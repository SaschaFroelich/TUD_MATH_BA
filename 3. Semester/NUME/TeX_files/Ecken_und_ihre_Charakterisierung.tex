\section{Ecken und ihre Charakterisierung}

\begin{definition}[konvex, konvexes Polyeder]
	Eine Menge $G\subseteq\real^n$ heißt \begriff{konvex}, wenn
	\begin{align}
		\lambda x + (1-\lambda)y\in G
	\end{align}
	für beliebige $x,y,\lambda\in G\times G\times (0,1)$ gilt. Gilt mit gewissen $d\in\real^p$ und $C\in\real^{p\times n}$, dass
	\begin{align}
		G=\{x\in\real^n\mid Cx\le d\}\notag
	\end{align}
	so heißt $G$ \begriff{konvexes Polyeder}.
\end{definition}

\begin{definition}[Ecke]
	Seien $G\subseteq\real^n$ und $z\in G$ gegeben. Dann heißt $z$ \begriff{Ecke} von $G$, wenn 
	\begin{align}
		\frac{1}{2}x + \frac{1}{2}y = z\Rightarrow x=y=z\notag
	\end{align}
	für alle $x,y\in G$ gilt.
\end{definition}

\begin{definition}[Basislösung]
	\proplbl{6_1_3}
	Sei $\rang(A)=m$. Ein Vektor $\overline{x}$ mit $A\overline{x}=b$ heißt \begriff{Basislösung} von $G_P$, wenn es Indexmengen $B,N\subseteq I=\{1,...,n\}$ gibt so dass
	\begin{itemize}
		\item $B\cup N=I$, $B\cap N=\emptyset$ und $\vert B\vert =m$
		\item $A_B$ regulär und
		\item $\overline{x}_N=0$
	\end{itemize}
	Eine Basislösung $\overline{x}$ heißt \begriff[Basislösung!]{zulässig}, wenn weiterhin $\overline{x}\ge 0$ gilt.
	
	Eine zulässige Basislösung $\overline{x}$ wird \begriff[Basislösung!]{nicht entartet} genannt, wenn $\overline{x}_B>0$ (das heißt $\overline{x}_i>0$ für $i\in B$). Andernfalls, wenn ein $j\in B$ existiert mit $\overline{x}_j=0$, so heißt $\overline{x}$ \begriff[Basislösung!]{entartet}. Die zu $i\in B$ bzw. $i\in N$ gehörenden Variablen $x_i$ werden \begriff{Basisvariable} bzw. \begriff{Nichtbasisvariable} zur Basislösung $\overline{x}$ genannt.
\end{definition}

\begin{*anmkerung}
$A_B$ ist die Matrix, die genau die Spalten von $A$ enthält, deren Nummer in $B$ ist. Beispiel $A_{\{1,2,4\}}$ enthält die erste, zweite und vierte Spalte von $A$.
\end{*anmerkung}

\begin{proposition}[Eckencharakterisierungssatz]
	\proplbl{6_1_4}
	Sei $\rang(A)=m$. Dann ist jede zulässige Basislösung von $G_P$ eine Ecke von $G_P$ und umgekehrt.
\end{proposition}
\begin{proof}
	Sei $\overline{x}$ eine zulässige Basislösung von $G_P$. Falls $\overline{x}=\frac{1}{2}x+\frac{1}{2}y$ für $x,y\in G_P$ gilt, so folgt wegen $x,y\ge 0$ und $\overline{x}_N=0$, dass
	\begin{align}
		\label{6.3}
		x_N=y_N=\overline{x}_N=0
	\end{align}
	Damit und wegen $\overline{x}\in G_P$ sowie der Regularität von $A_B$ erhält man weiter
	\begin{align}
		\label{6.4}
		\begin{split}
			b &= A\overline{x} = A_B\overline{x}_B + A_N\underline{\overline{x}_N}_{=0} = A_B\overline{x}_B \\
			\overline{x}_B &= A^{-1}_Bb
		\end{split}
	\end{align}
	Analog folgt $x_B=y_B=A^{-1}_Bb$. Also ist $\overline{x}_B=x_B=y_B$. Daher und wegen \cref{6.3} muss $\overline{x}$ Ecke von $G_P$ sein. \\
	Sei nun $\overline{x}$ Ecke von $G_P$ und $\mathcal{P}=\{i\in I\mid \overline{x}_i>0\}$. Angenommen die Spalten der Matrix $A_\mathcal{P}$ sind linear abhängig. Dann gibt es einen Vektor $w\in\real^n\backslash\{0\}$ mit $w_{I\setminus\mathcal{P}}=0$ und $Aw=0$. Für
	\begin{align}
		x(t) = \overline{x} + tw\notag
	\end{align}
	folgt $Ax(t)=A\overline{x} + tAw=b$. Wegen $\overline{x}_\mathcal{P}>0$ ergibt sich $x_\mathcal{P}(t)\ge 0$ und damit $x(t)\ge 0$ für alle $t$ mit $\vert t\vert$ hinreichend klein. Also gibt es ein $\overline{t}>0$, so dass $x(-\overline{t}),x(\overline{t})\in G_P$. Da $x(-\overline{t})\neq x(\overline{t})$ und $\frac{1}{2}x(-\overline{t}) + \frac{1}{2}x(\overline{t})=\overline{x}$ widerspricht dies der Voraussetzung, dass $\overline{x}$ Ecke von $G_P$ ist. Folglich ist die Annahme falsch, das heißt die Spalten von $A_\mathcal{P}$ sind linear unabhängig. Falls $\vert \mathcal{P}\vert=m$, setzen wir $B=\mathcal{P}$. Andernfalls kann wegen $\rang(A)=m$ die Menge $\mathcal{P}$ durch Hinzunahme von Indizes aus $I\setminus \mathcal{P}$ so zu $B$ ergänzt werden, dass die Spalten der Matrix $A_B$ linear unabhängig sind. Mit $N=I\setminus B$ sieht man, dass $\overline{x}$ alle Eigenschaften einer zulässigen Basislösung besitzt.
\end{proof}

\begin{proposition}
	\proplbl{6_1_5}
	Sei $G_P\neq 0$. Dann besitzt $G_P$
	\begin{enumerate}[label=(\alph*)]
		\item mindestens eine Ecke
		\item  höchstens endlich viele Ecken
	\end{enumerate}
\end{proposition}
\begin{proof}
	O.B.d.A. kann man zunächst $\rang(A)=m$ annehmen.
	\begin{enumerate}[label=(\alph*)]
		\item Zum Selbststudium empfohlen.
		\item $G_P$ besitzt höchstens endlich viele zulässige Basislösungen (und damit nach \propref{6_1_4} höchstens endlich viele Ecken), da es maximal $\binom{n}{m}$ Teilmengen der Menge $I$ mit Kardinalität $m$ gibt, die als Indexmenge $B$ in Frage kommen (und für jede dieser Indexmengen $B$ gibt es höchstens eine Basislösung).
	\end{enumerate}
\end{proof}

\begin{proposition}
	Falls die Optimierungsaufgabe \cref{6.2} lösbar ist, dann gibt es eine Ecke $\overline{x}$ von $G_P$, die \cref{6.2} löst.
\end{proposition}
\begin{proof}
	O.B.d.A. kann $\rang(A)=m$ angenommen werden. Sei $\mathcal{P}(x) = \{i\in I\mid x_i>0\}$. Unter allen Lösungen der Optimierungsaufgabe \cref{6.2} wähle man eine Lösung $\overline{x}$ so, dass $\mathcal{P}=\mathcal{P}(\overline{x})$ die minimal mögliche Anzahl von Elementen aufweist. 
	\begin{itemize}
		\item Falls die Matrix $A_\mathcal{P}$ Vollrang hat, so kann $\mathcal{P}$ erforderlichenfalls so durch geeignete Indizes aus $I\setminus\mathcal{P}$ zu $B$ ergänzt werden, dass $A_B$ dann regulär und $\overline{x}$ damit zulässige Basislösung (das heißt Ecke von $G_P$) ist.
		\item Angenommen $A_\mathcal{P}$ besitzt keinen Vollrang. Dann gibt es ein $w\in\real^n\backslash\{0\}$ mit $w_{I\setminus\mathcal{P}}=0$ und $Aw=0$. Ähnlich wie im Beweis von \propref{6_1_4} erhält man ein $\overline{t}\in\real$, so dass $x(-\overline{t}),x(\overline{t})\in G_P$, $x_{I\setminus\mathcal{P}}=0$ und $x_i(\overline{t})=0$ für ein $i\in\mathcal{P}$. Auf Grund der Minimaleigenschaft von $\overline{x}$ gilt $c^T\overline{x}\le c^T(\overline{x}+tw)$ für $t\in\{-\overline{t},\overline{t}\}$ und es folgt $c^Tw=0$. Dies liefert $c^Tx(\overline{t})=c^T(\overline{x}+w\overline{t})=c^T\overline{x}$. Folglich ist $\overline{x}$ eine Lösung der Optimierungsaufgabe \cref{6.2} mit weniger als $\vert\mathcal{P}\vert$ Nullelementen. Dies widerspricht der Auswahl von $\overline{x}$.
	\end{itemize}
\end{proof}
