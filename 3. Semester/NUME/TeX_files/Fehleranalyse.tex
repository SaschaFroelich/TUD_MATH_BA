\section{Fehleranalyse}

\subsection{Die Kondition einer Aufgabe}

Unter Aufgabe wird hier die Auswertung einer zumindest stetig differenzierbaren Abbildung
\begin{align}
	\Phi: D\to\real^n\notag
\end{align}
verstanden. Die Lösung der Aufgabe für ein Argument $a\in D\subset\real^n$ besteht also darin, $\Phi(a)$ zu ermitteln. Wir interessieren uns nun für die Frage, welchen Einfluss ein Fehler in $a$ (also die Verwendung der Maschinenzahl $\rd(a)$ statt $a$) auf das Ergebnis $\Phi(a)$ bei ansonsten exakter Rechnung hat. Dazu bezeichne $\tilde{a}\in D$ das fehlerbehaftete Argument. Für $\tilde{a}$ nahe bei $a$ erhält man aus der \person{Taylor}-Formel
\begin{align}
	\Phi(\tilde{a}) - \Phi(a) \approx\nabla\Phi(a)^T(\tilde{a}-a)= \left( \frac{\partial \Phi_i(a)}{\partial a_j}\right) _{\substack{i=1,...,m \\ j=1,...,n}} (\tilde{a}-a)\notag
\end{align}
und damit (unter der Bedingung $a_j\neq 0$ und $\Phi(a)_i\neq 0$)
\begin{align}
	\label{4.1}
	\frac{\Phi_i(\tilde{a}) - \Phi_i(a)}{\Phi_i(a)} \approx \frac{1}{\Phi_i(a)}\sum_{j=1}^{n}\frac{\partial \Phi_i(a)}{\partial a_j} (\tilde{a_j}-a_j) = \sum_{j=1}^n \frac{a_j}{\Phi_i(a)}\frac{\partial\Phi_i(a)}{\partial a_j}\frac{\tilde{a_j}-a_j}{a_j}
\end{align}

\begin{definition}[relative Konditionszahlen, gut/schlecht konditioniert]
	Es seien $D\subset\real^n$, $\Phi:D\to\real^m$ stetig differenzierbar und $a\in D$ mit $\Phi(a)\neq 0$. Dann heißen
	\begin{align}
		K_{ij} = \left|\frac{a_j}{\Phi_i(a)}\frac{\partial\Phi_i(a)}{\partial a_j}\right|\notag
	\end{align}
	\begriff[Konditionszahl!]{relative Konditionszahlen} der Aufgabe $\Phi(a)$. Die Aufgabe heißt \begriff[Konditionszahl!]{gut konditioniert}, wenn alle diese Konditionszahlen klein sind, sonst \begriff[Konditionszahl!]{schlecht konditioniert}.
\end{definition}

Mit \cref{4.1} folgt
\begin{align}
	\frac{\Phi_i(\tilde{a}) - \Phi_i(a)}{\Phi_i(a)}\approx \sum_{j=1}^n K_{ij}\frac{\tilde{a_j}-a}{a_j}\notag
\end{align}
Bei einer gut konditionierten Aufgabe verhält sich also der relative Fehler des Ergebnisses etwa wie die relativen Fehler der Eingangsdaten.

\begin{example}
	\begin{enumerate}[label=(\alph*)]
		\item Mit $\Phi(a)=\sqrt{a}$ für $a>0$ erhält man $m=n=1$ und
		\begin{align}
			K_{11} = \left|\frac{a}{\sqrt{a}}\frac{1}{2\sqrt{a}}\right| = \frac{1}{2}\notag
		\end{align}
		Die Aufgabe ist für alle $a>0$ gut konditioniert.
		\item Mit $\Phi(a) = a_1+a_2$ für $a_1+a_2\neq 0$ und $a_2\neq 0$ erhält man $m=1$, $n=2$ und
		\begin{align}
			K_{1j} = \left|\frac{a_j}{a_1+a_2}\cdot 1\right|\quad\text{für } j=1,2\notag
		\end{align}
		Die Aufgabe ist schlecht konditioniert, falls $a_1+a_2\approx 0$. Selbstauslöschung bei Subtraktion etwa gleich großer Zahlen, zum Beispiel:
		\begin{align}
			a_1 &= 1.23456789 &\tilde{a_1} = 1.234567885 \notag \\
			a_2 &= -1.23456788 &\tilde{a_2} = -1.23456788 \notag \\
			\Rightarrow a_1+a_2 &= 10^{-8} \notag \\
			\Rightarrow \tilde{a_1} + \tilde{a_2} &= -0.5\cdot 10^{-8} \notag
		\end{align}
		Der relative Fehler des Ergebnisses beträgt $\frac{\tilde{a_1}+\tilde{a_2} - (a_1+a_2)}{a_1+a_2}=0.5$, also 50\%!
	\end{enumerate}
\end{example}

\subsection{Stabilität von Algorithmen}

Nun wird untersucht, wie sich die Rundungsfehler in einzelnen Rechenschritten eines Algorithmus auswirken. Dazu wird die Aufgabe $\Phi:D\subset\real\to\real^m$ zu Grunde gelegt. ($m=1$ ist keine Einschränkung, da die folgende Untersuchung auch für weitere Komponenten analog möglich ist) und ein allgemeiner Algorithmus mit stetig differenzierbaren Abbildungen $r_i:\real^{n+1}\to\real$ (mit $i=0,...,N$) betrachtet.

\begin{algorithm}
	\proplbl{4_2_3}
	Input: $a$
	\begin{lstlisting}
%$y_0$% = %$r_0$%(a)
%$y_1$% = %$r_1$%(a, %$y_0$%)
...
	\end{lstlisting}
	Output: $y_N$ ($=\Phi(a)$ bei exakter Rechnung)
\end{algorithm}

Es stellt sich damit die Frage, wie sich die Rundungsfehler in in $y_0,...,y_{N-1}$ auf das Ergebnis des Algorithmus auswirken. Dazu betrachten wir die rekursiv definierten \begriff{Restabbildungen} $R_k:\real^{n+k+1}\to\real$ ($k=0,...,N-1$) mit
\begin{align}
	R_{N-1}(a,y_0,...,y_{N-1}) &= r_N(a,y_0,...,y_{N-1}) \notag \\
	R_{N-2}(a,y_0,...,y_{N-2}) &= r_N(a,y_0,...,y_{N-2},r_N(a,y_0,...,y_{N-2})) \notag \\
	&\vdots \notag \\
	R_(a,y_0) &= \dots \notag
\end{align}

\begin{*anmerkung}
	Wer sich fragt, was das soll: Die $y_i$ sind die Zwischenschritte zur Berechnung eines Terms. Soll zum Beispiel der Term $(1+2)^3-5$ berechnet werden, so ist
	\begin{align}
		y_0 &= 1+2 = 3 \notag \\
		y_1 &= y_0^3 \notag \\
		y_2 &= y_1-5 \notag
	\end{align}
	Die Restabbildungen $R_i$ setzen Zwischenschritte ein:
	\begin{align}
		R_2 &= y_1 -5\notag \\
		R_1 &= y_0^3 -5 \notag \\
		R_0 &= (1+2)^3 - 5\notag
	\end{align}
\end{*anmerkung}

Für $k=0,...,N-1$ wird nun die Kondition der durch $y_k\mapsto R_k(a,y_0,...,y_k)$ definierten Abbildung betrachtet, um festzustellen, wie sich Rundungsfehler in $y_k$ auf das Ergebnis $y_N$ auswirken.

\begin{definition}[relative Konditionszahlen, stabil, instabil]
	Die Zahlen
	\begin{align}
		AK_k = \left|\frac{y_k}{R_k(a,y_0,...,y_k)}\cdot\frac{\partial R_k(a,y_0,...,y_k)}{\partial y_k}\right|
	\end{align}
	heißen \begriff[Konditionszahl!]{relative Konditionszahlen} des \propref{4_2_3} \emph{bezüglich der Zwischenergebnisse} $y_k$. Der \propref{4_2_3} wird \begriff[Algorithmus!]{stabil} (oder gutartig) genannt, wenn alle diese Konditionszahlen nicht wesentlich größer sind als die Konditionszahlen der Aufgabe $\Phi(a)$. Andernfalls heißt der Algorithmus \begriff[Algorithmus!]{instabil}.
\end{definition}

\begin{example}
	Um die größere Nullstelle des Polynoms $x^2+2px+q$ zu bestimmen, kann man 
	\begin{enumerate}[label=(\alph*)]
		\item $\Phi(p,q)=-p+\sqrt{p^2-q}$ oder
		\item $\Theta(p.q)=\frac{q}{-p-\sqrt{p^2-q}}$ (entsprechend dem Satz von \person{Vieta}, sofern $p^2\neq q$)
	\end{enumerate}
	verwenden. Wir setzen dabei 
	\begin{align}
		\label{4.2}
		p>>1>q>0
	\end{align}
	voraus. Die relativen Konditionszahlen zur Aufgabe $\Phi(p,q)$ ergeben sich zu
	\begin{align}
		K_{11} &= \left|\frac{p}{-p+\sqrt{p^2-q}}\left(-1+\frac{p}{\sqrt{p^2-q}}\right)\right| = \left|\frac{p}{-p+\sqrt{p^2-q}}\left(\frac{-\sqrt{p^2-q}+p}{\sqrt{p^2-q}}\right)\right| \notag \\
		&= \left|\frac{p}{\sqrt{p^2-q}}\right|\approx 1\notag \\
		K_{12} &= \left|\frac{q}{-p+\sqrt{p^2-q}}\frac{1}{2\sqrt{p^2-q}}\right| = \left|\frac{q(p+\sqrt{p^2-q})}{-q} \frac{1}{2\sqrt{p^2-q}}\right| \notag \\
		&= \left|\frac{p+\sqrt{p^2-q}}{2\sqrt{p^2-q}}\right| \approx 1\notag
	\end{align}
	Also sind die Aufgaben $\Phi(p,q)$ und $\Theta(p,q)$ für \cref{4.2} gut konditioniert. Die Umsetzung der Aufgabe $\Phi(p,q)$ erfolge mit dem Algorithmus 
	\begin{align}
		\label{4.3}
		y_0 = p\cdot p\quad y_1 = y_0-q\quad y_2 = \sqrt{y_1}\quad y_3=-p+y_2
	\end{align}
	Es ergibt sich $N=3$, $n=2$, $R_2(p,q,y_0,y_1,y_2)=-p+y_2$ und
	\begin{align}
		AK_2 = \left|\frac{y_2}{R_2(p,q,y_0,y_1,y_2)}\frac{\partial R_2(p,q,y_0,y_1,y_2)}{\partial y_2}\right| = \left|\frac{y_2}{-p+y_2}\cdot 1\right| = \left|\frac{\sqrt{p^2-q}}{-p+\sqrt{p^2-q}}\right| >>1\notag
	\end{align}
	für \cref{4.2}. Der Algorithmus \cref{4.3} ist daher dann numerisch instabil. Betrachten wir nun den Algorithmus 
	\begin{align}
		\label{4.4}
		y_0 = p\cdot p\quad y_1=y_0-q\quad y_2=\sqrt{y_1}\quad y_3 = -p-y_2\quad y_4 = \frac{q}{y_3}
	\end{align}
	Es ist $N=4$. Die Restabbildungen lauten dann
	\begin{align}
		R_3(p,q,y_0,y_1,y_2,y_3) &= \frac{q}{y_3} \notag \\
		R_2(p,q,y_0,y_1,y_2) &= \frac{q}{-p-y_2} \notag \\
		R_1(p,q,y_0,y_1) &= \frac{q}{-p-\sqrt{y_1}} \notag \\
		R_0(p,q,y_0) &= \frac{q}{-p-\sqrt{y_0-q}} \notag 
	\end{align}
	Unter der Bedingung \cref{4.2} sind die Konditionszahlen $AK_0,...,AK_3$ kleiner gleich 1, so dass der Algorithmus \cref{4.4} dann numerisch stabil ist. Unter anderen Bedingungen kann das Stabilitätsverhalten der Algorithmen \cref{4.3} und \cref{4.4} anders sein, so dass bei der Implementierung eine Fallunterscheidung notwendig ist.
\end{example}