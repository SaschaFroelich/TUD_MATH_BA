\section{spezielle \person{Newton-Cotes}-Formeln}

Für $n=1$ erhält man die \begriff{Trapezformel} mit $x_0=a$, $x_1=b$ und $h=b-a$ wie folgt:
\begin{align}
	Q_1(f)=f_0\int_a^b L_0(x)\diff x+f_1\int_a^b L_1(x)\diff x\notag
\end{align}
Mit 
\begin{align}
	L_0(x)=\frac{x-x_1}{x_0-x_1}=\frac{b-x}{h}\quad&\text{und}\quad L_1(x)=\frac{x-x_0}{x_1-x_0} = \frac{x-a}{h} \notag \\
	\int_a^b L_0(x)\diff x=\frac{1}{h}\int_a^b (b-x)\diff x=\frac{1}{h}\left( bx-\frac{1}{2}x^2\right)\Bigg|_a^b&=\frac{1}{h} \left( \frac{b^2}{2}-ab+\frac{a^2}{2}\right) =\frac{h}{2}\notag \\
	\int_a^b L_1(x)\diff x&=\frac{h}{2}\notag
\end{align}
folgt
\begin{align}
	Q_1(f)=\frac{h}{2}(f_0+f_1)\notag
\end{align}

Für Polynomgrad $n=2$ erhält man auf ähnliche Weise die \begriff{\person{Simpson}-Formel} (auch \begriff{\person{Kepler}'sche Fassregel} genannt):
\begin{align}
	Q_2(f)=\frac{h}{3}(f_0+4f_1+f_2)\notag
\end{align}

Für Polynomgrade bis $n=6$ findet man weitere Formeln in der Literatur. Formeln nur $n>6$ werden aus numerischen Gründen nicht verwendet. Es können dann negative Gewichte auftreten.

\begin{proposition}
	\proplbl{2_3_1}
	\begin{enumerate}[label=(\alph*)]
		\item Sei $f\in C^2[a,b]$. Dann gilt:
		\begin{align}
			\vert Q_1(f)-J(f)\vert \le \frac{1}{12}h^3\Vert f''\Vert_\infty\notag
		\end{align}
		\item Sei $f\in C^4[a,b]$. Dann gilt:
		\begin{align}
			\vert Q_2(f)-J(f)\vert \le \frac{1}{12}h^5\Vert f^{(4)}\Vert_\infty\notag
		\end{align}
	\end{enumerate}
\end{proposition}
\begin{proof}
	\begin{enumerate}[label=(\alph*)]
		\item Für die Trapezformel erhält man mit \propref{1_2_9}
		\begin{align}
			\vert Q_1(f)-J(f)\vert = \left|\int_a^b f(x)-p_1(x)\diff x\right|\le \frac{\Vert f''\Vert_\infty}{2!}\int_a^b\vert (x-a)(x-b)\vert\diff x\notag
		\end{align}
		Mit $y:=x-\frac{a+b}{2}$ ergibt sich:
		\begin{align}
			x-a=y+\frac{b-a}{2}=y+\frac{h}{2}\quad\text{und}\quad x-b=y+\frac{a-b}{2}=y-\frac{h}{2}\notag \\
			\int_a^b\vert (x-a)(x-b)\vert\diff x = -\int_{-\sfrac{h}{2}}^{\sfrac{h}{2}}\left(y + \frac{h}{2}\right)\left(y - \frac{h}{2}\right)\diff y = -\left(\frac{1}{3}y^3 - \frac{1}{4}h^2y\right)\Bigg|_{-\sfrac{h}{2}}^{\sfrac{h}{2}}=\frac{1}{6}h^3 \notag
		\end{align}
		\item Zur Analyse des Quadraturfehlers der \person{Simpson}-Formel untersuchen wir zunächst
		\begin{align}
			W&:=\int_a^b (x-a)(x-x_1)(x-b)\diff x \notag \\
			\overline{W} &:=\int_a^b\vert (x-a)(x-x_1)(x-b)\vert\diff x\notag
		\end{align}
		Mit der Substitution $y:=x-x_1$ folgen $x-a=y+h$, $x-x_1=y$ und $x-b=y-h$, also
		\begin{align}
			\label{2.2}
			W=\int_a^b (y+h)y(y-h)\diff y=0
		\end{align}
		da die zu integrierende Funktion $y\mapsto (y+h)y(y-h)$ ungerade ist. Weiter erhält man
		\begin{align}
		\label{2.3}
			\overline{W}=\int_{-h}^{h}\vert (y+h)y(y-h)\vert\diff y=-2\int_{0}^{h} (y^2-h^2)y\diff y=\frac{1}{2}h^4
		\end{align}
		Wegen \propref{1_2_9} haben wir
		\begin{align}
			f(x)-p_2(x)=\frac{f'''(\xi(x))}{3!}w(x)=\frac{1}{6}f'''(x_1)w(x)+\frac{1}{6}(f'''(\xi(x))-f'''(x))w(x)\notag
		\end{align}
		für alle $x\in[a,b]$. Insbesondere ist daher $x\mapsto f'''(\xi(x))$ eine Funktion aus $C^4[a,b]$. Durch Integration folgt mit \cref{2.2}:
		\begin{align}
			\label{2.4}
			\begin{split}
				\left|\int_a^b f(x)-p_2(x)\diff x\right| &= \frac{1}{6}\left|f'''(x_1)\int_a^b w(x)\diff x+\int_a^b (f'''(\xi(x))-f'''(x_1))w(x)\diff x\right| \\
				&= \frac{1}{6}\left|\int_a^b (f'''(\xi(x))-f'''(x_1))w(x)\diff x\right| \\
				&\le \max_{x\in [a,b]}\left\lbrace\left|f'''(x)-f'''(x_1)\right|\cdot\int_a^b\vert w(x)\vert\diff x\right\rbrace
			\end{split}
		\end{align}
		Da $f\in C^[a,b]$ erhält man mit dem Mittelwertsatz
		\begin{align}
			\vert f'''(x)-f'''(x_1)\vert = \vert f^{(4)}(\zeta(x))\vert\cdot\vert x-x_1\vert\le h\Vert f^{(4)}\Vert_\infty\notag
		\end{align}
		mit $\zeta\in (a,b)$. Deshalb und wegen \cref{2.3} folgt aus \cref{2.4} weiter
		\begin{align}
			\vert Q_2(f)-J(f)\vert =\left|\int_a^b f(x)-p_2(x)\diff x\right|\le\frac{1}{12}h^5\Vert f^{(4)}\Vert_\infty
		\end{align}
	\end{enumerate}
\end{proof}

\begin{remark}
	Durch verfeinerte Abschätzungen kann man in \propref{2_3_1} b) $\vert Q_2(f)-J(f)\vert\le \frac{1}{90}h^5\Vert f^{(4)}\Vert_\infty$ erreichen. Auch für Quadraturformeln $Q_n(f)$ mit $n>2$ lassen sich entsprechende Ergebnisse für den Quadraturfehler herleiten, insbesondere hat der Quadraturfehler für $n=3$ mit $f\in C^4[a,b]$ die Ordnung $h^5$ und für $n=4$ mit $f\in C^6[a,b]$ die Ordnung $h^7$.
\end{remark}