\section{Das \person{Newton}-Verfahren}

Das \person{Newton}-Verfahren beruht auf der Idee, die Gleichung $F(x) = 0$ für eine gegebene Näherung $y\in\real^n$ einer Nullstelle durch die linearisierte Gleichung 
\begin{align}
	F(y) + F'(y)(x-y) \overset{!}{=} 0\notag
\end{align}
zu ersetzen, eine Lösung $x$ dieser Gleichung (sofern möglich) zu berechnen und als neue Näherung für die Nullstelle zu verwenden.

\begin{algorithm}[\person{Newton}-Verfahren]
	\proplbl{5_1_1}
	Input: $x^0\in\real^n$, $\varepsilon>0$ und $F$ in geeigneter Form
	\begin{lstlisting}
k = 0
do while %$\Vert F(x^k)\Vert>\varepsilon$%
 compute %$x^{k+1}$% 
 ! als Loesung von %$F'(x^k)(x-x^k)+F(x^k)=0$%
 k = k + 1
end do
	\end{lstlisting}
	Output: $x ^k$
\end{algorithm}

Vorausgesetzt $F$ ist differenzierbar, dann ist dieses Verfahren durchführbar, wenn alle auftretenden Matrizen $F'(x ^k)\in\real^{n\times n}$ invertierbar sind.

\begin{example}
	Sei $n=1$. Aus $F'(x^k)(x-x^k)+F(x^k)=0$ erhält man eine eindeutige Lösung $x^{k+1}$, falls $F'(x^k)\neq 0$. Es folgt dann durch Umstellen
	\begin{align}
		x^{k+1} = x^k - \frac{F(x^k)}{F'(x^k)}\notag
	\end{align}
	Speziell sei $F:\real\to\real$ gegeben durch $F(x) = e^{-x}-x$ und $x^0=0.5$. Damit erhält man
	\begin{align}
		x^1 &= 0.5663... \notag \\
		x^2 &= 0.56714316... \notag \\
		x^3 &= 0.567143290409781... \notag \\
		x^4 &= 0.567143290409784... \notag
	\end{align}
	also lokal sehr schnelle Konvergenz. Es werden nun Voraussetzungen dafür angegeben.
\end{example}

\begin{proposition}
	\proplbl{5_1_3}
	Es sei $D\subseteq\real^n$ offen, $F:D\to\real^m$ differenzierbar und $x^\ast\in D$ eine reguläre Nullstelle von $F$, das heißt $F(x ^\ast)=0$ und $F'(x^\ast)$ ist regulär. Weiter sei $F':D\to\real^{n\times n}$ \person{Lipschitz}-stetig in $D$, das heißt es gibt ein $L>0$, so dass
	\begin{align}
		\label{5.2}
		\Vert F'(x) - F'(y)\Vert \le L\Vert x-y\Vert\quad\forall x,y\in D 
	\end{align} 
	Dann gibt es eine Kugel $B(x^\ast,\delta)=\{x\in\real^n\mid \Vert x-x^\ast\Vert\le\delta\}\subset D$ um $x^\ast$, so dass \propref{5_1_1} für jeden Startvektor $x^0\in B(x^\ast,\delta)$ wohldefiniert ist. Falls der Algorithmus eine unendlich Folge $\{x ^k\}$ erzeugt und $\delta>0$ hinreichend klein ist, dann konvergiert $\{x^k\}$ gegen $x^\ast$ und es gibt ein $C>0$, so dass
	\begin{align}
		\label{5.3}
		\Vert x^{k+1} - x^k\Vert \le C\Vert x^k-x^\ast\Vert^2
	\end{align}
	für alle $k\in\natur$ gilt.
\end{proposition}
\begin{proof}
	Da $F'$ stetig in der offenen Menge $D$ ist, gibt es $\delta_1>0$ und $M\ge 1$, so dass $B(x^\ast,\delta_1)\subset D$ sowie
	\begin{align}
		\label{5.4}
		F'(x)\text{ regulär und } \Vert F''(x)^{-1}\Vert\le M \quad\forall x\in B(x^\ast,\delta_1)
	\end{align}
	Weiter folgt wegen \propref{5_0_1} und \cref{5.2}
	\begin{align}
		\Vert F(x) - F(y) - F'(y)(x-y)\Vert &= \left\Vert \int_0^1 \Big(F'(y+\tau(y-x)) - F'(y)\Big)(x-y)\diff \tau\right\Vert \notag \\
		&\le \int_0^1 \left\Vert \Big(F'(y+\tau(y-x)) - F'(y)\Big)(x-y) \right\Vert\diff\tau \notag \\
		&\le \frac{1}{2} L\Vert x-y\Vert^2 \notag
	\end{align}
	für alle $x,y\in B(x^\ast,\delta_1)$. Daher, mit \cref{5.4} und \propref{5_1_1}, erhält man für $x^k\in B(x^\ast,\delta_1)$
	\begin{align}
		\Vert x^{k+1}-x^\ast\Vert &= \Vert x^k - F'(x^k)^{-1}F(x^k) - x^\ast\Vert \notag \\
		&= \left\Vert -F'(x^k)^{-1}\Big(F(x^k) + F'(x ^k)(x^\ast-x^k)\Big)\right\Vert \notag \\
		&\le M\Vert F(x^k) + F'(x^k)(x^\ast-x^k)\Vert \notag \\
		&\le \frac{1}{2}ML\Vert x^k-x^\ast\Vert^2\notag
	\end{align}
	Sei nun $\delta\in [0,\delta_1)$ so gewählt, dass $ML\delta\le 1$. Dann folgt mit $x^k\in B(x^\ast,\delta)\subseteq B(x^\ast,\delta_1)$
	\begin{align}
		\label{5.5}
		\Vert x^{k+1}-x^\ast\Vert \le \frac{1}{2}ML\Vert x^k-x^\ast\Vert^2\le \frac{1}{2}\Vert x^k-x^\ast\Vert
	\end{align}
	Für $x^0\in B(x^\ast,\delta)$ erhält man damit und mit \cref{5.4} induktiv, dass der \propref{5_1_1} durchführbar ist und $\{x^k\}\subset B(x^\ast,\delta)$. Aus \cref{5.5} folgt daher die Konvergenz der Folge $\{x^k\}$ und \cref{5.3} für alle $k\in\natur$.
\end{proof}

Die Konvergenz der Folge $\{x^k\}$ gegen $x^\ast$ zusammen mit der Eigenschaft \cref{5.3} wird als \begriff{Q-quadratische Konvergenz} der Folge $\{x^k\}$ gegen $x^\ast$ bezeichnet.

\begin{*anmerkung}
	Es gibt neben der Q-quadratischen Konvergenz auch die R-quadratische Konvergenz. Das "'Q"' in Q-quadratischer Konvergenz kommt daher, dass man einen \textbf{Q}uotienten betrachtet: $\frac{\Vert x^{k+1}-x^\ast\Vert}{\Vert x^k-x^\ast\Vert}$. Das "'R"' in R-quadratischer Konvergenz steht für \textbf{R}adizieren, also das Wurzel-ziehen.
\end{*anmerkung}

\begin{remark}
	Es gibt zahlreiche Modifikationen des \person{Newton}-Verfahrens, zum Beispiel
	\begin{itemize}
		\item\textbf{Sekantenverfahren} für $n=1$: Dabei wird der Anstieg $F'(x^k)$ der Tangente $y=F'(x^k)(x-x^k)+F(x^k)$ an den Graphen von $F$ im Punkt $(x^k,F(x^k))$ durch den Anstieg
		\begin{align}
			\frac{F(x^k) - F(x^{k-1})}{x^k-x^{k-1}}\notag
		\end{align}
		der Sekante durch die Punkte $(x^k,F(x^k))$ und $(x^{k-1},F(x^{k-1}))$ des Graphen von $F$ ersetzt. Unter Voraussetzungen wie in \propref{5_1_3} lässt sich in einer Umgebung einer regulären Nullstelle $x^\ast$ die Konvergenz des Sekantenverfahrens gegen $x^\ast$ zeigen, genauer gibt es $c,\delta>0$ und $\kappa\in (0,1)$, so dass
		\begin{align}
			\Vert x^k-x^\ast\Vert \le c\kappa^{\tau^k}\notag
		\end{align}
		für alle Startwerte $x^0,x^1\in B(x^\ast,\delta)$ mit $x^0\neq x^1$ gilt, wobei $\tau = \frac{1+\sqrt{5}}{2}$. Die Folge $\{x^k\}$ ist mit der R-Ordnung $\tau$ gegen $x^\ast$ konvergent.
		\item\textbf{Quasi-\person{Newton}-Verfahren}: Hier wird $F'(x^k)$ durch eine Matrix $B_k\in\real^{n\times n}$ ersetzt, so dass die Quasi-\person{Newton}-Gleichung (oder Sekantengleichung)
		\begin{align}
			B_k(x^k-x^{k-1}) = F(x^k) - F(x^{k-1})\notag
		\end{align}
		erfüllt ist, speziell leistet dies etwa das \person{Broyden}-Verfahren mit
		\begin{align}
			B_k= B_{k-1} + \frac{\Big(F(x^k) - F(x^{k-1}) - B_{k-1}(x^k - x^{k-1})\Big)(x^k - x^{k-1})^T}{(x^k - x^{k-1})^T(x^k - x^{k-1})} \notag
		\end{align}
		\item\textbf{Inexakte \person{Newton}-Verfahren}: Dabei ist $x^{k+1}$ so zu wählen, dass
		\begin{align}
			\Vert F(x^k) + F'(x^k)(x^{k+1}-x^k)\Vert\le q_k\Vert F(x^k)\Vert\notag
		\end{align}
		mit einer Nullfolge $\{q_k\}\subset[0,1)$ erfüllt ist. Speziell lassen sich in diesem Rahmen Verfahren analysieren, bei denen etwa $F'(x ^k)$ durch eine Differenzapproximation ersetzt wird oder das lineare Gleichungssystem des \person{Newton}-Verfahrens nur näherungsweise gelöst wird.
	\end{itemize}
\end{remark}
