\section{\person{Gauss}'sche Quadraturformeln}

Wir gehen zunächst vom Interpolationsfehler (vgl. \propref{1_2_9})
\begin{align}
	f(x)-p_n(x) = \frac{f^{(n+1)}(\xi(x))}{(n+1)!}w_n(x)\quad\text{für } x\in [a,b]\notag
\end{align}
mit $w_n(x)=(x-x_0)...(x-x_n)$ aus. Bezogen auf das ganze Intervall $[a,b]$, kann man etwa $\Vert f-p_n\Vert_\infty$ oder $\Vert f-p_n\Vert_2$ als Maß für diesen Fehler verwenden. Da man über $f^{(n+1)}$ nicht verfügt, wird anstelle dessen $\Vert w_n\Vert_\infty$ oder $\Vert w_n\Vert_2$ untersucht. Dieses Fehlermaß ist offenbar nur von der Lage der Stützstellen $x_0,...,x_n\in [a,b]$ abhängig. Zur Vereinfachung beschränkt man sich zunächst auf das Intervall $[-1,1]$. Die Aufgabe, die Funktion
\begin{align}
	F_\infty :\begin{cases}
		\real^{n+1}&\to \real \\
		F_\infty(x_0,...,x_n) &\mapsto \max\limits_{x\in [-1,1]} \vert (x-x_0)...(x-x_n)\vert
	\end{cases}\notag
\end{align}
unter der Bedingung $(x_0,...,x_n)\in [-1,1]^{n+1}$ zu minimieren, hat als Lösung die Nullstellen des sogenannten \begriff{\person{Tschebyschow}-Polynoms} $T_{n+1}$ der Ordnung $n+1$. Die Funktion
\begin{align}
	F_2:\begin{cases}
		\real^{n+1} &\to \real \\
		F_2(x_0,...,x_n) &\mapsto \int_{-1}^{1} (x-x_0)^2...(x-x_n)^2\diff x
	\end{cases}\notag
\end{align}
wird unter der Bedingung $(x_0,...,x_n)\in [-1,1]^{n+1}$ durch die Nullstellen des sogenannten \begriff{\person{Legendre}-Polynoms} $P_{n+1}$ minimiert. Die \person{Legendre}-Polynome können rekursiv wie folgt definiert werden:
\begin{align}
	P_0(t) &= 1 \notag \\
	P_1(t) &= t \notag \\
	\vdots \notag \\
	(k+1)P_{k+1}(t) &= (2k+1)tP_k(t)-kP_{k-1}(t)\notag
\end{align}
für alle $t\in\real$ und $k=1,2,...$. Für $n=1$ erhält man zum Beispiel $P_2(t)=t^2-\frac{1}{3}$ mit den Nullstellen $x_{1/2}=\pm \frac{\sqrt{3}}{3}$. Das Interpolationspolynom zu diesem Stützstellen und den Stützwerten $f_0=f(x_0)$ sowie $f_1=f(x_1)$ lautet dann (in der \person{Lagrange}-Form)
\begin{align}
	q_1(x) = f_0\frac{x-x_1}{x_0-x_1}+f_1\frac{x-x_0}{x_1-x_0}\notag
\end{align}
Wegen
\begin{align}
	\int_{-1}^1 \frac{x-x_1}{x_0-x_1}\diff x = -\frac{\sqrt{3}}{2}\int_{-1}^1 \left(x-\frac{\sqrt{3}}{3}\right)\diff x&=1 \notag \\
	\int_{-1}^1 \frac{x-x_0}{x_1-x_0}\diff x &= 1\notag
\end{align}
hat man die \person{Gauss-Legendre}-Quadraturformel für das Integral $\int_{-1}^1 f(x)\diff x$ für $n=1$:
\begin{align}
	f\left(-\frac{\sqrt{3}}{3}\right) + f\left(\frac{\sqrt{3}}{3}\right)\notag
\end{align}
Man kann zeigen, dass die \person{Gauss-Legendre}-Quadraturformel  mit $n+1$ Stützstellen (also den Nullstellen von $P_{n+1}$) Polynome bis zum Grad $2n+1$ exakt integriert. Bei den geschlossenen \person{Newton-Cotes}-Formeln ist dies nur bis Grad $n$ (falls $n$ ungerade) bzw. bis zum Grad $n+1$ (falls $n$ gerade) möglich (vgl. Übungsaufgabe). Spezielle Modifikationen der \person{Gauss-Legendre}-Quadratur beziehen einen Randpunkt (\person{Gauss-Radau}) oder beide Randpunkte (\person{Gauss-Lobatto}) des Integrals mit ein.

Falls über $[a,b]$ zu integrieren ist, führt eine Variablentransformation auf ein Integral über $[-1,1]$ zum Ziel. Mit
\begin{align}
	y=\frac{2}{b-a}\left(x-\frac{a+b}{2}\right)\notag
\end{align}
ergibt sich $x=\frac{b-a}{2}y+\frac{a+b}{2}$, $\diff x=\frac{b-a}{2}\diff y$ und
\begin{align}
	\int_a^b f(x)\diff x = \frac{b-a}{2}\int_{-1}^1 f\left(\frac{b-a}{2}y+\frac{a+b}{2}\right)\diff y\notag
\end{align}