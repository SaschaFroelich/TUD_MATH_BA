\section{Interpolation durch Polynomsplines}

\subsection{Polynomsplines}

Zur Abkürzung bezeichne $\Delta$ eine Zerlegung des Intervall $[a,b]$ durch die Stützstellen $a=:x_0<...<x_n:=b$.

\begin{definition}[Polynomspline]
	Ein \begriff{Polynomspline} vom Grad $m\in\natur$ und Glattheit $l\in\natur$ zur Zerlegung $\Delta$ ist eine Funktion $s\in C^l[a,b]$ mit
	\begin{align}
		s_k := s\vert_{[x_k,x_{k+1}]}\in\Pi_n\quad\text{für } k=0,...,n-1\notag
	\end{align}
	Dabei bezeichnet $s\vert_{[x_k,x_{k+1}]}$ die Einschränkung von $s$ auf das Intervall $[x_k,x_{k+1}]$. Die Menge aller Splines wird mit $\mathcal{S}^l_m(\Delta)$ bezeichnet.
	
	Folglich ist ein Polynomspline $s\in\mathcal{S}^l_m(\Delta)$ auf jedem der Teilintervall $[x_k,x_{k+1}]$ ein Polynom vom Höchstgrad $m$. Außerdem ist $s\in\mathcal{S}^l_m(\Delta)$ in allen Punkten $x\in[a,b]$ (also auch in den Stützstellen) $l$-mal stetig differenzierbar. $\mathcal{S}^l_m(\Delta)$ ist mit der üblichen Addition und Multiplikation ein Vektorraum. Speziell ist $\mathcal{S}^0_1(\Delta)$ die Menge aller stetigen stückweise affin linearen Funktionen.
\end{definition}

\subsection{Interpolation durch kubische Polynomsplines}

Gegeben sei eine Zerlegung $\Delta$ und die Stützwerte $f_0,...,f_n$. Gesucht ist eine Funktion $s\in\mathcal{S}^l_3(\Delta)$ mit $l=1,2$ derart, dass
\begin{align}
	\label{1.6}
	s(x_k)=f_k\quad\text{für } k=0,...,n
\end{align}
Jede derartige Funktion heißt \begriff{kubischer Interpolationspline}.

\textbf{Konstruktion eines solchen Splines:}
\begin{align}
	h_k &:= x_{k-1}-x_k\quad\text{für } k=0,...,n-1 \notag \\
	m_k &:= s'(x_k) \quad\text{für } k=0,...,n-1\notag
\end{align}
Wegen $l\in\{1,2\}$ ist $s$ zunächst stetig differenzierbar. Wegen $s_k=s\vert_{[x_k,x_{k+1}]}$ für $k=0,...,n-1$ und $m=3$ kann man folgenden Ansatz für $s_k$ benutzen:
\begin{align}
	\label{1.7}
	s_k(x)=a_k(x-x_k)^3+b_k(x-x_k)^2+c_k(x-x_k)+d_k
\end{align}
Aus den Interpolationsbedingungen \cref{1.6} und der stetigen Differenzierbarkeit aller Funktionen in $s\in\mathcal{S}^l_m(\Delta)$ für $l\ge 1$ ergeben sich folgende Forderungen an $s_k$, $k=0,...,n-1$:
\begin{equation}
	\label{1.8}
	\begin{split}
		s_k(x_k) &= f_k \quad\text{und }\quad s_k(x_{k+1}) = f_{k+1} \\
		s'_k(x_k) &= m_k \quad\text{und }\quad s'_k(x_{k+1}) = m_{k+1}
	\end{split}
\end{equation}
Diese liefern:
\begin{equation}
	\label{1.9}
	\begin{split}
		d_k &= s_k(x_k)=f_k \\
		c_k &= s'_k(x_k)=m_k
	\end{split}
\end{equation}
und damit:
\begin{align}
	s_k(x_{k+1}) &= a_kh_k^3 + b_kh_k^2+m_kh_k + f_k = f_{k+1} \notag \\
	s'_k(x_{k+1}) &= 3a_kh_k^2 + 2b_kh_k + m_k = m_{k+1} \notag
\end{align}
Damit ergeben sich $a_k$ und $b_k$ als eindeutige Lösung für das lineare Gleichungssystem
\begin{align}
	\label{1.10}
	\begin{pmatrix}
		h_k^3 & h_k^2 \\ 3h_k^2 & 2h_k
	\end{pmatrix}
	\begin{pmatrix}
		a_k \\ b_k
	\end{pmatrix}=
	\begin{pmatrix}
		f_{k+1}-f_k-m_kf_k \\
		m_{k+1}-m_k
	\end{pmatrix}
\end{align}
Die Determinante ist $-h_k^4\neq 0$.