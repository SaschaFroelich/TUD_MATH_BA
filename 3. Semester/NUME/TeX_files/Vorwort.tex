Auch für die Vorlesung \textit{Einführung in die Numerik} im WS 2018/19 haben wir ein Skript geschrieben. Es ist eine in vielen Teilen verbesserte Version des Skriptes von Prof. \person{Fischer}, die uns leider nur in Papierform vorlag. Wir haben über das ganze Semester die 61 Seiten abgetippt und mit hilfreichen Notizen und Abbildungen aus der Vorlesung angereichert. Bei diesem Prozess hat sich die Nummerierung der Sätze, Bemerkungen, Gleichungen, ... deutlich verändert, aber immerhin gibt es jetzt nicht mehr eine Definition 2.3, einen Satz 2.3, einen Algorithmus 2.3, ....

Während der ersten Vorlesung ist die Diskussion über die Lauffähigkeit der Algorithmen entstanden. Wir sind damals zu dem Schluss gekommen, dass wohl einige Fehler in diesen enthalten sind, aber der grobe Ablauf stimmt überein. Es geht meiner Meinung nach (und wahrscheinlich auch der von Prof. \person{Fischer}) darum, dass man sieht, wie lange ein Algorithmus braucht um ein Problem lösen zu können, also um die Komplexität des Algorithmus.

Wie eigentlich immer, will auch hier gesagt sein, dass es sich lohnt die Vorlesung zu besuchen, auch wenn Prof. \person{Fischer} in der Regel sein Skript an die Tafel schreibt. Aber zwischendurch kommen immer mal wieder nützliche Bemerkungen, die das Verständnis des Stoffes deutlich erleichtern und jede Menge Arbeit in der Nachbereitung ersparen. Ich spreche da aus Erfahrung: Während ich das Skript geschrieben habe, habe ich vielleicht nur 50\% von dem verstanden, was ich da eigentlich geschrieben habe. Aber als ich dann die Vorlesung besucht habe, habe ich mich gefragt, wieso ich diesen Stoff nicht vorher komplett verstanden hatte.

Trotz sorgfältiger Kontrolle kann es vorkommen, dass hier und da noch ein Fehler versteckt ist. In diesem Fall bitten wir darum, dass du ein Issue auf \url{https://github.com/henrydatei/TUD_MATH_BA} erstellst und uns hilfst den Fehler zu beheben. Damit hilfst du nicht nur uns, sondern auch allen zukünftigen Studenten. Danke!