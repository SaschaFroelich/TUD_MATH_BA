\section{Maschinenzahlen und Rundungsfehler}

Ein Computer kann nur endlich viele Maschinenzahlen in normalisierter Gleitpunktdarstellung 
\begin{align}
	z = \sigma\cdot d_0d_1...d_{t-1}\cdot b^e\notag
\end{align}
exakt speichern, wobei
\begin{itemize}
	\item $b\in\natur$ mit $b\ge 2$ die Basis
	\item $d_0d_1...d_{t-1}$ mit $d_0,d_1,...,d_{t-1}\in\{0,...,b-1\}$ die Mantisse mit den Ziffern $d_i$
	\item $t\in\natur$ mit $t\ge 1$ die Mantissenlänge
	\item $e\in\natur$ mit $-m\le e\le M$ der Exponent
	\item $\sigma\in\{+1,-1\}$ das Vorzeichen
\end{itemize}
bedeuten. Zusätzlich ist 0 eine Maschinenzahl. Mit $\mathbb{M}=\mathbb{M}(b,t,m,M)$ wird die Menge aller Maschinenzahlen bezeichnet. Jede andere Zahl $x\in\real$, die im Computer gespeichert werden soll (auch Zwischenergebnisse), wird vorher auf eine Zahl $\rd(x)\in\mathbb{M}$ so gerundet, dass der durch die Rundung entstehende relative Fehler durch
\begin{align}
	\frac{\vert\rd(x)-x\vert}{\vert x\vert} = \min\limits_{z\in\mathbb{M}}\frac{\vert z-x\vert}{\vert x\vert}\quad\text{für } x\in\real\setminus\mathbb{M}\notag
\end{align}
gegeben ist. 

\begin{lemma}
	Für jedes $x\in\real\backslash\{0\}$ mit $b^{-m}\le \vert x\vert\le b^M$ gilt
	\begin{align}
		\frac{\vert\rd(x)-x\vert}{\vert x\vert} \le \text{\eps} = \frac{1}{2}b^{1-t}\notag
	\end{align}
\end{lemma}
\begin{proof}
	Ohne Beschränkung der Allgemeinheit sei $x>0$. Dann gibt es Zahlen $e\in\whole$ mit $m\le e\le M$ und eine (gegebenenfalls unendliche) Ziffernfolge $(x_k)\subset \{0,...,b-1\}$, so dass 
	\begin{align}
		x = (x_0x_1...x_{t-1}x_tx_{t+1}...)\cdot b^e\notag
	\end{align}
	Damit folgt
	\begin{align}
		\vert\rd(x)-x\vert &\le \frac{b}{2}b^{e-t} \notag \\
		\frac{\vert\rd(x)-x\vert}{\vert x\vert} &\le \frac{b^{e-t+1}}{2b^e}\le \frac{1}{2}b^{1-t} \notag
	\end{align}
\end{proof}

Die Zahl \eps wird als \begriff{Maschinengenauigkeit} bezeichnet und gibt den maximalen relativen Rundungsfehler für $x\in [-b^m,b^M]$ an.