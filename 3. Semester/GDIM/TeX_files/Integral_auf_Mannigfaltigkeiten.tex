\section{Integral auf Mannigfaltigkeiten}

\begin{underlinedenvironment}[Frage]
	$\displaystyle \int_M f\,\mathrm{d}a$ für Mannigfaltigkeit $M$?
\end{underlinedenvironment}

\begin{underlinedenvironment}[Idee]
	Überdecke $M$ mit Kartengebieten $U_{\beta}$ ($\beta \in \xi$) und suche Integrale $\int_{U_\beta}f\,\mathrm{d}a$ geeignet zusammen.
\end{underlinedenvironment}

\begin{underlinedenvironment}[Problem]
	$U_\beta$ überlappen sich im Allgemeinem
\end{underlinedenvironment}

\begin{underlinedenvironment}[Ausweg]
	Zerlege die Funktion $\alpha=1$ geeignet als $1 = \sum_{j=1}^\infty \alpha_j$.
\end{underlinedenvironment}

\begin{*definition}
	Die Menge der stetigen Funktionen $\alpha_j\colon M\to[0,1]$, $j\in\mathbb{N}$ heißt \begriff{Zerlegung der Eins} (ZdE) auf $M\subset\mathbb{R}^n$, falls \begin{enumerate}[label={\roman*)}]
		\item $\displaystyle \sum_{j=1}^\infty \alpha_j(u) = 1$ $\forall u\in M$
		\item Zerlegung ist lokal-endlich, d.h. $\forall u\in M$ existiert eine Umgebung $U(u)$ bezüglich $M$ mit \begin{align*}
			\alpha_j = 0\text{ auf }U(u)\text{ für f.a. }j\in\mathbb{N}
		\end{align*}
	\end{enumerate}
\end{*definition}
\begin{*definition}
	Sei $\mathcal{U}$ eine bezüglich $M$ offene Überdeckung von $M\subset\mathbb{R}^n$. Die Zerlegung der Eins $\{\alpha_j \}$ ist $\mathcal{U}$ untergeordnet, falls $\forall j$ $\exists U_j\in \mathcal{U}\colon$ $\supp \alpha_j\subset U_j$. $\supp \alpha_j := \overline{\{ u\in M \mid \alpha_j(u)\neq 0\}}$ ist der \begriff{Träger} von $\alpha_j$.
\end{*definition}

\begin{proposition}[Existenz der Zerlegung der Eins]
	Sei $M\subset\mathbb{R}^n$ und sei $\mathcal{U}$ eine bezüglich $M$ offene Überdeckung von $M$\\
	\hspace*{0.5em}$\Rightarrow$ es existiert eine Zerlegung der Eins $\{\alpha_j\}$ von $M$, die $\mathcal{U}$ untergeordnet ist.
\end{proposition}
\begin{*remark}\hspace*{0.5em}
	\vspace*{-1.5em}
	\begin{itemize}
		\item Betrachte später die Überdeckung $\mathcal{U}$ einer Mannigfaltigkeit $M$ mit Kartengebieten
		\item $\alpha_j$ in Wahrheit in $C^\infty$
	\end{itemize}
\end{*remark}

\begin{proof}
	Sei $\mathcal{U} = \bigcup_{\alpha\in A} U_\alpha$.
	\vspace*{-0.8\baselineskip}
	\begin{enumerate}[label={\alph*)}]
		\item $U_\alpha\in\mathcal{U}$ offen bezüglich $M$ $\Rightarrow$ $\exists W_\alpha\subset\mathbb{R}^n\colon U_\alpha = W_\alpha\cap M$. Setzte $W = \bigcup_{\alpha\in A} W_\alpha$ offen im $\mathbb{R}^n$.
		
		Sei $K_{j} := \{ u\in W \mid \mathrm{dist}_{W^\complement} u \ge \frac{1}{j} \} \cap \overline{B_j(0)}$. Offenbar sind die $K_j$ kompakt \\
		\hspace*{0.5em} $\Rightarrow$ $K_j \subset K_{j+1}$ $\forall j\in\mathbb{N}$ und $\bigcup_{i\in\mathbb{N}} K_j = W$. ($\{K_j\}$ heißt kompakte Ausschöpfung von $W$).
		
		\item Sei $u\in K_{j+1}\setminus \inn K_{j+1}$ (kompakt) $\subset \inn K_{j+2}\setminus K_{j-1}$ (offen) \\\begin{tabularx}{\linewidth}{r@{$\;\;$}X}
			$\Rightarrow$ & $\exists \alpha\in A$: $u\in W_\alpha$ \\
			$\Rightarrow$ & $\exists$ Kugel $B_r(u)$, offen im $\mathbb{R}^n$ ($r > 0$): $B_r(u) \subset W_\alpha \cap (\inn K_{j+2}\setminus K_{j-1})$ \\
			$\Rightarrow$ & $K_{j+1}\setminus \inn K_j$ wird von endlich vielen Kugeln $B_r(u)$ überdeckt \\
			$\Rightarrow$ & $\exists$ Folge $\{u_j\}$ in $W$ mit $\bigcup_{j=1}^\infty B_{r_j}(u_j) = W$ und für $u\in W$ gilt:
			
			\hspace*{0.5em}$\exists$ Umgebung $U$ mit $U\cap B_{r_j}(u_j)\neq \emptyset$ nur für endlich viele $j$
		\end{tabularx}
		\item Betrachte $\gamma_j\colon W\to [0,1]$ mit\begin{align*}
			\gamma_j (v) := \begin{cases}
				e^{\frac{1_j}{\vert v - u_j\vert - v_j}}, & \text{für }\vert v - u_j\vert \le r_j,\\
				0, & \text{sonst}
			\end{cases}
		\end{align*}
		Offenbar gilt $\gamma_j(r)>0$ auf $B_{r_j}(u_j)$, $\gamma_j\in C^\infty(W)$. Setzte $\gamma(u) = \sum_{j=1}^\infty \gamma_j(u)$, $\alpha_j(u) := \frac{\gamma_j(u)}{\gamma(u)}$ $\forall u\in W$.
		
		Offenbar ist $\{\alpha_j \}$ eine Zerlegung der Eins von $W$, damit auch von $M$ und ist offenbar $\mathcal{U}$ untergeordnet.
	\end{enumerate}
\end{proof}

\begin{*definition}
	Sei $M\subset\mathbb{R}^n$, $f\colon M\rightarrow\mathbb{R}^n$, $\supp f\subset U\subset M$, $U$ Kartengebiet von $M$.
	
	$f$ heißt \begriff{integrierbar auf $M$}, falls die Einschränkung $f|_U$ integrierbar auf Kartengebiet $U$ und \begin{align}
		\int_M f\mathrm{d}a := \int_U f|_U\,\mathrm{d}a
	\end{align}
	heißt \begriff{Integral} von $f$ auf $M$.
\end{*definition}

\begin{lemma}[Kriterium für Integrierbarkeit]
	\proplbl{integral_mf_lemma_2}
	Sei $M\subset\mathbb{R}^n$ eine Mannigfaltigkeit, $f\colon M\to\mathbb{R}$, $\supp f\subset U\subset M$, $U$ Kartengebiet von $M$ und sei $\{x_j\}$ eine Zerlegung der Eins auf $M$. Dann: \\
	\begin{tabularx}{\linewidth}{X@{$\;\;$}c@{$\;\;$}X}
		$f$ integrierbar auf $M$ & $\Leftrightarrow$ & \begin{minipage}[t]{\linewidth}
			\vspace*{-1\baselineskip}
			\begin{enumerate}[label={\roman*)}]
				\item $f_{x_j}$ integrierbar auf $M$ $\forall j\in\mathbb{N}$
				\item $\sum_{j=1}^\infty \int_m \vert f \vert \alpha_j\,\mathrm{d}a < \infty$
			\end{enumerate}
		\end{minipage}
	\end{tabularx}
	\begin{flalign}
		\proplbl{eq:integral_mf_2}
		\quad\Rightarrow\;\; & \sum_M f\mathrm{d}a = \sum_{j=1}^\infty \int_M \alpha_j \mathrm{d}a&
	\end{flalign}
\end{lemma}

\begin{proof}\hspace{0pt}
	\vspace{-0.8\baselineskip}	
	\begin{enumerate}[label={\alph*)}]
		\item Sei $f$ integrierbar auf $M$ $\xRightarrow{\propref{integration_mf_7}}$ i) und
		\begin{flalign}
			\sum_{j=1}^\infty \int_M \vert f \vert \alpha_j\,\mathrm{d}a = \lim\limits_{k\to\infty} \sum_{j=1}^k \int \vert f \vert \alpha_j\,\mathrm{d}a \le \int_M \vert f \vert \,\mathrm{d}a < \infty
		\end{flalign}
		$\Rightarrow$ ii) $\xRightarrow[\text{Konvergenz}]{\text{majorisierte}}$ \eqref{eq:integral_mf_2}
		\item gelten i) und ii) $\xRightarrow[\text{Konvergenz}]{\text{majorisierte}}$ $\vert f \vert$ integrierbar $\Rightarrow$ $f$ integrierbar
	\end{enumerate}
\end{proof}

\begin{*definition}
	Sei $M\subset\mathbb{R}^n$ eine Mannigfaltigkeit und $\mathcal{U}$ eine offene Überdeckung bezüglich $M$ von $M$ mit Kartengebieten.

	$f\colon M\to\mathbb{R}$ heißt \begriff{integrierbar auf Mannigfaltigkeit $M$}, falls die Zerlegung der Eins $\{\alpha_j\}$ auf $M$ existiert, die $\mathcal{U}$ untergeordnet ist, sodass \begin{enumerate}[label={\roman*)}]
		\item $f\alpha_j$ integrierbar $\forall j\in\mathbb{N}$ (auf $M$)
		\item $\sum_{j=1}^\infty \int_M \vert f \vert\alpha_j\,\mathrm{d}a < \infty$
	\end{enumerate}
	und damit definiere sich \begin{align}
		\proplbl{eq:integral_mf_3}
		\int_M f \mathrm{d} a = \sum_{j=1}^\infty \int_M f\alpha_j \,\mathrm{d}a,
	\end{align}
	und heißt \begriff{Integral von $f$ auf $M$}.
\end{*definition}

\begin{proposition}[Rechtfertigung des Integralbegriffs]
	$f$ ist integrierbar auf $M$ und $\int_M f \mathrm{d}a$ sind unabhängig von konkreter Überdeckung $\mathcal{U}$ und Zerlegung der Eins $\{\alpha_j\}$.
\end{proposition}

\begin{proof}
	Sei $\colon M\to\mathbb{R}$ integrierbar auf $M$ mit $\{\alpha_j\}$, $\mathcal{U}$ gemäß Definition. Sei $\{\tilde{\alpha}_j\}$ eine weitere Zerlegung der Eins, die einer Überdeckung $\tilde{\mathcal{U}}$ durch Kartengebiete untergeordnet ist. Dann sind zu zeigen: \begin{enumerate}[label={\roman*')},leftmargin=4em]
		\item $f\tilde{\alpha}_j$ ist integrierbar auf $M$ $\forall j$ und 
		\item $\sum_{j=1}^\infty \int_M \vert f \vert \alpha_j\,\mathrm{d}a < \infty$ und
		\item $\sum_{j=1}^\infty \int_M f\alpha_j\,\mathrm{d}a = \sum_{j=1}^\infty \int_m f \tilde{\alpha}_j\,\mathrm{d}a$.
	\end{enumerate}
	\begin{enumerate}[label={zu \roman*')},leftmargin=4em]
		\item $f\alpha_j$ sind integrierbar auf $M$ nach Voraussetzung \\
		\begin{tabularx}{\linewidth}{r@{$\;\;$}X}
				$\xRightarrow{\propref{integration_mf_7}}$ & \begin{minipage}[t]{0.5\linewidth}
					$f\tilde{\alpha}_k \alpha_j$ ist integrierbar auf $M$ $\forall k,j\in\mathbb{N}$ und
					\[
						\sum_{j=1}^\infty\int_M \vert f \tilde{\alpha}_k\vert \alpha_j\,\mathrm{d}a \le \sum_{j=1}^\infty \vert f \vert\alpha_j \,\mathrm{d}a < \infty
					\]
				\end{minipage} \\[1\parskip]
				$\xRightarrow{\propref{integral_mf_lemma_2}}$ & $f\tilde{\alpha}_k$ und $\vert f \tilde{\alpha}_k\vert$ integrierbar auf $M$ $\forall k$ \\
				$\Rightarrow$ & \begin{minipage}[t]{\linewidth} i') und
					\begin{equation}
					\proplbl{eq:integral_mf_star}
					\tag{\star} \begin{aligned}
					\begin{split}
						\int_M f \tilde{\alpha}_k \, \mathrm{d}a &= \int_{j=1}^\infty \int_M f \tilde{\alpha}_j \, \mathrm{d}a \qquad \text{bzw.} \\
						\int_M \vert f \vert \tilde{\alpha}_k \,\mathrm{d}a &= \int_{j=1}^\infty \int \vert f \vert \tilde{\alpha}_k \alpha_j \,\mathrm{d}a
					\end{split}\end{aligned}
					\end{equation}
				\end{minipage}
		\end{tabularx}
		\item $f\alpha_j$ integrierbar auf $M$ nach Voraussetzungen
		{\zeroAmsmathAlignVSpaces
		\begin{flalign}
			\;\;\xRightarrow[\text{mit }\{\tilde{\alpha}_j\}]{\propref{integral_mf_lemma_2}}\;\; & 
					\label{eq:integral_mf_star_star}
					\tag{\star\star} \int_M f \alpha_j\,\mathrm{d}a = \sum_{k=1}^\infty \int_M f\alpha_j \tilde{\alpha}_k\,\mathrm{d}a \quad\forall j& \\
					\notag
					& \text{und analog für $\vert f \vert$} \\
			\proplbl{eq:integral_mf_raute}
			\tag{\#}
			\Rightarrow\;\;& \sum_{j=1}^\infty \sum_{k=1}^\infty \int_M \vert f \vert\alpha_j\alpha_k\,\mathrm{d}a = \sum_{j=1}^\infty \int_m \vert f \vert \alpha_j\,\mathrm{d}a \overset{\text{ii)}}{<} \infty \\
			\notag
			& \text{Doppelreihensatz, \eqref{eq:integral_mf_star_star} mit $\vert f \vert$ und ii) eerlauben die Vertauschung der Summation in \eqref{eq:integral_mf_raute}} \\
			\proplbl{eq:integral_mf_plus}
			\tag{+}
			\xRightarrow{\eqref{eq:integral_mf_star}}\;\;& \sum_{k=1}^\infty \int_M \vert f \vert \tilde{\alpha}_k \,\mathrm{d}a = \sum_{j=1}^\infty \int_M \vert f \vert \alpha_j\,\mathrm{d}a \\
			\notag 
			\Rightarrow\;\;& \text{ii')}
		\end{flalign}}
		\item Analog erhält man \eqref{eq:integral_mf_plus} mit $f$ statt $\vert f \vert$ $\Rightarrow$ iii')
	\end{enumerate} 
\end{proof}

\begin{*definition}
	Sei $M\subset\mathbb{R}^n$, $M$ Mannigfaltigkeit, $A\subset M$ Teilmenge. Die Funktion $f\colon A\to\mathbb{R}$ heißt \begriff{integrierbar auf $A$}, falls \begin{align*}
		f_A := \begin{cases}
			f, &\text{auf $A$},\\
			0, & \text{sonst}
		\end{cases}
	\end{align*}
	integrierbar auf $M$ ist. $A\subset M$ heißt (endlich) \begriff{messbar} in $M$ falls die Funktion $f= \equiv 1$ auf $A$ integrierbar ist. \begin{align*}
		v_d(A) = \int_A \mathrm{d}a
	\end{align*}
	heißt \begriff{$d$-dimensionaler Inhalt} ($d$-dimensionales Maß) von $A$
\end{*definition}
\begin{underlinedenvironment}[beachte]
	Für $A\subset\mathbb{R}^n$ Lebesgue-messbar ist $\lambda^n(A) = \infty$ möglich. Hier ist $v_d(A) < \infty$ für $A\subset M$ messbar.
\end{underlinedenvironment}

\begin{*definition}
	$A\subset M$ heißt \begriff{$d$-Nullmenge}, falls $v_d(A) = 0$.
\end{*definition}
\begin{underlinedenvironment}[beachte]
	$d$-Nullmengen auf $M$ entsprechend $\mathcal{L}^d$-Nullmengen im Parameterbereich.
\end{underlinedenvironment}

\begin{proposition}
	Sei $M\subset\mathbb{R}^n$ eine Mannigfaltigkeit, $A\subset M$ kompakt bezüglich $M$, $f\colon A\to\mathbb{R}$ stetig
	
	\hspace*{0.5em}$\Rightarrow$ $f$ integrierbar auf $A$
\end{proposition}

\begin{underlinedenvironment}[Hinweis]
	\ \\
	\vspace{-1.5\baselineskip}
	\begin{itemize}
		\item $A\subset M$ ist kompakt bezüglich $M$, z.B. $A = \phi(U)$ für Parametrisierung und $U\subset\mathbb{R}^n$ kompakt
		\item somit sind alle kompakten $A\subset M$ messbar
	\end{itemize}
\end{underlinedenvironment}

\begin{proof}\hspace*{0pt}
	\vspace*{-0.8\baselineskip}
	\begin{enumerate}[label={\alph*)}]
	\item Sei $A\subset U$ für ein Kartengebiet $U\subset M$ mit zugehöriger Parametrisierung $\phi\colon V\subset\mathbb{R}^d\to U$ \\
	\hspace*{0.5em}$\Rightarrow$ $B:= \phi^{-1}(A)$ kompakt im $\mathbb{R}^d$ (da $\phi$ Homöomorphismus)
	
	Da $f(\phi(\,\cdot\,))\sqrt{g^\phi(\,\cdot\,)}$ stetig auf $B$ \\
	\hspace*{0.5em}$\Rightarrow$ auch integrierbar auf $B$ $\Rightarrow$ $f$ integrierbar auf $A$
	
	\item (allgemeiner Fall)
	
	Sei $\{\alpha_j\}$ eine Zerlegung der Eins zur Mannigfaltigkeit $M$. $\forall v\in A$ $\exists$ Umgebung $U(v)\subset M$: $\alpha_j = 0$ auf $U(v)$ für fast alle $j\in\mathbb{N}$.
	
	$\{ U(v)\}_{v\in A}$ ist eine offene Überdeckung von $A$.\\
	\begin{tabularx}{\linewidth}{>{$}r<{$}@{$\;\,$}X}
		\Rightarrow & bereits endlich viele überdecken kompaktes $A$ \\
		\Rightarrow & $\forall m\in\mathbb{N}$: $\alpha_j = 0$ auf $A$ $\forall j > m$ \\
		\Rightarrow & $f_A(u) = \sum_{j=1}^m f_A(u)\alpha_j$ $\forall u\in M$ 
	\end{tabularx}

	$\supp f_A \alpha_j$ ist abgeschlossene Teilmenge der kompakten Menge $A$ $\Rightarrow$ selbst kompakt \\
	\hspace*{0.5em}$\xRightarrow{\text{a)}}$ $f_A \alpha_j$ integrierbar auf $M$ $\forall j$
	
	Wegen \begin{align*}
		\sum_{j=1}^\infty \int_M \vert f_A\vert \alpha_j\,\mathrm{d}a = \sum_{j=1}^m \int \vert f_A\vert\alpha_j \,\mathrm{d}a < \infty
	\end{align*}
	$\Rightarrow$ $f_A$ integrierbar auf $M$
	\end{enumerate}
\end{proof}

Übertragung der Eigenschaften aus \propref{integration_mf_7}:

\begin{proposition}[Eigenschaften des Integrals]
	Sei $M\subset\mathbb{R}^n$ Mannigfaltigket und $f$, $g$, $f_k\colon M\to\mathbb{R}$. Dann: \begin{enumerate}[label={\arabic*)}]
		\item $f$ integrierbar auf $M$ $\Leftrightarrow$ $\vert f \vert$ integrierbar auf $M$ $\Leftrightarrow$ $f^+$ und $f^-$ integrierbar auf $M$
		\item $f$, $g$ integrierbar auf $M$, $c\in\mathbb{R}$ \begin{flalign*}
			\;\;\Rightarrow\;\;& \int_M cf \pm g\mathrm{d}a = c\int_M f\mathrm{d}a \pm \int_M g\mathrm{d}a&
		\end{flalign*}
		\item $f$, $g$ integrierbar auf $M$, $g$ beschränkt auf $M$ \\
		\hspace*{0.5em} $\Rightarrow$ $f\cdot g$ integrierbar auf $M$
		\item (Monotone Konvergenz)
		
		Seien $f_1\le f_2 \le \dotsc$ auf $M$, alle $f_k$ integrierbar auf $M$. Die Folge $\int_M f_k\,\mathrm{d}a$ sei beschränkt, $f(u) := \lim_{k\to\infty}\to\infty f_k(u)$ $\forall u\in M$\begin{flalign*}
		\quad\Rightarrow\;\;&f\text{ integrierbar auf $M$ mit }\int_M f\mathrm{da} = \lim\limits_{k\to\infty} \int_M f_k \mathrm{d}a&
		\end{flalign*}
		\item (Majorisierte Konvergenz)
		
		Seien $_k$, $g$ integrierbar auf $M$, $\vert f_k\vert \le g$ auf $M$ $\forall k$ und $f(u) := \lim_{k\to\infty} f_k(u)$ $\forall u\in M$ \begin{flalign*}
		\quad\Rightarrow\;\;&f \text{ integrierbar auf $M$ mit } \int_M f\mathrm{d}a = \lim\limits_{k\to\infty} \int_M f_k\mathrm{d}a &
		\end{flalign*}
	\end{enumerate}
\end{proposition}

\begin{proof}
	Sei $\{\alpha_j \}$ eine Zerlegung der Eins zu $M$.
	
	\begin{underlinedenvironment}[beachte]
		$f$ ist integrierbar auf $M$ $\xRightarrow{\text{Def.}}$ $f\alpha_j$ ist integrierbar auf einem Kartengebiet $U_j\subset M$
	\end{underlinedenvironment}
	\vspace*{-\dimexpr \parskip + \baselineskip\relax}
	Damit folgen (1) -- (3) leicht aus \propref{integration_mf_7}.
	\begin{enumerate}[label={zu \arabic*)},leftmargin=4em,start=4]
		\item ähnlich zu 5)
		\item Fixiere ein $j\in\mathbb{N}$. $f_k \alpha_j$ ist integrierbar auf einem Kartenbegiet $\forall U$, \begin{align*}
			\lim\limits_{k\to\infty} f_k(u)\alpha_j(u) = f(u) \alpha_j(u)
		\end{align*}
		Mit $\vert f_k \alpha_j\vert \le g\alpha_j$ $\xRightarrow{\propref{integration_mf_7}}$ $f\alpha_j$ integrierbar und \begin{align}
			\proplbl{eq:integral_mf_star_2}
			\tag{\star}
			\lim\limits_{k\to\infty} \int_M f_k \alpha_j \mathrm{d}a = \int_M f\alpha_j \mathrm{d}a
		\end{align}
		Wegen $\vert f\alpha_j \vert \le g\alpha_j$ $\forall j$: \begin{align*}
			\sum_{j=1}^\infty \int_M \vert f\alpha_j\vert\,\mathrm{d}a \le \sum_{j=1}^\infty \int_M g\alpha_j\,\mathrm{d}a \overset{\text{$g$ intbar}}{<} \infty
		\end{align*}
		$\Rightarrow$ $f$ integrierbar auf $M$ mit \begin{align*}
			\int_M f\mathrm{d}a = \sum_{j=1}^\infty \int_M f\alpha_j\,\mathrm{d}a
		\end{align*}
		
		Sei $\epsilon > 0$, dann existiert ein $m\in \mathbb{N}$ mit \begin{align*}
			\left\vert\sum_{j=m+1}^\infty \int_M f\alpha_j\,\mathrm{d}a \right\vert < \epsilon
		\end{align*}
		und es existiert ein $k_0\in\mathbb{N}$ mit \begin{align*}
			 \left\vert\int_M f\alpha_j\,\mathrm{d}a - \int_M f_k\alpha_j \,\mathrm{d}a \right\vert < \frac{\epsilon}{m}\quad\forall j=1,\dotsc,m \;\forall k\ge k_0\qquad \text{(nach \eqref{eq:integral_mf_star_2})}
		\end{align*}
		\begin{flalign*}
			\quad\Rightarrow\;\;&
			\begin{aligned}[t]
				\left\vert\int_M f\mathrm{d}a - \int_M f_k\,\mathrm{da} \right\vert &\le \begin{multlined}[t][0.7\linewidth]
				\left\lvert\sum_{j=1}^m \left( \int_M f\alpha_j\,\mathrm{d}a - \int_M f_k \alpha_j\,\mathrm{d}a\right)\right\vert \\ + \left\vert\sum_{j=m+1}^\infty \int_M f\alpha_j\,\mathrm{d}a \right\vert + \left\vert \sum_{j=m+1}^\infty \int_M g\alpha_j \,\mathrm{d}a\right\vert
				\end{multlined}\\
				& \le \frac{\epsilon}{m}\cdot m + \epsilon + \epsilon = 3\epsilon\quad\forall k\ge k_0
			\end{aligned}&
		\end{flalign*}
		$\xRightarrow[\text{bel}]{\epsilon > 0}$ $\lim_{k\to\infty} \int_M f_k\,\mathrm{d}a = \int_M f\mathrm{d}a$.
	\end{enumerate}
\end{proof}