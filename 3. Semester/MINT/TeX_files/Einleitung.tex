\section{Einleitung}

\textbf{messen:} Längen, Flächen, Volumina, $\natur \to $ zählen, Wahrscheinlichkeiten, Energie $\to$ Integrale, ... \\
Wenn man ein Integral hat: $\int_{t_0}^{t}F(t)\diff t$, also wird das $\diff t$ durch ein Maß $\mu(\diff t)$ ersetzt.
%TODO graph
\newline Wir messen Mengen:
\begin{align}
	\mu: \mathcal{F} \to [0,\infty] \text{ mit }\mathcal{F} \subset \mathcal{P}(E) \notag
\end{align}
Dabei ist:
\begin{itemize}
	\item $E$ eine beliebige Grundmenge
	\item $\mathcal{P}(E)=\{A\mid A\subset X\}$ die Potenzmenge von $E$
	\item $F \to \mu(F) \in [0,\infty]$
\end{itemize}

\textbf{Konvention:}
\begin{itemize}
	\item Familien von Mengen: $\mathcal{A}, \mathcal{B}, \mathcal{C}, \mathcal{F}, \dots, \mathcal{R}$
	\item Mengen: $A, B, E$
	\item Maße: $\mu, \lambda, \nu, \rho, \delta$
	\item Abbildungen: $\phi, \psi, \gamma, \eta$
\end{itemize}

\begin{*example}[Flächenmessung]
	%TODO needs graph! and no counting for this example!
	\begin{align}
		\mu(F) = g \cdot h &= \mu(F_1) + \mu(F_2) + \mu(F_3)\notag\\
		                   &= g^{\prime} \cdot h + h^{\prime}\cdot g^{\prime \prime} + h^{\prime \prime} \cdot g^{\prime \prime}\notag\\
		                   &= \dots \overset{!}{=} gh\notag
	\end{align}
	$F_1, F_2, F_3$ disjunkt bzw. nicht überlappend!\\
	$\mu(F) = \mu(\Delta_1)+\mu(\Delta_2)$ mit $\mu(\Delta) = 0.5 gh$\\ %TODO graph
	Allgemein für Dreiecke: \\%TODO graph
	$\mu(\Delta) = 0.5 gh \overset{!}{=} 0.5 g^{\prime}h^{\prime}$ und das ganze ist wohldefiniert!
\end{*example}
Dreiecke lassen allgemeine Flächenberechnung zu - Triangulierung!
%TODO graph

\begin{align}
	F = \biguplus_{n\in \natur} \Delta_n\, (\text{disjunkte Vereinigung } \Delta_i \cap \Delta_k = \emptyset \quad k \neq i)\notag
\end{align} %TODO fix error

%TODO do the rest of this chapter!


