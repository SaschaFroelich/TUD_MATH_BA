\section{Maße}

Sei $E$ eine beliebige nicht-leere Grundmenge.

\begin{definition}[Maß]
	Ein \begriff{Maß} $\mu$ ist eine Abbildung $\mu: \mathscr{A} \to [0,\infty]$ mit folgenden Eigenschaften:
	\begin{itemize}
		\item ($M_0$) $\mathscr{A}$ ist eine $\sigma$-Algebra auf $E$
		\item ($M_1$) $\mu(\emptyset) = 0$
		\item ($M_2$) $(A_n)_{n \in \natur} \subset \mathscr{A}$ paarweise disjunkt $\Longleftarrow \mu\big(\bigcup_{n\in \natur} A_n\big) = \sum_{n\in \natur} \mu(A_n)$
	\end{itemize}
	Gilt für $\mu: \mathscr{A} \to [0,\infty]$ nur $(M_1),(M_2)$, dann heißt $\mu$ \begriff{Prämaß}.
\end{definition}

Für auf- und absteigende Folgen von Mengen schreiben wir auch
\begin{align}
	A_n \uparrow A \Longleftrightarrow A_1 \subset A_2 \subset \dots &\text{ und } A = \bigcup_{n\in \natur} A_n \notag \\
	B_n \downarrow B \Longleftrightarrow B_1 \subset B_2 \subset \dots &\text{ und } B = \bigcap_{n\in \natur} B_n \notag
\end{align}

\begin{definition}
	\begin{itemize}
		\item Es sei $\mathscr{A}$ eine $\sigma$-Algebra auf $E$ und $\mu$ ein Maß. Dann heißt $(E,\mathscr{A})$ \begriff{Messraum} und $(E,\mathscr{A},\mu)$.
		\item Ein Maß mit $\mu(E) < \infty$ heißt \begriff{endliches Maß} und $(E,\mathscr{A},\mu)$ \begriff{endlicher Maßraum}.
		\item Gilt $\mu(E)=1$, dann sprechen wir von einem \begriff{Wahrscheinlichkeitsmaß} (\begriff{W-Maß}) und \\ \begriff{Wahrscheinlichkeitsraum} (\begriff{W-Raum}).
		\item Gibt es eine Folge $(A_n)_{n\in \natur} \subset \mathscr{A}$, sodass $A_n \uparrow E$ und $\mu(A_n) < \infty$, dann heißen $\mu$ und $(E,\mathscr{A},\mu)$ \begriff{$\sigma$-endlich}.
	\end{itemize}
\end{definition}

\begin{proposition}[Eigenschaften von Maßen]
	\proplbl{3_3} %TODO fix labels!
	Es sei $\mu$ ein Maß auf $(E,\mathscr{A})$ und $A,B,A_n, B_n \in \mathscr{A}, n \in \natur$.
	\begin{enumerate}
		\item $A\cap B = \emptyset \Longrightarrow \mu(A \cup B) = \mu(A) + \mu(B)$ (additiv)
		\item $A\subset B \Longrightarrow \mu(A) \leq \mu(B)$ (monoton)
		\item $A \subset B$ \& $\mu(A) < \infty \Longrightarrow \mu(B\setminus A) = \mu(B) - \mu(A)$
		\item $\mu(A \cup B) + \mu(A\cap B) = \mu(A) + \mu(B)$ (stark additiv)
		\item $\mu(A \cup B) \leq \mu(A) + \mu(B)$ (subadditiv)
		\item $A_n \uparrow A \Longrightarrow \mu(A)  = \sup_{n\in \natur} (A_n) = \lim_{n\to \infty} \mu(A_n)$ (stetig von unten)
		\item $B_n \downarrow B$ \& $\mu(B_1) < \infty \Longrightarrow \mu(B_n)  = \sup_{n\in \natur} (B_n) = \lim_{n\to \infty} \mu(B_n)$ (stetig von oben)
		\item $\mu\big(\bigcup_{n\in \natur} A_n\big) \leq \sum_{n\in \natur} \mu (A_n)$ ($\sigma$-additiv)
	\end{enumerate}
\end{proposition}

\begin{proof}
	Wird noch ergänzt später!
\end{proof}

\begin{remark}
	Die Aussagen von \propref{3_3} gelten auf für Prämaße, wenn das zu Grunge leigende Mengensystem groß genug ist. Genauer braucht man dafür:
	\begin{itemize} %TODO hyperlink maybe points from prop 3.3?
		\item a)-e) Stabilität unter endlichen vielen Wiederholungen von $\cup,\cap,\setminus$
		\item f) $A_{n+1}\setminus A_n,\bigcup_{n}^{\infty} A_n \in \mathscr{A}$
		\item g) $B_1 \setminus B_n,B_n \setminus B_{n+1},\bigcap_{n}^{\infty} B_n,B_1\setminus \bigcap_{n}^{\infty} \in \mathscr{A}$
		\item h) $\bigcup_{n}^{m} A_n,\bigcup_{n}^{\infty} A_n \in \mathscr{A}$
	\end{itemize}
\end{remark}

\begin{example}
	\begin{enumerate}
		\item (\begriff{Dirac-Maß}). Es sei $(E,\mathscr{A})$ ein beliebiger Messraum und $x \in E$ fest. Dann ist
		\begin{align}
			\delta_x: \mathscr{A} \to [0,1] \text{ mit } \delta_x(A) := \begin{cases}
				0 & x \not \in A,\\
				1 & x \in A
			\end{cases}\notag
		\end{align}
		ist ein W-Maß, das Dirac-Maß (auch \begriff{$\delta$-Funktion}, \begriff{Einheitsmasse})
		\item Es sei $E=\real$ und $\mathscr{A}$ wie in Beispiel 2.3 e) %TODO set reference, once chap 2 has been typed!
		(d.h. $A \in \mathscr{A} \Longleftrightarrow A \text{ oder } A^C \text{ abzählbar}$). Dann ist
		\begin{align}
		\gamma(A) := \begin{cases}
		0 & A \text{ ist abzählbar},\\
		1 & A^C \text{abzählbar}
		\end{cases}\notag
		\end{align} mit $A \in \mathscr{A}$ und $\gamma$ ist ein W-Maß.
%		\item Es sei $(E, \mathscr{A})$ ein beliebiger Messraum. Dann ist
%		\begin{align}
%	    \vert A\vert := \begin{cases}
%		\#A & x \not \in A,\\
%		+\infty & x \in A
%		\end{cases}\notag
%		\end{align}
		\item gibt noch mehr, werden später ergänzt!
	\end{enumerate}
\end{example}