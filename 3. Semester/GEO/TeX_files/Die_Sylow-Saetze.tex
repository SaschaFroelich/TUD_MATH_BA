\section{Die \person{Sylow}-Sätze}
Sei $G$ eine endliche Gruppe und $p \in \natur$ prim
.
\begin{definition}[$p$-\person{Sylow}-Untergruppe]
	Sei $H \leq G$.
	\begin{enumerate}[label=(\alph*)]
		\item $H$ ist \begriff{$p$-\person{Sylow}-Untergruppe} von $G$ (oder $p$-\person{Sylow}gruppe von $G$) $\Leftrightarrow H$ ist $p$-Gruppe und $p \nmid (G: H)$
		\item $\Syl_p(G) = \{H \leq G \mid H \text{ ist } p\text{-\person{Sylow}gruppe von } G\}$
	\end{enumerate}
\end{definition}

\begin{remark}
	Schreibe $\#G = p^k \cdot m$ mit $p \nmid m$. Dann gilt für $H \leq G$. $H \in \Syl_p(G) \Leftrightarrow \#H=p^k$.
\end{remark}

\begin{example}
	\begin{enumerate}[label=(\alph*)]
		\item $\Syl_3(S_3) = \{A_3\}$
		\item $\Syl_2(S_3) = \{\langle(12)\rangle , \langle(13)\rangle, \langle(23)\rangle\}$
		\item $\Syl_2(S_4) \ni D_4$
	\end{enumerate}
\end{example}

\begin{proposition}
	\proplbl{1_8_4}
	Es gilt $\Syl_p(G) \neq \emptyset$.
\end{proposition}

\begin{proof}
	Induktion nach $n:=\#G=p^k\cdot m$, $p\nmid m$. \\
	\emph{$n=1$:} $1 \in \Syl_2(1)$! \\
	\emph{$n>1$:} Ist $p\nmid n$, so ist $1 \in \Syl_p(G)$. Sei also $k \geq 1$.
	\begin{itemize}
		\item \textbf{1. Fall:} Es existiert $H \lneqq G$ mit $p \nmid (G:H)$. Nach Induktionshypothese existiert $S \in \Syl_p(H)$. Da $p \nmid (G:S)=(G:H)(H:S)$ ist $S \in \Syl_p(G)$.
		\item \textbf{2. Fall:} Es ist $p \mid (G:H)$ für alle $H \lneqq G$. Nach Klassengleichung \propref{1_6_16} ist $0 \equiv n = \#\Z(G) + \sum_{i=1}^{r} (G:C_G(x_i)) = \ord{\Z(G)} \mod p$, wobei $G \setminus \Z(G) = \biguplus_{i=1}^{r}x_i^G$, also $p \mid \#\Z(G)$. Nach \propref{1_7_3} (\person{Cauchy}) existiert ein $g \in \Z(G)$ mit $\ord(g) = p$. $\Rightarrow N:=\langle g \rangle \unlhd G$, $\#N = p$, $\# \lnkset{G}{N} = p^{k-1}m$. Nach Induktionshypothese existiert $\overline{S}\in \Syl_p(\lnkset{G}{N})$, das heißt $\overline{S} = p^{k-1}$. Setze $S:= \pi_N^{-1}(\overline{S}) \leq G$. Dann ist $\#S = \#\Ker(\pi_N)\#\overline{S}=p\cdot p^{k-1} = p^k$, das heißt $S \in \Syl_p(G)$.
	\end{itemize}
\end{proof}

\begin{conclusion}
	Ist $k \in \natur$ mit $p^k \mid \#G$, so existiert $H \leq G$ mit $\#H = p^k$.
\end{conclusion}
\begin{proof}
	\propref{1_8_4} und \propref{1_7_9}.
\end{proof}

\begin{theorem}[\person{Sylow}-Sätze]
	Sei $G$ eine endliche Gruppe.
	\begin{enumerate}[label=(\alph*)]
		\item Jede $p$-Gruppe $H \leq G$ ist in einer $p-$\person{Sylow}gruppe von $G$ enthalten.
		\item Je zwei $p$-\person{Sylow}gruppen von $G$ sind konjugiert.
		\item Für die Anzahl $s_p := \#\Syl_p(G)$ gilt 
		\begin{align}
			s_p = (G:N_G(S)) = 1 \mod p\notag
		\end{align}
		wobei $S \in \Syl_p(G)$ beliebig.
	\end{enumerate}
\end{theorem}

%\begin{proof}
%	Fixiere $S_0 \in \Syl_p(G)$ und betrachte Wirkung von $G$ auf $X = S_0^G \subseteq \Syl_p(G)$ durch Konjugation. Aus $p \nmid (G \colon S_0)$ folgt mit \propref{1_6_11}, dass $p \nmid \#X$. Ist also $H \leq G$ eine $p$-Gruppe, so gibt es ein $S \in \Fix_X(H)$. Man zeigt, dass auch $HS$ eine $p$-Gruppe ist, also $H \leq S$. Hieraus folgen (a) und (b). Setzt man $H = S_0$, so erhält man $\Fix_X(S_0) = \{S_0\}$, woraus (c) mit Hilfe von \propref{1_7_2} folgt.
%\end{proof} % was Fehm proof from the online pdf, very shortened! the oen from the lecture is much more detailed, maybe use this one!

\begin{conclusion}
	\proplbl{1_8_7}
	Sei $S \in \Syl_p(G)$. Genau dann ist $S\unlhd G$, wenn $s_p = 1$.
\end{conclusion}

\begin{conclusion}
	Schreibe $\#G = p^k m$, $p \nmid m$. Dann gilt
	\begin{align}
		s_p\mid m \text{ und } p \mid s_{p}-1\notag
	\end{align}
\end{conclusion}

\begin{example}
	Sei $\#G = pq$, mit Primzahlen $p<q$. Wähle $P \in \Syl_p(G)$, $Q \in \Syl_q(G)$.
	\begin{itemize}
		\item $s_q\mid p$ und $q \mid s_{q}-1 \xRightarrow{p<q} s_q = 1 \xRightarrow{\propref{1_8_7}} Q \unlhd G. \Rightarrow G = P \ltimes Q$ (denn $P \cap Q = 1$ und $PQ=G$).
		\item $s_p \mid q$ und $q \mid s_q -1 \Rightarrow s_p = 1$ oder ($s_p = q$ und $q \equiv 1 \mod p$)
		\begin{itemize}
			\item \textbf{1. Fall} mit $q\neq 1 \mod p$: Dann ist $s_p = 1 \Rightarrow P \unlhd G\Rightarrow G = P \times Q \cong C_p \times C_q \cong C_{pq}$
			\item \textbf{2. Fall} mit $q=1 \mod p$: $\Aut(Q) \cong \Aut(C_q) \overset{\propref{1_4_14}}{\cong} C_{q-1}$ hat genau eine Untergruppe mit Ordnung $p$, also ist $\Hom(P,\Aut(Q))\neq 1$. Es kann deshalb entweder $G =P \ltimes Q = P \times Q \cong C_{pq}$ abelsch oder $G = P \ltimes Q \cong C_p \ltimes_{\alpha} C_q$ mit $\alpha \neq 1$ nicht abelsch geben, z.B. $S_3 \cong C_2 \ltimes_{\alpha} C_3$.
		\end{itemize}
	\end{itemize}
\end{example}
