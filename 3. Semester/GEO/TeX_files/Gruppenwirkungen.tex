\section{Gruppenwirkungen}

Sei $G$ eine Gruppe und $X$ eine Menge.

\begin{definition}[Wirkung, $G$-Menge]
	Eine (rechts-)\begriff{Wirkung} von $G$ auf $X$ ist eine Abbildung
	\begin{align}
		\begin{cases}
			X\times G\to X \\ (x,g)\mapsto x^g
		\end{cases}\notag
	\end{align}
	mit $x\in X$ und $h,g\in G$, wobei
	\begin{itemize}
		\item (W1): $x^{1_G}=x$
		\item (W2): $(x^g)^h=x^{gh}$
	\end{itemize}
	Eine \begriff{$G$-Menge} ist eine Menge $X$ zusammen mit einer Wirkung von $G$ auf $X$.
\end{definition}

\begin{example}
	\proplbl{1_6_2}
	\begin{enumerate}[label=(\alph*)]
		\item Die symmetrische Gruppe $G=\Sym(X)$ wirkt auf $X$ durch $x^\sigma=\sigma(x)$ mit $x\in X$, $\sigma\in G$. So wirkt zum Beispiel $S_n$ auf $X=\{1,...,n\}$.
		\item $G$ wirkt auf $X=G$ durch Multiplikation $x^g=xg$, die sogenannte \begriff{reguläre Darstellung} von $G$.
		\item $G$ wirkt auf $X=G$ durch Konjugation: $x^g=g^{-1}xg$.
		\item $G$ wirkt auf der Menge $\UG(G)$ der Untergruppen von $G$ durch Konjugation $H^g=\{h^g\mid h\in H\}$ mit $H\le G$.
		\item Sind $H,N$ Gruppen, so liefert jedes $\alpha\in\Hom(H,\Aut(N))$ eine Wirkung von $H$ auf $N$ durch $n^h=n^{\alpha(h)}$.
		\item Ist $K$ ein Körper, so wirkt $K^\times$ auf $K$ durch $x^y=xy$ mit $x\in K$ und $y\in K^\times$.
		\item Ist $K$ ein Körper, $n\in\natur$, so wirkt $\GL_n(K)^{op}$ auf $K^n$ durch Multiplikation $x^A=Ax$
	\end{enumerate}
\end{example}

\begin{*anmerkung}
	$^{op}$ ist nötig, weil die Multiplikation "'falsch herum"' definiert wurde. g) wäre ein Beispiel für eine Linkswirkung, also ist es dann mit $^{op}$ eine Rechtswirkung.
\end{*anmerkung}

\begin{remark}
	\proplbl{1_6_3}
	Wirkt $G$ auf $X$, so ist für jedes $g\in G$ die Abbildung
	\begin{align}
		\sigma_g: \begin{cases}
			X\to X \\ x\mapsto x^g
		\end{cases} \notag
	\end{align}
	bijektiv, da $\sigma_g\circ\sigma_{g^{-1}}=\sigma_{g^{-1}}\sigma_g=\sigma_1=\id_X$, also $\sigma_g\in\Sym(X)$ und
	\begin{align}
		\begin{cases}
			G\to \Sym(G) \\ g\mapsto \sigma_g
		\end{cases}\notag
	\end{align}
	ist ein Gruppenhomomorphismus. Umgekehrt liefert jeder Homomorphismus $\sigma: G\to\Sym(G)$ eine Wirkung von $G$ auf $X$ durch $x^g=x^{\sigma(g)}$. Somit steht die Menge der Wirkungen von $G$ auf $X$ in natürlicher Bijektion zu $\Hom(G,\Sym(X))$.
\end{remark}

\begin{definition}[Fixpunkt, Stabilisator, Bahn, Bahnraum, $G$-invariant, treu, transitiv, frei]
	Sei $X$ eine $G$-Menge, $g_0\in G$, $x_0\in X$
	\begin{enumerate}[label=(\alph*)]
		\item $x_0$ ist ein \begriff{Fixpunkt} von $g_0\Leftrightarrow x_0^{g_0}=x_0$
		\item $\Fix(G) = X^G = \{x\in X\mid x^g=x\quad\forall g\in G\}$, die Menge der Fixpunkte von $X$ unter $G$
		\item $G_{x_0} = \Stab(x_0) = \{g\in G\mid x_0^g = x\}$ der \begriff{Stabilisator} von $x_0$ in $G$
		\item $x_0^G = \{x_0^g \mid g\in G\}$, die \begriff{Bahn} von $x_0$ unter G
		\item $\lnkset{X}{G} = \{x^G\mid x\in X\}$, der \begriff{Bahnenraum}
		\item $Y\le X$ ist \begriff{$G$-invariant} $\Leftrightarrow Y^g = \{y^g\mid y\in Y\}\le Y$
		\item Die Wirkung von $G$ auf $X$ ist
		\begin{itemize}
			\item \begriff{treu}, wenn $\bigcap_{x\in X} G_x=1$
			\item \begriff{transitiv}, wenn gilt: $\forall x,y\in X\exists g\in G$: $x^g=y$
			\item \begriff{frei}, wenn $G_x=1$ für alle $x\in X$
		\end{itemize}
	\end{enumerate}
\end{definition}

\begin{remark}
	\begin{enumerate}[label=(\alph*)]
		\item Der Stabilisator $G_{x_0}$ besteht aus den $g\in G$, die $x_0$ als Fixpunkt haben.
		\item Die Wirkung von $G$ auf $X$ ist
		\begin{itemize}
			\item transitiv, wenn es nur eine Bahn gibt, also $\vert\lnkset{X}{G}\vert = 1$
			\item frei, wenn kein $1\neq g\in G$ einen Fixpunkt hat
			\item treu, wenn kein $1\neq g\in G$ alle $x\in X$ als fixiert
		\end{itemize}
	\end{enumerate}
\end{remark}

\begin{example}
	Für $n>1$ wirkt $G=S_n$ auf $X=\{1,...,n\}$ transitiv, treu, aber für $n\ge 3$ nicht frei. Der Stabilisator $G_n$ von $n\in X$ ist eine Untergruppe von $S_n$ isomorph zu $S_{n-1}$.
\end{example}

\begin{example}
	\proplbl{1_6_7}
	Die reguläre Darstellung von $G$ auf $X=G$ ist frei und transitiv:
	\begin{itemize}
		\item frei: $x^g=x\Rightarrow xg=x\Rightarrow g=1$
		\item transitiv: $x,y\in X=G\Rightarrow$ für $g=x^{-1}y$ ist $x^g=y$
	\end{itemize}
\end{example}

\begin{lemma}
	\proplbl{1_6_8}
	Sei $X$ eine $G$-Menge.
	\begin{enumerate}[label=(\alph*)]
		\item Für $x\in X$ ist $G_x\le G$.
		\item Für $x,y\in X$ ist $x^G=y^G$ oder $x^G\cap y^G=\emptyset$.
		\item $\bigcap_{x\in X}=\Ker(\sigma)$, $\sigma:G\to\Sym(X)$ wie in \propref{1_6_3}
		\item Für $x\in X$ und $g\in G$ ist $G_{x^g}=(G_x)^g$
	\end{enumerate}
\end{lemma}
\begin{proof}
	Seien $x,y\in X$, $g,h\in G$
	\begin{enumerate}[label=(\alph*)]
		\item Sei $x^g=x$ und $x^h=x$. Dann
		\begin{align}
			x^{g^h}&=(x^g)^h=x^h=x\Rightarrow gh\in G_x \notag \\
			x^{g^{-1}} &= (x^g)^{g^{-1}} = x^1 = x\RightTorque g^{-1}\in G_x \notag
		\end{align}
		\item $x^g=y^h\Rightarrow x^G=(x^g)^G=(y^h)^H=y^H$
		\item $g\in\bigcap_{x\in X} G_x\Rightarrow\forall x\in X$: $x^g=x\Leftrightarrow \sigma_g=\sigma(g)\id_X$
		\item $h\in G_{x^g}\Leftrightarrow (x^g)^h=x^g\Leftrightarrow x^{ghg^{-1}}=x\Leftrightarrow h^{g^{-1}}\in G_x\Leftrightarrow h\in (G_x)^g$
	\end{enumerate}
\end{proof}

\begin{proposition}[\person{Cayley}]
	Ist $n=\#G<\infty$, so ist $G$ isomorph zu einer Untergruppe der $S_n$.
\end{proposition}
\begin{proof}
	Betrachte die reguläre Darstellung $\sigma:G\to\Sym(G)$. Da diese Wirkung frei ist (\propref{1_6_7}), also insbesondere treu, ist $\sigma$ injektiv (\propref{1_6_8} c), somit $G\cong \Image(\sigma)\le\Sym(G)$. Eine Aufzählung $G=\{g_1,...,g_n\}$ liefert einen Isomorphismus
	\begin{align}
		\phi:\begin{cases}
			S_n\to \Sym(X) \\ \tau\mapsto (g_i\mapsto g_{\tau(i)})
		\end{cases}\notag
	\end{align}
	und somit ist $G\cong \phi^{-1}(\Image(\sigma))\le S_n$.
\end{proof}

\begin{lemma}
	\proplbl{1_6_10}
	Für eine $G$-Menge $X$ und $x\in X$ ist
	\begin{align}
		\phi:\begin{cases}
			\rnkset{G}{G_x}\to x^G \\ G_xg\mapsto x^g
		\end{cases}\notag
	\end{align}
	eine Bijektion.
\end{lemma}
\begin{proof}
	\begin{itemize}
		\item $\phi$ wohldefiniert: $G_xg=G_xg'\Rightarrow g'=gh$ mit $h\in G_x\Rightarrow x^{g'}=x^{hg}=x^g$
		\item $\phi$ surjektiv: klar
		\item $\phi$ injektiv: $x^g=x^{g'}\Leftrightarrow x=x^{g'g^{-1}}\Leftrightarrow g'g^{-1}\in G_x\Leftrightarrow g'\in G_xg\Leftrightarrow Gx_g'=G_xg$
	\end{itemize}
\end{proof}

\begin{proposition}[Bahn-Stabilisator-Satz]
	\proplbl{1_6_11}
	Sei $X$ eine $G$-Menge, $x\in X$. Dann ist
	\begin{align}
		\# x^G = (G:G_x) \notag
	\end{align}
\end{proposition}
\begin{proof}
	\propref{1_6_10}
\end{proof}

\begin{conclusion}[Bahngleichung]
	Ist $X$ eine $G$-Menge und $X=\biguplus_{i=1}^n x_i^G$ die Zerlegung von $X$ in Bahnen (vgl. \propref{1_6_8} c) so ist
	\begin{align}
		\# X= \sum_{i=1}^{n} (G:G_i)\notag
	\end{align}
\end{conclusion}

\begin{definition}[Zentralisator, Normalisator]
	\begin{enumerate}[label=(\alph*)]
		\item Für $h\in H$ ist 
		\begin{align}
			C_G(h) = \{g\in G\mid gh=hg\}\notag
		\end{align}
		der \begriff{Zentralisator} von $h$.
		\item Für $H\le G$ ist 
		\begin{align}
		N_G(h) = \{g\in G\mid gH=Hg\}\notag
		\end{align}
		der \begriff{Normalisator} von $H$.
	\end{enumerate}
\end{definition}

\begin{remark}
	\begin{enumerate}[label=(\alph*)]
		\item Der Zentralisator von $h$ ist der Stabilisator von $h$ unter der Wirkung von $G$ auf $X=G$ durch Konjugation (\propref{1_6_2} c). Es ist die größte Untergruppe $H$ mit $h\in \Z(h)$.
		\item Der Normalisator von $H\le G$ ist der Stabilisator von $H$ unter der Wirkung von $G$ auf $X=\UG(G)$ durch Konjugation (\propref{1_6_2} d). Dies ist die größte Untergruppe $N$ von $G$ mit $H\unlhd N$.
	\end{enumerate}
\end{remark}

\begin{conclusion}
	\proplbl{1_6_15}
	Für $h\in G$ und $H\le G$ ist $C_G(h)\le G$ und $H\unlhd N_G(H)\le G$ und 
	\begin{enumerate}[label=(\alph*)]
		\item $(G:C_G(h))$ ist genau die Anzahl der zu $h$ konjugierten Elemente von $G$
		\item $(G:N_G(H))$ ist genau die Anzahl der zu $H$ konjugierten Untergruppen von $G$
	\end{enumerate}
\end{conclusion}
\begin{proof}
	\propref{1_6_11}
\end{proof}

\begin{conclusion}[Klassengleichung]
	\proplbl{1_6_16}
	Sei $G$ endlich mit Zentrum $Z=\Z(G)$ und sei $x_1,...,x_n$ ein Repräsentantensystem der Konjugationsklassen in $G\setminus Z$. Dann ist
	\begin{align}
		\#G = \#Z + \sum_{i=1}^n (G:C_G(x_i))\notag
	\end{align}
\end{conclusion}
\begin{proof}
	aus \propref{1_6_11} und \propref{1_6_15}, da $G=Z\uplus G\setminus Z=Z\uplus\biguplus_{i=1}^n x_i^G$.
\end{proof}