\section{Abelsche Gruppen}

Sei $G$ ein Gruppe.

\begin{definition}[zyklische Gruppe]
	Eine Gruppe $G$ ist \begriff[Gruppe!]{zyklisch} $\Leftrightarrow G=\langle g\rangle$ für ein $g\in G$.
\end{definition}

\begin{example}
	\begin{enumerate}[label=(\alph*)]
		\item $\whole=\langle 1\rangle$
		\item $\whole/n\whole=\langle\overline{1}\rangle$
		\item $C_n=\langle (1\, 2\, ...\, n)\rangle\le S_n$
		\item Ist $\#G=p$ eine Primzahl, so ist $G$ zyklisch (Übung 6)
	\end{enumerate}
\end{example}

\begin{lemma}
	\proplbl{1_4_3}
	Die Untergruppen von $(\whole,+)$ sind genau die $\langle k\rangle=\whole k$ mit $k\in\natur_0$ und für $k_1,...,k_r\in\whole$ ist $\langle k_1,...,k_r\rangle=\langle k\rangle$ mit
	\begin{align}
		k=\ggT(k_1,...,k_r)\notag
	\end{align}
\end{lemma}
\begin{proof}
	Zwei Beweise sind möglich:
	\begin{enumerate}
		\item Jede Untergruppe von $\whole$ ist ein Ideal von $(\whole,+,\cdot)$ und $\whole$ ist ein Hauptidealring.
		\item Sei $H\le\whole$. Setze $k=\min\{H\cap N\}$, ohne Einschränkung $H\neq\{0\}$.
		\begin{itemize}
			\item $H=\langle k\rangle$: $n\in H\Rightarrow n=qk+r$ mit $q,r\in\whole$, $0\le r<k\Rightarrow r=n-\underbrace{qk}_{k+...+k}\in H\xRightarrow[\text{mal}]{k\text{ mini-}}r=0\Rightarrow n\in\langle k\rangle$
			\item $\langle k_1,...,k_r\rangle=\langle k\rangle\Rightarrow k=\ggT(k_1,...,k_r)$: \\
			$k_i\in\langle k\rangle\Rightarrow k\vert k_i\quad\forall i$ \\
			$k\in\langle k_1,...,k_r\rangle\Rightarrow k=n_1k_1+...+n_rk_r$ mit $n_i\in\whole$ $\exists d\vert k_i\Rightarrow d\vert k\Rightarrow k=\ggT(k_1,...,k_r)$
		\end{itemize}
	\end{enumerate}
\end{proof}

\begin{proposition}[Klassifikation von zyklischen Gruppen]
	Sei $G=\langle g\rangle$ zyklisch. Dann ist $G$ abelsch und
	\begin{enumerate}[label=(\alph*)]
		\item $G\cong (\whole,+)$ \emph{oder}
		\item $G\cong (\whole/n\whole,+)$ mit $n=\#G<\infty$
	\end{enumerate}
\end{proposition}
\begin{proof}
	Betrachte 
	\begin{align}
		\phi: \begin{cases}
		\whole\to G\\ k\mapsto g^k
		\end{cases}\notag
	\end{align}
	$\phi$ ist ein Homomorphismus und surjektiv, da $G=\langle g\rangle$. Nach \propref{1_3_9} ist $G=\Image(\phi)\cong \lnkset{\whole}{\Ker(\phi)}$. Nach \propref{1_4_3} ist $\Ker(\phi)=\langle n\rangle$ für ein $n\in\natur_0$.
	\begin{itemize}
		\item \emph{$n=0$}, so ist $\Ker(\phi)=\langle 0\rangle$, also $\phi$ injektiv und $G\cong\whole$.
		\item \emph{$n>0$}, so ist $G\cong\whole/n\whole$ und $n=\#\whole/n\whole=\#G$.
	\end{itemize}
\end{proof}

\begin{proposition}
	Sei $G=(G,+)=\langle g \rangle$ zyklisch der Ordnung $n \in \natur$.
	\begin{itemize}
		\item Zu jedem $d \in \natur$ mit $d\mid n$ hat $G$ genau eine Untergruppe der Ordnung $d$, nämlich $U_d=\langle \frac{n}{d}g\rangle$
		\item Für $d \mid n$ und $d'\mid n$ ist $U_d \leq U_{d'} \Leftrightarrow d \mid d'$
		\item Für $h_1, \dots , h_k \in \whole$ ist $\langle h_1 g, \dots, h_r g \rangle = \langle eg\rangle = U_{\frac{n}{e}}$ mit $e = \ggT(h_1,\dots,h_r,n)$
		\item Für $k \in \whole$ ist $\ord(kg)= \frac{n}{\ggT(k,n)}$
	\end{itemize}
\end{proposition}
\begin{proof}
	Betrachte wieder $\phi: \begin{cases}
	\overline{k} &\to G \\
	k &\to kg
	\end{cases}$
	\begin{enumerate}
		\item Nach 3.7 und 4.3 liefert $\phi$ Bijektion $\{ e \in \natur \mid n\whole \leq e\whole \} \overset{1.1}{\rightarrow} \{ H \leq G \}$ und $n\whole \leq e\whole \Leftrightarrow e \mid n.$ Ist $H = \phi(e\whole) = \langle e\whole \rangle$, so ist $H \cong \lnkset{e\whole}{n\whole}$, also $n = (\whole \colon n \whole) = (\whole \colon e\whole)\cdot(e\whole \colon n\whole) = e \cdot \#H$
		\item $U_d \leq U_{d'} \Leftrightarrow \langle \frac{n}{d}g \rangle \leq \langle \frac{n}{d'}g \rangle = \frac{n}{d}\whole \leq \frac{n}{d'}\whole \Leftrightarrow \frac{n}{d'} \mid \frac{n}{d} \Leftrightarrow d \mid d'$
		\item Mit $H = \langle h_1, \dots, h_r,n \rangle \leq \whole$ ist $n\whole \leq H$, $\phi(H) = \langle h_1 g, \dots, h_r g$. Nach 4.3 ist $H = \langle e \rangle$ mit $e = \ggT(h_1, \dots, h_r, n)$, somit $\langle h_1 g, \dots, h_r g \rangle = \phi(e\whole) = U_{\frac{n}{e}}$
		\item $\ord(hg) = \#\langle hg \rangle \overset{c)}{=} \#U_{\frac{n}{e}}$ mit $e = \ggT(h,n)$
	\end{enumerate}
\end{proof}

\begin{lemma}
	Seien $a,b \in G$. Kommutieren $a \text{ und } b$ und sind $\ord(a)$ und $\ord(b)$ teilerfremd, so ist
	\begin{align}
		\ord(a,b) = \ord(a)\cdot \ord(b) \notag
	\end{align}
\end{lemma}
\begin{proof}
	Nach 2.12 ist $\langle a \rangle \cap \langle b \rangle = 1$. Ist $(ab)^k = 1 = a^k b^k$, so ist $a^k = b^{-k} \in \langle a \rangle \cap \langle b \rangle = 1$, also $a^k = b^k = 1$. Somit ist $(ab)^k = 1 \Leftrightarrow a^k = 1 \text{ und } b^k =1$ und damit $\ord(ab) = \kgV(\ord(a), \ord(b)) = \ord(a) \cdot \ord(b)$
\end{proof}

\begin{conclusion}
	Ist $G$ abelsch und sind $a,b \in G$ mit $\ord(a) = m < \infty$, $\ord(b) = n = \infty$, so existiert $c \in G$ mit
	\begin{align}
		\ord(c) = \kgV(\ord(a), \ord(b)) \notag
	\end{align}
\end{conclusion}
\begin{proof}
	Schreibe $m = m_0 m'$ und $n = n_0 n'$ mit $m_0 n_0 = \kgV(m,n)$ und $\ggT(m_0, n_0) = 1 \Rightarrow \ord(a^{m'}) = m_0$, $\ord(b^{n'}) = n_0 \Rightarrow \ord(b^{n'} \cdot a^{m'}) \overset{4.6}{=} m_0 \cdot n_0 = \kgV(m,n)$.
\end{proof}

\begin{theorem}[Struktursatz für endlich erzeugte abelsche Gruppen]
	Jede endliche erzeugte abelsche Gruppe $G$ ist eine direkte Summe zyklischer Gruppen
	\begin{align}
		G \cong \whole^{r} \oplus\bigoplus_{i=1}^n \qraum{\whole}{d_i \whole} \notag
	\end{align}
	mit eindeutig bestimmten $d_1, \dots, d_k > 1$ die $d_1 \mid d_{i+1}$ für alle $i$ erfüllen.
\end{theorem}
\begin{proof}
	\begin{itemize}
		\item Existenz: LAAG 2. VIII. 6.14
		\item Eindeutigkeit: Für $d \in \natur$ ist 
		\begin{align}
			\# \lnkset{G}{dG} &= \#\left( \lnkset{\whole}{d \whole}\right)^r \oplus \bigoplus_{i=1}^k \lnkset{\left(\lnkset{\whole}{d_\whole}\right)}{d\cdot\left(\lnkset{\whole}{d_i\whole}\right)} \notag \\
			&\overset{4.5}{=} d^r \cdot \prod_{i=1}^{n} \frac{d_i}{\ggT(d,d_i)}\notag
		\end{align} 
	\end{itemize}
und daraus kann man $r, k, d_1, \dots , d_k$ erhalten.
\end{proof}

\begin{lemma}
	Sei $G=(G,+) = \langle g\rangle$ zyklisch der Ordnung $n \in \natur_0$. Die Endomorphismen von $G$ sind genau die 
	\begin{align}
		\phi_{\overline{k}}: \begin{cases}
		G &\to G \\
		x &\to kx
		\end{cases} \text{ für } \overline{k} = k + n\whole \in \lnkset{\whole}{n\whole}
	\end{align}
	Dabei ist $\phi_{\overline{l}}\circ\phi_{\overline{k}} = \phi_{\overline{kl}}$ für $\overline{k}, \overline{l} \in \lnkset{\whole}{n\whole}$.\notag
\end{lemma}
\begin{proof}
	\begin{itemize}
		\item $\phi_{\overline{k}}$ wohldefiniert $\overline{k_1} = \overline{k_2} \Rightarrow k_2 = k_1 +an$ mit $a \in \whole$, $k_2 x = k_1 x + a n \cdot x = k_1 x \quad\forall x \in G$
		\item $\phi_{\overline{k}} \in \Hom(G,G)$: Klar, da $G$ abelsch
		\item $\overline{k}=\overline{l} \phi_{\overline{k}} = \phi_{\overline{l}}$\\
		$\phi_{\overline{k}} = \phi_{\overline{l}} \Rightarrow \phi_{\overline{k}}(g) = \phi_{\overline{k}}(g) \Rightarrow (k-l)g = 0 \overset{\ord(g) =n}{\Rightarrow} n \mid (k-l) \Rightarrow \overline{k} = \overline{l}$
		\item $\phi \in \Hom(G,G) \Rightarrow \phi = \phi_{\overline{k}}$ für ein $k \in \whole$, $\phi(g) = kg$ für ein $k \Rightarrow \phi = \phi_{\overline{k}}$
		\item $\phi_{\overline{l}} \circ \phi_{\overline{k}} = \phi_{\overline{kl}} := l(kx) = (lk)x$
	\end{itemize}
\end{proof}

\begin{proposition}
	Ist $G$ zyklisch von Ordnung $n \in \natur$, so ist
	\begin{align}
		\Aut(G) \cong \left( \lnkset{\whole}{\whole}\right)^{\times}\notag
	\end{align}
\end{proposition}
\begin{proof}
	$\Aut(G) \subseteq \Hom(G,G) = \{ \phi_{\overline{k}} \colon k \in \lnkset{\whole}{n\whole} \}$\\
	\begin{align}
		\phi_{\overline{k}} \in \Aut(G) &\Leftrightarrow \text{ existiert } \overline{l} \in \lnkset{\whole}{n\whole} \text{ mit }\notag\\ \phi_{\overline{l}} \circ \phi_{\overline{k}} = \phi_{\overline{1}} &\Leftrightarrow \text{ existiert } \overline{l} \in \lnkset{\whole}{n\whole}
	\end{align}
\end{proof}
