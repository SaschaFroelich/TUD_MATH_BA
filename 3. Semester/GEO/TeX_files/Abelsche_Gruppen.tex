\section{Abelsche Gruppen}

Sei $G$ ein Gruppe.

\begin{definition}[zyklische Gruppe]
	Eine Gruppe $G$ ist \begriff[Gruppe!]{zyklisch} $\Leftrightarrow G=\langle g\rangle$ für ein $g\in G$.
\end{definition}

\begin{example}
	\begin{enumerate}[label=(\alph*)]
		\item $\whole=\langle 1\rangle$
		\item $\whole/n\whole=\langle\overline{1}\rangle$
		\item $C_n=\langle (1\, 2\, ...\, n)\rangle\le S_n$
		\item Ist $\#G=p$ eine Primzahl, so ist $G$ zyklisch (Übung 6)
	\end{enumerate}
\end{example}

\begin{lemma}
	\proplbl{1_4_3}
	Die Untergruppen von $(\whole,+)$ sind genau die $\langle k\rangle=\whole k$ mit $k\in\natur_0$ und für $k_1,...,k_r\in\whole$ ist $\langle k_1,...,k_r\rangle=\langle k\rangle$ mit
	\begin{align}
		k=\ggT(k_1,...,k_r)\notag
	\end{align}
\end{lemma}
\begin{proof}
	Zwei Beweise sind möglich:
	\begin{enumerate}
		\item Jede Untergruppe von $\whole$ ist ein Ideal von $(\whole,+,\cdot)$ und $\whole$ ist ein Hauptidealring.
		\item Sei $H\le\whole$. Setze $k=\min\{H\cap N\}$, ohne Einschränkung $H\neq\{0\}$.
		\begin{itemize}
			\item $H=\langle k\rangle$: $n\in H\Rightarrow n=qk+r$ mit $q,r\in\whole$, $0\le r<k\Rightarrow r=n-\underbrace{qk}_{k+...+k}\in H\xRightarrow[\text{mal}]{k\text{ mini-}}r=0\Rightarrow n\in\langle k\rangle$
			\item $\langle k_1,...,k_r\rangle=\langle k\rangle\Rightarrow k=\ggT(k_1,...,k_r)$: \\
			$k_i\in\langle k\rangle\Rightarrow k\vert k_i\quad\forall i$ \\
			$k\in\langle k_1,...,k_r\rangle\Rightarrow k=n_1k_1+...+n_rk_r$ mit $n_i\in\whole$ $\exists d\vert k_i\Rightarrow d\vert k\Rightarrow k=\ggT(k_1,...,k_r)$
		\end{itemize}
	\end{enumerate}
\end{proof}

\begin{proposition}[Klassifikation von zyklischen Gruppen]
	Sei $G=\langle g\rangle$ zyklisch. Dann ist $G$ abelsch und
	\begin{enumerate}[label=(\alph*)]
		\item $G\cong (\whole,+)$ \emph{oder}
		\item $G\cong (\whole/n\whole,+)$ mit $n=\#G<\infty$
	\end{enumerate}
\end{proposition}
\begin{proof}
	Betrachte 
	\begin{align}
		\phi: \begin{cases}
		\whole\to G\\ k\mapsto g^k
		\end{cases}\notag
	\end{align}
	$\phi$ ist ein Homomorphismus und surjektiv, da $G=\langle g\rangle$. Nach \propref{1_3_9} ist $G=\Image(\phi)\cong \lnkset{\whole}{\Ker(\phi)}$. Nach \propref{1_4_3} ist $\Ker(\phi)=\langle n\rangle$ für ein $n\in\natur_0$.
	\begin{itemize}
		\item \emph{$n=0$}, so ist $\Ker(\phi)=\langle 0\rangle$, also $\phi$ injektiv und $G\cong\whole$.
		\item \emph{$n>0$}, so ist $G\cong\whole/n\whole$ und $n=\#\whole/n\whole=\#G$.
	\end{itemize}
\end{proof}