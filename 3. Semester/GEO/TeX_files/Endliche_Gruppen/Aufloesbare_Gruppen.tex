\section{Auflösbare Gruppen}

Sei $G$ eine endliche Gruppe.

\begin{definition}[Normalreihe, Faktoren, Verfeinerung, Kompositionsreihe]
	\proplbl{1_10_1}
	\begin{enumerate}[label=(\alph*)]
		\item Eine \begriff{Normalreihe} von $G$ ist eine Folge von Untergruppen
		\begin{align}
		\label{eq1}
		G = G_0 \properidealright G_1 \properidealright ... \properidealright G_n = 1\tag{$\ast$}
		\end{align}
		Dabei ist $n$ die Länge der Normalreihe, und die Quotienten $\lnkset{G_{i-1}}{G_i}$ heißen die \begriff{Faktoren} der Normalreihe.
		\item Eine Normalreihe $G_0, \dots, G_n$ von $G$ ist eine \begriff{Verfeinerung} einer Normalreihe $H_0,\dots,H_m$  von $G$, wenn $i_1,\dots,i_m$ mit $H_j = G_{i_{j}} \forall j$ gibt.
		\item Eine \begriff{Kompositionsreihe} ist eine Normalreihe, die maximal bezüglich Verfeinerung ist.
	\end{enumerate}
\end{definition}

\begin{remark}
	\begin{enumerate}[label=(\alph*)]
		\item Für eine Normalreihe \eqref{eq1} gilt nach \propref{1_3_7} + Ü27:
		Genau dann ist $\lnkset{G_{i-1}}{G_i}$ einfach, wenn es kein $G_{i-1} \properidealright N \properidealright G_i$ mit $N \unrhd G_{i-1}$ gibt. Das heißt, genau dann ist eine Normalreihe eine Kompositionsreihe, wenn alle ihre Faktoren einfach sind. % different to the lecutre notes, taken from the online version, because fehm avoided using the negation of the normal subgroup symbol!
		\item Jede Normalreihe besitzt eine Verfeinerung, die eine Kompositionsreihe ist.
	\end{enumerate}
\end{remark}

\begin{example}
	\begin{enumerate}[label=(\alph*)]
		\item $S_3$ hat eine Kompositionsreihe
		\begin{align}
			S_3 \properidealright A_3 \properidealright 1 \notag
		\end{align}
		mit Faktoren $\lnkset{S_3}{A_3} \cong C_2$, $\lnkset{A_3}{1} \cong C_3$.
		\item $S_4$ hat die Kompositionsreihe
		\begin{align}
			S_4 \properidealright A_4 \properidealright V_4 \properidealright H=\langle (1\, 2)(3\, 4)\rangle \properidealright 1 \notag
		\end{align}
		mit Faktoren $\lnkset{S_4}{A_4} \cong C_2$, $\lnkset{A_4}{V_4} \cong C_3$, $\lnkset{V_4}{H} \cong C_2$ und $\lnkset{H}{1}\cong C_2$.
		\item $S_5$ hat die Kompositionsreihe
		\begin{align}
			S_5 \properidealright A_5 \properidealright 1\notag
		\end{align}
		mit Faktoren $\lnkset{S_5}{A_5} \cong C_2$, $\lnkset{A_5}{1} \cong A_5$ nach \propref{1_9_11}.
	\end{enumerate}
\end{example}

\begin{theorem}[\person{Jordan}-\person{Hölder}]
	Je zwei Kompositionsreihen von $G$ haben die gleiche Länge und ihre Faktoren stimmen bis auf Isomorphie und Reihenfolge überein.
\end{theorem}
\begin{proof}
	 Induktion über die minimale Länge $n$ einer Kompositionsreihe: Seien
	 \begin{align}
	 	G &= A_0 \properidealright A_1 \properidealright \dotsm \properidealright A_n = 1,\notag \\ %TODO: Fix symbols
	 	G &= B_0 \properidealright B_1 \properidealright \dotsm \properidealright B_m = 1\notag
	\end{align}
	\begin{center}
		\begin{tikzpicture}
			\node at (0,0) (G) {$G$};
			
			\node at (1,1) (A1) {$A_1$};
			\node at (3,1) (A2) {$A_2$};
			\node at (5,1) (Adots) {$\dots$};
			\node at (7,1) (An) {$A_n$};
			
			\node at (1,-1) (B1) {$B_1$};
			\node at (3,-1) (B2) {$B_2$};
			\node at (5,-1) (Bdots) {$\dots$};
			\node at (7,-1) (Bm) {$B_m$};
			
			\node at (2,0) (N) {$N$};
			\node at (4,0) (N1) {$N_1$};
			\node at (6,0) (N2) {$N_2$};
			\node at (8,0) (Ndots) {$\dots$};
			\node at (10,0) (Nl) {$N_l$};
			
			\draw[red] (G) -- (A1);
			\draw[red] (B1) -- (N);
			\draw[blue] (G) -- (B1);
			\draw[blue] (A1) -- (N);
			
			\draw (A1) -- (A2) -- (Adots) -- (An);
			\draw (B1) -- (B2) -- (Bdots) -- (Bm);
			\draw (N) -- (N1) -- (N2) -- (Ndots) -- (Nl);
		\end{tikzpicture}
	\end{center}
	\emph{$n = 0$:} $G = 1$ klar \\
	\emph{$n > 0$:} $G \neq 1 \Rightarrow m > 0$. Es ist $N = A_1 \cap B_1 \unlhd G$, $A_1B_1 \unlhd G$
	\begin{itemize}
		\item \textbf{1. Fall:} $A_1 = B_1$: Behauptung aus Induktionshypothese für $N = A_1 = B_1$
	 	\item \textbf{2. Fall:} $A_1 \neq B_1$: Dann ist $A_1 \properideal A_1 B_1 \unlhd G$ und somit ist $A_1 B_1 = G$, denn $\lnkset{G}{A_1}$ ist einfach
	 	\begin{align}
	 		\lnkset{G}{A_1} &= \lnkset{A_1 B_1}{A_1} \overset{\propref{1_3_10}}{\cong} \lnkset{B_1}{A_1 \cap B_1} = \lnkset{B_1}{N} \notag \\
	 		\lnkset{G}{B_1} &\cong \lnkset{A_1}{N} \notag
	 	\end{align}
		Insbesondere sind $\lnkset{A_1}{N}$ und $\lnkset{B_1}{N}$ einfach. Sei $N=N_0 \properidealright N_1 \properidealright \dots \properidealright N_l = 1$ Kompositionsreihe. Dann ist auch $A_1 \properidealright N \properidealright N_1 \properidealright \dots \properidealright N_l$ ist Kompositionsreihe der Länge $l+1$. Da $A_1 \properidealright A_2 \properidealright \dots \properidealright A_n$ Kompositionsreihe minimaler Länge $n-1$ ist. Es folgt aus der Induktionshypothese, dass $n-1 = l+1$ und dass die Faktoren übereinstimmen. Ebenso sind $B_1 \properidealright B_2 \properidealright \dots \properidealright B_m$ und $B_1 \properidealright N_0 \properidealright N_1 \properidealright \dots \properidealright N_l$ Kompositionsreihe mit Länge $m-1$ und $l+1$. Da $l+1 = n-1 < n$ folgt aus der Induktionshypothese, dass $l+1 = m-1$ und dass die Faktoren übereinstimmen. Also $m = l+2 = n$ und $A_0 \properidealright \dots \properidealright A_n$ und $B_0 \properidealright \dots \properidealright B_n$ haben Faktoren $\lnkset{G}{A_1} \cong \lnkset{B_1}{N}$, $\lnkset{A_1}{N} \cong \lnkset{G}{B_2}$, $\lnkset{N}{N_1}$, $\lnkset{N_1}{N_2}$, $\dots$, $\lnkset{N_{l-1}}{N_l}$
	\end{itemize}
\end{proof}

\begin{definition}[Kompositionsfaktoren, auflösbar]
	\begin{enumerate}[label=(\alph*)]
		\item Die Faktoren einer Kompositionsreihe von $G$ heißen die \begriff{Kompositionsfaktoren} von $G$.
		\item $G$ ist \begriff{auflösbar}, wenn alle Kompositionsfaktoren von $G$ zyklisch sind.
	\end{enumerate}
\end{definition}

\begin{example}
	\proplbl{1_10_6}
	\begin{enumerate}[label=(\alph*)]
		\item $S_3$ hat Kompositionsfaktoren $C_2, C_3$: auflösbar
		\item $S_4$ hat Kompositionsfaktoren $C_2,C_3,C_2,C_2$: auflösbar
		\item $S_n$, $n \geq 5$ hat Kompositionsfaktoren $C_2, A_n$: nicht auflösbar
		\item $G$ ist abelsch $\Longrightarrow G$ ist auflösbar (\propref{1_9_3}c)
		\item $G$ ist $p$-Gruppe $\Longrightarrow G$ ist auflösbar (\propref{1_9_3}d))
		\item $C_4 \text{ und } V_4$ haben Kompositionsfaktoren $C_2$ und $C_2$, aber $C_4 \not\cong V_4$.
	\end{enumerate}
\end{example}

\begin{lemma}
	\proplbl{1_10_7}
	Sei $N\unlhd G$. Genau dann ist $G$ auflösbar, wenn $N$ und $\lnkset{G}{N}$ auflösbar sind.
\end{lemma}
\begin{proof}
	\begin{itemize}
		\item Hinrichtung: Ist $N = N_0 \properidealright \dots \properidealright N_l = 1$ Kompositionsreihe, so kann $G \properidealright N_0 \properidealright \dots \properidealright N_l = 1$ zu einer Kompositionsreihe von $G$ verfeinert werden, Kompositionsfaktoren von $N$ sind die Kompositionsfaktoren von $G$. Somit ist $N$ auflösbar. Ist $\lnkset{G}{H} = H_0 \properidealright \dots \properidealright H_k$ Kompositionsreihe von $\lnkset{G}{N}$, so kann $G = \pi^{-1}_{N}(H_0) \properidealright \pi^{-1}_{N}(H_1) \properidealright \dots \properidealright \pi^{-1}_{N}(H_k) = N \properidealright 1$ zu einer Kompositionsreihe verfeinert werden, die Kompositionsfaktoren von $\lnkset{G}{N}$ sind also Kompositionsfaktoren von $G$. Somit ist $\lnkset{G}{N}$ auflösbar.
		\item Rückrichtung: Sind $N = N_0 \properidealright \dots \properidealright N_l$ und $\lnkset{G}{N}=H_0 \properidealright \dots \properidealright H_k$ Kompositionsreihen, so ist $G = \pi^{-1}_{N}(H_0) \properidealright \dots \properidealright \pi^{-1}_{N}(H_k) = N \properidealright N_1 \properidealright \dots \properidealright N_l$ eine Kompositionsreihe von $G$ und ist damit auflösbar.
	\end{itemize}
\end{proof}

\begin{proposition}
	\proplbl{1_10_8}
	Für $G$ sind äquivalent:
	\begin{enumerate}[label=(\alph*)]
		\item $G$ ist auflösbar.
		\item $G$ hat eine Normalreihe mit zyklischen Faktoren.
		\item $G$ hat eine Normalreihe mit abelschen Faktoren.
		\item $G$ hat eine Normalreihe mit auflösbaren Faktoren.
	\end{enumerate}
\end{proposition}
\begin{proof}
		\begin{itemize}
			\item (a) $\Rightarrow$ (b) $\Rightarrow$ (c) $\xRightarrow{\propref{1_10_6}}$ (d)
			\item (d) $\Rightarrow$ (a): Induktion über die Länge der Normalreihe mit \propref{1_10_7}
		\end{itemize}
\end{proof}

\begin{definition}[Kommutator, Kommutatoruntergruppe]
	Seien $x,y \in G$, $H,K \leq G$.
	\begin{enumerate}[label=(\alph*)]
		\item $[x,y] := x^{-1}y^{-1}xy$, der \begriff{Kommutator} von $x$ und $y$
		\item $[H,K] := \langle[h,k]\mid h \in H, k\in K\rangle$
		\item $G' := [G,G]$, die \begriff{Kommutatoruntergruppe} von $G$
	\end{enumerate}
\end{definition}

\begin{remark}
	Genau dann kommutieren $x$ und $y$ (also $xy = yx$), wenn $[x,y] = 1$. Es gilt $[x,y]^{-1} = [y,x]$.
\end{remark}

\begin{lemma}
	\proplbl{1_10_11}
	Ist $\phi: G \to H$ ein Epimorphismus, so ist $\phi(G') = H'$.
\end{lemma}
\begin{proof}
	Da $\phi([x,y]) = [\phi(x),\phi(y)]$ ist
	\begin{align}
	\phi(G') &= \phi(\langle\{ [x,y] \mid x,y \in H \}\rangle)\notag\\
	&= \langle\phi(\{[x,y] \mid x,y \in G\}) \rangle \notag\\
	&= \langle\{[\phi(x),\phi(y)] \mid x,y \in G\}\rangle\notag \\
	&= \langle \{ [x,y] \mid x,y \in H \}\rangle = H' \notag
	\end{align}
\end{proof}

\begin{lemma}
	\proplbl{1_10_12}
	$G'$ ist der kleinste Normalteiler von $G$ mit $\lnkset{G}{G'}$ abelsch.
\end{lemma}
\begin{proof}
	\begin{itemize}
		\item $G'$ ist charakteristisch $\Rightarrow G' \unlhd G$
		\item $\lnkset{G}{G'} = \pi_{G'} (G) \xRightarrow{\propref{1_10_11}} (\lnkset{G}{G'})' = \pi_{G'}(G') = 1 \Rightarrow [x,y] = 1$ für alle $x,y \in \lnkset{G}{G'}$, das heißt $\lnkset{G}{G'}$ ist abelsch
		\item Sei $N \unlhd G$ mit $\lnkset{G}{N}$ abelsch $\Rightarrow \pi_{N}(G') \overset{\propref{1_10_11}}{=} (\lnkset{G}{N})' = 1 \Rightarrow G' \leq \Ker(\pi_{N}) = N$, das heißt $G'$ ist der kleinste Normalteiler.
	\end{itemize}
\end{proof}

\begin{definition}[Kommutatorreihe]
	Wir definieren die \begriff{Kommutatorreihe}
	\begin{align}
	G = G^{(0)} \properidealright G^{(1)} \properidealright \dots\notag
	\end{align}
	induktiv durch $G^{(0)} = G$ und $G^{(n+1)} = (G^{(n)})'$.
\end{definition}

\begin{proposition}
	\proplbl{1_10_14}
	Ist $G_n \properideal \dots \properideal G_1 \properideal G_0 = G$ eine Normalreihe mit abelschen Faktoren (wir fordern ausnahmsweise nicht, dass $G_n = 1$), so ist $G^{(i)} \le G_i$ für alle $i \leq n$. Insbesondere ist $G$ genau dann auflösbar, wenn $G^{(n)} = 1$ für ein $n$.
\end{proposition}

\begin{proof}
	Induktion über $n$ \\
	\emph{$n = 0$:} klar \\
	\emph{$n-1 \to n$:} Nach Induktionshypothese ist $G^{(n-1)} \leq G_{n-1}$, $\lnkset{G_{n-1}}{G_n}$ abelsch $\Rightarrow G^{(n)} = (G^{(n-1)})' \leq (G_{n-1})' \overset{\propref{1_10_12}}{\leq} G_n$.\\
	Für den "'Insbesondere"'-Fall gilt:
	\begin{itemize}
		\item Hinrichtung: Kompositionsreihe $G = G_0 \properidealright G_1 \properidealright \dots \properidealright G_n$ ist Normalreihe mit abelschen Faktoren $\Rightarrow G^{(n)} \leq G_n = 1 \Rightarrow G^{(n)} = 1$
		\item Rückrichtung: $G = G^{(0)} \properidealright G^{(1)} \properidealright \dots \properidealright G^{(n)} = 1$ mit $n$ minimal ist Normalreihe mit abelschen Faktoren $\xRightarrow{\propref{1_10_8}} G$ ist auflösbar.
	\end{itemize}
\end{proof}

\begin{conclusion}
	Ist $G$ auflösbar und $H \leq G$, so ist auch $H$ auflösbar.
\end{conclusion}

\begin{proof}
	$G$ auflösbar $\xRightarrow{\propref{1_10_14}} G^{(n)} = 1$ für ein $n \Rightarrow H^{(n)} \leq G^{(n)} = 1 \Rightarrow H^{(n)} = 1 \xRightarrow{\propref{1_10_14}} H$ auflösbar.
\end{proof}

\begin{remark}
	Das kleinste $n$ mit $G^{(n)} = 1$ heißt \begriff{Stufe} von $G$. (rank im englischen Sprachraum)
\end{remark}

\begin{remark}
	Es gelten die folgenden tiefen Sätze:
	\begin{itemize}
		\item Satz von \person{Burnside}: Ist $\#G = p^{a}q^{b}$ mit $p,q$ prim, so ist $G$ auflösbar.
		\item Satz von \person{Feit-Thompson}: Ist $\#G$ ungerade, so ist $G$ auflösbar.
	\end{itemize}
\end{remark}