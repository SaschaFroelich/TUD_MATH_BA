\section{Irreduzibilitatskriterien}

Sei $R$ ein faktorieller Ring, $K = \Quot(R)$.

\begin{remark}
	Sei $f \in K[x]$. Wir suchen hinreichende Kriterien dafür, dass $f \in K[x]$ irreduzibel ist.
	\begin{enumerate}[label=(\alph*)]
		\item Ist $c \in K^{\times}$ so gilt: $f$ ist irreduzibel $\Longrightarrow c \cdot f$ ist irreduzibel. Wir können also z.B. ohne Einschränkung annehmen, dass $f$ normiert ist.
		\item $\deg(f) = 1$: $f$ ist irreduzibel und hat Nullstellen in $K$.
		\item $\deg(f) \ge 2$: $f$ hat Nullstellen in $K \Rightarrow f$ reduzibel, da $f(a) = 0 \Rightarrow f(x) = (x-a)\cdot g(x)$, $\deg(g) = \deg(f) - 1 > 0$
		\item $\deg(f) \le 3$: $f$ hat keine Nullstelle in $K \Rightarrow f$ ist irreduzibel, da $f=gh$, $gh \not \in K^{\times}\Rightarrow \deg(g) = 1$ oder $\deg(h) = 1$\\
		\textbf{Achtung}: Für $\deg(f) \ge 4$ ist dies im Allgemeinen falsch! Zum Beispiel $f = x^4 + 2x^2 + 1 = (x^2 +1)^2 \in \ratio[x]$
	\end{enumerate}
\end{remark}

\begin{proposition}[\person{Eisenstein}'sches Irreduzibilitätskriterium]
	\proplbl{2_7_2}
	Sei $f = \sum_{i=0}^{n} a_i x^i \in R[x]\setminus R$ primitiv, und $p \in R$ prim mit $p\nmid a_n$, $p \mid a_i$ für $i = 0, \dots,n-1$, $p^2 \nmid a_0$. Dann ist $f$ irreduzibel in $R[x]$ und somit auch in $K[x]$.
\end{proposition}

\begin{proof}
	to be written ...
\end{proof}

\begin{example}
	Ist $p \in R$ prim, $n > 0$ ist $f = X^n-p$ nach \propref{2_7_2} irreduzibel in $K[x]$
	\begin{enumerate}[label=(\alph*)]
		\item $R = \whole$: $x^2 -5$, $x^7 - 3$ irreduzibel in $\ratio[x]$
		\item $R = F[t]$, $F$ Körper: $x^2-t$, $x^5 + t +1$ irreduzibel in $F[t][x] = F[t,x]$
	\end{enumerate}
\end{example}

\begin{proposition}[Reduktionskriterium]
	Sei $0 \neq f = \sum_{i=0}^{n} a_i x^i \in R[x]\setminus R$ und $p \in R$ prim mit $p \nmid a_n$. Ist $\bar{f} \in (R[x]/(p))[x]$
\end{proposition}