\section{Ringe von Brüchen}

Sei $R$ ein Ring.

\begin{remark}
	\proplbl{2_5_1}
	Wir möchten Unterringe von Körpern charakterisieren. So ist zum Beispiel jeder Unterring $R$ eines Körpers $K$ nullteilerfrei. Ist umgekehrt jeder nullteilerfreie Ring $R$ isomorph zu einem Unterring eines Körpers?
\end{remark}

\begin{definition}[multiplikativ]
	Sei $S \subseteq R$. $S$ ist \begriff{multiplikativ} $\Leftrightarrow 1 \in S$ und für $s,t \in S$ ist $st \in S$. 
\end{definition}

\begin{example}
	\begin{enumerate}[label=(\alph*)]
		\item $S = R^{\times}$
		\item $S = \{ 1, s, s^2, \dots \}$ für ein $s \in R$
		\item $S = \{ x \in R \mid x \text{ ist kein Nullteiler} \}$
		\item $S = R \setminus \mathfrak{p}$ für ein Primideal $\mathfrak{p} \unlhd R$
	\end{enumerate}
\end{example}

\begin{definition}
	Sei $S \subseteq R\backslash \{0\}$ multiplikativ und ohne Nullteiler. Definiere Äquivalenzrelation $\sim$ auf $R \times S$:
	\begin{align}
		(r,s) \sim (r', s') \Leftrightarrow r s' = r's\notag 
	\end{align}
	Schreibe $\frac{r}{s}$ für die $\sim$-Äquivalenzklasse von $(r,s)$ und 
	\begin{align}
		S^{-1}R = \lnkset{R\times S}{\sim} = \left\{ \frac{r}{s} \,\Big\vert\, r \in R,s \in S\right\}\notag	
	\end{align}
	Für $r_1,r_2 \in R$ und $s_1, s_2 \in S$ definiere
	\begin{align}
		\frac{r_1}{s_1} + \frac{r_2}{s_2} &= \frac{r_1 s_2 + r_2 s_1}{s_1 s_2} \notag \\
		\frac{r_1}{s_1} \cdot \frac{r_2}{s_2} &= \frac{r_1 r_2}{s_1 s_2} \notag
	\end{align}
\end{definition}

\begin{lemma}
	Addition und Multiplikation sind wohldefiniert und machen $S^{-1}R$ zu einem Ring.
\end{lemma}

\begin{proof}
	\begin{itemize}
	\item $\sim$ ist Äquivalenzrelation: reflexiv, transitiv, symmetrisch (siehe Analysis Konstruktion rationaler Zahlen)
	\item Multiplikation ist wohldefiniert:
	\begin{align}
	\frac{r_1}{s_1} \cdot \frac{r_2}{s_2} &= \frac{r_1 r_2}{s_1 s_2}, \frac{r_1}{s_1} = \frac{r_1'}{s_1'}\notag \\
	\Rightarrow \frac{r_1'}{s_1'} \cdot \frac{r_2}{s_2} &= \frac{r_1' r_2}{s_1' s_2}\notag \\
	\Rightarrow r_1 r_2 s_1' s_2 &= r_1' r_2 s_1 s_2\notag \\
	\Rightarrow \frac{r_1 r_2}{s_1 s_2} &= \frac{r_1' r_2}{s_1' s_2}\notag
	\end{align}
	\item Addition ist wohldefiniert: analog
	\item $(S^{-1}R, + ,\cdot)$ ist ein Ring: Übung
	\end{itemize}
\end{proof}

\begin{proposition}
	Sei $S \subseteq R \backslash \{0\}$ multiplikativ und ohne Nullteiler. Dann definiert
	\begin{align}
		\iota: \begin{cases}
			R &\to S^{-1}R \\
			a &\mapsto \frac{a}{1}
		\end{cases}\notag
	\end{align}
	einen injektiven Ringhomomorphismus mit $\iota(S) \subseteq (S^{-1}R)^{\times}$.
\end{proposition}

\begin{proof}
	\begin{itemize}
	\item $\iota$ ist Ringhomomorphismus: klar
	\item $\iota$ ist injektiv: $\iota(r) = 0 \Rightarrow \frac{r}{1} = 0 = \frac{0}{1} \Rightarrow r = 0$
	\item $\iota(S) \subseteq (S^{-1}R)^{\times}$: $\iota(s)\cdot \frac{1}{s} = \frac{s}{1} \cdot \frac{1}{s} = \frac{s}{s} = \frac{1}{1} = 1$
	\end{itemize}
\end{proof}

\begin{conclusion}
	Sei $R$ nullteilerfrei. Für $S = R \backslash \{0\}$ ist $S^{-1}R$ ein Körper und $\iota: R \to S^{-1}R$ ist injektiv.
\end{conclusion}

\begin{proof}
	$\frac{r}{s} \neq 0 \Rightarrow r \neq 0 \Rightarrow \frac{s}{r} \in S^{-1}R$, $\frac{r}{s}\cdot \frac{s}{r} = 1.$
\end{proof}

\begin{definition}
	Ist $R$ nullteilerfrei, so heißt
	\begin{align}
		\Quot(R) = (R\backslash \{0\})^{-1}R\notag
	\end{align}
	der \begriff{Quotientenkörper} von $R$. Wir identifizieren $R$ via $\iota$ mit einem Teilring von $\Quot(R)$.
\end{definition}

\begin{conclusion}
	$R$ lässt sich in einem Körper einbetten, das heißt er ist isomorph zu einem Unterring einer Körpers $\Leftrightarrow R$ ist nullteilerfrei.
\end{conclusion}

\begin{proof}
	\begin{itemize}
	\item Hinrichtung: \propref{2_5_1}
	\item Rückrichtung: $R \subseteq \Quot(R)$
	\end{itemize}
\end{proof}

\begin{example}
	\begin{enumerate}
	\item $\Quot(\whole) = \ratio$
	\item $\Quot(\real) = \real$
	\item Für einen Körper $K$ ist $K(x) : \Quot(K[x])$, der \begriff{rationale Funktionenkörper} einer Variable $x$ über $K$.
	\item Für ein Primideal $\mathfrak{p} \unlhd R$ ist $R_{\mathfrak{p}} := (\lnkset{R}{\mathfrak{p}})^{-1}R$, die \begriff{Lokalisierung} von $R$ in $\mathfrak{p}$, z.B. $\whole_{(0)} = \ratio, \whole_{(2)} = \{\frac{m}{n} \colon m \in \whole_{(0)}, n \in \natur, n \text{ ungerade}\}$.
	\item Mit $\whole[i] = \whole+\whole i := \{a+bi \colon a,b \in \whole\} \subset \comp$ ist $\Quot(\whole[i]) = \ratio[i] := \ratio + \ratio i = \{a+bi \colon a,b \in \ratio\}$ (siehe ÜA!, wie sieht Inverses zu $a+b i $ aus?)
	\end{enumerate}
\end{example}

\begin{remark}
	Ist $R$ Teilring einer Körpers $K$, so ist $\Quot(R) \cong \{s^{-1}r \colon r\in R, s \in R\setminus \{0\}\} \subset K$ und wir identifizieren $\Quot(R)$ mit diesem Teilkörper von $K$.
\end{remark}

\begin{proposition}
	Sei $R$ faktoriell. Ist $P$ ein Vertretersystem der Primelemente von $R$ modulo Einheiten, und $K = \Quot(R)$, so hat jedes $x \in K^{\times}$ eine eindeutige Darstellung:
	\begin{align}
		x = \mathfrak{u} \cdot \prod_{p\in P} p^{v_p(x)} \notag
	\end{align}
	mit $\mathfrak{u} \in R^{\times}$ und $v_p(x) \in \whole$ für alle $p \in P$, fast alle gleich Null.
\end{proposition}

\begin{proof}
	Für $x \in R$ folgt dies mit \propref{2_4_7}b), vgl. LAAG VIII 4.7, mit $v_p(x) \in \natur_0$ für alle $p \in P$.
	\begin{itemize}
		\item \textbf{Existenz:} Für $x=\frac{r}{s}$ mit $r,s \in R\setminus {0}$ ist $r = \mathfrak{u} \cdot \prod_{p\in P} p^{v_p(r)}$ und $s = \mathfrak{w}\cdot \prod_{p\in P} p{v_p(s)}$, $\mathfrak{u}, \mathfrak{w} \in R^{\times}, v_p(r), v_p(s) \in \natur_0$.
		\begin{align} %TODO \underbrace exponent maybe possbile?
		\Rightarrow x = \underbrace{\mathfrak{u}\mathfrak{w}^{-1}}_{\in R^{\times}}\cdot \prod_{p\in P} p^{v_p(r)-v_p(s)}\text{, wobei } v_p(r) - v_p(s) \in \whole \notag
		\end{align}
		\item Ist $x=\mathfrak{u} \cdot \prod_{p\in P} p^{\nu_p} = \mathfrak{w} \cdot \prod_{p\in P} p^{\mu_p}$ mit $\mathfrak{u},\mathfrak{w} \in R^{\times}$, $\nu_p,\mu_p \in \whole$, fast alle gleich Null und $\eta_p = - \min\{0, \nu_p,\mu_p\}$, $y:= \prod_{p\in P} p^{\eta_p} \in R$, so ist $xy = \mathfrak{u} \prod_{p\in P} p^{\nu_p + \eta_p} = \mathfrak{w} \prod_{p\in P} p ^{\mu_p + \eta_p}$ Exponenten jeweils $\ge 0$ $\xRightarrow{\text{in }R}$ $\mathfrak{u} = \mathfrak{w}$ und $\nu_p + \eta_p = \mu_p + \eta_p$ für alle $p \in P$.  
		\end{itemize}
\end{proof}

\begin{example} %TODO set option with alph
	\begin{enumerate}
		\item $R = \whole$; $R^{\times} = \{\pm 1\}, P = \{2,3,5,\dots\}, K = \ratio$
		\item $R = F[x], F$ Körper, $R^{\times} = F^{\times}$, $P = \{f \in F[x]\colon f \text{ irreduzibel}, \LC(f) = 1\}, K = F(x)$ (Primelemente von $\whole_{(0)}[i]$)
	\end{enumerate}
\end{example}

\begin{definition}[p-adische Bewertung, p-adic Valuation]
	Die Abbildung
	\begin{align} %TODO is it really v and not \nu?! literature uses \nu instead of v!
		v_p: \begin{cases}
		K^{\times} &\to \whole\\
		x &\to v_p(x)
		\end{cases}\notag
	\end{align}
	heißt die \begriff{p-adische Bewertung} auf $K=\Quot(R)$ mit $p\in R$ prim. Man setzt $v_p(0) := \infty$ mit $k \le \infty$ für alle $k \in \whole \cup \{\infty\}$.
\end{definition}

\begin{lemma}
	\proplbl{2_5_15}
	Sei $R$ faktoriell, $p \in R$ prim, $x,y \in \Quot(R)$.
	\begin{enumerate} %TODO set option i)
		\item $v_p(xy) = v_p(x) + v_p(y)$
		\item $v_p(x+y) \ge \min \{v_p(x), v_p(y)\}$
	\end{enumerate}
\end{lemma}

\begin{proof}
	\begin{enumerate} %TODO set option i), fix brackets!
		\item klar
		\item klar für $x,y \in R$: $p^n \mid x$ und $p^n \mid y \Rightarrow p^n \mid xy$. Für $x,y \in \Quot(R)$ schreibe $x = \frac{x_0}{y}$, $y = \frac{y_0}{a}$ mit $x_0, y_0, a \in R$
		\begin{align}
		\Rightarrow v_p(x+y) = v_p(\frac{x_0+y_0}{a}) &\overset{i)}{=} v_p(\frac{1}{a}) + v_p(x_0 + y_0) \notag \\
		&\ge v_p(\frac{1}{a}) + \min\{v_p(x_0), v_p(y_0)\}\notag \\
		&\overset{i)}{=} \min\{v_p(\frac{x_0}{a}), v_p(\frac{y_0}{a})\}\notag \\
		&\ge v_p(\frac{1}{a}) + \min\{v_p(x_0), v_p(y_0)\}\notag \\
		&\overset{i)}{=} \min\{v_p(\frac{x_0}{a}), v_p(\frac{y_0}{a})\}\notag
		\end{align}
		\end{enumerate}
\end{proof}

\begin{remark}
	Sei $R$ faktoriell.
	\begin{enumerate} %TODO alph
		\item Für $x \in K = \Quot(R)$ gilt: $x \in R \Leftrightarrow v_p(x) \ge 0$ für alle $p \in P$ prim
		\item Je zwei $x,y \in R$ haben ein $\ggT$ und ein $\kgV$:
		\begin{align}
		\ggT(x,y) &= \prod_{p\in P} p^{\min\{v_p(x), v_p(y)\}} \notag \\
		\kgV(x,y) &= \prod_{p\in P} p^{\max\{v_p(x), v_p(y)\}} \notag
		\end{align}
		wobei $P$ ein Vertretersystem der Primelemente von $R$ modulo Einheiten ist. 
	\end{enumerate}
\end{remark}

\begin{definition}[diskrete Bewertung]
	Ist $K$ ein Körper, so heißt jede Abbildung $v: K^{\times} \to \whole$, die (i) und (i) aus \propref{2_5_15} erfüllt, eine \begriff{diskrete Bewertung}. Wir setzen stets $v(0) := \infty$.
\end{definition}

\begin{remark}
	Jede diskrete Bewertung $v: K \to \whole \cup \{\infty\}$ erfüllt
	\begin{enumerate}
		\item $v(1) = v(-1) = 0$
		\item $v(-x) = v(x) \quad \forall x \in K$
		\item $v(x^{-1}) = -v(x) \quad \forall x \in K$
	\end{enumerate}
\end{remark}

\begin{definition}[p-adische Absolutbetrag]
	Sei $p \in \natur$ eine Primzahl. Die Abbildung
	\begin{align}
	\vert \cdot \vert_p: \begin{cases}
	K &\to \real_{\ge 0} \\ %TODO really real?!
	x &\mapsto p^{-v_p(x)} \notag
	\end{cases}
	\end{align}
	heißt der \begriff{p-adische Absolutbetrag} und erfüllt
	\begin{enumerate} %TODO i)
		\item $\vert xy\vert_p = \vert x \vert_p \cdot \vert y\vert_p$
		\item $\vert x+p\vert_p \le \max\{\vert x\vert_p, \vert y\vert_p\} (\le \vert x\vert_p + \vert y\vert_p)$
	\end{enumerate}
\end{definition}