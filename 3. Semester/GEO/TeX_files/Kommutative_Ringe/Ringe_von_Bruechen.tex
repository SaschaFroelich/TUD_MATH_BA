\section{Ringe von Brüchen}

Sei $R$ ein Ring.

\begin{remark}
	\proplbl{2_5_1}
	Wir möchten Unterringe von Körpern charakterisieren. So ist zum Beispiel jeder Unterring $R$ eines Körpers $K$ nullteilerfrei. Ist umgekehrt jeder nullteilerfreie Ring $R$ isomorph zu einem Unterring eines Körpers?
\end{remark}

\begin{definition}[multiplikativ]
	Sei $S \subseteq R$. $S$ ist \begriff{multiplikativ} $\Leftrightarrow 1 \in S$ und für $s,t \in S$ ist $st \in S$. 
\end{definition}

\begin{example}
	\begin{enumerate}[label=(\alph*)]
		\item $S = R^{\times}$
		\item $S = \{ 1, s, s^2, \dots \}$ für ein $s \in R$
		\item $S = \{ x \in R \mid x \text{ ist kein Nullteiler} \}$
		\item $S = R \setminus \mathfrak{p}$ für ein Primideal $\mathfrak{p} \unlhd R$
	\end{enumerate}
\end{example}

\begin{definition}
	Sei $S \subseteq R\backslash \{0\}$ multiplikativ und ohne Nullteiler. Definiere Äquivalenzrelation $\sim$ auf $R \times S$:
	\begin{align}
		(r,s) \sim (r', s') \Leftrightarrow r s' = r's\notag 
	\end{align}
	Schreibe $\frac{r}{s}$ für die $\sim$-Äquivalenzklasse von $(r,s)$ und 
	\begin{align}
		S^{-1}R = \lnkset{R\times S}{\sim} = \left\{ \frac{r}{s} \,\Big\vert\, r \in R,s \in S\right\}\notag	
	\end{align}
	Für $r_1,r_2 \in R$ und $s_1, s_2 \in S$ definiere
	\begin{align}
		\frac{r_1}{s_1} + \frac{r_2}{s_2} &= \frac{r_1 s_2 + r_2 s_1}{s_1 s_2} \notag \\
		\frac{r_1}{s_1} \cdot \frac{r_2}{s_2} &= \frac{r_1 r_2}{s_1 s_2} \notag
	\end{align}
\end{definition}

\begin{lemma}
	Addition und Multiplikation sind wohldefiniert und machen $S^{-1}R$ zu einem Ring.
\end{lemma}

\begin{proof}
	\begin{itemize}
	\item $\sim$ ist Äquivalenzrelation: reflexiv, transitiv, symmetrisch (siehe Analysis Konstruktion rationaler Zahlen)
	\item Multiplikation ist wohldefiniert:
	\begin{align}
	\frac{r_1}{s_1} \cdot \frac{r_2}{s_2} &= \frac{r_1 r_2}{s_1 s_2}, \frac{r_1}{s_1} = \frac{r_1'}{s_1'}\notag \\
	\Rightarrow \frac{r_1'}{s_1'} \cdot \frac{r_2}{s_2} &= \frac{r_1' r_2}{s_1' s_2}\notag \\
	\Rightarrow r_1 r_2 s_1' s_2 &= r_1' r_2 s_1 s_2\notag \\
	\Rightarrow \frac{r_1 r_2}{s_1 s_2} &= \frac{r_1' r_2}{s_1' s_2}\notag
	\end{align}
	\item Addition ist wohldefiniert: analog
	\item $(S^{-1}R, + ,\cdot)$ ist ein Ring: Übung
	\end{itemize}
\end{proof}

\begin{proposition}
	Sei $S \subseteq R \backslash \{0\}$ multiplikativ und ohne Nullteiler. Dann definiert
	\begin{align}
		\iota: \begin{cases}
			R &\to S^{-1}R \\
			a &\mapsto \frac{a}{1}
		\end{cases}\notag
	\end{align}
	einen injektiven Ringhomomorphismus mit $\iota(S) \subseteq (S^{-1}R)^{\times}$.
\end{proposition}

\begin{proof}
	\begin{itemize}
	\item $\iota$ ist Ringhomomorphismus: klar
	\item $\iota$ ist injektiv: $\iota(r) = 0 \Rightarrow \frac{r}{1} = 0 = \frac{0}{1} \Rightarrow r = 0$
	\item $\iota(S) \subseteq (S^{-1}R)^{\times}$: $\iota(s)\cdot \frac{1}{s} = \frac{s}{1} \cdot \frac{1}{s} = \frac{s}{s} = \frac{1}{1} = 1$
	\end{itemize}
\end{proof}

\begin{conclusion}
	Sei $R$ nullteilerfrei. Für $S = R \backslash \{0\}$ ist $S^{-1}R$ ein Körper und $\iota: R \to S^{-1}R$ ist injektiv.
\end{conclusion}

\begin{proof}
	$\frac{r}{s} \neq 0 \Rightarrow r \neq 0 \Rightarrow \frac{s}{r} \in S^{-1}R$, $\frac{r}{s}\cdot \frac{s}{r} = 1.$
\end{proof}

\begin{definition}
	Ist $R$ nullteilerfrei, so heißt
	\begin{align}
		\Quot(R) = (R\backslash \{0\})^{-1}R\notag
	\end{align}
	der \begriff{Quotientenkörper} von $R$. Wir identifizieren $R$ via $\iota$ mit einem Teilring von $\Quot(R)$.
\end{definition}

\begin{conclusion}
	$R$ lässt sich in einem Körper einbetten, das heißt er ist isomorph zu einem Unterring einer Körpers $\Leftrightarrow R$ ist nullteilerfrei.
\end{conclusion}

\begin{proof}
	\begin{itemize}
	\item Hinrichtung: \propref{2_5_1}
	\item Rückrichtung: $R \subseteq \Quot(R)$
	\end{itemize}
\end{proof}

\begin{example}
	\begin{enumerate}
	\item $\Quot(\whole) = \ratio$
	\item $\Quot(\real) = \real$
	\end{enumerate}
\end{example}