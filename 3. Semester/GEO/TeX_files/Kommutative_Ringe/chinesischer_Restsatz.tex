\section{Chinesischer Restsatz und Einheitengruppen}

Sei $R$ ein Ring.

\begin{definition}[teilerfremd]
	Zwei Ideale $I,J \unlhd R$ heißen \begriff{teilerfremd}, wenn $I+J = R$.
\end{definition}

\begin{example}
	\proplbl{2_3_2}
	\begin{enumerate}[label=(\alph*)]
		\item Sei $R = \whole$, $n,m \in \whole$
			\begin{itemize}
				\item $(n) \cap (m) = (\kgV(n,m))$
				\item $(n) + (m) = (\ggT(n,m))$ (LAAG VIII 3.8) oder \propref{1_4_3}
			\end{itemize}
		Insbesondere sind $(n)$ und $(m)$ genau dann teilerfremd, wenn $(\ggT(n,m)) = (1)$, also wenn $n$ und $m$ teilerfremd sind.
		\item Sind $I,J \unlhd R$ maximal und voneinander verschieden, so auch teilerfremd.
	\end{enumerate}
\end{example}

\begin{theorem}[Chinesischer Restsatz]
	\proplbl{2_3_3}
	Sind $I_1, \dots , I_k \unlhd R$ paarweise teilerfremd, so induzieren die Abbildungen
	\begin{align}
		\pi_i = \pi_{I_i} : R \to \lnkset{R}{I_i}\notag
	\end{align}
	einen Isomorphismus
	\begin{align}
		\bar{\pi}: \lnkset{R}{\bigcap_{i=1}^{k} I_i} \xrightarrow{\cong} \prod_{i=1}^{k} \lnkset{R}{I_i} \notag
	\end{align}
\end{theorem}

\begin{proof}
	Wende Homomorphiesatz an auf
	\begin{align}
	\pi: \begin{cases}
	R &\to \prod_{i=1}^{n} \lnkset{R}{I_i} \\
	x &\mapsto (\pi_1(x), \dots, \pi_n(x))
	\end{cases} \notag
	\end{align}
	\begin{itemize}
		\item $\Ker(\pi) = \bigcap_{i=1}^{k}\Ker(\pi_i) = \bigcap_{i=1}^{k} I_i$
		\item $\pi$ ist surjektiv: Sei $y=(y_1, \dots, y_k) \in \prod_{i=1}^{k} R/I_i$. Für $i = 1,\dots,k$ wähle $x_i \in R$ mit $\pi_i(x) = y_i$. Fixiere ein $I$. Für $j \neq i$ ist $I_i + I_j = R$. Insbesondere existiert $a_j \in I_i$ und $b_j \in I_j$ mit $a_j + b_j = 1$. Setze $e_i = \prod_{j\neq i} b_j \in R$.
		\begin{align}
		\pi_{\nu}(e_i) = \prod_{j \neq i} \pi_{\nu}(b_j) = \begin{cases}
		\prod_{j\neq i} \pi_i(1-a_j) \overset{(\ast)}{=} 1 & \nu = 1\\
		\underbrace{\pi_{\nu}(b_{\nu})}_{=0}\prod_{\substack{j\neq i \\ j \neq \nu}} \pi_{\nu}(b_j) = 0 & \nu \neq i
		\end{cases}\notag %TODO make the case clear when j \neq \nu!!! also what means the (*) over the equal sign?!
		\end{align}
		Für $x = \sum_{i=1}^{k}x_i \cdot e_i$ ist 
		\begin{align}
			\pi_{\nu}(x) = \sum_{i=1}^{k} \pi_{\nu}(x_i)\cdot\pi_{\nu}(e_i) = \pi_{\nu}(e_i) = y_{\nu} \notag
		\end{align}
		somit $\pi(x) = (\pi_1(x), \dots, \pi_k(x)) = (y_1, \dots, y_k)$. 
	\end{itemize}
\end{proof}

\begin{conclusion}
	\proplbl{3_3_4}
	Sind $u_1, \dots, u_k \in \natur$ paarweise teilerfremd, und $n = n_1\cdot n_2 \cdots n_k$, so ist
	\begin{align}
		\whole / n\whole &\cong \whole / n_1 \whole \times \cdots \times \whole / n_k \whole \notag \\
		(\whole / n\whole)^{\times} &\cong (\whole / n_1 \whole)^{\times} \times \cdots \times (\whole / n_k \whole)^{\times} \notag
	\end{align}
\end{conclusion}

\begin{proof}
	Sei $n = n_1 \cdot n_k = \kgV(n_1,\dots,n_k)$ \\
	$\xRightarrow{\propref{2_3_2}} (n) = \bigcap_{i=1}^{n} (n_i)$\\
	$\xRightarrow[\propref{2_3_2}]{\propref{2_3_3}} \whole / (n) \cong \prod_{i=1}^{k} \whole / (n_1)$\\
	und die andere Aussage folgt mit \propref{2_1_8}. 
\end{proof}

\begin{remark}
	\begin{enumerate}
		\item Insbesondere gilt: Sind $n_1,\dots,n_k \in \natur$ paarweise teilerfremd und $n=n_1\cdots n_k$, so hat das System von Kongruenzen
		\begin{align}
			x &= x_1 \mod n_1 \notag \\
			&\vdots \notag \\
			x & = x_k \mod n_k \notag
		\end{align}
		für jede Wahl von $x_1, \dots, x_k \in \whole$ eine eindeutig bestimmte Lösung $x \in \{ 0, \dots, n-1 \}$, und die anderen Lösungen in $\whole$ sind genau die Zahlen der Form $x +nm$ nur mit $m \in \whole$.
		\item Der Beweis liefert ein Verfahren, um $x$ zu bestimmen (modulo erweiterter euklidischer Algorithmus).
		\item Es folgt auch: Sind $n,m$ teilerfremd, so ist $C_{nm} \cong C_n \times C_m$.
		\item Die Vorraussetzung der Teilerfremdheit ist hier notwendig, so ist z.B. $C_4 \not \cong C_2 \times C_2$.
	\end{enumerate}
\end{remark}

\begin{erinnerung}
	\proplbl{2_3_6}
	\begin{align}
		\varphi(n) = \#(\whole / n \whole)^{\times}\notag
	\end{align}
	Für $n = p$ Primzahl: $\varphi(p) = p-1$ und $(\whole/p \whole)^{\times} \cong C_{p-1}$ (vgl \propref{1_4_14}). Es ist %TODO fix references in other chapters, thats not usefull :/
	\begin{align}
		(\whole / n \whole)^{\times} &= \{ \bar{k} \mid \exists \bar{m} \in \whole\colon \bar{m}\cdot \bar{k} = \bar{1}\} \notag\\
		&= \{ \bar{k} \mid \bar{1} \in \langle \bar{k} \rangle \} \notag\\
		&= \{ \bar{k} \mid \ord{\whole / n \whole}(\bar{k}) = n \} \notag\\
		&\overset{\propref{1_4_5}}{=} \{\bar{k} \mid \ggT(k,n) = 1\}\notag 
	\end{align}
\end{erinnerung}

\begin{proposition}
	Seien $n,m \in \natur$
	\begin{enumerate}
		\item Ist $\ggT(n,m) = 1$, so ist $\upphi(n\cdot m) = \phi(n)\cdot \varphi(m)$
		\item Für $p \in \natur$ prim, $r \in \natur$ ist $\upphi(p^r) = (p-1)p^{r-1}$
		\item Sind $p_1,\dots, p_k$ die verschiedenen Primteiler von $n$, so ist
		\begin{align}
		\upphi(n) = n \cdot \prod_{i=1}^{k}\left( 1- \frac{1}{p_i}\right) \notag
		\end{align}
	\end{enumerate}
\end{proposition}

\begin{proof}
	\begin{enumerate}
		\item \propref{2_3_4}
		\item Für $k \in \{0,\dots,p^{r-1}\}$ gilt:
		\begin{align}
			\bar{k} \in (\whole / p^r \whole)^{\times} &\Longleftrightarrow \ggT(k, p^k) = 1\notag \\
			&\Longleftrightarrow p \nmid k\notag \\
			&\Longleftrightarrow k \not \in \{0, p, 2p, \dots, p^r-p\}\notag
		\end{align}
		Also $\#(\whole /p^r \whole)^{\times} = p^r - p^{r-1} = (p-1)p^{r-1}$
		\item $n = p_i^{e_1}\cdots p_k^{e_k} \xRightarrow{a)} \upphi(n) = \prod_{i = 1}^{k} \upphi(p_i^{e_i}) \overset{b)}{=} \prod_{i=1}^{k} (p_i - 1)\cdot p_i^{e_i - 1}$\\
		$= n \prod_{i=1}^{k}(p_i - 1)\cdot p_i^{-1} = n \prod_{i=1}^{k}(1-\frac{1}{p_i})$
	\end{enumerate}
\end{proof}

\begin{lemma}
	\proplbl{2_3_8}
	Sei $p \in \natur$ prim, $e \in \natur$ mit $p^{e} > 2$. Dann gilt für $a,b \in \whole$:
	\begin{align}
		a \equiv 1 + bp^{e} \mod p^{e+1} \Rightarrow a^p \equiv 1 + bp^{e+1} \mod p^{e+2} \notag
	\end{align}
\end{lemma}

\begin{proof}
	Schreibe $a= 1 bp^{e} + b^{'}p^{e+1} = 1 + cp^{e}$ mit $c = b + b^{'}p$, $b^{'} \in \whole$.\\
	$\Rightarrow a^p = \sum_{i=0}^{p} \binom{p}{i} c^i p^{e_i} = 1+cp^{e+1} + \sum_{i=2}^{p} \binom{p}{i} c^i p^{e_i}$
	\begin{itemize}
		\item für $2 \le i < p$ gilt: $p \mid \binom{p}{i}$ und $e_i \ge e + 1$. Also folgt $\binom{p}{i}c^i p^{e_i} \equiv 0 \mod p^{e + 2}$
		\item für $i = p$ gilt: Es ist $e\cdot i = e\cdot p \ge 2$, dann $e \ge 2$ oder $p \ge 3$. Also folgt
		\begin{align}
			\binom{p}{p}c^{p}\cdot p^{cp} \equiv 0 \mod p^{e+2}\notag
		\end{align}
	\end{itemize}
	Insgesamt: $a^p \equiv cp^{e+1} \equiv 1 + bp^{e+1} \mod p^{e+2}$.
\end{proof}

\begin{theorem}
	\proplbl{2_3_9}
	Sei $n \in \natur$ mit Primzerlegung
	\begin{align}
		n = \prod_{i=1}^{k} p_i^{r_i}\notag
	\end{align}
	Dann ist
	\begin{align}
		(\whole / n \whole)^{\times} \cong \prod_{i=1}^{k} (\whole / p_i^{r_i} \whole)^{\times}\notag
	\end{align}
	wobei gilt:
	\begin{enumerate}
		\item Für $p > 2$ prim und $r \in \natur$ ist
		\begin{align}
			(\whole / p^{r}\whole)^{\times} &\cong \whole / (p-1)p^{r-1}\whole \notag \\
			&\cong \whole / (p-1) \whole \times \whole / p^{r-1} \whole \notag
		\end{align} 
		zyklisch.
		\item Für $r \ge 2$ und $p =2$ ist
		\begin{align}
			(\whole /2^r \whole)^{\times} \cong \whole / 2 \whole \times \whole / 2^{r-2}\whole \notag
		\end{align}
		\item $(\whole / 2 \whole)^{\times} = \{1\}$
	\end{enumerate}
\end{theorem}

\begin{proof}[\propref{2_3_9}1)]
	Zu zeigen ist, dass $G = (\whole / p^r \whole)^{\times}$ zyklisch ist, dann ist $G \cong \whole / \upphi(p^r) \whole \overset{\propref{2_3_7}}{=} \whole/ (p-1)p^{r-1}\whole \underset{\propref{2_3_4}}{\cong} \whole / (p-1) \whole \bigoplus \whole / p^{r-1} \whole$. Setze $H = (\whole / p \whole)^{\times}$. Der Ringhomomorphismus (z.B. aus \propref{2_2_8} mit $\phi = \pi(p)\cdot \whole \to \whole/p \whole (\ast)$)
	\begin{align}
		\pi: \begin{cases}
		\whole / p^r \whole &\to \whole / p \whole \\
		x + p^r \whole &\mapsto x + p\whole \notag
		\end{cases}
	\end{align}
	$(\ast)$ mit Ideal $I = p^r \whole)$\\
	liefert einen Gruppenhomomorphismus
	\begin{align}
		\pi^{\times} = \pi\mid_{G} \colon G \to H \notag
	\end{align}
	mit $N = \Ker(\pi^{\times}) \le G$. Es sind verschiedene Dinge zu zeigen:
	\begin{itemize}
		\item $\pi(G) \subseteq H$ und $\pi^{\times}(G) = H$: klar, denn für $x \in \whole$ gilt: $x + p^r \whole \in G \Leftrightarrow \ggT(x,p^r) = 1 \Leftrightarrow \ggT(x,p) = 1 \Leftrightarrow x + p\whole \in H$
		\item $G/N \cong H$ \propref{1_3_9}
		\item $\# N = p^{r-1}$: $\# N = \frac{\#G}{\#H} = \frac{\upphi(p^r)}{\upphi(p)} = \frac{(p\cdot 1)p^{r-1}}{p-1}$
		\item $H$ ist zyklisch: \propref{2_3_6}
		\item $N$ ist zyklisch: Sei $a = 1+p$ mit Restklassen $\bar{a} = 1+p + p\whole \in G \Rightarrow \bar{q} \in N$ und $\ord(\bar{a}) \mid \# N = p^{r-1}$. Es gilt $a \equiv 1+1p \mod p^2$. Mit \propref{2_3_8} folgt dass $a^p\equiv 1 + p^2 \mod p^3$. Die $r-2$-fache Anwendung von \propref{2_3_8} bringt $a^{p^{r-2}} \equiv 1 + p^{r-1 \mod p^r}$. Da $p^{r-1} \neq 0$ folgt $\bar{a}^{p^{r-2}} \not \equiv 1 \mod p^r$, also $\ord(\bar{a})_{G} = p^{r-1}$, damit $N = \langle \bar{a} \rangle$.
		\item $G$ ist zyklisch: Sei $H = \langle \bar{n}(\bar{b}) \rangle$ mit $b \in \whole$. Da $\#H = p-1$ folgt $\ord(\pi(\bar{b})) = p-1$. Das bedeutet, dass $\ord(\bar{b}) \mid p-1$. Nach \propref{1_4_7} existiert $\bar{c} \in G$ mit $\ord(\bar{c}) = \kgV(\ord(\bar{a}, \bar{b})) = (p-1)\cdot p^r$, d.h. $G = \langle \bar{c} \rangle$.
	\end{itemize}
\end{proof}