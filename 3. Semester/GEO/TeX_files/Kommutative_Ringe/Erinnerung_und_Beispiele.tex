\section{Erinnerung und Beispiele}

\begin{erinnerung}
	Ein \begriff{Ring} ist eine abelsche Gruppe $(R,+)$ zusammen mit einer Verknüpfung $\cdot : R\times R \to R$ die Assoziativität und Distributivität erfüllt. Eine Teilmenge $\emptyset \neq S \subseteq R$ ist ein \begriff{Unterring} oder \begriff{Teilring} von $R$, wenn $S$ abgeschlossen unter Addition, Subtraktion und Multiplikation ist. Eine Abbildung $\phi : R \to R'$ zwischen Ringen ist ein \begriff{Ringhomomorphismus}, wenn $\phi(r_1 + r_2) = \phi(r_1) + \phi(r_2) \text{ und } \phi(r_1 r_2) = \phi(r_1) \phi(r_2)$ und in diesem Fall ist
	\begin{align}
		\Ker(\phi)=\phi^{-1}(\{0\}) \notag
	\end{align}
	der \begriff{Kern} von $\phi$.
\end{erinnerung}

\begin{remark}
	In dieser Vorlesung bedeutet ``Ring'' \textbf{immer} kommutativer Ring mit Einselement, das heißt $(R,\cdot)$ bildet ein kommutatives Monoid mit Einselement $1_R$. Wir fordern dann zusätzlich, dass Unterringe von $R$ das Einselement von $R$ enthalten und dass Ringhomomorphismen $\phi : R \to R'$ das Einselement von $R$ auf das Einselement von $R'$ abbilden.
\end{remark}

\begin{example}
	\begin{enumerate}[label=(\alph*)]
		\item Der Ring $\whole$ der ganzen Zahlen.
		\item Der Restklassenring $\whole / n \whole$ für $n \in \natur$.
		\item Die Körper $\ratio$, $\real$, $\comp$.
		\item Der Nullring $R = \{0\}$
	\end{enumerate}
\end{example}

Seien $R$, $S$ Ringe. (Die meisten Beweise sind dem LAAG 1+2 Skript zu entnehmen!)

\begin{proposition}
	Ein Ringhomomorphismus $\phi: R \to S$ ist ein Isomorphismus (das heißt bijektiv), wenn es einen Ringhomomorphismus $\psi: S \to R$ mit $\psi \circ \phi = \id_R$ und $\phi \circ \psi = \id_{S}$ gibt.
\end{proposition}

\begin{proposition}
	Ein Ringhomomorphismus $\phi: R \to S$ ist genau dann injektiv, wenn $\Ker(\phi) =\{0\}$.
\end{proposition}

\begin{definition}[invertierbar, Einheit, Nullteiler, nullteilerfrei]
	Ein $x \in R$ heißt \begriff{invertierbar} oder eine \begriff{Einheit}, wenn es $y\in R$ mit $xy=1$ gibt, und die Menge $R^{\times}$ der Einheiten bildet eine Gruppe unter Multiplikation.
	
	Ein $x \in R$ ist ein \begriff{Nullteiler}, wenn es $0 \neq y \in R$ mit $xy=0$ gibt, und $R$ ist \begriff{nullteilerfrei}, wenn es keinen Nullteiler $0\neq x \in R$ gibt.
\end{definition}

\begin{example}
	\begin{enumerate}[label=(\alph*)]
		\item $\whole$ ist nullteilerfrei, $\whole^{\times} = \mu_2 = \{ \pm 1 \}$.
		\item $\whole / n \whole$ ist genau dann nullteilerfrei, wenn $n$ prim ist.
	\end{enumerate}
\end{example}

\begin{example}
	\proplbl{2_1_8}
	Für eine Familie von Ringen $(R_i)_{i \in I}$ wird $\prod_{i \in I} R_i$ durch komponentenweise Addition und Multiplikation zu einem Ring, genannt das \begriff{direkte Produkt} der $R_i$. Bezeichnet $1_{R_i}$ das Einselement von $R_i$, so ist $(1_{R_i})$ das Einselement von $\prod_{i \in I} R_i$ und 
	\begin{align}
		\left(\prod_{i\in I} R_i\right)^{\times} = \prod_{i \in I}R_i^\times \notag
	\end{align}
\end{example}

\begin{example}
	Der \begriff{Polynomring} einer Variablen $x$ über $R$ ist 
	\begin{align}
		R[x] = \left\{ \sum_{i=0}^{\infty} a_i x^i \mid a_i \in R, \text{ fast alle } a_i = 0\right\} \notag
	\end{align}
	mit der Addition und Multiplikation
	\begin{align}
		\sum_{i=0}^{\infty} a_i x^i + \sum_{i=0}^{\infty} b_i x^i &= \sum_{i=0}^{\infty} (a_i + b_i) x^i \notag \\
		\left(\sum_{i=0}^{\infty} a_i x^i\right) \cdot \left(\sum_{j=0}^{\infty} b_j x^j\right) &= \sum_{k=0}^{\infty} \left(\sum_{i+j=k}^{\infty} a_i b_j\right) x^k \notag
	\end{align}
	Ist $f = \sum_{i=0}^n a_i x^i \in R[x]$ mit $a_n \neq 0$, so ist $\deg(f) = n$ der \begriff{Grad} von $f$ (mit $\deg(0) = -\infty$) und $\LC(f) = a_n$ der \begriff{Leitkoeffizient} von $f$, $f$ heißt \begriff{normiert}, wenn $\LC(f) = 1$.
\end{example}

\begin{proposition}
	\proplbl{2_1_10}
	Seien $f,g \in R[x]$.
	\begin{enumerate}[label=(\alph*)]
		\item $\deg(f + g) \leq \max\{ \deg(f), \deg(g) \}$
		\item $\deg(f \cdot g) \leq \deg(f) + \deg(g)$
		\item Ist $f \neq 0$ und $\LC(f)$ kein Nullteiler, so ist $\deg(fg) = \deg(f) + \deg(g)$.
	\end{enumerate}
\end{proposition}

\begin{proof}
	Siehe LAAG I.6.4.
\end{proof}

\begin{conclusion}
	Ist $R$ nullteilerfrei, so auch $R[x]$ und $(R[x])^{\times} = R^{\times}$.
\end{conclusion}

\begin{proof}
	\begin{itemize}
		\item Ist $fg=0$, so ist
		\begin{align}
		-\infty = \deg(0) = \deg(fg) \overset{\propref{2_1_10}}{=} \deg(f) + \deg(g) \notag
		\end{align}
		folglich $f=0$ oder $g=0$.
		\item 	Ist $fg = 1$, so ist
		\begin{align}
		0 = \deg(1) = \deg(fg) \overset{\propref{2_1_10}}{=} \deg(f) + \deg(g) \notag
		\end{align}
		folglich $\deg(f) = \deg(g) = 0$, das heißt $f,g \in R$.
	\end{itemize}
\end{proof}

\begin{proposition}[universelle Eigenschaft des Polynomrings]
	\proplbl{2_1_12}
	Ist $\phi : R \to S$ ein Ringhomomorphismus und $s \in S$, so gibt es genau einen Ringhomomorphismus $\phi_s : R[x] \to S$ mit 
	\begin{align}
	\phi_{s}|_R = \phi \text{ und } \phi_{s} (x) = S\notag
	\end{align}
\end{proposition}

\begin{proof}
	Ist $\phi_s:R[x] \to S$ ein Ringhomomorphismus mit $\phi_{s}|_R = \phi$ und $\phi_{s}(x) = S$, so ist
	\begin{align}
		\phi_{s}\left(\sum_{i=0}^{\infty} a_i x^i\right) = \sum_{i=0}^{\infty} \phi_{s}(a_i) \phi_{s}(x^i) = \sum_{i=0}^{\infty} \phi(a_i) s^i \notag
	\end{align}
		eindeutig bestimmt. Umgekehrt ist das so definierte $\phi_s$ ein Ringhomomorphismus (Übung), der $\phi_{s}|_R$ und $\phi_{s}(x) =S$ erfüllt.
\end{proof}

\begin{remark}
	Insbesondere hat man für $a\in R$ den Einsetzungshomomorphismus:
	\begin{align}
		\upphi_{a}: \begin{cases}
		R[x] &\to R \\
		f &\mapsto f(a) 
		\end{cases}\notag
	\end{align}
	gegeben durch $\upphi_{a}\mid_{R} = \id_R$ und $\upphi_{a}(x) = a$. Dies liefert eine Abbildung
	\begin{align}
	\begin{cases}
	R[x] &\to \Abb(R,R) \\
	f &\mapsto \tilde{f},\, \tilde{f}(a) = \upphi_a(f) 
	\end{cases}\notag
	\end{align}
	Diese Abbildung ist im Allgemeinen \textbf{nicht injektiv}! Zum Beispiel für $R = \whole / 2\whole$ und $f = x^2 + x$ ist $f(\overline{0}) = \overline{0}$, $f(\overline{1}) = \overline{0}$, aber $\tilde{f} = \tilde{0}$, aber $f\neq 0$.
\end{remark}

\begin{proposition}[Polynomdivision]
	\proplbl{2_1_14}
	Sei $0 \neq g \in R[x]$ mit $\LC(g) \in R^{\times}$. Zu jedem Polynom $f \in R[x]$ gibt es eindeutig bestimmte $q,r \in R[x]$ mit $f = qg +r$ und $\deg(r) < \deg(g)$.
\end{proposition}

\begin{proof}
	Wie im Fall $R = K$ ein Körper.
	\begin{itemize}
		\item \textbf{Eindeutigkeit:} Sei $f = q_1 g+ r_1 = q_2 g + r_2$ und $\deg(r_1) <\deg(g) \Rightarrow r_1 - r_2 = (q_2 - q_1)g$. Da $\LC(g) \in R^{\times}$ ist $\LC(g)$ kein Nullteiler $\overset{\propref{2_1_10}}{\Rightarrow} \underbrace{\deg(r_1 - r_2)}_{<\deg(g)} = \deg(q_2 - q_1) + \deg(g)$ \\
		$\Rightarrow \deg(q_2 - q_1) < 0 \Rightarrow q_1 = q_2$ und $r_1 = r_2$
		\item \textbf{Existenz:} Sei $f = \sum_{i=0}^{n} a_i x^i$, $a_n \neq 0$ und $g = \sum_{j=0}^{m} b_j x^j$ mit $b_m \neq 0$. Nach Voraussetzung ist $b_m \in R^{\times}$, es existiert also $b_m^{-1} \in R$. Induktion nach $\deg(f) = n$: \\
		\emph{$n < m$:} $q = 0$, $r = f$ \\
		\emph{$n\geq m$:} $f_i = f - a_n b_m^{-1} x^{n-m} \cdot g \Rightarrow \deg(f_1) < \deg(f)$ mit Induktionshypothese folgt $f_1 = q_1 \cdot g + r_1$ mit $\deg(r) < m$ $\Rightarrow f = (q_1 + a_n b_m^{-1} x^{n-m})g + r$
	\end{itemize}
\end{proof}

\begin{conclusion}
	Ist $f \in R[x]$ und $a \in R$, $f(a) = 0$, so ist
	\begin{align}
		f(x) = (x-a)\cdot q(x) \text{ mit } q \in R[x]. \notag
	\end{align}
\end{conclusion}

\begin{proof}
	Sei $f = q(x-a) + r$, $\deg(r) < \deg(x-a)$, das heißt $\deg(r) \leq 0 \Rightarrow 0 = f(a) = q(a-a) + r(a) \Rightarrow r(a) = 0$.
\end{proof}

\begin{conclusion}
	\proplbl{2_1_16}
	Ist $R$ nullteilerfrei, so hat $0 = f \in R[x]$ höchstens $\deg(f)$ viele Nullstellen in $R$.
\end{conclusion}

\begin{definition}[Polynomring in kommutieren Variablen]
	Für eine Menge $I$ definieren wir das Monoid
	\begin{align}
	\natur_{0}^{(I)} : = \left\{ (\mu_i)_{i \in I} \in \prod_{i \in I} \natur_{0} \Bigg| \mu_i = 0 \text{ für fast alle } i \right\}\notag
	\end{align}
	mit Addition
	\begin{align}
		(\mu_i)_{i \in I} + (\nu_i)_{i \in I} :=(\mu_i+\nu_i)_{i \in I} \notag
	\end{align}
	sowie den Ring
	\begin{align}
		R[x_i \mid i \in I] = \left\{ (a_{\mu})_{\mu \in \natur_{0}^{(I)}} \mid a_{\mu} \in R, \text{ fast alle gleich } 0\right\} \notag
	\end{align}
	mit Addition und Multiplikation
	\begin{align}
		(a_{\mu})_{\mu \in \natur_{0}^{(I)}} + (b_{\mu})_{\mu \in \natur_{0}^{(I)}} &:= (a_{\mu} + b_{\mu})_{\mu \in \natur_{0}^{(I)}}\notag \\
		(a_{\lambda})_{\lambda \in \natur_{0}^{(I)}}\cdot (b_{\nu})_{\nu \in \natur_{0}^{(I)}} &:= \left( \sum_{\lambda + \nu = \mu} a_{\lambda}b_{\mu}\right)_{\mu \in \natur_{0}^{(I)}} \notag
	\end{align}
	genannt \begriff{Polynomring in den kommutierenden Variablen} $x_i$ mit $i \in I$. Wir identifizieren dem Ring $R$ mit den Unterring
	\begin{align}
		\left\{ (r\delta_{\mu,\underline{0}})_{\mu \in \natur_{0}^{(I)}}\mid r \in R \right\} \notag
	\end{align}
	Wir schreiben $x_i := (\delta_{\mu\nu})_{\mu \in \natur_{0}^{(I)}}$, $\nu := (\delta_{ij})_{j \in I}$ und $x^{\mu} := \prod_{i \in I}x_i^{\mu_i}$. Damit ist dann
	\begin{align}
		(a_{\mu})_{\mu \in \natur_{0}^{(I)}} = \sum_{\mu \in \natur_{0}^{(I)}} a_{\mu} x^{\mu}. \notag
	\end{align}
	Weiter schreiben wir $R[x_1, \dots, x_n] := R[x_i \mid i \in \{ i, \dots, n \}]$.
\end{definition}

\begin{example}
	Sei $R = \whole$ und $I = \{1,2\}$, dann
	\begin{align}
		\left( x_1 x_2 + x_2^2 \right)^2 = a_{(2,2)}x_1^2 x_2^2 + a_{(1,3)}x_1 x_2^3 + a_{(0,4)}x_2^4\notag	
	\end{align}
	mit $a_{(2,2)} = 1$, $a_{(1,3)} = 2$ und $a_{(0,4)} = 1$
\end{example}

\begin{remark}
	\propref{2_1_10} und \propref{2_1_12} kann man allgemein für $R[x_i \mid i \in I]$ anstatt $R[x]$ formulieren. Für \propref{2_1_14} bis \propref{2_1_16} gibt es keine Verallgemeinerung. So hat zum Beispiel $f = x_1 - x_2$ unendlich viele Nullstellen, da $f(a,a) = 0$ für alle $a \in \whole$.
\end{remark}