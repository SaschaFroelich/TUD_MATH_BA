\section{p-Gruppen}

Sei $G$ eine endliche Gruppe und $p$ eine Primzahl.

\begin{definition}[$p$-Gruppe]
	$G$ ist eine \begriff{$p$-Gruppe} $\Leftrightarrow\#G=p^n$ für ein $n\in\natur_0$.
\end{definition}

\begin{proposition}
	\proplbl{1_7_2}
	Sei $G$ eine $p$-Gruppe und $X$ eine endliche $G$-Menge. Dann ist
	\begin{align}
		\#\Fix_X(G) \equiv \#X\mod p\notag
	\end{align}
\end{proposition}
\begin{proof}
	Sei $x\in X$.
	\begin{itemize}
		\item $x\in\Fix_X(G)\Rightarrow (G:G_x)=1$
		\item $x\notin\Fix_X(G)\Rightarrow 1\neq (G:G_x)\mid \#G=p^n\Rightarrow (G:G_x)\equiv 0\mod p$
		\item Ist $X=\biguplus_{i=1}^n x_i^G$, so ist
		\begin{align}
			\#X = \sum_{i=1}^n (G:G_i) \equiv \#\Fix_X(G)\mod p\notag
		\end{align}
	\end{itemize}
\end{proof}

\begin{conclusion}[Satz von \person{Cauchy}]
	\proplbl{1_7_3}
	Teilt $p$ die Ordnung von $G$, so hat $G$ ein Element der Ordnung $p$.
\end{conclusion}
\begin{proof}
	Sei $X=\{g_1,...,g_p\in G^p\mid g_1\cdot ...\cdot g_p=1 \}$. Es ist $\#X=(\#P)^{p-1}$ und $C_p=\langle (1\, 2\, ...\, p)\rangle\le S_p$ wird auf $X$ durch $(g_1,...,g_p)^{\sigma}=(g_{\sigma(1)},...,g_{\sigma(p)})$ beschrieben. Mit \propref{1_7_2} gilt:
	\begin{align}
		\#\Fix_X(C_p) \equiv \#X \equiv (\#G)^{p-1} \equiv 0\mod p\notag
	\end{align}
	Da $(1,...,1)\in \Fix_X(C_p)$ folgt $\#\Fix_X(C_p)\ge p\ge 2$, es existiert also $1\neq g\in G$ mit $(g,...,g)\in X$, das heißt $\ord(g)=p$.
\end{proof}

\begin{conclusion}
	\proplbl{1_7_4}
	Jede nicht-triviale $p$-Gruppe hat eine nicht-triviales Zentrum.
\end{conclusion}
\begin{proof}
	Betrachte Wirkung von $G$ auf $X=G$ durch Konjugation (\propref{1_6_2} c). Dann
	\begin{align}
		\#\Z(G) \equiv \Fix_X(G)\overset{\propref{1_7_2}}{\equiv} \#G\equiv 0\mod p\notag
	\end{align}
	insbesondere ist $\Z(G)\neq 1$.
\end{proof}

\begin{lemma}
	\proplbl{1_7_5}
	$\#G=p\Rightarrow G$ ist zyklisch.
\end{lemma}
\begin{proof}
	Sei $1\neq g\in G\Rightarrow 1\neq\ord(g)\mid \# G\Rightarrow \ord(g)=p\Rightarrow G=\langle g\rangle$ ist zyklisch.
\end{proof}

\begin{lemma}
	\proplbl{1_7_6}
	$G/ Z(G)$ zyklisch $\Rightarrow G$ ist stabil. 
\end{lemma}

\begin{proof}
	Sei $a \in G$ mit $G/Z(G) = \langle aZ(G)\rangle$. \\
	Dann ist 
	\begin{align}
		G = \bigcup_{k \in \whole} a^k Z(G).\notag
	\end{align}
	Sind nun $x,y \in G$, so ist\\
	$x=a^k c$, $y=a^l d$ mit $k,l\in\whole,c,d \in Z(G)$\\
	$\Rightarrow xy=a^k c a^l d = a^l d a^k c = yx$
\end{proof}

\begin{proposition}
	Ist $\#G = p^2$, so ist $G$ abelsch.
\end{proposition}

\begin{proof}
	Nach \propref{1_7_4} ist $Z(G) \neq 1.$\\
	$\Rightarrow \#G/Z(G) \mid p$\\
	$\overset{\propref{1_7_5}}{\Rightarrow} G/Z(G)$ ist zyklisch \\
	$\overset{\propref{1_7_6}}{\Rightarrow} G$ ist abelsch.  
\end{proof}

\begin{remark}
	Mit dem Struktursatz \propref{1_4_8} erhalten wir
	\begin{align}
		\#G = p \Rightarrow &G \cong \whole/p\whole \\
		\#G = p^2 \Rightarrow &G \cong \whole/p^2\whole \text{ oder } \\
		                      &G \cong \whole / p\whole \bigoplus \whole /p\whole
	\end{align}
\end{remark}

\begin{proposition}
	Ist $\#G = p^k$ und $l \leq k$, so gibt es $H \leq G$ mit $\#H = p^l.$
\end{proposition}

\begin{proof}
	Induktion nach $l$:
	\begin{itemize}
		\item $l = 0$: trivial Untergruppe!
		\item $l-1 \to l:$ Nach $\propref{1_7_4}$ ist $\#Z(G) = p^a, a > 0$, nach \propref{1_7_3} (Cauchy) existiert somit ein $g \in Z(G)$ mit $\ord(g) = p$. Da $g \in Z(G)$ ist $\langle g \rangle \unlhd G$ und $\# G / \langle g \rangle = p^{k-1}$. Nach Induktionshypothese ist Untergruppe $H_0 \leq G/\langle g \rangle$ mit $\#H_0 = p^{l-1}$. Betrachte den $\hom \pi_{\langle g \rangle} : G \to G / \langle g \rangle \Rightarrow H:= \pi_{\langle g \rangle}^{-1}(H_0) \leq G$, $\#H = \# \ker (\pi_{\langle g \rangle}) \cdot \#H_0 = p\cdot p^{l-1} = p^l$.
	\end{itemize}
\end{proof}