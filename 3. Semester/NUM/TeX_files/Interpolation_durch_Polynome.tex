\section{Interpolation durch Polynome}

$\Pi_n$ bezeichne den Vektorraum der Polynome von Höchstgrad $n$ mit der üblichen Addition und Skalarmultiplikation. Für jedes $p\in\Pi_n$ gibt es $a_0,\dots,a_n\in\real$, sodass
\begin{align}
	p(x) = a_nx^n+a_{n-1}x^{n-1}+\dots+a_1x+a_0
\end{align}
und umgekehrt.

\subsection{Existenz und Eindeutigkeit}

\begin{proposition}
	Zu $n+1$ Datenpaaren $(x_0,f_0),\dots,(x_n,f_n)$ mit paarweise verschiedenen Stützstellen existiert genau ein Polynom $p\in\Pi_n$, dass die Interpolationsbedingung  \cref{interpolationsbedingung} erfüllt.
\end{proposition}
\begin{proof}
	\begin{itemize}
		\item Existenz: Sei $j\in\{0,\dots,n\}$ und $L_j:\real\to\real$ mit
		\begin{align}
			L_j(x) := \prod_{\substack{i=0\\ i\neq j}}^{n} \frac{x-x_i}{x_j-x_i}= \frac{(x-x_0)\cdot\dots\cdot(x-x_{j-1})(x-x_{j+1})\cdot\dots\cdot(x-x_n)} {(x_j-x_0)\cdot\dots\cdot(x_j-x_{j-1})(x_j-x_{j+1})\cdot\dots\cdot(x_j-x_n)}\notag
		\end{align}
		das \begriff{\person{Lagrange}-Basispolynom} vom Grad $n$. Offenbar gilt $L_j\in\Pi_n$ und 
		\begin{align}
			\label{1.3}
			L_j(x_k)=\begin{cases}
				1 & k=j \\ 0 & k\neq j
			\end{cases} = \delta_{jk}
		\end{align}
		Definiert man $p:\real\to\real$ durch
		\begin{align}
			\label{1.4}
			p(x) := \sum_{j=0}^{n} f_j\cdot L_j(x)
		\end{align}
		so ist $p\in\Pi_n$ und außerdem erfüllt $p$ wegen \cref{1.3} die Interpolationsbedingung \cref{interpolationsbedingung}
		\item Eindeutigkeit: Angenommen es gibt Interpolierende $p,\tilde{p}\in\Pi_n$ mit $p\neq\tilde{p}$. Dann folgt $p-\tilde{p}\in\Pi_n$ und $(p-\tilde{p})(x_k)=p(x_k)-\tilde{p}(x_k)=0$ für $k=0,\dots,n$. Also hat $(p-\tilde{p})$ mindestens $n+1$ Nullstellen, hat aber Grad $n$. Das heißt, dass $(p-\tilde{p})$ das Nullpolynom sein muss.
	\end{itemize}
\end{proof}

\begin{*definition}[Interpolationspolynom]
	Das Polynom, dass die Interpolationsbedingung erfüllt, heißt \begriff{Interpolationspolynom} zu $(x_0,f_0),\dots,(x_n,f_n)$.
\end{*definition}

\begin{remark}
	\begin{itemize}
		\item Die Darstellung \cref{1.4} heißt \begriff{\person{Lagrange}-Form} des Interpolationspolynoms.
		\item Um mittels \cref{1.4} einen Funktionswert $p(x)$ zu berechnen, werden $\mathcal{O}(n^2)$ Operationen genötigt; bei gleichabständigen Stützstellen kann man diesen Aufwand auf $\mathcal{O}(n)$ verringern. Ändern sich die Stützwerte, kann man durch Wiederverwendung von den $L_j(x)$ das $p(x)$ in $\mathcal{O}(n)$ Operationen berechnen.
		\item Man kann zeigen, dass $L_0$ bis $L_n$ eine Basis von $\Pi_n$ bilden.
	\end{itemize}
\end{remark}
