\section{Grundlagen}

\textbf{Aufgabe:} \\
Gegeben sind $n+1$ Datenpaare $(x_0,f_0),\dots, (x_n,f_n)$, alles reelle Zahlen und paarweise verschieden. \\
Gesucht ist eine Funktion $F:\real\to\real$, die die \begriff{Interpolationsbedingungen}
\begin{align}
	\label{interpolationsbedingung}
	F(x_0) = f_0, \, \dots, \, F(x_n)=f_n
\end{align}
genügt.

\begin{*definition}[Stützstellen, Stützwerte]
	Die $x_0$ bis $x_n$ werden \begriff{Stützstellen} genannt.
	
	Die $f_0$ bis $f_n$ werden \begriff{Stützwerte} genannt.
\end{*definition}

Die oben gestellte Aufgabe wird zum Beispiel durch 
\begin{align}
	F(x) = \begin{cases}
		0 & x\notin \{x_0,\dots,x_n\} \\
		f_i & x=x_i
	\end{cases}\notag
\end{align}
gelöst. Weitere Möglichkeiten sind: Polygonzug, Treppenfunktion, Polynom, \dots
\begin{itemize}
	\item In welcher Menge von Funktionen soll $F$ liegen?
	\item Gibt es im gewählten \begriff{Funktionenraum} für beliebige Datenpaare eine Funktion $F$, die den Interpolationsbedingungen genügt (eine solche Funktion heißt \begriff{Interpolierende})?
	\item Ist die Interpolierende in diesem Raum eindeutig bestimmt?
	\item Welche weiteren Eigenschaften besitzt die Interpolierende, zum Beispiel hinsichtlich ihrer Krümmung oder der Approximation einer Funktion $f:\real\to\real$ mit $f_k=f(x_k)$ für $k=0, \dots, n$
	\item Wie sollte man die Stützstellen wählen, falls nicht vorgegeben?
	\item Wie lässt sich die Interpolierende effizient bestimmen, gegebenenfalls auch unter der Berücksichtigung, dass neue Datenpaare hinzukommen oder dass sich nur die Stützwerte ändern? 
\end{itemize}

\begin{example}
	\hspace*{1.5em}
	\begin{center}
		\begin{tabular}{c|cccccc}
			$k$ & 0 & 1 & 2 & 3 & 4 & 5 \\
			\hline
			$x_k$ in s & 0 & 1 & 2 & 3 & 4 & 5 \\
			\hline
			$f_k$ in °C & 80 & 85,8 & 86,4 & 93,6 & 98,3 & 99,1
		\end{tabular}
	\end{center}
Interpolation im
\begin{itemize}
	\item Raum der stetigen stückweise affinen Funktionen
	\item Raum der Polynome höchstens 5. Grades
	\item Raum der Polynome höchstens 4. Grades (Interpolation im Allgemeinen nicht lösbar, Regression nötig)
\end{itemize}
\end{example}
