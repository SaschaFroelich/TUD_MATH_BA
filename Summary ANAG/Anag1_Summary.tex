\documentclass[ngerman,a4paper]{report}
\usepackage[table]{xcolor}

\usepackage[T1]{fontenc}
\usepackage{amsmath}
\usepackage{amssymb}
\usepackage{amsfonts}
\usepackage{latexsym}
\usepackage{ marvosym } %lighning
\usepackage{graphicx}
%\usepackage{fontspec} %ß, Umlaute etc.
\usepackage[ngerman]{babel}
\usepackage[utf8]{inputenc}
%\usepackage{url} 
\usepackage[top=1cm,bottom=1.5cm,left=1cm,right=1cm]{geometry}
\usepackage{bbm} %unitary matrix 1

\usepackage[texindy]{imakeidx}
\makeindex
\makeindex[name=symbols,title=Symbolverzeichnis]

\usepackage{enumerate}
\usepackage{enumitem} %customize label
\usepackage{tabularx}
\usepackage{multirow}
\usepackage{booktabs}
\usepackage{cleveref}
\usepackage{xfrac}%sfrac -> fractions e.g. 3/4
\usepackage{parskip}%split paragraphs by vspace instead of intendations

\usepackage{cancel}

\usepackage{chngcntr}

\usepackage{ulem} %better underlines

\usepackage{titlesec}%customize titles

\usepackage{xparse}%better macros

\usepackage[amsthm,thmmarks,hyperref]{ntheorem}%customize theorem-environments more effectively

\usepackage[xindy,acronym]{glossaries}
\makeglossaries

\usepackage[bookmarks=true]{hyperref}
\hypersetup{
	colorlinks,
	citecolor=green,
	filecolor=green,
	linkcolor=blue,
	urlcolor=green
}
\usepackage{bookmark}

\renewcommand{\mvchr}[1]{\mbox{\mvs\symbol{#1}}} %change the use of lightning symbol globally

\renewcommand{\thesection}{\arabic{section}}
\renewcommand{\thechapter}{\Roman{chapter}}
\titleformat{\chapter}[hang]{\huge\bfseries}{\thechapter}{15pt}{\huge\bfseries}
\titlespacing{\chapter}{0pt}{0pt}{0pt}
\titlespacing{\section}{0pt}{0pt}{0pt}

\theoremstyle{break}
\theorembodyfont{}
\theorempostskip{7pt}
\theorempreskip{3pt}

\newtheorem{theorem}{Theorem}[section]
\newtheorem*{*example}{Beispiel}
\newtheorem{example}[theorem]{Beispiel}
\newtheorem{corollar}[theorem]{Korollar}
\newtheorem{lemma}[theorem]{Lemma}
\newtheorem{satz}[theorem]{Satz}
\newtheorem{overview}[theorem]{Überblick}
\newtheorem*{definition}{Definition}
\newtheorem{remark}[theorem]{Bemerkung}
\newtheorem*{*remark}{Bemerkung} % removed counter
\newtheorem{conclusion}[theorem]{Folgerung}

\NewDocumentCommand{\begriff}{s O{} m O{}}{
	\IfBooleanTF{#1}
		{\index{#2#3#4}}
		{\uline{#3}\index{#2#3#4}}
}
\NewDocumentCommand{\mathsymbol}{s O{} m m O{}}{
	\IfBooleanTF{#1}
		{\index[symbols]{#2#3@\detokenize{#4}#5}}
		{#4\index[symbols]{#2#3@\detokenize{#4}#5}}
}

\newcommand{\person}[1]{\textsc{#1}}
\newcommand{\highlight}[1]{\emph{#1}}

\newcommand{\realz}{\mathfrak{Re}}
\newcommand{\imagz}{\mathfrak{Im}}

\renewcommand{\epsilon}{\varepsilon}


%\numberwithin{theorem}{section}
%\counterwithout{section}{chapter}
%\counterwithout{theorem}{section}


\DeclareMathOperator{\Abb}{Abb}
\DeclareMathOperator{\id}{id}
\DeclareMathOperator{\im}{Im}
\DeclareMathOperator{\Sym}{Sym}
\DeclareMathOperator{\spank}{span_\mathbb{K}}
\DeclareMathOperator{\Mat}{Mat}
\DeclareMathOperator{\diag}{diag}
\DeclareMathOperator{\End}{End}
\DeclareMathOperator{\aut}{Aut}
\DeclareMathOperator{\graph}{graph}
\DeclareMathOperator{\Int}{int}
\DeclareMathOperator{\Ext}{ext}
\DeclareMathOperator{\cl}{cl}
\DeclareMathOperator{\diam}{diam}
\DeclareMathOperator{\grad}{grad}

\newacronym{gdw}{gdw.}{genau dann wenn}
\newacronym{fa}{fa.}{fast alle}
\newacronym{obda}{oBdA}{ohne Beschränkung der Allgemeinheit}
\newacronym{tf}{TF}{\begriff{Teilfolge}}
\newacronym{hw}{Hw}{\begriff{Häufungswert}}
\newacronym{cf}{CF}{\begriff{\person{Cauchy}-Folge}}
\newacronym{hp}{HP}{\begriff{Häufungspunkt}}

\pagestyle{plain}

\begin{document}
	
%\tableofcontents
	
\section{Grundbegriffe aus Logik und Mengenlehre}

\begin{definition}[Aussage]
	\begriff{Aussage} ist ein Schverhalt, dem man entweder den Warheitswert wahr ($w$) oder falsch ($f$) zuordnen kann (und nichts anderes).
\end{definition}

\addtocounter{theorem}{1}
	
\begin{definition}[Menge]
	\begriff{Menge} ist (nach Cantor 1877) eine Zusammenfassung von bestimmten, wohlunterschiedenen Objekten der Anschauung oder des Denkens, welche die \begriff{Elemente} der Menge genannt werden, zu einem Ganzen.
\end{definition}

\addtocounter{theorem}{1}

\begin{definition}
	\begin{itemize}
		\item $M=N$, falls dieselben Elemente enthalten sind
		\item $N$\mathsymbol{c}{$\subset$}$M$ (\begriff{Teilmenge}), falls $n\in M$für jedes $n\in\mathbb{N}$
		\item $N$\mathsymbol{c=}{$\subsetneqq$}$M$ (\begriff{echte Teilmenge}), falls zusätzlich $N\neq M$.
		\item \begriff{Aussageform}: Sachverhalt mit Variablen, der durch geeignete Ersetzung der Variablen zur Aussage führt
	\end{itemize}
\end{definition}

\addtocounter{theorem}{1}

\begin{definition}[Quantoren]
	\begriff{Quantoren}
	\begin{itemize}
		\item $\forall x\in M: A(x)$ wahr \gls{gdw} $A(x)$ wahr für jedes $x\in M$
		\item $\exists x\in M: A(x)$ wahr \gls{gdw} $A(x)$ wahr für mindestens ein $x\in M$
	\end{itemize}
\end{definition}

\begin{definition}
	\begriff{Tautologie} bzw. \begriff{Kontradiktion}\slash\begriff{Widerspruch} ($\Lightning$) ist zusätzlich gesetzte Aussage, die unabhängig vom Wahrheitswert der Teilaussagen stets wahr bzw. falsch ist.
\end{definition}

\begin{satz}[\person{de Morgan}'sche Regeln]
	Folgende Aussagen sind stets Tautologien
	\begin{enumerate}[label={\alph*)}]
		\item $\neg(A\land B) \Leftrightarrow \neg A \lor \neg B$
		\item $\neg(A\lor B) \Leftrightarrow \neg A\land \neg B$
		\item $\neg (\forall x\in M: A(x))\Leftrightarrow \exists x\in M:\neg A(x)$
		\item $\neg (\exists x\in M: A(x)) \Leftrightarrow \forall x\in M:\neg A(x)$
	\end{enumerate}
\end{satz}

\begin{definition}
	\begin{itemize}
		\item \begriff{leere Menge} \mathsymbol{o}{$\emptyset$}$=:$ Menge, die kein Element enthält
		\item $M,N$ sind \begriff{disjunkt}, falls $M\cap N = \emptyset$
		\item Sei $\mathcal{M}$ \begriff{Mengensystem}, d.h. Mengen von Mengen, dann
		\begin{itemize}
			\item $\bigcup_{M\in\mathcal{M}} M := \{x | \exists M\in\mathcal{M}: x\in M\}$
			\item $\bigcap_{M\in\mathcal{M}} M:= \{ x|\forall M\in\mathcal{M}: x\in M \}$
		\end{itemize}
		\item \begriff{Potenzmenge}: \mathsymbol{p}{$\mathcal{P}$}$(XM):=\{\tilde{M} | \tilde{M}\in M\}$
		\item \begriff{\person{de Morgan}'sche Regeln} (für $\mathcal{N}\subset\mathcal{P}(M)$)
		\begin{itemize}
			\item $\left(\bigcup_{N\in\mathcal{N}} N\right)^C = \bigcap_{N\in\mathcal{N}} N^C$
			\item $\left(\bigcap_{N\in\mathcal{N}} N\right)^C = \bigcup_{N\in\mathcal{N}} N^C$
		\end{itemize}
		\item \begriff{kartesisches Produkt} $M$\mathsymbol{x}{$\times$}$N:=\{(m,n) | m\in M \text{ und } n\in N\}$
		\item $(m_1, \dotsc, m_n)$ ist \begriff{n-Tupel}
		\item \begriff{Auswahlaxiom} (AC / axiom of choice)
		
		Sei $\mathcal{M}$ Menge nichtleerer, paarweise disjunkter Mengen $M$\\
		$\Rightarrow$ es gibt immer (Auswahl-) Menge $\tilde{M}$, die mit jedem $M\in\mathcal{M}$ genau ein Element gemein hat.
	\end{itemize}
\end{definition}

\begin{example}
	\begin{itemize}
		\item Für Aussagen $A,B,C$: $A\land B \Rightarrow B$
		\begin{itemize}
			\item $B$ ist \begriff{notwendig} für $A$
			\item $A$ ist \begriff{hinreichend} für $B$
		\end{itemize}
	\end{itemize}
\end{example}

\subsection*{Mathematische Beweise}
\begin{definition}
	\begin{enumerate}
		\item \begriff[Beweis!]{direkt}\highlight{er Beweis}: $(A\Rightarrow A_1)\land(A_1\Rightarrow A_2)\land\dotsc\land(A_n\Rightarrow B)$ wahr für $A\Rightarrow B$
		\item \begriff[Beweis!]{indirekt}\highlight{er Beweis} durch Tautologie $(A\Rightarrow B)\Leftrightarrow (\neg B\rightarrow \not A)$
	\end{enumerate}
\end{definition}

\subsection*{Relation und Funktion}
\begin{definition}[Relation]
	\begin{itemize}
		\item \begriff{Relation} ist Teilmenge $R\subset M\times N$. $(x,y)\in R$ heißt: $x$ und $y$ stehen in Relation zueinander.
		\item Relation $R\subset M\times N$ heißt \begriff{Ordnungsrelation} (kurz \begriff{Ordnung}) auf $M$, falls $\forall a,b,c\in M$:
		\begin{enumerate}[label={\alph*)}]
			\item $(a,a)\in R$ (\begriff[Ordnung!]{reflexiv})
			\item $(a,b),(b,a)\in R \rightarrow a=b$ (\begriff[Ordnung!]{antisymmetrisch})
			\item $(a,b),(b,c)\in R \rightarrow (a,c)\in R$ (\begriff[Ordnung!]{transitiv})
		\end{enumerate}
		\item Ordnungsrelation $R$ auf $M$ heißt \begriff{Totalordnung}, falls $\forall a,b\in M: (a,b)\in R \lor (b,a)\in R$
		\item Relation auf $M$ heißt \begriff{Äquivalenzrelation}, falls $\forall a,b,c\in M$:
		\begin{enumerate}[label={\alph*)}]
			\item $(a,a)\in R$ (\begriff[Ordnung!]{reflexiv})
			\item $(a,b)\in R \Rightarrow (b,a)\in R$ (\begriff[Ordnung!]{symmetrisch})
			\item $(a,b),(b,c)\in R \Rightarrow (a,c)\in R$ (\begriff[Ordnung!]{transitiv})
		\end{enumerate}
		\item \mathsymbol{[a]}{$[a]$}$:=\{b\in M| (a,b)\in R\}$ heißt \begriff{Äquivalenzklasse} von $a\in M$ bzgl. $R$
		
		Jedes $b\in [a]$ ist ein \begriff{Repräsentant} von $[a]$
	\end{itemize}
\end{definition}

\begin{definition}[Abbildung]
	\begriff{Abbildung}/\begriff{Funktion} von $M$ nach $N$, kurz: $F:M\rightarrow N$ ist Vorschrift, die jedem \begriff{Argument} / \begriff{Urbild} $m\in M$ genau einen \begriff{Wert} / \begriff{Bild} $F(m)\in N$ zuordnet.
	
	\begin{itemize}
		\item \mathsymbol{D}{$\mathcal{D}$}$(F):=M$ heißt \begriff{Definitionsbereich} / \begriff{Urbildmenge}
		\item $N$ heißt \begriff{Zielbereich}
		\item $(M'):=\{n\in M | n=F(m)$ für ein $m\in M'\}$ ist \begriff{Bild}\highlight{ von $m'$}$\in M$
		\item $F^{-1}(N'):=\{ m\in M| n=F(m)$ für ein $N' \}$ ist \begriff{Urbild}\highlight{ von $N'$}$\subset N$
		\item \mathsymbol{R}{$\mathcal{R}$}$(F):= F(M)$ heißt \begriff{Wertebereich} / \begriff{Bildmenge}
		\item \mathsymbol{graph}{$\graph$}$(F) :=\{ (mn,)\in M\times N | n = F(m)\}$ heißt \begriff{Graph}\highlight{von $F$}
		\item \mathsymbol{fm}{$F|_{M'}$} ist \begriff{Einschränkung}\highlight{der Funktion} von $F$ auf $M'\subset M$
		\item \begriff{Komposition} von $F:M\rightarrow N$ und $G:N\rightarrow P$ ist Abbildung $G$\mathsymbol{o}{$\circ$}$F:M\rightarrow P$ mit $(G\circ F)(m):=G(F(m))$
		\item $Abbildung F:M\rightarrow N$ heißt
		\begin{itemize}
			\item \begriff{injektiv}, falls eineindeutig (d.h. $F(m_1) = F(m_2) \Rightarrow m_1 = m_2$)
			\item \begriff{surjektiv}, falls $F(M) = N$, d.h. $\forall n\in N\,\exists m\in M: F(m) = n$
			\item \begriff{bijektiv}, falls injektiv und surjektiv
		\end{itemize}
		\item Für bijektive Abb. $F:M\rightarrow N$ ist \begriff{Umkehrabbildung} / \begriff{inverse Abbildung} \mathsymbol{f-1}{$F^{-1}$}$:N\rightarrow M$ definiert durch $F^{-1}(n) = m \Leftrightarrow F(m) = n$
	\end{itemize}
\end{definition}

\stepcounter{theorem}
\begin{satz}
	Sei $F:M\rightarrow N$ surjektiv. Dann existiert Abbildung $G:N\rightarrow M$, sodass $F\circ G = \id_N$ (d.h. $F(G(n)) = n\;\forall n\in N$)
\end{satz}

\begin{definition}[Verknüpfung]
	Eine \begriff{Rechenoperation} / \begriff{Verknüpfung} auf $M$ ist Abb. $*:M\times M\rightarrow M$, d.h. $m,n\in M$ wird \begriff{Ergebnis} $m*n\in M$
	
	Rechenoperation
	\begin{itemize}
		\item hat \begriff{neutrales Element} $e\in M$, falls $m*e = e*m = m\;\forall m\in M$
		\item ist \begriff{kommutativ}, falls $m*n = n*m$
		\item ist \begriff{assoziativ}, falls $k*(m*n) = (k*m)*n\;\forall k,m,n\in M$
		\item hat \begriff{inverses Element} $m'\in M$ zu $m\in M$, falls $m*m' = m'*m = e$
	\end{itemize}
\end{definition}

\begin{*example}
	\begin{enumerate}[label={\alph*)}]
		\item \begriff{Addition}: $(m,n)\mapsto: m+n$ \begriff{Summe},
		\begin{itemize}
			\item neutrales Element heißt \begriff{Null} / \begriff{Nullelement}
			\item Inverses Element: \mathsymbol{-}{$-m$}
		\end{itemize}
		\item \begriff{Multiplikation} $\cdot:(m,n)\mapsto: m\cdot n$ \begriff{Produkt}
		\begin{itemize}
			\item neutrales Element heißt \begriff{Eins} / \begriff{Einselement}
			\item Inverses Element:\mathsymbol{-1}{$m^{-1}$}
		\end{itemize}
	\end{enumerate}
\end{*example}

\begin{definition}
	Addition und Multiplikation heißen \begriff{distributiv}, falls $k\cdot(m+n) = k\cdot m + k\cdot n\;\forall k,m,n\in M$
\end{definition}

\begin{definition}[Körper]
	Menge $K$ heißt \begriff{Körper}, falls auf $K$ eine Addition und Multiplikation existiert mit
	\begin{enumerate}[label={\alph*}]
		\item es existieren neutrale Elemente $0\in K$ und $1\in K_{\neg 0}$
		\item Addition und Multiplikation sind distributiv
		\item Es gibt Inverse
	\end{enumerate}
\end{definition}

\begin{definition}
	Menge $M$ habe Ordnung "$\le$", sowie Addition und Multiplikation.
	
	Ordnung ist \begriff{verträglich}\highlight{mit Addition und Multiplikation}, wenn $\forall a,b,c\in M$
	\begin{enumerate}[label={(\alph*)}]
		\item $a\le b \Leftrightarrow a+c \le b+c$
		\item $a\le b \Leftrightarrow a\cdot c \le b\cdot c$ mit $c > 0$
	\end{enumerate}
\end{definition}

\begin{definition}
	Körper $K$ heißt \begriff{angeordnet}, falls mit Addition und Multiplikation verträgliche Totalordnung existert.
\end{definition}

\begin{definition}[Isomorphismus]
	\begriff{Isomorphismus} bezüglich einer Struktur ist bijektive Abbildung $I:M_1\rightarrow M_2$, die auf $M_1$ und $M_2$ vorhandene Struktur erhält.
	
	Mengen $M_1$ und $M_2$ heißen \begriff{isomorph}.
\end{definition}

\chapter{Zahlenbereiche}\addtocounter{section}{2}
\section{Natürliche Zahlen}
\begin{definition}
$\mathbb{N}$ sei Menge, die die \begriff{\person{Peano}-Axiome} erfüllen, d.h.
\begin{enumerate}[label={P\arabic*)}]
	\item $\mathbb{N}$ sei indutkiv, d.h. es ex.
	\begin{itemize}
		\item Nullelement $0\in \mathbb{N}$ und
		\item injektive (Nachfolger-) Abb. \mathsymbol{nu}{$\nu$}$:\mathbb{N}\rightarrow\mathbb{N}$ mit $\nu(n)\neq 0\;\forall n\in \mathbb{N}$
	\end{itemize}
	\item (Induktionsaxiom)
	
	Falls $N\subset\mathbb{N}$ inuktiv in $\mathbb{N}$ (d.h. $0,\nu(n)\in\mathbb{N}$ falls $n\in\mathbb{N}$)\\
	$\Rightarrow N=\mathbb{N}$ ($N$ ist die kleinste indutkive Menge)
\end{enumerate}

Nach Mengenlehre ZF existiert eine Solche Menge der \begriff{natürliche Zahlen} mit üblichen Symbolen.
\end{definition}

\begin{theorem}
	Falls $\mathbb{N}$ und $\mathbb{N}*$ \person{Peano}-Axiome erfüllen, dann sind sie isomorph bezüglich Nachfolger-Abbildung und Nullelement (Anfangselement).
\end{theorem}

\begin{satz}[Prinzip der vollständigen Induktion]
	\begriff*{vollständigen Induktion}
	Sei $\{A_n | n\in\mathbb{N}\}$ Aussagenmenge mit d. Eigenschaften
	\begin{itemize}
		\item[(IA)] $A_0$ ist wahr (\begriff{Induktionsanfang})
		\item[(IS)] $\forall n\in\mathbb{N}$ gilt: $A_n$ (wahr) $\Rightarrow A_{n+1}$
	\end{itemize}
	$\Rightarrow A_n$ ist wahr $\forall n\in\mathbb{N}$
\end{satz}

\begin{lemma}
	Es gilt:
	\begin{enumerate}[label={\alph*)}]
		\item $\nu(\mathbb{N})\cup \{0\}=\mathbb{N}$
		\item $\nu(n)\neq n\;\forall n\in\mathbb{N}$
	\end{enumerate}
\end{lemma}

\begin{satz}[Rekusrive Definition / Rekursion]
	\begriff*{Rekursion}
	Sei b$B$ Menge, $b\in B$ u. $F:B\times\mathbb{N}\rightarrow B$ Abbildung. Dann liefert die Vorschrift \begin{align*}
		f(0) &:= b,\\f(n+1):=F(f(n),n)\quad\forall n\in \mathbb{N}
	\end{align*}
	genau eine Abbildung für $f:\mathbb{N}\rightarrow B$ (d.h. solche Abbildung ist eindeutig)
\end{satz}

\subsection*{Rechenoperationen}
\begin{definition}
	Definiere \begriff{Addition}[!natürliche Zahlen] $+:\mathbb{N}\times\mathbb{N}\rightarrow \mathbb{N}$ auf $\mathbb{N}$ durch $n+0:=n, n+\nu(m) :=\nu(n+m)\,\forall n,m\in\mathbb{N}$
	
	Definiere \begriff{Multiplikation}[!natürliche Zahlen] $\cdot:\mathbb{N}\times\mathbb{N} \rightarrow\mathbb{N}$ auf $\mathbb{N}$ durch $n\cdot 0 = 0, n\cdot\nu(m) = n\cdot m+n\;\forall m,n\in\mathbb{N}$
\end{definition}

\begin{satz}
	Addition und Multiplikation haben folgende Eigenschaften, d.h. $\forall k,m,n\in\mathbb{N}$ gilt:
	
	\begin{tabular}{clll}
		\toprule
		&& Addition & Multiplikation\\
		\midrule
		a)& $\exists$ neutrales Element & $n+0=n$ & $n\cdot 1 =  n$\\
		b)& kommutativ & $m+n=n+m$ & $m\cdot n = n\cdot m$ \\
		c)& assoziativ & $(k+m)+n = k+(m+n)$ & $(k\cdot m)\cdot n = k\cdot (m\cdot n)$ \\
		d)&distributiv & \multicolumn{2}{c}{$k(m+n) = k\cdot m + k\cdot n$} \\
		\bottomrule
	\end{tabular}
\end{satz}

\begin{conclusion}
	Es gilt $\forall k,m,n\in\mathbb{N}$:
	\begin{enumerate}[label={\alph*)}]
		\item $m\neg 0 \Rightarrow m+n \neg 0$
		\item $m\cdot n = 0 \Leftrightarrow m = 0 \lor n = 0$
		\item $m + k = n + k \Leftrightarrow m = n$ (Kürzungsregel Addition)
		\item $k\neg 0: m\cdot k = n\cdot k \Leftrightarrow m = n$ (Kürzungsregel Multiplikation)
	\end{enumerate}
\end{conclusion}

\subsection*{Ordnung auf $\boldsymbol{\mathbb{N}}$}
\begin{definition}
	Betr. Relation $R:=\{(m,n) \in\mathbb{N}\times\mathbb{N}|m \le n\}$
\end{definition}
\begin{satz}
	Es gilt auf $\mathbb{N}$:
	\begin{enumerate}[label={\arabic*)}]
		\item $m\le n \Rightarrow \exists!k\in\mathbb{N}: n = m + k$, nenne $n - m=:k$ \begriff{Differenz}
		\item Relation $R$ (bzw. "$\le$") ist Totalordnung auf $\mathbb{N}$
		\item Ordnung ``$\leq$'' ist verträglich mit Addition und Multiplikation
	\end{enumerate}
\end{satz}

\section{Ganze und rationale Zahlen}
\begin{definition}
	Definiere Äquivalenzrelation $Q:=\{ ((n_1,n_1'),(n_2,n_2'))\in((\mathbb{N}\times\mathbb{N})\times(\mathbb{N}\times\mathbb{N})) | n_1+n_2' = n_1' + n_2 \}$
\end{definition}
\begin{satz}
	$Q$ ist Äquivalenzrelation auf $\mathbb{N}\times\mathbb{N}$.
\end{satz}

\begin{satz}
	Sei $[(n,n')]\in\overline{\mathbb{Z}}$. Dann ex. eindeutige $n'\in\mathbb{N}:(n',0)\in[(n,n')]$ falls $n\ge n'$ bzw. $(0,n')\in[(n,n')]$ falls $n<n'$.
\end{satz}

\subsection*{Rechenoperationen}
\begin{definition}
	\begriff{Addition}[!ganze Zahlen]: $\overline{m}+\overline{n} = [(m,n')] + [(n,n')] :=[(m+n,m'+n')]$
	
	\begriff{Multiplikation}[!ganze Zahlen]: $\overline{m}\cdot\overline{n} = \overline{m}\overline{n} = [(m,m')]\cdot[(n,n')]:=[(mn+m'n',mn'+m'n)]$
\end{definition}

\begin{satz}
	Addition und Multiplikation sind eindeutig definiert, d.h. unabhängig vom Repräsentanten bzgl. $Q$.
\end{satz}

\begin{satz}
	Für Addition und Multiplikation auf $Z$ gilt $\forall \overline{m},\overline{n}\in\overline{\mathbb{Z}}$:
	\begin{enumerate}[label={\arabic*)}]
		\item Es ex. neutrales Element $0:=[(0,0)]$ (Add.), $1:=[(1,0)]$ (Mult., $=[(k,k)]$)
		\item Jeweils kommutativ, assoziativ und gemeinsam distributiv
		\item $-\overline{n} := [(n',n)]\in\overline{\mathbb{Z}}$ ist Inverses bzgl. Addition von $[(n,n')]=\overline{n}$
		\item $(-1)\cdot \overline{n} = -\overline{n}$
		\item $\overline{m}\cdot\overline{n} = 0 \Leftrightarrow \overline{m} = 0 \lor \overline{n} = 0$
	\end{enumerate}
\end{satz}

\begin{satz}
	Für $\overline{m},\overline{n}\in\overline{\mathbb{Z}}$ hat Gleichung $\overline{m} = \overline{n} + \overline{x}$ eindeutige Lösung $\overline{x} = \overline{m} + (-\overline{n}) = [(m+n'),(m'+n)]$.
\end{satz}

\subsection*{Ordnung auf $\overline{\mathbb{Z}}$}
\begin{definition}
	Betr. Relation $R:=\{(\overline{m},\overline{n})\in\overline{\mathbb{Z}}\times\overline{\mathbb{Z}} | \overline{m} \le \overline{n}\}$, wobei $\overline{m} = [(m,m')] \le [(n,n')]$ \gls{gdw} $(m+n'\le m'+n)$
\end{definition}

\begin{satz}
	$R$ ist Totalordnung auf $\overline{\mathbb{Z}}$, die verträglich ist mit Addition und Multiplikation.
\end{satz}

\begin{definition}
	Betr. $\mathbb{Z} = \mathbb{Z}\cup\{ (-k) | k\in\mathbb{N}_{>0} \}$ mit üblicher Addition, Multiplikation und Ordnung "$\ge$".
\end{definition}
\begin{satz}
	$\mathbb{Z},\overline{\mathbb{Z}}$ sind isomorph bzgl. Addition, Multiplikation, Ordnung.
\end{satz}

\subsection*{Rationale Zahlen}
\begin{definition}
	Betr. Relation $Q:=\left\lbrace \left. \left( \frac{n_1}{n_1'},\frac{n_2}{n_2'}\right) \in \left( \mathbb{Z}\times\mathbb{Z}_{\neq 0}\right)\times\left(\mathbb{Z}\times\mathbb{Z}_{\neq 0}\right) \right| n_1n_2' = n_1'n_2\right\rbrace$
	
	Setzte $\mathbb{Q} := \left\lbrace \left[ \left. \frac{n}{n'}\right] \right| (n,n')\in\mathbb{Z}\times\mathbb{Z}_{\neq 0}\right\rbrace$ Menge der \begriff{rationale Zahlen}.
	
	Offenbar gilt \begriff{Kürzungsregel}[!rationale Zahlen] $\left[ \frac{n}{n'}\right] = \left[ \frac{k\cdot n}{k\cdot n'}\right]\;\forall k\in\mathbb{Z}_{\neq 0}$.
\end{definition}

\subsection*{Rechenoperationen auf $\mathbb{Q}$}
\begin{definition}
	\begriff{Addition}[!rationale Zahlen]: $\left[ \frac{m}{m'}\right] + \left[ \frac{n}{n'}\right] := \left[ \frac{mn' + m'n}{m'+n'}\right]$
	
	\begriff{Multiplikation}[!rationale Zahlen]: $\left[\frac{m}{m'}\right]\cdot\left[\frac{n}{n'}\right]:=\left[\frac{m\cdot n}{m'\cdot n'}\right]$
	
	Addition und Multiplikation sind unabhängig vom Repräsentanten bzgl. $Q$ $\Rightarrow$ Operationen auf $Q$ eindeutig definiert.
\end{definition}

\begin{satz}
	Mit Addition und Multiplikation ist $\mathbb{Q}$ Körper mit
	\begin{itemize}
	\item neutralem Element $0:=\left[\frac{0_\mathbb{Z}}{1_\mathbb{Z}}\right] = \left[\frac{0_\mathbb{Z}}{n}\right], 1 :=\left[\frac{1_\mathbb{Z}}{1_\mathbb{Z}}\right] = \left[ \frac{n}{n}\right] \neq 0\;n\neq 0$
	\item Inverse Elemente $-\left[\frac{n}{n'}\right] = \left[ \frac{-n}{n'}\right], \left[\frac{n}{n'}\right]^{-1} = \left[\frac{n'}{n}\right]$
	\end{itemize}
\end{satz}

\subsection*{Ordnung auf $\mathbb{Q}$}
\begin{definition}
	Relation $R:=\left\lbrace \left. \left( \left[\frac{m}{m'}\right],\left[\frac{n}{n'}\right]\right)\in\mathbb{Q}\times\mathbb{Q} \right| mn'\le m'n'; m',n'>0\right\rbrace$ gibt Ordnung "$\le$".
\end{definition}

\begin{satz}
	$\mathbb{Q}$ ist angeordneter Körper ("$\le$" ist Totalordnung verträglich mit Addition und Multiplikation).
\end{satz}
\begin{conclusion}
	Körper $\mathbb{Q}$ ist \begriff{archimedisch angeordnet}, d.h. $\forall q\in\mathbb{Q} \exists n\in\mathbb{N}: q < n$.
\end{conclusion}

\section{Reelle Zahlen}
\subsection*{Struktur von archimedisch angeordneten Körpern}
\begin{satz}
	Sei $K$ Körper. Dann gilt $\forall a,b\in K$:
	\begin{enumerate}[label={\arabic*)}]
		\item $0,1,(-a),b^{-1} (b\neq 0)$ sind eindeutig bestimmt
		\item $(-0) = 0, 1^{-1} = 1$
		\item $-(-a) = a, (b^{-1})^{-1} = b (b\neq 0)$
		\item $-(a+b) = (-a) + (-b), (ab)^{-1} = a^{-1}b^{-1} (a,b\neq 0)$
		\item $-a = (-1) a, (-a)(-b) = ab, a\cdot 0 = 0$
		\item $ab = 0 \Leftrightarrow a=0\lor b = 0$
		\item $a+x = b$ hat eindeutige Lösung $x = b+(-a) =: b-a$ \begriff{Differenz}
		
		$ax=b (a\neq 0)$ hat eindeutige Lösung $x=a^{-1}b =:\frac{b}{a}$ \begriff{Quotient}
	\end{enumerate}
\end{satz}

\begin{definition}
	\begin{itemize}
	\item \begriff{Vielfache}: $na := \sum_{k=1}^{n}a$
	
	Damit:
	\begin{itemize}
		\item $(-n)a := n(-a), 0_\mathbb{N} a := a_K$ für $n\in\mathbb{N}_{\ge 1}$
		\item $ma + na = (m+n)a, na + nb = n(a+b)$
		\item $(ma)\cdot(na) = (mn)a^2, (-n)a = -(na)$
	\end{itemize}
	\item \begriff{Potenz}: $a^n$ von $a\in K, n\in\mathbb{Z}:=\prod_{k=1}^{n} a$
	
	Damit
	\begin{itemize}
		\item $a^{-n} :=(a^{-1})^n, a^{0_K}:=1_K$ für $n\in\mathbb{N}_{\ge 1}, a\neq 0$
		\item $a^m a^n = a^{m+n}, (a^m)^n = a^{mn}, a^nb^n = (ab)^n, a^{-n} = (a^n)^{-1}$
	\end{itemize}
	\item \begriff{Fakkultät} für $n\in\mathbb{N}:$\mathsymbol*{n}{$n"!$} $n!:=\prod_{k=1}^n k, 0!=1$
	\item \begriff{Binomialkoeffizient} \mathsymbol{noverm}{$\binom{n}{k}$}$:=\frac{n!}{k!(n-k)!}\in\mathbb{N}$ $\forall k,n\in\mathbb{N}, 0\le k\le n$
	\begin{itemize}
		\item $\binom{k+1}{n+1} = \binom{n}{k} + \binom{n}{k+1}$
		\item Rechenregel fürht auf \begriff{\person{Pascal}'sches Dreieck}
	\end{itemize}
	\end{itemize}
\end{definition}

\begin{satz}[Binomischer Satz]
	In Körper $K$ gilt: $(a+b)^n = \sum_{k=0}^n\binom{n}{k}a^n b^{n-k}, ,b\in K, n\in\mathbb{N}$
\end{satz}
\begin{satz}
	Sei $K$ angeordneter Körper. Dann gilt $\forall a,b,c,d\in K$:
	\begin{enumerate}[label={\alph*)}]
		\item $a < b \Leftrightarrow 0 < b-a$
		\item $a < b, c < d \Leftrightarrow a+c < b+d$
		
		$0 \le a < b, 0 \le c < d \Leftrightarrow a\cdot c < b\cdot d$
		\item $a < b \Leftrightarrow -b < -a$ (insbes. $a > 0 \Leftrightarrow -a < 0$)
		
		$a < b, c < 0 \Leftrightarrow a\cdot c > b \cdot c$
		\item $a\neq 0 \Leftrightarrow a^2 > 0$ (insbes. 1 > 0)
		\item $a > 0 \Leftrightarrow a^{-1} > 0$
		\item $0 < a < b \Leftrightarrow b^{-1} < a^{-1}$
	\end{enumerate}
\end{satz}

\begin{definition}
	\begriff{Absolutbetrag} $\vert\cdot\vert:K\rightarrow K$ (auf angeordneten Körper $K$) \[\vert a \vert:=\begin{cases}
	a&\text{für }a \ge 0 \\ -a& \text{für }a < 0\end{cases}\]
\end{definition}

\begin{satz}
	Sei $K$ angeordneter Körper. Dann gilt $\forall a,b\in K$:
	\begin{enumerate}[label={\arabic*)}]
		\item $\vert a\vert\ge 0, \vert a\vert\ge a$
		\item $\vert a\vert = 0$ \gls{gdw} $a=0$
		\item $\vert a\vert = \vert -a\vert$
		\item $\vert a\vert\cdot\vert b\vert = \vert a\cdot b\vert$
		\item $\left\vert \frac{a}{b}\right\vert = \frac{\vert a\vert}{\vert b\vert} (b\neq 0)$
		\item \begriff{Dreiecksungleichung}
		
		$\vert a+b\vert \le \vert a\vert + \vert b\vert$ ($\vert a-b\vert = \vert a+(-b)\vert \le \vert a\vert + \vert b\vert$)
		\item $\left\vert a\vert - \vert b\right\vert \le \vert a+b\vert$
		\item \begriff{\person{Bernoulli}-Ungleichung}
		
		$(1+a)^n \ge 1 + n\cdot a \;\forall a\ge -1, n\in\mathbb{N} (a\neq -1 \text{ bei }n = 0)$
		
		(Gleichheit \gls{gdw} $n=0,1$ oder $a=0$)
	\end{enumerate}
\end{satz}
\begin{definition}
	Betr. $f:\mathbb{Q}\rightarrow K$ mit $f\left(\frac{m}{n}\right):= \frac{m\cdot 1_K}{n\cdot 1_K}=(m 1_k)(n 1_K)^{-1}\;\forall m\in\mathbb{Z},k\in\mathbb{Z}_{\neq 0}$
\end{definition}
\begin{satz}
	Sei $K$ angeordneter Körper\\
	$\Rightarrow$ $f:\mathbb{Q}\rightarrow K$ ist injektiv und $f$ erhält die Körperstruktur und Ordnung, d.h. $\forall p,q\in\mathbb{Q}$:
	\begin{itemize}
		\item $f(p+q) = f(p) + f(q), f(0) = 0_K, f(-p) = -f(p)$
		\item $f(p\cdot q) = f(q)\cdot f(q), f(1) = 1_K, f(p^{-1}) = f(p)^{-1} (p\neq 0)$
		\item $p \le_\mathbb{Q} q \Leftrightarrow f(p) \le_K f(q)$
	\end{itemize}
\end{satz}

\begin{conclusion}
	Es gilt im angeordneten Körper:
	\begin{enumerate}[label={\arabic*)}]
		\item $\mathbb{Q}_K = f(\mathbb{Q})$ ist mit Addition, Multiplikation und Ordnung von $K$ selbst angeordneter Körper
		\item $\mathbb{Q}_K$ ist isomorph zu $\mathbb{Q}$ bzgl. Körperstruktur und Ordnung.
	\end{enumerate}
\end{conclusion}

\begin{definition}
	Angeordneter Körper heißt \begriff{archimedisch}, falls $\forall a\in K\,\exists n\in\mathbb{N}\subset K: a < n$.
\end{definition}
\begin{satz}
	Sei $K$ archimedisch angeordneter Körper. Dann\begin{enumerate}[label={\arabic*)}]
		\item $\forall a,b\in K$ mit $a,b>0\,\exists n\in\mathbb{N}: n\cdot a > b$
		\item $\forall a\in K\,\exists!\;[a]\in\mathbb{Z}: [a]\le a \le [a] +1$, \mathsymbol{a}{$[a]$} heißt \begriff{ganzer Anteil} von $a$
		\item $\forall \epsilon \in K$ mit $\epsilon > 0\,\exists n\in\mathbb{N}_{\neq 0}: \frac{1}{n}< \epsilon$ (beachte: $0 < \frac{1}{n}$)
		\item $\forall a,b\in K$ mit $a>1\,\exists n\in\mathbb{N}: a^n > b$
		\item $\forall a,\epsilon > 0\,\exists p,q\in\mathbb{Q}: p \le a  q$ und $q - p < \epsilon$
		
		(d.h. $a\in K$ kann auch rationale Zahlen beliebig genau approximiert werden, $\mathbb{Q}$ ``dicht'' in $K$)
		\item $\forall a,b\in K, a < b\,\exists q\in\mathbb{Q}:a < q < b$.
	\end{enumerate}
\end{satz}

\begin{definition}[Intervall]
	\begriff{Intervall} für angeordneten Körper $K$: Sei $a,b\in K$:
	\begin{itemize}
		\item \begriff{beschränktes Intervall}
		\begin{itemize}
			\item $[a,b]:=\{ x\in K | a \le x \le b \}$ \begriff[Intervall!]{abgeschlossen}
			\item $(a,b):=\{a < x < b\}$ \begriff[Intervall!]{offen}
			\item $[a,b) := \{a \le x < b\}, (a,b]:=\{a < x \le b\}$ \begriff[Intervall!]{halboffen}
		\end{itemize}
		\item \begriff{unbeschränktes Intervall}
		\begin{itemize}
			\item $[a,\infty]:=\{x\in K\mid a \le x\}$
			\item $(a,\infty):=\{x\in K\mid a > x\}$
            \item $(-\infty, b]:= \{x \in K \mid x< a\}$
            \item $(-\infty, b) := \{x\in K\mid x \leq b\}$
		\end{itemize}
	\end{itemize}
\end{definition}

\begin{definition}[Folge]
    Eine \begriff{Folge} in Menge $M$ ist eine Abbildung $\alpha:\mathbb{N}\rightarrow M$ (evtl. $\alpha:\mathbb{N}_{\ge n}\rightarrow M$), $\alpha_n := \alpha(n)$ heißen \begriff{Folgenglieder}, und \begriff{Folgenindex}.
    
	Notation: $\{a_n\}_{n\in\mathbb{N}}, \{\alpha_n\}_{k=1}^\infty$ bzw. $\alpha_0, \alpha_1, \dotsc$\\
	kurz: $\{\alpha_n\}_n, \{\alpha_n \}$
		
	Hinweis: $\{x\}_n$ ist \begriff{konstante Folge}, d.h. $\alpha_n = \alpha\;\forall n$
\end{definition}

Aussage gilt für \gls{fa} $n\in\mathbb{N}$, wenn höchstens für endlich viele $n$ falsch.

\begin{definition}[Intervallschachtelung]
	Folge $\{x_n\}_{n\in\mathbb{N}} =:\mathcal{X}$ von abgeschlossenen Intervallen $X_N[x_n, x_n']\subset K (x_n, x_n'\in K)$ heißt \begriff{Intervallschachtelung} (im angeordneten Körper K), falls
	\begin{enumerate}[label={\alph*)}]
		\item $X_n\neq \emptyset$ und $X_{n+1}\subset X_n\;\forall n\in\mathbb{N}$
		\item $\forall\epsilon > 0$ in $K$ existiert $n\in\mathbb{N}: l(X_n):= x_n' - x_n < \epsilon$, mit $l$ \begriff{Intervalllänge}
	\end{enumerate}
\end{definition}

\begin{lemma}
	Sei $\mathcal{X} = \{X_n\}_{n\in\mathbb{N}}$ Intervallschachtelung im angeordneten Körper $K$\\
	$\Rightarrow \bigcap_{n\in\mathbb{N}} X_n$ enthält höchstens ein Element.
\end{lemma}

\begin{definition}
	Archimedisch angeordneter Körper heißt \begriff{vollständig}, falls $\bigcap_{n\in\mathbb{N}} X_n\neq \emptyset$ für jede Intervallschachtelung $\mathcal{X} = \{x_n\}$ in $K$.
\end{definition}

\begin{definition}
	$Q:=\{ (\{x_n\}, \{y_n\})\in I_\mathbb{Q}\times I_\mathbb{Q} \}$ ist Relation auf $I_\mathbb{Q}$, $I_\mathbb{Q}:=$ Menge aller Intervallschachtelungen $\mathcal{X}=\{x_n\} \in \mathbb{Q}$.
\end{definition}

\begin{satz}
	$Q$ ist Äquivalenzrelation auf $I_\mathbb{Q}$.
\end{satz}

\begin{definition}
	setze $\mathbb{R} := \{ [\mathcal{X}] \mid \mathcal{X}\in I_\mathbb{Q} \}$ Menge der \begriff{reellen Zahlen}.
	
	\begin{itemize}
		\item $\bigcap_{n\in\mathbb{N}} x_n\neq 0 \rightarrow [\mathcal{X}]$ ist ``neue'' sog. \begriff{irrationale Zahl}
	\end{itemize}
\end{definition}

\subsection*{Rechenoperationen}
\begin{definition}
	Für Intervalle $X=[x,x'], Y=[y,y']$ in $\mathbb{Q}$ defineren wir Intervall in $\mathbb{Q}$:
	\begin{itemize}
		\item $X + Y := \{\xi + y \mid \xi \in X, y\in Y\} = [x + y, x' + y']$
		\item $X\cdot Y :=\{\xi \cdot y \mid \xi \in X, y\in Y\} = [\tilde{x}\tilde{y}, \tilde{x}'\tilde{y}'], x,x'\in\{x,x'\},y,y'\in\{y,y'\}$
		\item $-x := [-x,-x']$, $x^{-1}:=[\frac{1}{x'}, \frac{1}{x}]$ falls $0\in X$
	\end{itemize}

	Für relle Zahl $[\mathcal{X}] = [\{x_n\}], [\mathcal{Y}]=[\{y_n\}]$ sei
	\begin{itemize}
		\item $[\mathcal{X}]+\mathcal{Y} :=[\{x_n + y_n\}]$
		\item $[\mathcal{X}]\cdot[\mathcal{Y}] :=[\{x_n\cdot y_n\}]$
		\item $-[\mathcal{X}]:=[\{-x_n\}]$, $[\mathcal{X}]^{-1} := [\{x_n^{-1}\}]$ falls $[\mathcal{X}]\neq 0_\mathbb{R}$
	\end{itemize}
\end{definition}

\begin{satz}
	\begin{enumerate}[label={\arabic*)}]
		\item Addition, Multiplikation und Inverse sind in $\mathbb{R}$ eindeutig definiert
		\item $\mathbb{R}$ ist damit und neutralen Elementen ein Körper.
	\end{enumerate}
\end{satz}

\subsection*{Ordnung auf $\mathbb{R}$}
\begin{definition}
	Betr. Relation "$\le$": $R:=\{ ([\{x_n\}],[\{y_n\}])\in\mathbb{R}\times\mathbb{R} | x_n \le y_n\,\forall n\in\mathbb{N}\}$
\end{definition}
\begin{satz}
	$\mathbb{R}$ ist mit "`$\le$"' angeordneter Körper.
\end{satz}
\begin{satz}
	$\mathbb{R}$ ist archimedisch angeordneter Körper.
\end{satz}
\begin{theorem}
	$\mathbb{R}$ ist vollständiger, archimedisch angeordneter Körper.
\end{theorem}
\begin{theorem}
	Sei $K$ vollständiger, archimedisch angeordneter Körper\\
	$\Rightarrow K$ ist isomorph zu $\mathbb{R}$ bzgl. Körperstruktur und Ordnung.
\end{theorem}

\begin{definition}
	Sei $M\subset K$, $K$ angeordneter Körper.
	\begin{itemize}
		\item $s\in K$ ist \begriff[Schranke!]{obere} / \begriff[Schranke!]{untere} \begriff{Schranke} von $M$, falls $x \le s (x \ge s)\;\forall x\in M$
		
		$M$ ist nach \begriff[beschränkt!]{oben} / \begriff[beschränkt!]{unten} \highlight{beschränkt}, falls obere ( untere ) Schranke existiert.
		\item $M$ \begriff{beschränkt}[!Menge im Körper], falls $M$ nach oben und unten beschränkt.
		\item kleinste obere (größte untere) Schranke $\tilde{s}$ von $M$ ist \begriff{Supremum} (\begriff{Infimum}) von $M$, d.h. \\
		\mathsymbol{sup}{$\sup$}$ M:= \tilde{s} \le s ($\mathsymbol{inf}{$\inf$}$ M = s \ge \tilde{s}) \;$ obere (untere) Schranken $s\in M$.
		\item Falls $\sup M \in M (\inf M\in M)$ nennt man dies auch \begriff{Maximum} (\begriff{Minimum}) von $M$.
		
		kurz: \mathsymbol{max}{$\max$}$M = \sup M ($\mathsymbol{min}{$\min$}$M = \inf M)$
		\item falls $M$ nach oben (unten) \begriff{unbeschränkt}, d.h. nicht beschränkt, schreibt man auch $\sup M = \infty (\inf M = -\infty)$
	\end{itemize}

	Man hat
	\begin{align*}
	\sup M &= \min\{s \mid s \text{ obere Schranke von } M\}\\
	\inf M &= \max\{s \mid s \text{ untere Schranke von } M\}
	\end{align*}
\end{definition}
\stepcounter{theorem}
\begin{satz}
	Sei $K$ angeordneter Körper, $M\subset K$. Falls $\sup M\;(\inf M)$ existiert, dann
	\begin{enumerate}[label={\arabic*)}]
		\item $\sup M\;(\inf M)$ eindeutig
		\item $\forall \epsilon > 0\,\exists y\in M: \sup M < y + \epsilon\;(\inf M > y - \epsilon)$
	\end{enumerate}
\end{satz}

\begin{theorem}
	Sei $K$ archimedisch angeordneter Körper. Dann
	\[ K \text{ vollständig } \Leftrightarrow \sup M \slash \inf M \text{ ex. }\forall M\in K, M\neq \emptyset \text{ nach oben \slash unten beschränkt} \]
\end{theorem}

\subsection*{Anwendung: Wurzeln, Potenzen, Logarithmen in $\mathbb{R}$}
\begin{satz}[Wurzeln]
	Sei $a\in\mathbb{R}_{>0}, k\in\mathbb{N}_{>0} \Rightarrow \exists ! x\in \mathbb{R}_{>0}: x^k = a, \sqrt[k]{a}:=a^{\frac{1}{k}} = x$ heißt \highlight{k-te} \begriff{Wurzel} von $a$.
\end{satz}
\begin{definition}[Potenz]
	$n$-te \begriff{Potenz} von $a\in\mathbb{R}_{>0}, r\in\mathbb{R}$:
	
	Zunächst $r=\frac{m}{n}\in\mathbb{Q}$ (\gls{obda}) $n\in\mathbb{N}_{>0}$): $ a^{\frac{m}{n}}:= (a^m)^{\frac{1}{n}}$
	Allgemein für $a\ge 0, a > : a^r := \sup \{ a^q \mid 0 \le q \le r,q\in\mathbb{Q} \}$
	offenbar eindeutig definiert und allgemeine Definition konsistent mit Definition für $\frac{m}{n}\in\mathbb{Q}$.
	Damit: \begriff{Exponentialfunktion}
\end{definition}
\begin{satz}\label{satz_potenz_r}
	Seien $a,b\in\mathbb{R}_{>0}, r,s\in\mathbb{R}. Dann$
	\begin{enumerate}[label={\arabic*)}]
		\item $a^r b^r = (ab)^r, (a^r)^s = a^{rs}, a^ra^s = a^{r+s}$
		\item f. $r > 0: a < b \Leftrightarrow a^r < b^r$
		\item für $a > 1: r < s \Leftrightarrow a^r < a^s$
	\end{enumerate}
\end{satz}

\begin{definition}[Logarithmus]
	Sei $a,b\in\mathbb{R}_{<0}, a\neq 1$: \begriff{Logarithmus}\highlight{von $b$ zur Basis $a$} ist \begin{align*}
	 \log_a b :=\begin{cases}
	 \sup \{ r \in \mathbb{R} \mid a^r \le b\}& a > 1\\
 	\sup \{r\in\mathbb{R}\mid a^r \ge b\}& 0 < a < 1
	 \end{cases}
	\end{align*}
\end{definition}
\begin{satz}\label{satz_logarithmus_r}
	Se $a,b,c\in\mathbb{R}_{>0}, a\neq 1$. Dann
	\begin{enumerate}[label={\arabic*)}]
		\item $log_a b$ ist eindeutige Lösung von $a^x = b$, d.h. $a^{log_a b} = b$
		\item $\log_a a = 1, log_a 1 = 0$
		\item $\log_a b^\gamma = \gamma \log_a b \;\forall \gamma\in\mathbb{R}$
		\item $\log_a(bc) = \log_a b + \log_a c, \log_a \frac{b}{c} = \log_a b - \log_a c$
		\item $\log_a b = \frac{\log_\alpha b}{\log_\alpha a},;\forall \alpha\in\mathbb{R}_{>0},\alpha\neq 1$
	\end{enumerate}
\end{satz}

\subsection*{Mächtigkeit von Mengen}
\begin{definition}
	$M$ \begriff[Mächtigkeit!]{endlich}, falls $M$ endlich viele Elemente hat, sonst \begriff[Mächtigkeit!]{unendlich}.
	
	Unendliches $M$ ist \begriff{abzählbar}, falls bijektive Abbildung $f:\mathbb{N}\rightarrow M$ existiert, sonst ist $M$ \begriff{überabzählbar}.
\end{definition}
\begin{satz}
	Es gilt:
	\begin{enumerate}[label={\arabic*)}]
		\item $\mathbb{Z},\mathbb{Q}$ abzählbar
		\item $M$ abzählbar, $n\in\mathbb{N}_{>0} \Rightarrow M^n$ abzählbar ($\Rightarrow \mathbb{Z}^n, \mathbb{Q}^n$ abzählbar)
		\item Ein offenes Intervall $I\in\mathbb{R}\neq \emptyset $ ist überabzählbar
		\item $\mathcal{P}(\mathbb{N})$ ist überabzählbar.
	\end{enumerate}
\end{satz}

\section{Komplexe Zahlen}
\begin{definition}
	Betr. Menge der \begriff{komplexen Zahlen} $\mathbb{C}:=\mathbb{R}\times\mathbb{R} = \mathbb{R}^2$ mit Addition und Multiplikation:
	
	$(x,x') + (y,y') := (x+y, x'+y')$\\
	$(x,x')\cdot(y,y') :=(xy - x'y', xy' + x'y)$
	
	$\mathbb{C}$ ist ein Körper mit $0_\mathbb{C} = (0,0), 1_\mathbb{C} = (1,0), -(x,y)= (-x,-y), (x,y)^{-1} = \left(\frac{x}{x^2 + y^2}, \frac{-y}{x^2 + y^2}\right)$ mit \begriff{imaginäre Einheit}\mathsymbol{i}{$i$}$:=(0,1)$ schreibt man auch $z=x+iy$ statt $z = (x,y)$
	
	Nenne $x:=\realz(z)$ \begriff{Realteil}, $y:=\imagz(z)$ \begriff{Imaginärteil} von $z$.\\
	$\overline{z}:= x - iy$ zu $z$ \begriff{konjungiert}\highlight{komplexe Zahl}
	
	Komplexe Zahl $Z = x+i0 = x$ wird mit reellen Zahl $x\in\mathbb{R}$ identifiziert. Offenbar ist $i^2 = (0,1)^2 = -1$, d.h. $z = i\in\mathbb{C}$ löst Gleichung $z^2 = -1$.
	
	Betrag $\vert\cdot\vert:\mathbb{C}\rightarrow \mathbb{R}_{>0}$ mit $\vert z\vert :=\sqrt{x^2 + y^2}$ ist Beträg / Länge des Vektors.
	
	Es gilt:
	\begin{enumerate}[label={\alph*)}]
		\item $\Re z = \frac{z+\overline{z}}{z}, \im z = \frac{z - \overline{z}}{z}$
		\item $\overline{z_1 + z_2} = \overline{z_1} + \overline{z_2}, \overline{z_1 \cdot z_2} = \overline{z_1}\cdot \overline{z_2}$
		\item $|z| = 0 \Leftrightarrow z = 0$
		\item $|z | = |\overline{z}|$
		\item $|z_1 \cdot z_2 | = |z_1| \cdot |z_2|$
	\end{enumerate}
\end{definition}

\chapter{Metrische Räume und Konvergenz}\addtocounter{section}{6}
\section{Grundlegende Ungleichungen}
\begin{satz}[geoemtrisches / arithemtisches Mittel]
	Seien $x_1, \dotsc, x_n\in\mathbb{R}_{>0}$.\\
	\[\Rightarrow \underbrace{\sqrt[n]{x_1\cdot x_2\cdot \dotsc \cdot x_n}}_{\text{\begriff{geometrisches Mittel}}} \le \underbrace{\frac{x_1 + \dotsc + x_n}{n}}_{\text{\begriff{arithmetisches Mittel}}}\]
\end{satz}
\begin{satz}[allgemeine \person{Bernoulli}-Ungleichung]
	Seien $\alpha,x\in\mathbb{R}$. Dann
	\begin{enumerate}[label={\arabic*)}]
		\item $(1+x)^\alpha \ge 1 + \alpha x\;\forall x\ge -1, \alpha > 1$
		\item $(1+x)^\alpha \le 1+\alpha x \;\forall x\ge -1, 0 < \alpha < 1$
	\end{enumerate}
\end{satz}
\begin{satz}[\person{Young}-sche Ungleichung]
	Seien $p,q\in\mathbb{R}, p,q > 1$ mit $\frac{1}{p}+\frac{1}{q}=1$.\\
	$\Rightarrow a\cdot b \le \frac{a^p}{p} + \frac{b^q}{q}\;\forall a,b\ge 0$
	
	\uline{Spezialfall:} $p=q=2: ab \le \frac{a^2+b^2}{2} \;\forall a,b\in \mathbb{R}$
\end{satz}
\begin{satz}[\person{Hölder}'sche Ungleichung]
	Sei $p,q\in\mathbb{R}, p,q > 1$ mit $\frac{1}{p} + \frac{1}{q} = 1$\\
	$\Rightarrow \sum_{i=1}^{n} |x_i y_i| \le \left(\sum_{i=1}^n |x_i|^p \right)^{\frac{1}{p}}\left(\sum_{i=1}^n |y_i|^q\right)^{\frac{1}{q}}\;\forall x,y\in\mathbb{R}$
	
	Für $p=q=2$ heißt die Ungleichung \begriff{\person{Cauchy}-\person{Schwarz}'sche Ungleichung}
\end{satz}
\begin{satz}[\person{Minkowski}-Ungleichung]
	Sei $p\in\mathbb{R}, p>1$\\
%	$\Rightarrow \big(\sum_{i=1}\^n|x_i + y_i|^p\big)^{\frac{1}{p}} \le \big( \sum_{i=1}^n |x_i|^p \big)^{\frac{1}{p}}+\big( \sum_{i=1}^n |y_i|^p \big)^\frac{1}{p}$
    $\Rightarrow \big(\sum_{i=1}^{n} \vert x_i + y_i \vert^p \big)^\frac{1}{p} \leq \big(\sum_{i=1}^{n} \vert x_i \vert^p \big)^\frac{1}{p} + \big(\sum_{i=1}^{n} \vert y_i \vert^p \big)^\frac{1}{p}\;\forall x,y\in \mathbb{R}$
\end{satz}

\begin{remark}
	\begin{enumerate}[label={\arabic*)}]
    \item Ungleichung gilt auch für $x_i, y_i \in \mathbb{C}$
    \item ist $\Delta$-Ungleichung für $p$-Normen
    \end{enumerate}
\end{remark}

\section{Metrische Räume}
\begin{definition}[Metrik]
	Sei $X$ Menge, Abbildung $d:X\times X\rightarrow \mathbb{R}$ heißt \begriff{Metrik} auf $X$, falls $\forall x,y,z\in X$:
	\begin{enumerate}[label={\alph*)}]
		\item $d(x,y) = 0 \Leftrightarrow x=y$
		\item $d(x,y) = d(y,x)$ \begriff[Metrik!]{Symmetrie}\index{Symmetrie!Metrik}
		\item $d(x,z)\le d(x,y) + d(y,z)$ \begriff{Dreiecksungleichung}[!Metrik]
	\end{enumerate}

	$(X,d)$ heißt \begriff{metrischer Raum}.
\end{definition}
\stepcounter{theorem}
\begin{example}
	\begriff{Diskrete Metrik} auf bel. Menge $X$ ist \[ d(x,y) = \begin{cases}0& x=y \\ 1 & x\neq y \end{cases} \] ist offenbar Metrik.
\end{example}
\begin{example}
	Sei $(X,d)$ metrischer Raum, $Y\subset X$\\
	$\Rightarrow (Y,\tilde{d})$ ist metrischer Raum mit \begriff{induzierte Metrik} $\tilde{d}(x,y) := d(x,y)\;\forall x,y\in X$.
\end{example}

\begin{definition}[Norm]
	Sei $X$ Vektorraum über $K=\mathbb{R}$ bzw. $K=\mathbb{C}$.
	
	Abbildung \mathsymbol{.}{$\parallel.\parallel$}$: X\rightarrow\mathbb{R}$ heißt \begriff{Norm} auf $X$, falls $\forall x,y\in X$
	\begin{enumerate}[label={\alph*)}]
		\item $\parallel x\parallel = 0$ \gls{gdw} $x = 0$
		\item \label{norm_2} $\parallel \lambda\cdot x\parallel = |\lambda| \cdot \parallel x \parallel\;\forall \lambda\in K$ (\begriff{Homogenität})
		\item \label{norm_3} $\parallel x + y\parallel \le \parallel x \parallel + \parallel y \parallel$ \begriff{Dreiecksungleichung}[!Vektorraum]
	\end{enumerate}

	$(X,\parallel . \parallel)$ heißt \begriff{normierter Raum}
\end{definition}
\begin{definition}[Halbnorm]
	$\parallel . \parallel:X\rightarrow\mathbb{R}_{\ge0}$ heißt \begriff{Halbnorm}, falls nur \ref{norm_2} und \ref{norm_3} gelten.
\end{definition}
\begin{satz}
	Sei $(X,\parallel .\parallel)$ normierter Raum.\\
	$\Rightarrow X$ ist metrischer Raum mit Metrik $d(x,y):=\parallel x - y \parallel\;\forall x,y\in X$.
\end{satz}
\begin{example}
	\label{norm_r}
	Man hat u.a. folgende Normen auf $\mathbb{R}^n$:
	\begin{description}
		\item[\begriff{$p$-Norm}] $\vert x\vert_p:=\left(\sum_{i=1}^n |x_i|^p\right)^\frac{1}{p}\;(1\le p<\infty)$
		\item[\begriff{Maximum-Norm}] $|x|_\infty :=\max\{|x_i| \mid i=1,\dots,n\}$
	\end{description}

	Standardnorm im $\mathbb{R}^n: \vert \cdot \vert:=\vert \cdot \vert_{p=2}$ heißt \begriff{euklidische Norm}
\end{example}
\begin{definition}[Skalarprodukt]
	$\langle x,y\rangle:=\sum_{i=1}^n x_i y_i$ heißt \begriff{Skalarprodukt}[!$\mathbb{R}$] (\begriff{inneres Produkt}) von $x,y\in\mathbb{R}^n$.
	
	Offenbar ist $\langle x,y\rangle = |x|^2\;\forall x\in\mathbb{R}^n$ (\highlight{ausschließlich für Euklidische Norm})\\
	Man hat $|\langle x,y\rangle | \le |x|\cdot |x|\;\forall x,y\in\mathbb{R}^n$ (\begriff{\person{Cauchy}-\person{Schwarz}'sche Ungleichung})
\end{definition}
\begin{example}
	$X=\mathbb{C}^n$ ist Vektorraum über $\mathbb{C}$, $x=(x_1,\dotsc,x_n)\in\mathbb{C}^n, x_i\in\mathbb{C}$.
	
	Analog zu \ref{norm_r} sind $\vert\cdot\vert_p$ und $\vert\cdot\vert_\infty$ Normen auf $\mathbb{C}^n$
	
	$\langle x,y\rangle :=\sum_{i=1}^n \overline{x_i} y_i \;\forall x,y\in\mathbb{C}$ heißt \begriff{Skalarprodukt}[!$\mathbb{C}$] von $x,y\in\mathbb{C}^n$.
	
	$x,y\in\mathbb{R}^n (\mathbb{C}^n)$ heißen \begriff{orthogonal}, falls $\langle x,y\rangle = 0$.
\end{example}
\begin{example}
	Sei $M$ beliebige Menge, $f:M\rightarrow \mathbb{R}$.
	\begin{itemize}
		\item $\parallel f \parallel :=\sup\{ \vert f(x)\vert \mid x\in M\}$ \begriff{Supremumsnorm}
		\item \mathsymbol{B}{$B$}$(M):=\{ f:M\rightarrow \mathbb{R} \mid\; \parallel f \parallel < \infty \}$ \begriff{Menge der beschränkten Funktionen}
	\end{itemize}
\end{example}
\stepcounter{theorem}
\stepcounter{theorem}
\begin{definition}
	Normen $\parallel .\parallel_1, \parallel .\parallel_2$ auf $X$ heißen \begriff{äquivalent}, falls $\exists \alpha,\beta > 0:\alpha \parallel x \parallel_1 \le \parallel x\parallel_2 \le \beta \parallel x\parallel_1 \;\forall x\in X$
\end{definition}
\begin{conclusion}
	$\vert\cdot\vert_p, \vert\cdot\vert_q$ sind äquivalent auf $\mathbb{R}^n\;\forall p,q\ge 1$.
\end{conclusion}

\begin{definition}
    \begin{itemize}
    \item $B_r(a):=\{ x\in X \mid d(a,x) < r \}$ heißt (offene)\begriff{Kugel} um $a$ mit Radius $r > 0$
    \item $B_r[a]:=\bar{B}_r(a):=\{ x\in X \mid d(a,x) \le r \}$ heißt (abgeschlossene)\begriff{Kugel} um $a$ mit Radius $r > 0$
    \end{itemize}
    Hinweis: muss keine “übliche” Kugel sein, zum Beispiel $\{ x\in \mathbb{R}^n \mid d(0,x) = \Vert x\Vert_{\infty} < 1 \}$ hat die Form eines ``üblichen'' Quadrats.
    \begin{itemize}
        \item Menge $M\subset X$ heißt \begriff{offen}, falls $\forall x\in M\;\exists \epsilon > 0: B_\epsilon(x) \subset M$
        \item Menge $M\subset X$ ist \begriff{abgeschlossen}, falls $X\setminus M$ offen
        \item $U\subset X$ \begriff{Umgebung} von $M$, falls $\exists V\subset X$ offen mit $M\subset V\subset U$
        \item $x\in M$ \begriff{innerer Punkt}, von $M$, falls $\exists \epsilon > 0: B_\epsilon(x)\subset M$
        \item $x\in X\setminus M$ \begriff{äußerer Punkt} von $M$, falls $\exists \epsilon > 0: B_\epsilon(x)\subset X\setminus M$
        \item $x\in X$ heißt \begriff{Randpunkt}, von $M$, wenn $x$ weder innerer noch äußerer Punkt
        \item \mathsymbol{int}{$\Int$}$ M:=$ Menge aller inneren Punkte von $M$, heißt \begriff{Inneres} von $M$
        \item \mathsymbol{ext}{$\Ext$}$M:=$ Menge aller äußeren Punkte von $M$, heißt \begriff{Äußeres} von $M$.
        \item \mathsymbol{p}{$\partial$}$M:=$ Menge der Randpunkte von $M$, heißt \begriff{Rand} von $M$
        \item \mathsymbol{cl}{$\cl$}$:=\overline{M} = \int M \cup \partial M$ heißt \begriff{Abschluss} von $M$
        \item $M\subset X$ heißt \begriff{beschränkt}[!Menge], falls $\exists a\in X, r>0: M\subset B_r(a)$
        \item $x\in X$ heißt \gls{hp} von $M$, falls $\forall \epsilon > 0$ enthält $B_\epsilon(x)$ unendlich viele Elemente aus $M$
        \item $x\in M$ heißt \begriff{isolierter Punkt} von $M$, falls $x$ kein Häufungspunkt
        \end{itemize}
\end{definition}
\stepcounter{theorem}

\begin{lemma}
	Sei $(X,d)$ metrischer Raum. Dann
	\begin{enumerate}[label={\arabic*)}]
		\item $B_r(a)$ offene Menge $\forall r>0,a\in X$
		\item $M\subset X$ beschränkt $\Rightarrow \forall a\in X\,\exists r>0: M\subset B_r(a)$
	\end{enumerate}
\end{lemma}

\begin{satz}\label{satz_topologie}
	Sei $(X,d)$ metrischer Raum, $\tau:=\{U\subset X \mid U \text{ offen}\}$. Dann
	\begin{enumerate}[label={\arabic*)}]
		\item \label{topologie_1} $X,\emptyset\in \tau$ offen
		\item \label{topologie_2} $\bigcap_{i=1}^n U_i\subset \tau$ falls $U_i\in\tau$ für $i=1,\dotsc,n$
		\item \label{topologie_3} $\bigcup_{U\in\tau'} U\in\tau$ falls $\tau'\in\tau$ 
	\end{enumerate}
\end{satz}
\begin{conclusion}
	Sei $(X,d)$ metrischer Raum, $\sigma :=\{ V\subset X \mid  V \text{ abgeschlossen}\}$. Dann
	\begin{enumerate}[label={\arabic*)}]
		\item $X,\emptyset \in \sigma$ abgeschlossen
		\item $\bigcup_{i=1}^n V_i\subset\sigma$ falls $V_i\in\sigma_i$ für $i=1,\dotsc, n$
		\item $\bigcap_{V\in\sigma'} V\in\sigma$ falls $\sigma'\subset\sigma$
	\end{enumerate}
\end{conclusion}

\begin{definition}[Topologie]
	Sei $X$ Menge, und $\tau$ Menge von Teilmengen von $X$, d.h. $\tau\subset\mathcal{P}(X)$.\\
	$\tau$ ist \begriff{Topologie} und $(X,\tau)$ \begriff{topologischer Raum}, falls \ref{topologie_1},\ref{topologie_2},\ref{topologie_3} aus \ref{satz_topologie} gelten.
\end{definition}
\begin{satz}
	Seien $\parallel.\parallel_1, \parallel.\parallel_2$ äquivalente Normen in $X$ und $U\subset X$. Dann \[ U\text{ offen bezüglich } \parallel .\parallel_1\; \Leftrightarrow\; U\text{ offen bzgl. } \parallel .\parallel_2 \]
\end{satz}
\begin{satz}
	Sei $(X,d)$ metrischer Raum und $M\subset X$: Dann
	\begin{enumerate}[label={\arabic*)}]
		\item $\Int M, \Ext M$ offen
		\item $\partial M, \cl M$ abgeschlossen
		\item $M = \Int M$, falls $M$ offen, $M=\cl M$ falls $M$ abgeschlossen
	\end{enumerate}
\end{satz}

\section{Konvergenz}\setcounter{theorem}{0}
\begin{definition}[konvergent]
	Sei $(X,d)$ metrischer Raum. Folge $\{x_n\}_{n\in\mathbb{N}}$ in $X$, (d.h. $x_n\in X\,\forall n$) heißt \begriff{konvergent}, falls $x\in X$ existiert mit \[\forall \epsilon > 0 \exists n_0=n_0(\epsilon)\in\mathbb{N}: d(x_n, x) < \epsilon\quad \forall n\ge n_0\]
	
	$x$ heißt dann \begriff{Grenzwert} (auch Limes) der Folge.
	
	Notation: $x=$\mathsymbol{lim}{$\lim\limits_{n\rightarrow\infty}$}, $x_n\rightarrow x$ für $n\rightarrow\infty$, $x_n \overset{n\rightarrow\infty}{\longrightarrow}x$
	
	Folge heißt \begriff{divergent}, falls nicht konvergent.
\end{definition}

\begin{conclusion}
	Für Folge $\{x_n\}$ gilt: \[ x=\lim\limits_{n\rightarrow\infty}x_n \;\Leftrightarrow \text{Jede Kugel $B_\epsilon(x)$ enthält fast alle $x_n$} \]
\end{conclusion}
\addtocounter{theorem}{4}
\begin{satz}[Eindeutigkeit des Grenzwertes]
	Sei $(X,d)$ metr. Raum, $\{x_n\}$ Folge in $X$. Dann \[ x,x' \text{ Grenzwert von $\{x_n\}$} \;\Rightarrow\; x = x' \]
\end{satz}
\begin{satz}
	Sei $(X,d)$ metrischer Raum\\
	$\Rightarrow$ konvergente Folge $\{x_n\}$ ist stets beschränkt
\end{satz}
\addtocounter{theorem}{4}
\begin{definition}
	Sei $\{x_n\}$ beliebige Folge in $X$, $\{n_k\}_{k\in\mathbb{N}}$ Folge in $\mathbb{N}$ mit $n_{k+1} > n_k\;\forall k\in\mathbb{N}$. Dann heißt $\{x_{n_k}\}_{k\in\mathbb{N}}$ \gls{tf} von $\{x_n\}_{n\in\mathbb{N}}$.
	
	$\gamma\in X$ heißt \gls{hw} (auch Häufungspunkt) der Folge $\{x_n\}$, falls $\forall \epsilon > 0$ enthält $B_\epsilon(\gamma)$ unendlich viele $x_n$.
\end{definition}
\begin{satz}\label{tfprinzip}
	Sei $\{x_n\}$ Folge im metrischen Raum $(X,d)$. Dann
	\begin{enumerate}[label={\arabic*)}]
		\item $x_n\rightarrow x \;\Rightarrow\; x_{n_k} \overset{n\rightarrow\infty}{\longrightarrow} x$ für jede \gls{tf} $\{x_{n_k}\}_k$
		\item $\gamma$ ist \gls{hw} der Folge $\{x_n\}$ $\Leftrightarrow$ $\exists$\gls{tf} $\{x_{n_k}\}: x_{n_k} \overset{n\rightarrow\infty}{\longrightarrow} \gamma$
		\item \begriff{Teilfolgenprinzip}: Jede \gls{tf} $\{x_{k'}\}$ von $\{x_n\}$ hat \gls{tf} $\{x_{k''}\}$ mit $x_{n''}\rightarrow x$ $\Rightarrow$ $x_n \rightarrow x$
	\end{enumerate}
\end{satz}
\begin{satz}
	Sei $(X,d)$ metrischer Raum, $M\subset X$ Teilmenge. Dann
	\[ M\text{ abgeschlossen} \quad\Leftrightarrow\quad \text{für jede konv. Folge $\{x_n\}$ in $M$ gilt: }\lim\limits_{n\rightarrow\infty} x_n\in M \]
\end{satz}

\subsection*{Konvergenz im normierten Raum $X$}
\begin{satz}
	Sei $X$ normierter Raum, $\{x_n\}, \{y_n\}$ in $X$, $\{\lambda_n\}$ in $K$ mit $\lim x_n = x, \lim y_n = y$. Dann
	\begin{enumerate}[label={\arabic*)}]
		\item $\{x_n \pm y_n\}$ konvergiert und $\lim\limits_{n\rightarrow\infty}x_n + y_n = \lim\limits_{n\rightarrow\infty} x_n + \lim\limits_{n\rightarrow\infty} y_n$
		\item $\{\lambda_n x_n\}$ konvergiert und $\lim\limits_{n\rightarrow\infty} \lambda_n x_n = \lim\limits_{n\rightarrow\infty} x_n \cdot \lim\limits_{n\rightarrow\infty}x_n$
		\item $\lambda\neq 0 \;\Rightarrow\;\lim\limits_{n\rightarrow\infty} \frac{1}{\lambda_n} = \frac{1}{\lambda}$ (in $K$) für $\{\frac{1}{\lambda_n}\}_{n\ge\tilde{n}}$ ($\lambda_n\neq 0\;\forall n\ge\tilde{n}$)
	\end{enumerate}
\end{satz}
\begin{conclusion}
	Seien $\{\lambda_n\}, \{\mu_n\}$ Folgen in $K$ mit $\lambda_n\rightarrow\lambda,\mu_n\rightarrow\mu$. Dann
	\begin{enumerate}[label={\arabic*)}]
		\item $\lambda_n + \mu_n\rightarrow \lambda + \mu, \lambda_n \mu_n\rightarrow\lambda \mu$
		\item falls $\lambda\neq 0$ (\gls{obda} $\lambda_n\neq 0$): $\frac{\mu_n}{\lambda_n}\rightarrow\frac{\mu}{\lambda}$
	\end{enumerate}
\end{conclusion}
\stepcounter{theorem}
\begin{lemma}
	\begin{enumerate}[label={\arabic*)}]
		\item Im metrischen Raum $X$ gilt:$x_n\rightarrow x$ in $X$ $\Leftrightarrow\;d(x_n,x)\rightarrow 0$ in $\mathbb{R}$
		\item Sei $0\le \alpha_n\le\beta_n\;\forall n\in\mathbb{N}, \alpha_n, \beta_n\in\mathbb{R}, \beta_n\rightarrow 0$\\
		$\Rightarrow \alpha_n\rightarrow 0$ \begriff{Sandwitch-Prinzip}
	\end{enumerate}
\end{lemma}
\begin{satz}
	Sei $X$ normierter Raum, $\{x_n\}$ in $X$. Dann\\
	$x_n\rightarrow x$ in $X$ $\Rightarrow$ $\parallel x_n\parallel \rightarrow\parallel x\parallel$ in $\mathbb{R}$
\end{satz}
\begin{satz}
	Seien $(X,\parallel .\parallel_1)$, $(X,\parallel.\parallel_2)$ normierte Räume mit äquivalenten Normen. Dann
	
	$x_n\rightarrow x$ in $(X,\parallel.\parallel_1)$ $\Leftrightarrow$ $x_n\rightarrow x$ in $(X,\parallel.\parallel_2)$
\end{satz}
\stepcounter{theorem}
\begin{satz}[Konvergenz in $\mathbb{R}^n$/$\mathbb{C}^n$ bzgl. Norm]
	Sei $\{x_n\}$ Folge mit $x_n = (x_n^1, \dotsc, x_n^n)\in\mathbb{R} (\mathbb{C}^n)$, $x=(x^1, \dotsc,x^n)\in\mathbb{R}^n (\mathbb{C}^n)$.
	
	$\lim\limits_{n\rightarrow\infty} x_n = x$ in $\mathbb{R}^n (\mathbb{C}^n)$ $\Leftrightarrow$ $\lim\limits_{n\rightarrow\infty} x_k^j = xj$ in $\mathbb{R}$ bzw. $\mathbb{C}\;\forall j=1,\dotsc,n$
\end{satz}
\addtocounter{theorem}{3}
\subsection*{Konvergenz in $\mathbb{R}$}
\begin{satz}
	Seien $\{x_n\},\{y_n\},\{z_n\}$ Folgen in $\mathbb{R}$. Dann
	\begin{enumerate}[label={\arabic*)}]
		\item $x_n \le y_n\;\forall n\ge n_0, x_n\rightarrow x, y_n\rightarrow y\;\Rightarrow x\le y$
		\item $x_n\le y_n\le z_n\;\forall n\ge n_0, x_n\rightarrow c,z_n\rightarrow c \;\Rightarrow y_n\rightarrow c$ (\begriff{Sandwitch-Prinzip})
	\end{enumerate}
\end{satz}

\begin{definition}[monoton]
	Folge $\{x_n\}$ heißt \begriff[monoton!]{wachsend} / \begriff[monoton!]{fallend}, falls gilt:
	
	$x_n \le x_{n-1}\;(x_n\ge x_{n+1})\;\forall n\in\mathbb{N}$ (in beiden Fällen heißt Folge \begriff{monoton}).
	
	Falls stets ``$<$'' (``$>$'') ist $\{x_n\}$ \begriff[monoton!]{strikt}
\end{definition}
\begin{satz}
	Sei $\{x_n\}$ in $\mathbb{R}$ monoton und beschränkt.\[
	\{x_n\}\text{ konvergiert gegen }x:=
	\left\lbrace
		\begin{aligned}
			&\sup \{x_n \mid n\in\mathbb{N}\}, \\
			&\inf\{x_n \mid n\in\mathbb{N}\}, \\
		\end{aligned}
	\right.
	\text{ falls monoton }\;
	\begin{aligned}
		&\text{wachsend}\\
		&\text{fallend}
	\end{aligned}
	\]
\end{satz}
\addtocounter{theorem}{2}
\begin{theorem}[\person{Bolzano}-\person{Weierstraß}]\label{bolzano_weierstrass}
	$\{x_n\}$ beschränkte Folge in $\mathbb{R}$ $\Rightarrow$ $\{x_n\}$ hat konvergente \gls{tf}.
\end{theorem}
\stepcounter{theorem}

\subsection*{Oberer \slash Unterer Limes}
\begin{definition}
	Seien $\{x_n\}$ beschränkte Folgen in $\mathbb{R}$.\\
	$H:=\{ \gamma\in\mathbb{R} \mid \gamma \text{ ist \gls{hw} von }\{x_n\}\}$ ($\neq \emptyset$ nach \ref{bolzano_weierstrass})
	
	\begin{tabularx}{\textwidth}{ll}
		\mathsymbol*{limsup}{$\limsup$} $\limsup\limits_{n\rightarrow\infty} x_n := \overline{\lim}_{n\rightarrow\infty} x_n =:\sup H$ & \begriff{Limes superior} von $\{x_n\}$ \\[0.5cm]
		\mathsymbol*{liminf}{$\liminf$} $\liminf\limits_{n\rightarrow\infty} x_n = \underline{\lim}_{n\rightarrow\infty} x_n :=\inf H$  & \begriff{Limes inferior} von $\{x_n\}$
	\end{tabularx}
\end{definition}

\begin{satz}
	Sei $\{x_n\}$ beschränkte Folge in $\mathbb{R}$. Dann
	\begin{enumerate}[label={\arabic*)}]
		\item Sei $\{x_{n'}\}$ \gls{tf} mit $x_{n'}\rightarrow\gamma \;\Rightarrow \;\liminf\limits_{n\rightarrow\infty} x_n \le \gamma \le \limsup\limits_{n\rightarrow\infty} x_n$
		\item $\gamma' :=\liminf\limits_{n\rightarrow\infty} x_n$ und $\gamma'' := \limsup\limits_{n\rightarrow\infty} x_n$ sind \gls{hw} von $\{x_n\}$
		
		\begin{tabular}{ll}
		(folglich)& $\inf H = \min H, \sup H = \max H$ und \\
		& $\exists$ \gls{tf} $\{x_{n'}\}, \{x_{n''}\}, x_{n'}\rightarrow \gamma', x_{n''}\rightarrow\gamma''$
		\end{tabular}
		\item $x_n\rightarrow \alpha \;\Leftrightarrow \;\alpha = \liminf\limits_{n\rightarrow\infty} x_n = \limsup\limits_{n\rightarrow\infty} x_n$
	\end{enumerate}
\end{satz}
\stepcounter{theorem}
\section*{Uneigentliche Konvergenz}
\begin{definition}[Uneigentliche Konvergenz]
	Folge $\{x_n\}$ in $\mathbb{R}$ \begriff[Konvergenz!]{uneigentlich} gegen $+\infty (-\infty)$, falls $\forall R>0\,\exists n_0\in\mathbb{N}: x_n \ge R (x_n \le -R)\;\forall n\ge n_0$
	
	(heißt auch \begriff{bestimmt divergent}) gegen $\infty$, "uneigentlich" wird meist weggelassen.
	
	Notation: $\lim\limits_{n\rightarrow\infty} x_n = \pm \infty$ bzw. $\xi_n\rightarrow \pm \infty$
\end{definition}
\stepcounter{theorem}
\begin{satz}[Satz von \person{Stolz}]
	Sei $\{x_n\},\{y_n\}$ Folgen in $\mathbb{R}, \{y_n\}$ sei stren monoton wachsend, $\{y_n\}\rightarrow\infty$\\
	$\Rightarrow \lim\limits_{n\rightarrow\infty} \frac{x_n}{y_n} = \lim\limits_{n\rightarrow\infty} \frac{x_{n+1} - x_n}{y_{n+1} - y_n}$, falls rechter Grenzwert existiert (endlich oder unendlich)
\end{satz}
\stepcounter{theorem}
\begin{satz}
	Sei $\{x_n\}$ mit $x_n\rightarrow x$ im normierten Raum $X$.\\
	$\Rightarrow\frac{1}{n}\sum_{j=1}^n x_j \rightarrow n\rightarrow\infty x$
\end{satz}

\section{Vollständigkeit}
\begin{definition}[\person{Cauchy}-Folge]
	Folge $\{x_n\}$ im metrischen Raum $(X,d)$ heißt \gls{cf} (Fundamentalfolge), falls $\forall\epsilon > 0 \,\exists n_0\in\mathbb{N}: d(x_n, x_m) < \epsilon\;\forall n,m\ge n_0$
\end{definition}
\begin{satz}
	Sei $\{x_n\}$ Folge im metrischen Raum $(X,d)$. Dann
	\begin{enumerate}[label={\arabic*)}]
		\item $x_n\rightarrow x \Rightarrow \{x_n\}$ ist \person{Cauchy}-Folge
		\item $\{x_n\}$ \gls{cf} $\Rightarrow \{x_n\}$ ist beschränkt und hat maximal 1 \gls{hw}.
	\end{enumerate}
\end{satz}
\begin{definition}[Durchmesser]
	\begriff{Durchmesser} von $M\subset X$ beschränkt, $\neq 0$, $(X,d)$ metrischer Raum ist \mathsymbol{diam}{$\diam$}$M:=\sup\{d(x,y) | x,y\in M\}$
	
	Folge $\{A_n\}$ von abgeschlossenen Mengen heißt \begriff{Schachtelung} falls $A_n\neq\emptyset, A_{n+1}\subset A_n\;\forall n\in\mathbb{N}$ und $\diam A_n\overset{n\rightarrow\infty}{\longrightarrow}0$.
\end{definition}
\begin{lemma}
	Sei $M\subset X$ beschränkt, $\neq 0\;\Rightarrow\;\diam M = \diam (\cl M)$.
\end{lemma}
\begin{theorem}
	Sei $(X,d)$ metrischer Raum. Dann: für jede Schachtelung $A_n$ in $X$ gilt:\[ \bigcap_{n\in\mathbb{N}}\in\mathbb{N} A_n\neq \emptyset \;\Leftrightarrow \; \text{jede \gls{cf} in $\{x_n\}$ in $X$ ist konvergent} \]
\end{theorem}
\begin{lemma}
	In $\mathbb{R}$ gilt:
	\begin{center}
		\begin{tabular}{lcl}
			$\bigcap_{n\in\mathbb{N}} A_n\neq \emptyset$ & $\Leftrightarrow$ & $\bigcap_{n\in\mathbb{N}} X_n\neq \emptyset$ \\[5pt]
			$\forall$ Schachtelungen $\{A_n\}$ && $\forall$ Intervallschachtelungen $\{x_n\}$
		\end{tabular}
	\end{center}
\end{lemma}
\begin{definition}[Vollständigkeit]
	Metrischer Raum $(X,d)$ heißt \begriff{Vollständig}, falls jede \person{Cauchy}-Folge $\{x_n\}$ in $X$ konvergiert.
	
	Vollständiger, normierter Raum $(X,\parallel .\parallel)$ heißt \begriff{\person{Banach}-Raum}.
\end{definition}
\begin{conclusion}
	Sei $\{x_n\}$ Folge im vollständigen metrischen Raum $(X,d)$. Dann:\[ \{x_n\}\text{ konvergent}\;\Leftrightarrow\; \{x_n\} \text{ \person{Cauchy}-Folge} \]
\end{conclusion}
\begin{theorem}
	$\mathbb{R}^n$ und $\mathbb{C}^n$ mit $|.|_p$ ($1\le p \le \infty$) sind vollständige, normierte Räume (d.h. \person{Banach}-Räume).
\end{theorem}

\section{Kompaktheit}
\begin{definition}
Sei $(X,d)$ metrischer Raum, Mengensystem $\mathcal{U}\subset \{ U\subset X | U \text{ offen }\}$ heißt \begriff{offene Überdeckung} von $M\subset X$, falls $M\subset \bigcup_{U\in\mathcal{U}} U$.

Überdeckung $\mathcal{U}$ heißt endlich, falls $\mathcal{U}$ endlich (d.h. $\mathcal{U} = \{U_1,\dotsc,U_n\}$).

Menge $M\subset X$ heißt \highlight{(überdeckungs-)}\begriff{kompakt}, falls jede Überdeckung $\mathcal{U}$ eine endliche Überdeckung $\tilde{\mathcal{U}}\subset \mathcal{U}$ endhält (d.h. $\exists U_1,\dotsc, U_n\subset\mathcal{U}$ mit $M\subset\bigcup_{i=1}^n U_n$).

Menge $M\subset X$ heißt \begriff{folgenkompakt}, falls jede Folge $\{x_n\}$ aus $M$ (d.h. $x_n\in M\;\forall M$) eine konvergente Teilfolge $\{x_{n'}\}$ mit Grenzwert in $M$ bessitzt (d.h. $\{x_n\}$ hat \gls{hw} in $M$ nach \ref{tfprinzip}).
\end{definition}

\begin{theorem}
	Sei $(X,d)$ metrischer Raum, $M\subset X$. Dann:\[M\text{ kompakt} \;\Leftrightarrow\; M\text{ folgenkompakt}\]
\end{theorem}

\begin{satz}
	Sei $(X,d)$ metrischer Raum, $M\subset X$. Dann
	\begin{enumerate}[label={\arabic*)}]
		\item $M$ folgenkompakt $\Rightarrow$ $M$ beschränkt und abgeschlossen
		\item $M$ folgenkompakt, $A\subset M$ abgeschlossen $\Rightarrow$ $A$ folgenkompakt.
	\end{enumerate}
\end{satz}
\begin{theorem}[\person{Heine}-\person{Borell} kompakt, \person{Bolzano}-\person{Weierstraß} folgenkompakt]
	Sei $X=\mathbb{R}^n$ (bzw. $\mathbb{C}^n$) mit beliebiger Norm, $M\subset X$. Dann \[ M \text{ kompakt} \;\Leftrightarrow\; M \text{ abgeschlossen und beschränkt} \]
\end{theorem}
\begin{conclusion}
	Sei $\{x_n\}$ Folge in $X=\mathbb{R}^n$ (bzw. $\mathbb{C}^n$). Dann \[ \{x_n\}\text{ beschränkt} \;\Rightarrow \; \{x_n\} \text{ hat konvergente \gls{tf}}\]
\end{conclusion}
\begin{satz}
	Je 2 Normen aus $\mathbb{R}^n$ bzw. $\mathbb{C}^n$ sind äquivalent.
\end{satz}

\section{Reihen}
\begin{definition}[Partialsumme]
	Sei $X$ normierter Raum. $\{x_n\}$ Folge im normierten Raum.\\
	$s_n :=\sum_{k=1}^n x_k = x_0 + \dotsc + x_n$ heißt \begriff{Partialsumme}.
	
	Folge $\{s_n\}$ der Partialsumme heißt \highlight{(unendliche)}\begriff{Reihe} mit Gliedern $x_k$.\\
	Notation: durch Symbol $\sum_{k=0}^\infty x_k = x_0 + \dotsc = \sum_k x_k = \{s_k\}_{k\in\mathbb{N}}$
	
	Existiert der Grenzwert $s = \lim\limits_{n\rightarrow\infty} s_n$, so heißt der \begriff[Reihe!]{Summe} der Reihe.\\
	Notation: $s = \sum_{k=0}^\infty x_n$.
\end{definition}

\begin{satz}[\person{Cauchy}-Kriterium]
	Sei $X$ normierter Raum, $\{x_k\}$ Folge in $X$. Dann
	\begin{enumerate}[label={\arabic*)}]
		\item $\sum_k x_k$ konvergiert $\Rightarrow\;\forall \epsilon > 0\,\exists n_0: \left|\left|\sum_{k=n}^n x_k\right|\right| < \epsilon\;\forall k\ge n\ge n_0$
		\item falls $x$ vollständiger, normierter Raum, gilt $\Leftarrow$ oben.
	\end{enumerate}
\end{satz}
\begin{conclusion}
	Sei $X$ normierter Raum, $\{x_n\}$ Folge in $X$. Dann:\\
	$\sum_k x_k$ konvergiert $\Rightarrow$ $x_k\overset{k\rightarrow \infty}{\longrightarrow}0$
\end{conclusion}
\begin{example}
	\begriff{geometrische Reihe} $X=\mathbb{C}, a_k:= z^k, z\in\mathbb{C}$ fest.
	
	$\sum_{k=0}^\infty z^k = \frac{1}{1-z}\;\forall z\in\mathbb{C}$ mit $|z|<1$
	$\sum_{k=0}^\infty z^k$ divergent, falls $|z|>1$
\end{example}
\begin{example}
	\begriff{harmonische Reihe} $X=\mathbb{R}, x_k := \frac{1}{k}\;(k>1)$. Reihe divergiert.
\end{example}
\stepcounter{theorem}
\begin{example}
	$X=\mathbb{R}$:\[ \sum_{k=1}^\infty \frac{1}{k^s}\;\begin{cases}
	\text{konvergiert},& \text{für }s > 1\\ \text{divergiert},& \text{für }s \le 1
	\end{cases} \]
	Summe heißt \begriff{\person{Riemann}'sche Zetafunktion}\mathsymbol{zeta}{$\zeta(s)$} (für $s > 1$). Diese ist beschränkt und konvergent.
\end{example}
\begin{satz}
	Sei $X$ normierter Raum, $\{x_n\}, \{y_n\}$ in $X, \lambda,\mu\in K$ ($\mathbb{R}$ oder $\mathbb{C}$). Dann:\\
	$\sum_k x_k, \sum_k y_k$ konvergernt $\Rightarrow\;\sum_{k=0}^\infty \lambda x_k + \mu x_k$ konvergent gegen $\lambda\sum_k x_k + \mu \sum_k y_k$.
\end{satz}
\begin{definition}
	Reihe $\sum_k x_k$ heißt \begriff{absolut konvergent}, falls $\sum_k \parallel x_k\parallel$ konvergiert.
\end{definition}
\begin{satz}
	Sei $X$ vollständiger, normierter Raum. Dann:\\
	$\sum_k x_k$ absolut konvergent $\Rightarrow\;\sum_k x_k$ konvergent
\end{satz}
\begin{satz}[Konvergenzkriterien für Reihen]
	Sei $X$ normierter Raum, $\{x_k\}$ in $X, k_0\in\mathbb{N}$
	\begin{enumerate}[label={\alph*)}]
		\item Sei $\{x_k\}$ Folge in $\mathbb{R}$ \hfill\begriff{Majorantenkriterium}
		\begin{enumerate}[label={\alph*)}]
			\item $\parallel x_k\parallel \le \alpha_k\;\forall k\ge k_0,\sum_k \alpha_k$ konvergent $\Rightarrow\;\sum_k \parallel x_k\parallel$ konvergent
			\item $0 \le \alpha_k \le \parallel x_k\parallel\;\forall k\ge k_0,\sum_k \alpha_k$ divergent $\Rightarrow\sum_k\parallel x_k\parallel$ divergent.
		\end{enumerate}
		\item Sei $x_k\neq 0\;\forall k\ge k_0$\hfill\begriff{Quotientenkriterium}
		\begin{enumerate}[label={\alph*)}]
			\item $\frac{\parallel x_{k+1}\parallel}{\parallel x_k\parallel} \le q < 1\;\forall k\ge k_0 \;\Rightarrow\;\sum_k \parallel x_k\parallel$ konvergiert
			\item $\frac{\parallel x_{k+1}\parallel}{\parallel x_k\parallel}\;\forall k\ge k_0\;\Rightarrow \sum_k\parallel x_k\parallel$ divergiert.
		\end{enumerate}
		\item \hfill\begriff{Wurzelkriterium}
		\begin{enumerate}[label={\alph*)}]
			\item $\sqrt[k]{\parallel x_k\parallel}\le q < 1\;\forall k\ge k_0\;\Rightarrow\;\sum_k\parallel x_k\parallel$ konvergiert
			\item $\sqrt[k]{\parallel x_k\parallel} \ge 1\;\forall k\ge k_0\;\Rightarrow\;\sum_k \parallel x_k\parallel$ divergent.
		\end{enumerate}
	\end{enumerate}
\end{satz}
\begin{example}
	\begriff{Exponentialreihe} $\exp z := \sum_{k=0}^\infty \frac{z^k}{n!}$ absolut konvergent $\forall z\in \mathbb{C}$.
	
	\mathsymbol{e}{$e$}$:=\exp 1$ \begriff{\person{Euler}'sche Zahl}
\end{example}
\begin{example}
	\begriff{Potenzreihe}: $\sum_{k=0}^\infty a_k(z-z_0)^k$ für $z\in\mathbb{C}, a_k\in\mathbb{C}, z_0\in\mathbb{C}$.
	
	Sei \[L:=\begin{cases} \limsup\limits_{n\rightarrow\infty} \sqrt[k]{|a_k|},&\text{falls existiert}\\ \infty,&\text{sonst}\end{cases}\qquad R:=\frac{1}{L} \;(\text{mit }0 = \frac{1}{\infty}, \frac{1}{0} = \infty)\]
	
	$ |z - z_0| < R$: absolute Konvergenz,\\
	$|z-z_0| > R$: Divergenz,\\
	$|z-z_0| = R$: i.A. keine Aussage möglich.
	
	$B_R(z_0)$ heißt \begriff{Konvergenzkreis}, $R$ \begriff{Konvergenzradius}
\end{example}
\begin{example}
	\begriff{$p$-adische Brüche}. Sei $p\in\mathbb{N}_{\ge 2}$: betrachte $0,x_1x_2x_3\dotsc :=\sum_{k=1}^\infty x_k\cdot p^{-k}$ für $x_k\in\{0,1,\dotsc,p-1\}\;\forall k\in\mathbb{N}$.
\end{example}
\begin{satz}[\person{Leibnitz}-Kriterium für alternierende Reihen in $\mathbb{R}$]
	Sei $\{x_n\}$ monoton fallende Nullfolge in $\mathbb{R}$. Dann:\\
	alternierende Reihe $\sum_{k=0}^\infty (-1)^k x_k = x_0 - x_1 + x_2 - \dotsc$ ist konvergent.
\end{satz}
\stepcounter{theorem}
\begin{definition}[Umordnung]
	Sei $\beta:\mathbb{N}\rightarrow\mathbb{N}$ bijektive Abbildung: $\sum_{k=0}^\infty x_{\beta(k)}$ heißt \begriff{Umordnung} der Reihe $\sum_k x_k$.
\end{definition}
\begin{satz}
	Sei $X$ normierter Raum. Dann:\\
	$\sum_{k=0}^\infty x_k = x$ absolut konvergent $\Rightarrow\;\sum_{k=0}\infty x_{\beta(k)}$ absolut konvergent für jede Umordnung.
\end{satz}
\begin{satz}
	Sei $\sum_{k=0}^\infty x_k$ konvergierende Reihe in $\mathbb{R}$, die nicht absolut konvergent ist. Dann:\\
	$\forall s\in\mathbb{R}\cup \{\pm\infty\}$ existiert $\beta:\mathbb{N}\rightarrow\mathbb{N}$ bijektiv mit $s=\sum_{k=0}^\infty x_{\beta_k}$
\end{satz}
\begin{satz}[\person{Cauchy}-Produkt]
	Sei $X$ normierter Raum über $\mathbb{K}$, $\sum_j x_j$ und $\sum_i \lambda_i$ absolut konvergent in $X$ bzw. $\mathbb{K}$. $\beta:\mathbb{N}\times \mathbb{N}\rightarrow \mathbb{N}$ bijektiv, $Y_{\beta(i,j)} = \lambda_i x_i\;\forall i,j\in\mathbb{N}$
	
	$\Rightarrow \sum_{l=0}^\infty Y_l = \sum_{i=0}^\infty \lambda_i \sum_{j=0}^\infty x_j$, wobei linke Reihe absolut konvergiert in $X$.
	
	\begin{tabular}{ll}
		\highlight{Spezialfall:} & $\beta(i,j) = \frac{(i+j)(i+j+1)}{2} + i$ liefert\\[5pt]
		& $\sum_{k=0}^\infty \sum_{l=0}^k \lambda_k x_{k-l} = \sum_{i=0}^\infty \lambda_i \sum_{j=0}^\infty x_j$
	\end{tabular}
\end{satz}
\stepcounter{theorem}
\begin{satz}[Doppelreihensatz]
	Sei $\{x_{k,l}\}_{k,l\in\mathbb{N}}$ Doppelfolge im \person{Banach}-Raum $X$ und mögen $\sum_{l=0}^\infty \parallel x_{k,l}\parallel =:\alpha_k\;\forall k$ und $\sum_{k=0}^\infty x_k =: \alpha$ existieren.
	
	$\Rightarrow \sum_{k=0}^\infty \left(\sum_{l=0}^\infty x_{k,l}\right) = \sum_{l=0}^{\infty}\left( \sum_{k=0}^\infty x_{k,l}\right)$, wobei alle Reihen absolut konvergent sind.
\end{satz}

\chapter{Funktionen und Stetigkeit}\addtocounter{section}{12}
\section{Funktionen}
\begin{definition}
	$f:\mathbb{R}\to \mathbb{R}$ \begriff{monoton}\begriff[monoton!]{falled}/\begriff[monoton!]{wachsend}, falls $x < y, x,y\in M \,\Rightarrow \,f(x) \le f(y)$ bzw. $f(x) \ge f(y)$
	
	Falls rechts stets $<$ bzw. $>$, sagt man auch \begriff[monoton!]{streng} monoton.
\end{definition}

\begin{satz}
	Sei $f:\mathbb{R}\rightarrow \mathbb{R}$ streng monoton fallend / wachsend.\\
	$\Rightarrow$ inverse Funktion $f^{-1}:\mathcal{R}\rightarrow M$ existiert und ist streng monoton wachsend / fallend.
\end{satz}
\begin{example}
	\begriff{Allgemeine Potenzfunktion} in $\mathbb{R}$:\\
	$f:\mathbb{R}_{>0} \to \mathbb{R}$ mit $f(x) = x^r$ für $r\in\mathbb{R}$ fest.
	
	\begin{itemize}
		\item $r > 0:$ Satz \ref{satz_potenz_r} $\Rightarrow$ $f$ streng monoton wachsend
		\item $r < 0$: $x^r = \frac{1}{x^{-r}}$ $\Rightarrow$ $f$ streng monoton fallend
	\end{itemize}
	$\overset{\text{Satz 1}}{\Rightarrow}$ $f^{-1}$ existiert für $r\neq 0$ auf $(0,\infty)$, wegen $ y = (r^{\frac{1}{r}})^r$ ist $f^{-1}(y) = y^{\frac{1}{r}}$
\end{example}
\begin{example}
	\begriff{Allgemeine Exponentialfunktion} in $\mathbb{R}$:\\
	$f:\mathbb{R}\rightarrow\mathbb{R}$ mit $f(x) = a^x$ für $a\in\mathbb{R}_{>0}$ fest.
	
	\ref{satz_potenz_r} $\Rightarrow$ streng monoton wachsend für $a > 1$ bzw. fallend für $a < 1$ (benutze $\frac{1}{a} > 1$)\\
	$\overset{\text{Satz 1}}{\Rightarrow}$ $f^{-1}$ existiert auf $(0,\infty)$ für $a \neq 1$. Wegen $y = a^{\log_a y}$ (\ref{satz_logarithmus_r}) ist $f^{-1} (y) = \log_a y$.
\end{example}
\begin{example}
	\begriff{Polynom} in $\mathbb{C}$:\\
	Abbidlung $f:\mathbb{C}\rightarrow\mathbb{C}$ heißt \highlight{Polynom}, falls $f(z) = a_n z^n + \dotsc + a_1 z + a_0$ für $a_0,\dotsc, a_n\in\mathbb{C}$ fest.
	\begin{itemize}
		\item \mathsymbol{grad}{$grad$}$f = n$ falls $a_n\neq 0$
		\item $f$ ist \begriff{Nullpolynom}, falls $f(z) = 0\;\forall z\in\mathbb{C}$
		
		Notation: $f=0$
		
		(Menge der Polynome in $\mathbb{C}$ ist ein Vektorraum über $\mathbb{C}$)
	\end{itemize}
\end{example}
\begin{satz}\label{Polynomdiv}
	Seien $f,g$ Polynome mit $f(z) = \sum_{k=0}^n a_k z^k, g(z) = \sum_{k=0}^m a_k z^k$. Dann:
	\begin{enumerate}[label={\arabic*)}]
		\item $f,g\neq 0$, $\grad f\ge \grad g$\\
		$\Rightarrow$ existieren eindeutig bestimmte Polynome $q,r$ mit $f = q\cdot g + r$, wobei $r\neq 0$ oder $\grad r < \grad g$
		\item $z_0\in\mathbb{C}$ Nullstelle von $f\neq 0$ $\Leftrightarrow$ $f(z) = (z - z_0)q(z)$ für ein Plynom $q\neq 0$ mit $\grad q = \grad f -1$
		\item $f$ hat höchstens $\grad f$ Nullstellen falls $f\neq 0$
		\item $f(z_i) = g(z_j)$ für $n+1$ paarweise verschiedene Punkte $z_0, \dotsc, z_n\in\mathbb{C}, n = \grad f \ge \grad g$\\
		$\Rightarrow$ $f(z) = g(z) \;\forall z\in\mathbb{C}$ (d.hz. $a_k = b_k\;\forall k$)
	\end{enumerate}
\end{satz}
\stepcounter{theorem}
\begin{definition}
	Abbildung $f:X\rightarrow Y, Y$ metrischer Raum heißt \begriff{beschränkt}[!Funktion] auf $M\subset X$ , falls Menge $f(M)$ beschränkt in $Y$ ist, sonst unbeschränkt.
\end{definition}
\begin{definition}
	$f:X\to Y$ heißt \begriff{konstante Funktion}, falls $f(x) = a\;\forall x\in X$ und $a\in Y$ fest.
\end{definition}
\begin{definition}
	$M\subset X, X$ normierter Raum heißt \begriff{konvex}, falls $x,y\in M \,\Rightarrow \,tx+(1-t)y \in M\;\forall t\in(0,1)$
	
	$f:D\subset X\to \mathbb{R}$ heißt \begriff[konvex!]{strikt}\begriff{konvex}, falls $f(tx + (1-t)y) \underset{(<)}{\le} f(x) + (1-t)f(y)\;\forall x,y\in D, t\in(0,1)$
	
	$f$ heißt \begriff{konkav} (bzw. \begriff[konkav!]{strikt}), falls $-f$ (strikt) konvex.
\end{definition}
\stepcounter{theorem}

\subsection*{Lineare Funktionen}
\begin{definition}
	Seien $X,Y$ normierte Räume über $K$.\\
	$f: X\rightarrow Y$ heißt \begriff{linear}, falls
	\begin{itemize}
		\item $f$ \begriff{additiv}, d.h. $f(a+b) = f(a) + f(b) \;\forall a,b\in X$ und
		\item $f$ \begriff{homogon}, d.h. $f(\lambda a) = \lambda f(a)\;\forall a\in X,\lambda\in K$
	\end{itemize}

	$f:X\to Y$ heißt \begriff{affin linear}, falls $f+f_0$ linear für eine konstante Funktion $f_0$
	
	Offenbar $f$ linear $\Rightarrow\;f(0) = 0$
\end{definition}
\stepcounter{theorem}
\begin{definition}
	Lineare Abbildung $f:X\to Y$ heißt \begriff{beschränkt}[!lineare Funktion], falls $f$ beschränkt auf $\overline{B_1(0)}$, d.h. \begin{align}
		\tag{1}\exists\text{ konstante }c > 0: \parallel f(x)\parallel \le c\;\forall x: \parallel x\parallel \le 1
	\end{align}
	Wegen $\parallel f\left( \frac{x}{\parallel x \parallel}\right) = \frac{1}{\parallel x \parallel} \parallel f(x) \parallel$ ist (1) äquivalent zu
	\begin{align}
		\tag{1'} \parallel f(x) \parallel = \sup \{ \parallel f(x) \parallel | x \in \overline{B_1(0)}\}
	\end{align}
\end{definition}
\begin{satz}
	Seien $X,Y$ normierte Räume über $K$, dann:\\
	\mathsymbol{L}{$L$}$(X,Y):= \{ f:X\to Y \,|\, f \text{ linear und beschränkt} \}$ ist normierter Raum über $K$ mit $\parallel f \parallel = \sup \{ \parallel f(x) \parallel | x\in \overline{B_1(0)} \}$
\end{satz}

\subsection*{Exponentialfunktion}
\begin{definition}
	$\exp:\mathbb{C}\to \mathbb{C}$ mit $\exp(z) = \sum_{k=0}^\infty \frac{z^k}{k!}$
\end{definition}

\begin{satz}
	Sei $\{z_n\}$ Folge in $\mathbb{C}$ mit $z_n\to z$. Dann: $\lim\limits_{n\rightarrow\infty} \left( 1 + \frac{z_n}{n}\right)^n = \exp (z)$
\end{satz}

\begin{lemma}
	Sei $z_n\to 0$ in $\mathbb{C}\;\Rightarrow\; \lim \frac{\exp(z_n) - 1}{z^n} = 1$
\end{lemma}

\begin{satz}
	Sei $f:\mathbb{C}\rightarrow\mathbb{C}$ mit $f(z_1 + z_2) = f(z_1) \cdot f(z_2) \;\forall z_1, z_2\in\mathbb{C}$ und $\lim\limits_{n\rightarrow\infty} \dfrac{f\left( \frac{z}{n}\right) - 1}{\frac{z}{n}} = \gamma\in\mathbb{C}\;\forall z\in\mathbb{C}$ \\
	$\Rightarrow \;f(z) = \exp(\gamma z)\;\forall z\in\mathbb{C}$
\end{satz}

\begin{conclusion}
	Funktion $\exp$ ist durch obiges Lemma und Satz eindeutig definiert.
\end{conclusion}
\begin{satz}
	Es gilt: $e^x = \exp (x) \;\forall x\in \mathbb{R}$
	
	Definiert (!) in $\mathbb{C}:\; e^z := \exp(z) \;\forall z\in\mathbb{C}$ (als Potenz nicht erklärt)
\end{satz}

\begin{definition}
	\begriff{natürlicher Logarithmus}: $\ln x = \log_e x\;\forall x\in\mathbb{R}_{>0}$
	
	\begriff{Trigonometrische Funktion}:
	\begin{itemize}
		\item $\sin z := \frac{e^{iz} - e^{-iz}}{2i} = \sum_{k=0}^\infty (-1)^k \frac{z^{2k+1}}{(2k+1)!} = z - \frac{z^3}{3!} + \frac{z^5}{5!}+ \dotsc \;\forall z\in\mathbb{C}$
		\item $\cos z := \frac{e^{iz}+e^{-iz}}{z} = \sum_{k=0}^\infty (-1)^k \frac{z^{2k}}{(2k)!} = 1 - \frac{z^2}{4} + \frac{z^4}{24}+\dotsc \;\forall z\in\mathbb{C}$
	\end{itemize}
\end{definition}

\begin{satz}
	Es gilt:
	\begin{enumerate}[label={\arabic*)}]
		\item \begriff{\person{Euler}'sche Formel}: $e^{iz} ? \cos z + i \sin z$
		\item $\sin^2 z + \cos^2 z = 1\;\forall z\in\mathbb{C}$ (beachte: $\cancel{\rightarrow}\;|\sin z|\le1, |\cos z| \le 1$, $\sin, \cos$ unbeschränkt auf $\mathbb{C}$)
		\item $\sin(-z) = -\sin z, \cos z = \cos(-z)$
		\item (\begriff{Additionstheoreme})
		\begin{itemize}
			\item $\sin(z+w) = \sin z \cos w - \sin w \cos z \;\forall z,w\in\mathbb{C}$
			\item $\cos (z+w) = \cos z \cos w - \sin z \sin w \;\forall z,w\in\mathbb{C}$
		\end{itemize}
		\item $\sin(2z) = 2\sin z \cos z, \cos(2z) = \cos^2 z - \sin^2 z\;\forall z\in\mathbb{C}$
		\item $\sin z - \sin w = 2\cos \frac{z+w}{2} - \sin \frac{z+w}{2}$\\
			  $\cos z - \cos w = -2\sin\frac{z+2}{2}\sin\frac{z-w}{2}$
	\end{enumerate}
\end{satz}

\begin{satz}
	Es gilt $\forall x\in \mathbb{R}:$\\
	$\,\left| e^{ix}\right| = 1, \sin x = \Im e^{ix}, \cos = \Re e^{ix}$ (insbesondere $\sin x,\cos x \in\mathbb{R}$), somit $e^{ix} = \cos x + i \sin x$
\end{satz}

\begin{lemma}
	Es gilt in $\mathbb{R}$:
	\begin{enumerate}[label={\arabic*)}]
		\item $\cos$ streng fallend auf $[0,2]$
		\item $\cos 2 < 0$ und $\sin x > 0\;\forall x\in (0,2]$
		\item $\phi(x) 0 \phi(1) \;\forall x\in [0,2]$ und $45 < \phi(x) < 90$ (d.h. $\phi(x)$ proportional zu $x$)
		\item $\cos \frac{\pi}{2} = 0$ für $\pi := \frac{180°}{\phi(1)}$ ($=3,1415\dotsc$), $\frac{\pi}{2}$ einzige Nulsltelle in $[0,2]$
	\end{enumerate}
\end{lemma}
\stepcounter{theorem}
\begin{satz}
	Für alle $z\in\mathbb{C}, k\in\mathbb{Z}$ gilt:
	\begin{enumerate}[label={\arabic*)}]
		\item $^{z+2k\pi i} = e^z$, d.h. Periode $2\pi i$\\
		$\sin(z+2k\pi) = \sin z$ (d.h. Periode $2\pi$)\\
		$\cos(z+2k\pi) = \cos z$ (d.h. Periode $2\pi$)
		\item $e^{z+i\sfrac{\pi}{2}} = ie^z, e^{z+i\pi} = -e^z$
		\item $\sin(z+\pi) = -\sin z, \cos(z+\pi) = -\cos z$\\
		$\sin\left(z+\frac{\pi}{2}\right) = \cos z, \cos\left(z+\frac{\pi}{2}\right) = -\sin z$
	\end{enumerate}
\end{satz}

\begin{satz}
	Auf $\mathbb{C}$ gilt:
	\begin{itemize}
		\item $eez = 1 \,\Leftrightarrow\,z=2k\pi i,\;k\in\mathbb{Z}$
		\item $\sin z = 0\,\Leftrightarrow\,z=k\pi,\;k\in\mathbb{Z}$
		\item $\cos z = 0\,\Leftrightarrow\,z =k\pi + \frac{\pi}{2},\;k\in\mathbb{Z}$
	\end{itemize}
\end{satz}
\subsection*{$\sin$ / $\cos$ in $\mathbb{R}$}
\begin{centering}
	\begin{tabular}{c|ccccc}
		\toprule
		$x$ & 0 & $\frac{\pi}{6}$ & $\frac{\pi}{4}$ & $\frac{\pi}{3}$ & $\frac{\pi}{2}$ \\
		\midrule
		$\sin x$ & $0$ & $\frac{1}{2}$ & $\frac{\sqrt{2}}{2}$ & $\frac{\sqrt{3}}{2}$ & $1$ \\
		$\cos x$ & $1$ & $\frac{\sqrt{3}}{2}$ & $\frac{\sqrt{2}}{2}$ & $\frac{1}{2}$ & $0$ \\
		\bottomrule
	\end{tabular}
\end{centering}

\begin{definition}
	$\sin\left[ -\frac{\pi}{2},\frac{\pi}{2}\right]\to [-1,1]$ streng monoton und surjektiv,\\
	$\cos[0,\pi]\to[-1,1]$ streng monoton und surjektiv\\
	$\Rightarrow$ Umkehrfunktion existiert: \begriff{Arcussinus}, \begriff{Arcuscosinus}:
	\begin{itemize}
		\item $\arcsin := \sin^{-1}: [-1,1]\to\left[-\frac{\pi}{2},\frac{\pi}{2}\right]$
		\item $\arccos := \cos^{-1}: [-1,1]\to [0,\pi]$
	\end{itemize}
\end{definition}

\subsection*{Tangens und Cotangents}
\begin{definition}
	$\tan z z := \frac{\sin z}{\cos z}\;\forall z\in\mathbb{C}\setminus\{ \left.\frac{\pi}{2} + k\pi \right| k\in\mathbb{Z}\}$\\
	$\cot z := \frac{\cos z}{\sin z}\;\forall z\in\mathbb{C}\setminus \{ k\pi | k\in\mathbb{Z}\}$
	
	$\left.\begin{aligned}
		\text{Offenbar }\tan (z+\pi) &= \frac{\sin (z+\pi)}{\cos(z+\pi)} = \frac{-\sin z}{-\cos z} = \tan z\\
		\cot(z+\pi) &= \cot (z)
	\end{aligned}\right\rbrace
	\begin{gathered}
		\forall z\in\mathbb{C}, \text{ d.h. Periode $\pi$}
	\end{gathered}
	$
\end{definition}

\subsection*{Tangens auf $\mathbb{R}$}
\begin{definition}
	$0 \le x_1 < x_2 < \sfrac{\pi}{2} \,\Rightarrow\,\tan x_1 = \frac{\sin x_1}{\cos x_1} < \frac{\sin x_2}{\cos x_2} = \tan x_2$ \\
	$\Rightarrow\,\tan (-x) = - \tan(x) $ $\Rightarrow$ streng wachsend auf $\left( \frac{\pi}{2},\frac{\pi}{2}\right)$ \\
	$\Rightarrow\,\arctan = \tan^{-1}: \mathbb{R}\to \left(-\frac{\pi}{2},\frac{\pi}{2}\right)$ existiert.
\end{definition}
\begin{satz}
	Es gilt:
	\begin{enumerate}[label={\arabic*)}]
		\item $\Re(exp) = \mathbb{C}\setminus\{0\}$
		\item (\begriff{Polarkoordinaten} auf $\mathbb{C}$)
		
		Für $z\in\mathbb{C}\setminus\{0\}$ existiert eindeutiges $\gamma\in[0,2\pi] mit z = |z|e^{i\gamma} = |z|\left( \cos \gamma + i\sin \gamma\right)$ (auch $[-\pi,\pi]$)
		\item (Wurzeln)
		
		Für $Z=|z|e^{i\gamma}\in\mathbb{C}\setminus\{0\}, n\ge 2$ gilt:\\
		$w^n = z \,\Leftrightarrow\, w\in\left\{ \left. \sqrt[n]{z} e^{i \frac{k}{n} + \frac{2k\pi}{n}} =: w_k \right| k=1,\dotsc,n\right\}$ (Lösungen bilden ein regelmäßiges $N$-Eck auf dem Kreis mit dem Radius $\sqrt[n]{|z|}$)
	\end{enumerate}
\end{satz}

\subsection*{Logarithmen in $\mathbb{C}$} (sog. Hauptzweig)
\begin{definition}
	$exp\left( \{ z\in\mathbb{C}\,|\, \Im z < \pi \}\right) \to \mathbb{C}\setminus (\infty, 0]$ ist bijektiv \\
	$\Rightarrow$ Umkehrabbildung $\ln:\mathbb{C}\setminus(-\infty,0]$ gilt: $e^{\ln |z| + i\gamma} = |z|e^{i\gamma} = z$\\
	$\Rightarrow\,\ln z = \ln |z| + i\gamma \;\forall z=|z|e^{i\gamma}\in\mathbb{C}\setminus(-\infty,0)$\\
	$\Rightarrow \,\ln z$ stimmt auf $\mathbb{R}_{>0}$ mit rellen $\ln$ überein.
\end{definition}

\subsection*{Hyperbolische Funktionen}
\begin{definition}
	\begin{itemize}
		\item $\sinh (z) = \frac{e^z - e^{iz}}{2} = \sum_{k=0}^\infty \frac{z^{2k+1}}{(2k+1)!}\;\forall z\in\mathbb{C}$ (\begriff{Sinus Hyperbolicus})
		\item $\cosh (z) = \frac{e^z+e^{-z}}{2} = \sum_{k=0}^\infty \frac{z^{2k}}{(2k+1)!}\;\forall z\in\mathbb{C}$ (\begriff{Cosinus Hyperbolicus})
		\item $\tanh (z) = \frac{\sinh (z)}{\cosh (z)}\;\forall z\in\mathbb{C}\setminus\left\lbrace \left.\frac{\pi}{2} + k\pi  \right| k\in\mathbb{Z} \right\rbrace$ (\begriff{Tangens Hyperbolicus})
		\item $\coth(z) = \frac{\cosh(z)}{\sinh(z)} \;\forall z\in\mathbb{C}\setminus \{ k\pi | k\in\mathbb{Z}\}$ (\begriff{Cotangens Hyperbolicus})
	\end{itemize}
\end{definition}

\begin{satz}
	Es gilt $\forall z,w\in\mathbb{C}$
	\begin{enumerate}[label={\arabic*)}]
		\item $\sin h = -i\sin(z), \cos (z) = \cosh(iz), \sinh(-z) = -\sinh(z), \cosh(-z) = \cosh(x)$ (gibt auch Nullstellen vom $\sinh / \cosh$)
		\item $\sinh, \cosh$ haben Periode $2\pi i$, $\tanh, \coth$ haben Periode $\pi i$
		\item $\cosh^2 z - \sin^2 z = 1$
		\item $\sinh(z+w) = \sinh z \cosh w + \sinh w \cosh z$\\
		$\cosh (z+w) = \cosh z \cosh w + \sinh z \sin w$
	\end{enumerate}
\end{satz}
\rule{4cm}{0.4pt}
\begin{definition}
	Sei $f_n X\to Y$, $Y$ metrischer Raum ($X$ beliebige Menge), $n\in\mathbb{N}$. $\{f_n\}_{n\in\mathbb{N}}$ heißt \begriff{Funktionenfolge}.
	
	Funktionenfolge $\{f_n\}$ konvergiert \begriff[Konvergenz!]{punktweise} gegen $f:X\to Y$ auf $M\subset X$, falls $f_n(x) \overset{n\rightarrow\infty}{\longrightarrow} f(x) \;\forall x\in M$
	
	Funktionenfolge $\{f_n\}$ konvergiert \begriff[Konvergenz!]{gleichmäßig} gegen $f:X\to Y$ auf $M\subset X$, falls \[ \forall \epsilon > 0 \,\exists n_0\in\mathbb{N}: d(f_n(x), f(x)) < \epsilon\quad \forall n\ge n_0\,\forall x\in M \]
	Notation: \mathsymbol*{->}{$\rightrightarrows$} $f_n(x) \overset{n\rightarrow\infty}{\rightrightarrows} f(x)$ bzw. $f_n\overset{n\rightarrow\infty}{\longrightarrow}f$ gleichmäßig auf $M$.
\end{definition}

\begin{lemma}
	$f_n\to f$ gleichmäßig auf $M$ $\Rightarrow$ $f_n(x)\to f(x)\;\forall x\in M$ (d.h. punktweise auf $M$)
\end{lemma}

\begin{satz}
	Seien $f_n, f\in B(X,Y)$. Dann ($X$ metrischer Raum):
	\begin{center}
		$f_n \to f$ gleichmäßig auf $X$ $\Leftrightarrow$ $f_n \to f$ in $(B(X,Y),\parallel.\parallel_1\infty)$
	\end{center}
\end{satz}

\begin{definition}
	Sei $f_n.:X\to Y$, $Y$ normierter Raum ($X$ beliebige Menge), $n\in\mathbb{N}$: $\sum_{n=0}^\infty f_n$ heißt \begriff{Funktionenreihe}
	
	Reihe $\sum_n f_n$ heißt \begriff[Konvergenz!]{punktweise}[!Funktionenreihe] (\begriff[Konvergenz!]{gleichmäßig}[!Funktionenreihe]) konvergent gegen $f:X\to Y$ auf $M\subset X$, falls dies für die zugehörige Folge (Partialsumme!) $\{s_n\}$ gilt.
\end{definition}

\begin{satz}
	Sei $\sum_{k=0}^\infty a_k(z-z_0)^k$ Potenzreihe in $\mathbb{C}$ mit Konvergenzradius $R\in(0,\infty]$ und sei $M\subset B_R(z_0)$ kompakt\\
	$\Rightarrow$ Potenzreihe konvergiert gleichmäßig auf $M$.
\end{satz}

\section{Stetigkeit}
\begin{definition}
	Sei stets $f:D\subset X\rightarrow Y$, $X,Y$ metrischer Raum, $D=\mathcal{D}(f)\neq \emptyset, y_0\in Y$ heißt \begriff{Grenzwert}[!Funktion] der Funktion $f$ im Punkt $x_0\in \overline{D}$, falls gilt:
	\begin{center}
		$\{x_n\}$ Folge in $D$ mit $x_n\to x_0$ $\Rightarrow$ $f(x_n)\to y_0$
	\end{center}
	Notaton: $\lim\limits_{x\rightarrow x_0} = y_0, f(x)\overset{x\to x_0}{\longrightarrow } y_0$
\end{definition}
\stepcounter{theorem}
\begin{remark}
	Falls $x_0\in D$ \begriff{isolierter Punkt} von $D$, d.h. kein \gls{hp} von $D$, dann ist stets $\lim\limits_{x\rightarrow x_0} f(x) = f(x_0)$.
\end{remark}

\begin{satz}[$\epsilon\delta$-Kriterium]
	Sei $f:D\subset X\to Y, x_0\in\overline{D}$. Dann
	\begin{center}
	$\lim\limits_{x\rightarrow x_0} f(x) = y_0 \,\Leftrightarrow \, \forall\epsilon > 0\,\exists \delta > 0: f(B_\delta(x_0)\cap D)\subset B_\epsilon(x_0)$
	\end{center}
\end{satz}

\begin{satz}[Rechenregeln] \label{satz:rechenregel_stetigkeit}
	\begin{enumerate}[label={\arabic*)}]
		\item Sei $Y$ normierter Raum über $\mathbb{R}, f,g:D\subset X\to Y,\lambda: D\to K, x_0\in\overline{D}, f(x)\overset{x\to x_0}{\longrightarrow} y, g(x) \overset{x\to x_0}{\longrightarrow} \tilde{y}, \lambda(x)\overset{x\to x_0}{\longrightarrow} \alpha$. Dann:
		\begin{itemize}
			\item $(f+g)(x) \overset{x\to x_0}{\longrightarrow} y+\tilde{y}$
			\item $(\lambda \cdot f)(x) \overset{x\to x_0}{\longrightarrow} \alpha\cdot y$
			\item $\left(\frac{1}{\lambda}\right)(x) \overset{x\to x_0}{\longrightarrow} \frac{1}{\alpha}$ falls $\alpha\neq 0$
		\end{itemize}
		\item Sei $f: D\subset X\to Y, g:\tilde{D}\subset Y\to Z, \Re(f)\subset\tilde{D}, X,Y,Z$ metrische Räume, $x\in\overline{D}, f(x)\overset{x\to x_0}{\longrightarrow}y, g(y)\overset{y\to y_0}{\longrightarrow} z_0$. Dann:\\
		$g(f(x)) \overset{x\to x_0}{\longrightarrow} z_0$
	\end{enumerate}
\end{satz}

\begin{definition}
	Für $f:D\subset X\to Y$ mit $X=\mathbb{R}$ definieren wir einen \begriff{einseitiger Grenzwert} $y_0\in Y$ heißt \begriff[einseitiger Grenzwert!]{linksseitig} bzw. \begriff[einseitiger Grenzwert!]{rechtsseitig} von $f$ im \gls{hp} $x_0$ von $D\cap(-\infty, x_0)$ bzw. $D\cap(x_0,\infty)$, falls gilt: $x_n\in D\cap(-\infty, x_0)$ bzw. $x_n\in D\cap (x_0,\infty)$ mit $x_n\to x_0\,\Rightarrow \,f(x_n)\to y_0$
	
	$\begin{aligned}
		\text{Notation: } \lim\limits_{x\uparrow x_0} f(x) &= y_0 =: f(x_0^-)& f(x)&\overset{x\uparrow x_0}{\longrightarrow} y_0 \\
		\lim\limits_{x\downarrow x_0}f(x) &= y_0 =:f(x_0^+) & f(x) &\overset{x\downarrow x_0}{\longrightarrow} y_0
	\end{aligned}$
\end{definition}

\begin{remark}
	Satz \ref{satz:rechenregel_stetigkeit} gilt sinngemäß auch für einseitige Grenzwerte.
	
	Für $f:D\subset X\to Y$ mit $X=\mathbb{R}$ bzw. $Y=\mathbb{R}$ heißt der Grenzwert \begriff[Grenzwert!]{uneigentlich}\begriff*[Konvergenz!]{uneigentlich}[!Funktion]: \[\lim\limits_{x\to \pm \infty} f(x) = y_0, \lim\limits_{x\rightarrow x_0} f(x) = \pm \infty, \lim\limits_{x\to \pm \infty} f(x) = \pm \infty,\] indem wir einen Grenzwert definiert als $x_0=\pm \infty$ bzw. $y_0=\pm\infty$ wählen und bestimmte divergenzte Folgen $x_n\to \pm \infty$ mit $x_n\in D$) bzw. $f(x_n)\to \pm \infty$ betrachten.
\end{remark}
\stepcounter{theorem}
\subsection*{Landau-Symbole} (Vgl. von "`Konvergenzgeschwindigkeiten"')
\begin{definition}
	Sei $f:D\subset X\to Y, X$ metrischer Raum, $Y$ normierter Raum, $g:D\subset X\to \mathbb{R}$, $x_0\in \overline{D}$.
	\begin{itemize}
		\item $f(x)$ ist "`\begriff{klein o}"' von $g(x)$ für $x\to x_0$, falls \[ \lim\limits_{\stackrel{x\to x_0}{x\neq x_0}} \frac{\parallel f(x)\parallel}{g(x)} = 0 \]
		Notation: $f(x) = o(g(x))$\mathsymbol*{o}{$o$} (meist $x\neq x_0$ im "`$\lim$"' weggelassen)
		\item $f(x)$ ist "`\begriff{groß O}"' von $g(x)$ für $x\to x_0$, falls \[ \exists \delta > 0, c \ge 0: \frac{\parallel f(x)\parallel}{|g(x)|} \le c \quad \forall x\in (B_\delta(x_0) \setminus \{x_0\})\cap D \]
		Notation: $f(x) = \mathcal{O}(g(x))$\mathsymbol*{O}{$\mathcal{O}$} für $x\to x_0$
	\end{itemize}
\end{definition}
\stepcounter{theorem}
\stepcounter{theorem}
\subsection*{Relativtopologie}
\begin{definition}
	Sei $(X,d)$ metrischer Raum, für $D\subset X$ ist $(D,d)$ ein metrischer Raum mit der induzierten Metrik.
	\begin{itemize}
		\item $M\subset D$ heißt \begriff[Relativtopologie!]{offen} bzw. \begriff[Relativtopologie!]{abgeschlossen} \highlight{relativ zu $D$}, falls $M$ offen bzw. abgeschlossen im metrischen Raum $(D,d)$.
		\item $M\subset D$ heißt \begriff[Relativtopologie!]{Umgebung} von $x\in D$ relativ zu $D$, falls $M$ Umgebung von $x$ im metrischen Raum $(D,d)$.
	\end{itemize}
\end{definition}
\stepcounter{theorem}
\rule{4cm}{0.4pt}
\begin{definition}
	Sei $f:D\subset X\to Y$ metrischer Raum, $D=\mathcal{D}(f)$, Fkt. $f$ heißt \begriff{folgenstetig} im Punkt $x_0\in D$, falls \[ f(x_n)\to f(x_0) \forall \text{ Folgen $x_n\to x_0$ in $D$} \]
\end{definition}
\stepcounter{theorem}
\begin{definition}
	Funktion $f$ heißt \begriff{stetig} im Punkt $x_0\in D$, falls $\forall $ Umgebungen $V$ von $f(x_0)\,\exists $ Umgebung $U$ von $x_0$ in $D:\,f(U)\subset V$.
	
	\begin{tabularx}{\textwidth}{lX}
		\noindent\highlight{Interpretation:} & Input / Output Steuerung besteht Forderung, dass beliebig kleine Output-Toleranzen $\epsilon$ stets durch hinreichend kleine Input-Toleranzen $\delta$ erreicht werden können.
	\end{tabularx}
\end{definition}

\begin{satz}
	Sei $f:D\subset X\to Y$, $X,Y$ metrischer Raum, $x_0\in D$. Dann:
	\begin{center}
		$f$ stetig in $x_0$ $\Leftrightarrow$ $f \,\epsilon\delta$-Stetig in $x_0$ $\Leftrightarrow$ $f$ folgenstetig in $x_0$
	\end{center}
\end{satz}

\begin{definition}
	Funktion $f$ heißt stetig (folgen- / $\epsilon\delta$-stetig) auf $M\subset D$, falls $f$ stetig (folgen-/$\epsilon\delta$-stetig) in jedem Punkt $x_0\in M$.
\end{definition}
\stepcounter{theorem}
\begin{satz}
	Sei $f:D\subset X\to Y, X,Y$ metrische Räume, dann sind folgende Aussagen äquivalent:
	\begin{enumerate}[label={\arabic*)}]
		\item $f$ stetig auf $D$
		\item $f^{-1}(V)$ offen in $D$ $\forall V\subset Y$ offen
		\item $f^{-1}(A)$ abgeschlossen in $D$ $\forall A\subset Y$ abgeschlossen
	\end{enumerate}
\end{satz}
\begin{satz}[Rechenregeln]
	\begin{enumerate}[label={\arabic*)}]
		\item Sei $Y$ normierter Raum über $K$, $f,g:D\subset X\to Y, \lambda: D\to U, f,g, ,y $ stetig in $x_0\in D$\\
		$\Rightarrow$ $f+g, \lambda\cdot f$ stetig in $x_0$, $\frac{1}{\lambda}$ stetig in $x_0$ falls $\lambda(x_0) \neq 0$
		\item Sei $f:D\subset X\to Y, y:\tilde{D}\subset Y\to Z, X, Y, Z$ metrischer Raum, $f$ stetig in $x_0$, $g$ stetig in $f(x_0)\in \tilde{D}$\\
		$\Rightarrow \,g\circ f$ stetig in $x_0$ 
	\end{enumerate}
\end{satz}
\addtocounter{theorem}{3}
\begin{example}[\person{Dirichlet}-Funktion]
	$f:\mathbb{R}\to \mathbb{R}$ mit \[f(x) = \begin{cases}
	 1,&x\in\mathbb{Q}\\ 0,&\text{sonst}
	\end{cases} \] in keinem $x_0\in\mathbb{R}$ stetig.
\end{example}
\begin{satz}
	Sei $f_n, f:D\subset X\to X, f_n$ stetig in $x_0\in D$, $\forall n\in\mathbb{N}, f_n\to f$ gleichmäßig\\
	$\Rightarrow \, f$ stetig in $x_0$
\end{satz}

\begin{conclusion}
	Falls alle $f_n$ stetig auf $M\subset D$ und $f_n\to f$ gleichmäßig auf $M$ \\
	$\Rightarrow\, f$ stetig auf $M$.
\end{conclusion}

\begin{satz}
	Sei $f(z) := \sum_{k=0}^\infty a_k(z-z_0)^k\;\forall z\in B_r(z_0), R\in(0,\infty]$ Konvergrenzkreis, $a_k\in\mathbb{Z}\; \forall k\in \mathbb{N}$\\
	$\Rightarrow\, f:B_r(z_0) \to \mathbb{C}$ stetig auf $B_R(z_0)$
\end{satz}
\addtocounter{theorem}{2}
\begin{definition}
	Bijektive Abbildung $f:D\subset X\to R\subset Y, X,Y$ metrische Räume, $D=\mathcal{D}(f), R=\mathcal{R}(f)$ heißt \begriff{Homöomorphismus}, falls $f$ und $f^{-1}$ stetig.
	
	Mengen $D$ und $R$ heißen \begriff{homöomorph} zueinander, falls es einen Homöomorphismus $f:D\to R$ mit $D=\mathcal{D}(f), R=\mathcal{R}(f)$ gibt.
	
	\highlight{beachte:} Homöomorphismus bildet offene (abgeschlossene) Mengen auf offene (abgeschlossene) Mengen ab.
\end{definition}
\stepcounter{theorem}
\begin{example}
	\begriff{stereographische Projektion}
	
	$X=\mathbb{R}^{n+1}, X_0 := \{(x_0, \dotsc, x_n{n+1}) \in\mathbb{R}^{n+1} \,|\, x_{n+1}=0\}, N = (0,\dotsc, 0,1)$ (Nordpol), $S_n = \{ x\in\mathbb{R}^{n+1} \,|\, |x|=1\}$ $n$-dimensionale Einheitsspäre.
	
	Betrachte $\sigma: \mathbb{R}^{n+1} \setminus\{ N\} \rightarrow \mathbb{R}^{n+1}$ mit $\sigma(x) = N \frac{2}{(x-N)^2}\langle x-N\rangle$ stetig. $\sigma$ ist Homöomorphismus mit $\sigma^{-1}(y) = N - \frac{2}{(y-N)^2}\langle Y-N\rangle$
\end{example}
\rule{4cm}{0.4pt}
\begin{satz}
	Sei $f:D\subset \mathbb{R}\to \mathbb{R}$ streng monoton und stetig, $D$ Intervall \\
	$\Rightarrow f^{-1}$ existiert und ist stetig auf $\mathcal{R}(f)$.
\end{satz}
\stepcounter{theorem}
\begin{satz}
	Sei $f:X\to Y$ linear, $X,Y$ normierte Räume, $X=\mathcal{D}(f)$. Dann sind folgende Aussagen äquivalent:
	\begin{enumerate}[label={\arabic*)}]
		\item $f$ stetig in $x_0$
		\item $f$ ist stetig auf $X$
		\item $f$ ist beschränkt
	\end{enumerate}
\end{satz}
\rule{4cm}{0.4pt}
\begin{definition}
	Funktion $f:D\subset X\to Y, X,Y$ metrische Räume, heißt \begriff{gleichmäßig stetig} auf $M\subset D$, falls \[ \forall \epsilon > 0 \,\exists \delta > 0: d(f(x), f(\tilde{X})) < \epsilon\quad \forall x,\tilde{x}\in M \text{ mit $d(x,\tilde{x}) < \delta$}, \]
	d.h. $f$ ist $\epsilon\delta$-stetig in jedem $\tilde{x}\in M$ \highlight{und} $\delta > 0$ kann unabhängig von $x\in M$ gewählt werden.
\end{definition}

\begin{satz}
	Sei $f:D\subset X\to Y, X,Y$ metrischer Raum, $f$ stetig auf kompakten $M\subset D$ \\
	$\Rightarrow \,f$ gleichmäßig stetig auf $M$
\end{satz}

\begin{definition}
	Funktion $f:D\subset X\to Y, X,Y$ metrischer Raum, heißt \begriff{\person{Lipschitz}-stetig} auf $M\subset D$, falls \begriff{\person{Lipschitz}-Konstante} $L>0$ existiert mit \begin{align}
		\tag{L} d(f(x), f(\tilde{x})) \le Ld(x,\tilde{x})
	\end{align}
	
	\highlight{Spezialfall:} $X,Y$ normierte Räume, dann hat $L$ die Form \begin{align}
		\tag{L'} \parallel f(x) - f(\tilde{x})\parallel \le L\parallel x - \tilde{x}\parallel \quad\forall x,\tilde{x}\in M
	\end{align}
	
	\highlight{Interpretation:} für $X=Y=\mathbb{R}$ fixiere $\tilde{x}$
	\begin{itemize}
		\item Graph von $f$ liegt im schraffierten Kegel
		\item muss $\forall \tilde{x}\in M$ gelten mit gleichem $L$
	\end{itemize}
\end{definition}

\begin{satz}
	Sei $f:D\subset X\to Y$ \person{lipschitz}-stetig auf $M,X,Y$ metrische Räume\\
	$\Rightarrow$ $f$ gleichmäßig stetig auf $M$ (und damit auch stetig)
\end{satz}

\addtocounter{theorem}{2}

\begin{definition}[Fortsetzung, Einschränkung]
    Funktion $\tilde{f}: D(\tilde{f}) \to Y$ heißt Fortsetzung (bzw. Einschränkung) von $f \mathcal{D}(f) \to Y$ auf $\mathcal{D}(f)$ falls $\mathcal{D} \subset \mathcal{D}(\tilde{f})$ (bzw. $\mathcal{D}(\tilde{f}) \subset \mathcal{D}(f)$) und $\tilde{f}(x) = f(x) \forall x \in \mathcal{D}$ (bzw. $\forall x \in \mathcal{D}(\tilde{f}$). Für eine eingeschränkte Funktion $f$ auf $\mathcal{D}(\tilde{f})$, schreibe $\tilde{f} = f_{\vert \mathcal{D}(\tilde{f})}$.
\end{definition}

\begin{satz}
    Sei $f: D \subset X \to Y$ gleichmäßig stetig auf $D$, wobei $X,Y$  sind metrische Räume , $Y$ ist vollständig $\Rightarrow$ es existiert eindeutige stetige Fortsetzung $\tilde{f}$ von $f$ auf $\bar{D}$ und $\tilde{f}$ ist auf gleichmäßige stetige auf $\bar{D}$.
\end{satz}

\begin{*remark}
    Falls $x_0$ kein Häufungspukt von $D$ ist, so kann man stets stetig auf $D\cup \{x_0\}$ fortsetzen (aber nicht eindeutig).
\end{*remark}

\addtocounter{theorem}{6}

\begin{conclusion}
    Sei $f: D \subset X \to Y$ linear, stetig, $Y$ vollständig $\Rightarrow$ es existiert eindeutig stetige Fortsetzung von $f$ auf $\bar{D}$.
\end{conclusion}

\section{Anwendung}

Sei stets $f: D \subset X \to Y,X,Y$ metrische Räume, $D = \mathcal{D}(f)$.

\begin{satz}
    Sei $f: D \subset Y \to Y$ stetig, $M \subset D$ kompakt $\Rightarrow f(M)$ ist kompakt.
\end{satz}

\begin{satz}
    Sei $f; D \subset X \to Y$ stetig, injektiv, $D$ kompakt $\Rightarrow f^{^1}:f(D) \to D$ ist stetig.
\end{satz}

\begin{theorem}[\index{Weierstraß}Weierstraß]\label{weierstrass}
    Sei $f: D \subset X \to Y$ stetig, $X$ metrischer Raum, $M \subset D$ kompakt, $M \neq \emptyset \Rightarrow$
    \begin{align}
        \exists x_{min}, x_{max} : 
        \begin{cases}
            f(x_{min}) = \min\{f(x)\mid x \in M\} = \min_{x\in M} f(x),\\
            f(x_{max}) = \max\{f(x)\mid x \in M\} = \max_{x\in M} f(x)
        \end{cases}
    \end{align}
    %TODO Fix x \in M under \max und \min
\end{theorem}

\begin{remark}
    Theorem \ref{weierstrass} ist wichtiger Satz für Existenz von Optimallösungen (stetige Funktion beseitzt auf kompakter Menge eine Minimum und Maximum). Folglich sind stetige Funktionen auf kompakten Mengen.
\end{remark}

\begin{satz}
    Sei $f: \mathbb{R}^n \to Y$ linear, $Y$ normierter Raum $\Rightarrow f$ ist stetig auf $\mathbb{R}^n$.\\
    Hinweis: Etwas allgemeiner hat man sogar $f: X \to Y$ linear, $X,Y$ normierte Räume, $\dim X < \infty \Rightarrow f $ ist stetig. (Ist i.a nicht richtig für $\dim X = \infty$.)
\end{satz}

\begin{definition}[\index{Kurve}Kurve]
    Eine stetige Abbildung $f: I \subset X \to Y$, wobei $I$ Intervall und $Y$ metrischer Raum ist heißt Kurve in $Y$ (gelegentlich wird auch Mange $f(I)$ als Kurve und $f$ also zugehörige Parametrisierung bezeichnet).
\end{definition}

\begin{definition}[\index{bogenzusammenhängende Menge}bogenzusammenhängende Menge]
    Menge $M \subset X$, wobei $X$ ist metrische Raum heißt bogenzusammenhängend (bogenweise zusammenhängend) falls $\forall a,b \in M \exists$ Kurve $f: [a,b] \to M$ mit $f(\alpha) = a, f(\beta) = b$.\\
    Bemerkung: Eigentlich ist das die Definition für Wegzusammenhängend, leider ist das in der Literatur nicht eindeutig und manchmal wird zwischen Wegzusammenhängend und zusammenhängend noch das ``echt'' bogenzusammenhängend unterschieden. %TODO definition echt bogenzusammenhängend hinzufügen.
\end{definition}

\begin{definition}[\index{zusammenhängende Menge}zusammenhängende Menge]
    Menge $M \subset X$ heißt zusammenhängend, falls
    \begin{align}
        A, B \subset M \text{ sind offen in }M\text{, disjunkt, }\emptyset \Rightarrow M \neq A \cup B.
    \end{align}
\end{definition}

\begin{example}
    \begin{enumerate}[label={\arabic*)}]
    \item $x \in [0,2\pi] \to (x,\sin x) \in \mathbb{R}^2$ ist Kurve in $\mathbb{R}^2$
    \item $x \in [0,1] \to e^{î\pi x} \in \mathbb{C}$ oder $x \in [0,\pi]\to e^{i\pi} \in \mathbb{C}$ sind Kurven in $\mathbb{C}$
    \item Sei $Y$ normierter Raum, $a,b \in Y,f:[0,1] \to Y$ mit $f(t) = (1-t)\cdot a + t\cdot b$ ist Kurve (Strecke von $a$ nach $b$)
    \end{enumerate}
\end{example}

\begin{example}
    Sei $X=\mathbb{R}^2, M = \{(x,\sin x) \mid x \in (0,1]\} \cup \{(0,0)\}$. Dann ist $M$ zusammenhängend aber nicht bogenzusammenhängend.
\end{example}

\addtocounter{theorem}{1}
\begin{satz}
    Sei $X$ metrischer Raum, $M \subset X$. Dann
    \begin{enumerate}[label={\arabic*)}]
    \item $X = \mathbb{R}: M$ ist zusammenhängend $\Leftrightarrow M$ ist Intervall (offen, abgeschlossen, halboffen, beschränkt, unbeschränkt).
    \item $M$ ist bogenzusammenhängend $\Rightarrow M$ ist zusammenhängend.
    \item Sei $X$ normierter Raum, dann: $M$ ist offen, zusammenhängend $\Rightarrow M$ ist bogenzusammenhängend.
    \end{enumerate}
\end{satz}

\begin{definition}[\index{Gebiet}Gebiet]
    Sei $X$ metrischer Raum, $M \subset X$ heißt \begriff{Gebiet} falls $M$ offen und zusammenhängend ist.\\
    Beachte: Gebiet in einem normiertem Raum ist sogar bogenzusammenhängend.\\
    Offenbar: $M \subset X$ ist konvex $\Rightarrow M$ ist bogenzusammenhängend.
\end{definition}

\begin{satz}
    Sei $f: D\subset X\to Y$ stetig, wobei $X,Y$ metrische Räume sind, dann gilt: $M \subset D$ ist zusammenhängend $\Rightarrow f(M)$ ist zusammenhängend.
\end{satz}

\begin{theorem}[\index{Zwischenwertsatz}Zwischenwertsatz]
    Sei $f: D \subset X \to \mathbb{R}, M \subset D$ zusammenhängend, $a,b \in M \Rightarrow f$ nimmt auf $M$ jeden Wert zwischen $f(a)$ und $f(b)$ an.
\end{theorem}

\addtocounter{theorem}{1}
%TODO add the example here or not?

\begin{example}
    $f:[a,b] \to \mathbb{R}$ sei stetig mit $f([a,b]) \subset [a,b] \Rightarrow$ besitzt \begriff{Fixpunkt}, d.h. $\exists x \in [a,b]\colon f(x)=x$.
\end{example}

\begin{theorem}[\index{Fundamentalsatz der Algebra}Fundamentalsatz der Algebra]\label{Fundam_algebra}
    Sei $f: \mathbb{C} \to \mathbb{C}$ Polynom vom Grad $n\geq 1$ (d.h $f(z) = a_n z^n + \dots + a_1 z + a_0,a_j \in \mathbb{C}, a_n \neq 0, n\geq 1$) $\Rightarrow f$ besitzt (mindestens eine) Nullstelle $z_0 \in \mathbb{C}$ (d.h. $f(z_0) = 0$).
\end{theorem}

\begin{conclusion}
    Jedes Polynom $f: \mathbb{C} \to \mathbb{C}$ von Grad $n, f\neq 0$ besitzt genau $n$ Nullstellen in $\mathbb{C}$ gezählt mit Vielfachen, d.h. $\exists z_1,\dots,z_l \in \mathbb{C}$, paarweise verschieden (=verschieden) $k_1,\dots, k_l \in \mathbb{N}_{\geq 0}$, $a_n \in \mathbb{C}\setminus\{0\}$ mit $k_1 + \dots + k_l = n$ und $f(z) = a_n \cdot (z-z_1)^{k_1}\cdot\dots\cdot(z-z_l)^{l}\forall z \in \mathbb{C}$. Hier heißt $k_j$ Vielfachheit der Nullstelle $z_j$.\\
    Hinweis: In dem Satz \ref{Polynomdiv} wurde gezeigt, das $f$ höchstens $n$ Nullstellen besitzt.
\end{conclusion}

\begin{definition}[\index{analytische Funktion}analytische Funktion]
    Abbildung $f:\mathbb{C} \to \mathbb{C}$ heißt analytisch auf $B_R(z_0)\subset \mathbb{C}$ falls $f$ auf $B_R(z_0)$ durch Potenzreihe in $z_0$ darstellbar ist, d.h.
    \[
    f(z)=\sum_{k=0}^{\infty} a_k(z-z_0)^k \forall z \in B_R(z_0).
    \]
\end{definition}

\begin{satz}
    Sei $f:\mathbb{C}\to\mathbb{C}$ analytisch auf $B_R(z_0)$ und sei $B_r(z_1) \subset B_R(z_0)$ für $z_1 \in B_R(z_0),r>0 \Rightarrow f$ ist analytisch auf $B_r(z_1)$.
\end{satz}

\begin{satz}[\index{Identitätssatz}Identitätssatz]
    Seien $f,g:\mathbb{C} \to \mathbb{C}$ analytisch auf $B_R(z_0)$, sei $z_n \to \tilde{z},z_n\in B_R(z_0)\setminus\{\tilde{z}\}$ und $f(z_n) = g(z_n)\forall n \in \mathbb{N} \Rightarrow f(f) = g(z)\forall z \in B_R(z_0)$.
\end{satz}

\begin{remark}
    Analytische Funktionen sind durch Werte auf ``sehr kleinen'' Mengen bereits festgelegt (z.B $\exp,\sin\cos$ sind auf $\mathbb{C}$ eindeutig durch Werte auf $\mathbb{R}$ festgelgt).
\end{remark}

\begin{overview}
    Sei $X$ metrischer Raum, $Y$ normierter Raum.
    \begin{itemize}
    \item $B(X,Y):=\{f:X\to Y\mid \Vert f\Vert_{\infty} < \infty\}$ ist normierter Raum der bschränkten Funktionen mit $\Vert f\Vert_{\infty}=\sup\{\Vert f \Vert_{Y} \mid x \in X\}$.
    \item $C_b(X,Y):=\{f:X\to Y\mid \Vert f \Vert_{\infty} < \infty, f \text{ ist stetig}\}$ ist Menge der beschränkten stetigen Funktionen und offenbar eine linearer Unterraum von $B(X,Y)$ und damit auch Kern von $R \text{ mit } \Vert \cdot \Vert_{\infty}$.
    \item $C(X,Y):= \{f: X\to Y\mid f \text{ ist steig}\}$, Menge der stetigen Funktionen ist offenbar ein Vektorraum (enthält unbeschränkte Funktionen, z.B. $f(x)=\frac{1}{x} \text{ mit } x \in X = (0,1)$).
    \end{itemize}
\end{overview}

\begin{remark}
    Falls $X$ kompakt ist, dann kann man den Ausdruck $\Vert f \Vert_{\infty} < \infty$ in der Definition von $C_b(X,Y)$ weglassen (vgl. Theorem \ref{weierstrass}), d.h. $C_b(X,Y) = C(X,Y),f \text{ stetig }\Rightarrow X \to \Vert f(x)\Vert$ ist stetig $\overset{Theorem 15.3}{\Rightarrow} f$ ist beschränkt auf $X$. In diesem Fall ist auch $C(X,Y)$ mit $\Vert \cdot \Vert_{\infty}$ normierter Raum und $\Vert f\Vert_{\infty} = \max_{x\in M}\Vert f(x)\Vert_{Y}$.
\end{remark}

\begin{satz}
    Sei $X$ metrischer Raum, $Y$ Banachraum $\Rightarrow B(X,Y)$ und $C_b(X,Y)$ und Banachräume (mid $\Vert \cdot \Vert_{\infty}$).
\end{satz}

\begin{definition}[\index{Kontraktion}Kontraktion]
    Funktion $f: D \subset X \to X$, wobei $X$ metrischer Raum ist, heißt \begriff{Kontraktion} (bzw. kontraktiv) auf $M \subset D$ falls
    \[
    \exists L 0 \leq L < 1\colon d(f(x),f(y)) \leq L\cdot d(x,y) \forall x,y \in M.
    \]
    D.h. $f$ ist Lipschitz-stetig mit Lipschitzkonstante $L < 1$, folglich ist $f$ auch stetig.
\end{definition}

\begin{theorem}[\index{Banacherscher Fixpunktsatz}Banacherscher Fixpunktsatz]\label{Banach_Fixpunkt}
    Sei $f : D\subset X \to Y$ Kontraktion auf $M \subset D, X$ vollständiger metrischer Raum (z.B. Banachraum), $M$ abgeschlossen und $f(M) \subset M$. Dann
    \begin{enumerate}
        \item[(1)] $f$ besitzt genau einen Fixpunkt $\tilde{x}$ auf $M$ (d.h. $\exists$ genau ein $\tilde{x} \in M\colon f(\tilde{x} = \tilde{x})$).
        \item[(2)] Für $\{x_n\}$ in $M$ mit $x_{n+1}=f(x_n),x_0 \in M$ (beliebig) gilt:
            \[
            x_n \to x \text{ und } d(x_n,\tilde{x}) \leq \frac{L^n}{1-L}\cdot d(x_0,x_1) \forall n \in \mathbb{N}.
            \]
        \end{enumerate}
        Hinweis: Theorem \ref{Banach_Fixpunkt} ist eine wichtige Grundlage für Iterationsverfahren in der Numerik.
\end{theorem}

\subsection{Partialbruchzerlegung}

\begin{definition}[\index{Pol der Ordnung $k$}Pol der Ordnung $k$]
    Sei $R: \mathbb{C} \to \mathbb{C}$ rationale FUnktion, d.h. $R(z) = \frac{f(z)}{g(z)}$ für Polynome $f,g$ exitieren mit
    \[
    R(z) = \frac{\tilde{f}(z)}{(z-z_0)^k\cdot \tilde{g}} \text{ und } \tilde{f}(z_0) \neq 0, \tilde{g}(z_0) \neq 0.
    \]
    Motivation: Gelgentlich ist gewisse additive Zerlegung von rationalen Funktionen wichtig (Integration) z.B.
    \[
    \frac{2x}{x^2 - 1} = \frac{2x}{(x-1)(x+1)} = \frac{1}{x+1}+\frac{1}{x-1}.
    \]
\end{definition}

\begin{lemma}
    Sei $R: \mathbb{C} \to \mathbb{C}$ rationale Funktion, $z_0 \in \mathbb{C}$ Pol der Ordnung $k\geq 1 \Rightarrow \exists ! a_1,\dots,a_k \in \mathbb{C},a_k\neq 0$ und $\exists !$ Polynom $\tilde{p}$ mit 
    \[
    R(z) = \sum_{i=1}^{k}
    \frac{a_i}{(z-z_0)^{î}} + \frac{\tilde{p}(z)}{\tilde{g}(z)} = H(z) +\frac{\tilde{p}(z)}{\tilde{g}(z)}
    \]
    $H(z)$ heißt Hauptteil von $R \text{ in } z_0$. Beachte das $\frac{\tilde{p}}{\tilde{g}}$ keine Pole in $z_0$ hat.
\end{lemma}

\begin{satz}[\index{Partialbruchzerlegung}Partialbruchzerlegung]
    Sei $R: \mathbb{C} \to \mathbb{C}$ rationale Funktion, $R(z)=\frac{f(z)}{g(z)}$ für Polynome $f,g$. Sei $g(z) = \prod_{i=1}^{l}(z-z_i)^{k_i}$ gemäß Fundamentalsatz der Algebra(Theorem \ref{Fundam_algebra}). Seien $z_1,\dots,z_l$ keine Nullstellen von $f$ und seien $H_1,\dots,H_l$ Hauptteile von $R$ in $z_1,\dots,z_l \Rightarrow$
    \[
    \exists \text{ Polynom } p:R(z)=H_1(z)+\dots+H_l(z)+p(z) \forall z \neq z_j \forall j = 1,\dots,l
    \]
    wobei $f(z) = p(z)\cdot g(z) + r(z)\forall z$ für Polynom $r$. $p=0$ falls $\grad(f) < \grad(g)$ (vgl Satz \ref{Polynomdiv} Polynomdivision)
\end{satz}

\chapter*{Liste der Theoreme}
\addcontentsline{toc}{chapter}{Liste der Theoreme}
\theoremlisttype{allname}
\listtheorems{theorem}
\chapter*{Liste der benannten Sätze}
\addcontentsline{toc}{chapter}{Liste der benannten Sätze}
\theoremlisttype{optname}
\listtheorems{satz}
\printglossary[type=\acronymtype]
\addcontentsline{toc}{chapter}{Akronyme}
\printindex
\addcontentsline{toc}{chapter}{Index}
\printindex[symbols]
\addcontentsline{toc}{chapter}{Symbolverzeichnis}
\end{document}