So far, we have looked at statistical models of the form $Y\sim Normal(X\beta,\sigma\mathbbm{1})$. This is a flexible framework, allowing us to model linear and non-linear relationships between a response and predictors. We covered model formulation, model fitting, model checking, model selection, hypothesis test on model fits, predictions from models and the design of experiments to effectively collect data for statistical analysis. Today we will look at an even more general class of statistical models than linear models.

\subsection{Where linear models are not enough}

\begin{example}
	Consider yes/no outcomes or count data:
	
	The normal distribution for the errors $\epsilon$ is not appropriate here.
\end{example}

There is a more general class of models than linear models, called \begriff{Generalised Linear Models} (GLMs). They can be written as
\begin{align}
	\mathbb{E}(Y_i) &\equiv \mu_i =\gamma(X_i\beta) \notag \\
	Y_i &\overset{\text{independent}}{\sim} \text{ Exponential family distribution} \notag
\end{align}
where $\gamma$ is any smooth monotonic function. The Exponential family of distributions includes distributions such as Poisson, Gaussian, binomial and gamma.

GLMs are written in terms of the \begriff{link function}, $g$, which is the inverse of $\gamma$:
\begin{align}
	g(\mu_i) &= X_i\beta \notag \\
	Y_i &\overset{\text{independent}}{\sim} \text{ Exponential family distribution} \notag
\end{align}

\begin{example}
	\begin{align}
		Y_i &\sim \text{Binomial} \notag \\
		g(\mu) &= \ln\left(\frac{\mu}{1-\mu}\right)\notag
	\end{align}
	logit link function.
\end{example}

\begin{example}
	Linear models are a special case of GLMs.
\end{example}

The link functions in the following table are only examples. Other link functions can be used with distributions.
\begin{center}
	\begin{tabular}{c|c|c|c|c}
		\textbf{distribution} & \textbf{support} & \text{use} & \textbf{link name} & \textbf{link function} \\
		\hline
		normal & $(-\infty,\infty)$ & linear response & identity & $\mu$ \\
		exponential & $(0,\infty)$ & exponential response & inverse & $\mu^{-1}$ \\
		gamma & $(0,\infty)$ & gamma response & log & $\ln(\mu)$ \\
		Poisson & $\natur_0$ & count data & log & $\ln(\mu)$ \\
		binomial & & proportions of yes/no occurrences & logit & $\ln\left(\frac{\mu}{1-\mu}\right)$
	\end{tabular}
\end{center}

\begin{example}
	AIDS cases per year in Belgium at the start of the epidemic
	\begin{center}
		\begin{tikzpicture}
			\begin{axis}[
			xmin=0, xmax=13, xlabel=year since 1980,
			ymin=0, ymax=250, ylabel=new AIDS cases,
			axis x line=bottom,
			axis y line=left,
			]
			\addplot[blue, only marks, mark=x] coordinates {
				(1,10)
				(2,15)
				(3,30)
				(4,55)
				(5,65)
				(6,70)
				(7,130)
				(8,150)
				(9,170)
				(10,210)
				(11,250)
				(12,245)
				(13,240)
			};
			\end{axis}
		\end{tikzpicture}
	\end{center}
	Early in an epidemic, an exponential increase in cases can occur
	\begin{align}
		\mathbb{E}(Y_i) &\equiv \mu_i = \delta\exp(\alpha t_i) \notag \\
		Y_i &\sim Poisson(\mu_i) \notag
	\end{align}
	Taking the logarithm of both sides and letting $\beta_0\equiv\log(\delta)$ and $\beta_1\equiv\alpha$
	\begin{align}
		\log(\mu_i) &= \beta_0 + \beta_1 t_i \notag \\
		Y_i &\sim Poisson(\mu_i) \notag
	\end{align}
	which is a GLM with a log link.
\end{example}

\subsection{Model fitting in GLMs}

To fit GLMs to data, we use the principle of Maximum Likelihood Estimation (MLE). Given parameters $\beta$, we can write down $f(Y\mid \beta)$, the probability or probability density function of the response $Y$. For observed data, $Y_i^{\text{obs}}$, the likelihood function is
\begin{align}
	L(\beta) = \prod_i f(Y_i^{\text{obs}}\mid \beta)\notag
\end{align}
For GLMs, there is no closed form solution for the values of $\beta$ that maximise this function. Instead, MLE is performed numerically, using a technique called \begriff{iteratively re-weighted least squares} (IRLS). In principle, other optimisation algorithms could also be used for MLE.

\subsection{GLM assumptions and checking}

Important GLM assumptions:
\begin{itemize}
	\item Independence
	\item Distributional assumptions
	\item Weak exogeneity
	\item Linear relationship between transformed response and predictors (link function)
\end{itemize}
Residual plots are still useful to check if model assumptions hold.

As we use different distributions, we cannot simply use raw residuals, as for linear models. Two common types of residuals that attempt to mimic behaviour of residuals for LMs:
\begin{itemize}
	\item \person{Pearson} Residuals
	\item Deviance Residuals
\end{itemize}

\subsection{Hypothesis tests on GLM fits}

\subsection{Model selection on GLMs}

\subsection{Interpreting GLM parameters}