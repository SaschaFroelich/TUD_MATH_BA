\documentclass[british,a4paper,order=firstname]{mathscript}
\usepackage{mathoperators}

\title{\textbf{Applied statistics: Coursework 1}}
\author{\person{Henry Haustein}}

\begin{document}
\pagenumbering{roman}
\pagestyle{plain}

\maketitle

\hypertarget{tocpage}{}
\tableofcontents
\bookmark[dest=tocpage,level=1]{Table of contents}

\pagebreak
\pagenumbering{arabic}
\pagestyle{fancy}

\section{Task 1}
\subsection{Part (1)}
In the given data were two out of 26 data points with an Al/Be ratio of more than 4.5. That means
\begin{align}
	\hat{p} = \frac{2}{26} = \frac{1}{13}\notag
\end{align}

\subsection{Part (2)}
Using the following formula from the lecture we get the 95\% confidence interval:
\begin{align}
	\hat{p}&\pm 2\cdot\sqrt{\frac{\hat{p}(1-\hat{p})}{n}} \notag \\
	\frac{1}{13} &\pm \underbrace{2\cdot\sqrt{\frac{\frac{1}{13}\cdot\frac{12}{13}}{26}}}_{0.1045} \notag
\end{align}
Our 95\% confidence interval is [-0.0276,0.1814] which means that we are 95\% sure that the true proportion lies between -0.0276 and 0.1814. 

\subsection{Part (3)}

\subsection{Part (4)}

\pagebreak
\section{Task 2}
\subsection{Part (1)}
\begin{lstlisting}
x = [-4.5, -1, -0.5, -0.15, 0, 0.01, 0.02, 0.05, ...
0.15, 0.2, 0.5, 0.5, 1, 2, 3];
m = mean(x);
s = std(x);
\end{lstlisting}
\begin{center}
	\begin{tabular}{p{4cm}|p{7cm}}
		null hypothesis & $H_0$: $\mu = 0$ \\
		\hline
		alternative hypothesis & $H_A$: $\mu\neq 0$ \\
		\hline
		t-test for $\mu$ & $t=\frac{m-0}{\frac{s}{\sqrt{15}}} =\frac{0.0853}{\frac{1.6031}{\sqrt{15}}} = 0.2062$ \\
		\hline
		rejection region & \texttt{tinv(0.05,15)} = -1.7531 \\
		\hline
		conclusion & $t$ lies not in the rejection region so $H_0$ is accepted at the 10\% significance level.
	\end{tabular}
\end{center}
\begin{center}
	\begin{tikzpicture}
	\begin{axis}[
	xmin=-3, xmax=3, xlabel=$x$,
	ymin=0, ymax=1, ylabel=$y$,
	samples=400,
	axis y line=middle,
	axis x line=middle,
	domain=-3:3,
	restrict y to domain=0:1,
	width = 16cm,
	height = 8cm,
	]
	\addplot[name path=f,blue] {116640000000*sqrt(15)/(143*pi*(x^2+15)^(8))};
	\path[name path=axis] (axis cs:1.7531,0) -- (axis cs:3,0);
	\path[name path=axis2] (axis cs:-3,0) -- (axis cs:-1.7531,0);
	\draw (axis cs:1.7531,0) -- (axis cs:1.7531,1);
	\draw (axis cs:-1.7531,0) -- (axis cs:-1.7531,1);
	\draw [dotted] (axis cs:0.2062,0) -- (axis cs:0.2062,0.6);
	\node at (axis cs:1,0.5) (a) {0.05};
	\node at (axis cs:-1,0.5) (a2) {0.05};
	\draw (axis cs:1, 0.46) -- (axis cs: 2.2,0.02);
	\draw (axis cs:-1, 0.46) -- (axis cs: -2.2,0.02);
	\node at (axis cs: 2.0,0.95) (b) {1.7531};
	\node at (axis cs: -2.0,0.95) (b2) {-1.7531};
	\node at (axis cs: 0.45,0.55) (c) {0.2062};
	\node[red] at (axis cs: 2.4,0.78) (d) {rejection region};
	\node[red] at (axis cs: -2.4,0.78) (d2) {rejection region};
	\begin{scope}[transparency group]
	\begin{scope}[blend mode=multiply]
	\addplot [thick,color=blue,fill=blue,fill opacity=0.3] fill between[of=f and axis,soft clip={domain=1.7531:3},];
	\addplot [thick,color=blue,fill=blue,fill opacity=0.3] fill between[of=f and axis2,soft clip={domain=-3:-1.7531},];
	\draw[red,fill=red,opacity=0.2] (axis cs: 1.7531,0) -- (axis cs: 1.7531,1) -- (axis cs: 3,1) -- (axis cs: 3,0) -- (axis cs: 1.7531,0);
	\draw[red,fill=red,opacity=0.2] (axis cs: -1.7531,0) -- (axis cs: -1.7531,1) -- (axis cs: -3,1) -- (axis cs: -3,0) -- (axis cs: -1.7531,0);
	\end{scope}
	\end{scope}
	\end{axis}
	\end{tikzpicture}
\end{center}

\subsection{Part (2)}
If we reduce the significance level our rejection region gets smaller. With $\alpha = 0.05$ the rejection region will start at \texttt{tinv(0.025,15)} = -2.1314. The $t$ calculated in part (1) won't change $\Rightarrow$ our decision won't change too.

To get the type 2 error we use the MATLAB function \texttt{sampsizepwr} and $type\, 2\, error = 1-power$.
\begin{lstlisting}
testtype = 't';
p0 = [0 1.6031];
p1 = 0.0853;
n = 15;
power = sampsizepwr(testtype,p0,p1,[],n)
\end{lstlisting}
This gives $power = 0.0542\Rightarrow type\, 2\, error = 0.9458$. This is the probability of wrongly accepting $H_0$ when it is false.

\subsection{Part (3)}

\pagebreak
\section{Task 3}
\subsection{Part (1)}

\subsection{Part (2)}

\pagebreak
\section{Task 4}
\subsection{Part (1)}
The probability density function $f(t)$ is
\begin{align}
	f(t) = \frac{2t\cdot\frac{\exp(-t^2)}{100}}{100} = \frac{t\cdot\exp(-t^2)}{5000}\notag
\end{align}
\begin{center}
	\begin{tikzpicture}
	\begin{axis}[
	xmin=0, xmax=5, xlabel=$x$,
	ymin=0, ymax=0.0001, ylabel=$y$,
	samples=400,
	axis y line=middle,
	axis x line=middle,
	restrict y to domain=0:1,
	]
	\addplot+[mark=none] {(x*exp(-x^2))/5000};
	\end{axis}
	\end{tikzpicture}
\end{center}
The cumulative distribution function $F(t)$ is then
\begin{align}
	F(t) &= \int_0^t f(\xi)\,\diff\xi \notag \\
	&= \int_0^t \frac{\xi\cdot\exp(-\xi^2)}{5000}\,\diff\xi\notag \\
	&= \frac{\exp(-t^2)\Big(\exp(t^2)-1\Big)}{10000} \notag
\end{align}
\begin{center}
	\begin{tikzpicture}
	\begin{axis}[
	xmin=0, xmax=5, xlabel=$x$,
	ymin=0, ymax=0.0001, ylabel=$y$,
	samples=400,
	axis y line=middle,
	axis x line=middle,
	restrict y to domain=0:1,
	]
	\addplot+[mark=none] {(exp(-x^2)*(exp(x^2)-1))/10000};
	\end{axis}
	\end{tikzpicture}
\end{center}
For the survival function we get
\begin{align}
	R(t) &= 1 - F(t) \notag \\
	&= \frac{\exp(-t^2)+9999}{10000} \notag
\end{align}
\begin{center}
	\begin{tikzpicture}[scale=0.9]
	\begin{axis}[
	xmin=0, xmax=5, xlabel=$x$,
	ymin=0, ymax=1, ylabel=$y$,
	samples=400,
	axis y line=middle,
	axis x line=middle,
	restrict y to domain=0:1,
	]
	\addplot+[mark=none] {(exp(-x^2)+9999)/10000};
	\end{axis}
	\end{tikzpicture}
	\begin{tikzpicture}[scale=0.9]
	\begin{axis}[
	xmin=0, xmax=5, xlabel=$x$,
	ymin=0.9999, ymax=1, ylabel=$y$,
	samples=400,
	axis y line=middle,
	axis x line=middle,
	restrict y to domain=0:1,
	y tick label style={
		/pgf/number format/.cd,
		precision=5,
		/tikz/.cd
	},
	]
	\addplot+[mark=none] {(exp(-x^2)+9999)/10000};
	\end{axis}
	\end{tikzpicture}
\end{center}
To get the reliability of the component at $t=7$ we simply evaluate $R(7)$ which is 0.9999.

The hazard function is defined as
\begin{align}
	h(t) &= \frac{f(t)}{1-F(t)} \notag \\
	&= \frac{2t}{9999\cdot \exp(t^2)+1} \notag
\end{align}
\begin{center}
	\begin{tikzpicture}
	\begin{axis}[
	xmin=0, xmax=5, xlabel=$x$,
	ymin=0, ymax=0.0001, ylabel=$y$,
	samples=400,
	axis y line=middle,
	axis x line=middle,
	restrict y to domain=0:1,
	]
	\addplot+[mark=none] {(2*x)/(9999*exp(x^2)+1)};
	\end{axis}
	\end{tikzpicture}
\end{center}
The hazard function describes how an item ages where $t$ affects the risk of failure. It is the frequency with which the item fails, expressed in failures per unit of time.

\subsection{Part (2)}
Given $h(x)\sim(\sqrt{x})^{-1}$ we will try to find out the $shape$-parameter of the \person{Weibull} distribution first.
\begin{center}
	\begin{tikzpicture}
	\begin{axis}[
	xmin=0, xmax=5, xlabel=$x$,
	ymin=0, ymax=1, ylabel=$y$,
	samples=400,
	axis y line=middle,
	axis x line=middle,
	restrict y to domain=0:1,
	yticklabels={,,},
	xticklabels={,,}
	]
	\addplot+[mark=none] {1/sqrt(x)};
	\end{axis}
	\end{tikzpicture}
\end{center}
Comparing this graph to graphs of the hazard function with different $shape$-parameters we see that $shape=0.5$ fits best.
\begin{center}
	\begin{tabular}{p{5cm}|p{5cm}|p{5cm}}
		$shape = 0.5$ & $shape = 1$ & $shape = 2$ \\
		\hline
		\multicolumn{3}{c}{\cellcolor{gray!50}\textbf{Hazard function} $\left(h = \frac{\text{PDF}}{1-\text{CDF}}\right)$} \\
		\hline
		\begin{tikzpicture}[scale=0.6]
		\begin{axis}[
		xmin=0, xmax=2, xlabel=$x$,
		ymin=0, ymax=1, ylabel=$y$,
		samples=400,
		axis y line=middle,
		axis x line=middle,
		]
		\addplot+[mark=none] {(0.5*x^(-0.5)*exp(-x^0.5))/(1-(1-exp(-x^0.5)))};
		\end{axis}
		\end{tikzpicture} &
		\begin{tikzpicture}[scale=0.6]
		\begin{axis}[
		xmin=0, xmax=2, xlabel=$x$,
		ymin=0, ymax=1, ylabel=$y$,
		samples=400,
		axis y line=middle,
		axis x line=middle,
		restrict y to domain=0:1,
		]
		\addplot+[mark=none] {(1/exp(x))/(1-(1-1/exp(x)))};
		\draw[blue] (axis cs: 0,1) -- (axis cs: 2,1);
		\end{axis}
		\end{tikzpicture} &
		\begin{tikzpicture}[scale=0.6]
		\begin{axis}[
		xmin=0, xmax=2, xlabel=$x$,
		ymin=0, ymax=1, ylabel=$y$,
		samples=400,
		axis y line=middle,
		axis x line=middle,
		]
		\addplot+[mark=none] {(2*x^(1)*exp(-x^2))/(1-(1-exp(-x^2)))};
		\end{axis}
		\end{tikzpicture} \\
	    \end{tabular}
\end{center}
To get the $scale$-parameter of the distribution we use the other provided information:
\begin{align}
	5 &= \mu \notag \\
	&= scale\cdot\Gamma\left(1+\frac{1}{shape}\right) \notag \\
	&= scale\cdot\Gamma(3) \notag \\
	\Rightarrow scale &= \frac{5}{2} \notag
\end{align}
Let's build the survival function:
\begin{align}
	R(t) &= 1-\Bigg(1-\exp\left(-\sqrt{\frac{x}{\nicefrac{5}{2}}}\right)\Bigg) \notag \\
	&= \exp(-\sqrt{x}\cdot\sqrt{2.5})\notag
\end{align}
That mean that the probability of surviving 6 years (30 years) is $R(6) = 0.0208$ ($R(30) = 0.0002$).
\end{document}
