\documentclass[11pt]{article}
\usepackage[a4paper,left=2cm,right=2cm,top=2cm,bottom=4cm,bindingoffset=5mm]{geometry}
\usepackage{scrpage2}
\usepackage{amsmath}
\usepackage{paralist}
\usepackage{amssymb}
\usepackage{framed}

\title{\textbf{Lineare Algebra 1. Semester (WS2017/18)}}
\author{Dozent: Prof. Dr. Arno Fehm}
\date{}
\begin{document}

\maketitle

\raggedright 
\section{Grundgegriffe der Linearen Algebra}
	\subsection{Logik und Mengen}
		Wir werden die Grundlagen der Logik und der Mengenlehre kurz ansprechen.
		\subsubsection{\"Uberblick \"uber die Aussagenlogik}
			Jede mathematisch sinnvolle Aussage ist entweder wahr oder falsch, aber nie beides!
			\begin{compactitem}
				\item "$1+1=2$" $\to$ wahr
				\item "$1+1=3$" $\to$ falsch
				\item "Es gibt unendlich viele Primzahlen" $\to$ wahr
			\end{compactitem}
			Man ordnet jeder mathematischen Aussage $A$ einen Wahrheitswert "wahr" oder "falsch" zu. Aussagen
			lassen sich mit logischen Verkn\"upfungen zu neuen Aussagen zusammensetzen.
			\begin{compactitem}
				\item $\lor \to$ oder
				\item $\land \to$ und
				\item $\lnot \to$ nicht
				\item $\Rightarrow \to$ impliziert
				\item $\iff \to$ \"aquivalent
			\end{compactitem}
			Sind also $A$ und $B$ zwei Aussagen, so ist auch $A \lor B$, $A \land B$, $\lnot A$, 
			$A \Rightarrow B$ und $A \iff B$ Aussagen. Der Wahrheitswert einer zusammengesetzen Aussage ist
			eindeutig bestimmt durch die Wahrheitswerte ihrer Einzelaussagen.
			\begin{compactitem}
				\item $\lnot (1+1=3) \to$ wahr
				\item "2 ist ungerade" $\Rightarrow$ "3 ist gerade" $\to$ wahr
				\item "2 ist gerade" $\Rightarrow$ "Es gibt unendlich viele Primzahlen" $\to$ wahr
			\end{compactitem}
			$\newline$
			\begin{center}
				\begin{tabular}{|c|c|c|c|c|c|c|}
					\hline
						$A$ & $B$ & $A \lor B$ & $A \land B$ & $\lnot A$ & $A \Rightarrow B$ & $A \iff B$\\
					\hline
						w & w & w & w & f & w & w\\
					\hline
						w & f & w & f & f & f & f\\
					\hline
						f & w & w & f & w & w & f\\
					\hline
						f & f & f & f & w & w & w\\
					\hline
				\end{tabular}
			\end{center}
			
		\subsubsection{\"Uberblick \"uber die Pr\"adikatenlogik}
			Wir werden die Quantoren
			\begin{compactitem}
				\item $\forall$ (Allquantor, "f\"ur alle") und
				\item $\exists$ (Existenzquantor, "es gibt") verwenden.
			\end{compactitem}
			Ist $P(x)$ eine Aussage, deren Wahrheitswert von einem unbestimmten $x$ abh\"angt, so ist \\
			$\forall x: P(x)$ genau dann wahr, wenn $P(x)$ f\"ur alle $x$ wahr ist, \\
			$\exists x: P(x)$ genau dann wahr, wenn $P(x)$ f\"ur mindestens ein $x$ wahr ist. \\
			$\newline$
			Insbesondere ist $\lnot \forall x: P(x)$ genau dann wahr, wenn $\exists x: \lnot P(x)$ wahr ist. \\
			Analog ist $\lnot \exists x: P(x)$ genau dann wahr, wenn $\forall x: \lnot P(x)$ wahr ist.
			
		\subsubsection{\"Uberblick \"uber die Beweise}
			Unter einem Beweis verstehen wir die l\"uckenlose Herleitung einer mathematischen Aussage aus einer
			Menge von Axiomen, Vorraussetzungen und schon fr\"uher bewiesenen Aussagen. \\
			Einige Beweismethoden:
			\begin{compactitem}
				\item \textbf{Widerspruchsbeweis} \\
				Man nimmt an, dass eine zu beweisende Aussage $A$ falsch sei und leitet daraus ab, dass eine 
				andere Aussage sowohl falsch als auch wahr ist. Formal nutzt man die G\"ultigkeit der Aussage
				$\lnot A \Rightarrow (B \land \lnot B) \Rightarrow A$.
				\item \textbf{Kontraposition} \\
				Ist eine Aussage $A \Rightarrow B$ zu beweisen, kann man stattdessen die Implikation 
				$\lnot B \Rightarrow \lnot A$ beweisen.
				\item \textbf{vollst\"andige Induktion} \\
				Will man eine Aussage $P(n)$ f\"ur alle nat\"urlichen Zahlen zeigen, so gen\"ugt es, zu zeigen,
				dass $P(1)$ gilt und dass unter der Induktionsbehauptung $P(n)$ stets auch $P(n+1)$ gilt 
				(Induktionschritt). Dann gilt $P(n)$ f\"ur alle $n$. \\
				Es gilt also das Induktionsschema: $P(1) \land \forall n: (P(n) \Rightarrow P(n+1)) \Rightarrow
				\forall n: P(n)$.
			\end{compactitem}
			
		\subsubsection{\"Uberblick \"uber die Mengenlehre}
			Jede Menge ist eine Zusammenfassung bestimmter wohlunterscheidbarer Objekte zu einem Ganzen. Eine
			Menge enth\"alt also solche Objekte, die Elemente der Menge. Die Menge ist durch ihre Elemente
			vollst\"andig bestimmt. Diese Objekte k\"onnen f\"ur uns verschiedene mathematische Objekte, wie
			Zahlen, Funktionen oder andere Mengen sein. Man schreibt $x \in M$ bzw. $x \notin M$, wenn x ein
			bzw. kein Element der Menge ist. \\
			$\newline$
			Ist $P(x)$ ein Pr\"adikat, so bezeichnet man eine Menge mit $X := \{x \mid P(x)\}$. Hierbei muss
			man vorsichtig sein, denn nicht immer lassen sich alle $x$ f\"ur die $P(x)$ gilt, widerspruchsfrei
			zu einer Menge zusammenfassen. \\
			$\newline$
			
			\textbf{Beispiel: endliche Mengen} \\
			Eine Menge hei{\ss}t endlich, wenn sie nur endlich viele Elemente enth\"alt. Endliche Mengen
			notiert man oft in aufz\"ahlender Form: $M = \{1;23;4;5;6\}$. Hierbei ist die Reihenfolge
			der Elemente nicht relevant, auch nicht die H\"aufigkeit eines Elements. \\
			Sind die Elemente paarweise verschieden, dann ist die Anzahl der Elemente die M\"achtigkeit
			(oder Kardinalit\"at) der Menge, die wir mit $|M|$ bezeichnen. \\
			$\newline$
			\textbf{Beispiel: unendliche Mengen} \\
			\begin{compactitem}
				\item Menge der nat\"urlichen Zahlen: $\mathbb N := \{1,2,3,4,...\}$
				\item Menge der nat\"urlichen Zahlen mit der 0: $\mathbb N_0 := \{0,1,2,3,4,...\}$
				\item Menge der ganzen Zahlen: $\mathbb Z := \{...,-2,-1,0,1,2,...\}$
				\item Menge der rationalen Zahlen: $\mathbb Q := \{\frac p q \mid p,q \in \mathbb Z, q 
				\neq 0\}$
				\item Menge der reellen Zahlen: $\mathbb R := \{x \mid x$ ist eine reelle Zahl$\}$
			\end{compactitem}
			Ist $M$ eine Menge, so gilt $|M|=\infty$ \\
			$\newline$
			
			\textbf{Beispiel: leere Mengen} \\
			Es gibt genau eine Menge, die keine Elemente hat, die leere Menge $0 := \{\}$.
			
			\begin{framed}
				\textbf{Definition Teilmenge:} Sind $X$ und $Y$ zwei Mengen, so heißt $X$ eine Teilmenge von 
				$Y$, wenn jedes Element von $X$ auch Element von $Y$ ist, dass heißt wenn für alle 
				$x$ $(x \in X \Rightarrow x \in Y)$ gilt.
			\end{framed}
			
			Da eine Menge durch ihre Elemente bestimmt ist, gilt $X = Y \Rightarrow (X \subset Y)\land
			(Y \subset X)$. Will man Mengengleichheit beweisen, so gen\"ugt es, die beiden Inklusionen
			$X \subset Y$ und $Y \subset X$ zu beweisen. \\
			$\newline$
			
			Ist $X$ eine Menge und $P(x)$ ein Pr\"adikat, so bezeichnet man mit $Y:= \{x \in X \mid
			P(x)\}$ die Teilmenge von $X$, die das Pr\"adikat $P(x)$ erf\"ullen. \\
			
			\begin{framed}
				\textbf{Definition Mengenoperationen:} Seien $X$ und $Y$ Mengen. Man definiert daraus 
				weitere Mengen wie folgt:
				\begin{compactitem}
					\item $X \cup Y := \{x \mid x \in X \lor x \in Y\}$
					\item $X \cap Y := \{x \mid x \in X \land x \in Y\}$
					\item $X \backslash Y := \{x \in X \mid x \notin Y\}$
					\item $X \times Y := \{(x,y) \mid x \in X \land y \in Y\}$
					\item $\mathcal P(X) := \{Y \mid Y \subset X\}$
				\end{compactitem}
			\end{framed}
			
			Neben den offensichtlichen Mengengesetzen, wie dem Kommutaivgesetz, gibt es auch weniger 
			offensichtliche Gesetze, wie die Gesetze von de Morgan: F\"ur $X_1, X_2 \subset X$ gilt:
			\begin{compactitem}
				\item $X \backslash (X_1 \cup X_2) = (X \backslash X_1) \cap (X \backslash X_2)$
				\item $X \backslash (X_1 \cap X_2) = (X \backslash X_1) \cup (X \backslash X_2)$
			\end{compactitem}
			$\newline$
			
			Sind $X$ und $Y$ endliche Mengen, so gilt:
			\begin{compactitem}
				\item $|X \times Y| = |X| \cdot |Y|$
				\item $|\mathcal P(X)| = 2^{|X|}$
			\end{compactitem}
			
	\subsection{Abbildungen}
		\subsubsection{\"Uberblick \"uber Abbildungen}
			Eine Abbildung $f$ von eine Menge $X$ in einer Menge $Y$ ist eine Vorschrift, die jedem $x \in X$
			auf eindeutige Weise genau ein Element $f(x) \in Y$ zuordnet. Man schreibt dies als 
			\begin{equation*}
			f:
				\begin{cases}
					X \to Y \\ x \mapsto y
				\end{cases}
			\end{equation*}
			oder $f: X \to Y, x \mapsto y$ oder noch einfacher $f: X \to Y$. Dabei hei{\ss}t $X$ die
			Definitions- und $Y$ die Zielmenge von $f$. Zwei Abbildungen heißen gleich, wenn ihre
			Definitionsmengen und Zielmengen gleich sind und sie jedem $x \in X$ das selbe Element
			$y \in Y$ zuordnen. Die Abbildungen von $X$ nach $Y$ bilden wieder eine Menge, welche wir 
			mit \textbf{Abb($X$,$Y$)} bezeichnen. \\
			$\newline$
			
			Beispiele: \\
			\begin{compactitem}
				\item Abbildungen mit Zielmenge $\mathbb R$ nennt man Funktion: $f: \mathbb R \to \mathbb
				R, x \mapsto x^2$
				\item Abbildungen mit Zielmenge $\subset$ Definitionsmenge: $f: \mathbb R \to \mathbb
				R_{\le 0}, x \mapsto x^2$ \\
				$\to$ Diese Abbildungen sind verschieden, da sie nicht die selbe Zielmenge haben.
				\item $f: \{0,1\} \to \mathbb R, x \mapsto x^2$
				\item $f: \{0,1\} \to \mathbb R, x \mapsto x$ \\
				$\to$ Diese Funktionen sind gleich. Sie haben die gleichen Definitions- und Zielmengen 
				und sie ordnen jedem Element der Definitionsmenge das gleiche Element der Zielmenge zu.
			\end{compactitem}
			$\newline$
			
			Beispiele: \\
			\begin{compactitem}
				\item auf jeder Menge $X$ gibt es die identische Abbildung (Identit\"at) \\ $id: X \to X, x 
				\mapsto x$
				\item allgemein kann man zu jeder Teilmenge $A \subset X$ die Inklusionsabbildung zuordnen
				$\iota_A: A \to X, x \mapsto x$
				\item zu je zwei Mengen $X$ und $Y$ und einem festen $y_0 \in Y$ gibt es die konstante
				Abbildung $c_{y_0}: X \to Y x \mapsto y_0$
				\item zu jder Menge $X$ und Teilmenge $A \subset X$ definiert man die charackteristische 
				Funktion\\ $\chi_A: X \to \mathbb R,
				\begin{cases}
					x \mapsto 1 \quad(x \in A) \\ x \mapsto 0 \quad(x \notin A)
				\end{cases}
				$
				\item zu jeder Menge $X$ gibt es die Abbildung \\ $f: X \times X \to \mathbb R, (x,y) \mapsto
				\delta_{x,y} \begin{cases} 1 \quad (x=y) \\ 0 \quad (x \neq y) \end{cases}$
			\end{compactitem}
\end{document}
