\section{Koordinatendarstellung linearer Abbildungen}

Seien $V,W$ endlichdimensionale $K$-Vektorräume mit den Basen $B=(x_1,...,x_n)$ und $C=(y_1,...,y_m)$.

\begin{definition}[darstellende Matrix]
	Sei $f\in \Hom_K(V,W)$. Für $j=1,...,n$ schreiben wir $f(x_j)=\sum_{
		i=1}^m a_{ij}y_i$ mit eindeutig bestimmten $a_{ij}\in K$. Die Matrix $M_C^B(f)=(a_{ij})\in \Mat_{m\times n}(K)$ 
	heißt die \begriff{darstellende Matrix} von $f$ bezüglich der Basen $B$ und $C$.
\end{definition}

\begin{proposition}
	\proplbl{3_6_2}
	Sei $f\in \Hom_K(V,W)$. Die darstellende Matrix $M_C^B(f)$ ist die eindeutig bestimmte Matrix $A\in 
	\Mat_{n\times m}(K)$, für die das folgenden Diagramm kommutiert: \\
	\begin{center}\begin{tikzpicture}
		\matrix (m) [matrix of math nodes,row sep=3em,column sep=4em,minimum width=2em]
		{K^n & K^m \\ V & W \\};
		\path[-stealth]
		(m-1-1) edge node [left] {$\Phi_B$} (m-2-1)
		edge node [above] {$f_A$} (m-1-2)
		(m-2-1) edge node [below] {$f$} (m-2-2)
		(m-1-2) edge node [right] {$\Phi_C$} (m-2-2);
		\end{tikzpicture}\end{center}
	d.h. $f\circ \Phi_B=\Phi_C\circ f_A$.
\end{proposition}
\begin{proof}
	Sei zunächst $A=M_C^B(f)$. Für $j=1,...,n$ ist $\Phi_C(f_A(e_j))=\Phi_C((a_{1j},...,a_{mj})^t)=\sum_{i=1}^m 
	a_{ij}\cdot y_i=f(x_j)=f(\Phi_B(e_j))$, also $\Phi_C\circ f_A=f \circ \Phi_B$. \\
	Sei umgekehrt $A\in \Mat_{m\times n}(K)$ mit $\Phi_C\circ f_A=f\circ\Phi_B$. Da $\Phi_B$ und $\Phi_C$ Isomorphismen 
	sind, ist $f_A$ eindeutig bestimmt: $f_A=\Phi_C^{-1}\circ f \circ \Phi_B$ und deshalb auch $A$ (\propref{3_5_12}).
\end{proof}

\begin{conclusion}
	\proplbl{3_6_3}
	Die Abbildung $M_C^B$: $\Hom_K(V,W)\to \Mat_{m\times n}(K)$ ist ein Isomorphismus von $K$-Vektorräumen.
\end{conclusion}
\begin{proof}
	Definiere $A$: $\Hom_K(V,W)\to \Mat_{m\times n}(K)$ mit $f\mapsto \Phi_C^{-1}\circ f \circ \Phi_B$. $A(f)=F_{m\times n}
	(M_C^B(f))$, also $A=F_{m\times n}\circ M_C^B$. Die Abbildung ist bijektiv, da $\Phi_B$ und $\Phi_C$ bijektiv sind, 
	und linear, da $\Phi_B$ und $\Phi_C$ linear sind (\propref{3_5_8}). Also ist $A$ ein Isomorphismus. Da auch $F_{m\times n}^{-1}$ ein 
	Isomorphismus ist (\propref{3_5_12}), ist folglich auch $M_C^B=F_{m\times n}^{-1}\circ A$.
\end{proof}

\begin{lemma}
	\proplbl{3_6_4}
	Sei $U$ ein weiterer $K$-Vektorraum mit endlicher Basis $D$. Für $f\in \Hom_K(V,W)$ und $g\in \Hom_K(U,V)$ ist 
	\begin{align}
		M_C^B(f)\cdot M_B^D(g)=M_C^D(f\circ g)\notag
	\end{align}
\end{lemma}
\begin{proof}
	Sei $r=\dim_K(U)$ und $A=M_B^D(g)$ und $B=M_C^B(f)$. Nach \propref{3_6_2} kommutieren die beiden kleinen Quadrate in: 
	\begin{center}\begin{tikzpicture}
		\matrix (n) [matrix of math nodes,row sep=3em,column sep=4em,minimum width=2em]
		{K^r & K^n & K^m \\ U & V & W \\};
		\path[-stealth]
		(n-1-1) edge node [left] {$\Phi_D$} (n-2-1)
		edge node [above] {$f_A$} (n-1-2)
		(n-2-1) edge node [below] {$g$} (n-2-2)
		(n-2-2) edge node [below] {$f$} (n-2-3)
		(n-1-2) edge node [right] {$\Phi_B$} (n-2-2)
		(n-1-2) edge node [above] {$f_B$} (n-1-3)
		(n-1-3) edge node [right] {$\Phi_C$} (n-2-3);
		\end{tikzpicture}\end{center}
	Deshalb kommutiert auch:
	\begin{center}\begin{tikzpicture}
		\matrix (m) [matrix of math nodes,row sep=3em,column sep=4em,minimum width=2em]
		{K^r & K^m \\ U & W \\};
		\path[-stealth]
		(m-1-1) edge node [left] {$\Phi_D$} (m-2-1)
		edge node [above] {$f_B \circ f_A$} (m-1-2)
		(m-2-1) edge node [below] {$f\circ g$} (m-2-2)
		(m-1-2) edge node [right] {$\Phi_C$} (m-2-2);
		\end{tikzpicture}\end{center}
	Die Eindeutigkeit in \propref{3_6_2} impliziert deshalb, dass $F_{m\times n}(M_C^B(f))\circ F_{r\times m}(M_B^D(g))=F_{r\times n}
	(M_C^D(f\circ g))$. Da $F_{r\times n}$ injektiv ist, folgt mit \propref{3_5_11} und \propref{3_5_12} $M_C^B(f)\cdot M_B^D(g)=M_C^D(f\circ g)$.
\end{proof}

\begin{conclusion}
	\proplbl{3_6_5}
	Sei $f\in \Hom_K(V,W)$. Genau dann ist $f$ ein Isomorphismus, wenn $m=n$ und $M_C^B(f)=\GL_(K)$. In 
	diesem Fall ist $M_B^C(f^{-1})=M_C^B(f)^{-1}$.
\end{conclusion}
\begin{proof}
	Sei $A=M_C^B(f)$. Nach \propref{3_6_2} ist $f$ genau dann ein Isomorphismus, wenn $f_A$ einer ist, und in diesem Fall ist $m=n$. Zudem ist 
	$f_A$ genau dann ein Isomorphismus, wenn $A\in \GL_(K)$ (\propref{3_5_12}). Ist $f$ ein Isomorphismus, so ist $M_B^C(f^{-1})\cdot 
	M_C^B(f)=M_C^C(f^{-1}\circ f)=\mathbbm{1}_n$, also $M_B^C(f^{-1})=M_C^B(f)^{-1}$ (\propref{3_6_4}).
\end{proof}

\begin{conclusion}
	\proplbl{3_6_6}
	Die Abbildung $M_B:=M_B^B$: $\End_K(V)\to \Mat_n(K)$ ist ein Ringisomorphismus, der $\Aut_K(V)$ auf 
	$\GL_(K)$ abbildet.
\end{conclusion}
\begin{proof}
	\propref{3_6_3}, \propref{3_6_4}, \propref{3_6_5}
\end{proof}

\begin{definition}[Transformationsmatrix]
	Sind $B$ und $B'$ Basen von $V$, so nennt man $T_{B'}^B:=M_{B'}^B(\id_V)\in 
	\GL_(K)$ die \begriff{Transformationsmatrix} des Basiswechsels von $B$ nach $B'$.
\end{definition}

\begin{remark}
	Nach \propref{3_6_2} ist $T_{B'}^B$, also die Matrix $A$, die $f_A=\Phi_B^{-1}\circ \Phi_B$ 
	erfüllt. Ist $x=\Phi_B^{-1}(v)\in K^n$ der Koordinatenvektor von $v$ bezüglich $B$, so ist $T_{B'}^B\cdot 
	x=f_{T_{B'}^B}(x)=(\Phi_{B'}\circ \Phi_B)(\Phi_B^{-1}(v))=\Phi_{B'}^{-1}(v)$ der Koordinatenvektor von $v$ 
	bezüglich $B'$.
\end{remark}

\begin{proposition}[Transformationsformel]
	Seien $B,B'$ Basen von $V$ und $C,C'$ Basen von $W$. Für $f\in \Hom_K(V,W)$ ist 
	\begin{align}
		M_{C'}^B(f)=T_{C'}^C\cdot M_C^B(f)\cdot (T_{B'}^B)^{-1}\notag
	\end{align}
\end{proposition}
\begin{proof}
	$f=\id_W\circ f \circ \id_V$ mit den Basen $B',B,C,C'$ und erhält (\propref{3_6_4}) $M_{C'}^{B'}(f)=M_{C'}^C(\id_W)\cdot M_C^B(f)\cdot
	M_B^{B'}(\id_V)=T_{C'}^C\cdot M_C^B(f)\cdot T_B^{B'}$ und $T_B^{B'}=M_B^{B'}(\id_V)=M_B^{B'}(\id_V^{-1})=M_{B'}^B(\id_V)^
	{-1}=(T_{B'}^B)^{-1}$ nach \propref{3_6_5}.
\end{proof}

\begin{conclusion}
	Sind $B$ und $B'$ Basen von $V$ und $f\in \End_K(V)$, so gilt $M_{B'}(f)=T_{B'}^B \cdot M_B(f)
	\cdot (T_{B'}^B)^{-1}$.
\end{conclusion}