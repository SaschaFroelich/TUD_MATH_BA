\section{Linearkombinationen}

Sei $V$ ein $K$-Vektorraum.

\begin{definition}[Linearkombination]
	\begin{itemize}
		\item Sei $n \in \mathbb N_0$. Ein $x \in V$ ist eine \begriff{Linearkombination} eines $n$-Tupels $(x_1,...,x_n)$ von 
		Elementen von $V$, wenn es $\lambda_1,...,\lambda_n \in K$ gibt mit $x=\lambda_1\cdot x_1,...,\lambda_n \cdot
		x_n$. Der Nullvektor ist stets eine Linearkombination von $(x_1,...,x_n)$ auch wenn $n=0$.
		\item Ein $x\in V$ ist eine Linearkombination einer Familie $(x_i)$ von Elementen von $V$, wenn es $n \in \mathbb
		N_0$ und $i_1,...,i_n \in I$ gibt, für die $x$ Linearkombination von $(x\cdot i_1,...,x\cdot i_n)$ ist.
		\item Die Menge aller $x \in V$, die Linearkombination von $\mathcal F=(x_i)$ sind, wird mit $\Span_K(\mathcal F)$ 
		bezeichnet.
	\end{itemize}
\end{definition}

\begin{remark}
	\begin{itemize}
		\item Offenbar hängt die Menge der Linearkombinationen von $(x_1,...,x_n)$ nicht von der Reihenfolge der $x_i$ ab. 
		Wegen (V2)(ii) hängt sie sogar nur von der Menge $\{x_1,...,x_n\}$ ab.
		\item Deshalb stimmt 2. für endliche Familien $(x_1,...,x_n)$ mit 1. überein.
		\item Auch die Menge der Linearkombinationen einer Familie $\mathcal F=(x_1,...,x_n)$ hängt nur von der Menge $X=
		\{x_i \mid i \in I\}$ ab. Man sagt deshalb auch, $x$ ist Linearkombination von $X$ und schreibt $\Span_K(X)=\Span_K(
		\mathcal F)$, also $\Span_K(X)=\{\sum _{i=1}^n \lambda_i\cdot n_i \mid n \in \mathbb N_0, x_i \in X, \lambda_1,
		...,\lambda_n \in K\}$. Nach Definition in $0 \in \Span_K(X)$ auch für $X=\emptyset$.
		\item Wie schon bei Polynomen schreibt man hier gerne formal unendliche Summen $x=\sum_{i \in I} \lambda_i
		\cdot x_i$, bei denen nur endlich viele $\lambda_i$ von 0 verschieden sind.
	\end{itemize}
\end{remark}

\begin{lemma}
	Für jede Teilmenge $X \subseteq V$ ist $\Span_K(X)$ ein Untervektorraum von $V$.
\end{lemma}
\begin{proof}
	\begin{itemize}
		\item Sei $W=\Span_K(X)$. Nach Definition ist $0 \in W$, insbesondere $W\neq\emptyset$
		\item (UV1): Sind $x,y \in W$, also $x=\lambda_1\cdot x+...+\lambda_n\cdot x_n$ und $y=\mu_1\cdot x+...+
		\mu_n\cdot x_n$, so ist $x+y=(\lambda_1+\mu_1)x_1+...+(\lambda_n+\mu_n)x_n \in W$
		\item (UV2): Ist $\lambda \in K$ und $x \in W$, so ist $\lambda x=\lambda\cdot\sum_{i=1}^n \lambda_i\cdot x_i=
		\sum_{i=1}^n (\lambda\cdot\lambda_i)x_i \in W$
	\end{itemize}
\end{proof}

\begin{proposition}
	Für jede Teilmenge $X \subseteq V$ ist $\Span_K(X)=\langle X\rangle$.
\end{proposition}
\begin{proof}
	\begin{itemize}
		\item $\Span_K(X)$ ist Untervektorraum von $V$, der wegen $x=x\cdot 1$ die Menge $X$ enthält, und $\langle X\rangle$ ist der kleinste solche.
		\item Ist $W\subseteq V$ ein Untervektorraum von $V$, der $X$ enthält, so enthält er auch wegen (UV2) alle Elemente der Form 
		$\lambda\cdot x$, und wegen (UV1) dann auch alle Linearkombinationen aus $X$. Insbesondere gilt dies auch für $W=\langle X\rangle$
	\end{itemize}
\end{proof}

\begin{remark}
	Wir erhalten $\Span_K(X)=\langle X\rangle$ auf 2 verschiedenen Wegen. Erstens "'von oben"' als Schnitt über alle Untervektorraum 
	von $V$, die $X$ enthalten und zweitens "'von unten"' als Menge der Linearkombinationen. Man nennt $\Span_K(X)$ auch den 
	von $X$ aufgespannten Untervektorraum oder die lineare Hülle von $X$.
\end{remark}

\begin{example}
	\proplbl{2_2_6}
	\begin{itemize}
		\item Sei $V=K^n$ der Standardraum. Für $i=1,...,n$ sei $e_i=(\delta_{i,1},...,\delta_{i,n})$, also $e_1=(1,0,...0)$, 
		$e_2=(0,1,0,...,0),...,e_n=(0,...,1)$. Für $x=(x_1,...,x_n) \in V$ ist $x=\sum_{i=1}^n x_i\cdot e_1$, folglich 
		$\Span_K(e_1,..,e_n)=V$. Insbesondere ist $K^n$ eindeutig erzeugt. Man nennt $(e_1,...,e_n)$ die Standardbasis des 
		Standardraums $K^n$.
		\item Sei $V=K[X]$ Polynomring über $K$. Da $f=\sum_{i=1}^n a_i\cdot X^i$ ist $\Span_K((X^i)_{i \in I})=K[X]$. 
		Genauer ist $\Span_K(1,X,X^2,...,X^n)=K[X]_{\le n}$. Tatsächlich ist der $K$-Vektorraum $K[X]$ nicht endlich erzeugt. Sind 
		$f_1,...,f_r \in K[X]$ und ist $d=\max\{\deg(f_1),...,\deg(f_r)\}$, so sind $f_1,...,f_r \in K[X]_{\le d}$ und somit 
		$\Span_K(f_1,...,f_r) \subseteq K[X]_{\le d}$, aber es gibt Polynome, deren Grad größer $d$ ist.
		\item Für $x \in V$ ist $\langle x\rangle=\Span_K(x)=K\cdot x$. Im Fall $K=\mathbb R$, $V=\mathbb R^3$, $x\neq 0$ ist dies eine 
		Ursprungsgerade.
		\item Im $\mathbb R$-Vektorraum $\mathbb C$ ist $\Span_{\mathbb R}(1)=\mathbb R\cdot 1=\mathbb R$, aber im $\mathbb C$-Vektorraum 
		$\mathbb C$ ist $\Span_{\mathbb C}(1)=\mathbb C\cdot 1=\mathbb C$
	\end{itemize}
\end{example}

\begin{definition}[linear (un)abhängig]
	\begin{itemize}
		\item Sei $n\in \mathbb N_0$. Ein $n$-Tupel $(x_1,...,x_n)$ von Elementen von $V$ ist \begriff{linear abhängig}, wenn es 
		$\lambda_1,...,\lambda_n \in K$ gibt, die nicht alle 0 sind und $\lambda_1\cdot x_1+...+\lambda_n\cdot x_n=0$ (*) 
		erfüllen. Andernfalls heißt das Tupel \begriff{linear unabhängig}.
		\item Eine Familie $(x_i)$ von Elementen von $V$ ist linear abhängig, wenn es $n\in \mathbb N_0$ und paarweise 
		verschiedene $i_1,...,i_n \in I$ gibt, für die $(x_{i_1},...,x_{i_n})$ linear abhängig ist. Andernfalls linear 
		unabhängig.
	\end{itemize}
\end{definition}

\begin{mathematica}[Lineare Unabhängigkeit]
	In WolframAlpha kann man mittels 
	\begin{align}
		\texttt{linear independence (1,2,3), (4,5,6)}\notag
	\end{align}
	überprüfen, ob die Vektoren $(1,2,3)^T$ und $(4,5,6)^T$ linear unabhängig sind.
\end{mathematica}

\begin{remark}
	\begin{itemize}
		\item Offenbar hängt die Bedingung (*) nicht von der Reihenfolge der $x_1,...,x_n$ ab und ist $(x_1,...,x_k)$ linear 
		abhängig für ein $k \le n$, so ist auch $(x_1,...,x_n)$ linear abhängig. Deshalb stimmt die 2. Definition für 
		endliche Familien mit der 1. überein und $(x_i)$ ist genau dann linear abhängig, wenn es eine endliche Teilmenge 
		$J \subseteq I$ gibt, für die $(x_j)$ linear abhängig ist.
		\item Eine Familie ist genau dann linear unabhängig, wenn für jede endliche Teilmenge $J\subseteq I$ und für jede 
		Wahl an Skalaren $(\lambda_i)_{i\in J}$ aus $\sum \lambda_i\cdot x_i=0$ schon $\lambda_i=0$ folgt, also wenn sich 
		der Nullvektor nur trivial linear kombinieren lässt. 
	\end{itemize}
\end{remark}

\begin{proposition}
	\proplbl{2_2_9}
	Genau dann ist $(x_i)$ linear abhängig, wenn es $i_0 \in I$ gibt mit $x_{i_0} \in \Span_K((x_i)_{i\in 
		I\backslash\{i_0\}})$. In diesem Fall ist $\Span_K((x_i)_{i\in I})=\Span_K((x_i)_{i\in I\backslash\{i_0\}})$.
\end{proposition}
\begin{proof}
	Es reicht, die Aussage für $I=\{1,...,n\}$ zu beweisen.
	\begin{itemize}
		\item Hinrichtung: Ist $(x_1,...,x_n)$ linear anhängig, so existieren $\lambda_1,...,\lambda_n$ mit $\sum_{i=1}^n 
		\lambda_i\cdot x_i=0$. oBdA. sei $\lambda_n\neq 0$. Dann ist $x_n=\lambda_n^{-1}\cdot\sum_{i=1}^{n-1} \lambda_i
		\cdot x_i=\sum_{i=1}^{n-1} \lambda_n^{-1}\cdot\lambda_i\cdot x_i \in \Span_K(x_1,...,x_n)$.
		\item Rückrichtung: oBdA. $i_0=n$, also $\sum_{i=0}^{n-1} \lambda_i\cdot x_i$. Mit $\lambda_n=-1$ ist $\sum
		_{i=1}^n \lambda_i\cdot x_i=0$, was zeigt, dass $(x_1,...,x_n)$ linear abhängig ist. \\
		Sei nun $x_n=\sum_{i=1}^{n-1} \lambda_i\cdot x_i \in \Span_K(x_1,...,x_{n-1})$. Wir zeigen, dass $\Span_K(x_1,...,
		x_{n-1})=\Span_K(x_1,...,x_n)$
		\begin{itemize}
			\item klar, da bei mehr Elementen die Anzahl der Linearkombinationen nicht abnimmt
			\item Ist $y=\sum_{i=1}^n \mu_i\cdot x_i \in \Span_K(x_1,...,x_n)$, so ist $y=\sum_{i=1}^{n-1} \mu_i+
			\mu_n\cdot \lambda_i\cdot x_i \in \Span_K(x_1,...,x_n)$
		\end{itemize}
	\end{itemize}
\end{proof}

\begin{proposition}
	\proplbl{2_2_10}
	Genau dann ist $(x_i)$ linear unabhängig, wenn sich jedes $x\in \Span_K((x_i))$ in eindeutiger Weise 
	als Linearkombination der $(x_i)$ schreiben lässt, d.h. $x=\sum_{i\in I} \lambda_i\cdot x_i=\sum_{i
		\in I} \lambda'_i\cdot x_i$, so ist $\lambda_i=\lambda'_i$
\end{proposition}
\begin{proof}
	Es reicht, die Aussage für $I=\{1,...,n\}$ zu beweisen.
	\begin{itemize}
		\item Hinrichtung:  Ist $(x_,...,x_n)$ linear unabhängig und $x=\sum_{i\in I} \lambda_i\cdot x_i=\sum_{i\in I}
		\lambda'_i\cdot x_i$, so folgt daraus $\sum_{i\in I} (\lambda_i-\lambda'_i)x_i=0$ wegen der linearen 
		Unabhängigkeit der $x_i$, dass $\lambda_i=\lambda'_i=0$\\
		\item Rückrichtung: Lässt sich jedes $x\in \Span_K(x_1,...,x_n)$ in eindeutiger Weise als Linearkombination der $x_i$ schreiben, 
		so gilt dies insbesondere für $x=0$. Ist also $\sum_{i=1}^n \lambda_i\cdot x_i=0$, so folgt schon $\sum_{
			i=1}^n 0\cdot x_i=0$ schon $\lambda_i=0$
	\end{itemize}
\end{proof}

\begin{example}
	\begin{itemize}
		\item Die Standardbasis $(e_1,...,e_n)$ des $K^n$ ist linear unabhängig. Es ist $\sum_{i=1}^n \lambda_i\cdot 
		e_i=(\lambda_1,...,\lambda_n)$
		\item Im $K$-Vektorraum $K[X]$ sind die Monome $(X^i)$ linear unabhängig.
		\item Ein einzelner Vektor $x\in V$ ist genau dann linear abhängig, wenn $x=0$.
		\item Ein Paar $(x_1,x_2)$ von Elementen von $V$ ist linear abhängig, wenn es ein skalares Vielfaches des anderen ist, also z.B. $x_1=
		\lambda\cdot x_2$.
		\item Im $\mathbb R$-Vektorraum $\mathbb R^2$ sind die beiden Vektoren $(1,2)$ und $(2,1)$ linear unabhängig. \\
		Im $\mathbb Z\backslash 3\mathbb Z$-Vektorraum $(\mathbb Z\backslash 3\mathbb Z)^2$ sind diese Vektoren linear unabhängig, da 
		$x_1+x_2=(1,2)+(2,1)=(3,3)=(0,0)=0$. 
		\item Im $\mathbb R$-Vektorraum $\mathbb C$ ist $(1,i)$ linear unabhängig, aber im $\mathbb C$-Vektorraum $\mathbb C$ ist $(1,i)$ 
		linear abhängig, denn $\lambda_1\cdot 1+\lambda_2\cdot i =0$ für $\lambda_1=1$ und $\lambda_2=i$.
	\end{itemize}
\end{example}

\begin{remark}
	\begin{itemize}
		\item Ist $x_{i_0}=0$, ist $(x_i)$ linear abhängig: $1\cdot x_{i_0}=0$
		\item Gibt es $i,j\in I$ mit $i\neq j$, aber $x_i=x_j$, so ist $(x_i)$ linear abhängig: $x_i-x_j=0$
		\item Dennoch sagt man auch "'die Teilmenge $X\subseteq V$ ist linear abhängig"' und meint damit, dass die Familie $(x_x)
		_{x\in X}$ linear abhängig ist, d.h. es gibt ein $n\in \mathbb N_0$, $x_1,...,x_n \in X$ paarweise verschieden, mit 
		$\sum_{i=1}^n \lambda_i\cdot x_i=0$.
	\end{itemize}
\end{remark}