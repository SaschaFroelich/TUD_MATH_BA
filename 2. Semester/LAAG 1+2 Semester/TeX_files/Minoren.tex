\section{Minoren}

Seien $m,n\in \mathbb N$.

\begin{definition}[adjungierte Matrix]
	Sei $A=(a_{ij})\in \Mat_n(R)$. Für $i,j\in \{1,...,n\}$ definieren wir die $n\times n$-Matrix: \\
	\begin{align}
		A_{ij}=\begin{pmatrix}
		a_{11} & ... & a_{1,j-1} & 0 & a_{1,j+1} & ... & a_{1n} \\
		\vdots & \ddots & \vdots & \vdots & \vdots & \ddots & \vdots \\
		a_{i-1,1} & ... & a_{i-1,j.1} & 0 & a_{i-1,j+1} & ... & a_{i-1,n} \\
		0 & ... & 0 & 1 & 0 & ... & 0 \\
		a_{i+1,1} & ... & a_{i+1,j.1} & 0 & a_{i+1,j+1} & ... & a_{i+1,n} \\
		\vdots & \ddots & \vdots & \vdots & \vdots & \ddots & \vdots \\
		a_{n1} & ... & a_{n,j-1} & 0 & a_{n,j+1} & ... & a_{nn} \\
		\end{pmatrix}\notag
	\end{align}
	die durch Ersetzen der $i$-ten Zeile und der $j$-ten Spalte durch $e_j$ aus $A$ hervorgeht, sowie die $(n-1)\times(n-1)$-
	Matrix: \\
	\begin{align}
		A'_{ij}=\begin{pmatrix}
		a_{11} & ... & a_{1,j-1} & a_{1,j+1} & ... & a_{1n} \\
		\vdots & \ddots & \vdots & \vdots & \ddots & \vdots \\
		a_{i-1,1} & ... & a_{i-1,j.1} & a_{i-1,j+1} & ... & a_{i-1,n} \\
		a_{i+1,1} & ... & a_{i+1,j.1} & a_{i+1,j+1} & ... & a_{i+1,n} \\
		\vdots & \ddots & \vdots & \vdots & \ddots & \vdots \\
		a_{n1} & ... & a_{n,j-1} & a_{n,j+1} & ... & a_{nn} \\
		\end{pmatrix}\notag
	\end{align}
	die durch Streichen der $i$-ten Zeile und der $j$-ten Spalten entsteht. Weiterhin definieren wir die zu $A$ \begriff{adjungierte Matrix} 
	als $A^\#=(a_{ij}^\#)\in \Mat_n(R)$, wobei $a_{ij}^\#=\det(A_{ji})$.
\end{definition}

\begin{lemma}
	\proplbl{4_3_2}
	Sei $A\in \Mat_n(R)$ mit Spalten $a_1,...,a_n$. Für $i,j\in \{1,..,n\}$ gilt:
	\begin{itemize}
		\item $\det(A_{ij})=(-1)^{i+j}\cdot \det(A'_{ij})$
		\item $\det(A_{ij})=\det(a_1,...,a_{j-1},e_i,a_{j+1},...,a_n)$
	\end{itemize}
\end{lemma}
\begin{proof}
	\begin{itemize}
		\item Durch geeignete Permutation der ersten $i$ Zeilen und der ersten $j$ Zeilen erhält man 
		\begin{align}
			\det(A_{ij})&=(-1)^{(i-1)+
				(j-1)} \cdot \det(\begin{pmatrix}1&0&...&0 \\ 0 & \; & \; & \; \\ \vdots & \; & A'_{ij} & \; \\ 0 & \; & \; & \; \\ \end{pmatrix})\notag \\
			&\overset{\propref{4_2_9}}{=}(-1)^{i+j}\cdot \det(\mathbbm{1}_n)\cdot \det(A'_{ij})\notag
		\end{align}
		\item Man erhält $A_{ij}$ aus $(a_1,...,e_i,...,a_n)$ durch elementare Spaltenumformungen vom Typ II.
	\end{itemize}
\end{proof}

\begin{proposition}
	\proplbl{4_3_3}
	Für $A\in \Mat_n(R)$ ist 
	\begin{align}
		A^\#\cdot A=A\cdot A^\#=\det(A)\cdot \mathbbm{1}_n
	\end{align}
\end{proposition}
\begin{proof}
	\begin{align}
		(A^\#A)_{ij}&=\sum_{k=1}^n a^\#_{ik}\cdot a_{kj} \notag \\
		&=\sum_{k=1}^n a_{kj}\cdot \det(A_{kj}) \notag \\
		&\overset{\propref{4_3_2}}{=}\sum_{k=1}^n a_{kj}\cdot \det(a_1,...,a_{i-1},a_j,a_{i+1},...,a_n) \notag \\
		&=\det(a_1,...,a_{i-1},\sum_{k=1}^n a_{kj}e_k,a_{i+1},...,a_n) \notag \\
		&= \det(a_1,...,a_{i-1},a_j,a_{i+1},...,a_n) \notag \\
		&=\delta_{ij}\cdot \det(A) \notag \\
		&=(\det(A)\cdot \mathbbm{1}_n)_{ij} \notag
	\end{align}
	Analog bestimmt man die Koeffizienten von $AA^\#$, wobei man 
	$\det(A_{jk})=\det(A_{jk}^t)=\det((A^t)_{kj})$ benutzt.
\end{proof}

\begin{conclusion}
	\proplbl{4_3_4}
	Es ist $\GL_n(R)=\{A\in \Mat_n(R) \mid \det(A)\in R^{\times}\}$ und für $A\in \GL_n(R)$ ist $A^{-1}=
	\frac{1}{\det(A)}\cdot A^\#$.
\end{conclusion}
\begin{proof}
	\propref{4_3_3} und \propref{4_2_12}
\end{proof}

\begin{proposition}[\person{Laplace}'scher Entwicklungssatz]
	Sei $A=(a_{ij})\in \Mat_n(R)$. Für jedes $i,j\in \{1,..,n\}$ gilt die 
	Formel für die Entwicklung nach der $i$-ten Zeile: \\
	\begin{align}
		\det(A)=\sum_{j=1}^n (-1)^{i+j}\cdot a_{ij}\cdot \det(A'_{ij})\notag
	\end{align}
	Gleiches gilt auch für Spalten.
\end{proposition}
\begin{proof}
	Nach \propref{4_3_3} ist
	\begin{align}
		\det(A)=(AA^\#)_{ij}&=\sum_{j=1}^n a_{ij}\cdot a^\#_{ij} \notag \\
		&= \sum_{j=1}^n a_{ij}\cdot \det(A_{ij}) \notag \\
		&=\sum_{j=1}^n a_{ij}\cdot (-1)^{i+j}\cdot \det(A'_{ij}) \notag
	\end{align}
	Analog auch für Spalten.
\end{proof}

\begin{proposition}[\person{Cramer}'sche Regel]
	Sei $A\in \GL_n(R)$ mit Spalten $a_1,...,a_n$ und sei $b\in R^n$. Weiter sei 
	$x=(x_1,...,x_n)^t\in R^n$ die eindeutige Lösung des Linearen Gleichungssystems $Ax=b$. Dann ist für $i=1,...,n$ 
	\begin{align}
		x_i=\frac{\det(a_1,...,a_{i-1},b,a_{i+1},...,a_n)}{\det(A)}\notag
	\end{align}
\end{proposition}
\begin{proof}
	\begin{align}
		x_i&=(A^{-1}b)_i \notag \\
		&=\sum_{j=1}^n (A^{-1})_{ij}\cdot b_j \notag \\
		&\overset{\propref{4_3_4}}{=}\frac{1}{\det(A)}\cdot \sum_{j=1}^n a^\#_{ij}\cdot b_j  \notag \\
		&\overset{\propref{4_3_2}}{=} \frac{1}{\det(A)}\cdot \sum_{j=1}^n b_j\cdot\det(a_1,...,a_{i-1},e_i,a_{i+1},...,a_n) \notag \\
		&=\frac{1}{\det(A)}\cdot \det(a_1,...,a_{i-1},b_j,a_{i+1},...,a_n) \notag
	\end{align}
\end{proof}

\begin{definition}[Minor]
	Sei $A=(a_{ij})\in \Mat_{m\times n}(R)$ und $1\le r \le m$, $1\le s \le n$. Eine $r\times s$-
	Teilmatrix von $A$ ist eine Matrix der Form $(a_{i\mu,jv})_{\mu,v}\in \Mat_{r\times s}(R)$ mit $1\le i_1<...<i_r\le m$ 
	und $1\le j_1<...<j_s\le n$. Ist $A'$ eine $r\times r$-Teilmatrix von $A$, so bezeichnet man $\det(A')$ als einen 
	$r$-\begriff{Minor} von $A$.
\end{definition}

\begin{example}
	Ist $A\in \Mat_n(R)$ und $i,j\in \{1,...,n\}$, so ist $A'_{ij}$ eine Teilmatrix und $\det(A'_{ij})=(-1)^{i+j}
	\cdot a^\#_{ji}$ ein $(n-1)$-Minor von $A$.
\end{example}

\begin{proposition}
	Sei $A\in \Mat_n(R)$ und $r\in \mathbb N$. Genau dann ist $\rk(A)\ge r$, wenn es eine $r\times r$-
	Teilmatrix $A'$ von $A$ mit $\det(A^{\prime})\neq 0$ gibt.
\end{proposition}
\begin{proof}
	\begin{itemize}
		\item Hinrichtung: Ist $\rk(A)\ge r$, so hat $A$ $r$ linear unabhängige Spalten $a_1,...,a_r$. Die Matrix $\tilde A=(a_1,...,a_r)$ 
		hat den Rang $r$ und deshalb $r$ linear unabhängige Zeilen $\widetilde{a_1},...,\widetilde{a_r}$. Die $r\times r$-Matrix $A$ hat 
		dann Rang $r$, ist also invertierbar, und $\det(A)\neq 0$.
		\item Rückrichtung: Ist $A'$ eine $r\times r$-Teilmatrix von $A$ mit $\det(A')\neq 0$, so ist $\rk(A)\ge \rk(A')=r$.
	\end{itemize}
\end{proof}

\begin{conclusion}
	Sei $A\in \Mat_{m\times n}(K)$. Der Rang von $A$ ist das größte $r\in \mathbb N$, für das 
	$A$ einen von Null verschiedenen $r$-Minor hat.
\end{conclusion}