\section{Metrische Räume}
\begin{definition}[Metrik]
	Sei $X$ Menge, Abbildung $d:X\times X\rightarrow \mathbb{R}$ heißt \begriff{Metrik} auf $X$, falls $\forall x,y,z\in X$:
	\begin{enumerate}[label={\alph*)}]
		\item $d(x,y) = 0 \Leftrightarrow x=y$
		\item $d(x,y) = d(y,x)$ \begriff[Metrik!]{Symmetrie}\index{Symmetrie!Metrik}
		\item $d(x,z)\le d(x,y) + d(y,z)$ \begriff{Dreiecksungleichung}[!Metrik]
	\end{enumerate}

	$(X,d)$ heißt \begriff{metrischer Raum}.
\end{definition}
\stepcounter{theorem}
\begin{example}
	\begriff{Diskrete Metrik} auf bel. Menge $X$ ist \[ d(x,y) = \begin{cases}0& x=y \\ 1 & x\neq y \end{cases} \] ist offenbar Metrik.
\end{example}
\begin{example}
	Sei $(X,d)$ metrischer Raum, $Y\subset X$\\
	$\Rightarrow (Y,\tilde{d})$ ist metrischer Raum mit \begriff{induzierte Metrik} $\tilde{d}(x,y) := d(x,y)\,\forall x,y\in X$.
\end{example}

\begin{definition}[Norm]
	Sei $X$ Vektorraum über $K=\mathbb{R}$ bzw. $K=\mathbb{C}$.
	
	Abbildung \mathsymbol{.}{$\Vert.\Vert$}$: X\rightarrow\mathbb{R}$ heißt \begriff{Norm} auf $X$, falls $\forall x,y\in X$
	\begin{enumerate}[label={\alph*)}]
		\item $\Vert x\Vert = 0$ \gls{gdw} $x = 0$
		\item \label{norm_2} $\Vert \lambda\cdot x\Vert = |\lambda| \cdot \Vert x \Vert\,\forall \lambda\in K$ (\begriff{Homogenität})
		\item \label{norm_3} $\Vert x + y\Vert \le \Vert x \Vert + \Vert y \Vert$ \begriff{Dreiecksungleichung}[!Vektorraum]
	\end{enumerate}

	$(X,\Vert . \Vert)$ heißt \begriff{normierter Raum}
\end{definition}
\begin{definition}[Halbnorm]
	$\Vert . \Vert:X\rightarrow\mathbb{R}_{\ge0}$ heißt \begriff{Halbnorm}, falls nur \ref{norm_2} und \ref{norm_3} gelten.
\end{definition}
\begin{proposition}
	Sei $(X,\Vert .\Vert)$ normierter Raum.\\
	$\Rightarrow X$ ist metrischer Raum mit Metrik $d(x,y):=\Vert x - y \Vert\,\forall x,y\in X$.
\end{proposition}
\begin{example}
	\label{norm_r}
	Man hat u.a. folgende Normen auf $\mathbb{R}^n$:
	\begin{description}
		\item[\begriff{$p$-Norm}] $\vert x\vert_p:=\left(\sum_{i=1}^n |x_i|^p\right)^\frac{1}{p}\;(1\le p<\infty)$
		\item[\begriff{Maximum-Norm}] $|x|_\infty :=\max\{|x_i| \mid i=1,\dots,n\}$
	\end{description}

	Standardnorm im $\mathbb{R}^n: \vert \cdot \vert:=\vert \cdot \vert_{p=2}$ heißt \begriff{euklidische Norm}
\end{example}
\begin{definition}[Skalarprodukt]
	$\langle x,y\rangle:=\sum_{i=1}^n x_i y_i$ heißt \begriff{Skalarprodukt}[!$\mathbb{R}$] (\begriff{inneres Produkt}) von $x,y\in\mathbb{R}^n$.
	
	Offenbar ist $\langle x,x\rangle = |x|^2\,\forall x\in\mathbb{R}^n$ (\highlight{ausschließlich für Euklidische Norm})\\
	Man hat $|\langle x,y\rangle | \le |x|\cdot |y|\,\forall x,y\in\mathbb{R}^n$ (\begriff{\person{Cauchy}-\person{Schwarz}'sche Ungleichung})
\end{definition}
\begin{example}
	$X=\mathbb{C}^n$ ist Vektorraum über $\mathbb{C}$, $x=(x_1,\dotsc,x_n)\in\mathbb{C}^n, x_i\in\mathbb{C}$.
	
	Analog zu \ref{norm_r} sind $\vert\cdot\vert_p$ und $\vert\cdot\vert_\infty$ Normen auf $\mathbb{C}^n$
	
	$\langle x,y\rangle :=\sum_{i=1}^n \overline{x_i} y_i \,\forall x,y\in\mathbb{C}$ heißt \begriff{Skalarprodukt}[!$\mathbb{C}$] von $x,y\in\mathbb{C}^n$.
	
	$x,y\in\mathbb{R}^n (\mathbb{C}^n)$ heißen \begriff{orthogonal}, falls $\langle x,y\rangle = 0$.
\end{example}
\begin{example}
	Sei $M$ beliebige Menge, $f:M\rightarrow \mathbb{R}$.
	\begin{itemize}
		\item $\Vert f \Vert :=\sup\{ \vert f(x)\vert \mid x\in M\}$ \begriff{Supremumsnorm}
		\item \mathsymbol{B}{$B$}$(M):=\{ f:M\rightarrow \mathbb{R} \mid\; \Vert f \Vert < \infty \}$ \begriff{Menge der beschränkten Funktionen}
	\end{itemize}
\end{example}
\stepcounter{theorem}
\stepcounter{theorem}
\begin{definition}
	Normen $\Vert .\Vert_1, \Vert .\Vert_2$ auf $X$ heißen \begriff[Norm!]{äquivalent}, falls $\exists \alpha,\beta > 0:\alpha \Vert x \Vert_1 \le \Vert x\Vert_2 \le \beta \Vert x\Vert_1 \,\forall x\in X$
\end{definition}
\begin{conclusion}
	$\vert\cdot\vert_p, \vert\cdot\vert_q$ sind äquivalent auf $\mathbb{R}^n\,\forall p,q\ge 1$.
\end{conclusion}

\begin{definition}
    \begin{itemize}
    \item $B_r(a):=\{ x\in X \mid d(a,x) < r \}$ heißt (offene)\begriff{Kugel} um $a$ mit Radius $r > 0$
    \item $B_r[a]:=\bar{B}_r(a):=\{ x\in X \mid d(a,x) \le r \}$ heißt (abgeschlossene)\begriff{Kugel} um $a$ mit Radius $r > 0$
    \end{itemize}
    Hinweis: muss keine "`übliche"' Kugel sein, zum Beispiel $\{ x\in \mathbb{R}^n \mid d(0,x) = \Vert x\Vert_{\infty} < 1 \}$ hat die Form eines "`üblichen"' Quadrats.
    \begin{itemize}
        \item Menge $M\subset X$ heißt \begriff[Menge!]{offen}, falls $\forall x\in M\,\exists \epsilon > 0: B_\epsilon(x) \subset M$
        \item Menge $M\subset X$ ist \begriff[Menge!]{abgeschlossen}, falls $X\setminus M$ offen
        \item $U\subset X$ \begriff{Umgebung} von $M$, falls $\exists V\subset X$ offen mit $M\subset V\subset U$
        \item $x\in M$ \begriff{innerer Punkt}, von $M$, falls $\exists \epsilon > 0: B_\epsilon(x)\subset M$
        \item $x\in X\setminus M$ \begriff{äußerer Punkt} von $M$, falls $\exists \epsilon > 0: B_\epsilon(x)\subset X\setminus M$
        \item $x\in X$ heißt \begriff{Randpunkt}, von $M$, wenn $x$ weder innerer noch äußerer Punkt
        \item \mathsymbol{int}{$\Int$}$ M:=$ Menge aller inneren Punkte von $M$, heißt \begriff{Inneres} von $M$
        \item \mathsymbol{ext}{$\Ext$}$M:=$ Menge aller äußeren Punkte von $M$, heißt \begriff{Äußeres} von $M$.
        \item \mathsymbol{p}{$\partial$}$M:=$ Menge der Randpunkte von $M$, heißt \begriff{Rand} von $M$
        \item \mathsymbol{cl}{$\cl$}$:=\overline{M} = \Int M \cup \partial M$ heißt \begriff{Abschluss} von $M$
        \item $M\subset X$ heißt \begriff{beschränkt}[!Menge], falls $\exists a\in X, r>0: M\subset B_r(a)$
        \item $x\in X$ heißt \gls{hp} von $M$, falls $\forall \epsilon > 0$ enthält $B_\epsilon(x)$ unendlich viele Elemente aus $M$
        \item $x\in M$ heißt \begriff{isolierter Punkt} von $M$, falls $x$ kein Häufungspunkt
        \end{itemize}
\end{definition}
\stepcounter{theorem}

\begin{lemma}
	Sei $(X,d)$ metrischer Raum. Dann
	\begin{enumerate}[label={\arabic*)}]
		\item $B_r(a)$ offene Menge $\forall r>0,a\in X$
		\item $M\subset X$ beschränkt $\Rightarrow\; \forall a\in X\,\exists r>0: M\subset B_r(a)$
	\end{enumerate}
\end{lemma}

\begin{proposition}\label{proposition_topologie}
	Sei $(X,d)$ metrischer Raum, $\tau:=\{U\subset X \mid U \text{ offen}\}$. Dann
	\begin{enumerate}[label={\arabic*)}]
		\item \label{topologie_1} $X,\emptyset\in \tau$ offen
		\item \label{topologie_2} $\bigcap_{i=1}^n U_i\subset \tau$ falls $U_i\in\tau$ für $i=1,\dotsc,n$
		\item \label{topologie_3} $\bigcup_{U\in\tau'} U\in\tau$ falls $\tau'\in\tau$ 
	\end{enumerate}
\end{proposition}
\begin{conclusion}
	Sei $(X,d)$ metrischer Raum, $\sigma :=\{ V\subset X \mid  V \text{ abgeschlossen}\}$. Dann
	\begin{enumerate}[label={\arabic*)}]
		\item $X,\emptyset \in \sigma$ abgeschlossen
		\item $\bigcup_{i=1}^n V_i\subset\sigma$ falls $V_i\in\sigma_i$ für $i=1,\dotsc, n$
		\item $\bigcap_{V\in\sigma'} V\in\sigma$ falls $\sigma'\subset\sigma$
	\end{enumerate}
\end{conclusion}

\begin{definition}[Topologie]
	Sei $X$ Menge, und $\tau$ Menge von Teilmengen von $X$, d.h. $\tau\subset\mathcal{P}(X)$.\\
	$\tau$ ist \begriff{Topologie} und $(X,\tau)$ \begriff{topologischer Raum}, falls \ref{topologie_1},\ref{topologie_2},\ref{topologie_3} aus \ref{proposition_topologie} gelten.
\end{definition}
\begin{proposition}
	Seien $\Vert.\Vert_1, \Vert.\Vert_2$ äquivalente Normen in $X$ und $U\subset X$. Dann \[ U\text{ offen bezüglich } \Vert .\Vert_1\; \Leftrightarrow\; U\text{ offen bzgl. } \Vert .\Vert_2 \]
\end{proposition}
\begin{proposition}
	Sei $(X,d)$ metrischer Raum und $M\subset X$: Dann
	\begin{enumerate}[label={\arabic*)}]
		\item $\Int M, \Ext M$ offen
		\item $\partial M, \cl M$ abgeschlossen
		\item $M = \Int M$, falls $M$ offen, $M=\cl M$ falls $M$ abgeschlossen
	\end{enumerate}
\end{proposition}