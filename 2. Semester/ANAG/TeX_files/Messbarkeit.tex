\section{Messbarkeit}\setcounter{equation}{0}
Wir führen zunächst das \lebesque-Maß ein und behandeln dann messbare Mengen und messbare Funktionen.

\subsection{\lebesque-Maß}
\begin{*definition}[Quader, Volumen]
	Wir definieren die Menge \begin{align*}
		\mathcal{Q} &:= \left\{ I_1 \times \dotsc \times I_n \subset\mathbb{R}^n \mid I_j\subset\mathbb{R}\text{ beschränktes Intervall} \right\}
	\end{align*}
	$\emptyset$ ist auch als beschränktes Intervall zugelassen. $Q\in\mathcal{Q}$ heißt \begriff{Quader}.
	
	Sei $\vert I_j\vert :=$ Länge des Intervalls $I_j\subset\mathbb{R}$ (wobei $\vert\emptyset\vert = 0$), dann heißt \begin{align}
		\proplbl{messbarkeit_definition_volumen_eq}
		v(Q) &:= \vert I_1\vert \cdot \dots \cdot \vert I_n\vert \quad \text{für}\; Q = I_1\times \dotsc\times I_n \in\mathbb{Q}
	\end{align}
	\begriff{Volumen} von $Q$
	
	\emph{beachte:} $v(q) = 0$ für "`dünne"' Quader (d.h. falls ein $\vert I_j\vert = 0$). Insbesondere $v(\emptyset) = 0$.
\end{*definition}

Wir möchten für beliebige Mengen $M\subset\mathbb{R}^n$ ein "`Volumenmaß"' definieren, das mit dem Volumen für Quader kompatibel ist.
\begin{*definition}[\person{Lebesgue}-Maß]
	Dafür betrachte eine (Mengen-) Funktion $\vert .\vert :\mathcal{P}(\mathbb{R}^n)\to [0,\infty]$ mit \begin{align}
		\proplbl{messbarkeit_definition_lebesque_mass}
		\vert \mu \vert &= \inf \left\{ \left. \sum_{j=1}^{\infty} v(Q_j) \;\right|\; M\subset\bigcup\limits_{j=1}^\infty Q_j, \; \text{$Q_j\in\mathcal{Q}$ Quader} \right\}\quad\forall M\subset\mathbb{R}^n,
	\end{align}
	die man \begriff{\person{Lebegue}-Maß} auf $\mathbb{R}^n$ nennt.
	
	$\vert \mu \vert$ heißt (\lebesque-Maß) von $M$, oft schreibt man auch $\mathcal{L}^{\mu}(M)$.
\end{*definition}

\begin{*anmerkung}
	Man versucht das zu untersuchende Intervall mit Quadern zu überdecken und sucht dabei die 
	Überdeckung, bei der die Summe der Volumen am kleinsten wird. Also z.B. $\vert [2,3] \vert\in
	\natur=\vert\{2,3\}\vert=0$, da man für jede der beiden Zahlen genau einen Punkt als Quader braucht. 
	Der Punkt hat per Definition keine Dimension, also auch ein Volumen von 0. Damit gilt: $\vert [2,3]
	\vert=0+0=0$. Mit der gleichen Begründung gilt auch $\vert\natur\vert=0$.
\end{*anmerkung}

\begin{underlinedenvironment}[Hinweis]
	Das \lebesque-Maß wird in der Literatur vielfach nur für messbare Mengen definiert ($M\subset\mathbb{R}^n$) und die Erweiterung auf alle $M\subset\mathbb{R}^n$ wie in \eqref{messbarkeit_definition_lebesque_mass} wird dann als äußeres \lebesque-Maß bezeichnet.
\end{underlinedenvironment}

\begin{lemma}
	\proplbl{messbarkeit_nur_offene_mengen}
	Mann kann sich in \eqref{messbarkeit_definition_lebesque_mass} auf offene Mengen beschränken.
\end{lemma}
\begin{proof}
	Fixiere $\epsilon > 0$. Sei $M\subset\bigcup_{j=1}^\infty Q_j$, $Q_j\in\mathcal{Q}$ und $\alpha := \sum_{j=1}^\infty v(Q_j) < \vert M \vert + \epsilon$.
	
	Wähle offene Quader $\tilde{Q_j}\in\mathcal{Q}$ mit $Q_j\subset\tilde{Q_j}$, $v(\tilde{Q}_j)< v(Q_j) + \frac{\epsilon}{\alpha}$ \\
	$\Rightarrow$ $M\subset\bigcup_{j=1}^\infty \tilde{Q_j}$ und $\vert M \vert \le \sum_{j=1}^\infty v(\tilde{Q_j}) < \alpha + \epsilon < \vert M \vert + 2\epsilon$.
	
	Wegen $\epsilon > 0$ beliebig folgt die Behauptung.
\end{proof}

\begin{proposition}
	Es gilt: \begin{align}
		\proplbl{messbarkeit_satz_teilmenge_kleineres_mass_eq}
		M_1 \subset M_2 &\Rightarrow \vert M_1 \vert \le \vert M_2\vert
	\end{align}
	und die Abbildung $\mu\mapsto \vert \mu\vert$ ist \begriff{$\sigma$-subadditiv}, d.h. \begin{align}
		\proplbl{messbarkeit_sigma_subadditiv_eq}
		\left\vert \bigcup_{j=1}^\infty M_k\right\vert &\le \sum_{k=1}^\infty \vert M_k\vert, \quad\text{für } M_j\subset\mathbb{R}^n, \;j\in\mathbb{N}_{\ge 1}
	\end{align}
\end{proposition}

\begin{proof}
	\eqref{messbarkeit_satz_teilmenge_kleineres_mass_eq} folgt direkt aus \eqref{messbarkeit_definition_lebesque_mass} (Definition, das Infimum über eine größere Menge ist größer).
	
	Für \eqref{messbarkeit_sigma_subadditiv_eq} fixiere $\epsilon > 0$. Dann \begin{align*}
		\exists Q_{k_j} \in \mathcal{Q}&: M_k \subset\bigcup_{j=1}^\infty Q_{k_j},& \sum_{j=1}^\infty v(Q_{k_j}) &\le \vert M_k\vert + \frac{\epsilon}{2^k}
	\end{align*}
	Wegen $\bigcup_{k=1}^\infty M_k\subset \bigcup_{j,k=1}^\infty v(Q_{k_j}) \le \sum_{k=1}^\infty \vert M_k\vert + \epsilon$ folgt \begin{align*}
		\left\vert\bigcup_{k=1}^\infty M_k\right\vert \le \sum_{j,k=1}^\infty v(Q_{k_j}) \le \sum_{k=1}^\infty \vert M_k\vert + \epsilon
	\end{align*}
	Da $\epsilon>0$ beliebig, folgt die Behauptung.
\end{proof}

\begin{*definition}[Nullmenge]
	$N\subset\mathbb{R}^n$ heißt \begriff{Nullmenge}, falls $\vert N \vert = 0$. Offenbar gilt:\begin{align}
		\tilde{N}\subset N,\;\vert N \vert = 0 &\Rightarrow \;\vert \tilde{N}\vert = 0 \\
		\vert N_k\vert = 0 \;\forall k\in\mathbb{N}&\Rightarrow \;\left\vert \bigcup_{k=1}^\infty N_k \right\vert = 0
	\end{align}
\end{*definition}
Nach \eqref{messbarkeit_satz_teilmenge_kleineres_mass_eq} und \eqref{messbarkeit_sigma_subadditiv_eq} gilt:{\zeroAmsmathAlignVSpaces** \begin{align}
	\proplbl{messbarkeit_m_gleich_m_ohne_nullmenge}
	M\subset\mathbb{R}^n,\;\vert N \vert = 0 \;&\Rightarrow\; \vert M \vert = \vert M \setminus N\vert
\end{align}}
\begin{proof}\NoEndMark Dann $\vert M \setminus N\vert \overset{\eqref{messbarkeit_satz_teilmenge_kleineres_mass_eq}}{\le} \vert M \vert \overset{\eqref{messbarkeit_sigma_subadditiv_eq}}{\le} \underbrace{\vert M\cap N\vert}_{=0} + \vert M \setminus N\vert = \vert M \setminus N\vert$ $\Rightarrow$ Behauptung.\hfill\csname\InTheoType Symbol\endcsname\end{proof}

\begin{example} Es sind Nullmengen
	\begin{enumerate}[label={(\alph*)}]
		\item $\vert \emptyset\vert = 0$
		\item $\vert \{ x\} \vert = 0$ $\forall x\in\mathbb{R}^n$
		
		$\vert$abzählbar viele Punkte$\vert = 0$, folglich $\mathcal{L}^1(\mathbb{Q}) = 0$, $\mathcal{L}^1(\mathbb{N}) = 0$ (d.h. wir betrachten $\mathbb{Q}, \mathbb{N}$ als Teilmengen von $\mathbb{R}$, d.h. $n=1$)
		\item $\vert M \vert = 0$ falls $M\subsetneqq \mathbb{R}^n$ (echter affiner Unterraum)
		\item $\vert \partial Q\vert = 0$ für $Q\in\mathcal{Q}$
		\item "`schöne"' Kurven im $\mathbb{R}^2$
		
		"`schöne"' Kurven und Flächen im $\mathbb{R}^3$
	\end{enumerate}
\end{example}

\begin{conclusion}
	Es ist $v(q) = \vert Q\vert$ $\forall Q\in\mathcal{Q}$
	
	Damit im folgenden Stets $\vert Q\vert$ statt $v(Q)$
\end{conclusion}
\begin{proof}
	Sei $Q\in\mathcal{Q}$. Da offenbar $v(Q) = v(\cl Q)$ und $\vert Q\vert = \vert \cl Q\vert$ können wir $Q$ als abgeschlossen annehmen.
	
	Für ein fixiertes $\epsilon > 0$ existieren nach \propref{messbarkeit_nur_offene_mengen} offene $Q_j\in\mathcal{Q}$ mit \begin{alignat*}{3}
		Q&\subset\bigcup_{j=1}^\infty Q_j &\quad\text{und}\quad& \sum_{j=1}^\infty v(Q_j) &\le \vert Q \vert + \epsilon
	\end{alignat*}
	Da $Q$ kompakt ist, wird es durch endlich viele $Q_j$ überdeckt d.h. \gls{obda} $Q\subset\bigcup_{j=1}^\infty Q_j$. Mittels einer geeigneten Zerlegung der $Q_j$ folgt aus \eqref{messbarkeit_definition_volumen_eq}, dass $v(Q)\le \sum_{j=1}^\infty v(Q_j)$. Somit gilt: \begin{align*}
		\vert Q \vert \overset{\eqref{messbarkeit_definition_lebesque_mass}}{\le} v(Q) \le \vert Q \vert + \epsilon
	\end{align*}
	Da $\epsilon > 0$ beliebig, folgt die Behauptung.
\end{proof}

\begin{*definition}
	Eine Eigenschaft gilt \gls{fü} auf $M\subset\mathbb{R}^n$, falls eine Nullmenge existiert, sodass die Eigenschaft $\forall x\in M\setminus N$ gilt. Man sagt auch, dass die Eigenschaft für \gls{fa} $x\in M$ gilt.
\end{*definition}

\begin{example}
	\proplbl{messbarkeit_einfuehrung_dirichlet_funktion}
	Für die \person{Dirichlet}-Funktion \begin{align*}
		f(x) &=\begin{cases}
			1, &x\in\mathbb{Q} \\ 0,&x\in\mathbb{R}\setminus\mathbb{Q}
		\end{cases}
	\end{align*}
	ist $f=0$ \gls{fü} auf $\mathbb{R}$.
\end{example}

\subsection{Messbare Mengen}
\begin{boldenvironment}[Frage]
	gilt für paarweise disjunkte Mengen $M_k$ in  \eqref{messbarkeit_sigma_subadditiv_eq} Gleichheit?
	
	Obwohl es wünschenswert wäre, gibt es "`sehr exotische"' Mengen, für die dies nicht gilt (vgl. Bemerkung zum Auswahlaxiom in Kap. 2).
	
	Deshalb betrachten wir "`gutartige"' Mengen.
\end{boldenvironment}

\begin{*definition}[messbar]
Eine Menge $M\subset\mathbb{R}^n$ heißt \begriff{messbar}, falls \begin{align}
		\vert \tilde{M}\vert = \vert \tilde{M}\cap M\vert + \vert \tilde{M}\setminus M\vert \quad\forall \tilde{M}\in\mathbb{R}\marginnote{"`Sie können sich keine Menge vorstellen, die nicht messbar ist."' (Hr. Schönherr, 2014)}
	\end{align}
	
	Man beachte, dass nach \eqref{messbarkeit_sigma_subadditiv_eq} stets \begin{align}
		\proplbl{messbarkeit_definition_messbar_folgerung}
		\vert \tilde{M} \vert \le \vert \tilde{M} \cap M\vert + \vert \tilde{M}\setminus M\vert \quad\forall M,\tilde{M}\subset\mathbb{R}^n
	\end{align}
	Beim Nachweis der Messbarkeit muss man nur "`$\ge$"' prüfen.
\end{*definition}

\begin{proposition}
	\proplbl{messbarkeit_satz_grundlegende_messbare_mengen}
	\begin{enumerate}[label={(\alph*)}]
		\item \proplbl{messbarkeit_satz_sigma_algebra_eins} $\emptyset$, $\mathbb{R}^n$ sind messbar
		\item \proplbl{messbarkeit_satz_sigma_algebra_zwei} $M\subset\mathbb{R}^n$ messbar $\Rightarrow$ $M^C = \mathbb{R}^n\setminus M$ messbar 
		\item \proplbl{messbarkeit_satz_sigma_algebra_drei} $M_1, M_2, \dotsc\subset\mathbb{R}^n$ messbar $\Rightarrow$ $\bigcup_{j=1}^\infty M_j$, $\bigcap_{j=1}^\infty M_j$ messbar
	\end{enumerate}
\end{proposition}

\begin{*definition}[$\sigma$-algebra]
	Eine Menge von Teilmengen $\mu\subset X$  (hier $X=\mathbb{R}^n$) mit den Eigenschaften \cref{messbarkeit_satz_sigma_algebra_eins,messbarkeit_satz_sigma_algebra_zwei,messbarkeit_satz_sigma_algebra_drei} heißt \begriff{$\sigma$-algebra}
\end{*definition}

\begin{proof}\hspace*{0pt}
	\begin{itemize}[topsep=\dimexpr-\baselineskip / 2\relax]
		\item[\ref{messbarkeit_satz_sigma_algebra_eins}] wegen $\vert\emptyset\vert = 0$ und \eqref{messbarkeit_m_gleich_m_ohne_nullmenge}: $\vert \tilde{M}\vert \le \vert\tilde{M}\setminus\emptyset\vert = \vert\tilde{M}\vert$
		
		\item[\ref{messbarkeit_satz_sigma_algebra_zwei}] wegen $\tilde{M}\cap M = \tilde{M}\setminus M^C$, $\tilde{M}\setminus M = \tilde{M}\cap M^C$ $\Rightarrow$ Behauptung
	
		\item[\ref{messbarkeit_satz_sigma_algebra_drei}] \eqref{messbarkeit_sigma_subadditiv_eq} liefert \begin{align*}
			\vert\tilde{M}\vert \le \vert \tilde{M}\cap M\vert + \vert\tilde{M}\setminus M\vert \quad\forall \tilde{M}, \;M\subset\mathbb{R}^n,
		\end{align*}
		sodass man nur noch "`$\ge$"' zeigen muss.
		
		\begin{itemize}
		
			\item Seien $M_1$, $M_2$ messbar, dann gilt für beliebige $\tilde{M}\subset\mathbb{R}^n$: \begin{align*}
				\tilde{M}\cap (M_1\cup M_2) &=(\tilde{M}\cap M_1)\cup \big( (\tilde{M}\setminus M_1)\cap M_2 \big), \\
				\tilde{M}\setminus (M_1\cup M_2) &= (\tilde{M}\setminus M_1)\setminus M_2
			\end{align*}
			folglich \begin{align*}
				\vert \tilde{M}\vert &= \vert \tilde{M}\cap M_1\vert + \vert \tilde{M}\setminus M_1 \vert = \vert\tilde{M}\cap M_1\vert + \vert (\tilde{M}\setminus M_1)\cap M_2\vert + \vert (\tilde{M}\setminus M_1)\setminus M_2\vert \\
				&\ge \vert \tilde{M}\cap (M_1\cup M_2)\vert + \vert \tilde{M} \setminus (M_1\cup M_2) \vert,
			\end{align*}
			daher $M_1\cup M_2$ messbar.
			
			\item Da $(M_1\cap M_2)^C = M_1^C\cup M_2^C$ ist auch $M_1\cap M_2$ messbar.\\
			\begin{tabularx}{\linewidth}{r@{\ \ }X}
				$\Rightarrow$ & $M_1,\dotsc,M_k$ messbar \\
				$\Rightarrow$ & $M_1\cup \dotsc\cup M_k$ sowie $M_1\cap\dotsc\cap M_2$ messbar (Induktion).
			\end{tabularx}
		
			\item Seien jetzt $M_1,\dotsc\subset\mathbb{R}^n$ messbar und paarweise disjunkt \\
			\begin{tabularx}{\linewidth}{r@{\ \ }X}
				$\Rightarrow$ & alle $A_k := \bigcup_{j=1}^k M_j$ messbar. Für beliebige $\tilde{M}\subset\mathbb{R}^n$ folgt schrittweise
			\end{tabularx}
			\begin{align*}
				\vert \tilde{M} \cap A_k \vert + \sum_{j=2}^k \vert \tilde{M}\cap M_j\vert = \sum_{j=1}^k \vert \tilde{M}\cap M_j\vert
			\end{align*}
			Mit $A = \bigcup_{j=1}^\infty M_j$ folgt \begin{align}
				\proplbl{messbarkeit_sigma_algebra_beweis_eq}
				\vert \tilde{M}\vert = \vert \tilde{M}\cap A_k\vert + \vert \tilde{M}\setminus A_k\vert \ge \sum_{j=1}^k \vert \tilde{M}\cap M_j\vert + \vert \tilde{M}\setminus A\vert \quad \forall k\in\mathbb{N},
			\end{align}
			\begin{tabularx}{\linewidth}{r@{\ \ }X}
			$\xRightarrow{k\to\infty}$ & $\vert \tilde{M} \vert \ge \sum_{j=1}^\infty \vert \tilde{M}\cap M_j\vert + \vert \tilde{M} \setminus A\vert \overset{\eqref{messbarkeit_sigma_subadditiv_eq}}{\ge} \vert \tilde{M}\cap A\vert + \vert\tilde{M}\setminus A\vert$ \\
			$\Rightarrow$ & $A$ messbar
			\end{tabularx}
			
			\item Folglich sind die $M_j$ nicht paarweise disjunkt, ersetze $M_j$ durch $\underbrace{A_j \setminus A_{j-1}}_{=M_j'}$ und argumentiere wie oben (da $\bigcup_{k=1}^\infty M_k = \bigcup_{k=1}^\infty M_k^C$ $\Rightarrow$ $\bigcup_{k=1}^\infty M_k$ messbar, $\bigcap$ analog).
		\end{itemize}
	\end{itemize}
\end{proof}

\begin{proposition}
	\proplbl{messbarkeit_mengen_ober_unter_mengen}
	Seien $M_1$, $M_2$, $\dotsc\subset\mathbb{R}^n$ messbar. Dann \begin{enumerate}[label={(\alph*)}]
		\item \proplbl{messbarkeit_sigma_additiv}
		$M_j$ paarweise disjunkt $\Rightarrow$ $\left\vert \bigcup_{k=1}^\infty M_k\right\vert = \sum_{k=1}^\infty \vert M_k\vert$ ($\sigma$-additiv)
		\item \proplbl{messbarkeit_teilmengen_grenzwert_gleich_mass_vereinigung}
		$M_1\subset M_2\subset\dotsc$ $\Rightarrow$ $\lim\limits_{k\to\infty} \vert M_k\vert = \left\vert \bigcup_{k=1}^\infty M_k\right\vert$
		\item \proplbl{messbarkeit_mengen_ober_unter_mengen_c}
		 $M_1\supset M_2 \supset \dotsc$ und $\vert M_1 \vert < \infty$ $\Rightarrow$ $\lim\limits_{k\to\infty} \vert M_k\vert = \left\vert \bigcap_{k=1}^\infty M_k\right\vert$
	\end{enumerate}
\end{proposition}

\begin{proof}\hspace*{0pt}
	\begin{enumerate}[label={\alph*)}]
		\item Aus \eqref{messbarkeit_sigma_algebra_beweis_eq} mit $\tilde{M} = \mathbb{R}^n$ erhält man \begin{align*}
			\sum_{k=1}^{m} \vert M_k\vert = \left\vert\bigcup_{k=1}^m M_k\right\vert \overset{\eqref{messbarkeit_sigma_subadditiv_eq}}{\le} \sum_{k=1}^\infty \vert M_k\vert
		\end{align*}
		Der Grenzübergang $m\to\infty$ liefert die Behauptung.
		
		\item Nach \ref{messbarkeit_sigma_additiv} gilt: $\vert M_k\vert = \vert M_1 \vert + \sum_{k=1}^k \vert M_j\setminus M_{j-1}$, und folglich \begin{align*}
			\vert M_k\vert = \vert M_1 \vert + \sum_{k=1}^\infty \vert M_k\setminus M_{k-1}\vert \overset{\ref{messbarkeit_sigma_additiv}}{=} \left\vert \bigcup_{k=1}^\infty M_k\right\vert
		\end{align*}
		
		\item $A:= \bigcap_{k=1}^\infty M_k$. Wegen $\vert M_1\setminus M_k\vert = \vert M_1 \vert - \vert M_k\vert$ nach \eqref{messbarkeit_sigma_subadditiv_eq} hat man \begin{alignat*}{5}
			&\vert M_1\vert &\;\overset{\eqref{messbarkeit_sigma_subadditiv_eq}}{\le}\;& \vert A \vert + \vert M_1 \setminus A\vert &\;=\;& \vert A \vert + \left\vert \bigcup_{k=1}^\infty M_1 \setminus M_k\right\vert\\ &&\overset{\ref{messbarkeit_teilmengen_grenzwert_gleich_mass_vereinigung}}{=}\;& \vert A \vert + \lim\limits_{k\to\infty} \vert M_1 \setminus M_k\vert&=\;& \vert A \vert + \vert M_1\vert - \lim\limits_{k\to\infty} \vert M_k\vert \\
			&&\le\;\,& \lim\limits_{k\to\infty} \vert M_k\vert + \vert M_1\vert - \lim\limits_{k\to\infty} \vert M_k\vert &=\;& \vert M_1\vert&
		\end{alignat*}
		Subtraktion von $\vert M_1\vert$ liefert die Behauptung.
	\end{enumerate}
\end{proof}

\begin{proposition}
	\proplbl{messbarkeit_mengen_satz_acht}
	Es gilt: \begin{enumerate}[label={(\alph*)}]
		\item alle Quader sind Messbar ($Q\in\mathcal{Q}$)
		\item Offene und abgeschlossene $M\subset\mathbb{R}^n$ sind messbar
		\item alle Nullmengen sind messbar
		\item Sei $M\subset\mathbb{R}^n$ messbar, $M_0\subset\mathbb{R}^n$, beide Mengen unterscheiden sich voneinander nur um eine Nullmenge, d.h. $\vert (M\setminus M_0)\cup (M_0\setminus M)\vert = 0$ \\
		$\Rightarrow$ $M_0$ messbar.
	\end{enumerate}
\end{proposition}

\begin{proof}\hspace*{0pt}
	\begin{enumerate}[label={\alph*)},topsep=\dimexpr -\baselineskip / 2 \relax]
		\item Sei $Q\in\mathbb{Q}$ Quader. Für $\tilde{M}\subset\mathbb{R}^n$, $\epsilon > 0$ wähle $Q_j$ mit \begin{align*}
			\tilde{M} &\subset \bigcup_{j=1}^\infty Q_j,& \sum_{j=1}^\infty \vert Q_j\vert &\le \vert \tilde{M}\vert + \epsilon
		\end{align*}
		Aus \eqref{messbarkeit_definition_volumen_eq} folgert man $\vert Q_j\vert = \vert Q_j\cap Q\vert + \vert Q_j\setminus Q\vert$, da man $Q_j\setminus Q$ in endlich viele disjunkte Quader zerlegen kann.
		\begin{flalign*}
		\;\;\Rightarrow\;\; & \vert \tilde{M}\cap Q\vert + \vert \tilde{M}\setminus Q\vert \overset{\eqref{messbarkeit_sigma_subadditiv_eq}}{\le} \sum_{j=1}^\infty \vert Q_j\cap Q\vert + \sum_{j=1}^\infty \vert Q_j\setminus Q\vert = \sum_{j=1}^\infty \vert Q_j\vert\le \vert \tilde{M} \vert + \epsilon &
		\end{flalign*}
		Da $\epsilon$ beliebig, $\vert \tilde{M} \vert \ge \vert \tilde{M} \cap Q\vert + \vert \tilde{M} \setminus Q\vert$ und \eqref{messbarkeit_definition_messbar_folgerung}, ergibt sich die Behauptung.
		
		\item Sei $M\subset\mathbb{R}^n$ offen. Betrachte die Folge $\{x_n\}_{k=1}^\infty$ aller rationale Punkte in $M$ und $w_k\subset M$ sei jeweils der größte offene Würfel mit dem Mittelpunkt $x_k$ und Kantenlänge $\le 1$.
		
		Dann $M = \bigcup_{k=1}^\infty w_k$, denn für jedes $x\in M$ ist $B_{\epsilon}(x)\subset M$ für ein $\epsilon > 0$ und somit ist $x\in w_k$ für ein $x_k$ nahe genug bei $x$. Folglich ist $M$ messbar nach \propref{messbarkeit_satz_grundlegende_messbare_mengen}.
		
		Für $M\subset\mathbb{R}^n$ abgeschlossen ist das Komplement $\mathbb{R}^n\setminus M$ offen und somit messbar. Damit ist $M=\mathbb{R}^n\setminus (\mathbb{R}^n\setminus M)$ messbar.
		
		\item Für eine Nullmenge $N$, $\tilde{M}\subset\mathbb{R}^n$ ist $\vert \tilde{M} \vert \overset{\eqref{messbarkeit_sigma_subadditiv_eq}}{\le} \vert \tilde{M}\cap N\vert + \vert \tilde{M} \setminus N\vert \overset{\eqref{messbarkeit_satz_teilmenge_kleineres_mass_eq}}{\le} \vert N \vert + \vert \tilde{M} \setminus N\vert \overset{\eqref{messbarkeit_m_gleich_m_ohne_nullmenge}}{=} \vert\tilde{M}\vert$
		
		\item Mit den Nullmengen $N_1 := M\setminus M_0$, $N_2 = M_0\setminus M$ gilt $M_0 = (M\setminus N_1)\cup N_2$. Da $M\setminus N_1$ messbar ist, erhält man für beliebiges $\tilde{M}\subset \mathbb{R}^n$
		\begin{center}
		\begin{tabular}{r@{\ }c@{\ }l}
			$\vert \tilde{M}\cap M_0\vert + \vert \tilde{M}\setminus M_0\vert$ &=& $\vert\tilde{M} \cap ((M\setminus N_1)\cup N_2)\vert + \vert \tilde{M} \setminus ((M\setminus N_1)\cup N_2)\vert$ \\
			& $\overset{\eqref{messbarkeit_satz_teilmenge_kleineres_mass_eq},\eqref{messbarkeit_sigma_subadditiv_eq}}{\le}$ & $\vert M\cap(M\setminus N_1)\vert + \vert \tilde{M} \cap N_2\vert + \vert \tilde{M} \setminus (M \setminus N_1)\vert$ \\
			&=& $\vert \tilde{M}\vert$
		\end{tabular}
		\end{center}
	
		Mit \eqref{messbarkeit_definition_messbar_folgerung} folgt dann, dass $M_0$ messbar ist.
	\end{enumerate}
\end{proof}

\subsection{Messbare Funktionen}
Wir führen nun eine für die Integrationstheorie grundlegende Klasse von Funktionen ein. Dabei erlauben wir $\pm \infty$ als Funktionswerte und benutzen die Bezeichnung \begin{align*}
	\overline{\mathbb{R}} &= \mathbb{R}\cup \{ \pm \infty \} = [ -\infty, \infty ]
\end{align*}
sowie für $a\in\mathbb{R}$ \begin{align*}
	(a,\infty] &= (0,\infty)\cup\{ \infty \},
\end{align*}
und analog $[a,\infty]$, $(-\infty,a)$, $[-\infty,a]$. \marginnote{vgl. Kap. 5}

Für $\epsilon > 0$ definieren wir offene $\epsilon$-Kugeln um $\pm\infty$ durch \begin{align*}
	B_\epsilon(\infty) &:= \left( \frac{1}{\epsilon}, \infty \right] & &\text{bzw.} & B_\epsilon(\-\infty) := \left[ -\infty, -\frac{1}{\epsilon} \right)
\end{align*}

$U\subset\overline{\mathbb{R}}$ \emph{offen}, falls für jedes $x\in U$ ein $\epsilon > 0$ existiert, sodass $B_\epsilon \subset U$. Damit sind inbsesondere die offenen Mengen aus $\mathbb{R}$ auch offen in $\overline{\mathbb{R}}$ und die offenen Mengen in $\overline{\mathbb{R}}$ bilden eine Topologie. \marginnote{vgl. Kap. 8}

\begin{*definition}[messbar]
	Eine Funktion $f:D\subset\mathbb{R}\to\overline{\mathbb{R}}$ heißt \begriff{messbar}, falls $D$ messbar ist und $f^{-1}(U)$ für jede offene Menge $U\subset\overline{\mathbb{R}}$ messbar ist.
\end{*definition}

\begin{conclusion}
	Sei $f:D\subset\mathbb{R}\to\overline{\mathbb{R}}$ mit $D$ messbar. Dann sind folgende Aussagen äquivalent:\begin{enumerate}[label={(\alph*)}]
		\item \proplbl{messbarkeit_funktionen_satz_neun_eins}
		$f$ ist messbar
		\item \proplbl{messbarkeit_funktionen_satz_neun_zwei}
		$f^{-1}\left( [-\infty, a)\right)$ messbar $\forall a\in\mathbb{Q}$
		\item \proplbl{messbarkeit_funktionen_satz_neun_drei}
		$f^{-1}\left( [-\infty, a] \right)$ ist messbar $\forall a\in \mathbb{Q}$
	\end{enumerate}
\end{conclusion}

\begin{proof}
	Aus den Eigenschaften messbarer Mengen folgt mit \begin{align*}
		f^{-1}\left( [-\infty, a]\right) &= \bigcap_{k=1}^\infty f^{-1}\left( \left[ -\infty, a + \frac{1}{k} \right]\right) \\
		f^{-1} \left([-\infty, a)\right) &= \bigcup_{k=1}^\infty f^{-1}\left( \left[ -\infty, a - \frac{1}{k}\right] \right)
	\end{align*}
	die Äquivalenz von \ref{messbarkeit_funktionen_satz_neun_zwei} und \ref{messbarkeit_funktionen_satz_neun_drei}.
	
	Offenbar ist \ref{messbarkeit_funktionen_satz_neun_eins} $\Rightarrow$ \ref{messbarkeit_funktionen_satz_neun_zwei} $\Leftrightarrow$ \ref{messbarkeit_funktionen_satz_neun_drei}.
	
	Für $a,b\in\mathbb{Q}$ ist dann \begin{align*}
		f^{-1}\big( (a,b) \big) &= f^{-1}\left( [-\infty, b]\right) \cap f^{-1}\left( [a,\infty] \right) = f^{-1}\left( [-\infty, a)\right) \cap \left( f^{-1} \big( [-\infty, a] \big)\right)^C
	\end{align*}
	messbar und offensichtlich $f^{-1}\big( (a,\infty] \big)$ ebenfalls.
	
	Da jede offene Menge $U\subset\overline{\mathbb{R}}$ die abzählbare Vereinigung von Mengen der Form $(a,b)$, $[-\infty, a)$, $(a,]$ mit $a,b\in\mathbb{Q}$ ist, folgt die Messbarkeit von $f^{-1}(U)$ und somit \ref{messbarkeit_funktionen_satz_neun_eins}.
\end{proof}
\begin{underlinedenvironment}[Hinweis]
	Wir werden sehen, dass die Menge aller messbaren Funktionen die Menge der stetigen Funktionen enthält, aber auch noch viele Weitere.
\end{underlinedenvironment}

\begin{*definition}[charakteristische Funktion]
	Für $M\subset\mathbb{R}^n$ heißt $\chi_\mu:\mathbb{R}^n\to\mathbb{R}$ mit \begin{align*}
		\chi_\mu = \begin{cases}
			1, &x\in M\\ 0, &x\in\mathbb{R}^n\setminus M
		\end{cases}
	\end{align*}
	\begriff{charakteristische Funktion} von $M$.
\end{*definition}

Offenbar gilt
\begin{conclusion}
	$\chi_\mu:\mathbb{R}^n\to\mathbb{R}$ ist messbar \gls{gdw} $M\subset\mathbb{R}^n$ messbar ist.
\end{conclusion}

\begin{*definition}[Treppenfunktion]
	Eine Funktion $h:\mathbb{R}^n\to\mathbb{R}$ heißt \begriff{Treppenfunktion}, falls es $M_1, \dotsc, M_k\subset\mathbb{R}^n$  und $c_1,\dotsc,c_k\in\mathbb{R}$ gibt mit \begin{align}
		\proplbl{messbarkeit_definition_treppenfunktion_eq}
		h(x) = \sum_{j=1}^k a_j \chi_{\mu_j}(x)
	\end{align}
	
	Die Menge der Treppenfunktionen \mathsymbol{T}{$T(\mathbb{R})$} ist mit der üblichen Addition und skalarer Multiplikation für Funktionen ein Vektorraum.
	
	Man beachte, dass die Darstellung in \eqref{messbarkeit_definition_treppenfunktion_eq}, d.h. die Wahl der $\mu_j$ und $c_j = a_j$ nicht eindeutig ist. Insbesondere kann man $\mu_j$ stets paarweise disjunkt wählen.
\end{*definition}

Man sieht leicht
\begin{conclusion}
	\ \\
	\begin{tabularx}{\linewidth}{X@{\ }c@{\ }X}
		\hfill Die Treppenfunktion $h\in T(\mathbb{R}^n)$ ist messbar & $\Leftrightarrow$ & es gibt mindestens eine Darstellung \eqref{messbarkeit_definition_treppenfunktion_eq}, bei der alle $\mu_j$ messbar sind.
	\end{tabularx}
\end{conclusion}

\begin{*definition}[Nullfortsetzung]
	Für $f:D\subset\mathbb{R}^n\to\overline{\mathbb{R}}$ definieren wir die \begriff{Nullfortsetzung} $\overline{f}:\mathbb{R}^n\to\overline{\mathbb{R}}$ durch \begin{align}
		\overline{f}(x) &:= \begin{cases}
			f(x), &x\in D\\ 0,&x\in\mathbb{R}^n\setminus D
		\end{cases}
	\end{align}
\end{*definition}

\begin{proposition}
	\proplbl{messbarkeit_funktion_nullfortsetzung}
	Es gilt:\begin{enumerate}[label={\alph*)}]
		\item Sei $f:D\subset\mathbb{R}^n\to\overline{\mathbb{R}}$ messbar. Dann ist auch die Nullfortsetzung $\overline{f}:\mathbb{R}^n\to\overline{\mathbb{R}}$ messbar
		\item Sei $f:D\subset\mathbb{R}^n\to\overline{\mathbb{R}}$ messbar und $D'\subset D$ messbar. Dann ist $f$ auf $D'$ messbar, d.h. insbesondere $\left. f\right|_{D'}$ ist messbar.
		\item Seien $f,g:D\subset\mathbb{R}^n\to\overline{\mathbb{R}}$. Sei $f$ messbar und $f=g$ \gls{fü} auf $D$. Dann ist $g$ messbar.
	\end{enumerate}
\end{proposition}

\begin{example}
	Die \person{Dirichlet}-Funktion ist auf $\mathbb{R}$ messbar.
	
	$h=0$ ist messbare Treppenfunktion auf $\mathbb{R}$ und stimmt mit der \person{Dirichlet}-Funktion \gls{fü} überein.
\end{example}

\begin{proof}\hspace*{0pt}
	\begin{enumerate}[label={(\alph*)},topsep=\dimexpr -\baselineskip / 2\relax]
		\item Für ein offenes $U\subset\overline{\mathbb{R}}$ ist $\overline{f}^{-1}(U) = f^{-1}(U)$ falls $0\notin U$ und andernfalls $\overline{f}^{-1}(U) = f^{-1}(U)\cup (\mathbb{R}^n\setminus D)$.
		\item Für offenes $U\subset\overline{\mathbb{R}}$ ist $\left(\left. f\right|_{D'}\right)^{-1} (U) = f^{-1} (U)\cap D$.
		\item Für $U\subset\overline{R}$ offen: $f^{-1}(U)$ ist messbar und $g^{-1}(U)$ unterscheidet sich von $f^{-1}(U)$ nur um eine Nullmenge. Somit ist $g^{-1}(U)$ nach \propref{messbarkeit_mengen_satz_acht} messbar.
	\end{enumerate}
\end{proof}

\rule{0.4\linewidth}{0.1pt}

\begin{*definition}[positiver, negativer Teil]
Für $f:D\subset\mathbb{R}^n\to\overline{\mathbb{R}}$ und $\alpha\in\overline{\mathbb{R}}$ schreibt man verkürzt
\begin{align*}
\{ f > \alpha \} := \{ x\in D \mid f(x) > \alpha \}
\end{align*}
Man definiert mit \begin{align*}
	f^{+} &:= f\cdot \chi_{\{f > 0\}}, & f^{-} &:= -f\cdot \chi_{\{f \le 0\}}
\end{align*}
den \begriff{positive Teil} bzw. \begriff{negative Teil} von $f$, und man hat $f = f^+ - f^-$.

Weiterhin ist 
\begin{align*}
f:=\max(f_1, f_2):\mathbb{R}^n\to\mathbb{R},\;\;f(x) = \max \{ f_1(x), f_2(x) \} \quad \forall x\in\mathbb{R}^n
\end{align*}
und analog: $\min(f_1, f_2)$, $\sup\limits_{k\in\mathbb{N}} f_k$, $\inf\limits_{k\in\mathbb{N}} f_k$, $\limsup\limits_{k\to\infty} f_k$, $\liminf\limits_{k\in\mathbb{N}} f_k$

Bei punktweiser Konvergenz $f_k(x)\to f(x)$ für \gls{fa} $x\in D$ schreibt man auch $f_k\to f$ \gls{fü} auf $D$.
\end{*definition}

\begin{proposition}[zusammengesetzte messbare Funktionen]
	\proplbl{messbarkeit_funktionen_komposition}
	Für $D\subset\mathbb{R}^n$ messbar gilt \begin{enumerate}[label={\alph*)}]
		\item \proplbl{messbarkeit_funktion_zusammensetzung_addition_eins}
		$f,g:D\subset\mathbb{R}^n\to\mathbb{R}$ messbar $\Rightarrow$ $f\pm g$, $f\cdot g$ messbar,
		
		falls $g\neq 0$ auf $D$ $\Rightarrow$ $\frac{f}{g}$ messbar
		\item \proplbl{messbarkeit_funktion_zusammensetzung_addition_zwei}
		$f,g:D\subset\mathbb{R}^n\to\overline{\mathbb{R}}$ messbar, $c\in\mathbb{R}$ $\Rightarrow$ $f^{\pm}$, $\vert f \vert$, $\max(f,g)$, $\min(f,g)$ messbar
		\item \proplbl{messbarkeit_funktion_zusammensetzung_addition_drei}
		$f_k:D\subset\mathbb{R}^n\to\overline{\mathbb{R}}$ messbar $\forall k\in\mathbb{N}$ $\Rightarrow$ $\sup\limits_{k} f_k$, $\inf\limits_{k} f_k$, $\liminf\limits_{k} f_k$, $\limsup\limits_{k} f_k$ messbar
	\end{enumerate}
\end{proposition}

\begin{underlinedenvironment}[Hinweis]
	In \ref{messbarkeit_funktion_zusammensetzung_addition_eins} nur Funktionen mit Wertein in $\mathbb{R}$, nicht $\overline{\mathbb{R}}$, sonst ist die zusammengesetzte Funktion eventuell nicht erklärt.
\end{underlinedenvironment}

\begin{proof}\hspace*{0pt}
	\NoEndMark
	\begin{itemize}
	\item $\forall a\in\mathbb{Q}$ gilt: \begin{align*}
		\left( f + g\right)^{-1}\big( [-\infty, a) \big) &= \bigcup\limits_{\substack{\alpha,\beta\in\mathbb{Q}\\ \alpha + \beta \le a}} f^ {-1} ([-\infty,\alpha])\cap g^{-1}([-\infty,\beta))
	\end{align*}
	ist messbar, folglich $f+g$ messbar
	
	\item Für $c > 0$ ist \begin{align*}
		(cf)^{-1}([-\infty, a]) &= f^{-1} \left(\left[ -\infty,\frac{a}{c}\right)\right)&&\text{messbar als Menge,} \\
		(-cf)^{-1}) ([-\infty,a)) &= f^{-1}\left( \left(-\frac{a}{c},+\infty \right]\right)&&\text{messbar}
	\end{align*}
	$\Rightarrow$ $cf$ messbar ($c=0$ trivial) \\
	$\Rightarrow$ $-f$, $f+(-g)$ messbar
	
	\item Wegen \begin{align*}\left(f^2\right)^{-1}([-\infty,a)) = f^{-1}([-\infty,\sqrt{a}))\setminus f^{-1}([-\infty,-\sqrt{a}])\quad \forall a\ge 0\end{align*} ist $f^2$ messbar \\
	$\Rightarrow$ $f\cdot g = \frac{1}{2}\left( (f+g)^2 - f^2 - g^2\right)$ messbar
	
	\item Falls $g\neq 0$ auf $D$ ist für $a\ge 0$ \begin{align*}
		\left( \frac{1}{g}\right)^{-1} ([-\infty, -a)) &= g^{-1}\left( \left( -\frac{1}{a}, 0\right)\right) & \left( \frac{1}{g}\right)^{-1} ([a,\infty]) &= g^{-1} \left( \left( 0,\frac{1}{a}\right) \right)
	\end{align*}
	und mit $\left( \frac{1}{g}\right)^{-1}([-\infty,0)) = g^{-1}([-\infty, 0))$ folgt $\frac{1}{g}$ messbar \\
	$\Rightarrow$ Produkt $f\cdot \frac{1}{g} = \frac{f}{g}$ messbar
	
	\item Aus der Messbarkeit der Niveaumengen $\{ f > 0\}$, $\{ f < 0\}$ folgt die Messbarkeit von $f^{\pm} = f\chi_{\{ f \substack{>\\<}0\}}$, $\vert f\vert = f^{+} + f^{-}$, $\max(f,g) = (f - g)^+ + g$, $\min(f,g) = -(f - g)^- + g$ \\
	$\Rightarrow$ \ref{messbarkeit_funktion_zusammensetzung_addition_eins}, \ref{messbarkeit_funktion_zusammensetzung_addition_zwei}
	
	\item Zu \ref{messbarkeit_funktion_zusammensetzung_addition_drei}: Verwende \begin{align*}
		\left(\inf_{k\in\mathbb{N}} f_k\right)^{-1}([-\infty,a)) &= \bigcup_{k=1}^\infty f_k^{-1}([-\infty,a]) \\
		\left(\sup_{k\in\mathbb{N}} f_k\right)^{-1}([-\infty,a]) &= \bigcap_{k=1}^\infty f_k^{-1}([-\infty,a])
	\end{align*}
	$\Rightarrow$ $\inf f_k$, $\sup f_k$ messbar.
	
	\item Folglich \begin{align*}
		\left.\begin{aligned}
			\liminf\limits_{a\to\infty} f_k &= \sup\limits_{j\ge 1} \underbrace{\inf\limits_{k\ge j}}_{=g_j} f_k \\
			\limsup\limits_{k\to\infty} f_k &= \adjustlimits{\inf}_{j\ge 1} {\sup}_{k\ge j} f_k
		\end{aligned}\right\} \begin{gathered}
			\text{messbar}
		\end{gathered}
	\end{align*}
	\end{itemize}
	\hfill\csname\InTheoType Symbol\endcsname
\end{proof}

\begin{proposition}[Approximation messbarer Funktionen]
	\proplbl{messbarkeit_funktion_approximation}
	Sei $f:D\subset\mathbb{R}^n\to\overline{\mathbb{R}}$, $D$ messbar. Dann
	\begin{center}
	\begin{tabularx}{\linewidth}{r@{\ \ }c@{\ \ }X}
		\hspace*{0.19\linewidth} $f$ messbar & $\Leftrightarrow$ & $\exists$ Folge $\{ h_k\}$ von Treppenfunktionen mit $h_k\to f$ \gls{fü} auf $D$
	\end{tabularx}
	\end{center}
\end{proposition}
\begin{proof}\hspace*{0pt}
	\begin{itemize}[topsep=\dimexpr-\baselineskip/2\relax]
		\item["`$\Rightarrow$"'] $f$ messbar, somit auch $f^{\pm}$. Setzte mit $h_0^{\pm} := 0$ schrittweise \begin{align*}
			\left.\begin{aligned}
			M_k^{\pm} &:= \left\{ x\in D \;\left|\; f^{\pm}(x) \ge \frac{1}{k} + h_{k-1}^\pm \right.\right\}, \\
			h_k^{\pm} &:= \sum_{j=1}^k \frac{1}{j}\chi_{M_j^{\pm}}
			\end{aligned}\right\}\begin{gathered}
				\;\,\text{für $k\ge 1$}
			\end{gathered}
		\end{align*}
		da $h_{k-1}^\pm$ messbar ist, ist $M_k^{\pm} = \left( f^{\pm} - \frac{1}{k} - h_{k-1}^\pm\right)([0,\infty])$ messbar und 
		$h_k^{\pm}$ ist Treppenfunktion; $f^{\pm}\ge h_k^{\pm}$ auf $D$.
		
		\begin{itemize}
		\item Falls $f^\pm(x) = \infty$, dann $x\in M_k^\pm$ $\forall k\in\mathbb{N}$ und $h_k^\pm(x)\to f^{\pm}(x)$
		
		\item Falls $0\le f^{\pm}(x) < \infty$, dann gilt für unendlich viele $k\in\mathbb{N}$: $x\notin M_k^\pm$, somit $0\le f^{\pm}(x) - h_{k-1}^\pm < \frac{1}{k}$
		
		$\Rightarrow$ $h_k^\pm(x) \to f^{\pm}(x)$\\
		$\Rightarrow$ $h_k^+(x) - h_k^-(x) \to f^+(x) - f^-(x) = f(x)$
		\end{itemize}
		
		\item["`$\Leftarrow$"']
		Sei $\tilde{f}(x) := \limsup\limits_{k\to\infty} h_k(x)$ $\forall x\in D$ $\Rightarrow$ $f(x) = \tilde{f}(x)$ \gls{fü} auf $D$ \\
		Nach \propref{messbarkeit_funktionen_komposition}: $h_k$ messbar $\Rightarrow$ $\tilde{f}$ messbar
		
		Da $f=\tilde{f}$ \gls{fü} folgt $f$ messbar.
	\end{itemize}
\end{proof}

\begin{conclusion}
	\proplbl{messbarkeit_funktion_existenz_monotone_treppenfunktionen}
	Sei $f:D\subset\mathbb{R}^n\to\overline{\mathbb{R}}$ messbar mit $f\ge 0$
	
	$\Rightarrow$ $\exists$ Folge $\{h_k\}$ von Treppenfunktionen mit $0 \le h_1 \le h_2 \le \dotsc \le f$ auf $D$ und $h_k\to f$ \gls{fü} auf $D$.
\end{conclusion}

\begin{proposition}
	\proplbl{messbarkeit_funktion_funktion_messbar}
	Sei $f:D\subset\mathbb{R}^n\to \mathbb{R}$ und $D$ messbar, $N\subset\mathbb{R}^n$ mit $\vert N \vert = 0$ und $f$ stetig auf $D\subset N$
	
	$\Rightarrow$ $f$ messbar auf $D$
\end{proposition}

\begin{proof}
	\NoEndMark
	Offenbar $\tilde{D} = D\setminus N$ messbar. Da $f$ stetig auf $\tilde{D}$ ist, ist $f^{-1}(U)\setminus N$ offen in $\tilde{D}$ für $U\subset \mathbb{R}$ offen, d.h. $f^{-1}(U)\setminus N = M\cap \tilde{D}$ für ein $M\subset\mathbb{R}^n$ offen.
	
	\begin{tabularx}{\linewidth}{r@{\ }X}
	$\Rightarrow$ & $f^{-1}(U)\setminus N$ messbar \\
	$\xRightarrow{\text{\propref{messbarkeit_mengen_satz_acht}}}$ & $f^{-1}(U)$ messbar\\
	$\Rightarrow$ & $f$ messbar.\hfill\csname\InTheoType Symbol\endcsname
	\end{tabularx}
\end{proof}

\begin{example}
	Folgende Funktionen sind messbar
	\begin{itemize}
		\item Stetige Funktionen auf offenen und abgeschlossenen Mengen (wähle $N=\emptyset$ im obigen Satz), insbesondere konstante Funktionen sind messbar
		\item Funktionen auf offenen und abgeschlossenen Mengen, die \gls{fü} mit einer stetigen Funktion übereinstimmen
		\item $\tan$, $\cot$ auf $\mathbb{R}$ (setzte z.b. $\tan\left(\frac{\pi}{2}+k\pi\right) = \cot(k\pi) = 0$ $\forall k$)
		\item $x\to \sin\frac{1}{x}$ auf $[-1,1]$ (setzte beliebigen Wert in $x=0$)
		\item $\chi_M:\mathbb{R}\to\mathbb{R}$ ist für $\vert\partial M\vert = 0$ messbar auf $\mathbb{R}$ (dann ist $\chi$ auf $\inn M$, $\ext M$ stetig)
	\end{itemize}
\end{example}

\begin{underlinedenvironment}[Hinweis]
	Die \person{Dirichlet}-Funktion ist stetig auf $\mathbb{R}\setminus\mathbb{Q}$ und somit nach \propref{messbarkeit_funktion_funktion_messbar} messbar. Man beachte aber, das dies nicht bedeutet, dass die \person{Dirichlet}-Funktion auf $\mathbb{R}$ \gls{fü} stetig ist! (sie ist nirgends stetig auf $\mathbb{R}$)
\end{underlinedenvironment}

\begin{lemma}[Egorov]
	\proplbl{messbarkeit_funktion_egorov}
	Seien $f_k:D\subset\mathbb{R}^n\to\mathbb{R}$ messbar $\forall k\in\mathbb{N}$. Sei $A\subset D$ messbar mit $\vert A \vert < \infty$ und gelte $f_k(x)\to f(x)$ für \gls{fa} $x\in A$
	
	$\Rightarrow$ $\forall \epsilon > 0$ existieren messbare Menge $B\subset A$ mit $\vert A \setminus B \vert < \epsilon$ und $f_k \to f$ gleichmäßig auf $B$.
\end{lemma}

\begin{proof}\hspace*{0pt}
	\begin{itemize}[topsep=\dimexpr -\baselineskip / 2 \relax]
	\item 
	Offenbar $f$ messbar auf $A$ und Mengen \begin{align*}
		M_{m,l} := \bigcup_{j=l}^\infty \left\{ x\in A \;\left| \; \vert f_j(x) - f(x)\vert > \frac{1}{2^m} \right. \right\},\quad m,l\in\mathbb{N}
	\end{align*}
	sind messbar mit $M_{m,1} \supset M_{m,2} \supset \dotsc$ $\forall m\in\mathbb{N}$.
	
	Wegen $f_k(x) \to f(x)$ $\forall x\in A\setminus N$ für eine Nullmenge $N$ folgt $\bigcap_{l\in\mathbb{N}} M_{m,l} \subset N$ und $\vert \bigcap_{l\in\mathbb{N}} M_{m,l} \vert = 0$ $\forall m\in\mathbb{N}$ \\
	$\Rightarrow$ $\forall m\in\mathbb{N}$ $\exists l_m \in\mathbb{N}$ mit $\vert M_{m,l_m} \vert < \frac{\epsilon}{2^m}$ (vgl. \propref{messbarkeit_mengen_ober_unter_mengen} \ref{messbarkeit_mengen_ober_unter_mengen_c})
	
	Mit $M:=\bigcup_{m\in\mathbb{N}} M_{m,l_m}$ und $B:= A\setminus M$ folgt \begin{align*}
		\vert A \setminus B \vert &= \vert M \vert \le \sum_{m=1}^\infty \vert M_{m,l_m} \vert \le \underbrace{\sum_{m=1}^\infty \frac{\epsilon}{2^m}}_{\mathclap{\frac{1}{2^m}\text{ ist geometrische Reihe}}} = \epsilon
	\end{align*}
	\item Weiterhin hat man $\forall m\in\mathbb{N}$ \begin{align*}
		\vert f_k(x) - f(x) \vert \le \frac{1}{2^m}\quad\forall x\in B,\;k \ge l_m
	\end{align*}
	$\Rightarrow$ gleichmäßige Konvergenz auf $B$
	\end{itemize}
\end{proof}

\begin{example}
	Betrachte $f_k(x) = x^k$ auf $[0,1]$.
	
	Man hat $f_k(x) \to 0$ \gls{fü} auf $[0,1]$ und gleichmäßige Konvergenz auf $[0,\alpha]$ $\forall \alpha\in (0,1)$.
\end{example}
