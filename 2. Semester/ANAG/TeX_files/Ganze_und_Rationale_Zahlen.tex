\section{Ganze und rationale Zahlen}
\begin{definition}
	Definiere Äquivalenzrelation $Q:=\{ ((n_1,n_1'),(n_2,n_2'))\in((\mathbb{N}\times\mathbb{N})\times(\mathbb{N}\times\mathbb{N})) | n_1+n_2' = n_1' + n_2 \}$
\end{definition}
\begin{proposition}
	$Q$ ist Äquivalenzrelation auf $\mathbb{N}\times\mathbb{N}$.
\end{proposition}

\begin{proposition}
	Sei $[(n,n')]\in\overline{\mathbb{Z}}$. Dann ex. eindeutige $n^{*}\in\mathbb{N}:(n^{*},0)\in[(n,n')]$ falls $n\geq n'$ bzw. $(0,n^{*})\in[(n,n')]$ falls $n\leq n'$.
\end{proposition}

\subsection*{Rechenoperationen}
\begin{definition}
	\begriff{Addition}[!ganze Zahlen]: $\overline{m}+\overline{n} = [(m,n')] + [(n,n')] :=[(m+n,m'+n')]$
	
	\begriff{Multiplikation}[!ganze Zahlen]: $\overline{m}\cdot\overline{n} = \overline{m}\overline{n} = [(m,m')]\cdot[(n,n')]:=[(mn+m'n',mn'+m'n)]$
\end{definition}

\begin{proposition}
	Addition und Multiplikation sind eindeutig definiert, d.h. unabhängig vom Repräsentanten bzgl. $Q$.
\end{proposition}

\begin{proposition}
	Für Addition und Multiplikation auf $Z$ gilt $\forall \overline{m},\overline{n}\in\overline{\mathbb{Z}}$:
	\begin{enumerate}[label={\arabic*)}]
		\item Es ex. neutrales Element $0:=[(0,0)]$ (Add.), $1:=[(1,0)]$ (Mult., $=[(k,k)]$)
		\item Jeweils kommutativ, assoziativ und gemeinsam distributiv
		\item $-\overline{n} := [(n',n)]\in\overline{\mathbb{Z}}$ ist Inverses bzgl. Addition von $[(n,n')]=\overline{n}$
		\item $(-1)\cdot \overline{n} = -\overline{n}$
		\item $\overline{m}\cdot\overline{n} = 0 \Leftrightarrow \overline{m} = 0 \lor \overline{n} = 0$
	\end{enumerate}
\end{proposition}

\begin{proposition}
	Für $\overline{m},\overline{n}\in\overline{\mathbb{Z}}$ hat Gleichung $\overline{m} = \overline{n} + \overline{x}$ eindeutige Lösung $\overline{x} = \overline{m} + (-\overline{n}) = [(m+n'),(m'+n)]$.
\end{proposition}

\subsection*{Ordnung auf $\overline{\mathbb{Z}}$}
\begin{definition}
	Betr. Relation $R:=\{(\overline{m},\overline{n})\in\overline{\mathbb{Z}}\times\overline{\mathbb{Z}} | \overline{m} \le \overline{n}\}$, wobei $\overline{m} = [(m,m')] \le [(n,n')]$ \gls{gdw} $(m+n'\le m'+n)$
\end{definition}

\begin{proposition}
	$R$ ist Totalordnung auf $\overline{\mathbb{Z}}$, die verträglich ist mit Addition und Multiplikation.
\end{proposition}

\begin{definition}
	Betr. $\mathbb{Z} = \mathbb{Z}\cup\{ (-k) | k\in\mathbb{N}_{>0} \}$ mit üblicher Addition, Multiplikation und Ordnung "`$\ge$"'.
\end{definition}
\begin{proposition}
	$\mathbb{Z},\overline{\mathbb{Z}}$ sind isomorph bzgl. Addition, Multiplikation, Ordnung.
\end{proposition}

\subsection*{Rationale Zahlen}
\begin{definition}
	Betr. Relation $Q:=\left\lbrace \left. \left( \frac{n_1}{n_1'},\frac{n_2}{n_2'}\right) \in \left( \mathbb{Z}\times\mathbb{Z}_{\neq 0}\right)\times\left(\mathbb{Z}\times\mathbb{Z}_{\neq 0}\right) \right| n_1n_2' = n_1'n_2\right\rbrace$
	
	Setzte $\mathbb{Q} := \left\lbrace \left[ \left. \frac{n}{n'}\right] \right| (n,n')\in\mathbb{Z}\times\mathbb{Z}_{\neq 0}\right\rbrace$ Menge der \begriff{rationale Zahlen}.
	
	Offenbar gilt \begriff{Kürzungsregel}[!rationale Zahlen] $\left[ \frac{n}{n'}\right] = \left[ \frac{k\cdot n}{k\cdot n'}\right]\quad\forall k\in\mathbb{Z}_{\neq 0}$.
\end{definition}

\subsection*{Rechenoperationen auf $\mathbb{Q}$}
\begin{definition}
	\begriff{Addition}[!rationale Zahlen]: $\left[ \frac{m}{m'}\right] + \left[ \frac{n}{n'}\right] := \left[ \frac{mn' + m'n}{m'+n'}\right]$
	
	\begriff{Multiplikation}[!rationale Zahlen]: $\left[\frac{m}{m'}\right]\cdot\left[\frac{n}{n'}\right]:=\left[\frac{m\cdot n}{m'\cdot n'}\right]$
	
	Addition und Multiplikation sind unabhängig vom Repräsentanten bzgl. $Q$ $\Rightarrow$ Operationen auf $Q$ eindeutig definiert.
\end{definition}

\begin{proposition}
	Mit Addition und Multiplikation ist $\mathbb{Q}$ Körper mit
	\begin{itemize}
	\item neutralem Element $0:=\left[\frac{0_\mathbb{Z}}{1_\mathbb{Z}}\right] = \left[\frac{0_\mathbb{Z}}{n}\right], 1 :=\left[\frac{1_\mathbb{Z}}{1_\mathbb{Z}}\right] = \left[ \frac{n}{n}\right] \neq 0\;n\neq 0$
	\item Inverse Elemente $-\left[\frac{n}{n'}\right] = \left[ \frac{-n}{n'}\right], \left[\frac{n}{n'}\right]^{-1} = \left[\frac{n'}{n}\right]$
	\end{itemize}
\end{proposition}

\subsection*{Ordnung auf $\mathbb{Q}$}
\begin{definition}
	Relation $R:=\left\lbrace \left. \left( \left[\frac{m}{m'}\right],\left[\frac{n}{n'}\right]\right)\in\mathbb{Q}\times\mathbb{Q} \right| mn'\le m'n'; m',n'>0\right\rbrace$ gibt Ordnung "`$\le$"'.
\end{definition}

\begin{proposition}
	$\mathbb{Q}$ ist angeordneter Körper ("`$\leq$"') ist Totalordnung verträglich mit Addition und Multiplikation).
\end{proposition}
\begin{conclusion}
	Körper $\mathbb{Q}$ ist \begriff[Körper!]{archimedisch angeordnet}, d.h. $\forall q\in\mathbb{Q} \, \exists n\in\mathbb{N}: q < n$.
\end{conclusion}