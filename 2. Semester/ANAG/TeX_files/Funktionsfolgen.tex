\section{Funktionsfolgen}\setcounter{equation}{0}

Betrachte $f_k:D\subset K^n\to K^m$, $D$ offen, $f_k$ \gls{diffbar} für $k\in\mathbb{N}$

\begin{boldenvironment}[Frage:]
	Wann konvergiert $\{ f_k\}_{k\in\mathbb{N}}$ gegen \gls{diffbar}e Funktion $f$ mit $f_k'\to f'$
\end{boldenvironment}

\begin{boldenvironment}[Wiederholung]
	alle $f_k$ stetig, $f_k\to f$ gleichmäig auf $D$ $\xRightarrow{\text{\propref{chap_14_19}}}$ $f$ stetig
\end{boldenvironment}

\begin{example}
	Sei $f_k:\mathbb{R}\to\mathbb{R}$ mit $f_k(x) = \frac{\sinh^2 x}{k}$.
	
	Wegen $\vert f_k(x)\vert \le \frac{1}{k}$ $\forall k$ $\Rightarrow$ $f_k\to f$ gleichmäßig auf $\mathbb{R}$ für $f=0$

	Aber $f_k'(x) = k\cdot\cosh^2 x \cancel{\rightarrow} f'(x) = 0$
\end{example}

\begin{proposition}[Differentiation bei Funktionsfolgen]
	\proplbl{funktionsfolgen_differentiation}
	Sei $f_K:D\subset K^n\to K^n$, $D$ offen, beschränkt, $f_k$ \gls{diffbar} $\forall k$ und\begin{enumerate}[label={(\alph*)}]
		\item $f_k'\rightarrow: g$ gleichmäßig auf $B_r(x)\subset D$
		\item \proplbl{funktionsfolgen_differentiation_b} $\{ f_k(x_0)\}$ konvergiert für ein $x_0\in B_r(x)$
	\end{enumerate}
	$\Rightarrow$ $f_k\rightarrow: f$ gleichmäßig auf $B_r(x)$ und $f$ ist \gls{diffbar} auf $B_r(x)$ mit \begin{align*}
		f_k'(y) \rightarrow f'(y) \quad\forall y\in B_r(x)
	\end{align*}
\end{proposition}

\begin{underlinedenvironment}[Hinweis]
	Betrachte $f_k(x) := \frac{x^k}{k^2} + k$ auf $(0,1)$ um zu sehen, dass Voraussetzung \ref{funktionsfolgen_differentiation_b} wichtig ist.
\end{underlinedenvironment}

\begin{proof}
	Für $\epsilon > 0$ $\exists k\in \mathbb{N}$ mit \begin{align}
		\proplbl{funktionsfolgen_differentiation_beweis_1}
		\vert f_k(x_0) - f_l(x_0)\vert < \epsilon \quad k,l \ge k_0
	\end{align}
	Weiter gilt (eventuell für größeres $k_0$) $\Vert g(z) - f_k'(z)\Vert < \epsilon$ und  \begin{align}
		\proplbl{funktionsfolgen_differentiation_beweis_2}
		\Vert f_k'(y) - f_l'(y)\Vert < \epsilon \quad\forall k,l\ge k_0,\;z,y\in B_r(x)
	\end{align}
	Schrankensatz: $\forall z,y\in B_r(x)$, $k,l\ge k_0$ $\exists \xi\in [z,y]$ mit {\zeroAmsmathAlignVSpaces**\begin{align}
		\proplbl{funktionsfolgen_differentiation_beweis_3}
		\left\vert \left(f_k(y) - f_l(y)\right) - \left( f_k(z) - f_l(z)\right)\right\vert \le \Vert f_k'(\xi) - f_l'(\xi)\Vert \cdot \vert y - z\vert \le \epsilon \vert y -z\vert < 2r\cdot\epsilon
	\end{align}}
	{\zeroAmsmathAlignVSpaces*\begin{flalign}
		\notag\Rightarrow\;\;\vert f_k(y) - f_l(y)\vert &\le \vert \big(f_k(y) - f_l(y)\big) - \big(f_k(x_0) - f_l(x_0)\big)\vert + \vert f_k(x_0) - f_l(x_0)\vert& \\
		\proplbl{funktionsfolgen_differentiation_beweis_4}
		&\le 2r \epsilon + \epsilon = \epsilon (2r + 1)\quad y\in B_r(x),\; k,l\ge k_0&
	\end{flalign}}
	$\Rightarrow$ $\{ f_k(y)\}_{k\in\mathbb{N}}$ ist \gls{cf} in $K^m$ $\forall y$ \\
	$\Rightarrow$ $f_k(y) \xrightarrow{k\to\infty}: f(y)$ $\forall y\in B_r(x)$
	
	Mit $l\to\infty$ in \eqref{funktionsfolgen_differentiation_beweis_4}: $f_k\to f$ gleichmäßig auf $B_r(x)$
	
	Fixiere $\tilde{x}\in B_r(x)$, $k=k_0$. Dann liefert $l\to\infty$ in \eqref{funktionsfolgen_differentiation_beweis_3} \begin{align*}
		\vert f(y) - f(\tilde{x}) - \big( f_k(y) - f_k(\tilde{x} \big) \vert \le \epsilon \vert y - \tilde{x}\vert \quad\forall y\in B_r(x)
	\end{align*}
	Da $f_k$ \gls{diffbar} $\exists \rho = \rho (\epsilon) > 0$ mit {\zeroAmsmathAlignVSpaces**\begin{align*}
		\vert f_k(y) - f_k(\tilde{x}) - f_k'(\tilde{x})\cdot(y - \tilde{x})\vert \le \epsilon \vert y - \tilde{x}\vert\quad\forall y\in B_\rho(\tilde{x})\subset B_r(x)
	\end{align*}}
	{\zeroAmsmathAlignVSpaces*\begin{flalign}
		\notag \Rightarrow\;\; \vert f(y) - f(\tilde{x}) - g(\tilde{x}) \cdot(y-\tilde{x})\vert &\le 
		\begin{multlined}[t][\dimexpr\linewidth/2]\vert f(y) - f(\tilde{x})\vert  + \vert f_k(y) - f_k(\tilde{x}))\vert \\
		+ \vert f_k(y) - f_k(\tilde{x}) - f_k'(\tilde{x}) \cdot (y - \tilde{x})\vert \\
		  + \vert f_k'(\tilde{x})\cdot (y - \tilde{x}) - g(\tilde{x})(y - \tilde{x})\vert\end{multlined}& \\
		\proplbl{funktionsreihe_differentiation_beweis_5}
		&\le \epsilon\vert y - \tilde{x}\vert + \epsilon \vert y - \tilde{x}\vert + \epsilon \vert y - \tilde{x}\vert = 3\epsilon \vert y - \tilde{x}\vert \quad\forall y\in B_\rho(\tilde{x})&
	\end{flalign}}
	
	Beachte: $\forall \epsilon > 0$ $\exists \rho > 0$ und mit \eqref{funktionsreihe_differentiation_beweis_5}
	
	\begin{tabularx}{\linewidth}{r@{\ \ }X}
		$\Rightarrow$ & $f(y) - f(\tilde{x}) - g(\tilde{x})\cdot(y - \tilde{x}) = o(\vert y -\tilde{x}\vert)$, $y\to \tilde{x}$ \\
		$\Rightarrow$ & $f(\tilde{x}) = g(\tilde{x})$ $\xRightarrow{\text{$\tilde{x}$ beliebig}}$ Behauptung
	\end{tabularx}
\end{proof}

\subsection{Anwendung auf Potenzreihen}
Sei $f:B_r(x_0)\subset K\to K$ gegen durch eine Potenzreihe \begin{align}
	\proplbl{funktionsfolgen_potenzreihe}
	f(x) &= \sum_{k=0}^\infty a_k(x  - x_0)^k\quad\forall x\in B_{\underbrace{\text{\scriptsize$R$}}_{\mathclap{\text{Konvergenzradius}}}}(x_0)
\end{align}

\begin{boldenvironment}[Wiederholung]
	$R=\frac{1}{\limsup \sqrt[k]{\vert a_k\vert}}$
\end{boldenvironment}

\begin{boldenvironment}[Frage]
	Ist $f$ \gls{diffbar} und kann man gliedweise differenzieren?
\end{boldenvironment}

\begin{proposition}
	\proplbl{funktionsfolgen_satz_3}
	Sei $f:B_r(x_0)\subset K\to K$ Potenzreihe gemäß \eqref{funktionsfolgen_potenzreihe} \\
	\hspace*{1.5ex}$\Rightarrow$ $f$ ist \gls{diffbar} auf $B_r(x_0)$ mit \begin{align}
		\proplbl{funktionsfolgen_satz_3_eq}
		f'(x) &= \sum_{k=1}^\infty k a_k (x - x_0)^{k-1}\quad\forall x\in B_r(x_0)
	\end{align}
\end{proposition}

\begin{conclusion}
	Sei $f:B_r(x_0)\subset K\to K$ Potenzreihe gemäß \eqref{funktionsfolgen_potenzreihe} \\
	\hspace*{1.5ex}$\Rightarrow$ $f\in C^\infty (B_r(x_0), K)$ und \begin{align}
		\proplbl{funktionsfolgen_eq_8}
		a_k = \frac{1}{k!}\cdot f^{(k)})(x_0)
	\end{align}
	(d.h die Potenzreihe stimmt mit der Taylorreihe von $f$ in $x_0$ überein)
\end{conclusion}

\begin{proof}
	$k$-fache Anwendung von \cref{funktionsfolgen_satz_3} liefert $f\in C^k(B_r(x_0), K)$ $\forall k\in \mathbb{N}$\\
	\ $\xRightarrow{\eqref{funktionsfolgen_satz_3_eq}}$ $f'(x) = a_1$, $f''(x_0) = 2a_k$, $\dotsc$ rekursiv folgt \eqref{funktionsfolgen_eq_8}.
\end{proof}

\begin{proof}[\propref{funktionsfolgen_satz_3}]
	Betrache die Partialsummen\begin{align*}
		f_k(x) := \sum_{j=0}^k a_j(x- x_0)^j\quad\forall x\in B_r(x_0)
	\end{align*}
	\ $\Rightarrow$ $f_k(x_0)\xrightarrow{k\to\infty} f(x_0)$ und $f_k$ \gls{diffbar} mit \begin{align*}
		f_k'(x) = \sum_{j=1}^k j a_j(x - x_0)^{j-1}\quad\forall x\in B_r(x_0)
	\end{align*}
	Wegen \begin{align*}
	\limsup\limits_{k\to\infty} \sqrt[k]{(k+1)\vert a_{k+1}\vert} = \limsup \sqrt[k]{k\left(1 + \frac{1}{k}\right)} \cdot \left( \sqrt[k+1]{\vert a_{k+1}\vert}\right)^{\frac{k+1}{k}} = \limsup \sqrt[k]{\vert a_k\vert} = \frac{1}{R}
	\end{align*}
	hat die Potenzreihe \begin{align*}
		g(x) := \sum_{k=1}^\infty k a_k(x-x_0)^{k-1}
	\end{align*}
	den Konvergenzradius $R$ \\
	\begin{tabularx}{\linewidth}{r@{\ \ }X}
	\ $\Rightarrow$ & Reihe $g$ konvergiert gleichmäßig auf $B_r(x_0)$ $\forall r\in (0,R)$ (vgl. 13.1), d.h. $f_k'\to g$ gleichmäßig auf $B_r(x_0)$ \\
	$\xRightarrow{\text{\cref{funktionsfolgen_differentiation}}}$&  $f$ ist \gls{diffbar} auf $B_r(x_0)$ mit \eqref{funktionsfolgen_satz_3_eq} auf $B_r(x_0)$.
	\end{tabularx}
	
	Da $r\in(0,R)$ beliebig, folgt die Behauptung.
\end{proof}

\begin{example}
	Es gilt \begin{align}
		\proplbl{funktionsfolgen_eq_9}
		\ln(1+x) = \sum_{k=0}^\infty \frac{(-1)^k}{k+1}x^{k+1}\quad\forall x\in (-1,1)\subset \mathbb{R}
	\end{align}
\end{example}
	
\begin{proof}
	$f(x)$ sei Potenzreihe \eqref{funktionsfolgen_eq_9}, hat Konvergenzradius $R=1$, $x_0=0$ \\
	$\xRightarrow{\text{\cref{funktionsfolgen_satz_3}}}$ $f$ \gls{diffbar} auf $(-1,1)$ und \begin{align*}
		f'(x) = \sum_{k=0}^\infty (-x)^k = \frac{1}{1-(-x)} = \frac{1}{1+x}\qquad\text{geometrische Reihe}
	\end{align*}
	und\begin{align*}
		\frac{\D}{\D x}\ln (1+x) &= \frac{1}{1+x} = f'(x) \\
		f(x) &= \ln(1+x) + \mathrm{const}
	\end{align*}
	Wegen $f(0) = 0=\ln 1$ $\Rightarrow$ $f(x) = \ln(1+x)$ $\forall x\in(-1,1)$, d.h. \eqref{funktionsfolgen_eq_9} gilt.
\end{proof}