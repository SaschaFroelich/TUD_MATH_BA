\section{Grundbegriffe aus Logik und Mengenlehre}

\begin{definition}[Aussage]
	\begriff{Aussage} ist ein Schverhalt, dem man entweder den Warheitswert wahr ($w$) oder falsch ($f$) zuordnen kann (und nichts anderes).
\end{definition}

\addtocounter{theorem}{1}
	
\begin{definition}[Menge]
	\begriff{Menge} ist (nach Cantor 1877) eine Zusammenfassung von bestimmten, wohlunterschiedenen Objekten der Anschauung oder des Denkens, welche die \begriff{Elemente} der Menge genannt werden, zu einem Ganzen.
\end{definition}

\addtocounter{theorem}{1}

\begin{definition}
	\begin{itemize}
		\item $M=N$, falls dieselben Elemente enthalten sind
		\item $N$\mathsymbol{c}{$\subset$}$M$ (\begriff{Teilmenge}), falls $n\in M$für jedes $n\in\mathbb{N}$
		\item $N$\mathsymbol{c=}{$\subsetneqq$}$M$ (\begriff{echte Teilmenge}), falls zusätzlich $N\neq M$.
		\item \begriff{Aussageform}: Sachverhalt mit Variablen, der durch geeignete Ersetzung der Variablen zur Aussage führt
	\end{itemize}
\end{definition}

\addtocounter{theorem}{1}

\begin{definition}[Quantoren]
	\begriff{Quantoren}
	\begin{itemize}
		\item $\forall x\in M: A(x)$ wahr \gls{gdw} $A(x)$ wahr für jedes $x\in M$
		\item $\exists x\in M: A(x)$ wahr \gls{gdw} $A(x)$ wahr für mindestens ein $x\in M$
	\end{itemize}
\end{definition}

\begin{definition}
	\begriff{Tautologie} bzw. \begriff{Kontradiktion}\slash\begriff{Widerspruch} ($\Lightning$) ist zusätzlich gesetzte Aussage, die unabhängig vom Wahrheitswert der Teilaussagen stets wahr bzw. falsch ist.
\end{definition}

\begin{proposition}[\person{de Morgan}'sche Regeln]
	Folgende Aussagen sind stets Tautologien
	\begin{enumerate}[label={\alph*)}]
		\item $\neg(A\land B) \Leftrightarrow \neg A \lor \neg B$
		\item $\neg(A\lor B) \Leftrightarrow \neg A\land \neg B$
		\item $\neg (\forall x\in M: A(x)) \;\Leftrightarrow \; \exists x\in M:\neg A(x)$
		\item $\neg (\exists x\in M: A(x)) \;\Leftrightarrow \;\forall x\in M:\neg A(x)$
	\end{enumerate}
\end{proposition}

\begin{proof}
	Übung
\end{proof}

\begin{definition}
	\begin{itemize}
		\item \begriff{leere Menge} \mathsymbol{o}{$\emptyset$}$=:$ Menge, die kein Element enthält
		\item $M,N$ sind \begriff[Menge!]{disjunkt}, falls $M\cap N = \emptyset$
		\item Sei $\mathcal{M}$ \begriff{Mengensystem}, d.h. Mengen von Mengen, dann
		\begin{itemize}
			\item $\bigcup_{M\in\mathcal{M}} M := \{x \mid \exists M\in\mathcal{M}: x\in M\}$
			\item $\bigcap_{M\in\mathcal{M}} M:= \{ x\mid\forall M\in\mathcal{M}: x\in M \}$
		\end{itemize}
		\item \begriff{Potenzmenge}: \mathsymbol{p}{$\mathcal{P}$}$(XM):=\{\tilde{M} | \tilde{M}\in M\}$
		\item \begriff{\person{de Morgan}'sche Regeln} (für $\mathcal{N}\subset\mathcal{P}(M)$)
		\begin{itemize}
			\item $\left(\bigcup_{N\in\mathcal{N}} N\right)^C = \bigcap_{N\in\mathcal{N}} N^C$
			\item $\left(\bigcap_{N\in\mathcal{N}} N\right)^C = \bigcup_{N\in\mathcal{N}} N^C$
		\end{itemize}
		\item \begriff{kartesisches Produkt} $M$\mathsymbol{x}{$\times$}$N:=\{(m,n) | m\in M \text{ und } n\in N\}$
		\item $(m_1, \dotsc, m_n)$ ist \begriff{n-Tupel}
		\item \begriff{Auswahlaxiom} (AC / axiom of choice)
		
		Sei $\mathcal{M}$ Menge nichtleerer, paarweise disjunkter Mengen $M$\\
		$\Rightarrow$ es gibt immer (Auswahl-) Menge $\tilde{M}$, die mit jedem $M\in\mathcal{M}$ genau ein Element gemein hat.
	\end{itemize}
\end{definition}

\begin{example}
	\begin{itemize}
		\item Für Aussagen $A,B,C$: $A\land C \Rightarrow B$
		\begin{itemize}
			\item $B$ ist \begriff[Bedingung!]{notwendig} für $A$
			\item $A$ ist \begriff[Bedingung!]{hinreichend} für $B$
		\end{itemize}
	\end{itemize}
\end{example}

\subsection*{Mathematische Beweise}
\begin{definition}
	\begin{enumerate}
		\item \begriff[Beweis!]{direkt}\highlight{er Beweis}: $(A\Rightarrow A_1)\land(A_1\Rightarrow A_2)\land\dotsc\land(A_n\Rightarrow B)$ wahr für $A\Rightarrow B$
		\item \begriff[Beweis!]{indirekt}\highlight{er Beweis} durch Tautologie $(A\Rightarrow B)\Leftrightarrow (\neg B\rightarrow \neg A)$
	\end{enumerate}
\end{definition}

\subsection*{Relation und Funktion}
\begin{definition}[Relation]
	\begin{itemize}
		\item \begriff{Relation} ist Teilmenge $R\subset M\times N$. $(x,y)\in R$ heißt: $x$ und $y$ stehen in Relation zueinander.
		\item Relation $R\subset M\times N$ heißt \begriff{Ordnungsrelation} (kurz \begriff{Ordnung}) auf $M$, falls $\forall a,b,c\in M$:
		\begin{enumerate}[label={\alph*)}]
			\item $(a,a)\in R$ (\begriff[Ordnung!]{reflexiv})
			\item $(a,b),(b,a)\in R \rightarrow a=b$ (\begriff[Ordnung!]{antisymmetrisch})
			\item $(a,b),(b,c)\in R \rightarrow (a,c)\in R$ (\begriff[Ordnung!]{transitiv})
		\end{enumerate}
		\item Ordnungsrelation $R$ auf $M$ heißt \begriff{Totalordnung}, falls $\forall a,b\in M: (a,b)\in R \lor (b,a)\in R$
		\item Relation auf $M$ heißt \begriff{Äquivalenzrelation}, falls $\forall a,b,c\in M$:
		\begin{enumerate}[label={\alph*)}]
			\item $(a,a)\in R$ (\begriff[Ordnung!]{reflexiv})
			\item $(a,b)\in R \Rightarrow (b,a)\in R$ (\begriff[Ordnung!]{symmetrisch})
			\item $(a,b),(b,c)\in R \Rightarrow (a,c)\in R$ (\begriff[Ordnung!]{transitiv})
		\end{enumerate}
		\item \mathsymbol{[a]}{$[a]$}$:=\{b\in M\mid (a,b)\in R\}$ heißt \begriff{Äquivalenzklasse} von $a\in M$ bzgl. $R$
		
		Jedes $b\in [a]$ ist ein \begriff{Repräsentant} von $[a]$
	\end{itemize}
\end{definition}

\begin{definition}[Abbildung]
	\begriff{Abbildung}/\begriff{Funktion} von $M$ nach $N$, kurz: $F:M\rightarrow N$ ist Vorschrift, die jedem \begriff{Argument} / \begriff{Urbild} $m\in M$ genau einen \begriff{Wert} / \begriff{Bild} $F(m)\in N$ zuordnet.
	
	\begin{itemize}
		\item \mathsymbol{D}{$\mathcal{D}$}$(F):=M$ heißt \begriff{Definitionsbereich} / \begriff{Urbildmenge}
		\item $N$ heißt \begriff{Zielbereich}
		\item $F(M'):=\{n\in N \mid n=F(m)$ für ein $m\in M'\}$ ist \begriff{Bild}\highlight{ von $M'$}$\subset M$
		\item $F^{-1}(N'):=\{ m\in M\mid n=F(m)$ für ein $N' \}$ ist \begriff{Urbild}\highlight{ von $N'$}$\subset N$
		\item \mathsymbol{R}{$\mathcal{R}$}$(F):= F(M)$ heißt \begriff{Wertebereich} / \begriff{Bildmenge}
		\item \mathsymbol{graph}{$\graph$}$(F) :=\{ (mn,)\in M\times N | n = F(m)\}$ heißt \begriff{Graph}\highlight{von $F$}
		\item \mathsymbol{fm}{$F|_{M'}$} ist \begriff{Einschränkung}\highlight{der Funktion} von $F$ auf $M'\subset M$
		\item \begriff{Komposition} von $F:M\rightarrow N$ und $G:N\rightarrow P$ ist Abbildung $G$\mathsymbol{o}{$\circ$}$F:M\rightarrow P$ mit $(G\circ F)(m):=G(F(m))$
		\item $Abbildung F:M\rightarrow N$ heißt
		\begin{itemize}
			\item \begriff[Abbildung!]{injektiv}, falls eineindeutig (d.h. $F(m_1) = F(m_2) \Rightarrow m_1 = m_2$)
			\item \begriff[Abbildung!]{surjektiv}, falls $F(M) = N$, d.h. $\forall n\in N\,\exists m\in M: F(m) = n$
			\item \begriff[Abbildung!]{bijektiv}, falls injektiv und surjektiv
		\end{itemize}
		\item Für bijektive Abb. $F:M\rightarrow N$ ist \begriff{Umkehrabbildung} / \begriff{inverse Abbildung} \mathsymbol{f-1}{$F^{-1}$}$:N\rightarrow M$ definiert durch $F^{-1}(n) = m \Leftrightarrow F(m) = n$
	\end{itemize}
\end{definition}

\stepcounter{theorem}
\begin{proposition}
	Sei $F:M\rightarrow N$ surjektiv. Dann existiert Abbildung $G:N\rightarrow M$, sodass $F\circ G = \id_N$ (d.h. $F(G(n)) = n\,\forall n\in N$)
\end{proposition}

\begin{proof}
	Definiere Menge $\Gamma_n = \{m \in M \mid F(m) = n\} \overset{\text{surjektiv}}{\neq} \emptyset$. Nach Auswahlaxiom 	existiert Abbildung $G: N \to M$ mit $G(n) \in \Gamma_n;\ \forall n \in N \Rightarrow F(G(n)) = n;\ \forall n \in N 	\Rightarrow$ Behauptung.
\end{proof}



\begin{definition}[Verknüpfung]
	Eine \begriff{Rechenoperation} / \begriff{Verknüpfung} auf $M$ ist Abb. $*:M\times M\rightarrow M$, d.h. $m,n\in M$ wird \begriff{Ergebnis} $m*n\in M$
	
	Rechenoperation
	\begin{itemize}
		\item hat \begriff[Verknüpfung!]{neutrales Element} $e\in M$, falls $m*e = e*m = m\,\forall m\in M$
		\item ist \begriff[Verknüpfung!]{kommutativ}, falls $m*n = n*m$
		\item ist \begriff[Verknüpfung!]{assoziativ}, falls $k*(m*n) = (k*m)*n\,\forall k,m,n\in M$
		\item hat \begriff[Verknüpfung!]{inverses Element} $m'\in M$ zu $m\in M$, falls $m*m' = m'*m = e$
	\end{itemize}
\end{definition}

\begin{*example}
	\begin{enumerate}[label={\alph*)}]
		\item \begriff{Addition}: $(m,n)\mapsto: m+n$ \begriff{Summe},
		\begin{itemize}
			\item neutrales Element heißt \begriff{Null} / \begriff{Nullelement}
			\item Inverses Element: \mathsymbol{-}{$-m$}
		\end{itemize}
		\item \begriff{Multiplikation} $\cdot:(m,n)\mapsto: m\cdot n$ \begriff{Produkt}
		\begin{itemize}
			\item neutrales Element heißt \begriff{Eins} / \begriff{Einselement}
			\item Inverses Element:\mathsymbol{-1}{$m^{-1}$}
		\end{itemize}
	\end{enumerate}
\end{*example}

\begin{definition}
	Addition und Multiplikation heißen \begriff{distributiv}, falls $k\cdot(m+n) = k\cdot m + k\cdot n\,\forall k,m,n\in M$
\end{definition}

\begin{definition}[Körper]
	Menge $K$ heißt \begriff{Körper}, falls auf $K$ eine Addition und Multiplikation existiert mit
	\begin{enumerate}[label={\alph*}]
		\item es existieren neutrale Elemente $0\in K$ und $1\in K_{\neg 0}$
		\item Addition und Multiplikation sind distributiv
		\item Es gibt Inverse
	\end{enumerate}
\end{definition}

\begin{definition}
	Menge $M$ habe Ordnung "`$\le$"', sowie Addition und Multiplikation.
	
	Ordnung ist \begriff[Ordnung!]{verträglich}\highlight{mit Addition und Multiplikation}, wenn $\forall a,b,c\in M$
	\begin{enumerate}[label={(\alph*)}]
		\item $a\le b \Leftrightarrow a+c \le b+c$
		\item $a\le b \Leftrightarrow a\cdot c \le b\cdot c$ mit $c > 0$
	\end{enumerate}
\end{definition}

\begin{definition}
	Körper $K$ heißt \begriff[Körper!]{angeordnet}, falls mit Addition und Multiplikation verträgliche Totalordnung existert.
\end{definition}

\begin{definition}[Isomorphismus]
	\begriff{Isomorphismus} bezüglich einer Struktur ist bijektive Abbildung $I:M_1\rightarrow M_2$, die auf $M_1$ und $M_2$ vorhandene Struktur erhält.
	
	Mengen $M_1$ und $M_2$ heißen \begriff[Menge!]{isomorph}.
\end{definition}