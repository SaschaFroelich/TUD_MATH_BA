\section{Grundbegriffe aus Logik und Mengenlehre}

\textbf{Mengenlehre:} Universalität von Aussagen, Verwendung von Mengen \\
\textbf{Logik:} Regeln des Folgerns, wahre und falsche Aussagen \\
$\to$ hier werden einige Aspekte etwas vereinfacht, aber ausreichend genug behandelt

\begin{definition}[Aussage]
	\begriff{Aussage} ist ein Schverhalt, dem man entweder den Warheitswert wahr ($w$) oder falsch ($f$) zuordnen kann (und nichts anderes).
\end{definition}

\begin{example}
	\begin{itemize}
		\item 5 ist eine Quadratzahl (Aussage) $\to$ falsch
		\item Die Elbe fließt durch Dresden (Aussage) $\to$ wahr
		\item Mathematik ist rot (keine Aussage)
	\end{itemize}
\end{example}
	
\begin{definition}[Menge]
	\begriff{Menge} ist (nach Cantor 1877) eine Zusammenfassung von bestimmten, wohlunterschiedenen Objekten der Anschauung oder des Denkens, welche die \begriff{Elemente} der Menge genannt werden, zu einem Ganzen.
\end{definition}

\begin{example}
	\begin{itemize}
		\item $M_1=$ Menge aller Städte in Deutschland
		\item $M_2=\{1,2,3\}$
	\end{itemize}
\end{example}

\begin{definition}
	\begin{itemize}
		\item $M=N$, falls dieselben Elemente enthalten sind
		\item $N$\mathsymbol{c}{$\subset$}$M$ (\begriff{Teilmenge}), falls $n\in M$ für jedes $n\in\mathbb{N}$
		\item $N$\mathsymbol{c=}{$\subsetneqq$}$M$ (\begriff{echte Teilmenge}), falls zusätzlich $N\neq M$.
		\item \begriff{Aussageform}: Sachverhalt mit Variablen, der durch geeignete Ersetzung der Variablen zur Aussage führt
	\end{itemize}
\end{definition}

\begin{example}
	\begin{itemize}
		\item $A(X)=$ Die Elbe fließt durch $X$
		\item $B(X,Y,Z)=X+Y=Z$
		\item $\to$ $A(\text{Dresden})$ und $B(2,3,4)$ sind Aussagen
		\item $\to$ $A(\text{Mathematik})$ ist keine Aussage
		\item $\to$ $A(X)$ ist Aussage für jedes $X\in M_1$
	\end{itemize}
\end{example}

\begin{center}\begin{tabular}{|c c|c c c c c|}
	\hline 
	$A$& $B$ & $\neg A$ & $A\land B$ & $A\lor B$ & $A\Rightarrow B$ & $A\iff B$ \\ 
	\hline 
	W& W & F & W & W & W & W \\ 
	W& F & F & F & W & F &  F\\ 
	F& W & W & F & W & W & F \\ 
	F& F & W & F & F & W &  W\\ 
	\hline 
\end{tabular}\end{center}

\begin{example}
	\begin{itemize}
		\item $\neg$(3 ist gerade) - wahr
		\item (4 ist gerade) $\land$ (4 ist Primzahl) - falsch
		\item (3 ist gerade) $\lor$ (3 ist Primzahl) - wahr
		\item (Sonne ist heißt) $\Rightarrow$ (Es gibt Primzahlen) - w
		\item (3 ist gerade) $\iff$ ($\pi\in\natur$) - w
		\item Ausschließendes oder wird realisiert durch $\neg(A\iff B)$
	\end{itemize}
\end{example}

\begin{definition}[Quantoren]
	Neue Aussagen können mittels \begriff{Quantoren} gebildet werden:
	\begin{itemize}
		\item $\forall x\in M: A(x)$ wahr \gls{gdw} $A(x)$ wahr für jedes $x\in M$
		\item $\exists x\in M: A(x)$ wahr \gls{gdw} $A(x)$ wahr für mindestens ein $x\in M$
	\end{itemize}
\end{definition}

\begin{example}
	\begin{itemize}
		\item $\foralln\in\natur:n$ ist gerade - f
		\item $\exists n\in\natur:n$ ist gerade - w
	\end{itemize}
\end{example}

\begin{definition}[Tautologie, Kontraduktion]
	\begriff{Tautologie} bzw. \begriff{Kontradiktion}\slash\begriff{Widerspruch} ($\lightning$) ist zusätzlich gesetzte Aussage, die unabhängig vom Wahrheitswert der Teilaussagen stets wahr bzw. falsch ist.
\end{definition}

\begin{example}
	\begin{itemize}
		\item Tautologien: $A\lor\neg A$, $\neg(A\land\neg A)$, $(A\land B)\Rightarrow A$
		\item Widerspruch: $A\land\neg A$, $A\iff\neg A$
		\item besondere Tautologie: $(A\Rightarrow B)\iff (\neg B\Rightarrow \neg A)$
	\end{itemize}
\end{example}

\begin{proposition}[\person{de Morgan}'sche Regeln]
	Folgende Aussagen sind stets Tautologien
	\begin{enumerate}[label={\alph*)}]
		\item $\neg(A\land B) \Leftrightarrow \neg A \lor \neg B$
		\item $\neg(A\lor B) \Leftrightarrow \neg A\land \neg B$
		\item $\neg (\forall x\in M: A(x)) \;\Leftrightarrow \; \exists x\in M:\neg A(x)$
		\item $\neg (\exists x\in M: A(x)) \;\Leftrightarrow \;\forall x\in M:\neg A(x)$
	\end{enumerate}
\end{proposition}

\begin{proof}
	Übung
\end{proof}

\begin{definition}
	\begin{itemize}
		\item \begriff{leere Menge} \mathsymbol{o}{$\emptyset$}$=:$ Menge, die kein Element enthält
		\item $M,N$ sind \begriff[Menge!]{disjunkt}, falls $M\cap N = \emptyset$
		\item Sei $\mathcal{M}$ \begriff{Mengensystem}, d.h. Mengen von Mengen, dann
		\begin{itemize}
			\item $\bigcup_{M\in\mathcal{M}} M := \{x \mid \exists M\in\mathcal{M}: x\in M\}$
			\item $\bigcap_{M\in\mathcal{M}} M:= \{ x\mid\forall M\in\mathcal{M}: x\in M \}$
		\end{itemize}
		\item \begriff{Potenzmenge}: \mathsymbol{p}{$\mathcal{P}$}$(XM):=\{\tilde{M} | \tilde{M}\in M\}$
		\item \begriff{\person{de Morgan}'sche Regeln} (für $\mathcal{N}\subset\mathcal{P}(M)$)
		\begin{itemize}
			\item $\left(\bigcup_{N\in\mathcal{N}} N\right)^C = \bigcap_{N\in\mathcal{N}} N^C$
			\item $\left(\bigcap_{N\in\mathcal{N}} N\right)^C = \bigcup_{N\in\mathcal{N}} N^C$
		\end{itemize}
		\item \begriff{kartesisches Produkt} $M$\mathsymbol{x}{$\times$}$N:=\{(m,n) | m\in M \text{ und } n\in N\}$
		\item $(m_1, \dotsc, m_n)$ ist \begriff{n-Tupel}
		\item \begriff{Auswahlaxiom} (AC / axiom of choice)
		
		Sei $\mathcal{M}$ Menge nichtleerer, paarweise disjunkter Mengen $M$\\
		$\Rightarrow$ es gibt immer (Auswahl-) Menge $\tilde{M}$, die mit jedem $M\in\mathcal{M}$ genau ein Element gemein hat.
	\end{itemize}
\end{definition}

\subsection{Aufbau einer mathematischen Theorie}

Axiome (als wahr angenommene Aussagen) $\to$ Beweise $\to$ Sätze ("'neue"' wahre Aussagen) \\
$\Rightarrow$ ergibt Ansammlung (Menge) wahrer Aussagen \\
$\newline$

Formulierung mathematischer Aussagen:
\begin{itemize}
	\item typische Form eines mathematischen Satzes: $\underbrace{\text{Wenn A gilt,}}_\text{Vorraussetzung}\underbrace{\text{dann folgt B}}_\text{Behauptung}$
	\item formal: $A\Rightarrow B$
\end{itemize}

\begin{example}
	\begin{itemize}
		\item $n\in\natur$ ist durch 4 teilbar $\Rightarrow$ $n$ ist durch 2 teilbar
		\item genauer meint man sogar $A\land C\Rightarrow B$, wobei $C$ aus allen bekannten 
		wahren Aussagen besteht
		\item $B$ ist \begriff[Bedingung!]{notwendig} für $A$
		\item $A$ ist \begriff[Bedingung!]{hinreichend} für $B$
	\end{itemize}
\end{example}

\begin{*anmerkung}
	Aus dem Wikipedia-Artikel zu notwendiger und hinreichender Bedingung:
	\begin{itemize}
		\item notwendige Bedingung: Wenn $B$ wahr ist, dann muss auch $A$ wahr sein. Es kann nicht sein, dass $B$ wahr ist, ohne dass $A$ wahr ist. 
		\item Beispiel: Für jede Primzahl $>2$ gilt: Sie ist ungerade. Also: ist die Eigenschaft "'Primzahl"' 
		notwendig für die Eigenschaft "'ist ungerade"', denn es gibt keine Primzahl, die gerade ist.
		\item hinreichende Bedingung: Eine hinreichende Bedingung sorgt für das Eintreten des Ereignisses. Wenn die Bedingung nicht notwendig, sondern nur hinreichend ist, dann gibt es andere hinreichende Bedingungen, die zum Eintreten des Ereignisses führen.
		\item Beispiel: Cola trinken ist nicht notwendig zum überleben, da man auch Wasser trinken kann.
	\end{itemize}
\end{*anmerkung}

\begin{definition}[direkter Beweis, indirekter Beweis]
	\begin{itemize}
		\item \begriff[Beweis!]{direkt}\highlight{er Beweis}: $(A\Rightarrow A_1)\land(A_1\Rightarrow A_2)\land\dotsc\land(A_n\Rightarrow B)$ wahr für $A\Rightarrow B$
		\item \begriff[Beweis!]{indirekt}\highlight{er Beweis} durch Tautologie $(A\Rightarrow B)\Leftrightarrow (\neg B\rightarrow \neg A)$
	\end{itemize}
\end{definition}

\subsection{Relation und Funktion}
\begin{definition}[Relation]
	\begin{itemize}
		\item \begriff{Relation} ist Teilmenge $R\subset M\times N$. $(x,y)\in R$ heißt: $x$ und $y$ stehen in Relation zueinander.
		\item Relation $R\subset M\times N$ heißt \begriff{Ordnungsrelation} (kurz \begriff{Ordnung}) auf $M$, falls $\forall a,b,c\in M$:
		\begin{enumerate}[label={\alph*)}]
			\item $(a,a)\in R$ (\begriff[Ordnung!]{reflexiv})
			\item $(a,b),(b,a)\in R \rightarrow a=b$ (\begriff[Ordnung!]{antisymmetrisch})
			\item $(a,b),(b,c)\in R \rightarrow (a,c)\in R$ (\begriff[Ordnung!]{transitiv})
		\end{enumerate}
		\item Ordnungsrelation $R$ auf $M$ heißt \begriff{Totalordnung}, falls $\forall a,b\in M: (a,b)\in R \lor (b,a)\in R$
		\item Relation auf $M$ heißt \begriff{Äquivalenzrelation}, falls $\forall a,b,c\in M$:
		\begin{enumerate}[label={\alph*)}]
			\item $(a,a)\in R$ (\begriff[Ordnung!]{reflexiv})
			\item $(a,b)\in R \Rightarrow (b,a)\in R$ (\begriff[Ordnung!]{symmetrisch})
			\item $(a,b),(b,c)\in R \Rightarrow (a,c)\in R$ (\begriff[Ordnung!]{transitiv})
		\end{enumerate}
		\item \mathsymbol{[a]}{$[a]$}$:=\{b\in M\mid (a,b)\in R\}$ heißt \begriff{Äquivalenzklasse} von $a\in M$ bzgl. $R$
		
		Jedes $b\in [a]$ ist ein \begriff{Repräsentant} von $[a]$
	\end{itemize}
\end{definition}

\begin{example}
	\proplbl{beispiel_brueche_aequivalenzklassen}
	$B=\left\lbrace \frac{m}{n}\mid m,n\in\whole,n\neq 0 \right\rbrace$ Menge der Brüche \\
	man hat Äquivalenzrelation auf $B$ mit $R=\left\lbrace\left( \frac{m}{n},\frac{p}{q}\right)\in B\times B \mid mq=np\right\rbrace$ \\
	beachte: Menge der Äquivalenzklassen $\left\lbrace\left[ \frac{m}{n}\right]\mid \frac{m}{n}\in B \right\rbrace$ ist die Menge der rationalen Zahlen
\end{example}

\begin{*anmerkung}
	\begin{itemize}
		\item Mit einer Ordungsrelation kann man eigentlich unordenbare Dinge wie Funktionen (gilt $x^2 < x^3$ oder $x^2 > x^3$?) ordnen.
		\item Eine Äquivalenzrelation ist eine Art Gleichheitszeichen, nur eben für mathematische Objekte, die 
		keine Zahen sind.
		\item zu \propref{beispiel_brueche_aequivalenzklassen}: Zwei Brüche $\frac{m}{n}$ und $\frac p q$ 
		sind gleich, wenn $mq=np$, d.h. diese zwei Brüche gehören zu einer Äquivalenzklasse. So 
		gehören die Brüche $\frac 2 3$ und $\frac 4 6$ zu einer Äquivalenzklasse, nämlich zu $\left[ \frac 
		2 3\right]$, da $2\cdot 6=12=3\cdot 4$. Alle Äquivalenzklassen, also alle nicht mehr kürzbaren Brüche 
		ergeben dann die rationalen Zahlen $\ratio$.
	\end{itemize}
\end{*anmerkung}

\begin{definition}[Abbildung]
	\begriff{Abbildung}/\begriff{Funktion} von $M$ nach $N$, kurz: $F:M\rightarrow N$ ist Vorschrift, die jedem \begriff{Argument} / \begriff{Urbild} $m\in M$ genau einen \begriff{Wert} / \begriff{Bild} $F(m)\in N$ zuordnet.
	
	\begin{itemize}
		\item \mathsymbol{D}{$\mathcal{D}$}$(F):=M$ heißt \begriff{Definitionsbereich} / \begriff{Urbildmenge}
		\item $N$ heißt \begriff{Zielbereich}
		\item $F(M'):=\{n\in N \mid n=F(m)$ für ein $m\in M'\}$ ist \begriff{Bild}\highlight{ von $M'$}$\subset M$
		\item $F^{-1}(N'):=\{ m\in M\mid n=F(m)$ für ein $N' \}$ ist \begriff{Urbild}\highlight{ von $N'$}$\subset N$
		\item \mathsymbol{R}{$\mathcal{R}$}$(F):= F(M)$ heißt \begriff{Wertebereich} / \begriff{Bildmenge}
		\item \mathsymbol{graph}{$\graph$}$(F) :=\{ (mn,)\in M\times N | n = F(m)\}$ heißt \begriff{Graph}\highlight{von $F$}
		\item \mathsymbol{fm}{$F|_{M'}$} ist \begriff{Einschränkung}\highlight{ der Funktion} von $F$ auf $M'\subset M$
		\item Zwei Funktionen $F$ und $G$ sind gleich, wenn
		\begin{itemize}
			\item $\mathcal{D}(F)=\mathcal{D}(G)$
			\item $F(m)=G(m)\quad\forall m\in\mathcal{D}(F)$
		\end{itemize}
		\item \begriff{Komposition} von $F:M\rightarrow N$ und $G:N\rightarrow P$ ist Abbildung $G$\mathsymbol{o}{$\circ$}$F:M\rightarrow P$ mit $(G\circ F)(m):=G(F(m))$
		\item Abbildung $F:M\rightarrow N$ heißt
		\begin{itemize}
			\item \begriff[Abbildung!]{injektiv}, falls eineindeutig (d.h. $F(m_1) = F(m_2) \Rightarrow m_1 = m_2$)
			\item \begriff[Abbildung!]{surjektiv}, falls $F(M) = N$, d.h. $\forall n\in N\,\exists m\in M: F(m) = n$
			\item \begriff[Abbildung!]{bijektiv}, falls injektiv und surjektiv
		\end{itemize}
		\item Für bijektive Abb. $F:M\rightarrow N$ ist \begriff{Umkehrabbildung} / \begriff{inverse Abbildung} \mathsymbol{f-1}{$F^{-1}$}$:N\rightarrow M$ definiert durch $F^{-1}(n) = m \Leftrightarrow F(m) = n$
	\end{itemize}
\end{definition}

\begin{example}
	betrachte $f:\real\to\real$ und $f(x)=\sin(x)$ \\
	Zielmenge: $\real$, aber Wertebereich $[-1,1]$!
\end{example}

\stepcounter{theorem}
\begin{proposition}
	Sei $F:M\rightarrow N$ surjektiv. Dann existiert Abbildung $G:N\rightarrow M$, sodass $F\circ G = \id_N$ (d.h. $F(G(n)) = n\,\forall n\in N$)
\end{proposition}

\begin{proof}
	Definiere Menge $\Gamma_n = \{m \in M \mid F(m) = n\} \overset{\text{surjektiv}}{\neq} \emptyset$. Nach Auswahlaxiom 	existiert Abbildung $G: N \to M$ mit $G(n) \in \Gamma_n;\ \forall n \in N \Rightarrow F(G(n)) = n;\ \forall n \in N 	\Rightarrow$ Behauptung.
\end{proof}



\begin{definition}[Verknüpfung]
	Eine \begriff{Rechenoperation} / \begriff{Verknüpfung} auf $M$ ist Abb. $*:M\times M\rightarrow M$, d.h. $m,n\in M$ wird \begriff{Ergebnis} $m*n\in M$
	
	Rechenoperation
	\begin{itemize}
		\item hat \begriff[Verknüpfung!]{neutrales Element} $e\in M$, falls $m*e = e*m = m\,\forall m\in M$
		\item ist \begriff[Verknüpfung!]{kommutativ}, falls $m*n = n*m$
		\item ist \begriff[Verknüpfung!]{assoziativ}, falls $k*(m*n) = (k*m)*n\,\forall k,m,n\in M$
		\item hat \begriff[Verknüpfung!]{inverses Element} $m'\in M$ zu $m\in M$, falls $m*m' = m'*m = e$
	\end{itemize}
\end{definition}

\begin{example}
	\begin{itemize}
		\item \begriff{Addition}: $(m,n)\mapsto: m+n$ \begriff{Summe},
		\begin{itemize}
			\item neutrales Element heißt \begriff{Null} / \begriff{Nullelement}
			\item Inverses Element: \mathsymbol{-}{$-m$}
		\end{itemize}
		\item \begriff{Multiplikation} $\cdot:(m,n)\mapsto: m\cdot n$ \begriff{Produkt}
		\begin{itemize}
			\item neutrales Element heißt \begriff{Eins} / \begriff{Einselement}
			\item Inverses Element:\mathsymbol{-1}{$m^{-1}$}
		\end{itemize}
	\end{itemize}
\end{example}

\begin{definition}[distributiv]
	Addition und Multiplikation heißen \begriff{distributiv}, falls $k\cdot(m+n) = k\cdot m + k\cdot n\,\forall k,m,n\in M$
\end{definition}

\begin{definition}[Körper]
	Menge $K$ heißt \begriff{Körper}, falls auf $K$ eine Addition und Multiplikation existiert mit
	\begin{itemize}
		\item es existieren neutrale Elemente $0\in K$ und $1\in K_{\neg 0}$
		\item Addition und Multiplikation sind distributiv
		\item Es gibt Inverse
	\end{itemize}
\end{definition}

\begin{definition}
	Menge $M$ habe Ordnung "`$\le$"', sowie Addition und Multiplikation.
	Ordnung ist \begriff[Ordnung!]{verträglich}\highlight{ mit Addition und Multiplikation}, wenn $\forall a,b,c\in M$
	\begin{itemize}
		\item $a\le b \Leftrightarrow a+c \le b+c$
		\item $a\le b \Leftrightarrow a\cdot c \le b\cdot c$ mit $c > 0$
	\end{itemize}
\end{definition}

\begin{definition}[angeordnet]
	Körper $K$ heißt \begriff[Körper!]{angeordnet}, falls mit Addition und Multiplikation verträgliche Totalordnung existiert.
\end{definition}

\begin{definition}[Isomorphismus]
	\begriff{Isomorphismus} bezüglich einer Struktur ist bijektive Abbildung $I:M_1\rightarrow M_2$, die auf $M_1$ und $M_2$ vorhandene Struktur erhält, z.B. 
	\begin{itemize}
		\item Ordnung: $a\le b\iff I(a)\le I(b)$
		\item Rechenoperationen: $I(a*b)=I(a)*I(b)$
	\end{itemize}
	Mengen $M_1$ und $M_2$ heißen \begriff[Menge!]{isomorph}.
\end{definition}

\begin{*anmerkung}
	Mit einem Isomorphismus kann man die Elemente einer Menge, z.B. ganze Zahlen, den Elementen einer anderen Menge, z.B. den natürlichen Zahlen, zuordnen. Konkret würde das dann so aussehen: \\
	$0\mapsto 0$, $1\mapsto 1$, $-1\mapsto 2$, $2\mapsto 3$, $-2\mapsto 4$, ... \\
	Insbesondere wenn es darum geht, ob die ganzen Zahlen abzählbar sind, also ob ich diese mit den natürlichen Zahlen neu durchnummerieren kann, ist ein solcher Isomorphismus (denn dieses "'neunummerieren"' ist ein Isomorphismus) notwendig. Alle Aussagen, die die Struktur betreffen, z.B. die Kommutativität, bleiben erhalten und müssen nicht neu bewiesen werden.
\end{*anmerkung}

\begin{example}
	$M_1=\natur$, $M_2=\{\text{gerade Zahlen}\}$ jeweils mit Addition, Multiplikation, Ordnung \\
	$\Rightarrow I: M_1\to M_2$ mit $I(n)=2n$ ist ein Isomorphismus, denn alle geraden Zahlen werden 
	einfach nur neu durchgezählt \\
	$\Rightarrow$ Isomorphismus erhält Addtion, Ordnung und die 0, aber nicht die Multiplikation, da
	$I(a)*I(b)=2a*2b=4ab$ aber $I(a*b)=2(a*b)=2ab$, also $I(a)*I(b)\neq I(a*b)$
\end{example}

\subsection{Bemerkungen zum Fundament der Mathematik}

Forderungen an eine mathematische Theorie
\begin{itemize}
	\item widerspruchsfrei: Satz und seine Negation sind nicht gleichzeitig herleitbar
	\item vollständig: alle Aussagen innerhalb einer Theorie sind als wahr oder falsch beweisbar
\end{itemize}

2 Unvollständigkeitssätze
\begin{itemize}
	\item jedes System ist nicht gleichzeitig widerspruchsfrei und vollständig
	\item in einem System kann man nicht die eigenen Widerspruchsfreiheit zeigen
\end{itemize}