\setcounter{dummy}{16}
\addtocounter{section}{15}
\addtocounter{chapter}{4}

\chapter{Differentiation}
	Differentiation ist lokale Linearisierung.

\section{Wiederholung und Motivation}
	\begin{ueberblick}
	$K^n$ ist ein $n$-dimensionaler VR über dem vollständigen Körper $K=\real$ 
	oder $K=\comp$. Die Elemente sind $x=(x_1,...,x_n)\in K^n$ mit 
	$x_1,...,x_n\in K$. Basis ist die Standardbasis $(e_1,...,e_n)$. \\
	
	
	Alle Normen sind auf $K^n$ äquivalent $\Rightarrow$ Konvergenz ist 
	unabhängig von der Norm. Trotzdem verwenden wir die euklidische Norm: 
	$|x|=\sqrt{\sum\limits_{j=1}^n |x_j|^2}$. \\
	
	
	\underline{Skalarprodukt:} $\langle x,y \rangle=\sum\limits_{j=1}^n x_j\cdot y_j$ in 
	$\real$ bzw. $\langle x,y \rangle=\sum\limits_{j=1}^n \overline{x_j}
	\cdot y_j$ in $\comp$. \\
	
	
	\underline{\person{Cauchy}-\person{Schwarz}-Ungleichung:} $|\langle x,y \rangle| \le
	|x|\cdot |y|$. \\

	
	lineare Abbildung: $A:K^n \to K^m$, Darstellung mittels $m\times n$-Matrix 
	bezüglich Standardbasen in $K^n$ und $K^m$. Beachte: $A$ steht für die 
	lineare Abbildung und die Matrix, die die lineare Abbildung beschreibt. 
	Lineare Abbildungen sind stets stetig (unabhängig von der Norm). Hinweis: 
	$x=(x_1,...,x_n)$ in der Regel als Zeilenvektor geschrieben, aber bei 
	Matrixmultiplikation ist $x$ Spaltenvektor und $x^t$ Zeilenvektor, d.h. \\
	$x^t\cdot y=\langle x,y \rangle$, falls $m=n$ \\
	$x\cdot y^t=x\otimes y$, sogenanntes Tensor-Produkt \\
	
	
	$L(K^n,K^m)=\{A: K^n \to K^m \mid A \text{ linear}\}$ Menge aller 
	linearen Abbildungen mit $||A||=\sup\{|Ax| \mid |x|\le 1\} \to$ Norm 
	hängt im Allgemeinen von Normen auf $K^n$ und $K^m$ ab. \\
	$L(K^n,K^m)$ ist isomorph zu $Mat_{m\times n}(K)$ ist isomorph zu $K^{mn}$ 
	jeweils als VR $\Rightarrow$ $L(K^n,K^m)$ ist ein $m\cot n$-dimensionaler 
	VR $\Rightarrow$ alle Normen sind äquivalent $\Rightarrow$ Konvergenz von 
	$\{A_n\}$ in $L(K^n,K^m)$ unabhängig von Norm, nehme in der Regel statt 
	$||A||$ die euklidische Norm $|A|=\sqrt{\sum\limits_{k=1}^n \sum\limits_
	{l=1}^n |a_{kl}|^2} \Rightarrow$ es gilt: $|Ax|\le ||A||\cdot |x|$ und 
	$|Ax|\le |A|\cdot |x|$. \\
	
	
	Abbildung $f:K^n\to K^m$ heißt affin linear falls $f(x)=Ax+a$ für eine 
	lineare Abbildung $A:K^n\to K^m$. \\
	
	\underline{\person{Landau}-Symbole}: Sei $f:D\subset K^n \to K^m$, $g:D\subset K^n\to K$, 
	$x_0\in \overline{D}$
	\begin{compactitem}
		\item $f(x)=o(g(x))$ für $x\to x_0$ genau dann, wenn $\lim\limits_{\substack{x\to x_0 \\ x\neq x_0}} \frac{|f(x)|}{g(x)}=0$
		\item $f(x)=O(g(x))$ für $x\to x_0$ genau dann, wenn $\exists\delta>0$, $0\le c<\infty$ mit 
		$\frac{|f(x)|}{|g(x)|}\le c \quad \forall x\in (B_{\delta}(x_0)\backslash \{x_0\})\cap D$
	\end{compactitem}
	\end{ueberblick}

	%TODO was soll der wichtige Spezialfall im Skript? Ist der wichtig?

	\begin{beispiel}[$f: D\subset K^n \to K^m$]
		$x_0\in D$, $x_0$ Häufungspunkt von $D$. Dann: $f$ stetig in $x_0 \iff \lim\limits_{x\to x_0} f(x)
		=f(x_0)\iff \lim\limits_{x\to x_0} \frac{|f(x)-f(x_0)|}{1}=0 \iff f(x)=f(x_0)+o(1)$ für $x\to x_0$ \\
		Interpretation: Setze $r(x)=f(x)-f(x_0)\Rightarrow r(x)=o(1)$ für $x\to x_0\Rightarrow r(x)\to 0$, 
		d.h. $o(1)$ ersetzt die Restfunktion $f(x)-f(x_0)$. Wegen $o(1)=o(|x-x_0|^0)$ sagt man auch, 
		dass $f(x)=f(x_0)+o(1)$ die Approximation 0-ter Ordnung der Funktion $f$ in der Nähe von $x_0$.
	\end{beispiel}

	\begin{beispiel}[$f:D\subset \real^n \to \real$]
		$x_0\in D$, $D$ offen, das bedeutet: $f(x)=f(x_0)+o(|x-x_0|)$, $x\to x_0 \quad (*)$
		\begin{compactitem}
			\item betrachte $f$ auf Strahl $x=x_0+ty$, $y\in \real^n$ fest, $|y|=1$, $t\in \real$ \\
			$(*)\Rightarrow 0=\lim\limits_{\substack{x\to x_0 \\ x\neq x_0}} \frac{|f(x)-f(x_0)|}{|x-x_0|}=
			\lim\limits_{\substack{t\to 0 \\ t\neq 0}} \frac{|f(x_0+ty)-f(x_0)|}{|t|} \Rightarrow \frac
			{|\Delta f|}{|t|}=|\text{Anstieg der Sekante}|\to 0$
			\item $(*)\Rightarrow f(x)=f(x_0)+\underbrace{\frac{o(|x-x_0|)}{|x-x_0|}}_{o(1)}|x-x_0|
			\Rightarrow f(x)=f(x_0)+o(1)|x-x_0| \Rightarrow f(x)=f(x_0)+r(x)|x-x_0|\Rightarrow 
			|f(x)-f(x_0)| \le \varrho(t)|x-x_0|$ falls $|x-x_0|\le t$ mit $\varrho(t)=\sup\limits_{|x-x_0|\le t} 
			|r(x)|\to 0$ \\
			Graph von $f$ liegt nahe $x_0$ in "'immer flacheren kegelförmigen Mengen"' $\Rightarrow$ 
			Graph "'schmiegt sich"' an horizontale Ebene durch Punkt $(x_0,f(x_0))$
			\item $(*)$ erfüllt offenbar nicht die Beobachtung: horizontale Ebene ist Graph einer affin 
			linearen Funktion $\tilde A: \real^n \to \real$
		\end{compactitem}
	\end{beispiel}

	\textbf{zentrale Frage:} Gibt es zur Funktion $f:D\subset K^n \to K^m$, $x_0\in D$, eine affin 
	lineare Funktion $\tilde A:K^n\to K^m$, so dass sich in der Nähe von $x_0$ der Graph von $f$ 
	an den Graph von $\tilde A$ "'anschmiegt"'? \\
	\textbf{Antwort:} Ja, wegen $f(x_0)=\tilde A|x_0|$ folgt $\tilde Ax=A(x-x_0)+f(x_0)$
	
	\begin{definition}[Anschmiegen]
		$f(x_0)-(f(x_0)+A(x-x_0))=o(|x-x_0|)$ \\
		d.h. die Abweichung wird schneller kleiner als $|x-x_0|$! \\
		
		Vielleicht hatten Sie bisher eine andere Vorstellung von "'anschmiegen"', aber wir machen hier 
		Mathematik!
	\end{definition}

\section{Ableitung}
	Sei $f:D\subset K^n \to K^m$, $D$ offen.
	
	\begin{definition}[differenzierbar im Punkt]
		$f$ heißt differenzierbar im Punkt $x_0\in D$ falls es eine lineare Abbildung $A\in L(K^n,K^m)$ gibt,
		mit der Eigenschaft $f(x)=f(x_0)+A(x-x_0)+o(|x-x_0|)$ , $x\to x_0$.
	\end{definition}

	\begin{definition}[Ableitung]
		$A$ heißt Ableitung von $f$ an der Stelle $x=x_0$ und wird mit $f'(x_0)=A$ bzw. $Df(x_0)$ 
		bezeichnet. Man kann auch totales Differential, Fréchet-Ableitung, Jacobimatrix oder 
		Funktionalmatrix sagen. \\
		Andere Bezeichnungen sind: $\frac{\partial f}{\partial x}(x_0)$, $\frac{\partial f(x)}{\partial x}\vert_
		{x=x_0}$, d$f(x_0)$,... \\
		Somit gilt: $f(x)=f(x_0)+f'(x_0)(x-x_0)+o(|x-x_0|)$, $x\to x_0$
	\end{definition}

	\begin{bemerkung}
		$f'(x_0)$ ist im Allgemeinen eine von $x_0$ abhängige Matrix! \\
		$\Rightarrow$ lineare Funktion $\tilde A(x)=f(x_0)+f'(x_0)(x-x_0)$ appromximiert Funktion $f$ 
		in der Nähe von $x_0$ und heißt Linearisierung von $f$ in $x_0$.
	\end{bemerkung}

	\begin{satz}
		Sei $f:D\subset K^n \to K^m$, $D$ offen. $f$ ist differenzierbar in $x_0\in D$ mit Abbildung $f'(x_0)\in L(K^n,K^m)$ genau dann, wenn die folgenden Bedingungen erfüllt sind:
		\begin{compactitem}
			\item für ein $r:D\to K^m$ mit $\lim\limits_{\substack{x\to x_0 \\ x\neq x_0}} 
			\frac{r(x)}{|x-x_0|}=0$ \\
			$f(x)=f(x_0)+f'(x_0)(x-x_0)+r(x)\quad\forall x\in D$
			\item für ein $R:D\to L(K^n,K^m)$ mit $\lim\limits_{x\to x_0} R(x)=0$ \\
			$f(x)=f(x_0)+f'(x_0)(x-x_0)+R(x)(x-x_0)\quad\forall x\in D$
			\item für ein $Q:D\to L(K^n,K^m)$ mit $\lim\limits_{x\to x_0} Q(x)=f'(x_0)$ \\
			$f(x)=f(x_0)+Q(x)(x-x_0)\quad\forall x\in D$
		\end{compactitem}
	\end{satz}
	\begin{beweis}
		\begin{compactitem}
			\item offenbar ist $r(x)=o(|x-x_0|)$, $x\to x_0$, folglich ist dies äquivalent zu: $f$ 
			differenzierbar in $x_0$ mit Abbildung $f'(x_0)$
			\item $1\Rightarrow 2$: Sei $R:D\to K^{m\times n}$ gegeben durch $R(x_0)=0$, $R(x)=
			\frac{r(x)}{|x-x_0|^2}\cdot (x-x_0)^t$, $x\neq x_0 \Rightarrow R(x)(x-x_0)=\frac{r(x)}
			{|x-x_0|^2}\langle x-x_0,x-x_0\rangle=r(x)$ \\
			wegen $0=r(x_0)=R(x_0)(x-x_0)$ folgt $r(x)=R(x)(x-x_0)\quad\forall x\in D\Rightarrow 2$ \\
			wegen $|r(x)(x-x_0)^t|=|r(x)||x-x_0|$ folgt $\lim\limits_{x\to x_0} |R(x)|=\lim\limits_
			{\substack{x\to x_0 \\ x\neq x_0}} \frac{|r(x)\cdot (x-x_0)^t|}{|x-x_0|^2}=\lim\limits_
			{\substack{x\to x_0 \\ x\neq x_0}} \frac{|r(x)||x-x_0|}{|x-x_0|^2}=0\Rightarrow 2$
			\item $2\Rightarrow 3$: setze $Q(x)=f'(x_0)+R(x)\quad\forall x\in D\Rightarrow 3$ \\
			wegen $\lim\limits_{x\to x_0} Q(x)=f'(x_0)$ folgt 3
			\item $3\Rightarrow 1$: setze $r(x)=(Q(x)-f'(x))(x-x_0)$ wegen $|r(x)|\le |Q(x)-f'(x)||x-x_0|$ 
			folgt $\lim\limits_{\substack{x\to x_0 \\ x\neq x_0}} \frac{|r(x)|}{|x-x_0|}=\lim\limits_{x\to x_0} 
			|Q(x)-f'(x_0)|=0\Rightarrow$ Definition Ableitung
		\end{compactitem}
	\end{beweis}

	\begin{satz}
		Sei $f:D\subset K^n \to K^m$, $D$ offen, $f$ differenzierbar in $x_0\in D$. Dann:
		\begin{compactitem}
			\item $f$ ist stetig in $x_0$.
			\item Ableitung $f'(x_0)$ ist eindeutig bestimmt.
		\end{compactitem}
	\end{satz}
	\begin{beweis}
		\begin{compactitem}
			\item $\lim\limits_{x\to x_0} f(x)=\lim\limits_{x\to x_0} (f(x_0)+f'(x_0)(x-x_0)+R(x)(x-x_0))=f(x_0)
			\Rightarrow$ Behauptung
			\item angenommen $A_1,A_2\in L(K^n,K^m)$ sind Ableitungen von $f$ in $x_0$. Seien $R_1,R_2$ 
			zugehörige Terme. Dann gilt für $x=x_0+ty$: $|(A_1-A_2)(ty)|=|R_1(x_0+ty)(ty)|+|R_2(x_0+ty)
			(ty)| \le |R_1(x+ty)||ty|+|R_2(x_0+ty)||ty|\Rightarrow 0\le |(A_1-A_2)(y)|\le (|R_1(x_0+ty)|+|R_2
			(x_0+ty)|)|y|\to 0\Rightarrow (A_1-A_2)(y)=0\Rightarrow A_1=A_2\Rightarrow$ Behauptung
		\end{compactitem}
	\end{beweis}

\subsection{Spezialfälle für $K=\real$:}
	\begin{compactitem}
		\item $m=1$, $f:D\subset \real^n\to \real$ \\
		$f'(x_0)\in \real^{1\times n}$ ist Zeilenvektor, $f'(x_0)$ betrachtet als Vektor in $\real^n$ heißt auch 
		Gradient. Offenbar $f'(x_0)y=\langle f'(x_0),y\rangle \quad\forall y\in \real^n\Rightarrow f(x)=
		f(x_0)+\langle f'(x_0),x-x_0 \rangle + o(|x-x_0|)$
		\item $n=1$, $f:D\subset \real\to \real^m$, z.B. $D=(a,b)$ \\
		$f$ bzw. Bild $D(f)$ ist Kurve in $\real^m$, $f'(x_0)$ ist Spaltenvektor im $\real^m$. Man kann 
		schreiben: $f(x_0+t)=f(x_0)+tf'(x_0)+o(|t|)$ \\
		$\iff \underbrace{\frac{f(x_0+t)-f(x_0)}{t}}_{\text{heißt Differenzenquotient von
		 }f\text{ in }x_0}=f'(x_0)+o(1)$, $t\to 0\quad \frac{o(t)}{t}=o(1)$ \\
	 	$\iff \underbrace{\lim\limits_{t\to 0}\frac{f(x_0+t)-f(x_0)}{t}}_{\text{heißt Differentialquotient von
	 		}f\text{ in }x_0}=f'(x_0)$ \\
 		\textbf{Bemerkungen:} $f$ differentierbar $\iff$ Diffentialquotient existiert in $x_0$, aber nicht 
 		erklärt für den Fall $n>1$! \\
 		\textbf{Interpretation für $m>1$:} \begin{compactitem}
 			\item Tangente an Kurve: Bild von $\tilde A(\real)$ ist Gerade und heißt Tangente an Kurve 
 			$f(x_0)$
 			\item Tangentenvektor an Kurve in $f(x_0)$ ist $f'(x)$ \\
 			Falls $f$ nicht differenzierbar in $x_0$ bzw. $x_0$ Randpunkt von $D$ und $f(x_0)$ 
 			definiert, betrachtet man einseitige Grenzwerte.
 			\item rechtsseitige Ableitung: $\lim\limits_{t\downarrow 0}\frac{f(x_0+t)-f(x_0)}{t}=f'_r(x_0)$ 
 			heißt rechtsseitige Ableitung von $f$ in $x_0$ (falls existent), analog linksseitige Ableitung
 		\end{compactitem}
 		\item $n=m=1$, $f:D\subset \real\to \real$ \\
 		$f'(x_0)\in \real$ ist Zahl und es gilt: \begin{compactitem}
 			\item Graph von $f$ ist Kurve in $\real^2$
 			\item Graph von $\tilde{A}$ ist Tangente an Graph von $f$ in $(x_0,f(x_0))$ und hat Anstieg 
 			$f'(x_0)$
 		\end{compactitem}
	\end{compactitem}

	\begin{folgerung}
		Sei $f:D\subset K\to K^m$, $D$ offen. Dann: \\
		$f$ ist differenzierbar in $x_0\in D$ mit Ableitung $f'(x_0)\in L(K,K^m)\iff \exists f'(x_0)\in 
		L(K,K^m):\lim\limits_{y\to 0} \frac{f(x_0+t)-f(x_0)}{y}=f'(x_0)$.
	\end{folgerung}

\subsection{einfache Beispiele für Ableitungen}
	\begin{beispiel}[$f:K^n\to K^m$ affin linear]
		Für beliebige $x_0\in K^n$ gilt: $f(x)=Ax_0+a+A(x-x_0)=f(x_0)+A(x-x_0)+0\Rightarrow f$ ist 
		differenzierbar in $x_0$ mit $f'(x_0)=A$
	\end{beispiel}

	\begin{beispiel}[$f:\real^n\to \real$ mit $f(x)=|x|^2$]
		$|x-x_0|^2=\langle x-x_0,x-x_0\rangle=|x|^2-2\langle x_0,x\rangle+2\langle x_0,x_0\rangle-|x_0|^2=
		|x|^2-2\langle x_0,x-x_0\rangle-|x_0|^2\Rightarrow f(x)=f(x_0)+2\langle x_0,x-x_0\rangle+
		\underbrace{|x-x_0|^2}_{o(|x-x_0|)}$ \\
		wegen $2x_0\in L(\real^n,\real)$ folgt $f=|\cdot |^2$ ist differenzierbar in $x_0$ mit $f'(x_0)=2x_0
		\quad\forall x_0\in \real^n$
	\end{beispiel}

	\begin{beispiel}[$f:K\to K$ mit $f(x)=x^k$]
		\begin{compactitem}
			\item $k=0$: $f(x)=1\Rightarrow f'(x)=0$
			\item $k=1$: $f(x_0+y)=(x_0+y)^k=\sum\limits_{j=0}^{k} \binom{k}{j} x_0^{k-j}y^j=x_0^k+
			kx_0^{k-1}y+o(y)=f(x_0)+k\cdot f(x_0)y+o(y)$, $y\to 0\Rightarrow f'(x_0)=kx_0^{k-1} 
			\quad\forall x_0\in K$
		\end{compactitem}
	\end{beispiel}

	\begin{beispiel}[$f:\real^n\to\real$ mit $f(x)=|x|$]
		$f$ ist nicht differenzierbar in $x_0=0$, denn, angenommen Ableitung $f'(0)\in \real^n$ existiert, 
		fixiere $x\in \real^n$ mit $|x|=1\Rightarrow |tx|=0+\langle f'(0),tx\rangle+o(t)$, $t\to 0\Rightarrow
		\frac{|t|}{t}=\langle f'(x),x \rangle + \frac{o(t)}{t}=\pm 1 \Rightarrow$ Widerspruch \\
		anschaulich: es gibt keine Tangentialebene an Graph von $f$ in $(0,|0|)\in \real^{n+1}$ \\
		folglich: $f$ ist stetig in $x_0\not\Rightarrow f$ ist differenzierbar in $x_0$.
	\end{beispiel}
