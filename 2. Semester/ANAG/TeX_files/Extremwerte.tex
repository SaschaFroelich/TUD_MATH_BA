\section{Extremwerte} \setcounter{equation}{0}
\subsection{Lokale Extrema ohne Nebenbedingung}
Betrachte $f:D\subset\mathbb{R}^n\to\mathbb{R}$\marginnote{Zielraum $\mathbb{R}$ wegen Ordnung}, $D$ offen, $f$ \gls{diffbar}.

\begin{underlinedenvironment}[notwendige Bedingung]
	(\propref{mittelwertsatz_optimalitaetsbedingung}): $f$ hat lokales Minimum / Maximum in $x\in D$ $\Rightarrow$ $f'(x) = 0$
\end{underlinedenvironment}

\begin{underlinedenvironment}[Frage]
	Hinreichende Bedingung?
\end{underlinedenvironment}

\begin{*definition}
	$f^{(k)}(x)$ für $k\ge $ heißt positiv \begriff{definit} (negativ definit), falls \begin{align}
		\proplbl{extremwerte_definit_definition_eq}
		f^{(k)}(x) y^k > 0 \;(< 0) \quad\forall y\in\mathbb{R}\setminus \{0\}
	\end{align}
	und positiv (negativ) \begriff{semidefinit} mit $\ge$ ($\le$).
	
	$f^{(k)}$ heißt \begriff{indefinit}, falls \begin{align}
		\proplbl{extremwerte_indefinit_definition_eq}
		\exists y_1, y_2\in\mathbb{R}^n\setminus \{0\}: f^{(k)}(x) y_1^k < 0 < f^{(k)} (x) y_2^k
	\end{align}
\end{*definition}

\begin{underlinedenvironment}[Hinweis]
	$k$ ungerade, $f^{(k)}(x)\neq 0$ $\Rightarrow$ $f^{(k)}(x)$ indefinit, denn $f^{(k)}(-y)^k = (-1)^k f^{(k)}(x) y^k$
\end{underlinedenvironment}

\begin{proposition}[Hinreichende Extremwertbedingung]
	\proplbl{extremwerte_hinreichende_bedingung}
	Sei $f:D\subset\mathbb{R}^n\to\mathbb{R}$, $D$ offen, $f\in C^k(D,\mathbb{R})$, $x\in D$, $k\ge 2$ und sei \begin{align}
		\proplbl{extremwerte_hinreiche_bedingung_eq}
		f'(x) = \dotsc = f^{(k-1)} = 0
	\end{align}
	Dann: \begin{enumerate}[label={\alph*)}]
		\item $f$ hat strenges lokales Minimum (Maximum), falls $f^{(k)}(x)$ positiv (negativ) definit
		\item \proplbl{extremwerte_hinreichende_bedinung_b}
		$f$ hat weder Minimum noch Maximum, falls $f^{(k)}(x)$ indefinit.
	\end{enumerate}
\end{proposition}

\begin{remark}\vspace*{0pt}
	\begin{enumerate}[label={\arabic*)}]
		\item Falls $f^{(k)}(x)$ positiv (negativ) semidefinit $\Rightarrow$ keine Aussage möglich.
		
		(betrachte $f:\mathbb{R}^2\to\mathbb{R}$ mit $f(x_1, x_2) = x_1^2 + x_2^4$, hat Minimum in $x=0$, aber $f(x_1, x_2) = x_1^2 + x_2^3$ hat weder Minimum noch Maximum in $x=0$)
		
		\item \ref{extremwerte_hinreichende_bedinung_b} liefert: $f^{(k)}(x) \neq 0$ positiv (negativ) semidefinit ist notwendige Bedingung für ein lokales Minimum bzw. Maximum, falls \eqref{extremwerte_hinreiche_bedingung_eq} gilt
	\end{enumerate}
\end{remark}

\begin{proof}\hspace*{0pt}
	\begin{enumerate}[label={zu \alph*)},topsep=\dimexpr-\baselineskip / 2\relax,leftmargin=\widthof{\texttt{zu a)\ }}]
		\item \proplbl{extremwerte_hinreichende_bedingung_beweis_a}
		Für Minimum (Maximum analog):
		
		Sei $f^{(k)}(x)$ positiv definite Abbildung, $y\to f^{(k)}(x) y^k$ stetige Abbildung (folgt aus \propref{taylor_partielle_ableitung_isomorphismus_bemerkung}).
		
		Sei $S=\{ y\in\mathbb{R}^n \mid \vert y \vert = 1 \}$ ist kompakt \\
		{\renewcommand{\arraystretch}{1.5}
		\begin{tabularx}{\linewidth}{r@{\ \ }r@{\ }c@{\ }X}
		$\xRightarrow{\propref{chap_15_3}}$ & \multicolumn{3}{l}{$\exists\tilde{y}\in S: f^{(k)}(x) y^k \ge f^{(k)}(x)\tilde{y}^k =: \gamma > 0\;\forall y\in S$} \\
		$\xRightarrow{\propref{taylor_taylor}}$ & $f(x+y)$& = &$f(x) + \frac{1}{k!}f^{(k)}(x) y^k + o(\vert y\vert^k)$\\
		& & = &$f(x) + \frac{1}{k!}\vert y \vert^k \left(\underbrace{f^{(k)}(x) \left(\frac{y}{\vert y \vert}\right)^k}_{\ge \gamma} + \underbrace{o(1)}_{\mathclap{\ge -\frac{\gamma}{2}}}\right),\;\vert y \vert\to 0$ \\
		&& $\ge$ & $f(x) + \frac{\gamma}{2k!}\cdot \vert y \vert^k$ $\forall y\in B_r(0)$ falls $y\in B_r(0)$, $r>0$ klein \\
		$\Rightarrow$ &\multicolumn{3}{l}{$x$ ist strenges, lokales Minimum $\Rightarrow$ Behauptung}
		\end{tabularx}}
		
		\item Wähle $y_1, y_2$ gemäß \eqref{extremwerte_indefinit_definition_eq}, \gls{obda} $\vert y_1 \vert = \vert y_2 \vert = 1$ \\
		\begin{tabularx}{\linewidth}{r@{\ \ }X}
		$\xRightarrow[\text{$\vert t \vert$  klein}]{\text{analog \ref{extremwerte_hinreichende_bedingung_beweis_a}}}$ & $f(x + ty_1) = f(x) + \frac{t^k}{k!}\left(f^{(k)}(x) y_1^k + o(1) \right) < f(x)$, \\
		&$f(x + ty_2) = f(x) + \frac{t^k}{k!} \left( f^{(k)}(x) y_2^k + o(1) \right) > f(x)$
		\end{tabularx}
		$\Rightarrow$ Behauptung
	\end{enumerate}
\end{proof}

\begin{underlinedenvironment}[Test Deefinitheit in Anwendungen]
	$k=2$ wichtig (vgl. lineare Algebra).
	
	$f''(x) \in L^2(\mathbb{R}^2,\mathbb{R})\cong \mathbb{R}^{n\times n}$ (\person{Hesse}-Matrix)
	
	$f''(x) y^2 = f''(x) (y,y) = \langle (\mathrm{Hess} f)(x)y, y\rangle$, vgl. \propref{taylor_partielle_ableitung_beispiel_10}
	
	Matrix $A\in\mathbb{R}^{n\times}$ ist \begin{itemize}
		\item positiv (negativ) definit $\Leftrightarrow$ alle Eigenwerte sind positiv (negativ) 
		\item indefinit $\Rightarrow$ $\exists$ positive und negative Eigenwerte
	\end{itemize}
\end{underlinedenvironment}

\subsection{Sylvester'sches Definitheitskriterium}
Eine symmetrische Matrix $A=(a_{ij})\in\mathbb{R}^{n\times n}$ ist positiv definit \gls{gdw} alle führenden Hauptminoren positiv sind, d.h. \begin{align*}
	\alpha_k := \det\begin{pmatrix}
		\alpha_{11} & \dots & \alpha_{1k} \\ \vdots && \vdots \\ \alpha_{k1} & \dots & \alpha_{kk}
	\end{pmatrix} > 0\quad\forall k\in\{1,\dotsc,n\}
\end{align*}

\begin{underlinedenvironment}[beachte]
	$A$ negativ definit $\Leftrightarrow$ $-A$ positiv definit
\end{underlinedenvironment}
\begin{underlinedenvironment}[Spezialfall $n=2$]\vspace*{0pt}
	\begin{itemize}[topsep=\dimexpr-\baselineskip/2\relax]
		\item $\det A <0 $ $\Leftrightarrow$ indefinit
		\item $\alpha_1 < 0$ und $\det A > 0$ $\Leftrightarrow$ negativ definit
	\end{itemize}
\end{underlinedenvironment}

\begin{example}
	\proplbl{extremwerte_sylvester_beispiel_3}
	Sei $f:\mathbb{R}^2\to\mathbb{R}$ mit $f(x_1, x_2) = x_1^2 + \cos x_2$ \\
	\begin{tabularx}{\linewidth}{r@{\ \ }X}
	$\Rightarrow$ & $f'(x_1, x_2) = (2x_1) - \sin x_2 = 0$ \\
	$\Rightarrow$  &$x_1 = 0$, $x_2 = k\cdot\pi$, d.h. $\tilde{x} = (0, k\cdot\pi)$ für $k\in\mathbb{Z}$ sind Kandidaten für Extrema.
	\end{tabularx} \begin{align*}
		f''(x_1, x_2) &= \begin{pmatrix}
			2 & 0 \\ 0 & -\cos x_2
		\end{pmatrix} & \Rightarrow\;\; f(\tilde{x}) &= \begin{pmatrix}
			2 & 0 \\ 0 & (-1)^{k+1}
		\end{pmatrix}
	\end{align*}
	entsprechend ergeben sich folgende Fälle:\\
	\begin{tabularx}{\linewidth}{r@{\ \ }Xr@{\ \ }X}
	$\Rightarrow$ & $f''(\tilde{x})$ ist positiv definit für $k$ ungerade & $\Rightarrow$ & $f''(\tilde{x})$ ist indefinit für $k$ gerade \\
	$\Rightarrow$ & lokales Minimum,&
	$\Rightarrow$ & kein Extremum
	\end{tabularx}
\end{example}

\subsection{Lokale Extrema mit Gleichungsnebenbedingung}
Betrachte $f:D\subset\mathbb{R}^n\to\mathbb{R}$ \gls{diffbar}, $D$ offen, $g:D\subset\mathbb{R}^n\to\mathbb{R}$ \gls{diffbar}

\begin{underlinedenvironment}[Frage]
	Bestimmen von Extrema von $f$ auf der Menge $G:= \{ x\in\mathbb{R}\mid g(x) = 0 \}$, d.h. suche notwendige Bedingung (für hinreichende Bedingung sieh Vorlesung Optimierung)
\end{underlinedenvironment}

\begin{underlinedenvironment}[Motivation]
	Für $m\ge 1$: notwendige Bedingung: $f'(\mathrm{max})$ steht senkrecht auf der Niveaumenge $G$ \marginnote{(vgl. \propref{richtungsableitung_gradient_eigenschaften})} \\
	$\Rightarrow$ $\exists\lambda\in\mathbb{R}: f'(x_{\max}) + \lambda g'(x_{\max}) = 0$
\end{underlinedenvironment}

\begin{proposition}[Lagrange-Multiplikatorregel, notwendige Bedingung]
	Seien $f:D\subset\mathbb{R}^n\to\mathbb{R}$, $g:D\to\mathbb{R}^m$ stetig, \gls{diffbar}, $D$ offen und sei $x\in D$ lokales Extremum von $f$ bezüglich $G$, d.h. \begin{align*}\exists r > 0: f(x)\; \substack{\le \\ \ge}\; f(y)\quad\forall y\in B_r(x)\end{align*} mit $g(y) = 0$.
	
	Falls $g'(x)$ regulär, d.h. \begin{align}
		\proplbl{extremwerte_lokale_extrema_mit_gleichungsnebenbedingung_lagrange_multiplikator_eq_4}
	\mathrm{rang}\; g'(x) = m, \end{align}dann
	\begin{align}
	\proplbl{extremwerte_lokale_extrema_mit_gleichungsnebenbedingung_lagrange_multiplikator_eq_5}
	\exists \lambda\in\mathbb{R}^m: f'(x) + \transpose{\lambda} g'(x) = 0\end{align}
\end{proposition}

\begin{*definition}
	$\lambda$ oben heißt \begriff{Lagrangescher Multiplikator}
\end{*definition}

\begin{remark}\vspace*{0pt}
\begin{itemize}
	\item Offenbar nur für $m\le n$
	\item $x$ mit \eqref{extremwerte_lokale_extrema_mit_gleichungsnebenbedingung_lagrange_multiplikator_eq_4} heißt \begriff{reguläres Extrema}.
	\item Kandidaten für Extrema bestimmen: \eqref{extremwerte_lokale_extrema_mit_gleichungsnebenbedingung_lagrange_multiplikator_eq_5} liefert $n$ Gleichungen für $n+m$ Unbekannte $(x,\lambda)$, \emph{aber} \eqref{extremwerte_lokale_extrema_mit_gleichungsnebenbedingung_lagrange_multiplikator_eq_5} mit $g(x) = 0$ liefert $n+m$ Gleichungen für $(x,\lambda)$
\end{itemize}
\end{remark}

\begin{proof}
	Vgl. Literatur.
\end{proof}

\begin{example}
	Bestimme reguläre Extrema von $f$ auf $G=\{ g=0\}$ mit
	\begin{alignat*}{2}
		f&:\mathbb{R}^3\to\mathbb{R},&\, (x,y,z) \mapsto\;&x^2 + y^2 + z^2 \\
		g&:\mathbb{R}^3\to\mathbb{R}^2,& (x,y,z) \mapsto& \begin{pmatrix}
	x^2 + 4y^2 - 1 \\ z
	\end{pmatrix}
	\end{alignat*}
	Betrachte $\transpose{\lambda} = (\lambda_1,\lambda_2)$: \begin{align}
		0 &= f'(x,y,z) + \transpose{\lambda}g'(x,y,z) = (2x, 2y, 2z) + \transpose{\lambda}\cdot\begin{pmatrix}
			2x & 8y & 0 \\ 0 & 0 & 1
		\end{pmatrix} \\
		\notag 0 &= g(x,y,z)
	\end{align}
	Das heißt \begin{align*}
		2x + 2\lambda_1 x &= 0 & x^2 + 4y^2 &= 1 \\
		2y+8\lambda_1 y &= 0 & z &= 0 \\
		2z + \lambda_2 &= 0 & &
	\end{align*}
	$\Rightarrow$ $z=0$, $\lambda_2 = 0$, und \begin{align*}
		x(1 + \lambda_1) &= 0 & y(1 + 4\lambda_1) &= 0 & x^2 + 4y^2 &= 1
	\end{align*}
	\begin{tabularx}{\linewidth}{@{}r@{\ }l@{\ }l@{\ }c@{\ }r@{\ }l@{\ }c@{\ }r@{\ }l@{\ \ }c@{\ \ }r@{$\,$}r@{$\,$}r@{$\,$}r@{$\,$}lX}
	falls:& $\bullet$ $x\neq 0$:& $\lambda_1$ & = & $-1$, &$y$ & = & $0$, &$x$ = $\pm 1$ &$\Rightarrow$& $($ & $\pm1$, & 0, &0 & $)$ & \multirow{2}{*}{$\left.\phantom{\dfrac{1}{1}}\right\} \text{Kandidaten für reguläre Extrema}$} \\
f		  &	$\bullet$ $x = 0$:  & $y$ & = &$\pm\frac{1}{2}$, &$\lambda_1$ & = & $-\frac{1}{4}$& & $\Rightarrow$& $($ & 0,&$\pm\frac{1}{2}$, &0 &$)$	\end{tabularx}

	Offenbar ist $\mathrm{rang}\;g'(x,y,z)=2$ für alle Kandidaten.
	
	Da $G$ Ellipse in der $x$-$y$-Ebene ist, und $f$ die Norm in's Quadrat, prüft man leicht: Minimum in $(0,\pm\frac{1}{2},0)$ und Maximum in $(\pm 1,0,0)$.
\end{example}

\subsection{Globale Extrema mit Abstrakter Nebenbedinung}
Betrachte $f:\overline{D}\subset\mathbb{R}\to\mathbb{R}$, $D$ offen, $f$ stetig auf $\overline{D}$, \gls{diffbar} auf $D$.

\begin{underlinedenvironment}[Existenz]
	nach \propref{chap_15_3}:
	
	$D$ beschränkt $\xRightarrow{\text{$\overline{D}$ kompakt}}$ $f$ besitzt auf $\overline{D}$ ein Minimum und ein Maximum
\end{underlinedenvironment}

\begin{underlinedenvironment}[Frage]
	Bestimme sogenannte globale Extremalstelle $x_{\min}$, $x_{\max}$.
\end{underlinedenvironment}

\begin{underlinedenvironment}[Strategie]\vspace*{0pt}
	\begin{enumerate}[label={\alph*)},topsep=\dimexpr-\baselineskip/2\relax]
		\item Bestimmte lokale Extrema in $D$
		\item Bestimme globale Extrema auf $\partial D$
		\item Vergleiche Extrema aus a) und b)
	\end{enumerate}
\end{underlinedenvironment}

\begin{example}
	Sei $f(x_1, x_2) = x_1^2 + \cos x_2$ mit $D=(-1,1)\times(0,4)$ (vgl. \propref{extremwerte_sylvester_beispiel_3}).
	
	Lokale Extrema in $D$: $f(0,\pi) = -1$ Minimum.
	
	Globale Extrema auf $\partial D$:\begin{itemize}
		\item $x_1 = \pm 1:$ Betrachte $x_2 \to f(\pm 1, x_2) = 1 + \cos x_2$ auf $[0,4]$.
		
		Offenbar $0 = f(\pm 1, \pi) \le f(\pm 1, x_2) \le f(\pm 1, 0) = 2$
		
		\item $x_2=0$: $x_1\to f(x_1, 0) = x_1^2 + 1$ auf $[-1,1]$
		
		Offenbar $1=f(0,0) \le f(x_1,0)\le f(\pm 1, 0) = 2$
		
		\item $x_2 = 4$: Betrachte $x_1\to x_1^2+\cos 4$ mit $[-1,1]$
		
		$\cos 4 \le f(0,4) \le f(x_1, 4) \le f(\pm 1, 4) = 1 + \cos 4$
	\end{itemize}
	Vergleich liefert:$ x_{\min}=(0,\pi)$, $x_{\max} = (\pm 1,0)$
	
	\begin{underlinedenvironment}[Hinweis]
		Bentze für Extrema evtl. partielle Ableitungen \begin{align*}
			f_{x_2}(\pm 1, x_2) &= -\sin x_2 = 0 \\
			\text{bzw.}\quad f_{x_1}(x_1, 0) &= 2x_1 = 0 \qquad\qquad\text{usw.}
		\end{align*}
	\end{underlinedenvironment}
\end{example}