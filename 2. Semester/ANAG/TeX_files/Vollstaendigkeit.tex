\section{Vollständigkeit}
\begin{definition}[\person{Cauchy}-Folge]
	Folge $\{x_n\}$ im metrischen Raum $(X,d)$ heißt \gls{cf} (Fundamentalfolge), falls
    \[
    \forall\epsilon > 0 \,\exists n_0\in\mathbb{N}: d(x_n, x_m) < \epsilon\quad\forall n,m\ge n_0.
    \]
\end{definition}
\begin{proposition}
	Sei $\{x_n\}$ Folge im metrischen Raum $(X,d)$. Dann
	\begin{enumerate}[label={\arabic*)}]
		\item $x_n\rightarrow x \Rightarrow \{x_n\}$ ist \person{Cauchy}-Folge
		\item $\{x_n\}$ \gls{cf} $\Rightarrow \{x_n\}$ ist beschränkt und hat maximal einen \gls{hw}.
	\end{enumerate}
\end{proposition}
\begin{definition}[Durchmesser]
	\begriff{Durchmesser} von $M\subset X$ beschränkt, $\neq 0$, $(X,d)$ metrischer Raum ist \mathsymbol{diam}{$\diam$}$M:=\sup\{d(x,y) | x,y\in M\}$
	
	Folge $\{A_n\}$ von abgeschlossenen Mengen heißt \begriff{Schachtelung} falls $A_n\neq\emptyset, A_{n+1}\subset A_n\,\forall n\in\mathbb{N}$ und $\diam A_n\overset{n\rightarrow\infty}{\longrightarrow}0$.
\end{definition}
\begin{lemma}
	Sei $M\subset X$ beschränkt, $\neq 0\;\Rightarrow\;\diam M = \diam (\cl M)$.
\end{lemma}
\begin{theorem}
	Sei $(X,d)$ metrischer Raum. Dann: für jede Schachtelung $A_n$ in $X$ gilt:\[ \bigcap_{n\in\mathbb{N}} A_n\neq \emptyset \;\Leftrightarrow \; \text{jede \gls{cf} in $\{x_n\}$ in $X$ ist konvergent} \]
\end{theorem}
\begin{lemma}
	In $\mathbb{R}$ gilt:
	\begin{center}
		\begin{tabular}{lcl}
			$\bigcap_{n\in\mathbb{N}} A_n\neq \emptyset$ & $\Leftrightarrow$ & $\bigcap_{n\in\mathbb{N}} X_n\neq \emptyset$ \\[5pt]
			$\forall$ Schachtelungen $\{A_n\}$ && $\forall$ Intervallschachtelungen $\{x_n\}$
		\end{tabular}
	\end{center}
\end{lemma}
\begin{definition}[Vollständigkeit]
	Metrischer Raum $(X,d)$ heißt \begriff{Vollständig}, falls jede \person{Cauchy}-Folge $\{x_n\}$ in $X$ konvergiert.
	
	Vollständiger, normierter Raum $(X,\Vert .\Vert)$ heißt \begriff{\person{Banach}-Raum}.
\end{definition}
\begin{conclusion}
	Sei $\{x_n\}$ Folge im vollständigen metrischen Raum $(X,d)$. Dann:\[ \{x_n\}\text{ konvergent}\;\Leftrightarrow\; \{x_n\} \text{ \person{Cauchy}-Folge} \]
\end{conclusion}
\begin{theorem}
	$\mathbb{R}^n$ und $\mathbb{C}^n$ mit $|.|_p$ ($1\le p \le \infty$) sind vollständige, normierte Räume (d.h. \person{Banach}-Räume).
\end{theorem}