\section{Wiederholung und Motivation}
Sei $K^n$ $n$-dim. \gls{vr} über Körper mit $K=\mathbb{R}$ oder $K=\mathbb{C}, n\in\mathbb{N}_{\ge 0}$.
\begin{itemize}
	\item Elemente sind alle $x=(x_1, \dotsc, x_n)\in K^n$ mit $x_1, \dotsc, x_n\in K$.
	\item \begriff{Standardbasis} ist $\{e_1, \dotsc, e_n\}$ mit $e_j=(0,\dotsc,0,\underbrace{1}_{\text{$j$-te Stelle}},0,\dotsc,0)$
	\item alle Normen auf $K^n$ sind äquivalent (\propref{aeqv_norm}) \\
	$\Rightarrow$ Kovergenz unabhängig von der Norm
	
	Verwende idR euklidische Norm $\vert x \vert_2 = \vert x \vert = \sqrt{\sum_{i}\vert x_i \vert^2}$
	\item \begriff{Skalarprodukt}
	\begin{itemize}
		\item $\langle x,y \rangle = \sum_{j=1}^{n} x_j y_j$ in $\mathbb{R}$
		\item $\langle x,y \rangle = \sum_{j=1}^{n} \overline{x_j} y_j$ in $\mathbb{C}$
	\end{itemize}
	\item \textsc{Cauchy}-\text{Schwarz}-Ungleichung ($\vert \langle x,y\rangle \vert \le \vert x \vert + \vert y \vert\,\forall x,y\in K^n$)
\end{itemize}

\subsection*{Lineare Abbildungen}
\proplbl{definition_tensorprodukt}
Eine \begriff{lineare Abbildung} ist homogen und additiv (siehe \propref{defLinearFunction}).
\begin{itemize}
	\item Lineare Abbildung $A: K^n \rightarrow K^m$ ist darstellbar durch $m\times n$-Matrizen bezüglich der Standardbasis 
	(\emph{beachte:} $A$ sowohl Abbildung als auch Matrix)
	\begin{itemize}
		\item lineare Abbildung ist stetig auf endlich-dimensionalen Räumen (unabhängig von der Norm, siehe \propref{chap_15_5})
		\item transponierte Matrix: $A^T\in K^{m\times n}$
		
		\begin{hint}
		$x=(x_1,\dotsc, x_n)\in K^n$ idR platzsparender als Zeilenvektor geschrieben, \emph{aber} bei Matrix-Multiplikation $x$ Spalten-Vektor, $x^T$ Zeilenvektor, d.h.	\begin{align*}
		 x^T \cdot y &= \langle x,y\rangle, &&\text{falls $m=n$} \\
		 x \cdot y^T &= x \otimes y\in K^{m\times n}, && \text{sog. \begriff{Tensorprodukt}}
		 \end{align*}
		\end{hint}
	\end{itemize}	
	 \item \mathsymbol{L}{$L(K^n, K^m)$}$ = \{ A: K^n \to K^m, \text{ $A$ linear}\}$ (Menge der linearen Abbildung, ist normierter Raum)
	\begin{itemize}
		 \item \mathsymbol{|A|}{$\Vert A \Vert$}$= \sup\{ \vert Ax\vert \mid \vert x \vert \le 1 \}$ (\begriff{Operatornorm}, $\Vert A \Vert$ hängt i.A. von Normen auf $K^n, K^m$ ab)
		 \item $L(K^n, K^m)$ ist isomorph zu $K^{m\times n}$ als \gls{vr} \\
		 $\Rightarrow$ $L(K^n, K^m)$ ist $m\cdot n$-dim. \gls{vr} ($\Rightarrow$ alle Normen äquivalent, $\Rightarrow$ Konvergenz von $\{A_n\}$ von linearer Abbildungen in $L(K^n, K^m)$ ist normunabhängig)
		 
		 Nehmen idR statt $\Vert A \Vert$ \person{euklidische} Norm $\vert A \vert = \sqrt{\sum_{k,l} \vert A_{k,l} \vert ^2}$.\\
		 Es gilt: \[ \vert Ax \vert \le \Vert A \Vert \vert x \vert \text{ und } \vert Ax\vert \le \vert A \vert \vert x \vert \]
	\end{itemize}
	\item Abbildung $\tilde{f}: K^n \to K^m$ heißt \begriff[linear!]{affin} \highlight{linear}, falls $\tilde{f}(x) = Ax + a$ für lineare Abbildung $A:K^n\to K^m, a\in K^m$
\end{itemize}

\subsection*{\textsc{Landau}-Symbole}
Sei $f:D\subset K^n \to K^m$, $g:D\subset K^n \to K$, $x_0 \in \overline{D}$. Dann:
\begin{itemize}
	\item $f(x) = o(g(x))$ für $x\to x_o$ \gls{gdw} $\lim\limits_{\substack{x\to x_0 \\ x\neq x_0}} \frac{\vert f(x) \vert}{g(x)} = 0$
	\item $f(x) = \mathcal{O}(g(x))$ für $x\to x_0$ \gls{gdw} $\exists \delta > 0, c \ge 0: \frac{\vert f(x) \vert}{\vert g(x) \vert} \le c \;\forall x\in \left( B_\delta(x_0)\setminus \{ x_0\}\right) \cap D$
	
	\emph{wichtiger Spezialfall:} $g(x) = \vert x - x_0\vert ^k, k\in\mathbb{N}$
\end{itemize}

\begin{example}
	Sei $f:D\subset K^n\to K^m$, $x_0\in D$ \gls{hp} von $D$. Dann:
	\begin{align}
		\notag f\text{ stetig in } x_0 &\Leftrightarrow \lim\limits_{\substack{x\to x_0 \\ x\neq x_0}} f(x) = f(x_0) \\
		\notag &\Leftrightarrow \lim\limits_{\substack{x\to x_0 \\ x\neq x_0}} \frac{f(x) - f(x_0)}{1} = 0 \\
		&\Leftrightarrow \boxed{f(x) = f(x_0) + o(1)} \text{ für }x\to x_0\proplbl{chap15specialCase}
	\end{align}
	
	\begin{interpretation}[von \propref{chap15specialCase}]{}
	
	Setze $r(x) := f(x) - f(x_0)$
	\zeroAmsmathAlignVSpaces
	\begin{flalign}
		&\notag \overset{\text{(\ref{chap15specialCase})}}{\Rightarrow} r(x) = o(1) \text{ für } x\to x_0& \\
		&\label{chap15interpretationSpecialCase} \Rightarrow r(x) \overset{x\to x_0}{\longrightarrow} 0,&
	\end{flalign}
	d.h. $o(1)$ ersetzt eine "`Rest-Funktion"' $r(x)$ mit Eigenschaft (\ref{chap15interpretationSpecialCase}).
	\end{interpretation}
	Wegen $o(1) = o(\vert x-x_0\vert^0)$ (d.h. $k=0$) sagt man auch, \propref{chap15specialCase} ist die Approximation 0. Ordnung der Funktion $f$ in der Nähe von $x_0$.
\end{example}
\begin{example}
	Sei $f:D\subset \mathbb{R}^n\to \mathbb{R}$, $x_0\in D$, $D$ offen. Was bedeutet \begin{align}
		\proplbl{chap15meaningSpecialCase} f(x) = f(x_0) + o(\vert x-x_0\vert),\;x\to x_0?
	\end{align}
	\begin{enumerate}[label={\alph*)}]
		\item Betrachte $f$ auf Strahl $x=x_0 + ty$, $y\in\mathbb{R}^n$ fest, $\vert y \vert = 1$, $t\in\mathbb{R}$.
		\begin{align*}
			\text{\propref{chap15meaningSpecialCase}} \Rightarrow& \;0 = \lim\limits_{\substack{x\to x_0 \\ x\neq x_0}} \frac{\vert f(x) - f(x_0)\vert}{\vert x - x_0\vert} = \lim\limits_{\substack{x\to x_0 \\ x\neq x_0}} \frac{\vert f(x_0 + ty) - f(x_0) \vert}{\vert t \vert}
		\end{align*}
		
		\item \zeroAmsmathAlignVSpaces[\dimexpr -\baselineskip - \parskip\relax]
		\begin{flalign*}
			\text{\propref{chap15meaningSpecialCase}} \;\Rightarrow&\; f(x) = f(x_0) + \underbrace{\frac{o(\vert x - x_0\vert)}{\vert x - x_0\vert}}_{=o(1)} \cdot \vert x - x_0\vert, \;x\to \infty \\
			\Rightarrow& f(x) = f(x) + \underbrace{o(1)}_{\mathclap{\text{anstatt }r(x)\text{ gemäß (\ref{chap15interpretationSpecialCase})}}}\cdot \vert x-x_0\vert, \;x\to x_0 \\
			\Rightarrow& f(x) = f(x_0) + r(x) \cdot \vert x - x_0\vert \\
			\Rightarrow& \vert f(x) - f(x_0)\vert \le \rho(t) \cdot \vert x - x_0\vert,\\
			& \rho(t) := \sup\limits_{\vert x - x_0\vert \le t} \vert r(x)\vert \xrightarrow{t\to\infty} 0
		\end{flalign*}
		Der Graph von $f$ liegt in der Nähe von $x_0$ in immer flacheren, kegelförmigen Mengen\\
		$\Rightarrow$ Graph "`schmiegt sich"' an eine horizontale Ebene an den Punkt $(x_0, f(x_0))$ an.
		
		\item \emph{Beobachtung:} horizontale Ebene ist Graph einer affin linearen Funktion $\tilde{A}: \mathbb{R}^n\to\mathbb{R}^n$, daher\\
		\emph{Zentrale Frage:} Existiert zu gesuchter Funktion $f: D\subset\mathbb{R}^n \to K^m$, $x_0\in\mathbb{R}$ eine affin lineare Funktion $\tilde{A}:K^n\to K^m$, sodass sich in der Nähe von $x_0$ der Graph von $f$ an den Graph von $\tilde{A}$ "`anschmiegt"'?
		
		Wegen $f(x_0) = \tilde{A}(x_0)$ folgt $\tilde{A}(x) = A(x-x_0) + f(x_0)$. Was heißt "`anschmiegen"'? $f(x) + \underbrace{f(x_0) + A(x-x_0)}_{\tilde{A}(x)} = o(\vert x-x_0\vert)$, d.h. die Abweichung wird schneller klein als $\vert x-x_0\vert$!
	\end{enumerate}
\end{example}