\section{Konvergenz}\setcounter{theorem}{0}
\begin{*definition}[konvergent]
	Sei $(X,d)$ metrischer Raum. Folge $\{x_n\}_{n\in\mathbb{N}}$ in $X$, (d.h. $x_n\in X\,\forall n$) heißt \begriff[Folge!]{konvergent}, falls $x\in X$ existiert mit \[\forall \epsilon > 0 \,\exists n_0=n_0(\epsilon)\in\mathbb{N}: d(x_n, x) < \epsilon\quad \forall n\ge n_0\]
	
	$x$ heißt dann \begriff{Grenzwert} (auch Limes) der Folge.
	
	Notation: $x=$\mathsymbol{lim}{$\lim\limits_{n\rightarrow\infty}$}, $x_n\rightarrow x$ für $n\rightarrow\infty$, $x_n \overset{n\rightarrow\infty}{\longrightarrow}x$
	
	Folge heißt \begriff[Folge!]{divergent}, falls nicht konvergent.
\end{*definition}

\begin{conclusion}
	Für Folge $\{x_n\}$ gilt: \[ x=\lim\limits_{n\rightarrow\infty}x_n \;\Leftrightarrow \text{Jede Kugel $B_\epsilon(x)$ enthält fast alle $x_n$} \]
\end{conclusion}

\begin{example}
	\begin{itemize}
		\item konstante Folge: Sei $\{x_n\}_n=\{x\}_n\in\natur$, d.h. $x=x_n$
		\item $X=\real$: Folge $\{\frac{1}{n}\}$ konvergent, Grenzwert 0
		\item $X=\real$: $\lim\limits_{n\to\infty} \sqrt[n]{x}=1$
		\item $X=\real$: $\{-1\}^n$ ist divergent
	\end{itemize}
\end{example}

\begin{proposition}[Eindeutigkeit des Grenzwertes]
	Sei $(X,d)$ metr. Raum, $\{x_n\}$ Folge in $X$. Dann \[ x,x' \text{ Grenzwert von $\{x_n\}$} \;\Rightarrow\; x = x' \]
\end{proposition}
\begin{proof}
	Sei $\varepsilon:=\frac{1}{3}$, $d(x,x')>0\Rightarrow\exists m\in\natur:d(x_m,x)<\varepsilon$, $d(x_m,x')<\varepsilon$\\
	$3\varepsilon=d(x,x')\le d(x_m,x)+d(x_m,x')< 2\varepsilon\Rightarrow\lightning\Rightarrow d(x,x')=0$
\end{proof}

\begin{proposition}
	Sei $(X,d)$ metrischer Raum, $\{x_n\}$ konvergente Folge in $X$\\
    $\Rightarrow$ $\{x_n\}$ ist beschränkt.
\end{proposition}
\begin{proof}
	Sei $\lim\limits_{n\to\infty} x_n=x\Rightarrow$ für $\varepsilon=1\exists n_0:d(x_n,x)<1$ mit $r=\max\{d(x,x_n)\}+1$ folgt: $x_n\in B_r(x)\Rightarrow$ beschränkt
\end{proof}

\begin{example}
	$X=\real$ mit diskreter Metrik: betrachte $\{x_n\}$ \\
	angenommen $\lim\limits_{n\to\infty} x_n=x\Rightarrow$ für $\varepsilon=\frac{1}{2}\exists n_0: x_n\in B_{0,5}(x)=\{x\}$ \\
	$\Rightarrow$ fast alle $x_n$ sind gleich $x$ bei Konvergenz $\Rightarrow \left\lbrace \frac{1}{n}\right\rbrace$ ist divergent $\Rightarrow$ Konvergenz ist abhängig von Metrik
\end{example}

\begin{example}
	$X=\comp$ mit $\vert\cdot\vert$. betrachte $\{z^n\}$ für $z\in\comp$
	\begin{itemize}
		\item $\vert z \vert < 1$: $\forall\varepsilon >0\exists n_0:\vert z^n-n_0\vert<\varepsilon\Rightarrow\lim\limits_{n\to\infty} z^n=0$
		\item $\vert z \vert > 1$: $\forall r>0\exists n_0:\vert z^{n_0}-0\vert=\vert z\vert^{n_0}>r\Rightarrow$ es gibt also kein $r>0\Rightarrow \{z^n\}$ ist nicht beschränkt $\Rightarrow$ divergent
		\item $z=1$ offenbar $\lim\limits_{n\to\infty} 1^n=1$
		\item $\vert z\vert =1$, aber $z\neq 1$: angenommen $\lim\limits_{n\to\infty} z^n=\tilde z\Rightarrow \varepsilon=\frac{1}{2}\vert z-1\vert\Rightarrow \vert z-\tilde{z}\vert<\varepsilon\Rightarrow 2\varepsilon=\vert z-1\vert=\vert z^{n_0}\vert\cdot\vert z-1\vert=\vert z^{n_0}+1-\tilde{z}+\tilde{z}-z^{n_0}\vert\le\vert z^{n_0}+1-\tilde{z}\vert+\vert\tilde{z}-z^{n_0}\vert< 2\varepsilon\Rightarrow\lightning\Rightarrow \{z^n\}$ divergent
	\end{itemize}
\end{example}

\begin{example}
	$\lim\limits_{n\to\infty} \sqrt[n]{n}=1$, denn: \\
	\begin{align*}
		x_n:=\sqrt[n]{n}-1&\ge 0 \\
		n=(1+x_n)^n&\ge 1+\binom{n}{2}\cdot x_n^2 \\
		n-1 &\ge n\frac{n-1}{2x_n^2} \\
		x_n=\sqrt[n]{n}-1 &\le \sqrt{\frac{2}{n}}\le \varepsilon
	\end{align*}
\end{example}

\begin{example}
	$\lim\limits_{n\to\infty} \frac{\log_a n}{n}=0$ für $a>1$, denn \\
	$1<\sqrt[n]{n}< a^{\varepsilon}\Rightarrow 0 < \frac{\log_a n}{n} < \varepsilon$
\end{example}

\begin{*definition}[Teilfolge, Häufungswert]
	Sei $\{x_n\}$ beliebige Folge in $X$, $\{n_k\}_{k\in\mathbb{N}}$ Folge in $\mathbb{N}$ mit $n_{k+1} > n_k\,\forall k\in\mathbb{N}$. Dann heißt $\{x_{n_k}\}_{k\in\mathbb{N}}$ \gls{tf} von $\{x_n\}_{n\in\mathbb{N}}$.
	
	$\gamma\in X$ heißt \gls{hw} (auch Häufungspunkt) der Folge $\{x_n\}$, falls $\forall \epsilon > 0$ enthält $B_\epsilon(\gamma)$ unendlich viele $x_n$.
	\begin{underlinedenvironment}[beachte]
		HP der Folge muss nicht HP der Menge $\{x_n\}$ sein, z.B. konstante Folge
	\end{underlinedenvironment}
\end{*definition}

\begin{proposition}\label{tfprinzip}
	\proplbl{tfprinzip}
	Sei $\{x_n\}$ Folge im metrischen Raum $(X,d)$. Dann
	\begin{enumerate}[label={\arabic*)}]
		\item $x_n\rightarrow x \;\Rightarrow\; x_{n_k} \overset{n\rightarrow\infty}{\longrightarrow} x$ für jede \gls{tf} $\{x_{n_k}\}_k$
		\item $\gamma$ ist \gls{hw} der Folge $\{x_n\}$ $\Leftrightarrow$ $\exists$\gls{tf} $\{x_{n_k}\}: x_{n_k} \overset{n\rightarrow\infty}{\longrightarrow} \gamma$
		\item \begriff{Teilfolgenprinzip}: Jede \gls{tf} $\{x_{k'}\}$ von $\{x_n\}$ hat \gls{tf} $\{x_{k''}\}$ mit $x_{n''}\rightarrow x$ $\Rightarrow$ $x_n \rightarrow x$
	\end{enumerate}
\end{proposition}
\begin{proof}
	\begin{enumerate}
		\item folgt aus Definition
		\item $(\Rightarrow):\exists n_k:x_{n_k}\in B_{\frac{1}{k}}(x), n_{k+1}>n_k\Rightarrow \{x_{n_k}\}$ ist TF mit $x_{n_k}\to x$ \\
		$(\Leftarrow):x_{n_k}\to x\Rightarrow B_{\varepsilon}(x)$ fast alle $x\beha$
		\item Übungsaufgabe
	\end{enumerate}
\end{proof}

\begin{example}
	$\{(-1)^n\}$ hat TF $\{(-1)^{2k}\}$ und $\{(-1)^{2k+1}\}$ mit Grenzwert +1 und -1 $\Rightarrow \{(-1)^n\}$ ist divergent, da es 2 HW gibt.
\end{example}

\begin{proposition}
	Sei $(X,d)$ metrischer Raum, $M\subset X$ Teilmenge. Dann
	\[ M\text{ abgeschlossen} \quad\Leftrightarrow\quad \text{für jede konv. Folge $\{x_n\}$ in $M$ gilt: }\lim\limits_{n\rightarrow\infty} x_n\in M \]
\end{proposition}
\begin{proof}
	$(\Rightarrow):$ sei $\{x_n\}\in M$ mit $x_n\to x\notin M\Rightarrow\exists\varepsilon:B_{\varepsilon}\subset X\backslash M\Rightarrow x_n\not\to x\lightning\beha$ \\
	$(\Leftarrow):$ sei $X\backslash M$ nicht offen, also abgeschlossen $\Rightarrow\exists x\in X\backslash M:B_{\varepsilon}(x)\cap M\neq\emptyset\Rightarrow\exists x_n\in B_{\frac{1}{n}}(x)\cap M\Rightarrow x_n\to x\in M\lightning\Rightarrow X\backslash M$ offen
\end{proof}

\subsection{Konvergenz im normierten Raum \texorpdfstring{$X$}{X}}
$x_n\to x$ in $(X,\Vert .\Vert)$ und $\lambda_n\to\lambda$ in $(\real,\vert\cdot\vert)$

\begin{proposition}
	Sei $X$ normierter Raum, $\{x_n\}, \{y_n\}$ in $X$, $\{\lambda_n\}$ in $K$ mit $\lim x_n = x, \lim y_n = y$. Dann
	\begin{enumerate}[label={\arabic*)}]
		\item $\{x_n \pm y_n\}$ konvergiert und $\lim\limits_{n\rightarrow\infty}x_n \pm y_n = \lim\limits_{n\rightarrow\infty} x_n \pm \lim\limits_{n\rightarrow\infty} y_n$
		\item $\{\lambda_n x_n\}$ konvergiert und $\lim\limits_{n\rightarrow\infty} \lambda_n x_n = \lim\limits_{n\rightarrow\infty} \lambda_n \cdot \lim\limits_{n\rightarrow\infty}x_n$
		\item $\lambda\neq 0 \;\Rightarrow\;\lim\limits_{n\rightarrow\infty} \frac{1}{\lambda_n} = \frac{1}{\lambda}$ (in $K$) für $\{\frac{1}{\lambda_n}\}_{n\ge\tilde{n}}$ ($\lambda_n\neq 0\,\forall n\ge\tilde{n}$)
	\end{enumerate}
\end{proposition}
\begin{proof}
	\begin{enumerate}
		\item Übungsaufgabe
		\item $\{x_n\}$ beschränkt $\Rightarrow\exists r>\vert\lambda\vert>0:\Vert rx_n\Vert\le r$\\
		$\varepsilon>0\Rightarrow\exists n_0:\vert \lambda_n-\lambda\vert<\frac{\varepsilon}{2r},\Vert x_n-x\Vert < \frac{\varepsilon}{2r}$ \\
		$\Rightarrow \Vert \lambda_{x_n}-\lambda_n\Vert\le \Vert\lambda_nx_n-\lambda x_n\Vert+\Vert\lambda x_n\lambda x\Vert=\vert\lambda_n-\lambda\cdot\Vert x_n\Vert+\vert\lambda\vert\cdot\Vert x_n-x\Vert\le \frac{\varepsilon}{2r}\cdot r+r\cdot\frac{\varepsilon}{2r}=\varepsilon\beha$
		\item offenbar: $\exists\tilde n:\lambda_n\neq 0$ für $\varepsilon>0\exists n_0:\vert\lambda-\lambda_n\vert<m\cdot n\cdot \left\lbrace \left(\frac{\vert x\vert}{2}\right),\left(\frac{\varepsilon\cdot\vert\lambda\vert^2}{2}\right)\right\rbrace\Rightarrow\frac{1}{2}\cdot\vert\lambda\vert\le \vert\lambda\vert-\vert\lambda+\lambda_n\vert\le \lambda_n\Rightarrow ... \Rightarrow$ Behauptung
	\end{enumerate}
\end{proof}

\begin{conclusion}
	Seien $\{\lambda_n\}, \{\mu_n\}$ Folgen in $K$ mit $\lambda_n\rightarrow\lambda,\mu_n\rightarrow\mu$. Dann
	\begin{enumerate}[label={\arabic*)}]
		\item $\lambda_n + \mu_n\rightarrow \lambda + \mu, \lambda_n \mu_n\rightarrow\lambda \mu$
		\item falls $\lambda\neq 0$ (\gls{obda} $\lambda_n\neq 0$): $\frac{\mu_n}{\lambda_n}\rightarrow\frac{\mu}{\lambda}$
	\end{enumerate}
\end{conclusion}

\begin{*definition}[Nullfolge]
	$\{x_n\}$ im normierten Raum heißt \begriff{Nullfolge}, falls $x_n\to 0$
\end{*definition}

\begin{lemma}
	\proplbl{sandwich_lemma}
	\begin{enumerate}[label={\arabic*)}]
		\item Im metrischen Raum $X$ gilt:$x_n\rightarrow x$ in $X$ $\Leftrightarrow\;d(x_n,x)\rightarrow 0$ in $\mathbb{R}$
		\item Sei $0\le \alpha_n\le\beta_n\,\forall n\in\mathbb{N}, \alpha_n, \beta_n\in\mathbb{R}, \beta_n\rightarrow 0$\\
		$\Rightarrow \alpha_n\rightarrow 0$ \begriff{Sandwich-Prinzip}
	\end{enumerate}
\end{lemma}
\begin{proof}
	\begin{enumerate}
		\item benutze $d(x_n,x)<\varepsilon\iff\vert d(x_n,x)-0\vert<\varepsilon$
		\item $\varepsilon>0\Rightarrow\exists n:\beta_n=\vert\beta_n-0\beta_n\vert<\varepsilon\Rightarrow \alpha_n=\vert\alpha_n-0\vert\le \beta_n<\varepsilon\beha$
	\end{enumerate}
\end{proof}

\begin{proposition}
	Sei $X$ normierter Raum, $\{x_n\}$ in $X$. Dann\\
	$x_n\rightarrow x$ in $X$ $\Rightarrow$ $\Vert x_n\Vert \rightarrow\Vert x\Vert$ in $\mathbb{R}$
\end{proposition}
\begin{proof}
	$0\le \vert \;\Vert x_n\Vert-\Vert x\Vert\;\vert\le \Vert x_n-x\Vert\to 0\overset{\text{\propref{sandwich_lemma}}}{\Rightarrow}$ Behauptung
\end{proof}

\begin{proposition}
	Seien $(X,\Vert .\Vert_1)$, $(X,\Vert.\Vert_2)$ normierte Räume mit äquivalenten Normen. Dann
	
	$x_n\rightarrow x$ in $(X,\Vert.\Vert_1)$ $\Leftrightarrow$ $x_n\rightarrow x$ in $(X,\Vert.\Vert_2)$
\end{proposition}
\begin{proof}
	Es gibt $a,b>0:a\cdot\Vert y\Vert_1\le \Vert y\Vert_2\le b\cdot\Vert y\Vert_1$ \\
	$(\Rightarrow)$: es ist $0\le \Vert x_n-x\Vert_2\le b\cdot\Vert x_n-x\Vert_1\to 0\beha$ \\
	$(\Leftarrow)$: analog
\end{proof}

\begin{example}
	$X=\real^n$ bzw. $\comp^n$: $x_n\to x$ bezüglich $\Vert .\Vert_1\iff x_n\to x$ bezüglich $\Vert .\Vert_2$, somit Konvergenz in $\real^n$ bzw. $\comp^n$ unabhängig von Norm.
\end{example}

\begin{proposition}[Konvergenz in $\mathbb{R}^n$/$\mathbb{C}^n$ bzgl. Norm]
	Sei $\{x_n\}$ Folge mit $x_n = (x_n^1, \dotsc, x_n^n)\in\mathbb{R} (\mathbb{C}^n)$, $x=(x^1, \dotsc,x^n)\in\mathbb{R}^n (\mathbb{C}^n)$.
	
	$\lim\limits_{n\rightarrow\infty} x_n = x$ in $\mathbb{R}^n (\mathbb{C}^n)$ $\Leftrightarrow$ $\lim\limits_{n\rightarrow\infty} x_k^j = xj$ in $\mathbb{R}$ bzw. $\mathbb{C}\,\forall j=1,\dotsc,n$
\end{proposition}
\begin{proof}
	nur in $\real^n$ \\
	$(\Rightarrow)$: sei $x_k\to x$ in $\real^n$ bezüglich $\vert\cdot\vert_p\Rightarrow x_n\to x$ bezüglich $\vert\cdot\vert_\infty$. Wegen $\vert x_k^j-x^j\vert\le \vert x_k-x\vert_\infty\to 0$ hieraus folgt die Behauptung \\
	$(\Leftarrow)$: sei $x_k^j\to x^j\Rightarrow\vert x_k-x\vert_1=\vert x_k^1-x^1\vert +...+\vert x_k^n-x^n\vert\to 0\Rightarrow x_k\to x$ bezüglich $\vert\cdot\vert_1\beha$
\end{proof}

\begin{underlinedenvironment}[Hinweis]
	zukünftig bei Konvergenz in $\real^n$ oder $\comp^n$ in der Regel keine Angabe der konkreten Norm.
\end{underlinedenvironment}

\begin{remark}
	offenbar gilt: \\
	$z_n=x_n+iy_n\to z=x+iy\iff (x_n,y_n)\to (x,y)$ in $\real^2$ bezüglich $\vert\cdot\vert\iff \realz(z_n)\realz(z)$ und $\imagz(z_n)\to\imagz(z)$
\end{remark}

\begin{example}
	$\{x_k\}=\{(\sqrt{k+1}-\sqrt{k},\sqrt{k+\sqrt{k}}-\sqrt{k})\}$ Folgen in $\real^2$ \\
	es ist $0\le x_k^1=\sqrt{k+1}-\sqrt{k}=\frac{1}{\sqrt{k+1}+\sqrt{k}}<\frac{1}{\sqrt{k}}\to 0\Rightarrow x_k^1\to 0$ \\
	$x_k^2=\sqrt{k+\sqrt{k}}-\sqrt{k}=\frac{\sqrt{k}}{\sqrt{k+\sqrt{k}}+\sqrt{k}}=\frac{1}{\sqrt{1+\frac{1}{\sqrt{k}}}+1}\to \frac{1}{2}$ \\
	$\Rightarrow \lim\limits_{k\to\infty} x_k=\frac{1}{2}$
\end{example}

\begin{example}
	$z_k=\frac{1+ki}{1+k}\to i$, denn: \\
	$\realz(z_n)=\frac{1}{1+k}\to 0$ und $\imagz(z_n)=\frac{k}{k+1}\to 1\Rightarrow\to (0,1)=i$
\end{example}

\subsection{Konvergenz in \texorpdfstring{$\mathbb{R}$}{R}}
\begin{proposition}
	Seien $\{x_n\},\{y_n\},\{z_n\}$ Folgen in $\mathbb{R}$. Dann
	\begin{enumerate}[label={\arabic*)}]
		\item $x_n \le y_n\,\forall n\ge n_0, x_n\rightarrow x, y_n\rightarrow y\;\Rightarrow x\le y$
		\item $x_n\le y_n\le z_n\,\forall n\ge n_0, x_n\rightarrow c,z_n\rightarrow c \;\Rightarrow y_n\rightarrow c$ (\begriff{Sandwich-Prinzip})
	\end{enumerate}
\end{proposition}
\begin{proof}
	\begin{enumerate}
		\item angenommen $x>y$, sei $\varepsilon :=\frac{1}{2}(x-y)>0$ \\
		$\Rightarrow\exists m: x_n\in B_{\varepsilon}(x), y_n\in B_{\varepsilon}(y)$ \\
		$\Rightarrow y_n< y+\varepsilon=x-\varepsilon<x_n\Rightarrow\lightning\beha$
		\item offenbar $0\le y_n-x_n\le z_n-x_n\to 0\Rightarrow y_n-x_n\to 0\Rightarrow\to c$
	\end{enumerate}
\end{proof}

\begin{*definition}[monoton]
	Folge $\{x_n\}$ heißt \begriff[monoton!]{wachsend} / \begriff[monoton!]{fallend}, falls gilt:
	
	$x_n \le x_{n-1}\;(x_n\ge x_{n+1})\,\forall n\in\mathbb{N}$ (in beiden Fällen heißt Folge \begriff{monoton}).
	
	Falls stets "`$<$"' ("`$>$"') ist $\{x_n\}$ \begriff[monoton!]{strikt}
\end{*definition}

\begin{proposition}
	Sei $\{x_n\}$ in $\mathbb{R}$ monoton und beschränkt.\[
	\{x_n\}\text{ konvergiert gegen }x:=
	\left\lbrace
		\begin{aligned}
			&\sup \{x_n \mid n\in\mathbb{N}\}, \\
			&\inf\{x_n \mid n\in\mathbb{N}\}, \\
		\end{aligned}
	\right.
	\text{ falls monoton }\;
	\begin{aligned}
		&\text{wachsend}\\
		&\text{fallend}
	\end{aligned}
	\]
\end{proposition}
\begin{proof}
	Sei $\{x_n\}$ monoton wachsend und beschränkt $\Rightarrow x=\sup\{x_n\}$ existiert $\Rightarrow \varepsilon>0\Rightarrow \exists m:x-\varepsilon\le x_m\le x_n\le x\beha$ \\
	Monoton fallend analog
\end{proof}

\begin{example}
	Sei $x_{n+1}=\frac{1}{2}(x_n+\frac{a}{x_n})$ \\
	vollständige Induktion: $x_n>0$, somit $\{x_n\}$ rekursiv eindeutig definiert \\
	$\Rightarrow x_{n+1}^2-a=\frac{1}{4}(x_n+\frac{a}{x_n})^2-a=\frac{1}{4}(x_n-\frac{a}{x_n})^2\ge 0$\\
	$\Rightarrow x_n-x_{n+1}=\frac{1}{2x_n}(x_n^2-a)\ge 0$ \\
	$\Rightarrow \{x_n\}$ ist mon. fallend, beschränkt $\Rightarrow x_n\to x\in\real$ \\
	da $x_{n+1}\cdot x_n=\frac{1}{2}(x_n^2+a)\Rightarrow x^2=\frac{1}{2}(x^2+a)\Rightarrow x^2=a\Rightarrow \lim\limits_{n\to\infty} x_n=\sqrt{a}$
	
	\textbf{Fehlerabschätzung:} $x_{n+1}-\sqrt{a}=\frac{1}{2x_n}(x_n-\sqrt{a})^2\le \frac{1}{2\sqrt{a}}(x_n-\sqrt{a})^2$, so genannte \begriff[Konvergenz!]{quadratische} Konvergenz (schnelle Konvergenz, vgl. Newton-Verfahren), d.h. die Anzahl der signifikanten Dezimalstellen verdoppelt sich mit jedem Schritt!
\end{example}

\begin{example}
	$\lim\limits_{n\to\infty} \frac{z^n}{n!}=0$ \\
	betrachte reelle Folge $a_n:=\frac{\vert z^n\vert}{n!}\Rightarrow a_{n+1}=\frac{\vert z\vert}{n+1}a_n$ \\
	$\Rightarrow\exists\tilde{n}:\{a_n\}$ fallend $\left( \frac{\vert z\vert}{\tilde{n}+1}<1\right)\Rightarrow a_n\to a$ \\
	$\Rightarrow a=0\cdot a=0\Rightarrow \vert \frac{z^n}{n!}-0\vert=\frac{vert z\vert^n}{n!}\to 0\beha$
\end{example}

\begin{theorem}[\person{Bolzano}-\person{Weierstraß}]\label{bolzano_weierstrass}
	$\{x_n\}$ beschränkte Folge in $\mathbb{R}$ $\Rightarrow$ $\{x_n\}$ hat konvergente \gls{tf}.
\end{theorem}
\begin{proof}
	es gibt $y_0,y'_0:y_0\le x_n\le y'_0$ \\
	rekursive Definition von $y_n,y'_n\in\real$ \\
	$z{n+1}:=\frac{y_n+y'_n}{2}\Rightarrow \left\lbrace 
	\begin{aligned}
	 &\text{unendlich viele }y_n\in[z_{n+1},y'_n] & y_{n+1}=z_{n+1} &\quad y'_{n+1}=y'_n \\
	 &\text{sonst} & y_{n+1}=y_n &\quad y'_{n+1}=z_{n+1} \\
	\end{aligned}\right.\Rightarrow$ Folge $Y_n=[y_n,y'_n]$ ist Intervallschachtelung in $\real\Rightarrow\exists y\in\bigcap Y_n\Rightarrow y$ ist HW in $\{x_n\}\beha$ 
\end{proof}

\begin{example}
	$\{z_n\}$ für $z\in\comp,\vert z\vert=1,z\neq 1$: ist divergent, aber $\{\realz(z_n)\}$ und $\{\imagz(z_n)\}$ sind beschränkte Folgen in $\real$ \\
	$\Rightarrow\exists$ TF $\{n'\}$ von $\{n\}$ mit $\realz(z^{n'})\to \alpha$ \\
	$\Rightarrow\exists$ TF $\{n''\}$ von $\{n\}$ mit $\imagz(z^{n''})\to \beta$ \\
	$\Rightarrow z^n\to \alpha+i\beta\Rightarrow \{z_n\}$ hat konvergente TF in $\comp$!
\end{example}

\subsection{Oberer und Unterer Limes}
\begin{*definition}
	Seien $\{x_n\}$ beschränkte Folgen in $\mathbb{R}$.\\
	$H:=\{ \gamma\in\mathbb{R} \mid \gamma \text{ ist \gls{hw} von }\{x_n\}\}$ ($\neq \emptyset$ nach \ref{bolzano_weierstrass})
	
	\begin{tabularx}{\textwidth}{lX}
		\mathsymbol*{limsup}{$\limsup$} $\limsup\limits_{n\rightarrow\infty} x_n := \overline{\lim}_{n\rightarrow\infty} x_n =:\sup H$ & \begriff{Limes superior} von $\{x_n\}$ \\[0.5cm]
		\mathsymbol*{liminf}{$\liminf$} $\liminf\limits_{n\rightarrow\infty} x_n = \underline{\lim}_{n\rightarrow\infty} x_n :=\inf H$  & \begriff{Limes inferior} von $\{x_n\}$
	\end{tabularx}

\begin{underlinedenvironment}[beachte]
	$\limsup$ und $\liminf$ existieren stets für beschränkte Folgen!
\end{underlinedenvironment}
\end{*definition}

\begin{proposition}
	Sei $\{x_n\}$ beschränkte Folge in $\mathbb{R}$. Dann
	\begin{enumerate}[label={\arabic*)}]
		\item Sei $\{x_{n'}\}$ \gls{tf} mit $x_{n'}\rightarrow\gamma \;\Rightarrow \;\liminf\limits_{n\rightarrow\infty} x_n \le \gamma \le \limsup\limits_{n\rightarrow\infty} x_n$
		\item $\gamma' :=\liminf\limits_{n\rightarrow\infty} x_n$ und $\gamma'' := \limsup\limits_{n\rightarrow\infty} x_n$ sind \gls{hw} von $\{x_n\}$
		
		\begin{tabular}{ll}
		(folglich)& $\inf H = \min H, \sup H = \max H$ und \\
		& $\exists$ \gls{tf} $\{x_{n'}\}, \{x_{n''}\}, x_{n'}\rightarrow \gamma', x_{n''}\rightarrow\gamma''$
		\end{tabular}
		\item $x_n\rightarrow \alpha \;\Leftrightarrow \;\alpha = \liminf\limits_{n\rightarrow\infty} x_n = \limsup\limits_{n\rightarrow\infty} x_n$
	\end{enumerate}
\end{proposition}
\begin{proof}
	\begin{enumerate}
		\item $x\in H\overset{\text{\ref{tfprinzip}}}{\Rightarrow}$ Behauptung
		\item $\varepsilon>0\Rightarrow\exists x\in H\cap B_{\varepsilon}(x')$ \\
		$B_{\varepsilon}(x')$ offen $\Rightarrow\exists\tilde{\varepsilon}>0:B_{\tilde{\varepsilon}}(x')\subset B_{\varepsilon}(x')\Rightarrow$ unendlich viele $x_n$ in $B_{\varepsilon}(x')\Rightarrow$ Behauptung für $\liminf$
		\item Übungsaufgabe, Selbststudium
	\end{enumerate}
\end{proof}

\begin{example}
	$\{q_n\}\in\real$ sei Folge alle rationalen Zahlen in $(0,1)$ \\
	$\Rightarrow$ Menge aller HW ist $H=[0,1]\Rightarrow \liminf q_n=0$ und $\limsup q_n=1$
\end{example}

\subsection{Uneigentliche Konvergenz}
\begin{*definition}[Uneigentliche Konvergenz]
	Folge $\{x_n\}$ in $\mathbb{R}$ konvergiert \begriff[Konvergenz!]{uneigentlich} gegen $+\infty (-\infty)$, falls $\forall R>0\,\exists n_0\in\mathbb{N}: x_n \ge R (x_n \le -R)\,\forall n\ge n_0$
	
	(heißt auch \highlight{bestimmt divergent}) gegen $\infty$, "`uneigentlich"' wird meist weggelassen.
	
	\textbf{Notation}: $\lim\limits_{n\rightarrow\infty} x_n = \pm \infty$ bzw. $\xi_n\rightarrow \pm \infty$
\end{*definition}

\begin{example}
	$\lim\limits_{n\to\infty} \frac{n^2+1}{n+1}=+\infty$, denn für $R>0$ gilt: $\frac{n^2+1}{n+1}=\frac{n+\frac{1}{n}}{1+\frac{1}{n}}\ge \frac{n}{2}\ge R$ für $n\ge 2R$
\end{example}

\begin{proposition}[Satz von \person{Stolz}]
	Sei $\{x_n\},\{y_n\}$ Folgen in $\mathbb{R}, \{y_n\}$ sei stren monoton wachsend, $\{y_n\}\rightarrow\infty$\\
	$\Rightarrow \lim\limits_{n\rightarrow\infty} \frac{x_n}{y_n} = \lim\limits_{n\rightarrow\infty} \frac{x_{n+1} - x_n}{y_{n+1} - y_n}$, falls rechter Grenzwert existiert (endlich oder unendlich)
\end{proposition}
\begin{proof}
	Grenzwert rechts sei $g\in\real$, oBdA $y_n>0$. \\
	Sei $\varepsilon>0\Rightarrow n_0:\vert \frac{x_{n+1}-x_n}{y_{n+1}-y_n}-g\vert<\varepsilon\Rightarrow (g-\varepsilon)\cdot(y_{n+1}-y_n)\le x_{n+1}-x_n\le (g+\varepsilon)\cdot(y_{n+1}-y_n)\overset{(*)}{\Rightarrow} (g-\varepsilon)(y_m-y_{n_0})\le x_m-x_{n_0}\le (g+\varepsilon)(y_m-y_{n_0})\Rightarrow (g-\varepsilon)(1-\frac{y_{n_0}}{y_m})\le \frac{x_m}{y_m}\le (g+\varepsilon)(1-\frac{y_{n_0}}{y_m})+\frac{x_{n_0}}{y_m}\Rightarrow g-\varepsilon \le \liminf\frac{x_m}{y_m}\limsup\frac{x_m}{y_m}\le g+\varepsilon\Rightarrow\lim\limits_{m\to\infty}\frac{y_m}{x_m}=g$ \\
	$(*)\; \sum_{n=n_0}^{m-1}$
\end{proof}

\begin{example}
	$\lim\limits_{n\to\infty} \frac{n^k}{z^n}=0$ für $z\in\comp,\vert z\vert>1,k\in\natur_{>0}$ \\
	$k=1$: $\frac{n+1-n}{\vert z\vert^{n+1}-\vert z\vert^{n}}=\frac{1}{\vert z\vert}\to 0\beha$ \\
	$k>1$: $\frac{n^k}{\vert z\vert^n}=\left( \frac{n}{\sqrt[k]{\vert z\vert}}^n\right)^k\to 0^k=0\beha$
\end{example}

\begin{proposition}
	Sei $\{x_n\}$ mit $x_n\rightarrow x$ im normierten Raum $X$.\\
	$\Rightarrow\frac{1}{n}\sum_{j=1}^n x_j \overset{n\rightarrow\infty}{\longrightarrow} x$
\end{proposition}
\begin{proof}
	Es ist $\Vert\frac{1}{n}\sum_{j=1}^{n} x_j-x\Vert=\frac{1}{n}\sum_{j=1}^n x_j-x\le \frac{1}{n}\sum_{j=1}^n \Vert x_j-x\Vert =:c_n$ \\
	$\frac{\sum_{j=1}^{n+1}\Vert x_j-x\Vert-\sum_{j=1}^{n} \Vert x_j-x\Vert}{n+1-n}=\frac{\Vert x_j-x\Vert}{1}\to 0\Rightarrow c_n\to 0\beha$
\end{proof}