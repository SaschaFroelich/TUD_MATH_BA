\documentclass[ngerman,a4paper]{article}

\usepackage{amsmath}
\usepackage{amssymb}
\usepackage{enumitem}
\usepackage[left=2.1cm,right=3.1cm,bottom=3cm,top=1cm]{geometry}
\usepackage[ngerman]{babel}


\title{\textbf{\"Ubungsaufgaben f\"ur Lineare Algebra}}
\author{}
\date{}

\begin{document}
\maketitle

\renewcommand{\arraystretch}{1.5}

\section{Eigenwerte und Eigenvektoren}
\begin{align}
	\begin{pmatrix}
		3 & -1 & 0 \\ 2 & 0 & 0 \\ -2 & 2 & -1
	\end{pmatrix},
	\begin{pmatrix}
		1 & 2 & 3 \\ 4 & 3 & -2 \\ 0 & 0 & 5
	\end{pmatrix},
	\begin{pmatrix}
		8 & 12 & -4 \\ -40 & -60 & 20 \\ -100 & -150 & 50
	\end{pmatrix}\notag
\end{align}

\section{Diagonalisierung}
\begin{align}
	\begin{pmatrix}
		3 & 4 & -3 \\ 2 & 7 & -4 \\ 3 & 9 & -5
	\end{pmatrix},
	\begin{pmatrix}
		3 & 0 & 0 \\ 1 & 2 & 2 \\ 1 & 0 & 4
	\end{pmatrix},
	\begin{pmatrix}
		 1 & 0 & 1 \\ 0 & 2 & 0 \\ 1 & 0 & 1
	\end{pmatrix}\notag
\end{align}

\section{Trigonalisierung}
\begin{align}
	\begin{pmatrix}
		3 & 4 & 3 \\ -1 & 0 & -1 \\ 1 & 2 & 3
	\end{pmatrix},
	\begin{pmatrix}
		3 & 0 & -2 \\ -2 & 0 & 1 \\ 2 & 1 & 0
	\end{pmatrix},
	\begin{pmatrix}
		2 & 1 & 0 \\ 0 & 1 & 0 \\ 1 & 1 & 2
	\end{pmatrix}\notag
\end{align}

\section{Jordan-Normalform}
\begin{align}
	\begin{pmatrix}
		1 & 1 & 6 & -2 \\ 0 & 1 & -3 & 2 \\ 0 & 0 & 1 & 0 \\ 0 & 0 & -2 & 2
	\end{pmatrix},
	\begin{pmatrix}
		1 & 1 & 0 & 0 \\ -1 & 3 & 0 & 0 \\ -1 & 1 & 1 & 1 \\ -1 & 1 & -1 & 3
	\end{pmatrix},
	\begin{pmatrix}
		1 & 1 & -3 \\ 0 & 1 & 1 \\ 1 & 0 & 4
	\end{pmatrix}\notag
\end{align}

\section{Gram-Schmidt-Verfahren}
\begin{align}
	B = \left(\begin{pmatrix}3\\1\end{pmatrix}, \begin{pmatrix}2\\2\end{pmatrix}\right),
	C = \left(\begin{pmatrix}1\\1\\1\end{pmatrix}, \begin{pmatrix}1\\1\\0\end{pmatrix}, \begin{pmatrix}1\\0\\0\end{pmatrix}\right),
	D = \left(\begin{pmatrix}1\\1\\0\\0\end{pmatrix}, \begin{pmatrix}1\\0\\1\\0\end{pmatrix}, \begin{pmatrix}0\\1\\0\\1\end{pmatrix}, \begin{pmatrix}0\\0\\1\\-1\end{pmatrix}\right)\notag
\end{align}

\end{document}