\documentclass[ngerman,a4paper]{article}

\usepackage{amsmath}
\usepackage{amssymb}
\usepackage{enumitem}
\usepackage[left=2.1cm,right=3.1cm,bottom=3cm]{geometry}
\usepackage[ngerman]{babel}
\usepackage[bb=boondox]{mathalfa}
\usepackage{tabularx}

\renewcommand{\arraystretch}{1.2}

\title{\textbf{spezielle Definitionen von Vektorräumen, Matrizen und Endomorphismen}}
\author{\textsc{H. Haustein}, \textsc{P. Lehmann}}

\begin{document}
	\maketitle

\section{Vektorräume}
	\begin{tabularx}{\textwidth}{l|c|X}
		\textbf{Definition} & \textbf{Erklärung} & \textbf{Bemerkungen} \\
		\hline
		unitär / euklidisch & Es gibt ein Skalarprodukt. & Man kann Abstände und Winkel messen. \begin{align}
			\cos\sphericalangle(a,b)=\frac{\langle a,b\rangle}{\Vert a\Vert\cdot\Vert b\Vert}\notag
		\end{align}
	\end{tabularx}

\section{Matrizen}
	\begin{tabularx}{\textwidth}{l|c|X}
		\textbf{Definition} & \textbf{Erklärung} & \textbf{Bemerkungen} \\
		\hline
		hermitesch / symmetrisch & $A^T = \overline{A}$ & diagonalisierbar \\
		unitär / orthogonal & $A^{-1} = (\overline{A})^T$ & \\
		normal & $A(\overline{A})^T = (\overline{A})^TA$ & Jede selbstadjungierte oder hermitesche oder unitäre Matrix ist normal. \\
		selbstadjungiert & $A = (\overline{A})^T$ & \\
	\end{tabularx}

\section{Endomorphismen}
	\begin{tabularx}{\textwidth}{l|c|X}
		\textbf{Definition} & \textbf{Erklärung} & \textbf{Bemerkungen} \\
		\hline
		selbstadjungiert & $\langle f(v),w\rangle = \langle v,f(w)\rangle$ & $\Leftrightarrow A$ symmetrisch \\
		unitär / orthogonal & $\langle v,w\rangle = \langle f(v),f(w)\rangle$ & \\
		normal & $\langle (f\circ f^{adj})(v),w\rangle = \langle (f^{adj}\circ f)(v),w\rangle$ &
	\end{tabularx}
\end{document}