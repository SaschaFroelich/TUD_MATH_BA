\section{Körper}

\begin{definition}[Körper]
	Ein \begriff{Körper} ist ein kommutativer Ring $(K,+,\cdot)$ mit Einselement 
	$1 \neq 0$, in dem jedes Element $x \neq x \in K$ invertierbar ist.
\end{definition}

\begin{definition}[Teilkörper]
	Ein \begriff{Teilkörper} eines Körpers $(K,+,\cdot)$ ist die Teilemenge $L 
	\subset K$, die mit der geeigneten Einschränkung von Addition und Multiplikation wieder ein
	Körper ist.
\end{definition}

\begin{example}[Endliche Primkörper]
	Für jede Primzahl $p$ ist $\mathbb Z /p \mathbb Z$ ein Körper. Ist $\overline{a}\neq \overline{0}$, so gilt 
	$p$ teilt nicht $a$ und somit gibt es nach \propref{1_5_7} $b,k \in \mathbb Z$ mit \\
	\begin{align}
		ab+kp &= 1 \notag \\
		\overline{(ab+kp)} &= \overline{1} = \overline{(ab)} = \overline{a} \cdot \overline{b} \notag
	\end{align}
	und somit ist $\overline{a}$ invertierbar in $\mathbb Z /p \mathbb Z$. Somit sind für $n \in \mathbb N$
	äquivalent:
	\begin{itemize}
		\item $\mathbb Z /n \mathbb Z$ ist ein Körper
		\item $\mathbb Z /n \mathbb Z$ ist nullteilerfrei
		\item $n$ ist Primzahl
	\end{itemize}
\end{example}
\begin{proof}
	\begin{itemize}
		\item 1 $\Rightarrow$ 2: \propref{1_4_13}
		\item 2 $\Rightarrow$ 3: \propref{1_4_12}
		\item 3 $\Rightarrow$ 1: gegeben
	\end{itemize}
	Insbesondere ist $\mathbb Z /p \mathbb Z$ nullteilerfrei, d.h. aus $p\vert ab$ folgt $p\vert a$ oder $p\vert b$.
\end{proof}