\section{Koordinatendarstellung linearer Abbildungen}

Seien $V,W$ endlichdimensionale $K$-Vektorräume mit den Basen $B=(x_1,...,x_n)$ und $C=(y_1,...,y_m)$.

\begin{definition}[darstellende Matrix]
	Sei $f\in \Hom_K(V,W)$. Für $j=1,...,n$ schreiben wir $f(x_j)=\sum_{
		i=1}^m a_{ij}y_i$ mit eindeutig bestimmten $a_{ij}\in K$. Die Matrix $M_C^B(f)=(a_{ij})\in \Mat_{m\times n}(K)$ 
	heißt die \begriff{darstellende Matrix} von $f$ bezüglich der Basen $B$ und $C$.
\end{definition}

\begin{definition}[Transformationsmatrix]
	Sind $B$ und $B'$ Basen von $V$, so nennt man $T_{B'}^B:=M_{B'}^B(\id_V)\in 
	\GL_(K)$ die \begriff{Transformationsmatrix} des Basiswechsels von $B$ nach $B'$.
\end{definition}