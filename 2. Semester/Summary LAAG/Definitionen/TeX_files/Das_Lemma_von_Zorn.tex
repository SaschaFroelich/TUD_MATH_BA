\section{Das Lemma von Zorn}

Sei $K$ ein Körper und $U,V,W$ seien $K$-Vektorräume. Zudem sei $X$ eine Menge.

\begin{definition}[Relation]
	Eine \begriff{Relation} ist eine Teilmenge $R\subseteq X\times X$. Man schreibt $(x,x')\in R$ als $xRx'$. $R$ heißt
	\begin{itemize}
		\item \begriff[Relation!]{reflexiv}, wenn $\forall  x\in X$: $xRx$
		\item \begriff[Relation!]{transitiv}, wenn $\forall x,y,z\in X$: $xRy$ und $yRz\Rightarrow xRz$
		\item \begriff[Relation!]{symmetrisch}, wenn $\forall x,y\in X$: $xRy\Rightarrow yRx$
		\item \begriff[Relation!]{antisymmetrisch}, wenn $\forall x,y\in X$: $xRy$ und $yRx\Rightarrow y=x$
		\item \begriff[Relation!]{total}, wenn $\forall x,y\in X$: $(x,y)\notin R\Rightarrow (y,x)\in R$
	\end{itemize}
\end{definition}

\begin{definition}[Halbordnung]
	Eine \begriff{Halbordnung} (oder \begriff{partielle Ordnung}) ist eine reflexive, transitive und antisymmetrische Relation $\le$. Eine totale Halbordnung heißt \begriff{Totalordnung} oder \begriff{lineare Ordnung}. Man schreibt $x<y$ für $x\le y\land x\neq y$.
\end{definition}

\begin{definition}[Kette]
	Sei $(X,\le)$ eine Halbordnung, $Y\subseteq X$. $Y$ heißt \begriff{Kette}, wenn $(Y,\le\vert_Y)$ total ist.
	
	$x\in Y$ heißt ein \begriff[Kette!]{minimales Element} von $Y$, wenn $\forall x'\in Y$: $x<x'$.
	
	$x\in Y$ heißt \begriff[Kette!]{untere Schranke} von $Y$, wenn $\forall y\in Y$: $y\ge x$.
	
	$x\in Y$ heißt \begriff[Kette!]{kleinstes Element} von $Y$, wenn $x$ untere Schranke von $Y$ ist.
	
	Analog: \begriff[Kette!]{maximales Element}, \begriff[Kette!]{obere Schranke}, \begriff[Kette!]{größtes Element}.
\end{definition}

\begin{theorem}[Das Lemma von Zorn]
	Sei $(X,\le)$ eine Halbordnung, die nicht leer ist. Wenn jede Kette eine obere Schranke hat, dann hat $X$ ein maximales Element.
\end{theorem}
\begin{proof}
	Das Lemma von Zorn hat axiomatischen Charakter - es ist äquivalent zum Auswahlaxiom, seine Gültigkeit ist somit abhängig von unseren grundlegenden mengentheoretischen Annahmen. Für einen Beweis des Lemmas von Zorn aus dem Auswahlaxiom siehe die Vorlesung \textit{Mengenlehre}. Wir zeigen hier zumindest die andere Richtung, nämlich dass das Auswahlaxiom aus dem Lemma von Zorn folgt.
\end{proof}

\begin{conclusion}[Auswahlaxiom]
	Zu jeder Familie $(x_i)$, nicht leer, gibt es eine \begriff{Auswahlfunktion}, das heißt eine Abbildung:
	\begin{align}
		f: I\to \bigcup_{i\in I} X_i\text{ mit } f(i)\in X_i\quad\forall i\notag
	\end{align}
\end{conclusion}
\begin{proof}
	Sei $\mathcal{F}$ die Menge der Paare $(J,f)$ bestehend aus einer Teilmenge $J\subseteq I$ und einer Abbildung $f:I\to \bigcup_{i\in I} X_i$ mit $f(i)\in X_i\quad\forall i\in J$. Definieren wir $(J,f)\le (J',f')\iff J\subseteq J'$ und $f'\vert_J = f$, so ist $\le$ eine Halbordnung auf $\mathcal{F}$. Da $(\emptyset,\emptyset)\in\mathcal{F}$ ist $\mathcal{F}$ nichtleer. Ist $\mathcal{G}\subseteq\mathcal{F}$ eine nichtleere Kette, so wird auf $J':=\bigcup_{(J,f)\in\mathcal{G}} J$ durch $f'(j)=f(j)$ falls $(J,f)\in\mathcal{G}$ und $j\in J$ eine wohldefinierte Abbildung $f':J\to \bigcup_{i\in J}X_i$ mit $f'(i)\in X_i\quad\forall i\in J'$ gegeben. Das Paar $(J',f')$ ist eine obere Schranke der Kette $\mathcal{G}$. Nach dem Lemma von Zorn besitzt $\mathcal{F}$ ein maximales Element $(J,f)$. Wir behaupten, dass $J=I$. Andernfalls nehmen wir ein $i'\in I\backslash J$ und ein $x'\in X_{i'}$ und definieren $J':= U\cup\{i'\}$ und $f':J'\to \bigcup_{i\in J'} X_i$, $j\mapsto\begin{cases}f(j)&j\in J\\ x'&j=i'\end{cases}$. Dann ist $(J',f')\in\mathcal{F}$ und $(J,f)<(J',f')$ im Widerspruch zur Maximalität von $(J,f)$.
\end{proof}


\begin{conclusion}[Basisergänzungssatz]
	\proplbl{7_1_11}
	Sei $V$ ein $K$-Vektorraum. Jede linear unabhängige Teilmenge $X_0\subseteq V$ ist in einer Basis von $V$ enthalten.
\end{conclusion}
\begin{proof}
	Sei $\mathfrak{X}=\{X\subseteq V\mid X\text{ ist linear unabhängig, } X_0\subseteq X\}$ geordnet durch Inklusion. Dann ist $X_0\in\mathfrak{X}$, also $\mathfrak{X}\neq\emptyset$. Ist $\mathcal{Y}$ eine nichtleere Kette in $\mathfrak{X}$, so ist auch $Y=\bigcup\mathcal{Y}\subseteq V$ linear unabhängig. Sind $y_1,...,y_n\in Y$ paarweise verschieden, so gibt es $Y_1,...,Y_n\in\mathcal{Y}$ mit $y_i\in Y_i$ für $i=1,...,n$. Da $\mathcal{Y}$ total geordnet ist, besitzt $\{Y_1,...,Y_n\}$ ein größtes Element, o.E. $Y_1$. Also sind $y_1,...,y_n\in Y_1$ und somit linear unabhängig. Folglich ist $Y_1\in \mathfrak{X}$ eine obere Schranke von $\mathcal{Y}$. Nach dem Lemma von Zorn besitzt $\mathfrak{X}$ ein maximales Element $X$. Das heißt, $X$ ist eine maximal linear unabhängige Teilmenge von $V$, nach LAAG1 II.3.5 also eine Basis von $V$. %TODO: Verlinkung
	\end{proof}