\section{Das Lemma von Zorn}

Sei $K$ ein Körper und $U,V,W$ seien $K$-Vektorräume. Zudem sei $X$ eine Menge.

\begin{definition}[Relation]
	Eine \begriff{Relation} ist eine Teilmenge $R\subseteq X\times X$. Man schreibt $(x,x')\in R$ als $xRx'$. $R$ heißt
	\begin{itemize}
		\item \begriff[Relation!]{reflexiv}, wenn $\forall  x\in X$: $xRx$
		\item \begriff[Relation!]{transitiv}, wenn $\forall x,y,z\in X$: $xRy$ und $yRz\Rightarrow xRz$
		\item \begriff[Relation!]{symmetrisch}, wenn $\forall x,y\in X$: $xRy\Rightarrow yRx$
		\item \begriff[Relation!]{antisymmetrisch}, wenn $\forall x,y\in X$: $xRy$ und $yRx\Rightarrow y=x$
		\item \begriff[Relation!]{total}, wenn $\forall x,y\in X$: $(x,y)\notin R\Rightarrow (y,x)\in R$
	\end{itemize}
\end{definition}

\begin{example}[Äquivalenzrelation]
	Eine \begriff{Äquivalenzrelation} ist eine reflexive, transitive und symmetrische Relation. Wir haben schon verschiedene Äquivalenzrelationen kennengelernt: Isomorphie von $K$-Vektorräumen und Ähnlichkeit von Matrizen.
\end{example}

\begin{definition}[Halbordnung]
	Eine \begriff{Halbordnung} (oder \begriff{partielle Ordnung}) ist eine reflexive, transitive und antisymmetrische Relation $\le$. Eine totale Halbordnung heißt \begriff{Totalordnung} oder \begriff{lineare Ordnung}. Man schreibt $x<y$ für $x\le y\land x\neq y$.
\end{definition}

\begin{example}
	\begin{enumerate}
		\item Die natürliche Ordnung $\le$ auf $\real$, $\ratio$, $\whole$ und $\natur$ ist eine Z
		Totalordnung.
		\item Teilbarkeit $\vert$ ist eine Halbordnung auf $\natur$, aber Teilbarkeit ist keine Halbordnung auf $\whole$, da $1\vert -1$ und $-1\vert 1$, aber $1\neq -1$!
		\item $\mathcal{P}(X)$ ist die Potenzmenge. ``$\subseteq$'' ist eine Halbordnung auf $\mathcal{P}$, aber für $\vert X\vert>1$ ist ``$\subseteq$'' keine Totalordnung.
		\item Sei $(X,\le)$ eine Halbordnung, sei $Y\subseteq X$, so ist $(Y,\subseteq\vert_Y)$ eine Halbordnung.
	\end{enumerate}
\end{example}

\begin{definition}[Kette]
	Sei $(X,\le)$ eine Halbordnung, $Y\subseteq X$. $Y$ heißt \begriff{Kette}, wenn $(Y,\le\vert_Y)$ total ist.
	
	$x\in Y$ heißt ein \begriff[Kette!]{minimales Element} von $Y$, wenn $\forall x'\in Y$: $x<x'$.
	
	$x\in Y$ heißt \begriff[Kette!]{untere Schranke} von $Y$, wenn $\forall y\in Y$: $y\ge x$.
	
	$x\in Y$ heißt \begriff[Kette!]{kleinstes Element} von $Y$, wenn $x$ untere Schranke von $Y$ ist.
	
	Analog: \begriff[Kette!]{maximales Element}, \begriff[Kette!]{obere Schranke}, \begriff[Kette!]{größtes Element}.
\end{definition}

\begin{center}
	\begin{tikzpicture}
		\node[place] (A) {1};
		\node[place] (B) [below=of A] {3};
		\node[place] (C) [left=of B] {2};
		\node[place] (D) [right=of B] {5};
		\node[place] (E) [right=of D] {7};
		\node[place] (dots) [right=of E] {...};
		
		\node[place] (F) [below=of C] {4};
		\node[place] (G) [below=of B] {6};
		\node[place] (H) [below=of E] {15};
		\node[place] (I) [right=of G] {10};
		\node[place] (dots2) [right=of H] {...};
		
		\node[place] (J) [below=of F] {8};
		\node[place] (K) [below=of J] {16};
		\node[place] (L) [below=of K] {32};
		
		\draw[->,thick] (A.west) -- (C.north);
		\draw[->,thick] (A.south) -- (B.north);
		\draw[->,thick] (A.east) -- (D.north);
		\draw[->,thick] (A.east) -- (E.north);
		\draw[->,thick] (A.east) -- (dots.north);
		
		\draw[->,thick] (C.south) -- (F.north);
		\draw[->,thick] (C.south) -- (G.north);
		\draw[->,thick] (B.south) -- (G.north);
		\draw[->,thick] (C.south) -- (I.north);
		\draw[->,thick] (D.south) -- (I.north);
		\draw[->,thick] (B.south) -- (H.north);
		\draw[->,thick] (D.south) -- (H.north);
		\draw[->,thick] (E.south) -- (dots2.north);
		\draw[->,thick] (D.south) -- (dots2.north);
		
		\draw[->,thick] (F.south) -- (J.north);
		\draw[->,thick] (J.south) -- (K.north);
		\draw[->,thick] (K.south) -- (L.north);
	\end{tikzpicture}
	$Y=\{2^n\mid n\in\natur\}$ ist eine Kette
\end{center}

\begin{remark}
	\begin{itemize}
		\item Hat $Y$ ein kleinstes Element, so ist dies eindeutig bestimmt. Ein kleinstes Element ist minimal.
		\item Jede endliche Halbordnung hat minimale Elemente. Jede endliche Totalordnung hat ein kleinstes Element. Analog für maximale Elemente und größtes Element.
	\end{itemize}
\end{remark}

\begin{example}
	$(\natur,\le)$ hat als kleinstes Element die 1, aber kein größtes Element oder maximale Elemente.
\end{example}

\begin{example}
	$V=\real^3$, $\mathfrak{X}$ die Menge der Untervektorräume des $\real^3$. $(\mathfrak{X},\le)$ ist eine Halbordnung auf $Y\subseteq X$ mit $Y=\{U\in\mathfrak{X}\mid \dim_\real(U)\le 2\}$. 
	\begin{itemize}
		\item $Y$ hat ein kleinstes Element: $\{0\}$.
		\item Es gibt unendlich viele maximale Elemente in $Y$, nämlich die Untervektorräume von $V$, die die Dimension 2 haben. Es gibt also kein größtes Element.
		\item $V$ ist die obere Schranke von $Y$.
	\end{itemize}
\end{example}

\begin{theorem}[Das Lemma von Zorn]
	Sei $(X,\le)$ eine Halbordnung, die nicht leer ist. Wenn jede Kette eine obere Schranke hat, dann hat $X$ ein maximales Element.
\end{theorem}
\begin{proof}
	Das Lemma von Zorn hat axiomatischen Charakter - es ist äquivalent zum Auswahlaxiom, seine Gültigkeit ist somit abhängig von unseren grundlegenden mengentheoretischen Annahmen. Für einen Beweis des Lemmas von Zorn aus dem Auswahlaxiom siehe die Vorlesung \textit{Mengenlehre}. Wir zeigen hier zumindest die andere Richtung, nämlich dass das Auswahlaxiom aus dem Lemma von Zorn folgt.
\end{proof}

\begin{conclusion}[Auswahlaxiom]
	Zu jeder Familie $(x_i)$, nicht leer, gibt es eine \begriff{Auswahlfunktion}, das heißt eine Abbildung:
	\begin{align}
		f: I\to \bigcup_{i\in I} X_i\text{ mit } f(i)\in X_i\quad\forall i\notag
	\end{align}
\end{conclusion}
\begin{proof}
	Sei $\mathcal{F}$ die Menge der Paare $(J,f)$ bestehend aus einer Teilmenge $J\subseteq I$ und einer Abbildung $f:I\to \bigcup_{i\in I} X_i$ mit $f(i)\in X_i\quad\forall i\in J$. Definieren wir $(J,f)\le (J',f')\iff J\subseteq J'$ und $f'\vert_J = f$, so ist $\le$ eine Halbordnung auf $\mathcal{F}$. Da $(\emptyset,\emptyset)\in\mathcal{F}$ ist $\mathcal{F}$ nichtleer. Ist $\mathcal{G}\subseteq\mathcal{F}$ eine nichtleere Kette, so wird auf $J':=\bigcup_{(J,f)\in\mathcal{G}} J$ durch $f'(j)=f(j)$ falls $(J,f)\in\mathcal{G}$ und $j\in J$ eine wohldefinierte Abbildung $f':J\to \bigcup_{i\in J}X_i$ mit $f'(i)\in X_i\quad\forall i\in J'$ gegeben. Das Paar $(J',f')$ ist eine obere Schranke der Kette $\mathcal{G}$. Nach dem Lemma von Zorn besitzt $\mathcal{F}$ ein maximales Element $(J,f)$. Wir behaupten, dass $J=I$. Andernfalls nehmen wir ein $i'\in I\backslash J$ und ein $x'\in X_{i'}$ und definieren $J':= U\cup\{i'\}$ und $f':J'\to \bigcup_{i\in J'} X_i$, $j\mapsto\begin{cases}f(j)&j\in J\\ x'&j=i'\end{cases}$. Dann ist $(J',f')\in\mathcal{F}$ und $(J,f)<(J',f')$ im Widerspruch zur Maximalität von $(J,f)$.
\end{proof}


\begin{conclusion}[Basisergänzungssatz]
	\proplbl{7_1_11}
	Sei $V$ ein $K$-Vektorraum. Jede linear unabhängige Teilmenge $X_0\subseteq V$ ist in einer Basis von $V$ enthalten.
\end{conclusion}
\begin{proof}
	Sei $\mathfrak{X}=\{X\subseteq V\mid X\text{ ist linear unabhängig, } X_0\subseteq X\}$ geordnet durch Inklusion. Dann ist $X_0\in\mathfrak{X}$, also $\mathfrak{X}\neq\emptyset$. Ist $\mathcal{Y}$ eine nichtleere Kette in $\mathfrak{X}$, so ist auch $Y=\bigcup\mathcal{Y}\subseteq V$ linear unabhängig. Sind $y_1,...,y_n\in Y$ paarweise verschieden, so gibt es $Y_1,...,Y_n\in\mathcal{Y}$ mit $y_i\in Y_i$ für $i=1,...,n$. Da $\mathcal{Y}$ total geordnet ist, besitzt $\{Y_1,...,Y_n\}$ ein größtes Element, o.E. $Y_1$. Also sind $y_1,...,y_n\in Y_1$ und somit linear unabhängig. Folglich ist $Y_1\in \mathfrak{X}$ eine obere Schranke von $\mathcal{Y}$. Nach dem Lemma von Zorn besitzt $\mathfrak{X}$ ein maximales Element $X$. Das heißt, $X$ ist eine maximal linear unabhängige Teilmenge von $V$, nach LAAG1 II.3.5 also eine Basis von $V$. %TODO: Verlinkung
	\end{proof}