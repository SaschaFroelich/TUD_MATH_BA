\section{Der Dualraum}

Sei $V$ ein $K$-Vektorraum.

\begin{definition}[Dualraum]
	Der \begriff{Dualraum} zu $V$ ist der $K$-Vektorraum
	\begin{align}
		V^*=\Hom_K(V,K)=\{\phi:V\to K\text{ linear}\}\notag
	\end{align}
	Die Elemente von $V^*$ heißen \begriff{Linearformen} auf $V$.
\end{definition}

\begin{definition}[duale Basis]
	Ist $B=(x_i)_{i\in I}$ eine endliche Basis von $V$, so nennt man $B^*=(x_i^*)_{i\in I}$ die zu $B$ \begriff{duale Basis}.
\end{definition}

\begin{definition}[Bidualraum]
	Der \begriff{Bidualraum} zu $V$ ist der $K$-Vektorraum
	\begin{align}
		V^{**}=(V^*)^*=\Hom_K(V^*,K)\notag
	\end{align}
\end{definition}

\begin{definition}[Annulator]
	Für eine Teilmenge $U\subseteq V$ bezeichne
	\begin{align}
		U^0 =\{\phi\in V^*\mid \phi(x)=0\quad\forall x\in U\}\notag
	\end{align}
	den \begriff{Annulator} von $U$.
\end{definition}