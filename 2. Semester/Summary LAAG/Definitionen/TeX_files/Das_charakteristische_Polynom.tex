\section{Das charakteristische Polynom}

\begin{definition}[charakteristisches Polynom]
	Das \begriff{charakteristische Polynom} einer Matrix $A\in\Mat_n(K)$ ist die Determinante der Matrix $t\cdot \mathbbm{1}_n-A\in\Mat_n(K[t])$. 
	\begin{align}
		\chi_A(t)&=\det(t\cdot \mathbbm{1}_n-A)\in K[t] \notag
	\end{align}
	Das charakteristische Polynom eines Endomorphismus $f\in\End_K(V)$ ist $\chi_f(t)=\chi_{M_B(f)}(t)$, wobei $B$ eine Basis von $V$ ist.
\end{definition}

\begin{definition}[normiertes Polynom]
	Ein Polynom $0\neq P\in K[t]$ mit Leitkoeffizient 1 heißt \begriff{normiert}.
\end{definition}