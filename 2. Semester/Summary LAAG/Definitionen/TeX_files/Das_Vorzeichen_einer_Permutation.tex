\section{Das Vorzeichen einer Permutation}

In diesem Kapitel sei $K$ ein Körper und $R$ ein kommutativer Ring mit Einselement.

\begin{definition}[Fehlstand, Vorzeichen]
	Sei $\sigma\in S_n$.
	\begin{itemize}
		\item Ein \begriff{Fehlstand} von $\sigma$ ist ein Paar $(i,j)$ mit $1\le i<j\le n$ und $\sigma(i)>\sigma(j)$.
		\item Das \begriff{Vorzeichen} (oder \begriff{Signum}) von $\sigma$ ist $\sgn(\sigma)=(-1)^{f(\sigma)}\in \{-1,1\}$, wobei $f(\sigma)$ die 
		Anzahl der Fehlstände von $\sigma$ ist.
		\item Man nennt $\sigma$ \begriff[Vorzeichen!]{gerade}, wenn $\sgn(\sigma)=1$, sonst \begriff[Vorzeichen!]{ungerade}.
	\end{itemize}
\end{definition}