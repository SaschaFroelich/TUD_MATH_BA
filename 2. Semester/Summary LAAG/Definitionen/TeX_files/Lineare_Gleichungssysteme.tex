\section{Lineare Gleichungssysteme}

Sei $A\in \Mat_{m\times n}(K)$ und $b\in K^m$.

\begin{definition}[Lineares Gleichungssystem]
	Unter einem \begriff{Linearen Gleichungssystem} verstehen wir eine Gleichung der Form $Ax=b$. 
	Diese heißt \begriff{homogen}, wenn $b=0$, sonst \begriff{inhomogen} und $L(A,b)=\{x\in K^n\mid Ax=b\}$ ist sein \begriff{Lösungsraum}.
\end{definition}

\begin{definition}[Zeilenstufenform]
	Die Matrix $A=(a_{ij})$ hat \begriff{Zeilenstufenform}, wenn es ganze Zahlen $0\le r \le m$ und $1\le 
	k_1<...<k_r\le n$ gibt mit:
	\begin{itemize}
		\item für $1\le i \le r$ und $1\le j < k_i$ ist $a_{ij}=0$
		\item für $1\le i \le r$ ist $a_{ik_{i}}\neq 0$ (sogenannte \begriff{Pivotelemente})
		\item für $r<i\le m$ und $1\le j\le n$ ist $a_{ij}=0$
	\end{itemize}
	\begin{align}
		\begin{pmatrix}
		0 & \dots & 0 & a_{1k_{1}} & * & \dots & \dots & *\\
		0 & \dots & \dots & 0 & a_{2k_{2}} & * & \dots & *\\
		\vdots & \vdots & \vdots & \vdots & \vdots & \vdots & \vdots & \vdots\\
		0 & \dots & \dots & \dots & \dots & \dots & \dots & a_{rk_{r}}\\
		0 & \dots & \dots & \dots & \dots & \dots & \dots & 0\\
		\vdots & \; & \; & \; & \; & \; & \; & \vdots\\
		0 & \dots & \dots & \dots & \dots & \dots & \dots & 0\\
		\end{pmatrix}\notag
	\end{align}
\end{definition}

\begin{definition}[Elementarmatrizen]
	Für $i,j\in \{1,...,m\}$, $\lambda \in K^{\times}$ und $\mu\in K$ definieren wir 
	$m\times m$-Matrizen, die sogenannten \begriff{Elementarmatrizen}:
	\begin{itemize}
		\item $S_i(\lambda):=\mathbbm{1}_m + (\lambda-1)E_{ii}$
		\item $Q_{ij}(\mu):= \mathbbm{1}_m + \mu E_{ij}$
		\item $P_{ij}:= \mathbbm{1}_m + E_{ij} + E_{ji} - E_{ii} - E_{jj}$
	\end{itemize}
\end{definition}

\begin{theorem}[Eliminierungsverfahren nach \person{Gauß}]
	\proplbl{3_9_11}
	Zu jeder Matrix $A\in \Mat_{m\times n}(K)$ gibt es $l\in \mathbb N_0$ und 
	Elementarmatrizen $E_1,...,E_l$ vom Typ II und III für die $E_l\cdot ... \cdot E_1\cdot A$ in Zeilenstufenform ist. 
\end{theorem}
\begin{proof}
	Seien $a_1,...,a_n$ die Spalten von $A$. \\
	Ist $A=0$ so ist nichts zu tun. \\
	Sei nun $A\neq 0$ und sei $k_1$ minimal mit $a_{k_1}\neq 0$. Es gibt also ein $i$ mit $a_{ik_1}\neq 0$. Durch Vertauschen der ersten 
	und der $i$-ten Zeile erreichen wir, dass $a_{1k_1}=0$, d.h. wir multiplizieren $A$ mit $E_1=P_{1i}$. Nun addieren wir für $i=2,..,m$ 
	ein geeignetes Vielfaches der ersten Zeile zur $i$-ten Zeile, um $a_{ik_1}=0$, d.h. wir multiplizieren $A$ mit $E_i=Q_{i1}(\mu_i)$ für 
	$\mu_i=\frac{a_{ik_1}}{a_{1k_1}}$. Nach diesen Umformungen haben wir eine Matrix der Form:
	\begin{align}\begin{pmatrix}
		0 & \dots & 0 & a_{1k_1} & * & \dots & *\\
		0 & \dots & \dots & 0 & \textcolor{red}{*} & \textcolor{red}{\dots} & \textcolor{red}{*}\\
		\vdots & \vdots & \vdots & \vdots & \textcolor{red}{\vdots} & \textcolor{red}{\vdots} & \textcolor{red}{\vdots}\\
		0 & \dots & \dots & 0 & \textcolor{red}{*} & \textcolor{red}{\dots} & \textcolor{red}{*}\\
		\end{pmatrix}\notag\end{align}
	und können nun mit dem \textcolor{red}{Rest der Matrix $A=:A'$} von vorne beginnen. Die nun folgenden Zeilenumformungen werden die erste Zeile und die ersten $k_1$ Spalten nicht mehr ändern, und weil $A'$ weniger Zeilen und Spalten als $A$ hat, bricht das Verfahren nach endlich vielen Schritten ab.
\end{proof}