\section{Euklidische und unitäre Vektorräume}

\begin{definition}[quadratische Form]
	Sei $s$ eine hermitesche Sesquilinearform auf $V$. Die \begriff{quadratische Form} zu $s$ ist die Abbildung
	\begin{align}
		q_s:\begin{cases}
		V\to \real \\ x\mapsto s(x,x)
		\end{cases}\notag
	\end{align}
\end{definition}

\begin{definition}[(semi)definit, euklidischer VR, unitärer VR]
	Sei $s$ eine hermitesche Sesquilinearform auf $V$. Ist $s(x,x)\ge 0$ für alle $x\in V$, so heißt $s$ \emph{positiv} \begriff{semidefinit}. Ist $s(x,x)>0$ für alle $0\neq x\in V$, so heißt $s$ \emph{positiv} \begriff{definit} (oder ein \emph{Skalarprodukt}).
	
	Eine hermitesche Matrix $A\in\Mat_n(K)$ heißt \emph{positiv (semi)definit}, wenn $s_A$ dies ist.
	
	Einen endlichdimensionalen $K$-VR zusammen mit positiv definiten hermiteschen Sesquilinearformen nennt man einen \begriff{euklidischen} bzw. \begriff{unitären} VR (oder auch \emph{Prähilbertraum}). Wenn nicht anderes angegeben, notieren wir die Sesquilinearform mit $\skalar{\cdot}{\cdot}$.
\end{definition}

\begin{definition}
	Ist $V$ ein unitärer VR, so definiert man die Norm von $x\in V$ als
	\begin{align}
		\Vert x\Vert = \sqrt{\skalar{x}{x}}\in\real_{\ge 0}\notag
	\end{align}
\end{definition}