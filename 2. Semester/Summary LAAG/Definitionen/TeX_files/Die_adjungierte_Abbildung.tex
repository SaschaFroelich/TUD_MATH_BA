\section{Die adjungierte Abbildung}

Sei $K=\real$ oder $K=\comp$ und $V$ ein endlichdimensionaler unitärer $K$-Vektorraum.

\begin{definition}[weitere Skalarmultiplikation]
	Wir definieren auf $V$ eine Skalarmultiplikation
	\begin{align}
		\lambda\ast x=\overline{\lambda}\cdot x\notag
	\end{align}
	und schreiben $\overline{V}=(V,+,\ast)$.
\end{definition}

\begin{lemma}
	$\overline{V}$ ist ein $K$-Vektorraum und $\End_K(V)=\End_K(\overline{V})$.
\end{lemma}
\begin{proof}
	Mit LAAG1 VI.1.7 nachprüfen, zum Beispiel:
	\begin{itemize}
		\item $\lambda\ast (x+y)=\overline{\lambda}\cdot (x+y)=\overline{\lambda} x+\overline{\lambda} y=\lambda\ast x+\lambda\ast y$
		\item $\lambda\ast(\mu\ast x)=\overline{\lambda}(\overline{\mu}\cdot x)=\overline{\lambda\mu}x=(\lambda\mu)\ast x$
	\end{itemize}
	Weiterhin sei: $f\in\End_K(V)$, $x\in V$, $\lambda\in K$ \\
	$\Rightarrow f(\lambda\ast x)=f(\overline{\lambda}x)=\lambda\ast f(x)$ \\
	$\Rightarrow f\in \End_K(\overline{V})$. \\
	Umgekehrt sei $g\in\End_K(\overline{V})$, $x\in V$, $\lambda\in K$ \\
	$\Rightarrow g(\lambda\cdot x)=g(\overline{\lambda}\ast x)=\lambda\cdot g(x)$ \\
	$\Rightarrow g\in \End_K(V)$. \\
\end{proof}

\begin{lemma}
	Für $y\in V$ ist
	\begin{align}
		\Phi_y:
		\begin{cases}
			V\to K \\ x\mapsto\skalar{x}{y}
		\end{cases}\notag
	\end{align}
	eine Linearform auf $V$.
	
	Die Abbildung $y\mapsto\Phi_y$ liefert einen Isomorphismus $\Phi:\overline{V}\to V^*$.
\end{lemma}
\begin{proof}
	\begin{itemize}
		\item $\Phi_y\in V^*$: Linearität in ersten Argument.
		\item $\Phi\in \Hom_K(\overline{V},V^*)$: Für $y,y'\in V$, $\lambda\in K$, $x\in V$ ist
		\begin{itemize}
			\item $\Phi_{y+y'}(x)=\skalar{x}{y+y'}=\skalar{x}{y}+\skalar{x}{y'}=\Phi_y(x)+\Phi_{y'}(x)$
			\item $\Phi_{\lambda\ast y}(x)=\skalar{x}{\lambda\ast x}=\skalar{x}{\overline{\lambda}y}=\lambda \skalar{x}{y}=\lambda\Phi_y(x)$
		\end{itemize}
		\item $\Phi$ injektiv: Skalarprodukt ist nicht ausgeartet.
		\item Da $\dim_K(\overline{V})=\dim_K(V)=\dim_K(V^*)$ ist $\Phi$ somit ein Isomorphismus.
	\end{itemize}
\end{proof}

\begin{proposition}
	Zu $f\in\End_K(V)$ gibt es ein eindeutig bestimmtes $f^{adj}\in\End_K(V)$ mit 
	\begin{align}
		\skalar{f(x)}{y}=\skalar{x}{f^{adj}(y)}\quad\forall x,y\in V\notag
	\end{align}
\end{proposition}
\begin{proof}
	Existenz und Eindeutigkeit sind zu zeigen.
	\begin{itemize}
		\item Existenz:
		\begin{center}
		\begin{tikzpicture}
		\matrix (m) [matrix of math nodes,row sep=3em,column sep=4em,minimum width=2em]
		{\overline{V} & \overline{V} \\ V^* & V^* \\};
		\path[-stealth]
		(m-1-1) edge node [left] {$\Phi$} (m-2-1)
		edge node [above] {$f$} (m-1-2)
		(m-1-2) edge node [below] {$f^{adj}$} (m-1-1)
		(m-2-2) edge node [below] {$f^*$} (m-2-1)
		(m-1-2) edge node [right] {$\Phi$} (m-2-2);
		\end{tikzpicture}
		\end{center}
		Für $f^{adj}=\Phi{-1}\circ f^*\circ \Phi\in \End_K(\overline{V})=\End_K(V)$ ist 
		\begin{align}
			\Phi_y\circ = (f^*\circ \Phi)(y)=(\Phi\circ f^{adj.})(y)=\Phi_{f^{adj}(y)}\notag
		\end{align}
		also
		\begin{align}
			\skalar{f(x)}{y}=(\Phi_y\circ f)(x)=\Phi_{f^{adj}(y)}(x)=\skalar{x}{f^{adj}(y)}\quad\forall x,y\in V\notag
		\end{align}
		\item Eindeutigkeit: Erfüllen $f_1,f_2$ für Gleichung
		\begin{align}
			\skalar{f(x)}{y}=\skalar{x}{f^{adj}(y)}\quad\forall x,y\in V\notag
		\end{align}
		so ist
		\begin{align}
			0=\skalar{x}{f_1(y)}-\skalar{x}{f_2(y)}=\skalar{x}{f_1(y),f_2(y)}\quad\forall x,y\in V\notag
		\end{align}
		da $\skalar{\cdot}{\cdot}$ nicht ausgeartet ist, folgt daraus, dass $f_1=f_2$.
	\end{itemize}
\end{proof}

\begin{definition}[adjungierter Endomorphismus]
	Die Abbildung $f^{adj}$ heißt der zu $f$ \begriff[Endomorphismus!]{adjungierte Endomorphismus}.
\end{definition}

\begin{example}
	\begin{itemize}
		\item Ist $f$ selbstadjungiert, so ist $f^{adj}=f$.
		\item Ist $f$ unitär, so ist $f\in\Aut_K(V)$ und 
		\begin{align}
		\skalar{f(x)}{y}=\skalar{x}{f^{-1}(y)}\quad\forall x,y\in V\notag
		\end{align}
		also $f^{adj}=f^{-1}$.
	\end{itemize}
\end{example}

\begin{lemma}
	\proplbl{7_4_7}
	Ist $B$ eine Orthonormalbasis vin $V$, so ist
	\begin{align}
		M_B(f^{adj})=M_B(f^*)\notag
	\end{align}
\end{lemma}
\begin{proof}
	Ist $A=M_B(f)$ und $B=M_B(f^{adj})$, $v=\Phi_B(x)$, $w=\Phi_B(y)$, so ist
	\begin{align}
		(Ax)^t\overline{y}=\skalar{f(v)}{w}&=\skalar{v}{f^{adj}(w)}\notag \\
		x^tA^t\overline{y} &= x^t\overline{B}\overline{y} \notag \\
		\Rightarrow B&= \overline{A^t}=A^*\notag
	\end{align}
\end{proof}

\begin{lemma}
	\proplbl{7_4_8}
	Für $f,g\in\End_K(V)$ und $\lambda,\mu\in K$ ist
	\begin{align}
		(\lambda f+\mu g)^{adj} &= \overline{\lambda}f^{adj}+\overline{\mu}g^{adj}\notag \\
		(f^{adj})^{adj} &= f\notag
	\end{align}
\end{lemma}
\begin{proof}
	Für $x,y\in V$ ist
	\begin{align}
		\skalar{(\lambda f+\mu g)(x)}{y}&=\lambda\skalar{f(x)}{y}+\mu\skalar{g(x)}{y}\notag \\
		&= \lambda\skalar{x}{f^{adj}(y)}+\mu\skalar{x}{g^{adj}} \notag \\
		&= \skalar{x}{(\overline{\lambda}f^{adj}+\overline{\mu}g^{adj})(y)}\notag
	\end{align}
	und
	\begin{align}
		\skalar{f^{adj}(x)}{y}=\overline{\skalar{y}{f^{adj}(y)}}=\overline{\skalar{f(y)}{x}}=\skalar{x}{f(y)}\notag
	\end{align}
\end{proof}