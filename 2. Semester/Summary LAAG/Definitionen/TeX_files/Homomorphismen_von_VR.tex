\section{Homomorphismen von Vektorräumen}

Seien $U,V,W$ drei $K$-Vektorraum. \\

\begin{definition}[linear]
	Eine Abbildung $f: V \to W$ heißt $K$-\begriff{linear}er Homomorphismus von $K$-Vektorraum, wenn für 
	alle $x,y\in V$ und $\lambda\in K$ gilt:
	\begin{itemize}
		\item (L1): $f(x+y)=f(x)+f(y)$
		\item (L2): $f(\lambda x)=\lambda \cdot f(x)$
	\end{itemize}
	Die Menge der $K$-linearen Abbildungen $f: V\to W$ wird mit $\Hom_K(V,W)$ bezeichnet. Die Elemente von $\End_K(V)
	:= \Hom_K(V,V)$ nennt man die Endomorphismen von $V$. Ein $f\in \Hom_K(V,W)$ ist ein Mono-, Epi- bzw. Isomorphismus, 
	falls $f$ injektiv, surjektiv bzw. bijektiv ist. Einen Endomorphismus der auch ein Isomorphismus ist, nennt man 
	\begriff{Automorphismus} von $V$ und bezeichnet die Menge der Automorphismen von $V$ mit $\Aut_K(V)$. Der Kern einer linearen 
	Abbildung $f: V\to W$ ist $\Ker(f):= f^{-1}(\{0\})$.
\end{definition}