\section{Tensorprodukte}

\begin{definition}[billineare Abbildung]
	Eine Abbildung $\xi:V\times W\to U$ ist \begriff[Abbildung!]{bilinear}, wenn für jedes $v\in V$ die Abbildung 
	\begin{align}
		\begin{cases}
		W\to U \\ w\mapsto \xi(v,w)
		\end{cases}\notag
	\end{align}
	und für jedes $w\in W$ die Abbildung
	\begin{align}
	\begin{cases}
	V\to U \\ v\mapsto \xi(v,w)
	\end{cases}\notag
	\end{align}
	linear sind.
	
	Wir definieren
	\begin{align}
		\Bil_K(V,W,U)=\{\xi\in\Abb(V\times W,U)\mid \xi\text{ bilinear}\}\notag
	\end{align}
\end{definition}

\begin{definition}[Tensorprodukt]
	Ein \begriff{Tensorprodukt} von $V$ und $W$ ist ein Paar $(T,\tau)$ bestehend aus einem $K$-Vektorraum $T$ und einer bilinearen Abbildung $\tau\in\Bil_K(V,W,T)$ welche die folgende \begriff{universelle Eigenschaft} erfüllt: \\
	\textit{Ist $U$ ein weiterer $K$-Vektorraum und $\xi\in\Bil_K(V,W,U)$ so gibt es genau ein $\xi_\otimes\in\Hom_K(T,U)$ mit $\xi=\xi_\otimes\circ\tau$.}
	\begin{center}
		\begin{tikzpicture}
		\matrix (m) [matrix of math nodes,row sep=3em,column sep=4em,minimum width=2em]
		{V\times W & T \\ \; & U \\};
		\path[-stealth]
		(m-1-1) edge node [below] {$\xi$} (m-2-2)
		edge node [above] {$\tau$} (m-1-2)
		(m-1-2) edge [dashed] node [right] {$\xi_\otimes$} (m-2-2);
		\end{tikzpicture}
	\end{center}
\end{definition}

\begin{definition}[Vektorraum mit Basis $X$]
	\proplbl{7_6_8}
	Sei $X$ eine Menge. Der $K$-\begriff{Vektorraum mit Basis $X$} ist der Untervektorraum $V=\Span_K((\delta_x)_{x\in X})$ des $K$-Vektorraum $\Abb(X,K)$ mit $\delta_x(y)=\delta_{x,y}=\begin{cases}1&x=y\\0&x\neq y\end{cases}$
\end{definition}