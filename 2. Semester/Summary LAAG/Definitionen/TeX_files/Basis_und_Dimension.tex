\section{Basis und Dimension}

\begin{definition}[Basis]
	Eine Familie $(x_i)$ von Elementen von $V$ ist eine \begriff{Basis} von $V$, wenn gilt:
	\begin{itemize}
		\item (B1): Die Familie ist linear unabhängig.
		\item (B2): Die Familie erzeugt $V$, also $\Span_K(x_i) = V$.
	\end{itemize}
\end{definition}

\begin{theorem}[Basisauswahlsatz]
	\proplbl{2_3_6}
	Jedes endliche Erzeugendensystem von $V$ besitzt eine Basis  als 
	Teilfamilie: Ist $(x_i)$ ein endliches Erzeugendensystem von $V$, so gibt es eine Teilmenge $J\subseteq I$, 
	für die $(x_i)_{i\in J}$ eine Basis von $V$ ist. 
\end{theorem}
\begin{proof}
	Sei $(x_i)$ ein endliches Erzeugendensystem von $V$. Definiere $\mathcal J:=\{J \subseteq I \mid (x_i)_{i\in J}\; 
	\text{J ist Erzeugendensystem von }V\}$. Da $I$ endlich ist, ist auch $\mathcal J$ endlich. Da $(x_i)$ 
	Erzeugendensystem ist, ist $I\in J$, insbesondere $\mathcal J\neq\emptyset$. Es gibt deshalb ein bezüglich 
	Inklusion minimales $J_0\in \mathcal J$, d.h. $J_1 \in \mathcal J$ so gilt nicht $J_1 \subsetneq J_0$. Deshalb 
	ist $(x_i)_{i\in J_0}$ eine Basis von $V$ (\propref{2_3_5}).
\end{proof}

\begin{lemma}[Austauschlemma]
	\proplbl{2_3_9}
	Sei $B=(x_1,...,x_n)$  eine Basis von $V$. Sind $\lambda_1,...,\lambda_n \in K$ und 
	$y=\sum_{i=1}^n \lambda_i\cdot x_i$, so ist für jedes $j\in \{1,2,...,n\}$ mit $\lambda_j\neq 0$ auch 
	$B'=(x_1,...,x_{j-1},y,x_{j+1},...,x_n)$ eine Basis von $V$.
\end{lemma}
\begin{proof}
	oBdA. sei $j=1$, also $B'=(y,x_2,...,x_n)$. Wegen $\lambda_1\neq 0$ ist $x_1=\lambda_1^{-1}\cdot y - \sum
	_{i=2}^n \lambda_i\cdot x_i \in \Span_K(y,x_2,...,x_n)$ und somit ist $B'$ ein Erzeugendensystem. Sind 
	$\mu_1,...,\mu_n \in K$ mit $\mu_1\cdot y - \sum_{i=2}^n \mu_i\cdot x_i=0$, so folgt $0=\mu_1(\sum
	_{i=1}^n \lambda_i\cdot x_i + \sum_{i=2}^n \mu_i\cdot x_i)=\mu_1\cdot \lambda_1\cdot x_1 + \sum
	_{i=2}^n (\mu_1\cdot \lambda_i + \mu_i)x_i$ und aus der linearen Unabhängigkeit von $B$ somit $\mu_1\cdot 
	\lambda_1=0$, $\mu_1\cdot \lambda_2 + \mu_2 =0$, ..., $\mu_1\cdot\lambda_n + \mu_n=0$. Wegen $\lambda_1\neq 0$ folgt 
	$\mu_1=0$ und daraus $\mu_i=0$. Folglich ist $B'$ linear unabhängig.
\end{proof}

\begin{theorem}[\person{Steinitz}'scher Austauschsatz]
	\proplbl{2_3_10}
	Sei $B=(x_1,...,x_n)$ eine Basis von $V$ und $\mathcal F=(y_1,...
	,y_r)$ eine linear unabhängige Familie in $V$. Dann ist $r\le n$ und es gibt $i_1,...,i_{n-r} \in \{1,...,n\}$, für 
	die $B'=(y_1,...,y_r,x_{i_1},...,x_{i_{n-r}})$ eine Basis von $V$ ist. 
\end{theorem}
\begin{proof}
	Induktion nach $r$\\
	Für $r=0$ ist nichts zu zeigen. \\
	Sei nun $r\ge 1$ und gelte die Aussage für $(y_1,...,y_{r-1})$. Insbesondere ist $r-1\le n$ und es gibt $i_1,..,
	i_{n-(r-1)} \in \{1,...,n\}$ für die $B'=(y_1,...,y_r,x_{i_1},...,x_{i_{n-(r-1)}})$ eine Basis von $V$ ist. Da $y_r
	\in V=\Span_K(B')$ ist $y_r=\sum_{i=1}^{r-1} \lambda_i\cdot y_1 + \sum_{j=0}^{n-(r-1)} \mu_j\cdot 
	x_{i_j}$. Da $(y_1,...,y_r)$ linear unabhängig, ist $y_r \notin \Span_K(y_1,...,y_{r-1})$. Folglich gibt es $j_0 \in 
	\{1,...,n-(r-1)\}$ mit $\mu_{j_0}\neq 0$. Insbesondere ist $n-(r-1)\ge 1$, also $r\le n$. oBdA. $j_0=1$, dann 
	ergibt sich mit dem Austauschlemma (\propref{2_3_9}), dass auch $(y_1,...,y_{r-1},y_r,x_{i_2},...,x_{i_{n-(r-1)}})$ eine Basis von 
	$V$ ist.
\end{proof}

\begin{conclusion}[Basisergänzungssatz]
	\proplbl{2_3_12}
	Ist $V$ endlich erzeugt, so lässt sich jede linear unabhängige Familie zu einer Basis ergänzen: 
	Ist $(x_1,...,x_n)$ linear unabhängig, so gibt es $m\ge n$ und $x_{n+1},x_{n+2},...,x_m$ für die $(x_1,...,x_n,
	x_{n+1},...,x_m)$ eine Basis von $V$ ist.
\end{conclusion}
\begin{proof}
	Nach dem Basisauswahlsatz (\propref{2_3_6} und \propref{2_3_7}) besitzt $V$ eine endliche Basis, die Behauptung folgt somit aus dem \person{Steinitz}'schen Austauschsatz (\propref{2_3_10}).
\end{proof}