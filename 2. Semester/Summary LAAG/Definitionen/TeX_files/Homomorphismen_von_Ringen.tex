\section{Homomorphismen von Ringen}

Seien $R,S$ und $T$ Ringe.

\begin{definition}[Ringhomomorphismus]
	Eine Abbildung $f:R\to S$ ist ein \begriff{Ringhomomorphismus}, wenn für $x,y\in R$ 
	gilt:
	\begin{itemize}
		\item (RH1:) $f(x+y)=f(x)+f(y)$
		\item (RH2:) $f(xy)=f(x)\cdot f(y)$
	\end{itemize}
	Die Menge der Ringhomomorphismen $f:R\to R$ wird mit $\Hom(R,S)$ bezeichnet. Ein Homomorphismus $f:R\to S$ ist ein 
	Mono-, Epi- oder Isomorphismus, wenn $f$ injektiv, surjektiv oder bijektiv ist. Gibt es einen Isomorphismus 
	$f:R\to S$, so nennt man $R$ und $S$ isomorph und schreibt $R\cong S$. Die Elemente von $\End(R):= \Hom(R,R)$ nennt 
	man \begriff{Endomorphismen}. Der Kern eines Ringhomomorphismus $f:R\to S$ ist $\Ker(f):= f^{-1}(\{0\})$.
\end{definition}

\begin{definition}[Ideal]
	Ist $I$ eine Untergruppe von $(R,+)$ und $xa,ax\in I$ mit $x\in R$ und $a\in I$, so nennt 
	man $I$ ein \begriff{Ideal} von $R$ und schreibt $I\unlhd R$.
\end{definition}