\documentclass[ngerman,a4paper]{article}

\usepackage{amsmath}
\usepackage{amssymb}
\usepackage{enumitem}
\usepackage[left=2.1cm,right=3.1cm,bottom=3cm]{geometry}
\usepackage[ngerman]{babel}
\usepackage[bb=boondox]{mathalfa}

\DeclareMathOperator{\Eig}{Eig}
\DeclareMathOperator{\Ker}{Ker}
\DeclareMathOperator{\diag}{diag}
\DeclareMathOperator{\id}{id}

\title{\textbf{Wichtige Methoden der Linearen Algebra}}
\author{\textsc{H. Haustein}, \textsc{P. Lehmann}}

\begin{document}
\maketitle

\section{Basiswechsel}
\begin{enumerate}[label=\textbf{\arabic*.}]
	\item Alle Vektoren $b_j$ der alten Basis als Linearkombination mit Vektoren $b'$ der neuen Basis darstellen
	\begin{align}
		b_j = \sum_{j=0}^n a_{ij}b'_i\notag
	\end{align}
	\item Matrix zusammenbauen
	\begin{align}
		T^B_{B'} = M^B_{B'}(\id) = \left(\begin{array}{c|c|c|c|c}
			a_{11} & & a_{1j} & & a_{1n} \\
			\vdots & \dots & \vdots & \dots & \vdots \\
			a_{n1} & & a_{nj} & & a_{nn}
		\end{array}\right)\notag
	\end{align}
\end{enumerate}

\section{Lineare Gleichungssysteme}
\subsection{Eliminierungsverfahren nach Gauss}
Matrix in Zeilenstufenform bringen mit folgenden Methoden
\begin{enumerate}[label=\textbf{\arabic*.}]
	\item Multiplikation einer Zeile mit einer Zahl
	\item Addition eines Vielfachen einer Zeile zu einer anderen Zeile
	\item Vertauschung von zwei Zeilen
\end{enumerate}

\subsection{Cramer'sche Regel}
\begin{align}
	x_i = \frac{\det(a_1,...,a_{i-1},b,a_{i+1},...,a_n)}{\det(A)}\notag
\end{align}

\section{Determinanten}
\begin{enumerate}[label=\textbf{\arabic*.}]
	\item Laplace'scher Entwicklungssatz: Entwicklung nach der $i$-ten Zeile
	\begin{align}
		\det(A) = \sum_{j=1}^n (-1)^{i+j}\cdot a_{ij}\cdot\det(A')\notag
	\end{align}
	\item Determinante einer Diagonalmatrix:
	\begin{align}
		\det(D) = \prod_{i=1}^n a_{ii}\notag
	\end{align}
\end{enumerate}

\section{Eigenwerte und Eigenvektoren}
\begin{enumerate}[label=\textbf{\arabic*.}]
	\item Das charakteristische Polynom bestimmen
	\begin{align}
		\chi = \det(A-\lambda\mathbb{1}_n)\notag
	\end{align}
	\item Das charakteristische Polynom 0 setzen und die $\lambda_i$'s ausrechnen.
	\item Für jedes $\lambda_i$ die Eigenräume berechnen
	\begin{align}
		\Eig(A,\lambda_i) = \Ker(A-\lambda_i\mathbb{1}_n)\notag
	\end{align}
	\item Die $\lambda_i$ sind dann die Eigenwerte und die Eigenräume sind alle Vielfachen des Eigenvektors zum Eigenwert $\lambda_i$.
\end{enumerate}

\section{Diagonalisierung und Trigonalisierung}
\subsection{Diagonalisierung}
\begin{enumerate}[label=\textbf{\arabic*.}]
	\item Die Eigenwerte und dazu die Vielfachheiten ausrechnen. Die Diagonalmatrix besteht dann auf der Hauptdiagonalen aus Eigenwerten mit ihrer Vielfachheit, also
	\begin{align}
		D = \diag(\underbrace{\lambda_1, ..., \lambda_1}_{\mu(\chi,\lambda_1)}, ..., \underbrace{\lambda_n, ..., \lambda_n}_{\mu(\chi,\lambda_n)})\notag
	\end{align}
	Allerdings ist $A$ nur dann diagonalisierbar, wenn für jeden Eigenwert gilt
	\begin{align}
		\dim(\Eig(A,\lambda)) = \mu(\chi,\lambda)\notag
	\end{align}
	\item Will man noch die Transformationsmatrizen $S$ und $S^{-1}$ berechnen, so muss man zu jedem Eigenraum eine Basis finden. Die Vereinigung bildet dann eine Basis $B$ von $V$.
	\item Die Matrix $S^{-1}$ besteht dann aus den Basiselementen von $B$ als Spaltenvektoren, also
	\begin{align}
		S^{-1} = \left(\begin{array}{c|c|c}
		&& \\
		b_1 & \dots & b_n \\
		&&
		\end{array}\right)\notag
	\end{align}
	\item Das Invertieren von $S^{-1}$ gibt dann $S$ und es gilt $SAS^{-1}=D$.
\end{enumerate}

\subsection{Trigonalisierung}
\begin{enumerate}[label=\textbf{\arabic*.}]
	\item Eigenwerte und Eigenvektoren von $A$ bestimmen und zu einer Basis von $V$ erweitern
	\begin{align}
		S_1^{-1} = \left(\begin{array}{c|c|c|c}
		&&& \\
		v_1 & e_2 & \dots & e_n \\
		&&&
		\end{array}\right)\notag
	\end{align}
	\item Matrix $A_2$ berechnen
	\begin{align}
		A_2 = S_1AS_1^{-1}\notag
	\end{align}
	\item Den Vorgang mit der noch nicht trigonalisierten Matrix unten rechts wiederholen,
	\begin{align}
		S_2^{-1} = \left(\begin{array}{c|c|c|c|c}
		&&&& \\
		v_1 & v_2 & e_3 & \dots & e_n \\
		&&&&
		\end{array}\right)\notag
	\end{align}
\end{enumerate}

\subsection{Minimal Polynom}
Betrachte Potenzen des Endomorphismus $a \in End(V)$ ($A$ ist die Matrix dazu) und finde geeignete Linearkombination. z.B:
	\begin{align}
		A=\begin{pmatrix}
			1 & -1 & -1\\
			1 & -2 & 1\\
			0 & 1 & -3
		\end{pmatrix}\notag
	\end{align}
Betrachte Basisvektor $e_1$ und Multipliziere mit Potenzen von $A$. Das sieht dann so aus:
	\begin{align}
	A^3 \cdot e_1 = -4A^2\cdot e_1 - T\cdot e_1 + e_1\notag \\
	\Rightarrow \text{Minimalpolynom: } \mu_A (X) = X^3 + 4X^2+X -I = \underline{0}\notag
	\end{align}

\section{Jordan-Normalform}
\begin{enumerate}[label=\textbf{\arabic*.}]
	\item Eigenwerte und deren Vielfachheit bestimmen
	\item zu jedem Eigenwert den Jordan-Block mit Größe = Vielfachheit des Eigenwertes bestimmen
	\item Die Jordan-Normalform besteht aus den Jordan-Blocken auf der Hauptdiagonalen
	\item Will man noch die Transformationsmatrizen bestimmen, muss man noch zu jedem Eigenwert einen Eigenvektor bestimmen.
	\item Kommt ein Eigenwert mehrfach (z.B. $n$-fach) vor, so muss man noch $\Ker\big((A-\lambda\mathbb{1}_n)^2\big)$, $\Ker\big((A-\lambda\mathbb{1}_n)^3\big)$, ..., $\Ker\big((A-\lambda\mathbb{1}_n)^{d-1}\big)$ bestimmen.
	\item Dann kann man sich die Transformationsmatrix $S$ zusammenbauen:
	\begin{align}
		S = \left(\begin{array}{c|c|c|c|c|c|c}
		EV & EV &  & EV & EV &  & EV \\
		\text{zu }\lambda_1 & \text{zu }\lambda_1 & \dots & \text{zu }\lambda_1 & \text{zu }\lambda_2 & \dots & \text{zu }\lambda_n \\
		\text{Potenz }1 & \text{Potenz }2 & & \text{Potenz }d-1 & \text{Potenz }1 & & \text{Potenz }d-1
		\end{array}\right)\notag
	\end{align}
	\item Dann gilt $SAS^{-1}=J$.
\end{enumerate}

\section{Gram-Schmidt-Verfahren}

\section{Smith-Normalform}
\end{document}