\RequirePackage{ifluatex,ifpdf}
\documentclass[ngerman,a4paper]{report}
\usepackage[left=2.1cm,right=3.1cm,bottom=3cm]{geometry}
\usepackage[ngerman]{babel}
\ifpdf
\usepackage[utf8]{inputenc} %not recommended with lualatex
\usepackage[T1]{fontenc}
\fi

\usepackage{zref-base}
\usepackage{etoolbox}
\usepackage{xparse}%better macros
\usepackage{chngcntr}
\usepackage{calc}

\usepackage{scalerel,stackengine}
\usepackage{tocloft}

\ifluatex
\usepackage{fontspec}
%\usepackage{luacode}
\fi

\usepackage[texindy]{imakeidx}
\indexsetup{
	level=\chapter*
}
\makeindex[intoc]

\usepackage[xindy,acronym]{glossaries}
\makeglossaries

\usepackage[title,titletoc]{appendix}

\usepackage{amsmath}
\usepackage{amssymb}
\usepackage{amsfonts}
\usepackage{mathtools}
\usepackage{latexsym}
\usepackage{marvosym} %lighning
\usepackage{bbm} %unitary matrix 1
\usepackage{cancel}
\usepackage{xfrac}%sfrac -> fractions e.g. 3/4

\usepackage[table]{xcolor}
\usepackage{graphicx}
\usepackage{pgfplots}
\pgfplotsset{compat=1.10}
\usepgfplotslibrary{fillbetween}
\usepackage{pgf}
\usepackage{tikz}
\usetikzlibrary{patterns,arrows,calc,decorations.pathmorphing}
\usetikzlibrary{matrix}
\usepackage{color}
\usepackage{wasysym}

\usepackage{enumerate}
\usepackage{enumitem} %customize label
\usepackage{stmaryrd} % Lightning symbol

\usepackage{tabularx}
\usepackage{multirow}
\usepackage{booktabs}

\usepackage{ulem} %better underlines

\usepackage{parskip}%split paragraphs by vspace instead of intendations
\usepackage{fancyhdr}
\usepackage{titlesec}%customize titles
\usepackage{marginnote}

\usepackage[amsmath,amsthm,thmmarks,hyperref]{ntheorem}%customize theorem-environments more effectively
\usepackage[ntheorem,framemethod=TikZ]{mdframed}

\usepackage[unicode,bookmarks=true]{hyperref}
\hypersetup{
	colorlinks,
	citecolor=green,
	filecolor=green,
	linkcolor=blue,
	urlcolor=green
}
\usepackage{cleveref}
\usepackage{bookmark}

\newcommand{\coloredRule}[3][black]{\textcolor{#1}{\rule{#2}{#3}}}
\newlength{\blacktrianglewidth}
\settowidth{\blacktrianglewidth}{$\blacktriangleright$}

\definecolor{lightgrey}{gray}{0.91}
\definecolor{lightred}{rgb}{1,0.6,0.6}
\definecolor{darkgrey}{gray}{0.6}
\definecolor{darkgreen}{rgb}{0,0.6,0}

%numbered theorems
\theoremstyle{break}
\theorembodyfont{}

\mdfdefinestyle{boxedtheorem}{%
	outerlinewidth=3pt,%
	skipabove=5pt,%
	skipbelow=10pt,%
	frametitlefont=\normalfont\bfseries\color{black},%
}

\newmdtheoremenv[%
	style=boxedtheorem,%
	innertopmargin=\topskip,%
	innerbottommargin=\topskip,%
	linecolor=darkgrey,%
	backgroundcolor=lightgrey,%
]{theorem}{Theorem}[section]

\newmdtheoremenv[%
	style=boxedtheorem,%
	linecolor=darkgrey,%
	topline=false,%
	rightline=false,%
	bottomline=false,%
	innertopmargin=\topskip,%
	innerbottommargin=\topskip,%
	backgroundcolor=lightgrey,%
]{proposition}[theorem]{Satz}

\newmdtheoremenv[%
	style=boxedtheorem,%
	linecolor=darkgrey,%
	topline=false,%
	rightline=false,%
	bottomline=false,%
	backgroundcolor=lightgrey,%
	innertopmargin=\topskip,%
	innerbottommargin=\topskip,%
]{lemma}[theorem]{Lemma}

\newmdtheoremenv[%
	style=boxedtheorem,%
	linecolor=red,%
	topline=false,%
	rightline=false,%
	bottomline=false,%
	innertopmargin=0,%
	innerbottommargin=-3pt,%
]{definition}[theorem]{Definition}

\newmdtheoremenv[%
	outerlinewidth=3pt,%
	linecolor=black,%
	topline=false,%
	rightline=false,%
	bottomline=false,%
	innertopmargin=0pt,%
	innerbottommargin=-0pt,%
	frametitlefont=\normalfont\bfseries\color{black},%
	skipabove=5pt,%	
	skipbelow=10pt,%
]{conclusion}[theorem]{Folgerung}

\newmdtheoremenv[%
	hidealllines=true,%
	frametitlefont=\normalfont\bfseries\color{black},%
	innerleftmargin=0pt,%
	skipabove=5pt,%
	innerleftmargin=10pt,%
]{remark}[theorem]{\hspace*{-10pt}$\blacktriangleright$\hspace*{\dimexpr 10pt - \blacktrianglewidth\relax}Bemerkung}

\newmdtheoremenv[%
	hidealllines=true,%
	frametitlefont=\normalfont\bfseries\color{black},%
	innerleftmargin=10pt,%
]{example}[theorem]{\hspace*{-10pt}\rule{5pt}{5pt}\hspace*{5pt}Beispiel}

%unnumbered theorems
\theoremstyle{nonumberbreak}
\theoremindent0cm
\newmdtheoremenv[%
	style=boxedtheorem,%
	linecolor=red,%
	topline=false,%
	rightline=false,%
	bottomline=false,%
	innertopmargin=1pt,%
	innerbottommargin=1pt,%
]{*definition}{Definition}

\newmdtheoremenv[%
	hidealllines=true,%
	frametitlefont=\normalfont\bfseries\color{black},%
	skipabove=5pt,%
	innerleftmargin=10pt,%
]{*remark}{\hspace*{-10pt}$\blacktriangleright$\hspace*{\dimexpr 10pt - \blacktrianglewidth\relax}Bemerkung}

\newmdtheoremenv[%
	hidealllines=true,%
	innerleftmargin=10pt,%
]{*example}{\hspace*{-10pt}\rule{5pt}{5pt}\hspace*{5pt}Beispiel}
\newtheorem{overview}[theorem]{Überblick}

\newmdtheoremenv[%
	style=boxedtheorem,%
	topline=false,%
	rightline=false,%
	leftline=false,
	bottomline=false,%
	innertopmargin=\topskip,%
	innerbottommargin=\topskip,%
	backgroundcolor=lightgrey,%
]{*anmerkung}{Anmerkung}

%Hinweis-Theoremstyle and environment
%To get rid of the parentheses, a new theorem style is neccessary (definition of nonumberbreak from ntheorem.sty)
%to achieve the underlining, this needed to put in the theoremstyle definition
\theoremheaderfont{\mdseries}
\theoremseparator{:}
\theorempostskip{0pt}
\makeatletter
\newtheoremstyle{noparentheses}%
	{\item[\rlap{\vbox{\hbox{\hskip\labelsep \theorem@headerfont
					\underline{##1}\theorem@separator}\hbox{\strut}}}]}%
	{\item[\rlap{\vbox{\hbox{\hskip\labelsep \theorem@headerfont
					\underline{##1\ ##3\theorem@separator}}\hbox{\strut}}}]}
\newtheoremstyle{underlinedPlain}%
	{\item[\hskip\labelsep \uline{\theorem@headerfont ##1\theorem@separator}]}%
	{\item[\hskip\labelsep \uline{\theorem@headerfont ##1\ \theorem@headerfont(##3)\theorem@separator}]}
\newtheoremstyle{underlinedEnvironment}{}%
{\item[\hskip\labelsep \uline{##1\theorem@headerfont ##3\theorem@separator}]}
\newtheoremstyle{boldEnvironment}{}%
{\item[\hskip\labelsep \textbf{##1\theorem@headerfont ##3\theorem@separator}]}
\newtheoremstyle{proofstyle}%
{\item[\hskip\labelsep {\theorem@headerfont ##1}\theorem@separator]}%
{\item[\hskip\labelsep {\theorem@headerfont ##1}\ (##3)\theorem@separator]}
\makeatother

\theoremstyle{noparentheses}
\newmdtheoremenv[%
	hidealllines=true,%
	innerleftmargin=1em,%
	innerbottommargin=0pt,%
	innerrightmargin=0,%
	skipbelow=0pt,%
]{interpretation}{\hspace*{\dimexpr - \mdflength{innerleftmargin}\relax}Interpretation}
\theoremstyle{underlinedPlain}
\newmdtheoremenv[%
	hidealllines=true,%
	innerleftmargin=1em,%
	innerrightmargin=0,%
	skipbelow=0pt,%
]{hint}{\hspace*{\dimexpr - \mdflength{innerleftmargin}\relax}Hinweis}

\theoremstyle{underlinedEnvironment}
\newmdtheoremenv[%
	hidealllines=true,%
	innerleftmargin=1em,%
	innerrightmargin=0,%
	skipbelow=0pt,%
]{underlinedenvironment}{\hspace*{\dimexpr -\mdflength{innerleftmargin}\relax}}
\theoremheaderfont{\bfseries}
\theoremstyle{boldEnvironment}
\newmdtheoremenv[%
	hidealllines=true,%
	innerleftmargin=1em,%
	innerrightmargin=0,%
	skipbelow=0pt,%
]{boldenvironment}{\hspace*{\dimexpr -\mdflength{innerleftmargin}\relax}}

\theoremstyle{proofstyle}
\theoremheaderfont{\normalfont\normalsize\itshape}
\theorembodyfont{\normalfont\small}
\theoremseparator{.}
\theorempreskip{5pt}
\theorempostskip{5pt}
\theoremsymbol{$\square$}
\renewtheorem{proof}{Beweis}

%for \cref: printed environment names
\crefname{theorem}{Theorem}{Theoreme}
\crefname{proposition}{Satz}{Sätze}
\crefname{lemma}{Lemma}{Lemmata}
\crefname{conclusion}{Folgerung}{Folgerungen}
\crefname{definition}{Definition}{Definitionen}
\crefname{remark}{Bemerkung}{Bemerkungen}
\crefname{example}{Beispiel}{Beispiele}
\crefname{*definition}{Definition}{Definitionen}
\crefname{*remark}{Bemerkung}{Bemerkungen}
\crefname{*example}{Beispiel}{Beispiele}

\makeatletter
\newcommand*{\rom}[1]{\expandafter\@slowromancap\romannumeral #1@}
\newcommand*{\proplbl}[1]{%
	\@bsphack
	\begingroup
	\label{#1}%
	\zref@setcurrent{default}{\arabic{chapter}}%
%		\zref@wrapper@immediate{%
		\zref@labelbyprops{#1@chapter}{default}
%		}
	\endgroup
	\@esphack
}
\newcommand*{\propref}[1]{%
	\ifcsdef{r@#1}%in first compilation the label may not be defined yet
	{%
		\zref@refused{#1@chapter}%
		\ifnumcomp{\value{chapter}}{=}{\zref@extractdefault{#1@chapter}{default}{0}}%
		{%same chapter
			\ifmmode 
				\cref{#1}%
			\else
				\mbox{\cref{#1}}%
			\fi
		}%
		{%otherwise
			\def\propositionref@current@type{}%
			\cref@gettype{#1}{\propositionref@current@type}%get the environment's name
			%example for following line:
			%\crefformat{truetheorem}{\cref@truetheorem@name~##2\rom{\zref@extractdefault{#1}{#1chapter}{1}}.##1##3}
			%this changes the format used by \cref to <environtment name> <chapter-number>.<section-number>.<theorem number>
			\crefformat{\propositionref@current@type}{%
				\csname cref@\propositionref@current@type @name\endcsname ~##2\rom{\zref@extractdefault{#1@chapter}{default}{1}}.##1##3%
			}%
			\ifmmode 
				\cref{#1}%
			\else
				\mbox{\cref{#1}}%
			\fi
			\crefformat{\propositionref@current@type}{%
				\csname cref@\propositionref@current@type @name\endcsname~##2##1##3%
			}%
		}%
	}%
	{??}%similar to \ref\cref: question marks in case of undefined labels
}
\makeatother

\NewDocumentCommand{\begriff}{s O{} m O{}}{%
	\IfBooleanTF{#1}%
	{\index{#2#3#4}}%
	{%
		\uline{#3}%
		\ifnumcomp{\value{section}}{<}{16}%
		{\index[semester1]{#2#3#4}}%
		{\index[semester2]{#2#3#4}}%
		\index{#2#3#4}%
	}%
}
\NewDocumentCommand{\mathsymbol}{s O{} m m O{}}{%
	\IfBooleanTF{#1}%
	{\index[symbols]{#2#3@\detokenize{#4}#5}}%
	{#4\index[symbols]{#2#3@\detokenize{#4}#5}}%
}
\NewDocumentCommand{\zeroAmsmathAlignVSpaces}{s s O{0 pt} O{0 pt}}{%
	\IfBooleanTF{#1}%
	{%
		\IfBooleanTF{#2}%
			{\setlength{\belowdisplayskip}{#4}}%
			{\setlength{\abovedisplayskip}{#3}}%
	}%
	{%
		\setlength{\abovedisplayskip}{#3}%
		\setlength{\belowdisplayskip}{#4}%
	}%
}

\NewDocumentCommand{\transpose}{m}{\ensuremath{#1^\mathsf{T}}}

\NewDocumentCommand{\itemEq}{s m}{%
	\begingroup%
	\setlength{\abovedisplayskip}{\dimexpr -\parskip + 1pt\relax}%
	\setlength{\belowdisplayskip}{0pt}%
	\IfBooleanTF{#1}%
		{\parbox[c]{\linewidth}{\begin{flalign*}#2&&\end{flalign*}}}%}
		{\parbox[c]{\linewidth}{\begin{flalign}#2&&\end{flalign}}}%}
	\endgroup% 
}
\newcommand\equalhat{\mathrel{\stackon[1.5pt]{=}{\stretchto{%
	\scalerel*[\widthof{=}]{\wedge}{\rule{1ex}{3ex}}}{0.5ex}}}}

\makeatletter
\newcommand{\leqnos}{\tagsleft@true\let\veqno\@@leqno}
\newcommand{\reqnos}{\tagsleft@false\let\veqno\@@eqno}
\reqnos

\pdfstringdefDisableCommands{%
	\def\\{}%
	\def\texttt#1{<#1>}%
	\def\mathbb#1{<#1>}%
}
\makeatother

%General newcommands!
\newcommand{\comp}{\mathbb{C}} % complex set C
\newcommand{\real}{\mathbb{R}} % real set R
\newcommand{\whole}{\mathbb{Z}} % whole number Symbol
\newcommand{\natur}{\mathbb{N}} % natural number Symbol
\newcommand{\ratio}{\mathbb{Q}} % rational number symbol
\newcommand{\field}{\mathbb{K}} % general field for the others above!
\newcommand{\diff}{\mathrm{d}} % differential d
\newcommand{\s}{\,\,}     % space after the function in the intergral
\newcommand{\cont}{\mathcal{C}} % Contour C
\newcommand{\fuk}{f(z) \s\diff z} % f(z) dz
\newcommand{\diffz}{\s\diff z}
\newcommand{\subint}{\int\limits} % lower boundaries for the integral
\newcommand{\poly}{\mathcal{P}} % special P - polygon
\newcommand{\defi}{\mathcal{D}} % D for the domain of a function
\newcommand{\cover}{\mathcal{U}} % cover for a set
\newcommand{\setsys}{\mathcal{M}} % set system M
\newcommand{\setnys}{\mathcal{N}} % set system N
\newcommand{\zetafunk}{f(\zeta)\s\diff \zeta} %f(zeta) d zeta
\newcommand{\ztfunk}{f(\zeta)} % f(zeta)
\newcommand{\bocirc}{S_r(z)}
\newcommand{\prop}{\,|\,}
\newcommand*{\QEDA}{\hfill\ensuremath{\blacksquare}} %tombstone
\newcommand{\emptybra}{\{\varnothing\}} % empty set with set-bracket
\newcommand{\realpos}{\real_{>0}}
\newcommand{\realposr}{\real_{\geq0}}
\newcommand{\naturpos}{\natur_{>0}}
\newcommand{\Imag}{\operatorname{Im}} % Imaginary symbol
\newcommand{\Realz}{\operatorname{Re}} % Real symbol
\newcommand{\norm}{\Vert \cdot \Vert}
\newcommand{\metric}{\vert \cdot \vert}
\newcommand{\foralln}{\forall n} %all n
\newcommand{\forallnset}{\forall n \in \natur} %all n € |N
\newcommand{\forallnz}{\forall n \geq _0} % all n >= n_0
\newcommand{\conjz}{\overline{z}} % conjugated z
\newcommand{\tildz}{\tilde{z}} % different z
\newcommand{\lproofar}{"`$ \Leftarrow $"'} % "`<="'
\newcommand{\rproofar}{"`$ \Rightarrow $"'} % "`=>"'
\newcommand{\beha}{\Rightarrow \text{ Behauptung}}
\newcommand{\powerset}{\mathcal{P}}
\newcommand{\person}[1]{\textsc{#1}}
\newcommand{\highlight}[1]{\emph{#1}}
\newcommand{\realz}{\mathfrak{Re}}
\newcommand{\imagz}{\mathfrak{Im}}
\renewcommand{\epsilon}{\varepsilon}
\renewcommand{\phi}{\varphi}
\newcommand{\lebesque}{\person{Lebesgue}}
\renewcommand{\Re}{\mathfrak{Re}}
\renewcommand{\Im}{\mathfrak{Im}}
\renewcommand*{\arraystretch}{1.4}

% Math Operators
\DeclareMathOperator{\inn}{int} % Set of inner points
\DeclareMathOperator{\ext}{ext} % Set of outer points
\DeclareMathOperator{\cl}{cl} % Closure
\DeclareMathOperator{\grad}{grad}
\DeclareMathOperator{\D}{d}
\DeclareMathOperator{\id}{id}
\DeclareMathOperator{\graph}{graph}
\DeclareMathOperator{\Int}{int}
\DeclareMathOperator{\Ext}{ext}
\DeclareMathOperator{\diam}{diam}

\DeclareMathOperator{\End}{End}
\DeclareMathOperator{\Aut}{Aut}
\DeclareMathOperator{\Hom}{Hom}
\DeclareMathOperator{\Eig}{Eig}
\DeclareMathOperator{\Mat}{Mat}
\DeclareMathOperator{\Ker}{Ker}
\DeclareMathOperator{\diag}{diag}

%change headings:
\titlelabel{\thetitle.\quad}%. behind section/sub... (3. instead of 3)
\counterwithout{section}{chapter}
\renewcommand{\thechapter}{\Roman{chapter}}
\renewcommand{\thepart}{\Alph{part}}
%italic chapters (due to titlesec package some more stuff)
%\titleformat{command}[shape]{format}{label}{sep}{before-code}[after-code]
\titleformat{\chapter}[display]{\bfseries}{\Large\chaptername\;\thechapter}{-5pt}{\huge\bfseries\itshape}
\titlespacing{\chapter}{0pt}{0pt}{10pt}
\titleformat{\section}[hang]{\bfseries\Large}{\thesection.}{8pt}{\Large\bfseries}
%\titlespacing{command}{left}{before-sep}{after-sep}
\titlespacing{\subsection}{0pt}{0pt}{5pt}

%change appearence of heading of toc: 0 space above, bold, italic huge toc-heading
\renewcommand{\cftbeforetoctitleskip}{0pt}
\renewcommand{\cfttoctitlefont}{\itshape\Huge\bfseries}
%change indentations due to width of capital roman numbers
\renewcommand{\cftchapnumwidth}{2.5em}
\renewcommand{\cftsecindent}{2.5em}
%\renewcommand{\cftsecnumwidth}{3.3em}
\renewcommand{\cftsubsecindent}{4.8em}
%\renewcommand{\cftsubsecnumwidth}{4.2em}

%change header:
\renewcommand{\headrulewidth}{0.75pt}
\renewcommand{\footrulewidth}{0.3pt}
\lhead{\rightmark}%left: section-number. section-title
\rhead{\leftmark}%right: chapter chapternumber: chapter-title

% Add new page-style (just footer), patch \chapter command to use this page style
\fancypagestyle{plainChapter}{%
	\fancyhf{}%
	\fancyfoot[C]{\thepage}%
	\renewcommand{\headrulewidth}{0pt}% Line at the header invisible
	\renewcommand{\footrulewidth}{0.4pt}% Line at the footer visible
}
\patchcmd{\chapter}{\thispagestyle{plain}}{\thispagestyle{plainChapter}}{}{}

\pagestyle{fancy}
\pagenumbering{arabic}
%remember chapter-title in \leftmark and \rightmark
\renewcommand{\chaptermark}[1]{%
	\markboth{\chaptername
		\ \thechapter:\ #1}{}}
%remember section title in \leftmark
\renewcommand{\sectionmark}[1]{%
	\markright{\thesection.\ #1}{}}

%change numbering of equations to be section by section
\counterwithout{equation}{section}

\newacronym{gdw}{gdw.}{genau dann wenn}
\newacronym{fa}{fa.}{fast alle}
\newacronym{fü}{f.ü.}{fast überall}
\newacronym{obda}{oBdA}{ohne Beschränkung der Allgemeinheit}
\newacronym{tf}{TF}{\begriff{Teilfolge}}
\newacronym{hw}{Hw}{\begriff{Häufungswert}}
\newacronym{cf}{CF}{\begriff{\person{Cauchy}-Folge}}
\newacronym{hp}{HP}{\begriff{Häufungspunkt}}
\newacronym{vr}{VR}{Vektorraum}
\newacronym{diffbar}{diffbar}{differenzierbar}
\newacronym{mws}{MWS}{Mittelwertsatz}

\title{\textbf{Lineare Algebra SS2018}}
\author{Dozent: Prof. Dr. Arno Fehm}

%remove page number from part{}-pages
\makeatletter
\let\sv@endpart\@endpart
\def\@endpart{\thispagestyle{empty}\sv@endpart}
\makeatother

\begin{document}
\pagenumbering{roman}
\pagestyle{plain}

\maketitle

\hypertarget{tocpage}{}
\tableofcontents
\bookmark[dest=tocpage,level=1]{Inhaltsverzeichnis}

\pagebreak
\pagestyle{fancy}
\pagenumbering{arabic}
\pagestyle{fancy}

\chapter{Endomorphismen}
In diesem Kapitel seien $K$ ein Körper, $n\in\natur$ eine natürliche Zahl, $V$ ein $n$-dimensionaler $K$-Vektorraum und $f\in\End_K(V)$ ein Endomorphismus.

Das Ziel dieses Kapitels ist, die Geometrie von $f$ besser zu verstehen und Basen zu finden, für die $M_B(f)$ eine besonders einfache oder kanonische Form hat.

\section{Eigenwerte}

\begin{remark}
	Wir erinnern uns daran, dass $\End_K(V)=\Hom_K(V,V)$ sowohl einen $K$-Vektorraum als auch einen Ring bildet. Bei der Wahl einer Basis $B$ von $V$ wird $f\in\End_K(V)$ durch die Matrix $M_B(f)=M_B^B(f)$ beschrieben.	
\end{remark}

\begin{example}
	$K=\real, A=\begin{henrysmatrix}1&2\\2&1\end{henrysmatrix}\in\Mat_2(\real),f=f_A\in\End_K(K^2)$ \\
	\begin{align}
		A\cdot \begin{henrysmatrix}1\\1\end{henrysmatrix}=\begin{henrysmatrix}3\\3\end{henrysmatrix},\;A\cdot\begin{henrysmatrix} 1\\-1\end{henrysmatrix}=\begin{henrysmatrix}-1\\1\end{henrysmatrix}\notag
	\end{align}
	$\Rightarrow$ mit $B=\left( \begin{henrysmatrix}1\\1\end{henrysmatrix},\begin{henrysmatrix}1\\-1\end{henrysmatrix}\right)$ ist $M_B(f)=\begin{henrysmatrix}3&0\\0&-1\end{henrysmatrix}$. \\
	Der Endomorphismus $f=f_A$ streckt also entlang der Achse $\real\cdot \begin{henrysmatrix}1\\1\end{henrysmatrix}$ um den Faktor 3 und spiegelt entlang der Achse $\real\cdot \begin{henrysmatrix}1\\-1\end{henrysmatrix}$
	\begin{center}
		\begin{tikzpicture}
		\draw[->,thick] (-2,0) -- (2,0) node[right] {$x_1$};
		\draw[->,thick] (0,-2) -- (0,2) node[above] {$x_2$};
		\draw[->, thin] (0,0) -- (1.414,1.414);
		\draw[->, thin] (0,0) -- (-1.414,-1.414);
		\draw[->, thin] (0,0) -- (-0.707,0.707);
		\draw[->, thin] (0,0) -- (0.707,-0.707);
		\draw[dashed, rotate=+45, blue] (0,0) ellipse (2cm and 1cm);
		\draw (0,0) circle (1);
		\end{tikzpicture}
	\end{center}
\end{example}

\begin{definition}[Eigenwert, Eigenvektor, Eigenraum]
	Sind $0\neq x\in V$ und $\lambda\in K$ mit $f(x)=\lambda x$ so nennt man $\lambda$ einen \begriff{Eigenwert} von $f$ und $x$ einen \begriff{Eigenvektor} von $f$ zum Eigenwert $\lambda$. Der \begriff{Eigenraum} zu $\lambda\in K$ ist $\Eig (f,\lambda)=\{x\in V\mid f(x)=\lambda x\}$.
\end{definition}

\begin{remark}
	Für jedes $\lambda\in K$ ist $\Eig (f,\lambda)$ ein Untervektorraum von $V$, da
	\begin{align}
		\Eig (f,\lambda) &= \{x\in V\mid f(x)=\lambda x\} \notag \\
		&= \{x\in V\mid f(x)-\lambda\cdot\id_V(x)=0\} \notag \\
		&= \{x\in V\mid (f-\lambda\cdot\id_V)(x)=0\} \notag \\
		&= \Ker (f-\lambda\cdot\id_V) \notag
	\end{align}
	und $f-\lambda\cdot\id_V\in\End_K(V)$.
\end{remark}

\begin{remark}
	Achtung! Der Nullvektor ist nach Definition kein Eigenvektor, aber $\lambda=0$ kann ein Eigenwert sein, nämlich genau dann, wenn $f\notin\Aut_K(V)$, siehe Übung. Die Menge der Eigenvektoren zu $\lambda$ ist also $\Eig (f,\lambda)\backslash\{0\}$ und $\lambda$ ist genau dann ein Eigenwert von $f$, wenn $\Eig (f,\lambda)\neq\{0\}$.
\end{remark}

\begin{example}
	Ist $A=\diag(\lambda_1,...,\lambda_n)$ und $f=f_A\in\End_K(K^n)$, so sind $\lambda_1,...,\lambda_n$ Eigenwerte von $f$ und jedes $e_i$ ist ein Eigenvektor zum Eigenwert $\lambda_i$.
\end{example}

\begin{proposition}
	\proplbl{satz_diagonal_ev}
	Sei $B$ eine Basis von $V$. Genau dann ist $M_B(f)$ eine Diagonalmatrix, wenn $B$ aus Eigenvektoren von $f$ besteht.
\end{proposition}
\begin{proof}
	Ist $B=(x_1,...x_n)$ eine Basis aus Eigenvektoren zu Eigenwerten $\lambda_1,....,\lambda_n$, so ist $M_B(f)= \diag(\lambda_1,...,\lambda_n)$ und umgekehrt.
\end{proof}

\begin{example}
	Sei $K=\real$, $V=\real^2$ und $f_{\alpha}\in\End_K(\real^2)$ die Drehung um den Winkel $\alpha\in [0,2\pi)$ \\
	\[\Rightarrow M_{\mathcal{E}}(f_{\alpha})=\begin{pmatrix}\cos(\alpha)&-\sin(\alpha) \\ \sin(\alpha) & \cos(\alpha)\end{pmatrix}\]
	Für $\alpha=0$ hat $f_{\alpha}=\id_{\real^2}$ nur den Eigenwert 1. \\
	Für $\alpha=\pi$ hat $f_{\alpha}=-\id_{\real^2}$ nur den Eigenwert -1. \\
	Für $\alpha\neq 0,\pi$ hat $f_{\alpha}$ keine Eigenwerte. %TODO figure
\end{example}

\begin{lemma}
	\proplbl{lemma_EW_lin_unabh}
	Sind $\lambda_1,...,\lambda_n$ paarweise verschiedene Eigenwerte von $f$ und ist $x_i$ ein Eigenvektor zu $\lambda_i$ für $i=1,...,m$, so ist $(x_1,...,x_m)$ linear unabhängig.
\end{lemma}
\begin{proof}
	Induktion nach $m$\\
	\emph{$m=1$}: klar, denn $x_1\neq 0$ \\
	\emph{$m-1\to m$}: Sei $\sum_{i=1}^m \mu_i x_i=0$ mit $\mu_1,...,\mu_m\in K$.
	\begin{align}
		0&= (f-\lambda\cdot\id_V)\left( \sum\limits_{i=1}^m \mu_i x_i\right) \notag \\
		&= \sum\limits_{i=1}^m \mu_i(f(x_i)-\lambda_m\cdot x_i) \notag \\
		&= \sum\limits_{i=1}^{m-1} \mu_i(\lambda_i-\lambda_m)\cdot x_i \notag
	\end{align} 
	Nach IB ist $\mu_i(\lambda_i-\lambda_m)=0$ für $i=1,...,m-1$, da $\lambda_i\neq\lambda_m$ für $i\neq m$ also $\mu_i=0$ für $i=1,...,m-1$. Damit ist auch $\mu_m=0$. Folglich ist $(x_1,...,x_m)$ linear unabhängig.
\end{proof}

\begin{proposition}
	\proplbl{satz_eig_direkte_summe}
	Sind $\lambda_1,...,\lambda_m\in K$ paarweise verschieden, so ist 
	\[\sum\limits_{i=1}^m \Eig(f,\lambda_i)=\bigoplus_{i=0}^{m}\Eig(f,\lambda_i).\]
\end{proposition}
\begin{proof}
	Seien $x_i,y_i\in\Eig(f,\lambda_i)$ für $i=1,...,m$. Ist $\sum_{i=1}^m x_i=\sum_{i=1}^m y_i$, so ist $\sum_{i=1}^m \underbrace{x_i-y_i}_{z_i}=0$.\\
	o. E. seien $z_i\neq 0$ für $i=1,...,r$ und $z_i=0$ für $i=r+1,...,m$. Wäre $r>0$, so wären $(z_1,...,z_r)$ linear abhängig, aber $z_i=x_i-y_i\in\Eig(f,\lambda_i)\backslash\{0\}$, im Widerspruch zu \propref{lemma_EW_lin_unabh}. Somit ist $x_i=y_i$ für alle $i$ und folglich ist die Summe $\sum\Eig(f,\lambda_i)$ direkt.
\end{proof}

\begin{definition}[Eigenwerte und Eigenvektoren für Matrizen]
	Sei $A\in\Mat_n(K)$. Man definiert Eigenwerte, Eigenvektoren, etc. von $A$ als Eigenwerte, Eigenvektoren von $f_A\in\End_K(K^n)$.
\end{definition}

\begin{mathematica}[Eigenwerte und Eigenvektoren]
	Um die Eigenwerte und Eigenvektoren einer Matrix $A$ zu berechnen, gibt es in Mathematica bzw. WolframAlpha verschiedene Möglichkeiten:
	\begin{itemize}
		\item \texttt{Eigenvalues[A]}: liefert eine Liste der Eigenwerte
		\item \texttt{Eigenvectors[A]}: liefert eine Liste der Eigenvektoren
		\item \texttt{Eigensystem[A]}: liefert zu jeden Eigenwert den Eigenvektor
	\end{itemize}
\end{mathematica}

\begin{proposition}
	Sei $B$ eine Basis von $V$ und $\lambda\in K$. Genau dann ist $\lambda$ ein Eigenvektor von $f$, wenn $\lambda$ ein Eigenwert von $A=M_B(f)$ ist. Insbesondere haben ähnliche Matrizen die selben Eigenwerte.
\end{proposition}
\begin{proof}
	Dies folgt aus dem kommutativen Diagramm
	\begin{center}\begin{tikzpicture}
		\matrix (m) [matrix of math nodes,row sep=3em,column sep=4em,minimum width=2em]
		{K^n & K^n \\ V & V \\};
		\path[-stealth]
		(m-1-1) edge node [left] {$\Phi_B$} (m-2-1)
		edge node [above] {$f_A$} (m-1-2)
		(m-2-1) edge node [below] {$f$} (m-2-2)
		(m-1-2) edge node [right] {$\Phi_B$} (m-2-2);
		\end{tikzpicture}\end{center}
	denn $f_A(x)=\lambda x\iff (\Phi_B\circ f_A)(x)=\Phi_B(\lambda x)\iff f(\Phi_B(x))=\lambda\Phi_B(x)$. \\
	Ähnliche Matrizen beschreiben den selben Endomorphismus bezüglich verschiedener Basen, vgl. \propref{4_4_1}
\end{proof}
\section{Das charakteristische Polynom}

\begin{proposition}
	\proplbl{satz_det_null}
	Sei $\lambda\in K$. Genau dann ist $\lambda$ ein EW von $f$, wenn $\det(\lambda\cdot\id_V-f)=0$.
\end{proposition}
\begin{proof}
	Da $\Eig(f,\lambda)=\Ker(\lambda\cdot\id_V-f)$ ist $\lambda$ genau dann ein EW von $f$, wenn $\dim_K(\Ker(\lambda\cdot\id_V-f))>0$, also wenn $\lambda\cdot\id_V-f\notin\Aut_K(V)$. Nach IV.4.6 bedeutet dies, dass $\det(\lambda\cdot\id_V-f)=0$ %TODO: Verlinkung setzen
\end{proof}

\begin{definition}[charakteristisches Polynom]
	Das \begriff{charakteristische Polynom} einer Matrix $A\in\Mat_n(K)$ ist die Determinante der Matrix $t\cdot \mathbbm{1}_n-A\in\Mat_n(K[t])$. 
	\begin{align}
		\chi_A(t)&=\det(t\cdot \mathbbm{1}_n-A)\in K[t] \notag
	\end{align}
	Das charakteristische Polynom eines Endomorphismus $f\in\End_K(V)$ ist $\chi_f(t)=\chi_{M_B(f)}(t)$, wobei $B$ eine Basis von $V$ ist.
\end{definition}

\begin{proposition}
	\proplbl{satz_2_3}
	Sind $A,B\in\Mat_n(K)$ mit $A\sim B$, so ist $\chi_A=\chi_B$. Insbesondere ist $\chi_f$ wohldefiniert.
\end{proposition}
\begin{proof}
	Ist $B=SAS^{-1}$ mit $S\in\GL_n(K)$, so ist $t\cdot \mathbbm{1}_n-B = S(t\cdot \mathbbm{1}_n-A)S^{-1}$, also $t\cdot \mathbbm{1}_n-B\sim t\cdot \mathbbm{1}_n-A$ und ähnliche Matrizen haben die selben Determinante (IV.4.4). \\
	Sind $B,B'$ Basen von $V$, so sind $M_B(f)\sim M_{B'}(f)$, also $\chi_{M_B(f)}=\chi_{M_{B'}(f)}$ %TODO: Verlinkung setzen
\end{proof}

\begin{lemma}
	\proplbl{lemma_chi_det}
	Für $\lambda\in K$ ist $\chi_f(\lambda)=\det(\lambda\cdot\id_V-f)$.
\end{lemma}
\begin{proof}
	Sei $B$ eine Basis von $V$ und $A=M_B(f)=(a_{ij})_{i,j}$. Dann ist $M_B(\lambda\cdot\id_V-f)= \lambda\cdot \mathbbm{1}_n-A$. Aus IV.2.8 und I.6.8 folgt $\det(t\cdot \mathbbm{1}_n-A)(\lambda)=\det(\lambda\cdot \mathbbm{1}_n-A)$. Folglich ist 
	\begin{align}
		\chi_f(\lambda)&=\chi_A(\lambda)\notag \\
		&=\det(t\cdot \mathbbm{1}_n-A)(\lambda)\notag \\
		&=\det(\lambda\cdot \mathbbm{1}_n-A)\notag \\
		&= \det(\lambda\cdot\id_V-f) \notag
	\end{align}
\end{proof}

\begin{proposition}
	\proplbl{satz_chi_polynom}
	Sei $\dim_K(V)=n$ und $f\in\End_K(V)$. Dann ist $\chi_f(t)=\sum_{i=0}^n \alpha_i t^i$ ein Polynom vom Grad $n$ mit 
	\begin{align}
		\alpha_n&=1\notag \\
		\alpha_{n-1}&=-\tr(f) \notag \\
		\alpha_0 &= (-1)^n\cdot\det(f) \notag
	\end{align}
	Die Nullstellen von $\chi_f$ sind genau die EW von $f$.
\end{proposition}
\begin{proof}
	Sei $B$ eine Basis von $V$ und $A=M_B(f)=(a_{ij})_{i,j}$. Wir erinnern uns daran, dass $\tr(f)=\tr(A=\sum_{i=1}^n a_{ii}$. Es ist $\chi_f(t)=\det(t-\cdot 1_n-A)=\sum_{\sigma\in S_n}\sgn(\sigma)\prod_{i=1}^n (t\delta_{i,\sigma(i)}-a_{i,\sigma(i)})$. \\
	Der Summand für \emph{$\sigma=\id$} ist $\prod_{i=1}^n (t-a_{ii})=t^n+\sum_{i=1}^n (-a_{ii})t^{n-1}+...+\prod_{i=1}^n(-a_{ii})$ \\
	Für \emph{$\sigma\neq\id$} ist $\sigma(i)\neq i$ für mindestens zwei $i$, der entsprechende Summand hat also Grad höchstens $n-2$. Somit haben $\alpha_n$ und $\alpha_{n-1}$ die oben behauptete Form, und $\alpha_0=\chi_A(0)=\det(-A)=(-1)^n\cdot\det(f)$. \\
	Die Aussage über die Nullstellen von $\chi_f$ folgt aus \propref{satz_det_null} und \propref{lemma_chi_det}.
\end{proof}

\begin{conclusion}
	Ist $\dim_K(V)=n$, so hat $f$ höchstens $n$ Eigenwerte.
\end{conclusion}
\begin{proof}
	\propref{satz_chi_polynom} und I.6.10 %TODO: Verlinkung
\end{proof}

\begin{definition}[normiertes Polynom]
	Ein Polynom $0\neq P\in K[t]$ mit Leitkoeffizient 1 heißt \begriff{normiert}.
\end{definition}

\begin{example}
	\proplbl{beispiel_2_8}
	\begin{enumerate}
		\item Ist $A=(a_{ij})_{i,j}$ eine obere Dreiecksmatrix, so ist $\chi_A(t)=\prod_{i=1}^n (t-a_{ii})$, vgl. IV.2.9.c \\ %TODO: Verlinkung
		Insbesondere ist $\chi_{1_n}(t)=(t-1)^n$, $\chi_0(t)=t^n$
		\item Für eine Blockmatrix $A=\begin{pmatrix}A_1&B \\ 0&A_2\end{pmatrix}$ mit quadratischen Matrizen $A_1,A_2$ ist $\chi_A=\chi_{A_1}\cdot \chi_{A_2}$ vgl. IV.2.9.e %TODO: Verlinkung
		\item Für
		\begin{align}
			\begin{pmatrix}
			0&...&...&...&0&-c_0  \\ 
			1& \ddots&\;&\;&\vdots&\vdots  \\ 
			0&\ddots&\ddots&\;&\vdots&\vdots  \\ 
			\vdots&\ddots&\ddots&\ddots&\vdots&\vdots  \\ 
			0&...&0&1&0&-c_{n-1} 
			\end{pmatrix} \quad c_0,...,c_{n-1}\in K \notag
		\end{align}
		ist $\chi_A(t)=t^n+\sum_{i=0}^{n-1} c_i t^i$ \\
		Man nennt diese Matrix die Begleitmatrix zum normierten Polynom $P=t^n+\sum_{i=0}^{n-1} c_i t^i$ und schreibt $M_P:=A$
	\end{enumerate}
\end{example}

\chapter{Skalarprodukte}
\input{./TeX_files/chapter3}

\chapter{Dualität}

\chapter{Moduln}

\part*{Anhang}
\addcontentsline{toc}{part}{Anhang}
\appendix
\patchcmd{\chapter}{\thispagestyle{plainChapter}}{\thispagestyle{fancy}}{}{}
%\titleformat{command}[shape]{format}{label}{sep}{before-code}[after-code]
%\titlespacing{command}{left}{before-sep}{after-sep}
\renewcommand{\chaptername}{Anhang}
\renewcommand{\thechapter}{\Alph{chapter}}
\titleformat{\chapter}[hang]{\bfseries}{\LARGE\chaptername\ \thechapter:}{0.5em}{\LARGE\bfseries}
\titlespacing{\chapter}{0pt}{-0.75cm}{0pt}
\renewcommand{\thesection}{\Alph{chapter}.\arabic{section}}

%from ntheorem.sty
%\def\thm@@thmline@name#1#2#3#4#5{%
%	\ifx\\#5\\%
%		\@dottedtocline{-2}{0em}{2.3em}%
%		{#1 \protect\numberline{#2}#3}%
%		{#4}
%	\else
%		\ifHy@linktocpage\relax\relax
%			\@dottedtocline{-2}{0em}{2.3em}%
%			{#1 \protect\numberline{#2}#3}%
%			{\hyper@linkstart{link}{#5}{#4}\hyper@linkend}%
%		\else
%			\@dottedtocline{-2}{0em}{2.3em}%
%			{\hyper@linkstart{link}{#5}%
%			{#1 \protect\numberline{#2}#3}\hyper@linkend}%
%			{#4}%
%		\fi
%	\fi
%}

\makeatletter
%update ntheorem macro to provide space between theorem numbers and any optional comment
\renewcommand{\thm@@thmline@name}[5]{%
	\def\thm@@thmline@name@tmp{%
		\if\relax\detokenize{#3}\relax%
			{:}%
		\else%
			{:\hspace*{1em}#3}%
		\fi%
	}%
	\ifx\\#5\\%
		\@dottedtocline{-2}{0em}{2.3em}%
		{#1 \protect\numberline{#2}{\thm@@thmline@name@tmp}}%
		{#4}
	\else
		\ifHy@linktocpage\relax\relax
			\@dottedtocline{-2}{0em}{2.3em}%
			{#1 \protect\numberline{#2}{\thm@@thmline@name@tmp}}%
			{\hyper@linkstart{link}{#5}{#4}\hyper@linkend}%
		\else
			\@dottedtocline{-2}{0em}{2.3em}%
			{%
			{#1 \protect\numberline{#2}{\thm@@thmline@name@tmp}}}%
			{\hyper@linkstart{link}{#5}{#4}\hyper@linkend}%
		\fi
	\fi
}
\makeatother

\chapter{Listen}
\section{Liste der Theoreme}
\theoremlisttype{allname}
\listtheorems{theorem}

\pagebreak
\section{Liste der benannten Sätze}
\theoremlisttype{optname}
\listtheorems{proposition}

\printglossary[type=\acronymtype]
\addcontentsline{toc}{chapter}{Akronyme}

\printindex

\end{document}