\RequirePackage{ifluatex,ifpdf}
\documentclass[ngerman,a4paper]{report}
\usepackage[left=2.1cm,right=3.1cm,bottom=3cm,footskip=0.75cm,headsep=0.5cm]{geometry}
\usepackage[ngerman]{babel}
\ifpdf
\usepackage[utf8]{inputenc} %not recommended with lualatex
\usepackage[T1]{fontenc}
\fi

\usepackage{zref-base}
\usepackage{etoolbox}
\usepackage{xparse}%better macros
\usepackage{chngcntr}
\usepackage{calc}

\usepackage{scalerel,stackengine}
\usepackage{tocloft}

\ifluatex
\usepackage{fontspec}
%awesome package for debugging spacing issues
%\usepackage{lua-visual-debug}
%\usepackage{luacode}
\fi

\usepackage[texindy]{imakeidx}
\indexsetup{
	level=\chapter*
}
\makeindex[intoc]

\usepackage[xindy,acronym]{glossaries}
\makeglossaries

\usepackage[title,titletoc]{appendix}

\usepackage{amsmath}
\usepackage{amssymb}
\usepackage{amsfonts}
\usepackage{mathtools}
\usepackage{latexsym}
\usepackage{marvosym} %lighning
\usepackage{bbm} %unitary matrix 1
\usepackage{cancel}
\usepackage{xfrac}%sfrac -> fractions e.g. 3/4

\usepackage[table]{xcolor}
\usepackage{graphicx}
\usepackage{pgfplots}
\pgfplotsset{compat=1.10}
\usepgfplotslibrary{fillbetween}
\usepackage{pgf}
\usepackage{tikz}
\usetikzlibrary{patterns,arrows,calc,decorations.pathmorphing}
\usetikzlibrary{matrix}
\usepackage{color}
\usepackage{wasysym}
\usepackage{tcolorbox}

\usepackage{enumerate}
\usepackage{enumitem} %customize label
\usepackage{stmaryrd} % Lightning symbol

\usepackage{tabularx}
\usepackage{multirow}
\usepackage{booktabs}

\usepackage{ulem} %better underlines

\usepackage{parskip}%split paragraphs by vspace instead of intendations
\usepackage{fancyhdr}
\usepackage{titlesec}%customize titles
\usepackage{marginnote}

\usepackage[amsmath,amsthm,thmmarks,hyperref]{ntheorem}%customize theorem-environments more effectively
\usepackage[ntheorem,framemethod=TikZ]{mdframed}

\usepackage[unicode,bookmarks=true]{hyperref}
\hypersetup{
	colorlinks,
	citecolor=green,
	filecolor=green,
	linkcolor=blue,
	urlcolor=green
}
\usepackage{cleveref}
\usepackage{bookmark}

\newcommand{\coloredRule}[3][black]{\textcolor{#1}{\rule{#2}{#3}}}
\newlength{\blacktrianglewidth}
\settowidth{\blacktrianglewidth}{$\blacktriangleright$}

\definecolor{lightgrey}{gray}{0.91}
\definecolor{lightred}{rgb}{1,0.6,0.6}
\definecolor{darkgrey}{gray}{0.6}
\definecolor{darkgreen}{rgb}{0,0.6,0}

%numbered theorems
\theoremstyle{break}
\theorembodyfont{}

\mdfdefinestyle{boxedtheorem}{%
	outerlinewidth=3pt,%
	skipabove=5pt,%
	skipbelow=10pt,%
	frametitlefont=\normalfont\bfseries\color{black},%
	nobreak,%enforce no pagebrakes in the whole frame
}

\newmdtheoremenv[%
	style=boxedtheorem,%
	innertopmargin=\topskip,%
	innerbottommargin=\topskip,%
	linecolor=darkgrey,%
	backgroundcolor=lightgrey,%
]{theorem}{Theorem}[section]

\newmdtheoremenv[%
	style=boxedtheorem,%
	linecolor=darkgrey,%
	topline=false,%
	rightline=false,%
	bottomline=false,%
	innertopmargin=\topskip,%
	innerbottommargin=\topskip,%
	backgroundcolor=lightgrey,%
]{proposition}[theorem]{Satz}

\newmdtheoremenv[%
	style=boxedtheorem,%
	linecolor=darkgrey,%
	topline=false,%
	rightline=false,%
	bottomline=false,%
	backgroundcolor=lightgrey,%
	innertopmargin=\topskip,%
	innerbottommargin=\topskip,%
]{lemma}[theorem]{Lemma}

\newmdtheoremenv[%
	style=boxedtheorem,%
	linecolor=red,%
	topline=false,%
	rightline=false,%
	bottomline=false,%
	innertopmargin=0,%
	innerbottommargin=-3pt,%
]{definition}[theorem]{Definition}

\newmdtheoremenv[%
	outerlinewidth=3pt,%
	linecolor=black,%
	topline=false,%
	rightline=false,%
	bottomline=false,%
	innertopmargin=0pt,%
	innerbottommargin=-0pt,%
	frametitlefont=\normalfont\bfseries\color{black},%
	skipabove=5pt,%	
	skipbelow=10pt,%
]{conclusion}[theorem]{Folgerung}

\newmdtheoremenv[%
	hidealllines=true,%
	frametitlefont=\normalfont\bfseries\color{black},%
	innerleftmargin=0pt,%
	skipabove=5pt,%
	innerleftmargin=10pt,%
]{remark}[theorem]{\hspace*{-10pt}$\blacktriangleright$\hspace*{\dimexpr 10pt - \blacktrianglewidth\relax}Bemerkung}

\newmdtheoremenv[%
	hidealllines=true,%
	frametitlefont=\normalfont\bfseries\color{black},%
	innerleftmargin=10pt,%
]{example}[theorem]{\hspace*{-10pt}\rule{5pt}{5pt}\hspace*{5pt}Beispiel}

%unnumbered theorems
\theoremstyle{nonumberbreak}
\theoremindent0cm
\newmdtheoremenv[%
	style=boxedtheorem,%
	linecolor=red,%
	topline=false,%
	rightline=false,%
	bottomline=false,%
	innertopmargin=1pt,%
	innerbottommargin=1pt,%
]{*definition}{Definition}

\newmdtheoremenv[%
	hidealllines=true,%
	frametitlefont=\normalfont\bfseries\color{black},%
	skipabove=5pt,%
	innerleftmargin=10pt,%
]{*remark}{\hspace*{-10pt}$\blacktriangleright$\hspace*{\dimexpr 10pt - \blacktrianglewidth\relax}Bemerkung}

\newmdtheoremenv[%
	hidealllines=true,%
	innerleftmargin=10pt,%
]{*example}{\hspace*{-10pt}\rule{5pt}{5pt}\hspace*{5pt}Beispiel}
\newtheorem{overview}[theorem]{Überblick}

\newmdtheoremenv[%
	style=boxedtheorem,%
	topline=false,%
	rightline=false,%
	leftline=false,
	bottomline=false,%
	innertopmargin=\topskip,%
	innerbottommargin=\topskip,%
	backgroundcolor=lightgrey,%
]{*anmerkung}{Anmerkung}

%Hinweis-Theoremstyle and environment
%To get rid of the parentheses, a new theorem style is neccessary (definition of nonumberbreak from ntheorem.sty)
%to achieve the underlining, this needed to put in the theoremstyle definition
\theoremheaderfont{\mdseries}
\theoremseparator{:}
\theorempostskip{0pt}
\makeatletter
\newtheoremstyle{noparentheses}%
	{\item[\rlap{\vbox{\hbox{\hskip\labelsep \theorem@headerfont
					\underline{##1}\theorem@separator}\hbox{\strut}}}]}%
	{\item[\rlap{\vbox{\hbox{\hskip\labelsep \theorem@headerfont
					\underline{##1\ ##3\theorem@separator}}\hbox{\strut}}}]}
\newtheoremstyle{underlinedPlain}%
	{\item[\hskip\labelsep \uline{\theorem@headerfont ##1\theorem@separator}]}%
	{\item[\hskip\labelsep \uline{\theorem@headerfont ##1\ \theorem@headerfont(##3)\theorem@separator}]}
\newtheoremstyle{underlinedEnvironment}{}%
{\item[\hskip\labelsep \uline{##1\theorem@headerfont ##3\theorem@separator}]}
\newtheoremstyle{boldEnvironment}{}%
{\item[\hskip\labelsep \textbf{##1\theorem@headerfont ##3\theorem@separator}]}
\newtheoremstyle{proofstyle}%
{\item[\hskip\labelsep {\theorem@headerfont ##1}\theorem@separator]}%
{\item[\hskip\labelsep {\theorem@headerfont ##1}\ (##3)\theorem@separator]}
\makeatother

\theoremstyle{noparentheses}
\newmdtheoremenv[%
	hidealllines=true,%
	innerleftmargin=1em,%
	innerbottommargin=0pt,%
	innerrightmargin=0,%
	skipbelow=0pt,%
]{interpretation}{\hspace*{\dimexpr - \mdflength{innerleftmargin}\relax}Interpretation}
\theoremstyle{underlinedPlain}
\newmdtheoremenv[%
	hidealllines=true,%
	innerleftmargin=1em,%
	innerrightmargin=0,%
	skipbelow=0pt,%
]{hint}{\hspace*{\dimexpr - \mdflength{innerleftmargin}\relax}Hinweis}

\theoremstyle{underlinedEnvironment}
\newmdtheoremenv[%
	hidealllines=true,%
	innerleftmargin=1em,%
	innerrightmargin=0,%
	skipbelow=0pt,%
]{underlinedenvironment}{\hspace*{\dimexpr -\mdflength{innerleftmargin}\relax}}
\theoremheaderfont{\bfseries}
\theoremstyle{boldEnvironment}
\newmdtheoremenv[%
	hidealllines=true,%
	innerleftmargin=1em,%
	innerrightmargin=0,%
	skipbelow=0pt,%
]{boldenvironment}{\hspace*{\dimexpr -\mdflength{innerleftmargin}\relax}}

\theoremstyle{proofstyle}
\theoremheaderfont{\normalfont\normalsize\itshape}
\theorembodyfont{\normalfont\small}
\theoremseparator{.}
\theorempreskip{5pt}
\theorempostskip{5pt}
\theoremsymbol{$\square$}
\renewtheorem{proof}{Beweis}

%for \cref: printed environment names
\crefname{theorem}{Theorem}{Theoreme}
\crefname{proposition}{Satz}{Sätze}
\crefname{lemma}{Lemma}{Lemmata}
\crefname{conclusion}{Folgerung}{Folgerungen}
\crefname{definition}{Definition}{Definitionen}
\crefname{remark}{Bemerkung}{Bemerkungen}
\crefname{example}{Beispiel}{Beispiele}
\crefname{*definition}{Definition}{Definitionen}
\crefname{*remark}{Bemerkung}{Bemerkungen}
\crefname{*example}{Beispiel}{Beispiele}

\makeatletter
%output a number in upper roman letters
\newcommand*{\rom}[1]{\expandafter\@slowromancap\romannumeral #1@}
%declare a new label; store current chapter number
\newcommand*{\proplbl}[1]{%
	\@bsphack
	\begingroup
	\label{#1}%
	\zref@setcurrent{default}{\arabic{chapter}}%
%		\zref@wrapper@immediate{%
		\zref@labelbyprops{#1@chapter}{default}
%		}
	\endgroup
	\@esphack
}

%refer to a label set by proplbl.
%If the label is not defined (yet), question marks are output at the calling position. If the label is defined, the chapter number is prepended to the link output by \cref if the current chapter number and the one set when calling \proplbl differ.
%the macro handels both text and math mode. mbox is needed due to a feature concerning ulem / cleveref
\newcommand*{\propref}[1]{%
	\ifcsdef{r@#1}%in first compilation the label may not be defined yet
	{%
		\zref@refused{#1@chapter}%
		\ifnumcomp{\value{chapter}}{=}{\zref@extractdefault{#1@chapter}{default}{0}}%
		{%same chapter
			\ifmmode 
				\cref{#1}%
			\else
				\mbox{\cref{#1}}%
			\fi
		}%
		{%otherwise
			\def\propositionref@current@type{}%
			\cref@gettype{#1}{\propositionref@current@type}%get the environment's name
			%example for following line:
			%\crefformat{truetheorem}{\cref@truetheorem@name~##2\rom{\zref@extractdefault{#1}{#1chapter}{1}}.##1##3}
			%this changes the format used by \cref to <environtment name> <chapter-number>.<section-number>.<theorem number>
			\crefformat{\propositionref@current@type}{%
				\csname cref@\propositionref@current@type @name\endcsname ~##2\rom{\zref@extractdefault{#1@chapter}{default}{1}}.##1##3%
			}%
			\ifmmode 
				\cref{#1}%
			\else
				\mbox{\cref{#1}}%
			\fi
			\crefformat{\propositionref@current@type}{%
				\csname cref@\propositionref@current@type @name\endcsname~##2##1##3%
			}%
		}%
	}%
	{??}%similar to \ref\cref: question marks in case of undefined labels
}
\makeatother

%declare new term to the index, output if no star is given to call position
\NewDocumentCommand{\begriff}{s O{} m O{}}{%
	\IfBooleanTF{#1}%
	{\index{#2#3#4}}%
	{%
		\uline{#3}%
		\index{#2#3#4}%
	}%
}

%append a new mathsymbol to the index, output if no star is given at the call position
\NewDocumentCommand{\mathsymbol}{s O{} m m O{}}{%
	\IfBooleanTF{#1}%
	{\index[symbols]{#2#3@\detokenize{#4}#5}}%
	{#4\index[symbols]{#2#3@\detokenize{#4}#5}}%
}

%remove skip before / after amsmath-environments: default to 0pt. 1star: just before the environment, 2stars: just after the environment, no star: both
\NewDocumentCommand{\zeroAmsmathAlignVSpaces}{s s O{0 pt} O{0 pt}}{%
	\IfBooleanTF{#1}%
	{%
		\IfBooleanTF{#2}%
			{\setlength{\belowdisplayskip}{#4}}%
			{\setlength{\abovedisplayskip}{#3}}%
	}%
	{%
		\setlength{\abovedisplayskip}{#3}%
		\setlength{\belowdisplayskip}{#4}%
	}%
}

%general transpose-makro
\NewDocumentCommand{\transpose}{m}{\ensuremath{#1^\mathsf{T}}}

%unused
\NewDocumentCommand{\itemEq}{s m}{%
	\begingroup%
	\setlength{\abovedisplayskip}{\dimexpr -\parskip + 1pt\relax}%
	\setlength{\belowdisplayskip}{0pt}%
	\IfBooleanTF{#1}%
		{\parbox[c]{\linewidth}{\begin{flalign*}#2&&\end{flalign*}}}%}
		{\parbox[c]{\linewidth}{\begin{flalign}#2&&\end{flalign}}}%}
	\endgroup% 
}

%new macro for "equals" ^=
\newcommand\equalhat{\mathrel{\stackon[1.5pt]{=}{\stretchto{%
	\scalerel*[\widthof{=}]{\wedge}{\rule{1ex}{3ex}}}{0.5ex}}}}

%macro that defines the spacing between bracket and content of a matrix
\NewDocumentCommand{\matrixBracketSpacing}{}{\mspace{4.0mu plus 3.0mu minus 1.0mu}}
%macro width customized spacing between bracktes / content, lineheight and columnwidth
\newenvironment{henrysmatrix}{%
	\renewcommand*{\arraystretch}{1.2}
	\setlength\arraycolsep{5pt}
	\left(\matrixBracketSpacing
	\begin{matrix}
}{%
	\end{matrix}
	\matrixBracketSpacing\right)
}

\makeatletter
%redefine \overline to customize the space between text / line (currently 0.4mm + height of the content)
%ATTENTION: when changing the 0.4mm unfortunately, in \kringel the 0.4mm need to be changed accordingly
\let\@old@overline\overline
\renewcommand*{\overline}[1]{%
	\@old@overline{\raisebox{0pt}[\dimexpr\height+0.4mm\relax]{$#1$}}%
}

%encircle some content. Arguments: border color (optional), background color (mandatory), content (mandatory)
%two lengths to get width / height of content (important for width / height of the circle)
\newlength{\@kringel@contentheight}
\newlength{\@kringel@contentwidth}
\newlength{\@kringel@depth}
\NewDocumentCommand{\kringel}{O{blue} m m}{%
%as the macro should work for both text and math mode, add some macros for later use to distinguish
%in text mode, nothing happens (except discarding the 1st argument for the raisebox, that is permantently given), in math mode, the content needs to be enbraced by \ensuremath, the tcolorbox-environment by a raisebox
%ATTENTION: when changing the height-factor of tcolorbox, the depth correction needs to be changed as well
	\let\@kringel@inner\relax
	\let\@kringel@outer\@secondoftwo
	\ifmmode
		\let\@kringel@inner\ensuremath
		\let\@kringel@outer\raisebox
	\fi
%set the width and height
	\settoheight{\@kringel@contentheight}{\hbox{\@kringel@inner{#3}}}
	\settowidth{\@kringel@contentwidth}{\@kringel@inner{#3}}
	\settodepth{\@kringel@depth}{\@kringel@inner{#3}}
%change the depth correction dependend whethere there is a depth (e.g. y) or not (e.g. a)
	\ifdim \@kringel@depth > 0pt%
		\setlength{\@kringel@depth}{\dimexpr\@kringel@depth+0.5mm\relax}
	\else
		\settodepth{\@kringel@depth}{y}
		\setlength{\@kringel@depth}{\dimexpr\@kringel@depth+0.3mm\relax}
	\fi
%output the colorbox width given parameter: frame color, background color, computed width and height, and escaped content depending on math / text mode
	\@kringel@outer{\dimexpr-\@kringel@contentheight/2-\@kringel@depth\relax}{\begin{tcolorbox}[colframe=#1,halign=center,valign=center,width=\dimexpr1.5\@kringel@contentwidth+1mm\relax,height=2.5\@kringel@contentheight,left=0pt,right=0pt,bottom=0pt,top=0pt,boxrule=0.8pt,colback=#2,boxsep=0pt,bean arc]
		\@kringel@inner{#3}
	\end{tcolorbox}}
}

%switch numbering of equations (amsmath-environments)
\newcommand{\leqnos}{\tagsleft@true\let\veqno\@@leqno}
\newcommand{\reqnos}{\tagsleft@false\let\veqno\@@eqno}
\reqnos

\pdfstringdefDisableCommands{%
	\def\\{}%
	\def\texttt#1{<#1>}%
	\def\mathbb#1{<#1>}%
}
\makeatother

%General newcommands!
\newcommand{\comp}{\mathbb{C}} % complex set C
\newcommand{\real}{\mathbb{R}} % real set R
\newcommand{\whole}{\mathbb{Z}} % whole number Symbol
\newcommand{\natur}{\mathbb{N}} % natural number Symbol
\newcommand{\ratio}{\mathbb{Q}} % rational number symbol
\newcommand{\field}{\mathbb{K}} % general field for the others above!
\newcommand{\diff}{\mathrm{d}} % differential d
\newcommand{\s}{\,\,}     % space after the function in the intergral
\newcommand{\cont}{\mathcal{C}} % Contour C
\newcommand{\fuk}{f(z) \s\diff z} % f(z) dz
\newcommand{\diffz}{\s\diff z}
\newcommand{\subint}{\int\limits} % lower boundaries for the integral
\newcommand{\poly}{\mathcal{P}} % special P - polygon
\newcommand{\defi}{\mathcal{D}} % D for the domain of a function
\newcommand{\cover}{\mathcal{U}} % cover for a set
\newcommand{\setsys}{\mathcal{M}} % set system M
\newcommand{\setnys}{\mathcal{N}} % set system N
\newcommand{\zetafunk}{f(\zeta)\s\diff \zeta} %f(zeta) d zeta
\newcommand{\ztfunk}{f(\zeta)} % f(zeta)
\newcommand{\bocirc}{S_r(z)}
\newcommand{\prop}{\,|\,}
\newcommand*{\QEDA}{\hfill\ensuremath{\blacksquare}} %tombstone
\newcommand{\emptybra}{\{\varnothing\}} % empty set with set-bracket
\newcommand{\realpos}{\real_{>0}}
\newcommand{\realposr}{\real_{\geq0}}
\newcommand{\naturpos}{\natur_{>0}}
\newcommand{\Imag}{\operatorname{Im}} % Imaginary symbol
\newcommand{\Realz}{\operatorname{Re}} % Real symbol
\newcommand{\norm}{\Vert \cdot \Vert}
\newcommand{\metric}{\vert \cdot \vert}
\newcommand{\foralln}{\forall n} %all n
\newcommand{\forallnset}{\forall n \in \natur} %all n € |N
\newcommand{\forallnz}{\forall n \geq _0} % all n >= n_0
\newcommand{\conjz}{\overline{z}} % conjugated z
\newcommand{\tildz}{\tilde{z}} % different z
\newcommand{\lproofar}{"`$ \Leftarrow $"'} % "`<="'
\newcommand{\rproofar}{"`$ \Rightarrow $"'} % "`=>"'
\newcommand{\beha}{\Rightarrow \text{ Behauptung}}
\newcommand{\powerset}{\mathcal{P}}
\newcommand{\person}[1]{\textsc{#1}}
\newcommand{\highlight}[1]{\emph{#1}}
\newcommand{\realz}{\mathfrak{Re}}
\newcommand{\imagz}{\mathfrak{Im}}
\renewcommand{\epsilon}{\varepsilon}
\renewcommand{\phi}{\varphi}
\newcommand{\lebesque}{\person{Lebesgue}}
\renewcommand{\Re}{\mathfrak{Re}}
\renewcommand{\Im}{\mathfrak{Im}}
\renewcommand*{\arraystretch}{1.4}

% Math Operators
\DeclareMathOperator{\inn}{int} % Set of inner points
\DeclareMathOperator{\ext}{ext} % Set of outer points
\DeclareMathOperator{\cl}{cl} % Closure
\DeclareMathOperator{\grad}{grad}
\DeclareMathOperator{\D}{d}
\DeclareMathOperator{\id}{id}
\DeclareMathOperator{\graph}{graph}
\DeclareMathOperator{\Int}{int}
\DeclareMathOperator{\Ext}{ext}
\DeclareMathOperator{\diam}{diam}

\DeclareMathOperator{\End}{End}
\DeclareMathOperator{\Aut}{Aut}
\DeclareMathOperator{\Hom}{Hom}
\DeclareMathOperator{\Eig}{Eig}
\DeclareMathOperator{\Mat}{Mat}
\DeclareMathOperator{\Ker}{Ker}
\DeclareMathOperator{\diag}{diag}
\DeclareMathOperator{\GL}{GL}
\DeclareMathOperator{\tr}{tr}
\DeclareMathOperator{\sgn}{sgn}
\DeclareMathOperator{\Span}{span}
\DeclareMathOperator{\Image}{Im}
\DeclareMathOperator{\Hau}{Hau}

%change headings:
\titlelabel{\thetitle.\quad}%. behind section/sub... (3. instead of 3)
\counterwithout{section}{chapter}
\renewcommand{\thechapter}{\Roman{chapter}}
\renewcommand{\thepart}{\Alph{part}}
%italic chapters (due to titlesec package some more stuff)
%\titleformat{command}[shape]{format}{label}{sep}{before-code}[after-code]
\titleformat{\chapter}[display]{\bfseries}{\Large\chaptername\;\thechapter}{-5pt}{\huge\bfseries\itshape}
\titlespacing{\chapter}{0pt}{0pt}{10pt}
\titleformat{\section}[hang]{\bfseries\Large}{\thesection.}{8pt}{\Large\bfseries}
%\titlespacing{command}{left}{before-sep}{after-sep}
\titlespacing{\subsection}{0pt}{0pt}{5pt}

%change appearence of heading of toc: 0 space above, bold, italic huge toc-heading
\renewcommand{\cftbeforetoctitleskip}{0pt}
\renewcommand{\cfttoctitlefont}{\itshape\Huge\bfseries}
%change indentations due to width of capital roman numbers
\renewcommand{\cftchapnumwidth}{2.5em}
\renewcommand{\cftsecindent}{2.5em}
%\renewcommand{\cftsecnumwidth}{3.3em}
\renewcommand{\cftsubsecindent}{4.8em}
%\renewcommand{\cftsubsecnumwidth}{4.2em}

%change header:
\renewcommand{\headrulewidth}{0.75pt}
\renewcommand{\footrulewidth}{0.3pt}
\lhead{\rightmark}%left: section-number. section-title
\rhead{\leftmark}%right: chapter chapternumber: chapter-title

% Add new page-style (just footer), patch \chapter command to use this page style
\fancypagestyle{plainChapter}{%
	\fancyhf{}%
	\fancyfoot[C]{\thepage}%
	\renewcommand{\headrulewidth}{0pt}% Line at the header invisible
	\renewcommand{\footrulewidth}{0.4pt}% Line at the footer visible
}
%changes pagestyle; instead of empty page the normal footer is printed
\patchcmd{\chapter}{\thispagestyle{plain}}{\thispagestyle{plainChapter}}{}{}
%usually, after a new chapter the section counter needs to be reset manually. Instead, automatic reset
\pretocmd{\chapter}{\setcounter{section}{0}}{}{}

\pagestyle{fancy}
\pagenumbering{arabic}
%remember chapter-title in \leftmark and \rightmark
\renewcommand{\chaptermark}[1]{%
	\markboth{\chaptername
		\ \thechapter:\ #1}{}}
%remember section title in \leftmark
\renewcommand{\sectionmark}[1]{%
	\markright{\thesection.\ #1}{}}

%change numbering of equations to be section by section
\counterwithout{equation}{section}

\title{\textbf{Lineare Algebra SS2018}}
\author{Dozent: Prof. Dr. Arno Fehm}

%remove page number from part{}-pages
\makeatletter
\let\sv@endpart\@endpart
\def\@endpart{\thispagestyle{empty}\sv@endpart}
\makeatother

\begin{document}
\pagenumbering{roman}
\pagestyle{plain}

\maketitle

\hypertarget{tocpage}{}
\tableofcontents
\bookmark[dest=tocpage,level=1]{Inhaltsverzeichnis}

\pagebreak
\pagestyle{fancy}
\pagenumbering{arabic}
\pagestyle{fancy}

\chapter{Endomorphismen}
In diesem Kapitel seien $K$ ein Körper, $n\in\natur$ eine natürliche Zahl, $V$ ein $n$-dimensionaler $K$-Vektorraum und $f\in\End_K(V)$ ein Endomorphismus.

Das Ziel dieses Kapitels ist, die Geometrie von $f$ besser zu verstehen und Basen zu finden, für die $M_B(f)$ eine besonders einfache oder kanonische Form hat.

\section{Eigenwerte}

\begin{remark}
	Wir erinnern uns daran, dass $\End_K(V)=\Hom_K(V,V)$ sowohl einen $K$-Vektorraum als auch einen Ring bildet. Bei der Wahl einer Basis $B$ von $V$ wird $f\in\End_K(V)$ durch die Matrix $M_B(f)=M_B^B(f)$ beschrieben.	
\end{remark}

\begin{example}
	$K=\real, A=\begin{henrysmatrix}1&2\\2&1\end{henrysmatrix}\in\Mat_2(\real),f=f_A\in\End_K(K^2)$ \\
	\begin{align}
		A\cdot \begin{henrysmatrix}1\\1\end{henrysmatrix}=\begin{henrysmatrix}3\\3\end{henrysmatrix},\;A\cdot\begin{henrysmatrix} 1\\-1\end{henrysmatrix}=\begin{henrysmatrix}-1\\1\end{henrysmatrix}\notag
	\end{align}
	$\Rightarrow$ mit $B=\left( \begin{henrysmatrix}1\\1\end{henrysmatrix},\begin{henrysmatrix}1\\-1\end{henrysmatrix}\right)$ ist $M_B(f)=\begin{henrysmatrix}3&0\\0&-1\end{henrysmatrix}$. \\
	Der Endomorphismus $f=f_A$ streckt also entlang der Achse $\real\cdot \begin{henrysmatrix}1\\1\end{henrysmatrix}$ um den Faktor 3 und spiegelt entlang der Achse $\real\cdot \begin{henrysmatrix}1\\-1\end{henrysmatrix}$
	\begin{center}
		\begin{tikzpicture}
		\draw[->,thick] (-2,0) -- (2,0) node[right] {$x_1$};
		\draw[->,thick] (0,-2) -- (0,2) node[above] {$x_2$};
		\draw[->, thin] (0,0) -- (1.414,1.414);
		\draw[->, thin] (0,0) -- (-1.414,-1.414);
		\draw[->, thin] (0,0) -- (-0.707,0.707);
		\draw[->, thin] (0,0) -- (0.707,-0.707);
		\draw[dashed, rotate=+45, blue] (0,0) ellipse (2cm and 1cm);
		\draw (0,0) circle (1);
		\end{tikzpicture}
	\end{center}
\end{example}

\begin{definition}[Eigenwert, Eigenvektor, Eigenraum]
	Sind $0\neq x\in V$ und $\lambda\in K$ mit $f(x)=\lambda x$ so nennt man $\lambda$ einen \begriff{Eigenwert} von $f$ und $x$ einen \begriff{Eigenvektor} von $f$ zum Eigenwert $\lambda$. Der \begriff{Eigenraum} zu $\lambda\in K$ ist $\Eig (f,\lambda)=\{x\in V\mid f(x)=\lambda x\}$.
\end{definition}

\begin{remark}
	Für jedes $\lambda\in K$ ist $\Eig (f,\lambda)$ ein Untervektorraum von $V$, da
	\begin{align}
		\Eig (f,\lambda) &= \{x\in V\mid f(x)=\lambda x\} \notag \\
		&= \{x\in V\mid f(x)-\lambda\cdot\id_V(x)=0\} \notag \\
		&= \{x\in V\mid (f-\lambda\cdot\id_V)(x)=0\} \notag \\
		&= \Ker (f-\lambda\cdot\id_V) \notag
	\end{align}
	und $f-\lambda\cdot\id_V\in\End_K(V)$.
\end{remark}

\begin{remark}
	Achtung! Der Nullvektor ist nach Definition kein Eigenvektor, aber $\lambda=0$ kann ein Eigenwert sein, nämlich genau dann, wenn $f\notin\Aut_K(V)$, siehe Übung. Die Menge der Eigenvektoren zu $\lambda$ ist also $\Eig (f,\lambda)\backslash\{0\}$ und $\lambda$ ist genau dann ein Eigenwert von $f$, wenn $\Eig (f,\lambda)\neq\{0\}$.
\end{remark}

\begin{example}
	Ist $A=\diag(\lambda_1,...,\lambda_n)$ und $f=f_A\in\End_K(K^n)$, so sind $\lambda_1,...,\lambda_n$ Eigenwerte von $f$ und jedes $e_i$ ist ein Eigenvektor zum Eigenwert $\lambda_i$.
\end{example}

\begin{proposition}
	\proplbl{satz_diagonal_ev}
	Sei $B$ eine Basis von $V$. Genau dann ist $M_B(f)$ eine Diagonalmatrix, wenn $B$ aus Eigenvektoren von $f$ besteht.
\end{proposition}
\begin{proof}
	Ist $B=(x_1,...x_n)$ eine Basis aus Eigenvektoren zu Eigenwerten $\lambda_1,....,\lambda_n$, so ist $M_B(f)= \diag(\lambda_1,...,\lambda_n)$ und umgekehrt.
\end{proof}

\begin{example}
	Sei $K=\real$, $V=\real^2$ und $f_{\alpha}\in\End_K(\real^2)$ die Drehung um den Winkel $\alpha\in [0,2\pi)$ \\
	\[\Rightarrow M_{\mathcal{E}}(f_{\alpha})=\begin{pmatrix}\cos(\alpha)&-\sin(\alpha) \\ \sin(\alpha) & \cos(\alpha)\end{pmatrix}\]
	Für $\alpha=0$ hat $f_{\alpha}=\id_{\real^2}$ nur den Eigenwert 1. \\
	Für $\alpha=\pi$ hat $f_{\alpha}=-\id_{\real^2}$ nur den Eigenwert -1. \\
	Für $\alpha\neq 0,\pi$ hat $f_{\alpha}$ keine Eigenwerte. %TODO figure
\end{example}

\begin{lemma}
	\proplbl{lemma_EW_lin_unabh}
	Sind $\lambda_1,...,\lambda_n$ paarweise verschiedene Eigenwerte von $f$ und ist $x_i$ ein Eigenvektor zu $\lambda_i$ für $i=1,...,m$, so ist $(x_1,...,x_m)$ linear unabhängig.
\end{lemma}
\begin{proof}
	Induktion nach $m$\\
	\emph{$m=1$}: klar, denn $x_1\neq 0$ \\
	\emph{$m-1\to m$}: Sei $\sum_{i=1}^m \mu_i x_i=0$ mit $\mu_1,...,\mu_m\in K$.
	\begin{align}
		0&= (f-\lambda\cdot\id_V)\left( \sum\limits_{i=1}^m \mu_i x_i\right) \notag \\
		&= \sum\limits_{i=1}^m \mu_i(f(x_i)-\lambda_m\cdot x_i) \notag \\
		&= \sum\limits_{i=1}^{m-1} \mu_i(\lambda_i-\lambda_m)\cdot x_i \notag
	\end{align} 
	Nach IB ist $\mu_i(\lambda_i-\lambda_m)=0$ für $i=1,...,m-1$, da $\lambda_i\neq\lambda_m$ für $i\neq m$ also $\mu_i=0$ für $i=1,...,m-1$. Damit ist auch $\mu_m=0$. Folglich ist $(x_1,...,x_m)$ linear unabhängig.
\end{proof}

\begin{proposition}
	\proplbl{satz_eig_direkte_summe}
	Sind $\lambda_1,...,\lambda_m\in K$ paarweise verschieden, so ist 
	\[\sum\limits_{i=1}^m \Eig(f,\lambda_i)=\bigoplus_{i=0}^{m}\Eig(f,\lambda_i).\]
\end{proposition}
\begin{proof}
	Seien $x_i,y_i\in\Eig(f,\lambda_i)$ für $i=1,...,m$. Ist $\sum_{i=1}^m x_i=\sum_{i=1}^m y_i$, so ist $\sum_{i=1}^m \underbrace{x_i-y_i}_{z_i}=0$.\\
	o. E. seien $z_i\neq 0$ für $i=1,...,r$ und $z_i=0$ für $i=r+1,...,m$. Wäre $r>0$, so wären $(z_1,...,z_r)$ linear abhängig, aber $z_i=x_i-y_i\in\Eig(f,\lambda_i)\backslash\{0\}$, im Widerspruch zu \propref{lemma_EW_lin_unabh}. Somit ist $x_i=y_i$ für alle $i$ und folglich ist die Summe $\sum\Eig(f,\lambda_i)$ direkt.
\end{proof}

\begin{definition}[Eigenwerte und Eigenvektoren für Matrizen]
	Sei $A\in\Mat_n(K)$. Man definiert Eigenwerte, Eigenvektoren, etc. von $A$ als Eigenwerte, Eigenvektoren von $f_A\in\End_K(K^n)$.
\end{definition}

\begin{mathematica}[Eigenwerte und Eigenvektoren]
	Um die Eigenwerte und Eigenvektoren einer Matrix $A$ zu berechnen, gibt es in Mathematica bzw. WolframAlpha verschiedene Möglichkeiten:
	\begin{itemize}
		\item \texttt{Eigenvalues[A]}: liefert eine Liste der Eigenwerte
		\item \texttt{Eigenvectors[A]}: liefert eine Liste der Eigenvektoren
		\item \texttt{Eigensystem[A]}: liefert zu jeden Eigenwert den Eigenvektor
	\end{itemize}
\end{mathematica}

\begin{proposition}
	Sei $B$ eine Basis von $V$ und $\lambda\in K$. Genau dann ist $\lambda$ ein Eigenvektor von $f$, wenn $\lambda$ ein Eigenwert von $A=M_B(f)$ ist. Insbesondere haben ähnliche Matrizen die selben Eigenwerte.
\end{proposition}
\begin{proof}
	Dies folgt aus dem kommutativen Diagramm
	\begin{center}\begin{tikzpicture}
		\matrix (m) [matrix of math nodes,row sep=3em,column sep=4em,minimum width=2em]
		{K^n & K^n \\ V & V \\};
		\path[-stealth]
		(m-1-1) edge node [left] {$\Phi_B$} (m-2-1)
		edge node [above] {$f_A$} (m-1-2)
		(m-2-1) edge node [below] {$f$} (m-2-2)
		(m-1-2) edge node [right] {$\Phi_B$} (m-2-2);
		\end{tikzpicture}\end{center}
	denn $f_A(x)=\lambda x\iff (\Phi_B\circ f_A)(x)=\Phi_B(\lambda x)\iff f(\Phi_B(x))=\lambda\Phi_B(x)$. \\
	Ähnliche Matrizen beschreiben den selben Endomorphismus bezüglich verschiedener Basen, vgl. \propref{4_4_1}
\end{proof}
\section{Das charakteristische Polynom}

\begin{proposition}
	\proplbl{satz_det_null}
	Sei $\lambda\in K$. Genau dann ist $\lambda$ ein EW von $f$, wenn $\det(\lambda\cdot\id_V-f)=0$.
\end{proposition}
\begin{proof}
	Da $\Eig(f,\lambda)=\Ker(\lambda\cdot\id_V-f)$ ist $\lambda$ genau dann ein EW von $f$, wenn $\dim_K(\Ker(\lambda\cdot\id_V-f))>0$, also wenn $\lambda\cdot\id_V-f\notin\Aut_K(V)$. Nach IV.4.6 bedeutet dies, dass $\det(\lambda\cdot\id_V-f)=0$ %TODO: Verlinkung setzen
\end{proof}

\begin{definition}[charakteristisches Polynom]
	Das \begriff{charakteristische Polynom} einer Matrix $A\in\Mat_n(K)$ ist die Determinante der Matrix $t\cdot \mathbbm{1}_n-A\in\Mat_n(K[t])$. 
	\begin{align}
		\chi_A(t)&=\det(t\cdot \mathbbm{1}_n-A)\in K[t] \notag
	\end{align}
	Das charakteristische Polynom eines Endomorphismus $f\in\End_K(V)$ ist $\chi_f(t)=\chi_{M_B(f)}(t)$, wobei $B$ eine Basis von $V$ ist.
\end{definition}

\begin{proposition}
	\proplbl{satz_2_3}
	Sind $A,B\in\Mat_n(K)$ mit $A\sim B$, so ist $\chi_A=\chi_B$. Insbesondere ist $\chi_f$ wohldefiniert.
\end{proposition}
\begin{proof}
	Ist $B=SAS^{-1}$ mit $S\in\GL_n(K)$, so ist $t\cdot \mathbbm{1}_n-B = S(t\cdot \mathbbm{1}_n-A)S^{-1}$, also $t\cdot \mathbbm{1}_n-B\sim t\cdot \mathbbm{1}_n-A$ und ähnliche Matrizen haben die selben Determinante (IV.4.4). \\
	Sind $B,B'$ Basen von $V$, so sind $M_B(f)\sim M_{B'}(f)$, also $\chi_{M_B(f)}=\chi_{M_{B'}(f)}$ %TODO: Verlinkung setzen
\end{proof}

\begin{lemma}
	\proplbl{lemma_chi_det}
	Für $\lambda\in K$ ist $\chi_f(\lambda)=\det(\lambda\cdot\id_V-f)$.
\end{lemma}
\begin{proof}
	Sei $B$ eine Basis von $V$ und $A=M_B(f)=(a_{ij})_{i,j}$. Dann ist $M_B(\lambda\cdot\id_V-f)= \lambda\cdot \mathbbm{1}_n-A$. Aus IV.2.8 und I.6.8 folgt $\det(t\cdot \mathbbm{1}_n-A)(\lambda)=\det(\lambda\cdot \mathbbm{1}_n-A)$. Folglich ist 
	\begin{align}
		\chi_f(\lambda)&=\chi_A(\lambda)\notag \\
		&=\det(t\cdot \mathbbm{1}_n-A)(\lambda)\notag \\
		&=\det(\lambda\cdot \mathbbm{1}_n-A)\notag \\
		&= \det(\lambda\cdot\id_V-f) \notag
	\end{align}
\end{proof}

\begin{proposition}
	\proplbl{satz_chi_polynom}
	Sei $\dim_K(V)=n$ und $f\in\End_K(V)$. Dann ist $\chi_f(t)=\sum_{i=0}^n \alpha_i t^i$ ein Polynom vom Grad $n$ mit 
	\begin{align}
		\alpha_n&=1\notag \\
		\alpha_{n-1}&=-\tr(f) \notag \\
		\alpha_0 &= (-1)^n\cdot\det(f) \notag
	\end{align}
	Die Nullstellen von $\chi_f$ sind genau die EW von $f$.
\end{proposition}
\begin{proof}
	Sei $B$ eine Basis von $V$ und $A=M_B(f)=(a_{ij})_{i,j}$. Wir erinnern uns daran, dass $\tr(f)=\tr(A=\sum_{i=1}^n a_{ii}$. Es ist $\chi_f(t)=\det(t-\cdot 1_n-A)=\sum_{\sigma\in S_n}\sgn(\sigma)\prod_{i=1}^n (t\delta_{i,\sigma(i)}-a_{i,\sigma(i)})$. \\
	Der Summand für \emph{$\sigma=\id$} ist $\prod_{i=1}^n (t-a_{ii})=t^n+\sum_{i=1}^n (-a_{ii})t^{n-1}+...+\prod_{i=1}^n(-a_{ii})$ \\
	Für \emph{$\sigma\neq\id$} ist $\sigma(i)\neq i$ für mindestens zwei $i$, der entsprechende Summand hat also Grad höchstens $n-2$. Somit haben $\alpha_n$ und $\alpha_{n-1}$ die oben behauptete Form, und $\alpha_0=\chi_A(0)=\det(-A)=(-1)^n\cdot\det(f)$. \\
	Die Aussage über die Nullstellen von $\chi_f$ folgt aus \propref{satz_det_null} und \propref{lemma_chi_det}.
\end{proof}

\begin{conclusion}
	Ist $\dim_K(V)=n$, so hat $f$ höchstens $n$ Eigenwerte.
\end{conclusion}
\begin{proof}
	\propref{satz_chi_polynom} und I.6.10 %TODO: Verlinkung
\end{proof}

\begin{definition}[normiertes Polynom]
	Ein Polynom $0\neq P\in K[t]$ mit Leitkoeffizient 1 heißt \begriff{normiert}.
\end{definition}

\begin{example}
	\proplbl{beispiel_2_8}
	\begin{enumerate}
		\item Ist $A=(a_{ij})_{i,j}$ eine obere Dreiecksmatrix, so ist $\chi_A(t)=\prod_{i=1}^n (t-a_{ii})$, vgl. IV.2.9.c \\ %TODO: Verlinkung
		Insbesondere ist $\chi_{1_n}(t)=(t-1)^n$, $\chi_0(t)=t^n$
		\item Für eine Blockmatrix $A=\begin{pmatrix}A_1&B \\ 0&A_2\end{pmatrix}$ mit quadratischen Matrizen $A_1,A_2$ ist $\chi_A=\chi_{A_1}\cdot \chi_{A_2}$ vgl. IV.2.9.e %TODO: Verlinkung
		\item Für
		\begin{align}
			\begin{pmatrix}
			0&...&...&...&0&-c_0  \\ 
			1& \ddots&\;&\;&\vdots&\vdots  \\ 
			0&\ddots&\ddots&\;&\vdots&\vdots  \\ 
			\vdots&\ddots&\ddots&\ddots&\vdots&\vdots  \\ 
			0&...&0&1&0&-c_{n-1} 
			\end{pmatrix} \quad c_0,...,c_{n-1}\in K \notag
		\end{align}
		ist $\chi_A(t)=t^n+\sum_{i=0}^{n-1} c_i t^i$ \\
		Man nennt diese Matrix die Begleitmatrix zum normierten Polynom $P=t^n+\sum_{i=0}^{n-1} c_i t^i$ und schreibt $M_P:=A$
	\end{enumerate}
\end{example}
\section{Diagonalisierbarkeit}

\begin{definition}[diagonalisierbar]
	Man nennt $f$ \begriff{diagonalisierbar}, wenn $V$ eine Basis $B$ besitzt, für die $M_B(f)$ eine Diagonalmatrix ist.
\end{definition}

\begin{lemma}
	\proplbl{lemma_diag_summe_eig}
	Genau dann ist $f$ diagonalisierbar, wenn
	\begin{align}
		V=\sum\limits_{\lambda\in K} \Eig(f,\lambda) \notag
	\end{align}.
\end{lemma}
\begin{proof}
	$(\Rightarrow)$: Ist $B$ eine Basis aus EV von $f$ (vgl. \propref{satz_diagonal_ev}), so ist $B\le \bigcup\limits_{\lambda\in K}\Eig(f,\lambda)$, also $V=\Span_K(\bigcup\limits_{\lambda\in K}\Eig(f, \lambda))=\sum\limits_{\lambda\in K}\Eig(f,\lambda)$. \\
	$(\Leftarrow)$: Ist $V=\sum\limits_{\lambda\in K}\Eig(f,\lambda)$, so gibt es $\lambda_1,...,\lambda_n \in K$ mit $V=\sum\limits_{i=1}^r \Eig(f,\lambda_i)$. Wir wählen Basen $B_i$ von $\Eig(f,\lambda_i)$. Dann ist $\bigcup\limits_{i=1}^r B_i$ ein endliches Erzeugendensystem von $V$, enthält also eine Basis von $V$ (II.3.6). Diese besteht aus EV von $f$. %TODO: Verlinkung
\end{proof}

\begin{proposition}
	Ist $\dim_K(V)=n$, so hat $f$ höchstens $n$ Eigenwerte. Hat $f$ genau $n$ Eigenwerte, so ist $f$ diagonalisierbar.
\end{proposition}
\begin{proof}
	Ist $\lambda$ ein EW von $f$, so ist $\dim_K(\Eig(f,\lambda))\ge 1$. Sind also $\lambda_1,...,\lambda_n$ paarweise verschiedene EW von $f$, so ist
	\begin{align}
		n=\dim_K(V)&\ge \dim_K\left( \sum\limits_{i=1}^m \Eig(f,\lambda_i)\right) \notag \\
		&\overset{\text{\propref{satz_eig_direkte_summe}}}{=} \dim_K\left( \bigoplus_{i=0}^{m} \Eig(f,\lambda_i)\right) \notag \\
		&= \sum\limits_{i=1}^m \dim_K(\Eig(f,\lambda_i)) \notag \\
		&\ge m \notag
	\end{align}
	Ist zudem $m=n$, so muss 
	\begin{align}
		\dim_K(V) &= \dim_K(\sum\limits_{i=1}^m \Eig(f,\lambda_i))\text{ sein, also }\notag \\
		V&= \sum\limits_{i=1}^m \Eig(f,\lambda_i) \notag
	\end{align}
	Nach \propref{lemma_diag_summe_eig} ist $f$ genau dann diagonalisierbar.
\end{proof}

\begin{definition}[$a$ teilt $b$]
	Sei $R$ ein kommutativer Ring mit seien $a,b\in R$. Man sagt, $a$ \begriff{teilt} $b$ (in Zeichen $a\vert b$), wenn es $x\in R$ mit $b=ax$ gibt.
\end{definition}

\begin{definition}[Vielfachheit]
	Für $0\neq P\in K[t]$ und $\lambda\in K$ nennt man $\mu(P,\lambda)=\max\{r\in \natur_{>0}\mid (t-r)^r\vert P\}$ die \begriff{Vielfachheit} der Nullstelle $\lambda$ von $P$.
\end{definition}

\begin{lemma}
	\proplbl{lemma_3_6}
	Genau dann ist $\mu(P,\lambda)\ge 1$, wenn $\lambda$ eine Nullstelle von $P$ ist.
\end{lemma}
\begin{proof}
	$(\Rightarrow)$: $t-\lambda\vert P\Rightarrow P(t)=(t-\lambda)\cdot Q(t)$ mit $Q(t)\in K[t]\Rightarrow P(\lambda)=0\cdot Q(\lambda)=0$. \\
	$(\Leftarrow)$: $P(\lambda)=0\overset{I.6.9}{=}t-\lambda\vert P(t)\Rightarrow \mu(P,\lambda)\ge 1$.
	%TODO: Verlinkung
\end{proof}

\begin{lemma}
	\proplbl{lemma_3_7}
	Ist $P(t)=(t-\lambda)^r\cdot Q(t)$ mit $Q(t)\in K[t]$ und $Q(\lambda)\neq 0$, so ist $\mu(P,\lambda)=r$
\end{lemma}
\begin{proof}
	Offensichtlich ist $\mu(P,\lambda)\ge r$. Wäre $\mu(P,\lambda)\ge r+l$, so $(t-\lambda)^{r+l}\vert P(t)$ also $(t-\lambda)^r\cdot Q(t)=(t-\lambda)^{r^+l}\cdot R(t)$ mit $R(t)\in K[t]$, folglich $t-\lambda\vert Q(t)$, insbesondere $Q(\lambda)=0$. \\
	(Denn wir dürfen kürzen: $R$ ist nullteilerfrei, genau so wie $K[t]$). \\
	$(t-\lambda)^r(Q(t)-(t-\lambda)R(t))=0\Rightarrow Q(t)=(t-\lambda)R(t)$.
\end{proof}

\begin{lemma}
	\proplbl{lemma_3_8}
	Sind $P,Q,R\in K[t]$ mit $PQ=PR$, und ist $P\neq 0$, so ist $Q=R$.
\end{lemma}
\begin{proof}
	$PQ=PR\Rightarrow P(Q-R)=0\overset{K[t]\text{ nullteilerfrei}}{\Rightarrow} Q-R=0$, d.h. $Q=R$.
\end{proof}

\begin{lemma}
	\proplbl{lemma_3_9}
	Es ist $\sum\limits_{\lambda\in K} \mu(P,\lambda)\le \deg(P)$, mit Gleichheit genau dann, wenn $P$ in Linearfaktoren zerfällt.
\end{lemma}
\begin{proof}
	Schreibe $P(t)=\prod\limits_{\lambda\in K}(t-\lambda)^{r_\lambda}\cdot Q(t)$, wobei $Q(t)\in K[t]$ keine Nullstellen mehr besitzt. Nach \propref{lemma_3_7} ist $\mu(P,\lambda)=r_\lambda$ für alle $\lambda$ und somit $\deg(P)=\sum\limits_{\lambda\in K} r_\lambda+\deg(Q)\ge \sum\limits_{\lambda\in K} \mu(P,\lambda)$ mit Gleichheit genau dann,wenn $\deg(Q)=0$, also $Q=c\in K$, d.h. genau dann, wenn $P(t)=c\cdot \prod\limits_{\lambda\in K} (t-\lambda)^{r_\lambda}$.
\end{proof}

\begin{lemma}
	\proplbl{lemma_3_10}
	Für $\lambda\in K$ ist
	\begin{align}
		\dim_K(\Eig(f,\lambda))\ge \mu(x_f,\lambda)\notag
	\end{align}
\end{lemma}
\begin{proof}
	Ergänze eine Basis $B$ von $\Eig(f,\lambda)$ zu einer Basis $B$ von $V$. Dann ist 
	\begin{align}
		A=M_B(f)=\begin{pmatrix}\lambda\mathbbm{1}_s&*\\0&A'\end{pmatrix}\notag
	\end{align}
	mit einer Matrix $A'\in \Mat_{n-s}(K)$, also $\chi_f(t)=\chi_A(t)\overset{\text{\propref{beispiel_2_8}}}{=}\chi_{\lambda\mathbbm{1}}\cdot\chi_{A'}(t)=(t- \lambda)^s\cdot \chi_{A'}(t)$ und somit $\dim_K(\Eig(f,\lambda))=s\le \mu(x_f,\lambda)$.
\end{proof}

\begin{proposition}
	Genau dann ist $f$ diagonalisierbar, wenn $\chi_f$ in Linearfaktoren zerfällt und $\dim_K(\Eig(f,\lambda))=\mu(x_f,\lambda)$ für alle $\lambda\in K$.
\end{proposition}
\begin{proof}
	Es gilt
	\begin{align}
		\dim_K(\sum\limits_{\lambda\in K}\Eig(f,\lambda))&\overset{\text{\propref{satz_eig_direkte_summe}}}{=} \dim_K(\bigoplus\limits_{\lambda\in K}\Eig(f,\lambda)) \notag \\
		&\overset{\text{II.4.12}}{=}\sum\limits_{\lambda\in K}\dim_K(\Eig(f,\lambda)) \notag \\
		&\overset{\text{\propref{lemma_3_10}}}{\le}\sum\limits_{\lambda\in K}\mu(\chi_f,\lambda) \\
		&\le \deg(\chi_f) \\
		&= n \notag
	\end{align}
	Nach \propref{lemma_diag_summe_eig} ist $f$ genau dann diagonalisierbar, wenn $\dim_K(\sum\limits_{\lambda\in K}\Eig(f,\lambda))=n$, also wenn bei (1) und (2) Gleichheit herrscht. Gleichheit bei (1) bedeutet $\dim_K(\Eig(f,\lambda))=\mu(\chi_f,\lambda)$ für alle $\lambda\in K$, und Gleichheit bei (2) bedeutet nach \propref{lemma_3_9}, dass $\chi_f$ in Linearfaktoren zerfällt. %TODO: Verlinkung
\end{proof}

\begin{definition}[algebraische und geometrische Vielfachheit]
	Man nennt $\mu_a(f,\lambda)=\mu(\chi_f,\lambda)$ die \begriff[Vielfachheit!]{algebraische Vielfachheit} und $\mu_g(f,\lambda)=\dim_K(\Eig(f,\lambda))$ die  \begriff[Vielfachheit!]{geometrische Vielfachheit} des Eigenwertes $\lambda$ von $f$.
\end{definition}

\begin{remark}
	Wieder nennt man $A\in\Mat_n(K)$ diagonalisierbar, wenn $f_A\in\End_K(K^n)$ diagonalisierbar ist, also wenn $A\sim D$ für eine Diagonalmatrix $D$.
\end{remark}
\section{Trigonalisierbarkeit}

\begin{definition}
	Man nennt $f$ \begriff{trigonalisierbar}, wenn $V$ eine Basis $B$ besitzt, für die $M_B(f)$ eine obere Dreiecksmatrix ist.
\end{definition}

\begin{example}
	Ist $f$ diagonalisierbar, so ist $f$ auch trigonalisierbar.
\end{example}

\begin{lemma}
	\proplbl{lemma_4_3}
	Ist $f$ trigonalisierbar, so zerfällt $\chi_f$ in Linearfaktoren.
\end{lemma}
\begin{proof}
	Klar aus \propref{beispiel_2_8} und \propref{satz_2_3}.
\end{proof}

\begin{definition}[invariant]
	Ein Untervektorraum $W\le V$ ist $f$-\begriff{invariant}, wenn $f(W)\le W$.
\end{definition}

\begin{remark}
	Ist $W$ ein $f$-invarianter UVR von $V$, so ist $f\vert_W\in \End_K(W)$.
\end{remark}

\begin{example}
	\proplbl{beispiel_4_6}
	\begin{enumerate}
		\item $V$ hat stets die $f$-invarianten UVR $W=\{0\}$ und $W=V$.
		\item Jeder UVR $W\le \Eig(f,\lambda)$ ist $f$-invariant.
		\item Ist $B=(x_1,...,x_n)$ eine Basis von $V$, für die $M_B(f)$ eine obere Dreiecksmatrix ist, so sind alle UVR $W_i=\Span_K(x_1,...,x_i)$ $f$-invariant.
		\item Sei $V=W\oplus U$, $B_1=(x_1,...,x_r)$ Basis von $W$, $B_2(x_{r+1},...,x_n)$ Basis von $U$ und $B=(x_1,...,x_n)$. Ist $W$ $f$-invariant, so ist 
		\begin{align}
			M_B(f)=\begin{pmatrix}M_{B_1}(f\vert_W)&*\\0&*\end{pmatrix}\notag
		\end{align}
		Sind $W$ und $U$ $f$-invariant, so ist 
		\begin{align}
			M_B(f)=\begin{pmatrix}M_{B_1}(f\vert_W)&0\\0&M_{B_2}(f\vert_U)\end{pmatrix}\notag
		\end{align}
	\end{enumerate}
\end{example}

\begin{lemma}
	Ist $W\subset V$ ein $f$-invarianter UVR, so gilt $\chi_{f\vert_W}\vert \chi_f$. Hat $W$ ein lineares Komplement $U$, dass auch $f$-invariant ist, so $\chi_f=\chi_{f\vert_W}\cdot \chi_{f\vert_U}$.
\end{lemma}
\begin{proof}
	Ergänze eine Basis $B_0=(x_1,...,x_r)$ von $W$ zu einer Basis $B=(x_1,...,x_n)$ von $V$. Sei $A=M_B(f)$, $A_0=M_{B_0}(f\vert_W)$. Dann ist 
	\begin{align}
		A=\begin{pmatrix}A_0&*\\0&C\end{pmatrix}\quad C\in\Mat_{n-r}(K)\notag
	\end{align}
	folglich $\chi_f=\chi_A=\chi_{A_0}\cdot \chi_C$, insbesondere $\chi_{f\vert_W}\vert\chi_f$.\\
	Ist auch $U=\Span_K(x_{r+1},...,x_n)$ $f$-invariant, so ist 
	\begin{align}
		A=\begin{pmatrix}A_0&0\\0&C\end{pmatrix}\notag
	\end{align}
	und folglich $\chi_f=\chi_A=\chi_{A_0}\cdot\chi_C=\chi_{f\vert_W}\cdot\chi_{f\vert_U}$.
\end{proof}

\begin{theorem}
	Genau dann ist $f$ trigonalisierbar, wenn $\chi_f$ in Linearfaktoren zerfällt.
\end{theorem}
\begin{proof}
	$(\Rightarrow)$: \propref{lemma_4_3}\\
	$(\Leftarrow)$: Induktion nach $n=\dim_K(V)$. \\
	\emph{$n=1$}: trivial \\
	\emph{$n-1\to n$}: Nach Annahme ist $\chi_f(t)=\prod\limits_{i=1}^n (t-\lambda_i)$ mit $\lambda_1,...,\lambda_n\in K$. Sei $x_1$ ein EV zum EW $\lambda_1$. Dann ist $V_1=K\cdot x_1$ ein $f$-invarianter UVR. Ergänze $B_1=(x_1)$ zu einer Basis $B=(x_1,...,x_n)$ von $V$ und setze $B_2=(x_2,...,x_n)$, $V_2=\Span_K(B_2)$.
	\begin{align}
		\Rightarrow M_B(f)&=\begin{pmatrix}\lambda_1&*\\0&A_2\end{pmatrix}\quad A_2\in\Mat_{n-1}(K)\notag\\
		\chi_f(t)&=\chi_{\lambda_1\mathbbm{1}_1}\cdot \chi_{A_2}=(t-\lambda_1)\cdot\chi_{A_2}(t)\notag \\
		\overset{\text{\propref{lemma_3_7}}}{\Rightarrow} \chi_{A_2}(t)&=\prod\limits_{i=2}^n(t-\lambda_i)\notag
	\end{align}
	Seien $\pi_1,\pi_2\in\End_K(V)$ gegeben durch $M_B(\pi_1)=\diag(1,0,...,0)$ und $M_B(\pi_2)=\diag(0,1,...,1)$. Dann ist $\pi_1+\pi_2=\id_V$ und $f_i=\pi_1\circ f$ ist $f=\id_V\circ f=f_1+f_2$ und $f_2\vert_{V_2}\in\End_K(V_2)$. Nach Induktionshypothese ist $f_2\vert_{V_2}$ trigonalisierbar, da $M_B(f_2\vert_{V_2})=A_2$, also $\chi_{f_2\vert_{V_2}}=\chi_{A_2}$. Dies bedeutet, es gibt also eine Basis $B'_2=(x'_2,...,x'_n)$ von $V_2$, für die $M_{B'_2}(f_2\vert_{V_2})$ eine obere Dreiecksmatrix ist. Somit ist für $B'=(x_1,x'_2,...,x'_n)$ auch 
	\begin{align}
		M_{B'}(f)&=M_{B'}(f_1)+M_{B'}(f_2)\notag \\
		&= \begin{pmatrix}\lambda_1&*\\0&0\end{pmatrix} + \begin{pmatrix}0&0\\0&M_{B'_2}(f_2\vert_{V_2})\end{pmatrix}\notag
	\end{align}
	eine obere Dreiecksmatrix.
\end{proof}

\begin{conclusion}
	Ist $K$ algebraisch abgeschlossen, so ist jedes $f\in\End_K(V)$ trigonalisierbar.
\end{conclusion}
\begin{proof}
	Ist $K$ algebraisch abgeschlossen, so zerfällt nach I.6.14 jedes Polynom über $K$ in Linearfaktoren, insbesondere also $\chi_f$. %TODO:Verlinkung
\end{proof}

\begin{conclusion}
	Ist $V$ ein endlichdimensionaler $\comp$-VR, so ist jedes $f\in\End_\comp(V)$ trigonalisierbar.
\end{conclusion}
\begin{proof}
	Nach dem Fundamentalsatz der Algebra I.6.16 ist $\comp$ algebraisch abgeschlossen. %TODO:Verlinkung
\end{proof}
\section{Das Minimalpolynom}

\begin{definition}
	Für ein Polynom $P(t)=\sum_{i=0}^n c_it^i\in K[t]$ definieren wir $P(f)=\sum_{i=0}^m c_if^i\in\End_K(V)$, wobei $f^0=\id_V$, $f^1=f$, $f^2=f\circ f$, ...
	
	Analog definiert man $P(A)$ für $A\in\Mat_n(K)$.
\end{definition}

\begin{remark}
	Die Abbildung $\quad\begin{cases}K[t]\to \End_K(V)\\ P\mapsto P(f)\end{cases}$ ist ein Homomorphismus von $K$-VR und Ringen. Sein Kern ist das Ideal 
	\begin{align}
		\mathcal{I}_f:=\{P\in K[t]\mid P(f)=0\}\notag
	\end{align}
	und sein Bild ist der kommutative Unterring 
	\begin{align}
		K[f]:&=\{P(f)\mid P\in K[t]\}\notag \\
		&= \Span_K(f^0,f^1,f^2,...)\notag
	\end{align}
	des (im Allgemeinen nicht kommutativen) Rings $\End_K(V)$.
	
	Analog definiert man $\mathcal{I}_A$ und $K[A]\le \Mat_n(K)$.
\end{remark}

\begin{lemma}
	\proplbl{lemma_5_3}
	$\mathcal{I}_f\neq\{0\}$
\end{lemma}
\begin{proof}
	Wäre $\mathcal{I}_f=\{0\}$, so wäre $K[t]\to \End_K(V)$ injektiv, aber $\dim_K(K[t])= \infty>n^2=\dim_K(\End_K(V))$, ein Widerspruch.
\end{proof}

\begin{proposition}
	\proplbl{satz_5_4}
	Es gibt ein eindeutig bestimmtes normiertes Polynom $0\neq P\in K[t]$ kleinsten Grades mit $P(f)=0$. Dieses teilt jedes $Q\in K[t]$ mit $Q(f)=0$.
\end{proposition}
\begin{proof}
	Nach \propref{lemma_5_3} gibt es $0\neq P\in K[t]$ mit $P(f)=0$ von minimalem Grad $d$. Indem wir durch den Leitkoeffizienten von $P$ teilen, können wir annehmen, dass $P$ normiert ist. \\
	Sei $Q\in\mathcal{I}_f$. Polynomdivision liefert $R,H\in K[t]$ mit $Q=P\cdot H+R$ und $\deg(R)<\deg(P)=d$. Es folgt $R(f)=\underbrace{Q(f)}_{=0}-\underbrace{P(f)}_{=0}\cdot H(f)=0$. Aus der Minimalität von $d$ folgt $R=0$ und somit $P\mid Q$. \\
	Ist $Q$ zudem normiert vom Grad $d$, so ist $H=1$, also $Q=P$, was die Eindeutigkeit zeigt.
\end{proof}

\begin{definition}[Minimalpolynom]
	Das eindeutig bestimmte normierte Polynom $0\neq P\in K[t]$ kleinsten Grades mit $P(f)=0$ nennt man das \begriff{Minimalpolynom} $P_f$ von $f$.
	
	Analog definiert man das Minimalpolynom $P_A\in K[t]$ einer Matrix $A\in\Mat_n(K)$.
\end{definition}

\begin{mathematica}[Minimalpolynom]
	Die Funktion für das Minimalpolynom $p$ mit der Variable $t$ in Mathematica bzw. WolframAlpha lautet:
	\begin{align}
		\texttt{MinimalPolynomial[p,x]}\notag
	\end{align}
\end{mathematica}

\begin{example}
	\begin{enumerate}
		\item $A=\mathbbm{1}_n$, $\chi_A(t)=(t-1)^n$, $P_A(t)=t-1$
		\item $A=0$, $\chi_A(t)=t^n$, $P_A(t)=t$
		\item Ist $A=\diag(a_1,...,a_n)$ mit paarweise verschiedenen Eigenwerten $\lambda_1,...,\lambda_r$, so ist $\chi_A(t)=\prod_{i=1}^n (t-a_i)=\prod_{i=1}^n (t-\lambda_i)^{\mu_a(f_A,\lambda_i)}$, $P_A(t)=\prod_{i=1}^r (t-\lambda_i)$ und es folgt $\deg(P_A)\ge \vert \{a_1,...,a_n\}\vert=r$.
	\end{enumerate}
\end{example}

\begin{definition}[$f$-zyklisch]
	Ein $f$-invarianter UVR $W\le V$ heißt $f$-\begriff{zyklisch}, wenn es ein $x\in W$ mit $W=\Span_K(x,f(x),f^2(x),...)$ gibt.
\end{definition}

\begin{lemma}
	\proplbl{lemma_5_8}
	Sei $x\in V$ und $x_i=f(x)$. Es gibt ein kleinstes $k$ mit $x_k\in\Span_K(x_0,x_1,...,x_{k-1})$, und $W=\Span_K(x_0,...,x_{k-1})$ ein $f$-zyklischer UVR von $V$ mit Basis $B=(x_0,...,x_{k-1})$ und $M_B(f\vert_W)=M_{\chi_{f\vert_W}}$.
\end{lemma}
\begin{proof}
	Da $\dim_K(V)=n$ ist $(x_0,...,x_n)$ linear abhängig, es gibt also ein kleinstes $k$ mit $(x_0,...,x_{k-1})$ linear unabhängig, aber $(x_0,...,x_k)$ linear abhängig, folglich $x_k\in\Span_K(x_0,...,x_{k-1})$. Mit $x_k=f(x_{k-1})=\sum_{i=0}^{k-1}-c_ix_i$ ist dann 
	Da $\dim_K(V)=n$ ist $(x_0,...,x_n)$ linear abhängig, es gibt also ein kleinstes $k$ mit $(x_0,...,x_{k-1})$ linear unabhängig, aber $(x_0,...,x_k)$ linear abhängig, folglich $x_k\in\Span_K(x_0,...,x_{k-1})$. Mit $x_k=f(x_{k-1})=\sum_{i=0}^{k-1}-c_ix_i$ ist dann 
	\begin{align}
		M_B(f\vert_W)=\begin{pmatrix}0&...&...&...&0&-c_0\\
		1&\ddots&\;&\;&\vdots&\vdots\\
		0&\ddots&\ddots&\;&\vdots&\vdots\\
		\vdots&\ddots&\ddots&\ddots&\vdots&\vdots\\
		0&...&0&1&0&-c_{k-1}\end{pmatrix}\notag
	\end{align}
	somit $\chi_{f\vert_W}=t^k+\sum_{i=0}^{k-1}c_it^i$, also $M_B(f\vert_W)=M_{\chi_{f\vert_W}}$.
\end{proof}

\begin{theorem}[Satz von \person{Cayley-Hamiltion}]
	\proplbl{theorem_5_9}
	Für $f\in\End_K(V)$ ist $\chi_f(f)=0$.
\end{theorem}
\begin{proof}
	Sei $x\in V$. Definiere $x_i=f^i(x)$ und $W=\Span_K(x_0,...,x_{k-1})$ wie in \propref{lemma_5_8}. Sei $\chi_{f\vert_W}=t^k+\sum_{i=0}^{k-1} c_it^i$, also $f(x_{k-1})=\sum_{i=0}^{k-1} -c_ix_i$. Wenden wir $\chi_{f\vert_W}(f)\in\End_K(V)$ auf $x$ an, so erhalten wir 
	\begin{align}
		\chi_{f\vert_W}(f)(x)&=\left( f^k+\sum\limits_{i=1}^{k-1} c_if^i\right)(x)\notag \\
		&= \sum\limits_{i=1}^{k-1} -c_ix_i+\sum\limits_{i=1}^{k-1}c_ix_i\notag \\
		&= 0\notag
	\end{align}
	Aus $\chi_{f\vert_W}\mid \chi_f$ (\propref{beispiel_4_6}) folgt somit $\chi_f(f)(x)=0$, denn ist $\chi_f=Q\cdot \chi_{f\vert_W}$ mit $Q\in K[t]$, so ist $\chi_f(f)=Q(f)\circ\chi_{f\vert_W}(f)$, also $\chi_f(f)(x)=Q(f)(\underbrace{\chi_{f\vert_W}(f)(x)}_{=0})=0$. Da $x\in V$ beliebig war, folgt $\chi_f(f)=0\in\End_K(V)$.
\end{proof}

\begin{conclusion}
	\proplbl{folgerung_5_10}
	Es gilt $P_f\mid \chi_f$. Insbesondere ist $\deg(P_f)\le n$.
\end{conclusion}
\begin{proof}
	\propref{theorem_5_9} + \propref{satz_5_4}
\end{proof}

\begin{remark}
	Ist $B$ eine Basis von $V$ und $A=M_B(f)$, so ist $P_A=P_f$. Insbesondere ist $P_A=P_B$ für $A\sim B$. Als Spezialfall von \propref{theorem_5_9} erhält man $\chi_A(A)=0$ und $P_A\mid \chi_A$.
\end{remark}

\begin{remark}
	Der naheliegende "'Beweis"' $\underbrace{\chi_A}_{\in\Mat_n(K)}=\det(t\mathbbm{1}_n-A)(A) =\det(A\mathbbm{1}_n-A)=\det(0)=\underbrace{0}_{\in K}$ ist falsch!
\end{remark}

\section{Nilpotente Endomorphismen}

\begin{remark}
	Für $f\in\End_K(V)$ sind 
	\begin{itemize}
		\item $f\{0\}=\Ker(f^0)\subseteq \Ker(f^1)\subseteq \Ker(f^2)\subseteq ...$
		\item $V=\Image(f^0)\supseteq \Image(f^1)\supseteq \Image(f^2)\supseteq ...$
	\end{itemize}
Folgen von UVR von $V$. Nach der Kern-Bild-Formel III.7.13 ist %TODO: Verlinkung
\begin{align}
	\dim_K(\Ker(f^i))+\dim_K(\Image(f^i))=\dim_K(V)\quad\forall i\notag
\end{align}
Da $\dim_K(V)=n<\infty$ gibt es ein $d$ mit $\Ker(f^d)=\Ker(f^{d+i})$ und $\Image(f^d)=\Image(f^{d+i})$ für jedes $i\ge 0$.
\end{remark}

\begin{example}
	$f=f_A$, $A\in\Mat_2(K)$.
	\begin{itemize}
		\item $A=\begin{pmatrix}1&0\\0&1\end{pmatrix}$: $\{0\}=\Ker(f^0)=\Ker(f^1)=...$
		\item $A=\begin{pmatrix}1&0\\0&0\end{pmatrix}$: $\{0\}=\Ker(f^0)\subset\Ker(f^1)=\Ker(f^2)=...=\Span_K(e_2)$
		\item $A=\begin{pmatrix}0&1\\0&0\end{pmatrix}$: $\{0\}=\Ker(f^0)\subset\underbrace{\Ker(f^1)}_{=\Span_K(e_1)}\subset \Ker(f^2)=... = K^2$
		\item $A=\begin{pmatrix}0&0\\0&0\end{pmatrix}$: $\{0\}=\Ker(f^0)\subset\Ker(f^1)=\Ker(f^2)=...=K^2$
	\end{itemize}
\end{example}

\begin{lemma}
	\proplbl{lemma_6_3}
	Seien $f,g\in\End_K(V)$. Wenn $f$ und  $g$ kommutieren, d.h. $f\circ g=g\circ f$, so sind die UVR $\Ker(g)$ und $\Image(g)$ $f$ invariant.
\end{lemma}
\begin{proof}
	Ist $x\in\Ker(f)$, so ist $g(f(x))=f(g(x))=f(0)=0$, also $f(x)\in\Ker(g)$. Für $g(x)\in\Image(g)$ ist $f(g(x))=g(f(x))\in\Image(g)$.
\end{proof}

\begin{proposition}[Lemma von \person{Fitting}]
	\proplbl{satz_6_4}
	Seien $V_i=\Ker(f^i)$, $W_i=\Image(f^i)$, $d=\min\{i:V_i=V_{i+1}\}$. Dann sind 
	\begin{align}
		\{0\}&=V_0\subsetneq V_1\subsetneq ...\subsetneq V_d=V_{d+1}=...\notag \\
		V&= W_0\supsetneq W_1\supsetneq ... \supsetneq W_d=W_{d+1}=...\notag
	\end{align}
	Folgen $f$-invarianter UVR und $V=V_d\oplus W_d$.
\end{proposition}
\begin{proof}
	Da $f^i$ und $f^j$ für beliebige $i,j$ kommutieren, sind $V_i$ und $V_j$ nach \propref{lemma_6_3} $f$-invariant für jedes $i$. Aus $\dim_K(V_i)+\dim_K(W_i)=n$ folgt $d=\min\{i:W_i=W_{i+1}\}$, insbesondere ist $\Image(f^d)=\Image(f^{d+1})=f(\Image(f^d))$, somit $W_{d+i}=\Image(f^{d+i})=W_d$ für $i\ge 0$, also auch $V_d=V_{d+i}$ für alle $i\ge 0$. \\
	Insbesondere ist $f^d\vert_{W_d}:W_d\to W_{2d}=W_d$ surjektiv, also auch injektiv, also $V_d\cap W_d=\{0\}$. Aus der Dimensionsformel II.4.12 folgt dann $\dim_K(V_d+W_d)=\dim_K(V_d)+\dim_K(W_d)=\dim_K(V)$. Folglich ist $V_d+W_d=V$ und $V_d\cap W_d=\{0\}$, also $V=V_d\oplus W_d$.
\end{proof}

\begin{definition}[nilpotent]
	Ein $f\in\End_K(V)$ heißt \begriff{nilpotent}, wenn $f^k=0$ für ein $k\in\natur$. Analog heißt $A\in\Mat_n(K)$ nilpotent, wenn $A^k=0$ für $k\in\natur$. Das kleinste $k$ mit $f^k=0$ bzw. $A^k$ heißt die \begriff{Nilpotenzklasse} von $f$ bzw. $A$.
\end{definition}

\begin{lemma}
	\proplbl{lemma_6_6}
	Ist $f$ nilpotent, so gibt es eine Basis $B$ von $V$, für die $M_B(f)$ eine strikte obere Dreiecksmatrix ist.
\end{lemma}
\begin{proof}
	Induktion nach $n=\dim_K(V)$. \\
	\emph{$n=1$}: $f^k=0\Rightarrow f=0$ \\
	\emph{$n>1$}: Sei $k$ die Nilpotenzklasse von $f$ und $U=\Ker(f^{k-1})$. Dann ist $U\subset V$. Da $f^k=f^{k-1}\circ f$ ist $f(V)\subset U$, insbesondere $f\vert_U\in\End_K(U)$. Da $f\vert_U$ nilpotent ist, gibt es nach I.H. eine Basis $B_0$ von $U$, für die $M_B(f\vert_U)$ eine strikte obere Dreiecksmatrix ist. Ergänze $B_0$ zu einer Basis $B$ von $V$. Da $f(V)\subset U$ ist dann auch 
	\begin{align}
		M_B(f)=\begin{pmatrix}M_{B_0}(f\vert_U)&*\\0&0\end{pmatrix}\notag
	\end{align}
	eine strikte obere Dreiecksmatrix.
\end{proof}

\begin{proposition}
	\proplbl{satz_6_7}
	Für $f\in\End_K(V)$ sind äquivalent:
	\begin{enumerate}[label={\arabic*)}]
		\item $f$ ist nilpotent
		\item $f^n=0$ für $n \in \natur$
		\item $P_f(t)=t^r$ für ein $r\leq n$
		\item $\chi_f(t)=t^n$
		\item Es gibt eine Basis $B$ von $V$, mit 
		\[
		M_B(f) = 
		\begin{pmatrix}
		0 & \textasteriskcentered & \dots & \textasteriskcentered\\
		& \ddots & \ddots & \vdots \\
		& & \ddots & \textasteriskcentered \\
		& & & 0
		\end{pmatrix}
		\] eine strikte obere Dreiecksmatrix ist.
	\end{enumerate}
\end{proposition}
\begin{proof}
	\hspace{0pt}
	\begin{itemize}
		\item $1)\Rightarrow 5)$: \propref{lemma_6_6}
		\item $5)\Rightarrow 4)$: \propref{beispiel_2_8}
		\item $4)\Rightarrow 3)$: Nach \propref{folgerung_5_10} ist $P_f\vert \chi_f=t^n$, also $t^n=P_f(t)Q(t)$ mit $Q\in K[t]$. Schreibe $P_f(t)=t^a\cdot P_1(t), Q(t)=t^b\cdot Q_1(t)$ mit $a,b\in\natur$, $P_1,Q_1\in K[t]$, $P_1(0)\neq 0$, $Q_1(0)\neq 0$ \\
		$\overset{\propref{lemma_3_8}}{\Rightarrow} t^{n-(a+b)}=P_1(t)Q_1(t)$ und $(P_1Q_1)(0)\neq 0$ \\
		$\Rightarrow n-(a+b)=0\Rightarrow P_1=1$, somit $P_f(t)=t^a$
		\item $3)\Rightarrow 2)$: $t^r=0$, $r\leq n\Rightarrow f^n=0$
		\item $2)\Rightarrow 1)$: nach Definition
	\end{itemize}
\end{proof}

\begin{conclusion}
	Die Nilpotenzklasse eines nilpotenten Endomorphismus $f\in\End_K(V)$ ist höchstens $\dim_K(V)$.
\end{conclusion}

\begin{conclusion}
	Ist $d:=\min\{i\mid \Ker(f^i)=\Ker(f^{i+1})\}$, so ist $d\le \dim_K(\Ker(f))=\mu_a(f,0)$.
\end{conclusion}
\begin{proof}
	Sei $V_d=\Ker(f^d)$, $W_d=\Image(f^d)$, $k=\dim_K(V_d)$. Da $V=V_d\oplus W_d$ ist $\chi_f=\chi_{f\vert_{V_d}}\cdot \chi_{f\vert_{W_d}}$. Da $f\vert_{V_d}$ nilpotent ist, ist $\chi_{f\vert_{V_d}}=t$ nach \propref{satz_6_7}. Da $f\vert_{W_d}$ injektiv ist, ist $\chi_{f\vert_{W_d}}(0)\neq 0$. Somit ist $\mu_a(f,0)= \mu(\chi_f,0) \overset{\propref{lemma_3_6}}{=}k$. Da $\dim_K(\Ker(f^d))>...>\dim_K(\Ker(f))>0$ ist $k=\dim_K(\Ker(f^d))\ge d$, falls $d>0$, sonst klar. 
\end{proof}

\begin{remark}
	Die Bedeutung nilpotenter Endomorphismen beim Finden geeigneter Basen ergibt sich aus der folgenden Beobachtung: \\
	Ist $A$ eine obere Dreiecksmatrix, so ist $A=D+N$, wobei $D$ eine Diagonalmatrix ist und $N$ eine strikte obere Dreiecksmatrix ist. Anders gesagt: Jeder trigonalisierbare Endomorphismus ist Summe aus einem diagonalisierbaren und einem nilpotenten Endomorphismus.
\end{remark}

\begin{definition}[\person{Jordan}-Matrix]
	Für $k\in\natur$ definieren wir die \begriff{\person{Jordan}-Matrix}
	\begin{align}
		J_k=\begin{pmatrix}0&1&0&...&0 \\
		\vdots&\ddots&\ddots&\ddots&\vdots\\
		\vdots&\;&\ddots&\ddots&0\\
		\vdots&\;&\;&\ddots&1\\
		0&...&...&...&0\end{pmatrix} \in \Mat_k(K)\notag
	\end{align}
	weiter setzen wir für $\lambda\in K$ $J_k(\lambda):=\lambda\mathbbm{1}+J_k$.
\end{definition}

\begin{lemma}
	Die \person{Jordan}-Matrix $J_k$ ist nilpotent von Nilpotenzklasse $k$.
\end{lemma}
\begin{proof}
	Es ist $(J_k)^r=(\delta_{i+r,j})_{i,j}$ für $r\ge 1$.
\end{proof}

\begin{proposition}
	Ist $f$ nilpotent von Nilpotenzklasse $k$, so gibt es eindeutig bestimmte $r_1,..,r_k\in\natur_{>0}$ mit $\sum\limits_{d=1}^k dr_d-n$ und eine Basis $B$ von $V$ mit 
	\begin{align}
		M_B(f)=\diag(\underbrace{J_k,...,J_k}_{r_k\text{ viele}},...,\underbrace{J_1,...,J_1}_{r_1\text{ viele}})\notag
	\end{align}
\end{proposition}
\begin{proof}
	Sei $U_i=\Ker(f^i)$. Nach \propref{satz_6_4} haben wir eine Folge $\{0\}=U_0\subset U_1\subset ...\subset U_k=V$ mit $f(U_i)\subseteq U_{i-1}$ für alle $i>0$. \\
	Wir konstruieren eine Zerlegung $V=\bigoplus\limits_{d=1}^k W_d$ mit $U_i=U_{i-1}\oplus W_i$, $f(W_i)\subseteq W_{i-1}$, $f\vert_{W_d}$ injektiv für $i>1$.
	\begin{align}
		V&= U_k\notag \\
		V&= U_{k-1}\oplus W_k \notag \\
		V&= U_{k-2}\oplus W_{k-1}\oplus W_k \notag \\
		\vdots \notag \\
		V&= U_0 \oplus W_1\oplus ... \oplus W_k\notag
	\end{align}
	Wähle $W_k$ mit $V=U_k=U_{k-1}\oplus W_k$. Ist $k>1$, so ist $W_k\cap \Ker(f)\subseteq W_k\cap U_{k-1}=\{0\}$, also $f\vert_{W_k}$ ist injektiv. Des weiteren ist $f(W_k)\subseteq U_{k-1}$ und aus $W_k\cap U_{k-1}=\{0\}$ folgt $f(W_k)\cap U_{k-2}=\{0\}$. Wir können deshalb $W_{k-1}$ mit $U_{k-1}=U_{k-2}\oplus W_{k-1}$ und $f(W_k)\subseteq W_{k-1}$ wählen. Somit ist $V=U_{k-1}\oplus W_k=U_{k-2}\oplus W_{k-1}\oplus W_k$. Wir setzen dies fort und erhalten $V= U_0 \oplus W_1\oplus ... \oplus W_k$ mit $f(W_i)\subseteq W_{i-1}$ und $f\vert_{W_i}$ injektiv für $i>1$, wobei $U_0=\{0\}$ und $W_1=\Ker(f)$. \\
	Sie $r_d=\dim_K(W_d)-\dim_K(W_{d+1})$, wobei wir $W_{k+1}=\{0\}$. Wähle nun eine Basis $(x_{k,1},...,x_{k,r_k})$ von $W_k$. Ist $k>1$, so ist $f\vert_{W_k}$ injektiv und wir können $(f(x_{k,1}),...,f(x_{k,r_k}))$ durch Elemente $x_{k-1,1},...,x_{k-1,r_{k-1}}$ zu einer Basis von $W_{k-1}$ ergänzen, und so weiter.\\
	Da $V=\bigoplus\limits_{d=1}^k W_d$ ist
	\begin{align}
		B=\{f^i(x_{d,j})\mid d=1,...,k,j=1,...,r_d,i=0,...,d-1\}\notag
	\end{align}
	eine Basis von $V$, die bei geeigneter Anordnung das Gewünschte leistet. \\
	Es bleibt zu zeigen, dass $r_1,...,r_k$ eindeutig bestimmt sind. Ist $B_0$ eine Basis, für die $M_{B_0}(f)$ in der gewünschten Form ist, so ist 
	\begin{align}
		\dim_K(U_1) &= \sum\limits_{d=1}^k r_d \notag \\
		\dim_K(U_2) &= \sum\limits_{d=2}^k r_d + \sum\limits_{d=1}^k r_d \notag \\
		\vdots \notag \\
		\dim_K(U_k) &= \sum\limits_{d=k}^k r_d + ... + \sum\limits_{d=1}^k r_d\notag 
	\end{align}
	woraus man sieht, dass $r_1,...,r_k$ durch $U_1,...,U_k$, also durch $f$ eindeutig bestimmt.
\end{proof}

\begin{example}
	Sei $f=f_A$ mit $A=\begin{pmatrix}0&1&3\\\;&0&2\\\;&\;&0\end{pmatrix}\in\Mat_3(\real)$
	\begin{align}
		A^2=\begin{pmatrix}0&0&2\\\;&0&0\\\;&\;&0\end{pmatrix}, A^3=0\notag
	\end{align}
	$\Rightarrow k=3, U_0=\{0\}, U_1=\real e_1, U_2=\real e_1+\real e_2, U_3=V$. \\
	Wähle $W_3$ mit $V=U_3=U_2\oplus W_3$, z.B. $W_3=\real e_3$. \\
	Wähle $W_2$ mit $U_2=U_1\oplus W_2$ und $f(W_3)\subseteq W_2$, also 
	\begin{align}
		W_2=\real\begin{pmatrix}3\\2\\0\end{pmatrix}\notag
	\end{align}
	Setze $W_1=U_1=\Ker(f)=\real e_1\Rightarrow$ Basis $B=(f^2(e_3),f(e_3),e_3)$
	\begin{align}
		M_B(f)=\begin{pmatrix}0&1&0\; \\ \;&0&1\\ \;&\;&0\end{pmatrix}\notag
	\end{align}
\end{example}
\section{Die \person{Jordan}-Normalform}

\begin{definition}[Hauptraum]
	Der \begriff{Hauptraum} von $f$ zum EW $\lambda$ der Vielfachheit $r=\mu_a(f,\lambda)$ ist
	\begin{align}
		\Hau(f,\lambda)=\Ker\Big( (f-\lambda\id_V)^r\Big) \notag
	\end{align}
\end{definition}

\begin{lemma}
	\proplbl{lemma_7_2}
	$\Hau(f,\lambda)$ ist ein $f$-invarianter UVR der Dimension $\dim_K(\Hau(f,\lambda))= \mu_a(f,\lambda)$, auf dem $f-\lambda\id_V$ nilpotent ist und $\chi_{f\vert_{\Hau(f,\lambda)}}= (t-\lambda)^{\mu_a(f,\lambda)}$
\end{lemma}
\begin{proof}
	$f$ kommutiert sowohl mit $f$ als auch mit $\id_V$, somit auch mit $(f-\lambda\id_V)^r$. Die $f$-Invarianz von $U=\Hau(f,\lambda)$ folgt aus \propref{lemma_6_3}. Nach \propref{folgerung_6_9} ist $\dim_K(U)=\mu_a(f-\lambda\id_V,0)$ und da $\chi_f(t)=\chi_{f-\lambda\id_V}(t-\lambda)$ ist $\mu_a(f,\lambda)=\mu(\chi_f,\lambda)= \mu_a(f-\lambda\id_V,0)$. Da $f-\lambda\id_V\vert_U$ nilpotent ist $\chi_{f-\lambda\id_V\vert_U}(t)= t^r$, somit $\chi_{f\vert_U}(t)=(t-\lambda)^r$.
\end{proof}

\begin{proposition}[Hauptraumzerlegung]
	\proplbl{satz_7_3}
	Ist $\chi_f(t)=\prod_{i=1}^m (t-\lambda_i)^{r_i}$ mit $\lambda_1,...,\lambda_m\in K$ paarweise verschieden und $r_1,...,r_m\in\natur$, so ist $V=\bigoplus_{i=1}^m V_i$ mit $V_i=\Hau(f,\lambda_i)$ eine Zerlegung in $f$-invariante UVR und für jedes $i$ ist $\chi_{f\vert_{V_i}}(t)=(t-\lambda_i)^{r_i}$.
\end{proposition}
\begin{proof}
	Induktion nach $m$.\\
	\emph{$m=1$}: $r_1=n\overset{\propref{lemma_7_2}}{\Rightarrow} V=V_1$.\\
	\emph{$m-1\to m$}: Nach \propref{satz_6_4} ist $V=V_1\oplus W_1$ mit $W_1=\Image((f-\lambda_i\id_V)^r)$ eine Zerlegung in $f$-invariante UVR mit $\dim_K(V_1)=r_1$, $\dim_K(W_1)=n-r_1$. Somit ist $\chi_f=\chi_{f\vert_{V_1}}\cdot \chi_{f\vert_{W_1}}$ und $\chi_{f\vert_{V_1}}\overset{\propref{lemma_7_2}}{=}(t-\lambda_1)^{r_1}$ also $\chi_{f\vert_{W_1}}=\prod_{i=2}^m (t-\lambda_i)^{r_i}$. Nach I.H. ist also $W_1=\bigoplus_{i=2}^m \Hau(f\vert_{W_1},\lambda_i)$. Es ist für $i\ge 2$ $\Hau(f\vert_{W_1},\lambda_i)\subseteq\Hau(f,\lambda_i)=V_i$ und da $\dim_K(\Hau(f\vert_{W_1},\lambda_i))=r_i=\dim_K(\Hau(f,\lambda_i))$ gilt Gleichheit. Damit ist
	\begin{align}
		V&=V_1\oplus W_1 \notag\\
		&=V_1\oplus\bigoplus_{i=2}^m\Hau(f\vert_{W_1},\lambda_i)\notag \\
		&= V_1\oplus\bigoplus_{i=2}^m V_i \notag\\
		&= \bigoplus_{i=1}^m V_i\notag
	\end{align}
\end{proof}

\begin{example}
	$f=f_A$
	\begin{align}
		A=\begin{pmatrix}1&3&\; \\ \;&1&4 \\ \;&\; & 2\end{pmatrix}\in\Mat_3(\real)\notag
	\end{align}
	$\chi_A(t)=(t-1)^2(t-2)$
	$\Rightarrow \real^3=\underbrace{\Hau(f,1)}_{\dim = 2}\oplus\underbrace{\Hau(f,2)}_{\dim = 1}$ \\
	$\Hau(f,1)=\Ker((f-\id)^2)=L((A-\mathbbm{1})^2,0)$ \\
	$\Hau(f,2)=\Ker(f-2\id)=\Eig(f,2)=L(A-2\mathbbm{1},0)$ \\
	\begin{align}
		A-\mathbbm{1}&=\begin{henrysmatrix}0&3&\; \\ \; & -1&4 \\ \;&\;&0\end{henrysmatrix}, (A-\mathbbm{1})^2=\begin{henrysmatrix}0&\;&12 \\ \;&0&4 \\ \;&\;&1\end{henrysmatrix}&\Rightarrow \Hau(f,1)=\real e_1+\real e_2\notag \\
		A-2\mathbbm{1}&=\begin{henrysmatrix}-1&3&\; \\ \; & -1&4 \\ \;&\;&0\end{henrysmatrix}&\Rightarrow\Hau(f,2)=\real\begin{henrysmatrix}12\\4\\1\end{henrysmatrix}\notag
	\end{align}
	Mit $B=\left( \begin{henrysmatrix}1\\0\\0\end{henrysmatrix}, \begin{henrysmatrix}0\\1\\0\end{henrysmatrix}, \begin{henrysmatrix}12\\4\\1\end{henrysmatrix}\right) $ ist
	\begin{align}
		M_B(f)=\begin{pmatrix}\begin{pmatrix}1&3\\\; & 1\end{pmatrix}&\; \\ \; & 2\end{pmatrix}\notag
	\end{align} 
\end{example}

\begin{theorem}[\person{Jordan}-Normalform]
	Sei $f\in\End_K(V)$ ein Endomorphismus, dessen charakteristisches Polynom $\chi_f$ in Linearfaktoren zerfällt. Dann gibt es $r\in\natur$, $\mu_1,...,\mu_r\in K$ und $k_1,...,k_r\in \natur$ mit $\sum_{i=1}^r k_i=\dim_K(V)$ und eine Basis $B$ von $V$ mit
	\begin{align}
		M_B(f)=\diag(J_{k_1}(\mu_1),...,J_{k_r}(\mu_r))\notag
	\end{align} 
	Die Paare $(\mu_1,k_1),...,(\mu_r,k_r)$ heißen die \begriff{\person{Jordan}-Invarianten} von $f$ und sind bis auf Reihenfolge eindeutig bestimmt.
\end{theorem}
\begin{proof}
	Schreibe $\chi_f(t)=\prod_{i=1}^m (t-\lambda_i)^{r_i}$ mit $\lambda_1,...,\lambda_m\in K$ paarweise verschieden, $r_i\in\natur$. Sei $V_i=\Hau(f,\lambda_i)$. Nach \propref{satz_7_3} ist $V=\bigoplus_{i=1}^m V_i$ eine Zerlegung in $f$-invariante UVR. Für jedes $i$ wenden wir \propref{satz_6_13} auf $(f-\lambda_i\id_V)\vert_{V_i}$ an und erhalten eine Basis $B_i$ von $V_i$ und $k_{i,1}\ge ...\ge k_{i,s_i}$ mit 
	\begin{align}
		M_B((f-\lambda_i\id)\vert_{V_i})=\diag(J_{k_{i,1}},...,J_{k_{i,s_i}})\notag
	\end{align}
	Es folgt $M_{B_i}(f\vert_{V_i})=M_{B_i}(\lambda_i\id_{V_i})+M_{B_i}((f-\lambda_i\id_V)\vert_{V_i})$. Ist nun $B$ die Vereinigung der $B_i$, so hat $M_B(f)$ die gewünschte Form. Die Eindeutigkeit der \person{Jordan}-Invarianten folgt aus der Eindeutigkeit der $k_{i,j}$ in \propref{lemma_6_3}.
\end{proof}

\begin{remark}
	Ist $K$ algebraisch abgeschlossen, so haben wir nun eine (bis auf Permutationen) eindeutige Normalform für Endomorphismen $f\in\End_K(V)$ gefunden. Aus ihr lassen sich viele Eigenschaften des Endomorphismus leicht ablesen.
\end{remark}

\begin{conclusion}
	\proplbl{folgerung_7_7}
	Sei $f\in\End_K(V)$ trigonalisierbar mit $\chi_f(t)=\prod_{i=1}^m (t-\lambda_i)^{\mu_a(f,\lambda_i)}$, $P_f(t)=\prod_{i=1}^m (t-\lambda_i)^{d_i}$ und \person{Jordan}-Invarianten $(\mu_1,k_1),...,(\mu_r,k_r)$. Mit $J_i=\{j\mid \mu_j=\lambda_i\}$ ist dann 
	\begin{align}
		\mu_g(f,\lambda_i)&= \vert J_i \vert \notag \\
		\mu_a(f,\lambda_i) &= \sum_{j\in J_i} k_j\notag \\
		d_i&= \max\{k_j\mid j\in J_i\}\notag
	\end{align}
\end{conclusion}
\begin{proof}
	\begin{itemize}
		\item $\mu_a$: klar, da $\chi_f(t)=\prod_{j=1}^r (t-\mu_j)^{k_j}=\prod_{i=1}^m (t-\lambda_i)^{\mu_a(f,\lambda_i)}$
		\item $\mu_g$: lese Basis von $\Eig(f,\lambda_i)$ aus \person{Jordan}-NF: Jeder Block $J_{k_j}(\lambda_i)$ liefert ein Element der Basis.
		\item $d_i$: folgt, da $J_{k_j}$ nilpotent von Nilpotenzklasse $k_j$ ist (\propref{lemma_6_12}).
	\end{itemize}
\end{proof}

\begin{conclusion}
	Genau dann ist $f$ diagonalisierbar, wenn 
	\begin{align}
		\chi_f(t)&=\prod_{i=1}^m (t-^\lambda_i)^{r_i}\quad \lambda_1,...,\lambda_m\in K\text{ paarweise verscheiden und} \notag \\
		P_f(t) &= \prod_{i=1}^m (t-\lambda_i)\notag
	\end{align}
\end{conclusion}
\begin{proof}
	Genau dann ist $f$ diagonalisierbar, wenn $f$ trigonalisierbar ist und die \person{Jordan}-NF die Diagonalmatrix ist (Eindeutigkeit der JNF), also $k_j=1$ für alle $j$. Nach \propref{folgerung_7_7} ist dies äquivalent dazu, dass $d_i=1$ für alle $i$, also $P_f=\prod_{i=1}^m (t-\lambda_i)$.
\end{proof}

\begin{remark}
	Wieder definiert man die \person{Jordan}-Invarianten, etc. von einer Matrix $A\in\Mat_n(K)$ als die \person{Jordan}-Invarianten von $f_A\in\End_K(K^n)$.
\end{remark}

\begin{conclusion}
	Seien $A,B\in\Mat_n(K)$ trigonalisierbar. Genau dann ist $A\sim B$, wenn $A$ und $B$ die gleichen \person{Jordan}-Invarianten haben.
\end{conclusion}
\begin{proof}
	Existenz und Eindeutigkeit der \person{Jordan}-Normalform.
\end{proof}

\chapter{Skalarprodukte}
In diesem ganzen Kapitel seien
\begin{itemize}
	\item $K=\real$ oder $K=\comp$
	\item $n\in\natur$
	\item $V$ ein $n$-dimensionaler $K$-VR
\end{itemize}

\section{Das Standardskalarprodukt}

Sei zunächst $K=\real$.

\begin{definition}[Standardskalarprodukt in $\real$]
	Auf den Standardraum $V=\real^n$ definiert man das \begriff{Standardskalarprodukt in $\real$} $\langle.\rangle:\real^n\times\real^n\to \real$ durch
	\begin{align}
		\skalar{x}{y}=x^ty=\sum_{i=1}^n x_iy_i\notag
	\end{align}
\end{definition}

\begin{proposition}
	\proplbl{6_1_2}
	Das Standardskalarprodukt erfüllt die folgenden Eigenschaften:
	\begin{itemize}
		\item Für $x,x',y,y'\in\real^n$ und $\lambda\in\real$ ist:
		\begin{align}
			\langle x+x',y\rangle &= \langle x,y\rangle + \langle x',y\rangle\notag\\
			\langle \lambda x,y\rangle =&= \lambda \langle x,y \rangle\notag \\
			\langle x,y+y' \rangle &= \langle x,y \rangle + \langle x,y'\rangle\notag \\
			\langle x,\lambda y \rangle &= \lambda \langle x,y \rangle\notag
		\end{align}
		\item Für $x,y\in\real^n$ ist $\langle x,y \rangle=\langle y,x\rangle$
		\item Für $x\in\real^n$ ist $\langle x,y \rangle\ge 0$ und $\langle x,x\rangle=0\iff x=0$
	\end{itemize}
\end{proposition}
\begin{proof}
	\begin{itemize}
		\item klar
		\item klar
		\item $\langle x,x\rangle=\sum_{i=1}^n x_i^2\ge x_j^2$ für jedes $j\Rightarrow \langle x,x\rangle\ge 0$ und $\langle x,x \rangle>0$ falls $x_j\neq 0$ für ein $j$.
	\end{itemize}
\end{proof}

\begin{definition}[euklidische Norm in $\real$]
	Auf $K=\real^n$ definiert man \begriff{euklidische Norm in $\real$} $\Vert \cdot \Vert:\real^n\to \real_{\ge 0}$ durch
	\begin{align}
		\Vert x\Vert =\sqrt{\langle  x,x\rangle}\notag
	\end{align}
\end{definition}

\begin{proposition}[Ungleichung von \person{Cauchy-Schwarz}]
	\proplbl{6_1_4}
	Für $x,y\in\real^n$ gilt
	\begin{align}
		\vert \langle x,y \rangle\vert \le \Vert x\Vert \cdot \Vert y\Vert\notag
	\end{align}
	Gleichheit genau dann, wenn $x$ und $y$ linear abhängig sind.
\end{proposition}
\begin{proof}
	siehe Analysis, siehe VI.§3
\end{proof}

\begin{proposition}
	Die euklidische Norm erfüllt die folgenden Eigenschaften:
	\begin{itemize}
		\item Für $x\in\real^n$ ist $\Vert x\Vert=0\iff x=0$
		\item Für $x\in\real^n$ und $\lambda\in\real$ ist $\Vert \lambda x\Vert =\vert \lambda \vert \cdot \Vert x\Vert$
		\item Für $x,y\in\real^n$ ist $\Vert x+y\Vert \le \Vert x\Vert +\Vert y\Vert$
	\end{itemize}
\end{proposition}
\begin{proof}
	\begin{itemize}
		\item \propref{6_1_2}
		\item \propref{6_1_2}
		\item $\Vert x+y\Vert^2=\langle x+y,x+y \rangle=\langle x,x \rangle+2\langle x,y\rangle+\langle y,y\rangle\le \Vert x\Vert^2+2\Vert x\Vert \Vert y\Vert+\Vert y\Vert^2=(\Vert x\Vert +\Vert y\Vert)^2\overset{\propref{6_1_4}}{\Rightarrow}\Vert x+y\Vert \le \Vert x\Vert +\Vert y\Vert$
	\end{itemize}
\end{proof}

Sei nun $K=\comp$.

\begin{definition}[komplexe Konjugation, Absolutbetrag]
	Für $x,y\in\real$ und $z=x+iy\in\comp$ definiert man $\overline{z}=x-iy$ heißt \begriff{komplexe Konjugation}.. Man definiert den \begriff{Absolutbetrag} von $z$ als
	\begin{align}
		\vert z\vert &=\sqrt{z\overline{z}}=\sqrt{x^2+y^2}\in\real_{\ge 0}\notag
	\end{align}
	Für $A=(a_{ij})_{i,j}\in\Mat_{m\times n}(\comp)$ sehen wir
	\begin{align}
		\overline{A}&= (\overline{a_{ij}})_{i,j}\in\Mat_{m\times n}(\comp)\notag
	\end{align}
\end{definition}

\begin{proposition}
	\proplbl{6_1_7}
	Komplexe Konjugation ist ein Ringautomorphismus von $\comp$ mit Fixkörper
	\begin{align}
		\{z\in\comp\mid z=\overline{z}\}&=\real\notag
	\end{align}
\end{proposition}
\begin{proof}
	siehe LAAG1 H47
\end{proof}

\begin{conclusion}
	\proplbl{6_1_8}
	Für $A,B\in\Mat_n(\comp)$ und $S\in\GL_n(\comp)$ ist $\overline{A+B}=\overline{A}+\overline{B}, \overline{AB}=\overline{A}\cdot \overline{B},\overline{A^t}=\overline{A}^t, \overline{S^{-1}}=\overline{S}^{-1}$
\end{conclusion}
\begin{proof}
	\propref{6_1_7}, einfache Übung
\end{proof}

\begin{definition}[Standardskalarprodukt in $\comp$]
	Auf $K=\comp^n$ definiert man das \begriff{Standardskalarprodukt in $\comp$} $\langle\cdot,\cdot\rangle:\comp^n\times\comp^n\to \comp$ durch
	\begin{align}
	\langle x,y\rangle=x^t\overline{y}=\sum_{i=1}^n x_i\overline{y}_i\notag
	\end{align}
\end{definition}

\begin{proposition}
	Das komplexe Standardskalarprodukt erfüllt die folgenden Eigenschaften:
	\begin{itemize}
		\item Für $x,x',y,y'\in\comp^n$ und $\lambda\in\comp$ ist:
		\begin{align}
		\langle x+x',y\rangle &= \langle x,y\rangle + \langle x',y\rangle\notag\\
		\langle \lambda x,y\rangle =&= \lambda \langle x,y \rangle\notag \\
		\langle x,y+y' \rangle &= \langle x,y \rangle + \langle x,y'\rangle\notag \\
		\langle x,\lambda y \rangle &= \kringel{lightgrey}{\overline{\lambda}} \langle x,y \rangle\notag
		\end{align}
		\item Für $x,y\in\comp^n$ ist $\langle x,y \rangle=\kringel{lightgrey}{\overline{\langle y,x\rangle}}$
		\item Für $x\in\comp^n$ ist $\langle x,y \rangle\in\real_{\ge 0}$ und $\langle x,x\rangle=0\iff x=0$
	\end{itemize}
\end{proposition}
\begin{proof}
	\begin{itemize}
		\item klar
		\item klar
		\item $\langle x,x\rangle=\sum_{i=1}^n x_i\overline{x_i}=\sum_{i=1}^n \vert x_i\vert^2$
	\end{itemize}
\end{proof}

\begin{definition}[euklidische Norm in $\comp$]
	Auf $V=\comp$ definiert man die \begriff{euklidische Norm in $\comp$} $\Vert \cdot \Vert:\comp^n\to \real_{\ge 0}$ durch
	\begin{align}
	\Vert x\Vert =\sqrt{\langle  x,x\rangle}\notag
	\end{align}
\end{definition}

\begin{remark}
	\proplbl{6_1_12}
	Schränkt man das komplexe Skalarprodukt auf den $\real^n$ ein, so erhält man das Standardskalarprodukt auf dem $\real^n$. Wir werden ab jetzt die beiden Fälle $K=\real$ und $K=\comp$ parallel behandeln. Wenn nicht anders angegeben, werden wir die Begriffe für den komplexen Fall benutzen, aber auch den reellen Fall einschließen.
\end{remark}
\section{Bilinearformen und Sesquilinearformen}

Sei $K=\real$ oder $K=\comp$.

\begin{definition}[Bilinearform, Sesquilinearform]
	Eine \begriff{Bilinearform} ($K=\real$) bzw. \begriff{Sesquilinearform} ($K=\comp$) ist eine Abbildung $s:V\times V\to K$ für die gilt:
	\begin{itemize}
		\item Für $x,x',y\in V$ ist $s(x+x',y)=s(x,y)+s(x',y)$
		\item Für $x,y,y'\in V$ ist $s(x,y+y')=s(x,y)+s(x,y')$
		\item Für $x,y\in V$, $\lambda\in K$ ist $s(\lambda x,y)=\lambda s(x,y)$
		\item Für $x,y\in V$, $\lambda\in K$ ist $s(x,\lambda y)=\kringel{white}{\overline{\lambda}} s(x,y)$
	\end{itemize}
\end{definition}

\begin{remark}
	Im Fall $K=\real$ ist $\lambda=\overline{\lambda}$. Wir werden der Einfachheit halber auch in diesem Fall von Sesquilinearformen sprechen, vgl. \propref{6_1_12}
\end{remark}

\begin{example}
	Für $A=(a_{ij})_{i,j}\in\Mat_n(K)$ ist $s_A:K^n\times K^n\to K^n$ gegeben durch
	\begin{align}
		s_A(x,y)=x^tA\overline{y}=x^t\left( \sum_{j=1}^n a_{ij}\overline{y}_j\right)_i=\sum_{i,j=1}^n a_{ij}x_i\overline{y}_j\notag
	\end{align}
	eine Sesquilinearform auf $V=K^n$.
\end{example}

\begin{definition}
	Sei $s$ eine Sesquilinearform auf $V$ und $B=(v_1,...,v_n)$ eine Basis von $V$. Die \begriff[Sesquilinearform!]{darstellende Matrix} von $s$ bzgl. $B$ ist
	\begin{align}
		M_B(s)=(s(v_i,v_j))_{i,j}\in\Mat_n(K)\notag
	\end{align}
\end{definition}

\begin{example}
	Die darstellende Matrix des Standardskalarprodukts $s=s_{\mathbbm{1}_n}$ auf den Standardraum $V=K^n$ bzgl. der Standardbasis $\mathcal{E}$ ist
	\begin{align}
		M_{\mathcal{E}}(s)=\mathbbm{1}_n\notag
	\end{align}
\end{example}

\begin{lemma}
	\proplbl{6_2_6}
	Seien $v,w\in V$. Mit $x=\Phi_B^{-1}(v)$, $y=\Phi_B^{-1}(w)$ und $A=M_B(s)$ ist $s(v,w)=x^tA\overline{y}=s_A(x,y)$.
\end{lemma}
\begin{proof}
	Achtung: $v_i$ beschreibt das $i$-te Element der Basis $B$!\\
	$s(v,w)=s(\sum_{i=1}^n x_iv_i,\sum_{j=1}^n y_jv_j)=\sum_{i,j=1}^n x_i\overline{y}s(v,v_j)=x^tA\overline{y}$
\end{proof}

\begin{proposition}
	ISei $B$ eine Basis von $V$. Die Abbildung $s\mapsto M_B(s)$ ist eine Bijektion zwischen den Sesquilinearformen auf $V$ und $\Mat_n(K)$.
\end{proposition}
\begin{proof}
	\begin{itemize}
		\item injektiv: \propref{6_2_6}
		\item surjektiv: Für $A\in\Mat_n(K)$ wird durch $s(v,w)=\Phi_B^{-1}(v)^t\cdot A\cdot \overline{\Phi_B^{-1}(w)}$ eine Sesquilinearform auf $V$ mit $M_B(s)=(s(v_i,w_j))_{i,j}= (e_i^tA\overline{e_j})_{i,j}=(e_iAe_j)_{i,j}=A$ definiert.
	\end{itemize}
\end{proof}

\begin{proposition}[Transformationsformel]
	\proplbl{6_2_8}
	Seien $B$ und $B'$ Basen von $V$ und $s$ eine Sesquilinearform auf $V$. Dann gilt:
	\begin{align}
		M_{B'}(s)=(T_B^{B'})^t\cdot M_B(s)\cdot \overline{T_B^{B'}}\notag
	\end{align}
\end{proposition}
\begin{proof}
	Seien $v,w\in V$. Definiere $A=M_B(s)$, $A'=M_{B'}(s)$, $T=T_B^{B'}$ und $x,y,x',y'\in K^n$ mit $v=\Phi_B(x)=\Phi_B(x')$, $w=\Phi_B(y)=\Phi_B(y')$. Dann ist $x=Tx'$, $y=Ty'$ und somit
	\begin{align}
		(x')^tA'\overline{y'}&\overset{\propref{6_2_6}}{=}s(v,w)\notag \\
		&\overset{\propref{6_2_6}}{=}x^tA\overline{y}\notag \\
		&= (Tx')^tA\overline{Ty'} \notag \\
		&= (x')^tT^tA\overline{T}\overline{y'}\notag
	\end{align} 
	Da $v,w\in V$ und somit $x',y'\in K$ beliebig waren, folgt $A=T^tA\overline{T}$.
\end{proof}

\begin{example}
	\proplbl{6_2_9}
	Sei $s$ das Standardskalarprodukt auf dem $K^n$ und $B=(b_1,...,b_n)$ eine Basis des $K^n$. Dann ist 
	\begin{align}
		M_B(s)=(T_{\mathcal{E}}^B)^t\cdot M_{\mathcal{E}}(s)\cdot \overline{T_{\mathcal{E}}^B}=B^t\cdot \mathbbm{1}_n\cdot \overline{B}=B^tB\notag
	\end{align}
	wobei $B=(b_1,...,b_n)\in\Mat_n(K)$.
\end{example}

\begin{proposition}
	\proplbl{6_2_10}
	Sei $s$ eine Sesquilinearform auf $V$. Dann sind äquivalent:
	\begin{itemize}
		\item Es gibt $0\neq v\in V$ mit $s(v,w)=0$ für alle $w\in V$.
		\item Es gibt $0\neq w\in V$ mit $s(v,w)=0$ für alle $v\in V$.
		\item Es gibt eine Basis $B$ von $V$ mit $\det(M_B(s))=0$.
		\item Für jede Basis $B$ von $V$ gilt $\det(M_B(s))=0$.
	\end{itemize}
\end{proposition}
\begin{proof}
	Sei $B$ eine Basis von $V$, $v=\Phi_B(x)$ und $A=M_B(s)$. Genau dann ist die (semilineare) Abbildung $w\mapsto s(v,w)$ die Nullabbildung, wenn $x^tA\overline{y}=0$ für alle $y\in K^n$, also wenn $0=x^tA$, d.h. $A^tx=0$. Somit ist $(1)$ genau dann erfüllt, wenn $A^t$ nicht invertierbar ist, also wenn $0=\det(A^t)=\det(A)$. Damit $(1)\Rightarrow (4)\Rightarrow (3)\Rightarrow (1)$ gezeigt und $(2)\iff (4)$ zeigt man analog.
\end{proof}

\begin{definition}[ausgeartet]
	Eine Sesquilinearform $s$ auf $V$ heißt \begriff{ausgeartet}, wenn eine der äquivalenten Bedingungen aus \propref{6_2_10} erfüllt ist, sonst \emph{nicht-ausgeartet}.
\end{definition}

\begin{definition}[symmetrisch, hermitesch]
	Eine Sesquilinearform $s$ auf $V$ heißt \begriff{symmetrisch}, wenn bzw. \begriff{hermitesch}, wenn
	\begin{align}
		s(x,y)=\overline{s(y,x)}\quad\text{ für alle }x,y\in V\notag
	\end{align}
	
	Eine Matrix $A\in\Mat_n(K)$ heißt \emph{symmetrisch} bzw. \emph{hermitesch}, wenn $A=A^*=\overline{A}^t=\overline{A^t}$.
\end{definition}

\begin{mathematica}[symmetrische bzw. hermitesche Matrizen]
	Wie für vieles Andere auch, hat Mathematica bzw. WolframAlpha auch dafür eine Funktion:
	\begin{align}
		\texttt{SymmetricMatrixQ[A]}\notag \\
		\texttt{HermitianMatrixQ[A]}\notag
	\end{align}
\end{mathematica}

\begin{proposition}
	\proplbl{6_2_13}
	Sei $s$ eine Sesquilinearform auf $V$ und $B$ eine Basis von $V$. Genau dann ist $s$ hermitesch, wenn $M_B(s)$ dies ist.
\end{proposition}
\begin{proof}
	$(\Rightarrow)$: klar aus Definition von $M_B(s)$. \\
	$(\Leftarrow)$: $x=\Phi_B^{-1}$, $y=\Phi_B^{-1}(w)$, $\overline{s(v,w)}=\overline{s(v,w)^t}=\overline{(x^tA\overline{y})^t}=y^t\overline{A^t}\overline{x}=s(w,v)$
\end{proof}

\begin{proposition}
	Für $A,B\in\Mat_n(K)$ und $S\in\GL_n(K)$ ist $(A+B)^*=A^*+B^*$, $(AB)^*=B^*A^*$, $(A^*)^*=A$ und $(S^{-1})^*=(S^*)^{-1}$.
\end{proposition}
\begin{proof}
	\propref{6_1_8}, III.1.14, III.1.15 %TODO: Verlinkung
\end{proof}

\chapter{Dualität}

\chapter{Moduln}

\part*{Anhang}
\addcontentsline{toc}{part}{Anhang}
\appendix
\patchcmd{\chapter}{\thispagestyle{plainChapter}}{\thispagestyle{fancy}}{}{}
%\titleformat{command}[shape]{format}{label}{sep}{before-code}[after-code]
%\titlespacing{command}{left}{before-sep}{after-sep}
\renewcommand{\chaptername}{Anhang}
\renewcommand{\thechapter}{\Alph{chapter}}
\titleformat{\chapter}[hang]{\bfseries}{\LARGE\chaptername\ \thechapter:}{0.5em}{\LARGE\bfseries}
\titlespacing{\chapter}{0pt}{-0.75cm}{0pt}
\renewcommand{\thesection}{\Alph{chapter}.\arabic{section}}

%from ntheorem.sty
%\def\thm@@thmline@name#1#2#3#4#5{%
%	\ifx\\#5\\%
%		\@dottedtocline{-2}{0em}{2.3em}%
%		{#1 \protect\numberline{#2}#3}%
%		{#4}
%	\else
%		\ifHy@linktocpage\relax\relax
%			\@dottedtocline{-2}{0em}{2.3em}%
%			{#1 \protect\numberline{#2}#3}%
%			{\hyper@linkstart{link}{#5}{#4}\hyper@linkend}%
%		\else
%			\@dottedtocline{-2}{0em}{2.3em}%
%			{\hyper@linkstart{link}{#5}%
%			{#1 \protect\numberline{#2}#3}\hyper@linkend}%
%			{#4}%
%		\fi
%	\fi
%}

\makeatletter
%update ntheorem macro to provide space between theorem numbers and any optional comment
\renewcommand{\thm@@thmline@name}[5]{%-       		
	\def\thm@@thmline@name@tmp{%
		\if\relax\detokenize{#3}\relax\else%
			{\hspace*{2.2ex}#3}%
		\fi%
	}%
	\ifx\\#5\\%
		\@dottedtocline{-2}{0em}{2.3em}%
		{#1 \protect\numberline{#2\thm@@thmline@name@tmp}{}}%
		{#4}
	\else
		\ifHy@linktocpage\relax\relax
			\@dottedtocline{-2}{0em}{1.3em}%
			{#1 \protect\numberline{#2:}\thm@@thmline@name@tmp}%
			{\hyper@linkstart{link}{#5}{#4}\hyper@linkend}%
		\else
			\@dottedtocline{-2}{0em}{1.3em}%
			{#1 \protect\numberline{#2:}\thm@@thmline@name@tmp}%
			{\hyper@linkstart{link}{#5}{#4}\hyper@linkend}%
		\fi
	\fi
}
\makeatother

\chapter{Listen}
\section{Liste der Theoreme}
\theoremlisttype{allname}
\listtheorems{theorem}

\pagebreak
\section{Liste der benannten Sätze}
\theoremlisttype{optname}
\listtheorems{proposition}

\printglossary[type=\acronymtype]
\addcontentsline{toc}{chapter}{Akronyme}

\printindex

\end{document}