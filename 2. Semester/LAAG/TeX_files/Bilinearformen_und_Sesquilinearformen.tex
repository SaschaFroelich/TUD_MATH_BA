\section{Bilinearformen und Sesquilinearformen}

Sei $K=\real$ oder $K=\comp$.

\begin{definition}[Bilinearform, Sesquilinearform]
	Eine \begriff{Bilinearform} ($K=\real$) bzw. \begriff{Sesquilinearform} ($K=\comp$) ist eine Abbildung $s:V\times V\to K$ für die gilt:
	\begin{itemize}
		\item Für $x,x',y\in V$ ist $s(x+x',y)=s(x,y)+s(x',y)$
		\item Für $x,y,y'\in V$ ist $s(x,y+y')=s(x,y)+s(x,y')$
		\item Für $x,y\in V$, $\lambda\in K$ ist $s(\lambda x,y)=\lambda s(x,y)$
		\item Für $x,y\in V$, $\lambda\in K$ ist $s(x,\lambda y)=\kringel{white}{\overline{\lambda}} s(x,y)$
	\end{itemize}
\end{definition}

\begin{remark}
	Im Fall $K=\real$ ist $\lambda=\overline{\lambda}$. Wir werden der Einfachheit halber auch in diesem Fall von Sesquilinearformen sprechen, vgl. \propref{6_1_12}
\end{remark}

\begin{example}
	Für $A=(a_{ij})_{i,j}\in\Mat_n(K)$ ist $s_A:K^n\times K^n\to K^n$ gegeben durch
	\begin{align}
		s_A(x,y)=x^tA\overline{y}=x^t\left( \sum_{j=1}^n a_{ij}\overline{y}_j\right)_i=\sum_{i,j=1}^n a_{ij}x_i\overline{y}_j\notag
	\end{align}
	eine Sesquilinearform auf $V=K^n$.
\end{example}

\begin{definition}
	Sei $s$ eine Sesquilinearform auf $V$ und $B=(v_1,...,v_n)$ eine Basis von $V$. Die \begriff[Sesquilinearform!]{darstellende Matrix} von $s$ bzgl. $B$ ist
	\begin{align}
		M_B(s)=(s(v_i,v_j))_{i,j}\in\Mat_n(K)\notag
	\end{align}
\end{definition}

\begin{example}
	Die darstellende Matrix des Standardskalarprodukts $s=s_{\mathbbm{1}_n}$ auf den Standardraum $V=K^n$ bzgl. der Standardbasis $\mathcal{E}$ ist
	\begin{align}
		M_{\mathcal{E}}(s)=\mathbbm{1}_n\notag
	\end{align}
\end{example}

\begin{lemma}
	\proplbl{6_2_6}
	Seien $v,w\in V$. Mit $x=\Phi_B^{-1}(v)$, $y=\Phi_B^{-1}(w)$ und $A=M_B(s)$ ist $s(v,w)=x^tA\overline{y}=s_A(x,y)$.
\end{lemma}
\begin{proof}
	Achtung: $v_i$ beschreibt das $i$-te Element der Basis $B$!\\
	$s(v,w)=s(\sum_{i=1}^n x_iv_i,\sum_{j=1}^n y_jv_j)=\sum_{i,j=1}^n x_i\overline{y}s(v,v_j)=x^tA\overline{y}$
\end{proof}

\begin{proposition}
	ISei $B$ eine Basis von $V$. Die Abbildung $s\mapsto M_B(s)$ ist eine Bijektion zwischen den Sesquilinearformen auf $V$ und $\Mat_n(K)$.
\end{proposition}
\begin{proof}
	\begin{itemize}
		\item injektiv: \propref{6_2_6}
		\item surjektiv: Für $A\in\Mat_n(K)$ wird durch $s(v,w)=\Phi_B^{-1}(v)^t\cdot A\cdot \overline{\Phi_B^{-1}(w)}$ eine Sesquilinearform auf $V$ mit $M_B(s)=(s(v_i,w_j))_{i,j}= (e_i^tA\overline{e_j})_{i,j}=(e_iAe_j)_{i,j}=A$ definiert.
	\end{itemize}
\end{proof}

\begin{proposition}[Transformationsformel]
	\proplbl{6_2_8}
	Seien $B$ und $B'$ Basen von $V$ und $s$ eine Sesquilinearform auf $V$. Dann gilt:
	\begin{align}
		M_{B'}(s)=(T_B^{B'})^t\cdot M_B(s)\cdot \overline{T_B^{B'}}\notag
	\end{align}
\end{proposition}
\begin{proof}
	Seien $v,w\in V$. Definiere $A=M_B(s)$, $A'=M_{B'}(s)$, $T=T_B^{B'}$ und $x,y,x',y'\in K^n$ mit $v=\Phi_B(x)=\Phi_B(x')$, $w=\Phi_B(y)=\Phi_B(y')$. Dann ist $x=Tx'$, $y=Ty'$ und somit
	\begin{align}
		(x')^tA'\overline{y'}&\overset{\propref{6_2_6}}{=}s(v,w)\notag \\
		&\overset{\propref{6_2_6}}{=}x^tA\overline{y}\notag \\
		&= (Tx')^tA\overline{Ty'} \notag \\
		&= (x')^tT^tA\overline{T}\overline{y'}\notag
	\end{align} 
	Da $v,w\in V$ und somit $x',y'\in K$ beliebig waren, folgt $A=T^tA\overline{T}$.
\end{proof}

\begin{example}
	\proplbl{6_2_9}
	Sei $s$ das Standardskalarprodukt auf dem $K^n$ und $B=(b_1,...,b_n)$ eine Basis des $K^n$. Dann ist 
	\begin{align}
		M_B(s)=(T_{\mathcal{E}}^B)^t\cdot M_{\mathcal{E}}(s)\cdot \overline{T_{\mathcal{E}}^B}=B^t\cdot \mathbbm{1}_n\cdot \overline{B}=B^tB\notag
	\end{align}
	wobei $B=(b_1,...,b_n)\in\Mat_n(K)$.
\end{example}

\begin{proposition}
	\proplbl{6_2_10}
	Sei $s$ eine Sesquilinearform auf $V$. Dann sind äquivalent:
	\begin{itemize}
		\item Es gibt $0\neq v\in V$ mit $s(v,w)=0$ für alle $w\in V$.
		\item Es gibt $0\neq w\in V$ mit $s(v,w)=0$ für alle $v\in V$.
		\item Es gibt eine Basis $B$ von $V$ mit $\det(M_B(s))=0$.
		\item Für jede Basis $B$ von $V$ gilt $\det(M_B(s))=0$.
	\end{itemize}
\end{proposition}
\begin{proof}
	Sei $B$ eine Basis von $V$, $v=\Phi_B(x)$ und $A=M_B(s)$. Genau dann ist die (semilineare) Abbildung $w\mapsto s(v,w)$ die Nullabbildung, wenn $x^tA\overline{y}=0$ für alle $y\in K^n$, also wenn $0=x^tA$, d.h. $A^tx=0$. Somit ist $(1)$ genau dann erfüllt, wenn $A^t$ nicht invertierbar ist, also wenn $0=\det(A^t)=\det(A)$. Damit $(1)\Rightarrow (4)\Rightarrow (3)\Rightarrow (1)$ gezeigt und $(2)\iff (4)$ zeigt man analog.
\end{proof}

\begin{definition}[ausgeartet]
	Eine Sesquilinearform $s$ auf $V$ heißt \begriff{ausgeartet}, wenn eine der äquivalenten Bedingungen aus \propref{6_2_10} erfüllt ist, sonst \emph{nicht-ausgeartet}.
\end{definition}

\begin{definition}[symmetrisch, hermitesch]
	Eine Sesquilinearform $s$ auf $V$ heißt \begriff{symmetrisch}, wenn bzw. \begriff{hermitesch}, wenn
	\begin{align}
		s(x,y)=\overline{s(y,x)}\quad\text{ für alle }x,y\in V\notag
	\end{align}
	
	Eine Matrix $A\in\Mat_n(K)$ heißt \emph{symmetrisch} bzw. \emph{hermitesch}, wenn $A=A^*=\overline{A}^t=\overline{A^t}$.
\end{definition}

\begin{mathematica}[symmetrische bzw. hermitesche Matrizen]
	Wie für vieles Andere auch, hat Mathematica bzw. WolframAlpha auch dafür eine Funktion:
	\begin{align}
		\texttt{SymmetricMatrixQ[A]}\notag \\
		\texttt{HermitianMatrixQ[A]}\notag
	\end{align}
\end{mathematica}

\begin{proposition}
	\proplbl{6_2_13}
	Sei $s$ eine Sesquilinearform auf $V$ und $B$ eine Basis von $V$. Genau dann ist $s$ hermitesch, wenn $M_B(s)$ dies ist.
\end{proposition}
\begin{proof}
	$(\Rightarrow)$: klar aus Definition von $M_B(s)$. \\
	$(\Leftarrow)$: $x=\Phi_B^{-1}$, $y=\Phi_B^{-1}(w)$, $\overline{s(v,w)}=\overline{s(v,w)^t}=\overline{(x^tA\overline{y})^t}=y^t\overline{A^t}\overline{x}=s(w,v)$
\end{proof}

\begin{proposition}
	Für $A,B\in\Mat_n(K)$ und $S\in\GL_n(K)$ ist $(A+B)^*=A^*+B^*$, $(AB)^*=B^*A^*$, $(A^*)^*=A$ und $(S^{-1})^*=(S^*)^{-1}$.
\end{proposition}
\begin{proof}
	\propref{6_1_8}, III.1.14, III.1.15 %TODO: Verlinkung
\end{proof}