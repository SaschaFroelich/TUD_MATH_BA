\section{Tensorprodukte}

\begin{definition}[billineare Abbildung]
	Eine Abbildung $\xi:V\times W\to U$ ist \begriff[Abbildung!]{bilinear}, wenn für jedes $v\in V$ die Abbildung 
	\begin{align}
		\begin{cases}
		W\to U \\ w\mapsto \xi(v,w)
		\end{cases}\notag
	\end{align}
	und für jedes $w\in W$ die Abbildung
	\begin{align}
	\begin{cases}
	V\to U \\ v\mapsto \xi(v,w)
	\end{cases}\notag
	\end{align}
	linear sind.
	
	Wir definieren
	\begin{align}
		\Bil_K(V,W,U)=\{\xi\in\Abb(V\times W,U)\mid \xi\text{ bilinear}\}\notag
	\end{align}
\end{definition}

\begin{example}
	Seien $V=W=K[t]_{\le d}$, $U=K[t]_{\le 2d}$. Die Abbildung
	\begin{align}
	\xi:\begin{cases}
	V\times W\to U \\ (f.g)\mapsto fg
	\end{cases}\quad\text{ ist bilinear}\notag
	\end{align}
	Wir sehen, dass $\Image(\xi)$ im Allgemeinen kein Untervektorraum von $U$ ist. Ist zum Beispiel $K=\ratio$, $d=1$, so liegen $t^2=\xi(t,t)$ und $-2=\xi(-2,1)$ im $\Image(\xi)$ nicht jedoch $t^2-2$, denn wäre $t^2-2=fg$ mit $f,g\in\ratio[t]$ linear, so hätte $t^2-2$ eine Nullstelle in $\ratio$, aber $\sqrt{2}\notin\ratio$.
\end{example}

\begin{lemma}
	$\Bil_K(V,W,U)$ bildet einen Untervektorraum des $K$-Vektorraum $\Abb(V\times W,U)$.
\end{lemma}
\begin{proof}
	klar, zum Beispiel
	\begin{align}
		(\xi+\xi')(\lambda v,w)=\xi(\lambda v,w)+\xi'(\lambda v,w)=\lambda\xi(v,w)+\lambda\xi(v,w)= \lambda(\xi+\xi')(v,w)\notag
	\end{align}
\end{proof}

\begin{lemma}
	Ist $\xi\in\Bil_K(V,W,U)$ und $f\in\Hom_K(U,U')$ für einen $K$-Vektorraum, so ist 
	\begin{align}
		f\circ \xi\in\Bil_K(V,W,U')\notag
	\end{align}
\end{lemma}
\begin{proof}
	klar, zum Beispiel
	\begin{align}
		(f\circ\xi)(\lambda v,w)=f(\xi(\lambda v,w))=f(\lambda\xi(v,w))=\lambda\cdot(f\circ\xi)(v,w)\notag
	\end{align}
\end{proof}

\begin{lemma}
	\proplbl{3_6_5}
	Sei $(v_i)_{i\in I}$ eine Basis von $V$ und $(w_j)_{j\in J}$ eine Basis von $W$. Zu jeder Familie $(u_{ij})_{(i,j)\in I\times J}$ in $U$ gibt es genau ein $\xi\in\Bil_K(V,W,U)$ mit
	\begin{align}
		\xi(v_i,w_i)=u_{ij}\quad\forall i\in I, j\in J\notag
	\end{align}
\end{lemma}
\begin{proof}
	\begin{itemize}
		\item Eindeutigkeit: Ist $\xi$ bilinear, $v=\sum_{i\in I} \lambda_i v_i$, $w=\sum_{j\in J} \mu_j w_j$ so ist 
		\begin{align}
			\xi(v,w)&=\xi\left(\sum_{i\in I} \lambda_i v_i,\sum_{j\in J}\mu_j w_j\right) \notag \\
			&= \sum_{i\in I}\lambda_i\xi\left(v_i,\sum_{j\in J}\mu_j w_j\right)\notag \\
			&= \sum_{i,j}\lambda_i\mu_j u_{ij}
		\end{align}
		durch die Familie $(u_{ij})_{i,j}$ bestimmt.
		\item Existenz: Wird $\xi$ durch (1) definiert, so ist $\xi$ bilinear: Für festes $w=\sum_{j\in J}\mu_j w_j$ ist
		\begin{align}
			\begin{cases}
			V&\to U \\ v=\sum\limits_{i\in I}\lambda_i v_i&\mapsto \xi(v,w)=\sum\limits_{i\in I}\lambda_i\left(\sum\limits_{j\in J}\mu_j u_{ij}\right)
			\end{cases}\notag
		\end{align}
		linear (LAAG1 III.5.1), analog für festes $v$. %TODO: Verlinkung
	\end{itemize}
\end{proof}

\begin{definition}[Tensorprodukt]
	Ein \begriff{Tensorprodukt} von $V$ und $W$ ist ein Paar $(T,\tau)$ bestehend aus einem $K$-Vektorraum $T$ und einer bilinearen Abbildung $\tau\in\Bil_K(V,W,T)$ welche die folgende \begriff{universelle Eigenschaft} erfüllt: \\
	\textit{Ist $U$ ein weiterer $K$-Vektorraum und $\xi\in\Bil_K(V,W,U)$ so gibt es genau ein $\xi_\otimes\in\Hom_K(T,U)$ mit $\xi=\xi_\otimes\circ\tau$.}
	%TODO: Abbildung
%	V\times W -> \tau -> T
%									|
%					\xi				\/ \xi_\otimes
%									U
\end{definition}

\begin{lemma}
	\proplbl{3_6_7}
	Sind $(T,\tau)$ und $(T',\tau')$ Tensorprodukte von $V$ und $W$, so gibt es einen eindeutig bestimmten Isomorphismus $\Theta:T\to T'$ mit $\tau'=\Theta\circ\tau$.
	%TODO: Abbildung
\end{lemma}
\begin{proof}
	Da $(T,\tau)$ die universelle Eigenschaft erfüllt, gibt es ein eindeutig bestimmtes $\Theta=(\tau')_\otimes\in\Hom_K(T,T')$ mit $\tau'=\Theta\circ\tau$. Analog gibt es $\Theta'\in\Hom_K(T',T)$ mit $\tau=\Theta'\circ\tau'$. Es folgt, dass $\tau=\Theta'\circ\tau'= \Theta'\circ\Theta\circ\tau$. Da auch $\tau=\id_T\circ\tau$ liefert die Eindeutigkeitsaussage in der universellen Eigenschaft von $(T,\tau)$, für $U=T$, $\xi=\tau$, dass $\Theta\circ\Theta'=\id_T$. Analog sieht man, dass $\Theta\circ\Theta'=\id_{T'}$. Somit ist $\Theta$ ein Isomorphismus.
\end{proof}

\begin{definition}[Vektorraum mit Basis $X$]
	\proplbl{3_6_8}
	Sei $X$ eine Menge. Der $K$-\begriff{Vektorraum mit Basis $X$} ist der Untervektorraum $V=\Span_K((\delta_x)_{x\in X})$ des $K$-Vektorraum $\Abb(X,K)$ ($\delta_x(y)=\delta_{x,y}=\begin{cases}1&x=y\\0&x\neq y\end{cases}$)
\end{definition}

\begin{lemma}
	Sei $X$ eine Menge und $V$ der $K$-Vektorraum mit Basis $X$. Dann ist $V$ ein $K$-Vektorraum und $(\delta_x)_{x\in X}$ ist eine Basis von $V$.
\end{lemma}
\begin{proof}
	Zu zeigen ist nur, dass $(\delta_x)_{x\in X}$ linear unabhängig ist. Ist $f=\sum_{x\in X}\lambda_x \delta_x$, $\lambda_x\in K$, fast alle gleich 0, und $f=0$, so ist $\lambda_x=f(x)=0$ für jedes $x\in X$. 
\end{proof}

\begin{lemma}
	\proplbl{3_6_10}
	Sei $(v_i)_{i\in I}$ eine Basis von $V$ und $(w_j)_{j\in J}$ eine Basis von $W$. Sei $T$ der $K$-Vektorraum mit der Basis $I\times J$ (im Sinne von \propref{3_6_8}) und $\tau:V\times W\to T$ die bilineare Abbildung gegeben durch $(v_i,w_j)\mapsto \delta_{i,j}$, vergleiche \propref{3_6_5}. Dann ist $(T,\tau)$ ein Tensorprodukt von $V$ und $W$.
\end{lemma}
\begin{proof}
	Wir schreiben $v_i\otimes w_j$ für $\delta_{i,j}$. Sei $U$ ein weiterer $K$-Vektorraum und $\xi\in\Bil_K(V,W,U)$. Da $(v_i\otimes w_j)_{(ij)\in I\times J}$ eine Basis von $T$ ist, gibt es genau ein $\xi_\otimes\in\Hom_K(T,U)$ mit $\xi_\otimes(v_i\otimes w_j)=\xi(v_i,w_j)$ für alle $i,j$, also mit $\xi_\otimes\circ\tau=\xi$ nach \propref{3_6_5}. Die universelle Eigenschaft ist somit erfüllt.
\end{proof}

\begin{proposition}
	Es gibt ein bis auf Isomorphie (im Sinne von \propref{3_6_7}) eindeutig bestimmtes Tensorprodukt
	\begin{align}
		(V\otimes_K W,\otimes)\notag
	\end{align}
	von $V$ und $W$. Sind $V$ und $W$ endlichdimensional, so ist
	\begin{align}
		\dim_K(V\otimes_K W)=\dim_K(V)\cdot\dim_K(W)\notag
	\end{align}
\end{proposition}
\begin{proof}
	\propref{3_6_10} und \propref{3_6_7}
\end{proof}

\begin{example}
	Durch die Wahl der Standardbasis erhält man einen kanonischen Isomorphismus $K^m\otimes_K K^n\cong\Mat_{m\times n}(K)$.
\end{example}