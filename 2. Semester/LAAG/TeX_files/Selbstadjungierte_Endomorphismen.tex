\section{Selbstadjungierte Endomorphismen}

Sei $V$ ein euklidischer bzw. unitärer Vektorraum und $f\in\End_K(V)$.

\begin{definition}[selbstadjungiert]
	$f$ ist \begriff{selbstadjungiert}, wenn
	\begin{align}
		\skalar{f(x)}{y}=\skalar{x}{f(y)}\quad\forall x,y\in V\notag
	\end{align}
\end{definition}

\begin{proposition}
	\proplbl{6_6_2}
	Sei $B$ eine Orthonormalbasis von $V$. Genau dann ist $f$ selbstadjungiert, wenn $M_B(f)$ hermitesch ist.
\end{proposition}
\begin{proof}
	Seien $A=M_B(f),v=\Phi_B(x),w=\Phi_B(y)$. Es ist 
	\begin{align}
		\skalar{f(v)}{w}=(Ax)^t\overline{y}=x^tA^t\overline{y}\notag \\
		\skalar{v}{f(w)}=x^t\overline{Ay}=x^t\overline{A}\overline{y}\notag
	\end{align}
	Somit ist $\skalar{f(v)}{w}=\skalar{v}{f(w)}$ genau dann, wenn $A^t=\overline{A}$, d.h. $A=A^*$, also $A$ hermitesch.
\end{proof}

\begin{lemma}
	\proplbl{6_6_3}
	Ist $f$ selbstadjungiert und $\lambda$ ein Eigenwert von $f$, so ist $\lambda\in\real$.
\end{lemma}
\begin{proof}
	Ist $0\neq x\in V$ mit $f(x)=\lambda x$, so ist 
	\begin{align}
		\lambda\skalar{x}{x}=\skalar{f(x)}{x}=\skalar{x}{f(x)}=\overline{\lambda}\skalar{x}{x}\notag
	\end{align}
	und mit $\skalar{x}{x}\neq 0$ folgt $\lambda=\overline{\lambda}$, also $\lambda\in\real$.
\end{proof}

\begin{proposition}
	\proplbl{6_6_4}
	Ist $f$ selbstadjungiert, so ist $\chi_f\in\real[t]$ und $\chi_f$ zerfällt über $\real$ in Linearfaktoren.
\end{proposition}
\begin{proof}
	Sei $B$ eine Orthonormalbasis von $V$. Nach \propref{6_6_2} ist $A=M_B(f)\in\Mat_n(K)\subseteq \Mat_n(\comp)$ hermitesch. Da $\comp$ algebraisch abgeschlossen ist, ist $\chi_f(t)\prod_{i=1}^{n} (t-\lambda_i)$ mit $\lambda_1,...,\lambda_n\in\comp$. Nach \propref{6_6_3} ist aber schon $\lambda_1,...,\lambda_n\in\real$. Somit zerfällt $\chi_f\chi_A\in\real[t]$ über $\real$ in Linearfaktoren.
\end{proof}

\begin{theorem}
	\proplbl{6_6_5}
	Ist $f$ selbstadjungiert, so besitzt $V$ eine Orthonormalbasis aus Eigenvektoren von $f$.
\end{theorem}
\begin{proof}
	Induktion über $n=\dim_K(V)$. \\
	\emph{$n=0$}: klar \\
	\emph{$n-1\to n$}: Nach \propref{6_6_4} hat $f$ einen reellen Eigenwert $\lambda\in\real$. Wähle $x_1\in V$ mit $f(x_1)=\lambda x_1$ und $\Vert x_1\Vert=1$. Sei $W=K\cdot x_1$. Für $y\in W^\perp$ ist 
	\begin{align}
		\skalar{x_1}{f(y)}=\skalar{f(x_1)}{y}=\lambda\skalar{x_1}{y}=0\notag
	\end{align}
	und folglich ist $W^\perp$ $f$-invariant. Nach \propref{6_4_11} ist $V=W\oplus W^\perp$ und $f\vert_{W^\perp}$ ist wieder selbstadjungiert. Nach Induktionshypothese hat $W^\perp$ eine Orthonormalbasis $(x_1,...,x_n)$ aus Eigenvektoren von $f\vert_{W^\perp}$. Da $V=W\oplus W^\perp$ und $W\perp W^\perp$ ist $(x_1,...,x_n)$ eine Orthonormalbasis von $V$ aus Eigenvektoren von $f$.
\end{proof}

\begin{conclusion}
	Jeder selbstadjungierte  Endomorphismus eines euklidischen oder unitären Vektorraums ist diagonalisierbar.
\end{conclusion}

\begin{conclusion}
	\proplbl{6_6_7}
	Ist 
	\begin{itemize}
		\item $f$ selbstadjungiert ($K=\comp$ oder $\real$)
		\item $f$ unitär ($K=\comp$)
	\end{itemize}
	so ist 
	\begin{align}
		V=\bigoplus_{\lambda\in K} \Eig(f,\lambda)\notag
	\end{align}
	eine Zerlegung von $V$ in paarweise orthogonale Untervektorräume.
\end{conclusion}
\begin{proof}
	Nach \propref{6_5_9} bzw. \propref{6_6_5} existiert eine Orthonormalbasis $B$ aus Eigenvektoren. Insbesondere ist $f$ diagonalisierbar, also
	\begin{align}
		V=\bigoplus_{\lambda\in K} \Eig(f,\lambda)\notag
	\end{align}
	Zu jedem $\lambda$ gibt es eine Teilfamilie von $B$ die eine Basis von $\Eig(f,\lambda)$ bildet. Da $B$ eine Orthonormalbasis ist, folgt, dass die Eigenräume paarweise orthogonal sind.
\end{proof}

\begin{remark}
	Um eine Orthonormalbasis aus Eigenvektoren wie in \propref{6_5_9} oder \propref{6_6_5} zu bestimmen, kann man entweder wie im Induktionsbeweis vorgehen, oder man bestimmt zunächst Basen $B$ von $\Eig(f,\lambda_i)$, $i=1,...,n$ und orthonormalisiert diese mit \propref{6_4_9} zu Basen $B'$. Nach \propref{6_6_7} ist $\bigcup B'$ dann eine Orthonormalbasis von $V$ aus Eigenvektoren von $f$.
\end{remark}