\section{Das charakteristische Polynom}

\begin{proposition}
	\proplbl{satz_det_null}
	Sei $\lambda\in K$. Genau dann ist $\lambda$ ein EW von $f$, wenn $\det(\lambda\cdot\id_V-f)=0$.
\end{proposition}
\begin{proof}
	Da $\Eig(f,\lambda)=\Ker(\lambda\cdot\id_V-f)$ ist $\lambda$ genau dann ein EW von $f$, wenn $\dim_K(\Ker(\lambda\cdot\id_V-f))>0$, also wenn $\lambda\cdot\id_V-f\notin\Aut_K(V)$. Nach IV.4.6 bedeutet dies, dass $\det(\lambda\cdot\id_V-f)=0$ %TODO: Verlinkung setzen
\end{proof}

\begin{definition}[charakteristisches Polynom]
	Das \begriff{charakteristische Polynom} einer Matrix $A\in\Mat_n(K)$ ist die Determinante der Matrix $t\cdot \mathbbm{1}_n-A\in\Mat_n(K[t])$. 
	\begin{align}
		\chi_A(t)&=\det(t\cdot \mathbbm{1}_n-A)\in K[t] \notag
	\end{align}
	Das charakteristische Polynom eines Endomorphismus $f\in\End_K(V)$ ist $\chi_f(t)=\chi_{M_B(f)}(t)$, wobei $B$ eine Basis von $V$ ist.
\end{definition}

\begin{proposition}
	\proplbl{satz_2_3}
	Sind $A,B\in\Mat_n(K)$ mit $A\sim B$, so ist $\chi_A=\chi_B$. Insbesondere ist $\chi_f$ wohldefiniert.
\end{proposition}
\begin{proof}
	Ist $B=SAS^{-1}$ mit $S\in\GL_n(K)$, so ist $t\cdot \mathbbm{1}_n-B = S(t\cdot \mathbbm{1}_n-A)S^{-1}$, also $t\cdot \mathbbm{1}_n-B\sim t\cdot \mathbbm{1}_n-A$ und ähnliche Matrizen haben die selben Determinante (IV.4.4). \\
	Sind $B,B'$ Basen von $V$, so sind $M_B(f)\sim M_{B'}(f)$, also $\chi_{M_B(f)}=\chi_{M_{B'}(f)}$ %TODO: Verlinkung setzen
\end{proof}

\begin{lemma}
	\proplbl{lemma_chi_det}
	Für $\lambda\in K$ ist $\chi_f(\lambda)=\det(\lambda\cdot\id_V-f)$.
\end{lemma}
\begin{proof}
	Sei $B$ eine Basis von $V$ und $A=M_B(f)=(a_{ij})_{i,j}$. Dann ist $M_B(\lambda\cdot\id_V-f)= \lambda\cdot \mathbbm{1}_n-A$. Aus IV.2.8 und I.6.8 folgt $\det(t\cdot \mathbbm{1}_n-A)(\lambda)=\det(\lambda\cdot \mathbbm{1}_n-A)$. Folglich ist 
	\begin{align}
		\chi_f(\lambda)&=\chi_A(\lambda)\notag \\
		&=\det(t\cdot \mathbbm{1}_n-A)(\lambda)\notag \\
		&=\det(\lambda\cdot \mathbbm{1}_n-A)\notag \\
		&= \det(\lambda\cdot\id_V-f) \notag
	\end{align}
\end{proof}

\begin{proposition}
	\proplbl{satz_chi_polynom}
	Sei $\dim_K(V)=n$ und $f\in\End_K(V)$. Dann ist $\chi_f(t)=\sum_{i=0}^n \alpha_i t^i$ ein Polynom vom Grad $n$ mit 
	\begin{align}
		\alpha_n&=1\notag \\
		\alpha_{n-1}&=-\tr(f) \notag \\
		\alpha_0 &= (-1)^n\cdot\det(f) \notag
	\end{align}
	Die Nullstellen von $\chi_f$ sind genau die EW von $f$.
\end{proposition}
\begin{proof}
	Sei $B$ eine Basis von $V$ und $A=M_B(f)=(a_{ij})_{i,j}$. Wir erinnern uns daran, dass $\tr(f)=\tr(A=\sum_{i=1}^n a_{ii}$. Es ist $\chi_f(t)=\det(t-\cdot 1_n-A)=\sum_{\sigma\in S_n}\sgn(\sigma)\prod_{i=1}^n (t\delta_{i,\sigma(i)}-a_{i,\sigma(i)})$. \\
	Der Summand für \emph{$\sigma=\id$} ist $\prod_{i=1}^n (t-a_{ii})=t^n+\sum_{i=1}^n (-a_{ii})t^{n-1}+...+\prod_{i=1}^n(-a_{ii})$ \\
	Für \emph{$\sigma\neq\id$} ist $\sigma(i)\neq i$ für mindestens zwei $i$, der entsprechende Summand hat also Grad höchstens $n-2$. Somit haben $\alpha_n$ und $\alpha_{n-1}$ die oben behauptete Form, und $\alpha_0=\chi_A(0)=\det(-A)=(-1)^n\cdot\det(f)$. \\
	Die Aussage über die Nullstellen von $\chi_f$ folgt aus \propref{satz_det_null} und \propref{lemma_chi_det}.
\end{proof}

\begin{conclusion}
	Ist $\dim_K(V)=n$, so hat $f$ höchstens $n$ Eigenwerte.
\end{conclusion}
\begin{proof}
	\propref{satz_chi_polynom} und I.6.10 %TODO: Verlinkung
\end{proof}

\begin{definition}[normiertes Polynom]
	Ein Polynom $0\neq P\in K[t]$ mit Leitkoeffizient 1 heißt \begriff{normiert}.
\end{definition}

\begin{example}
	\proplbl{beispiel_2_8}
	\begin{enumerate}
		\item Ist $A=(a_{ij})_{i,j}$ eine obere Dreiecksmatrix, so ist $\chi_A(t)=\prod_{i=1}^n (t-a_{ii})$, vgl. IV.2.9.c \\ %TODO: Verlinkung
		Insbesondere ist $\chi_{1_n}(t)=(t-1)^n$, $\chi_0(t)=t^n$
		\item Für eine Blockmatrix $A=\begin{pmatrix}A_1&B \\ 0&A_2\end{pmatrix}$ mit quadratischen Matrizen $A_1,A_2$ ist $\chi_A=\chi_{A_1}\cdot \chi_{A_2}$ vgl. IV.2.9.e %TODO: Verlinkung
		\item Für
		\begin{align}
			\begin{pmatrix}
			0&...&...&...&0&-c_0  \\ 
			1& \ddots&\;&\;&\vdots&\vdots  \\ 
			0&\ddots&\ddots&\;&\vdots&\vdots  \\ 
			\vdots&\ddots&\ddots&\ddots&\vdots&\vdots  \\ 
			0&...&0&1&0&-c_{n-1} 
			\end{pmatrix} \quad c_0,...,c_{n-1}\in K \notag
		\end{align}
		ist $\chi_A(t)=t^n+\sum_{i=0}^{n-1} c_i t^i$ \\
		Man nennt diese Matrix die Begleitmatrix zum normierten Polynom $P=t^n+\sum_{i=0}^{n-1} c_i t^i$ und schreibt $M_P:=A$
	\end{enumerate}
\end{example}