\section{Die duale Abbildung}

Sei $f\in\Hom_K(V,W)$.

\begin{remark}
	Ist $\phi\in W^*=\Hom_K(W,K)$ eine Linearform auf $W$, so ist $\phi\circ f\in \Hom_K(V,K)=V^*$ eine Linearform auf $V$.
	%TODO: Bild von Pascal
\end{remark}

\begin{center}
	\begin{tikzcd}
		V \arrow[r, "f"] \arrow[dr, dashrightarrow, "f^{\ast}(\phi)"]
		& W \arrow[d, "\phi"]\\
		& K\\
		V^{\ast} & \arrow[l, "f^{\ast}"] W^{\ast}
	\end{tikzcd}
\end{center}

\begin{definition}[duale Abbildung]
	Die zu $f$ duale Abbildung ist
	\begin{align}
		f^*:
		\begin{cases}
			W^*\to V^* \\
			\phi\mapsto \phi\circ f
		\end{cases} \notag
	\end{align}
\end{definition}

\begin{lemma}
	Es ist $f^*\in\Hom_K(W^*,V^*)$.
\end{lemma}
\begin{proof}
	Sind $\phi,\psi\in W^*$ und $\lambda\in K$ ist 
	\begin{align}
		f^*(\phi+\psi) &= (\phi+\psi)\circ f \notag \\
		&= \phi\circ f + \psi\circ f \notag \\
		&= f^*(\phi) + f^*(\psi) \notag \\
		f^*(\lambda\phi) &= (\lambda\phi)\circ f \notag \\
		&= \lambda\cdot(\phi\circ f) \notag \\
		&= \lambda\cdot f^*(\phi) \notag
	\end{align}
\end{proof}

\begin{proposition}
	\proplbl{3_3_4}
	Sind $B=(x_1,...,x_n)$ und $C=(y_1,...,y_m)$ Basen von $V$ bzw. $W$, so ist
	\begin{align}
		M_{B^*}^{C^*}(f^*)=\left(M_C^B(f) \right)^t\notag
	\end{align}
\end{proposition}
\begin{proof}
	Sei $A=M_C^B(f)=(a_{ij})_{i,j}$ und $B=M_{B^*}^{C^{*}}(f^*)=(b_{ji})_{j,i}$. Dann ist $f(x_j)=\sum_{i=1}^m a_{ij}y_i$, also $a_{ji}=y_i^*(f(x_j))=f^*(y_i^*)(x_j)$ und $f^*(y_i^*)=\sum_{j=1}^n b_{ji}x_j^*$, also $b_{ji}=f^*(y_i^*)(x_j)=a_{ij}$.
\end{proof}

\begin{conclusion}
	\proplbl{3_3_5}
	Sind $V$ und $W$ endlichdimensional, und identifizieren wir $V=V^{**}$ und $W=W^{**}$, so ist $f=f^{**}$, das heißt $\iota\circ f=f^{**}\circ\iota$.
	%TODO: Bild von Pascal
	
\end{conclusion}

\begin{center}
	\begin{tikzcd}
		V \arrow[r, "f"] \arrow[d, "\iota_V \cong"] 
		& W \arrow[d, "\iota_W \cong"] \\
		V^{\ast \ast} \arrow[r, "f^{\ast \ast}"] 
		& W^{\ast \ast}
	\end{tikzcd}
\end{center}

\begin{proof}
	Seien $B$ und $C$ Basen von $V$ bzw. $W$. Unter der Identifizierung ist $B^{**}=B$ und $C=C^{**}$, das heißt $\iota(x_i)=x_i^{**}$ bzw. $\iota(y_j)=y_j^{**}$, denn $\iota(x_i)(x_j^*)=x_j^*(x_i)=\delta_{ij} = x_i^{**}(x_j^*)\quad\forall i,j$ und somit 
	\begin{align}
		M_C^B(f^{**}) \overset{\propref{3_3_4}}{=} \left( M_{B^*}^{C^*}(f^*)\right)^t \overset{\propref{3_3_4}}{=} \left( M_C^B(f)\right)^{tt}=M_C^B(f)\notag
	\end{align}
	Also $f^{**}=f$.
\end{proof}

\begin{conclusion}
	Sind $V,W$ endlichdimensional, so liefert die Abbildung $f\mapsto f^*$ einen Isomorphismus von $K$-Vektorräumen.
	\begin{align}
		\Hom_K(V,W)\to \Hom_K(W^*,V^*)\notag
	\end{align}
\end{conclusion}
\begin{proof}
	Sind $f,g\in\Hom_K(V,W)$ und $\lambda\in K$, $\phi\in W^{*}$, so ist
	\begin{align}
		(f+g)^*(\phi)&=\phi\circ(f+g)=\phi\circ f+\phi\circ g=f^*(\phi)+g^*(\phi)=(f^*+g^*)(\phi) \notag \\
		(\lambda f)^*(\phi)&=\phi\circ (\lambda f)=\lambda\cdot(\phi\circ f)=\lambda\circ f^*(\phi)=(\lambda f^*)(\phi)\notag
	\end{align}
	Die Abbildung ist somit linear. Nach \propref{3_3_5} ist sie injektiv. Da 
	\begin{align}
		 \dim_K(V,W)&=\dim_K(V)\cdot \dim_K(W)\notag \\
		 &=\dim_K(V^*)\cdot \dim_K(W^*) \notag \\
		 &= \dim_K(\Hom_K(W^*,V^*))\notag
	\end{align}
	ist sie auch ein Isomorphismus.
\end{proof}

\begin{proposition}
	Sind $V,W$ endlichdimensional so ist
	\begin{align}
		\Image(f^*)&=\Ker(f)^0\notag \\
		\Ker(f^*)&=\Image(f)^0\notag
	\end{align}
\end{proposition}