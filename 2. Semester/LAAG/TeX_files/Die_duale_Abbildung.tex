\section{Die duale Abbildung}

Sei $f\in\Hom_K(V,W)$.

\begin{remark}
	Ist $\phi\in W^*=\Hom_K(W,K)$ eine Linearform auf $W$, so ist $\phi\circ f\in \Hom_K(V,K)=V^*$ eine Linearform auf $V$.
	\begin{center}
		\begin{tikzpicture}
		\matrix (m) [matrix of math nodes,row sep=3em,column sep=4em,minimum width=2em]
		{V & W \\ \; & K \\};
		\path[-stealth]
		(m-1-1) edge node [above] {$f$} (m-1-2)
		(m-1-1) edge [dashed,.] node [below] {$f^*(\phi)$} (m-2-2)
		(m-1-2) edge node [right] {$\phi$} (m-2-2);
		\end{tikzpicture}
	\end{center}
\end{remark}

\begin{definition}[duale Abbildung]
	Die zu $f$ duale Abbildung ist
	\begin{align}
		f^*:
		\begin{cases}
			W^*\to V^* \\
			\phi\mapsto \phi\circ f
		\end{cases} \notag
	\end{align}
\end{definition}

\begin{lemma}
	Es ist $f^*\in\Hom_K(W^*,V^*)$.
\end{lemma}
\begin{proof}
	Sind $\phi,\psi\in W^*$ und $\lambda\in K$ ist 
	\begin{align}
		f^*(\phi+\psi) &= (\phi+\psi)\circ f \notag \\
		&= \phi\circ f + \psi\circ f \notag \\
		&= f^*(\phi) + f^*(\psi) \notag \\
		f^*(\lambda\phi) &= (\lambda\phi)\circ f \notag \\
		&= \lambda\cdot(\phi\circ f) \notag \\
		&= \lambda\cdot f^*(\phi) \notag
	\end{align}
\end{proof}

\begin{proposition}
	\proplbl{7_3_4}
	Sind $B=(x_1,...,x_n)$ und $C=(y_1,...,y_m)$ Basen von $V$ bzw. $W$, so ist
	\begin{align}
		M_{B^*}^{C^*}(f^*)=\left(M_C^B(f) \right)^t\notag
	\end{align}
\end{proposition}
\begin{proof}
	Sei $A=M_C^B(f)=(a_{ij})_{i,j}$ und $B=M_{B^*}^{C^{*}}(f^*)=(b_{ji})_{j,i}$. Dann ist $f(x_j)=\sum_{i=1}^m a_{ij}y_i$, also $a_{ji}=y_i^*(f(x_j))=f^*(y_i^*)(x_j)$ und $f^*(y_i^*)=\sum_{j=1}^n b_{ji}x_j^*$, also $b_{ji}=f^*(y_i^*)(x_j)=a_{ij}$.
\end{proof}

\begin{conclusion}
	\proplbl{7_3_5}
	Sind $V$ und $W$ endlichdimensional, und identifizieren wir $V=V^{**}$ und $W=W^{**}$, so ist $f=f^{**}$, das heißt $\iota\circ f=f^{**}\circ\iota$.
	\begin{center}
		\begin{tikzpicture}
		\matrix (m) [matrix of math nodes,row sep=3em,column sep=4em,minimum width=2em]
		{V & W \\ V^{**} &W^{**} \\};
		\path[-stealth]
		(m-1-1) edge node [left] {$\iota_V\cong$} (m-2-1)
		edge node [above] {$f$} (m-1-2)
		(m-2-1) edge node [below] {$f^{**}$} (m-2-2)
		(m-1-2) edge node [right] {$\iota_W\cong$} (m-2-2);
		\end{tikzpicture}
	\end{center}
\end{conclusion}

\begin{proof}
	Seien $B$ und $C$ Basen von $V$ bzw. $W$. Unter der Identifizierung ist $B^{**}=B$ und $C=C^{**}$, das heißt $\iota(x_i)=x_i^{**}$ bzw. $\iota(y_j)=y_j^{**}$, denn $\iota(x_i)(x_j^*)=x_j^*(x_i)=\delta_{ij} = x_i^{**}(x_j^*)\quad\forall i,j$ und somit 
	\begin{align}
		M_C^B(f^{**}) \overset{\propref{7_3_4}}{=} \left( M_{B^*}^{C^*}(f^*)\right)^t \overset{\propref{7_3_4}}{=} \left( M_C^B(f)\right)^{tt}=M_C^B(f)\notag
	\end{align}
	Also $f^{**}=f$.
\end{proof}

\begin{conclusion}
	Sind $V,W$ endlichdimensional, so liefert die Abbildung $f\mapsto f^*$ einen Isomorphismus von $K$-Vektorräumen.
	\begin{align}
		\Hom_K(V,W)\to \Hom_K(W^*,V^*)\notag
	\end{align}
\end{conclusion}
\begin{proof}
	Sind $f,g\in\Hom_K(V,W)$ und $\lambda\in K$, $\phi\in W^{*}$, so ist
	\begin{align}
		(f+g)^*(\phi)&=\phi\circ(f+g)=\phi\circ f+\phi\circ g=f^*(\phi)+g^*(\phi)=(f^*+g^*)(\phi) \notag \\
		(\lambda f)^*(\phi)&=\phi\circ (\lambda f)=\lambda\cdot(\phi\circ f)=\lambda\circ f^*(\phi)=(\lambda f^*)(\phi)\notag
	\end{align}
	Die Abbildung ist somit linear. Nach \propref{7_3_5} ist sie injektiv. Da 
	\begin{align}
		 \dim_K(V,W)&=\dim_K(V)\cdot \dim_K(W)\notag \\
		 &=\dim_K(V^*)\cdot \dim_K(W^*) \notag \\
		 &= \dim_K(\Hom_K(W^*,V^*))\notag
	\end{align}
	ist sie auch ein Isomorphismus.
\end{proof}

\begin{proposition}
	\proplbl{7_3_7}
	Sind $V,W$ endlichdimensional so ist
	\begin{align}
		\Image(f^*)&=\Ker(f)^0\notag \\
		\Ker(f^*)&=\Image(f)^0\notag
	\end{align}
\end{proposition}
\begin{proof}
	\begin{itemize}
		\item $\Image(f^*)\subseteq\Ker(f)^0$: Ist $\phi\in W^*$, $x\in \Ker(f)$, so ist
		\begin{align}
			f^*(\phi)(x)=(\phi\circ f)(x)=\phi(0)=0\notag
		\end{align}
		\item $\Ker(f)^0\subseteq\Image(f^*)$: Sei $\phi\in\Ker(f)^0$. Setze eine Basis $(x_1,...,x_r)$ von $\Ker(f)$ zu einer Basis $(x_1,...,x_n)$ von $V$ fort. Dann sind $f(x_{r+1}),...,f(x_n)$ linear unabhängig nach der Kern-Bild-Formel (LAAG 1 III.7.13), es gibt also $\psi\in W^*$ mit 
		\begin{align}
			\psi(f(x_i))=\phi(x_i)\quad\forall i\notag
		\end{align}
		Es folgt
		\begin{align}
			f^*(\psi)(x_i)=\psi(f(x_i))=\phi(x_i)\quad\forall i\notag
		\end{align}
		also $\phi=f^*(\psi)$. %TODO: Verlinkung
		\item Mit der Identifizierung $V=V^{**}$ ist
		\begin{align}
			\Image(f)^0\overset{\propref{7_3_5}}{=}\Image(f^{**})^0=\Ker(f^*)^{00}\overset{\propref{7_2_15}}{=}\Ker(f^*)\notag
		\end{align}
	\end{itemize}
\end{proof}

\begin{conclusion}
	Sind $V,W$ endlichdimensional, so ist
	\begin{align}
		\rk(f)=\rk(f^*)\notag
	\end{align}
\end{conclusion}
\begin{proof}
	\begin{align}
		\rk(f) &= \dim_K(\Image(f))\notag \\
		&\overset{\propref{7_2_14}}{=} \dim_K(W)-\dim_K(\Image(f)^0)\notag \\
		&\overset{LAAG1.III.7.13}{=} \dim_K(W^*)-\dim_K(\Ker(f^*)) \notag \\
		&= \rk(f^*)\notag %TODO: Verlinkung
	\end{align}
\end{proof}

\begin{conclusion}
	\proplbl{7_3_9}
	Ist $\dim_K(V)<\infty$ und $U\subseteq V$ ein Untervektorraum, so lässt sich jede Linearform auf $U$ zu einer Linearform auf $V$ fortsetzen.
\end{conclusion}
\begin{proof}
	Ist $f:U\to V$ die Inklusionsabbildung, so ist $f^*:V^*\to U^*$, $\phi\mapsto\phi\vert_U$ und
	\begin{align}
		\rk(f^*)=\rk(f)=\dim_K(U)=\dim_K(U^*)\notag
	\end{align}
	$f^*$ ist somit surjektiv.
\end{proof}

\begin{remark}
	\propref{7_3_9} gilt auch ohne die Voraussetzung $\dim_K(V)<\infty$, siehe Übung.
\end{remark}

\begin{remark}
	Ein homogenes lineares Gleichungssystem $Ax=0$ hat als Lösungsraum $L(A,0)\subseteq K^n$ ein Untervektorraum des $K^n$. Unter der Identifizierung $K^n=(K^n)^{**}$ ist $L(A,0)$ der Annulator der Linearformen beschrieben durch die Zeilen $a_1,...,a_m\in (K^n)^*$ von $A$. Wir wollen umgekehrt zu einem Untervektorraum $W\subseteq K^n$ ein $A=(a_1,...,a_m)\in\Mat_{n\times m}(K)$ mit $W=L(A,0)$ finden. Ist $W=\Span_K(b_1,...,b_r)$, so ist $W=\Image(f_B)$ mit $B=(b_1,...,b_r)\in\Mat_{n\times r}(K)$. \\
	$\Rightarrow W\overset{\propref{7_3_7}}{=}\Ker(f^*_B)^0$ und $M_{\mathcal{E}^t}(f^*_B)=B^t$. Wenn man also eine Basis $(a_1,...,a_s)$ von $L(B^t,0)$ bestimmt und daraus eine Matrix $A=(a_1^t,...,a_s^t)\in\Mat_{s\times n}(K)$ bildet, so ist $W=L(A,0)$.
\end{remark}