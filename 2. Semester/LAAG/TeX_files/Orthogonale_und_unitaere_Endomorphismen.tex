\section{Orthogonale und unitäre Endomorphismen}

Sei $V$ ein euklidischer bzw. unitärer Vektorraum und $f\in\End_K(V)$.

\begin{definition}[orthogonale, unitäre Endomorphismen]
	$f$ ist \begriff[Endomorphismus!]{orthogonal} bzw. \begriff[Endomorphismus!]{unitär}, wenn 
	\begin{align}
		\skalar{f(x)}{f(y)}=\skalar{x}{y}\quad\forall x,y\in V\notag
	\end{align}
\end{definition}

\begin{proposition}
	\proplbl{6_5_2}
	Ist $f$ unitär, so gelten
	\begin{itemize}
		\item Für $x\in V$ ist $\Vert f(x)\Vert =\Vert x\Vert$.
		\item Sind $x,y\in V$ mit $x\perp y$, so ist $f(x)\perp f(y)$.
		\item Es ist $f\in\Aut_K(V)$ und auch $f^{-1}$ ist unitär.
		\item Das Bild einer Orthonormalbasis unter $f$ ist eine Orthonormalbasis.
		\item Ist $\lambda$ ein Eigenwert von $f$, so ist $\vert\lambda\vert=1$.
	\end{itemize}
\end{proposition}
\begin{proof}
	\begin{itemize}
		\item klar
		\item klar
		\item $f(x)=0\iff \Vert f(x)\Vert=0\iff \Vert x\Vert =0\iff x=0$, also ist $f$ injektiv, somit $f\in\Aut_K(V)$ und
		\begin{align}
			\skalar{f^{-1}(x)}{f^{-1}(y)}\overset{f\text{ unitär}}{=}\skalar{f(f^{-1}(x))}{f(f^{-1}(y))}=\skalar x y \notag
		\end{align}
		\item Folgt aus 1, 2 und 3
		\item Ist $f(x)=\lambda x$, $x\neq 0$, so ist
		\begin{align}
			\Vert x\Vert=\Vert f(x)\Vert =\Vert \lambda x\Vert =\vert\lambda\vert \cdot \Vert x\Vert \Rightarrow \vert\lambda\vert=1\notag 
		\end{align}
	\end{itemize}
\end{proof}

\begin{proposition}
	Ist $\Vert f(x)\Vert=\Vert x\Vert$ für alle $x\in V$, so ist $f$ unitär.
\end{proposition}
\begin{proof}
	Aus $\Vert f(x)\Vert=\Vert x\Vert$ folgt $\skalar{f(x)}{f(x)}=\skalar{x}{x}$. Die Polarisierung (\propref{6_3_4}) für $\skalar{f(x)}{f(y)}$ und die Linearität von $f$ liefern $\skalar{f(x)}{f(y)}=\skalar{x}{y}$. Zum Beispiel im Fall $K=\real$:
	\begin{align}
		\skalar{f(x)}{f(y)}&=\frac 1 2 \left(\skalar{\underbrace{f(x)+f(y)}_{f(x+y)}} {\underbrace{f(x)+f(y)}_{f(x+y)}}-\skalar{f(x)}{f(x)}-\skalar{f(y)}{f(y)}\right) \notag \\
		&= \frac 1 2 \left( \skalar{x+y}{x+y}-\skalar{x}{x}-\skalar{y}{y}\right) \notag \\
		&= \skalar{x}{y}\notag
	\end{align}
\end{proof}

\begin{definition}[orthogonale, unitäre Matrizen]
	Eine Matrix $A\in\Mat_n(K)$ heißt \begriff[Matrix!]{orthogonal} bzw. \begriff[Matrix!]{unitär}, wenn
	\begin{align}
		A^*A=\mathbbm{1}_n\notag
	\end{align}
\end{definition}

\begin{mathematica}[orthogonale bzw. unitäre Matrizen]
 	Auch für orthogonale bzw. unitäre Matrizen $A$ gibt es eine Mathematica bzw. WolframAlpha-Funktion
 	\begin{align}
 		\texttt{OrthogonalMatrixQ[A]}\notag \\
 		\texttt{UnitaryMatrixQ[A]}\notag
 	\end{align}
\end{mathematica}

\begin{remark}
	Offenbar ist $A$ genau dann unitär, wenn $A^*$ das Inverse zu $A$ ist. Die folgenden Bedingungen sind daher äquivalent dazu, dass $A$ unitär ist:
	\begin{align}
		AA^*=\mathbbm{1}_n, \overline{A}A^t=\mathbbm{1}_n,  A^t\overline{A}=\mathbbm{1}_n,  A^t=\overline{A^{-1}}\notag
	\end{align}
\end{remark}

\begin{proposition}
	Sei $B$ eine Orthogonalbasis von $V$. Genau dann ist $f$ unitär, wenn $M_B(f)$ unitär ist.
\end{proposition}
\begin{proof}
	Sei $A=M_B(f)$, $v=\Phi_B(x)$, $\Phi_B(y)$. Dann ist $\skalar{v}{w}=x^t\underbrace{M_B(\skalar{\cdot}{\cdot})}_{=\mathbbm{1}}\cdot \overline{y}=x^t\cdot \overline{y}$. Somit ist $f$ genau dann unitär, wenn $(Ax)^t\overline{Ay}=x^t\overline{y}$ für alle $x,y\in K^n$, also wenn $A^t\overline{A}=\mathbbm{1}$, d.h. $A$ unitär.
\end{proof}

\begin{proposition}
	Die folgenden Mengen bilden Untergruppen der $\GL_n(K)$.
	\begin{itemize}
		\item $\Orth_n=\{A\in\GL_n(\real)\mid A\text{ ist orthogonal}\}$ die \begriff{orthogonale Gruppe}
		\item $\SO_n=\{A\in \Orth_n\mid \det(A)=1\}$ die \begriff{spezielle orthogonale Gruppe}
		\item $\Uni_n=\{A\in\GL_n(\comp)\mid A\text{ ist unitär}\}$ die \begriff{unitäre Gruppe}
		\item $\SU_n=\{A\in \Uni_n\mid \det(A)=1\}$ die \begriff{spezielle unitäre Gruppe}
	\end{itemize}
\end{proposition}
\begin{proof}
	z.B. für $\Uni_n$: Sind $A^{-1}=A^*$, $B^{-1}=B^*$, so ist $(AB)^{-1}=B^{-1}A^{-1}=B^*A^*=(AB)^*$, $(A^{-1})^{-1}=A=(A^*)^{-1}=(A^{-1})^*$
\end{proof}

\begin{proposition}
	\proplbl{6_5_8}
	Genau dann ist $A\in\Mat_n(K)$ unitär, wenn die Spalten (oder die Zeilen) von $A$ eine Orthonormalbasis des $K^n$ bilden. 
\end{proposition}
\begin{proof}
	Sei $s$ das Standardskalarprodukt und $B=(a_1,...,a_n)$. Nach \propref{6_4_5} ist $B$ genau dann eine Orthonormalbasis, wenn $M_B(s)=\mathbbm{1}_n$, und $M_B(s)=A^t\cdot\mathbbm{1}_n\cdot\overline{A}$, vgl. \propref{6_2_9}
\end{proof}

\begin{theorem}
	\proplbl{6_5_9}
	Sei $K=\comp$ und $f\in\End_K(V)$. Ist $f$ unitär, so besitzt $V$ eine Orthonormalbasis aus Eigenvektoren von f.
\end{theorem}
\begin{proof}
	Induktion über$n=\dim_K(V)$. \\
	\emph{$n=0$}: klar \\
	\emph{$n-1\to n$}: Da $K$ algebraisch abgeschlossen ist, hat $\chi_f$ eine Nullstelle $\lambda$, es gibt also einen Eigenvektor $x_1$ von $f$ zum Eigenwert $\lambda$. Ohne Einschränkung nehmen wir $\Vert x\Vert=1$ an. Sei $W=K\cdot x_1$. Nach \propref{6_4_11} ist dann $V=W\oplus W^\perp$. Für $v\in W^\perp, w\in W$ ist
	\begin{align}
		0=\skalar{v}{w}=\skalar{f(v)}{f(w)}=\overline{\lambda}\skalar{f(v)}{w}\notag
	\end{align}
	da $\lambda\neq 0$ ($f$ unitär) also $f(W^\perp)\perp W$. Somit ist $f(W^\perp)\subseteq W^\perp$, d.h. $W^\perp$ ist $f$-invariant. Da auch $f\vert_{W^\perp}$ unitär ist, gibt es nach Induktionshypothese eine Orthonormalbasis $(x_1,...,x_n)$ aus Eigenvektoren von $f\vert_{W^\perp}$. Da $V=W\oplus W^\perp$ und $W\perp W^\perp$ ist $(x_1,...,x_n)$ eine Orthonormalbasis von $V$ aus Eigenvektoren von $f$.
\end{proof}

\begin{conclusion}
	Jeder unitäre Endomorphismus eines unitären Vektorraums ist diagonalisierbar.
\end{conclusion}

\begin{conclusion}
	Zu jeder $A\in\Uni_n$ gibt es $S\in\Uni_n$ so, dass 
	\begin{align}
		S^*AS=S^{-1}AS=\diag(\lambda_1,...,\lambda_n)\notag
	\end{align}
	mit $\vert\lambda_i\vert=1$ für $i=1,...,n$.
\end{conclusion}
\begin{proof}
	Da $A$ unitär ist, ist $f_A\in\End_\comp(\comp^n)$ unitär, nach \propref{6_5_9} existiert also eine Orthonormalbasis $B$ des $\comp^n$ aus Eigenvektoren von $A$. Die Transformationsmatrix $S=T_{\mathcal{E}}^B$ hat als Spalten die Elemente von $B$ und somit ist $S$ nach \propref{6_5_8} unitär. Nach \propref{6_5_2} ist $\vert\lambda\vert=1$ für alle Eigenwerte von $f_A$.
\end{proof}

\begin{remark}
	Dies (\propref{6_5_9}) gilt nicht im Fall $K=\real$. Man kann aber auch orthogonale Endomorphismen immer "'fast diagonalisieren"'.
\end{remark}