\section{Der Dualraum}

Sei $V$ ein $K$-Vektorraum.

\begin{definition}[Dualraum]
	Der \begriff{Dualraum} zu $V$ ist der $K$-Vektorraum
	\begin{align}
		V^*=\Hom_K(V,K)=\{\phi:V\to K\text{ linear}\}\notag
	\end{align}
	Die Elemente von $V^*$ heißen \begriff{Linearformen} auf $V$.
\end{definition}

\begin{example}
	Ist $V=K^n=\Mat_{n\times 1}(K)$, so wird $V^*=\Hom_K(V,K)$ durch $\Mat_{1\times n}(K)\cong K^n$. Wir können also die Elemente von $V$ als Spaltenvektoren und die Linearformen auf $V$ als Zeilenvektoren auffassen.
\end{example}

\begin{lemma}
	Ist $B(x_1)_{i\in I}$ eine Basis von $V$, so gibt es zu jedem $i\in I$ genau $x_i^*\in V^*$ mit 
	\begin{align}
		x_i^*(x_j)=\delta_{ij}\quad\forall j\in I\notag
	\end{align}
\end{lemma}
\begin{proof}
	Siehe LAAG1 III.5.1, angewandt auf die Familie $(y_j)_{j\in I}$, $y_j\delta_{i.j}$ in $W=K$. %TODO: Verlinkung
\end{proof}

\begin{proposition}
	Ist $B=(x_1)_{i\in I}$ eine Basis von $V$, so ist $B^*=(x_i^*)_{i\in I}$ linear unabhängig. Ist $I$ endlich, so ist $B^*$ eine Basis von $V^*$.
\end{proposition}
\begin{proof}
	Ist $\phi=\sum_{i\in I} \lambda_ix_i^*$, $\lambda_i\in K$, fast alle gleich 0, so ist $\phi(x_j)=\sum_{i\in I} \lambda_j x_i^*(x_j)=\lambda_j$ für jedes $j\in I$. Ist also $\phi=0$, so ist $\lambda_j=\phi(x_j)=0\quad\forall j\in I$, $B^*$ ist somit linear unabhängig. \\
	Ist zudem $I$ endlich und $\psi\in V^*$, so ist $\psi=\psi'=\sum_{i\in I} \psi(x_i)x_i^*$, denn $\psi'(x_j)=\sum_{i\in I} \psi(x_i)x_i^*(x_j)=\psi(x_i)\quad\forall j\in I$, und somit ist $B^*$ ein Erzeugendensystem von $V^*$.
\end{proof}

\begin{definition}[duale Basis]
	Ist $B=(x_i)_{i\in I}$ eine endliche Basis von $V$, so nennt man $B^*=(x_i^*)_{i\in I}$ die zu $B$ \begriff{duale Basis}.
\end{definition}

\begin{conclusion}
	\proplbl{7_2_6}
	Zu jeder Basis $B$ von $V$ gibt es einen eindeutig bestimmtem Monomorphismus
	\begin{align}
		f_V\to V^*\text{ mit } f(B)=B^*\notag
	\end{align}
	Ist $\dim_K(V)<\infty$, so ist dieser ein Isomorphismus.
\end{conclusion}

\begin{conclusion}
	\proplbl{7_2_7}
	Zu jedem $=0\neq x\in V$ gibt es eine Linearform $\phi\in V$ mit $\phi(x)=1$.
\end{conclusion}
\begin{proof}
	Ergänze $x_1=x$ zu einer Basis $(x_i)_{i\in I}$ von $V$ (\propref{7_1_11}) und $\phi=x_1^*$.
\end{proof}

\begin{example}
	Ist $V=K^n$ mit Standardbasis $\mathcal{E}=(e_1,...,e_n)$, so können wir $V^*$ mit dem Vektorraum der Zeilenvektoren identifizieren, und dann ist
	\begin{align}
		e_i^* = e_i^t\notag
	\end{align}
\end{example}

\begin{definition}[Bidualraum]
	Der \begriff{Bidualraum} zu $V$ ist der $K$-Vektorraum
	\begin{align}
		V^{**}=(V^*)^*=\Hom_K(V^*,K)\notag
	\end{align}
\end{definition}

\begin{proposition}
	Die kanonische Abbildung
	\begin{align}
		\iota:\begin{cases}
		V\to V^{**} \\ x\to \iota_x
		\end{cases}\text{ wobei } \iota_x(\phi)=\phi(x)\notag
	\end{align}
	ist ein Monomorphismus. Ist $\dim_K(V)<\infty$, so ist $\iota$ ein Isomorphismus.
\end{proposition}
\begin{proof}
	\begin{itemize}
		\item $\iota_x\in V^{**}$: 
		\begin{itemize}
			\item $\iota_x(\phi+\psi)=(\phi+\psi)(x)=\phi(x)+\psi(x)=\iota_x(\phi)+ \iota_x(\psi)$
			\item $\iota_x(\lambda \phi)=(\lambda\phi)(x)=\lambda\phi(x)=\lambda\iota_x(\phi)$
		\end{itemize}
		\item $\iota$ linear: 
		\begin{itemize}
			\item $\iota_{x+y}(\phi)=\phi(x+y)=\phi(x)+\phi(y)=\iota_x(\phi)+\iota_y(\phi)= (\iota_x+\iota_y)(\phi)$
			\item $\iota_{\lambda x}(\phi)=\phi(\lambda x)=\lambda\iota_x(\iota)=(\lambda\iota_x)(\phi)$
		\end{itemize}
		\item $\iota$ injektiv: Sei $0\neq x\in V$. Nach \propref{7_2_7} existiert $\phi\in V^*$ mit $\iota_x(\phi)=\phi(x)=1\neq 0$. Somit ist $\iota_x\neq 0$.
		\item Ist $\dim_K(V)<\infty$, so ist $V\overset{\propref{7_2_6}}{\cong} V^* \overset{\propref{7_2_6}}{\cong} V^{**}$, insbesondere $\dim_K(V)=\dim_K(V^{**})$. Der Monomorphismus $\iota$ ist somit ein Isomorphismus.
	\end{itemize}
\end{proof}

\begin{remark}
	Sei $\dim_K(V)<\infty$. Im Gegensatz zu den Isomorphismen $V\to V^*$, die von der Wahl der Basis $B$ abhängen, ist der Isomorphismus $\iota:V\to V^{**}$ kanonisch (von der Wahl der Basis $B$ unabhängig).
	
	Die Voraussetzung, dass $\dim_K(V)<\infty$ ist hier essentiell: Für $\dim_K(V)=\infty$ ist $\iota$ nicht surjektiv.
\end{remark}

\begin{definition}[Annulator]
	Für eine Teilmenge $U\subseteq V$ bezeichne
	\begin{align}
		U^0 =\{\phi\in V^*\mid \phi(x)=0\quad\forall x\in U\}\notag
	\end{align}
	den \begriff{Annulator} von $U$.
\end{definition}

\begin{*anmerkung}
	Für eine Gerade $L=x\cdot\real\subset\real^2$ sind $a=(a_1,a_2)\in (\real^2)^*$ gesucht mit
	\begin{align}
		a_1x_1 + a_2x_2 = 0\notag
	\end{align}
	\renewcommand{\arraystretch}{0.8}
	\begin{center}\begin{tikzpicture}
		\draw[->] (-7,0) -- (-1,0);
		\draw[->] (-4,-3) -- (-4,3);
		\draw[thick] (-7,-1.5) -- (-1,1.5);
		\draw[fill=black] (-3,0.5) circle (0.05);
		\node at (-5,2) (real) {$\mathbb{R}^2$};
		\node at (-1,1.8) (gerade) {$L$};
		\node at (-1.7,0.5) (punkt) {$x=\left(\begin{array}{c}x_1 \\ x_2\end{array}\right)$};
		
		\draw[->] (1,0) -- (7,0);
		\draw[->] (4,-3) -- (4,3);
		\draw[thick] (2.5,3) -- (5.5,-3);
		\draw[fill=black] (4.5,-1) circle (0.05);
		\node at (5,2) (real) {$\left(\mathbb{R}^2\right)^*$};
		\node at (2.2,3) (annulator) {$L^0$};
		\node at (5.7,-1) (punkt2) {$a=(a_1,a_2)$};
		\end{tikzpicture}\end{center}
	Diese $a\in (\real^2)^*$ liegen auf einer Geraden $L^0$ in $(\real^2)^*$, die senkrecht auf $L$ steht, wenn man $\real^2$ und $(\real^2)^*$ nicht unterscheidet.
	\renewcommand{\arraystretch}{1.2}
\end{*anmerkung}

\begin{lemma}
	$U^0$ ist ein Untervektorraum von $V^*$.
\end{lemma}
\begin{proof}
	Klar.
\end{proof}

\begin{proposition}
	\proplbl{7_2_14}
	Ist $\dim_K(V)<\infty$ und $U\subseteq V$ ein Untervektorraum, so ist
	\begin{align}
		\dim_K(V)=\dim_K(U)+\dim_K(U^0)\notag
	\end{align}
\end{proposition}
\begin{proof}
	Ergänze eine Basis $(x_1,...,x_r)$ von $U$ zu einer Basis $B=(x_1,...,x_n)$ von $V$. Dann ist $B^*(x_1^*,...,x_n^*)$ eine Basis von $V^*$. Sei $C=(x_{r+1}^*,...,x_n^*)$. Dann ist $C$ eine Basis von $U^0$: 
	\begin{itemize}
		\item $B^*$ ist Basis $\Rightarrow C$ ist linear unabhängig.
		\item $C\subseteq U^0$: Für $1\le j\le r < i\le n$ ist $x_i^*(x_j)=\delta_{ij}=0$.
		\item $U^0\subseteq\Span_K(C)$: Ist $\phi=\sum_{i=1}^n \lambda_ix_i^*\in U^0$, so $0=\phi(x_j)= \lambda_j$ für alle $j\le r$, \\also $\phi\in\Span_K(x_{r+1}^*,...,x_n^*)$.
	\end{itemize}
\end{proof}

\begin{conclusion}
	\proplbl{7_2_15}
	Ist $\dim_K(V)<\infty$ und $U\subset V$ ein Untervektorraum, so ist
	\begin{align}
		\iota(U)=U^{00}\notag
	\end{align}
\end{conclusion}
\begin{proof}
	Es ist klar, dass $\iota(U)\le U^{00}$. \\
	Für $\phi\in U^0$ und $x\in U$ ist $\iota_x(\phi) = \phi(x) =0$. Mit \propref{7_2_14} ist
	\begin{align}
		\dim_K(U^{00}) &= \dim_K(V^*)-\dim_K(U^0) \notag \\
		&=\dim_K(V^*) - (\dim_K(V)-\dim_K(U)) \notag \\
		&\overset{\propref{7_2_6}}{=} \dim_K(U) \notag
	\end{align}
	und da $\iota$ injektiv ist, folgt $\iota(U)=U^{00}$.
\end{proof}