\section{Der Vektorraum der linearen Abbildungen}

Seien $V$ und $W$ zwei $K$-Vektorräume.

\begin{proposition}
	\proplbl{3_5_1}
	Sei $(x_i)$ eine Basis von $V$ und $(y_i)$ eine Familie in $W$. Dann gibt es genau eine lineare 
	Abbildung $f:V\to W$ mit $f(x_i)=y_i$. Diese Abbildung ist durch $f(\sum \lambda_ix_i)=\sum \lambda_iy_i$ 
	(*) ($\lambda_i\in K$, fast alle gleich 0) gegeben und erfüllt
	\begin{itemize}
		\item $\Image(f)=\Span_K(y_i)$
		\item genau dann ist $f$ injektiv, wenn $(y_i)$ linear unabhängig ist
	\end{itemize}
\end{proposition}
\begin{proof}
	Ist $f:V\to W$ linear mit $f(x_i)=y_i$, so folgt aus \propref{3_4_6} $f(\sum \lambda_ix_i)=\sum \lambda_iy_i$. Da sich jedes 
	$x\in V$ als $x=\sum  \lambda_ix_i$ schreiben lässt, ist $f$ dadurch schon eindeutig bestimmt. Andererseits wird 
	durch (*) eine wohldefinierte Abbildung beschrieben, da die Darstellung von $x$ eindeutig ist (denn $x_i$ sind 
	linear unabhängig). Es bleibt zu zeigen, dass die durch (*) definierte Abbildung $f:V\to W$ tatsächlich linear ist. 
	Ist $x=\sum \lambda_ix_i$ und $x'=\sum \lambda'_ix_i$ so ist $f(x+x')=f(\sum (\lambda_i+\lambda'_i)x_i)=
	\sum (\lambda_i+\lambda'_i)y_i=\sum \lambda_iy_i+\sum \lambda'_iy_i=f(x)+f(x')$. $f(\lambda x)=f(\sum \lambda
	\lambda_ix_i)=\sum \lambda\lambda_iy_i=\lambda\sum\lambda_iy_i=\lambda f(x)$.
	\begin{itemize}
		\item $\Image(f)$ ist ein Untervektorraum nach \propref{3_4_6} von $W$ und $\{y_i\}\subset \Image(f)\subset \Span_K(y_i)$, somit $\Image(f)=\Span_K(y_i)$
		\item \propref{3_4_11}: $f$ ist injektiv $\iff \Ker(f)=\{0\}$ \\ 
		$\iff \lambda_i\in K$ gilt: $f(\sum \lambda_ix_i)=0\Rightarrow \sum \lambda_ix_i=0$ \\ 
		$\iff \lambda_i\in K$ gilt: $\sum\lambda_iy_i=0\Rightarrow \lambda_i=0$ \\ 
		$\iff (y_i)$ linear unabhängig.
	\end{itemize}
\end{proof}

\begin{conclusion}
	Sei $\dim_K<\infty$. Ist $(x_1,...,x_n)$ eine linear unabhängige Familie in $V$ und $(y_1,...,y_n)$ 
	eine Familie in $W$, so gibt es eine lineare Abbildung $f:V\to W$ mit $f(x_i)=y_i$
\end{conclusion}
\begin{proof}
	Nach dem Basisergänzungssatz (\propref{2_3_12}) können wir die Familie $(x_i)$ zu einer Basis $x_1,...,x_m$ ergänzen. Die Behauptung 
	folgt aus \propref{3_5_1} für beliebige $y_{n+1},...,y_m\in W$.
\end{proof}

\begin{conclusion}
	\proplbl{3_5_3}
	Ist $(x_i)$ eine Basis von $V$ und $(y_i)$ eine Basis in $W$, so gibt es genau einen Isomorphismus 
	$f:V\to W$ mit $f(x_i)=y_i$.
\end{conclusion}
\begin{proof}
	Sei $f$ wie in \propref{3_5_1}. $(y_i)$ ist Erzeugendensystem $\Rightarrow \Image(f)=\Span_K(y_i)=W$, also $f$ surjektiv. 
	$(y_i)$ linear abhängig $\Rightarrow f$ ist injektiv.
\end{proof}

\begin{conclusion}
	Zwei endlichdimensionale $K$-Vektorräume sind genau dann isomorph, wenn sie dieselbe Dimension haben.
\end{conclusion}
\begin{proof}
	\propref{3_5_3} und \propref{3_4_10}
\end{proof}

\begin{conclusion}
	\proplbl{3_5_5}
	Ist $B=(v_1,...,v_n)$ eine Basis von $V$, so gibt es genau einen Isomorphismus $\Phi_B:K^n\to 
	V$ mit $\Phi_B(e_i)=v_i$. Insbesondere ist jeder endlichdimensionale $K$-Vektorraum zu einem Standardraum isomorph, nämlich zu
	$K^n$ für $n=\dim_K(V)$.
\end{conclusion}

\begin{definition}[Koordinatensystem]
	Die Abbildung $\Phi_B$ heißt \begriff{Koordinatensystem} zu $B$. Für $v\in V$ ist 
	$(x_1,...,x_n)^t=\Phi^{-1}_B(v)\in K^n$ der Koordinatenvektor zu $v$ bezüglich $B$ und $(x_1,...,x_n)$ sind die 
	Koordinaten von $v$ bezüglich $B$.
\end{definition}

\begin{proposition}
	Die Menge $\Hom_K(V,W)$ ist ein Untervektorraum des $K$-Vektorraums $\Abb(V,W)$.
\end{proposition}
\begin{proof}
	Seien $f,g\in \Hom_K(V,W)$ und $\eta \in K$.
	\begin{itemize}
		\item $f+g\in \Hom_K(V,W)$: Für $x,y\in V$ und $\lambda,\mu\in K$ ist $(f+g)(\lambda x+\mu y)=f(\lambda x+\mu y)+
		g(\lambda x+\mu y)=\lambda f(x)+\mu f(y)+\lambda g(x)+\mu g(y)=\lambda(f+g)(x)+\mu(f+g)(y)$
		\item $\eta f\in \Hom_K(V,W)$: Für $x,y\in V$ und $\lambda,\mu\in K$ ist $(\eta f)(\lambda x+\mu y)=\eta\cdot 
		f(\lambda x+\mu y)=\eta(\lambda f(x)+\mu f(y))=\lambda(\eta f)(x)+\mu(\eta f)(y)$
		\item $\Hom_K(V,W)\neq\emptyset$: $c_0\in \Hom_K(V,W)$
	\end{itemize}
\end{proof}

\begin{lemma}
	\proplbl{3_5_8}
	Sei $U$ ein weiterer $K$-Vektorraum. Sind $f,f_1,f_2\in \Hom_K(V,W)$ und $g,g_1,g_2\in \Hom_K(U,V)$, so ist 
	$f\circ (g_1+g_2)=f\circ g_1+f\circ g_2$ und $(f_1+f_2)\circ g=f_1\circ g+f_2\circ g$.
\end{lemma}
\begin{proof}
	Für $x\in U$ ist
	\begin{itemize}
		\item $(f\circ(g_1+g_2))(x)=f((g_1+g_2)(x))=f(g_1(x)+g_2(x))=f(g_1(x))+f(g_2(x))=(f\circ g_1+f\circ g_2)(x)$
		\item $((f_1+f_2)\circ g)(x)=(f_1+f_2)(g(x))=f_1(g(x))+f_2(g(x))=(f_1\circ g+f_2\circ g)(x)$
	\end{itemize}
\end{proof}

\begin{conclusion}
	\proplbl{3_5_9}
	Unter der Komposition wird $\End_K(V)$ zu einem Ring mit Einselement $\id_V$ und $\End_K(V)^{\times}=
	\Aut_K(V)$.
\end{conclusion}
\begin{proof}
	$(\End_K(V),+)$ ist eine abelsche Gruppe (\propref{3_4_9}), die Komposition eine Verknüpfung auf $\End_K(V)$ ist assoziativ und die 
	Distributivgesetze gelten (\propref{3_5_8}).
\end{proof}

\begin{remark}
	Die Menge der strukturverträglichen Abbildungen zwischen $K$-Vektorräumen trägt also wieder die Struktur 
	eines $K$-Vektorraums. Wir können diesen mit unseren Mitteln untersuchen und z.B. nach Dimension und Basis fragen.
\end{remark}

\begin{lemma}
	\proplbl{3_5_11}
	Seien $m,n,r\in \mathbb N$, $A\in \Mat_{m\times n}(K)$, $B\in \Mat_{n\times r}(K)$. Für die linearen 
	Abbildungen $f_A\in \Hom_K(K^n,K^m)$, $f_B\in \Hom_K(K^r,K^n)$ aus \propref{3_4_5} gilt dann $f_{AB}=f_A\circ f_B$.
\end{lemma}
\begin{proof}
	Sind $A=(a_{ij})$ und $B=(b_{jk})$, so ist $(f_A\circ f_B)(e_k)=f_A(f_B(e_k))=f_A(Be_k)=f_A(b_{1k},...,b_{nk})^t=
	A\cdot (b_{1k},...,b_{nk})^t=(\sum_{j=1}^n a_{ij}b_{jk},...,\sum_{j=1}^n a_{mj}b_{jk})^t=AB\cdot e_k=
	f_{AB}(e_k)$ für $k=1,...,r$, also $f_A\circ f_B=f_{AB}$ nach \propref{3_5_1}.
\end{proof}

\begin{proposition}
	\proplbl{3_5_12}
	Die Abbildung $A\to f_A$ aus \propref{3_4_5} liefert einen Isomorphismus von $K$-Vektorräumen $F_{m\times n}\!\!: \Mat_{m\!\times\!n}(K)
	\!\to\! \Hom_K(K^n\!,K^m)$ sowie einen Ringisomorphismus $F_n:\Mat_n(F)\to \End_K(K^n)$ der $\GL_n(K)$ auf $\Aut_K(K^n)$ 
	abbildet.
\end{proposition}
\begin{proof}
	Wir schreiben $F$ für $F_{m\times n}$
	\begin{itemize}
		\item $F$ ist linear: Sind $A,B\in Mat-{n\times m}(K)$ und $\lambda,\mu\in K$, so ist $F(\lambda A+\mu B)(x)=
		f_{\lambda A+\mu B}(x)=(\lambda A+\mu B)x=\lambda Ax+\mu Bx=\lambda f_A(x)+\mu f_B(x)=(\lambda F(A)+\mu F(B))(x)$, 
		also ist $F$ linear.
		\item $F$ ist injektiv: Es genügt zu zeigen, dass $\Ker(f)=\{0\}$ (\propref{3_4_11}). Ist $A=(a_{ij})\in \Mat_{n\times m}(K)$ mit $F(A)=0$, 
		so insbesondere $0=F(A)(e_j)=f_A(e_j)=Ae_j=(a_{1j},...,a_{mj})^t$, also $A=0$.
		\item $F$ ist surjektiv: Sei $f\in \Hom_K(V,W)$. Schreibe $f(e_j)=(a_{1j},...,a_{mj})^t$ und setze $A=(a_{ij})\in 
		\Mat_{n\times m}(K)$. Dann ist $f_A\in \Hom_K(K^n,K^m)$ mit $f_A(e_j)=Ae_j=f(e_j)$, also $f=f_A=F(A)\in \Image(f)$ nach \propref{3_5_1}.
		\item $F_n$ ist eine Ringhomomorphismus (\propref{3_5_11}): \\
		(RH1) aus (L1) \\
		(RH2) aus $f_{AB}=f_A\circ f_B$.
		\item Somit ist $F_n$ eine Ringisomorphismus $\Rightarrow F_n(\Mat_n(K)^{\times})=\End_K(V)^{\times}$, also $F_n(
		\GL_n(K))=\Aut_K(V)$ nach \propref{3_5_9}.
	\end{itemize}
\end{proof}

\begin{remark}
	Wir sehen also, dass die linearen Abbildungen zwischen Standardräumen sehr konkret durch Matrizen beschrieben werden können. Da jeder endlichdimensionale Vektorraum zu einem Standardraum isomorph ist (\propref{3_5_5}), kann man diese Aussage auf solche Vektorräume erweitern. Dies wollen wir im nächsten Abschnitt tun.
\end{remark}