\section{Logik und Mengen}

Wir werden die Grundlagen der Logik und der Mengenlehre kurz ansprechen.
\begin{overview}[Aussagenlogik]
	Jede mathematisch sinnvolle Aussage ist entweder wahr oder falsch, aber nie beides!
	\begin{itemize}
		\item "'$1+1=2$"' $\to$ wahr
		\item "'$1+1=3$"' $\to$ falsch
		\item "'Es gibt unendlich viele Primzahlen"' $\to$ wahr
	\end{itemize}
	Man ordnet jeder mathematischen Aussage $A$ einen Wahrheitswert "'wahr"' oder "'falsch"' zu. Aussagen
	lassen sich mit logischen Verknüpfungen zu neuen Aussagen zusammensetzen.
	\begin{itemize}
		\item $\lor \to$ oder
		\item $\land \to$ und
		\item $\lnot \to$ nicht
		\item $\Rightarrow \to$ impliziert
		\item $\iff \to$ äquivalent
	\end{itemize}
	Sind also $A$ und $B$ zwei Aussagen, so ist auch $A \lor B$, $A \land B$, $\lnot A$, 
	$A \Rightarrow B$ und $A \iff B$ Aussagen. Der Wahrheitswert einer zusammengesetzten Aussage ist
	eindeutig bestimmt durch die Wahrheitswerte ihrer Einzelaussagen.
	\begin{itemize}
		\item $\lnot (1+1=3) \to$ wahr
		\item "'2 ist ungerade"' $\Rightarrow$ "'3 ist gerade"' $\to$ wahr
		\item "'2 ist gerade"' $\Rightarrow$ "'Es gibt unendlich viele Primzahlen"' $\to$ wahr
	\end{itemize}
	$\newline$
	\begin{center}
		\begin{tabular}{|c|c|c|c|c|c|c|}
			\hline
			$A$ & $B$ & $A \lor B$ & $A \land B$ & $\lnot A$ & $A \Rightarrow B$ & $A \iff B$\\
			\hline
			w & w & w & w & f & w & w\\
			\hline
			w & f & w & f & f & f & f\\
			\hline
			f & w & w & f & w & w & f\\
			\hline
			f & f & f & f & w & w & w\\
			\hline
		\end{tabular}
	\end{center}
\end{overview}

\begin{overview}[Prädikatenlogik]
	Wir werden die Quantoren
	\begin{itemize}
		\item $\forall$ (Allquantor, "'für alle"') und
		\item $\exists$ (Existenzquantor, "'es gibt"') verwenden.
	\end{itemize}
	Ist $P(x)$ eine Aussage, deren Wahrheitswert von einem unbestimmten $x$ abhängt, so ist \\
	$\forall x: P(x)$ genau dann wahr, wenn $P(x)$ für alle $x$ wahr ist, \\
	$\exists x: P(x)$ genau dann wahr, wenn $P(x)$ für mindestens ein $x$ wahr ist. \\
	$\newline$
	Insbesondere ist $\lnot \forall x: P(x)$ genau dann wahr, wenn $\exists x: \lnot P(x)$ wahr ist. \\
	Analog ist $\lnot \exists x: P(x)$ genau dann wahr, wenn $\forall x: \lnot P(x)$ wahr ist.
\end{overview}

\begin{overview}[Beweise]
	Unter einem Beweis verstehen wir die lückenlose Herleitung einer mathematischen Aussage aus einer
	Menge von Axiomen, Voraussetzungen und schon früher bewiesenen Aussagen. \\
	Einige Beweismethoden:
	\begin{itemize}
		\item \textbf{Widerspruchsbeweis} \\
		Man nimmt an, dass eine zu beweisende Aussage $A$ falsch sei und leitet daraus ab, dass eine 
		andere Aussage sowohl falsch als auch wahr ist. Formal nutzt man die Gültigkeit der Aussage
		$\lnot A \Rightarrow (B \land \lnot B) \Rightarrow A$.
		\item \textbf{Kontraposition} \\
		Ist eine Aussage $A \Rightarrow B$ zu beweisen, kann man stattdessen die Implikation 
		$\lnot B \Rightarrow \lnot A$ beweisen.
		\item \textbf{vollständige Induktion} \\
		Will man eine Aussage $P(n)$ für alle natürlichen Zahlen zeigen, so genügt es, zu zeigen,
		dass $P(1)$ gilt und dass unter der Induktionsbehauptung $P(n)$ stets auch $P(n+1)$ gilt 
		(Induktionschritt). Dann gilt $P(n)$ für alle $n$. \\
		Es gilt also das Induktionsschema: $P(1) \land \forall n: (P(n) \Rightarrow P(n+1)) \Rightarrow
		\forall n: P(n)$.
	\end{itemize}
\end{overview}

\begin{overview}[Mengenlehre]
	Jede Menge ist eine Zusammenfassung bestimmter wohlunterscheidbarer Objekte zu einem Ganzen. Eine
	Menge enthält also solche Objekte, die Elemente der Menge. Die Menge ist durch ihre Elemente
	vollständig bestimmt. Diese Objekte können für uns verschiedene mathematische Objekte, wie
	Zahlen, Funktionen oder andere Mengen sein. Man schreibt $x \in M$ bzw. $x \notin M$, wenn x ein
	bzw. kein Element der Menge ist.
	
	Ist $P(x)$ ein Prädikat, so bezeichnet man eine Menge mit $X := \{x \mid P(x)\}$. Hierbei muss
	man vorsichtig sein, denn nicht immer lassen sich alle $x$ für die $P(x)$ gilt, widerspruchsfrei
	zu einer Menge zusammenfassen.
\end{overview}

\begin{example}[endliche Mengen]
	Eine Menge heißt endlich, wenn sie nur endlich viele Elemente enthält. Endliche Mengen
	notiert man oft in aufzählender Form: $M = \{1;2;3;4;5;6\}$. Hierbei ist die Reihenfolge
	der Elemente nicht relevant, auch nicht die Häufigkeit eines Elements. \\
	Sind die Elemente paarweise verschieden, dann ist die Anzahl der Elemente die Mächtigkeit
	(oder Kardinalität) der Menge, die wir mit $|M|$ bezeichnen.
\end{example}

\begin{example}[unendliche Mengen]
	\begin{itemize}
		\item Menge der natürlichen Zahlen: $\mathbb N := \{1,2,3,4,...\}$
		\item Menge der natürlichen Zahlen mit der 0: $\mathbb N_0 := \{0,1,2,3,4,...\}$
		\item Menge der ganzen Zahlen: $\mathbb Z := \{...,-2,-1,0,1,2,...\}$
		\item Menge der rationalen Zahlen: $\mathbb Q := \{\frac p q \mid p,q \in \mathbb Z, q 
		\neq 0\}$
		\item Menge der reellen Zahlen: $\mathbb R := \{x \mid x$ ist eine reelle Zahl$\}$
	\end{itemize}
	Ist $M$ eine Menge, so gilt $|M|=\infty$
\end{example}

\begin{example}[leere Menge]
	Es gibt genau eine Menge, die keine Elemente hat, die leere Menge $\emptyset := \{\}$.
\end{example}


\begin{definition}[Teilmenge]
	\proplbl{1_1_8}
	Sind $X$ und $Y$ zwei Mengen, so heißt $X$ eine \begriff{Teilmenge} von 
	$Y$, wenn jedes Element von $X$ auch Element von $Y$ ist, dass heißt wenn für alle 
	$x$ $(x \in X \Rightarrow x \in Y)$ gilt.
\end{definition}

Da eine Menge durch ihre Elemente bestimmt ist, gilt $X = Y \Rightarrow (X \subset Y)\land
(Y \subset X)$. Will man Mengengleichheit beweisen, so genügt es, die beiden Inklusionen
$X \subset Y$ und $Y \subset X$ zu beweisen. \\


Ist $X$ eine Menge und $P(x)$ ein Prädikat, so bezeichnet man mit $Y:= \{x \in X \mid
P(x)\}$ die Teilmenge von $X$, die das Prädikat $P(x)$ erfüllen. \\

\begin{definition}[Mengenoperationen]
	Seien $X$ und $Y$ Mengen. Man definiert daraus 
	weitere Mengen wie folgt (\begriff{Mengenoperationen}):
	\begin{itemize}
		\item $X \cup Y := \{x \mid x \in X \lor x \in Y\}$
		\item $X \cap Y := \{x \mid x \in X \land x \in Y\}$
		\item $X \backslash Y := \{x \in X \mid x \notin Y\}$
		\item $X \times Y := \{(x,y) \mid x \in X \land y \in Y\}$
		\item $\mathcal P(X) := \{Y \mid Y \subset X\}$
	\end{itemize}
\end{definition}

Neben den offensichtlichen Mengengesetzen, wie dem Kommutativgesetz, gibt es auch weniger 
offensichtliche Gesetze, wie die Gesetze von \person{de Morgan}: Für $X_1, X_2 \subset X$ gilt:
\begin{itemize}
	\item $X \backslash (X_1 \cup X_2) = (X \backslash X_1) \cap (X \backslash X_2)$
	\item $X \backslash (X_1 \cap X_2) = (X \backslash X_1) \cup (X \backslash X_2)$
\end{itemize}


Sind $X$ und $Y$ endliche Mengen, so gilt:
\begin{itemize}
	\item $|X \times Y| = |X| \cdot |Y|$
	\item $|\mathcal P(X)| = 2^{|X|}$
\end{itemize}
