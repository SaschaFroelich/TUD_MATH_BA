\section{Priority Queue}

Eine Priority Queue ist eine Art von Warteschlange, die nicht nach dem FIFO-Prinzip arbeitet, sondern die Elemente entsprechend ihrer Priorität (ein spezieller Key) behandelt. Implementiert wird dies durch einen Heap in einem eindimensionalen Feld.

\begin{*anmerkung}
	Die Grundoperationen einer Priority Queue und deren Komplexitäten werden gerne in Klausuren abgefragt.
\end{*anmerkung}

Die Grundoperationen einer Priority Queue sind wie folgt:
\begin{itemize}
	\item \texttt{init(A)} $\to$ initiiert eine leeres Feld \texttt{A}
	\item \texttt{empty(A)} $\to$ ist Feld leer?
	\item \texttt{HeapInsert(A,key)} $\to$ neues Element mit \texttt{key} einfügen in \texttt{A}
	\item \texttt{HeapExtractMax(A,key)} $\to$ key des Wurzelknotens wird in 2. Parameter (oder als Funktionswert) zurückgegeben und dieses Element aus \texttt{A} herausgenommen
	\item \texttt{HeapUpdate(A,key,max)} $\to$ Kombination aus \texttt{HeapExtractMax} und \texttt{HeapInsert}: liefert in \texttt{max} den Key der Wurzel und ersetzt diesen Wert durch \texttt{key}
	\item \texttt{Maximum(A)} $\to$ Funktion, liefert den key der Wurzel
\end{itemize}

\begin{lstlisting}
subroutine HeapExtractMax(A, max)
 if (size <= 0) error("empty heap")
 
 max = A_1
 A_1 = A_size
 size = size - 1
 Heapify(A,1)
end subroutine HeapExtractMax
\end{lstlisting}
$\Rightarrow T(n)=\mathcal{O}(\log_2 n)$

\begin{lstlisting}
subroutine HeapUpdate(A, key, max)
 if (size <= 0) error("empty heap")
 
 max = A_1
 A_1 = key
 Heapify(A,1)
end subroutine HeapUpdate
\end{lstlisting}
$\Rightarrow T(n)=\mathcal{O}(\log_2 n)$

\begin{lstlisting}
subroutine HeapInsert(A, key)
 size = size + 1
 i = size
 
 do while (i > 1 .and. A_Parent(i) < key)
  A_i = A_Parent(i)
  i = Parent(i)
 end do
 A_i = key
end subroutine HeapInsert
\end{lstlisting}
$\Rightarrow T(n)=\mathcal{O}(\log_2 n)$

\begin{lstlisting}
subroutine BuildHeapNew(A)
 size = 1
 
 do i = 2, n
  HeapInsert(A,A_i)
 end do
end subroutine BuildHeapNew
\end{lstlisting}
$\Rightarrow T(n)=\mathcal{O}(\log_2 n)$