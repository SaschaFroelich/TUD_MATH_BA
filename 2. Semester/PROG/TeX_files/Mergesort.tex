\section{Mergesort}

\subsection{2-Wege-Mergesort}

Den Mergesort-Algorithmus gibt es in 2 Varianten: rekursiv und iterativ. Schauen wir uns zuerst die rekursive Variante an:
\begin{enumerate}
	\item falls $L$ leer oder nur 1 Element enthält $\to$ ok, return
	\item divide: Teile $L$ in 2 möglichst gleich lange Teillisten $L_1$ und $L_2$ und mache darauf rekursive Aufrufe \texttt{Mergesort(L\_1)} und \texttt{Mergesort(L\_2)}.
	\item conquer: Merge von $L_1$ und $L_2$
\end{enumerate}

\begin{lstlisting}
! kann nur 10 Elemente sortieren, kann man aber anpassen

subroutine _merge(lst, a, middle, b)
 integer a
 integer b
 integer middle
 integer lst(10)
 integer tmp(10)
 integer ai
 integer bi
 integer ti
 integer x
 ai = a
 bi = middle
 ti = a

 do while ((ai < middle) .or. (bi < b))
  if (ai == middle) then
   tmp(ti+1) = lst(bi+1)
   bi = bi + 1
  else if (bi == b) then
   tmp(ti+1) = lst(ai+1)
   ai = ai + 1
  else if (lst(ai+1) < lst(bi+1)) then
   tmp(ti+1) = lst(ai+1)
   ai = ai + 1
  else
   tmp(ti+1) = lst(bi+1)
   bi = bi + 1
  end if
  
  ti = ti + 1
 end do
 do x = a, b - 1
  lst(x + 1) = tmp(x + 1)
 end do
end subroutine _merge

recursive subroutine mergesort(lst, a, b)
 integer a
 integer b
 integer lst(10)
 integer diff
 diff = b - a
 
 if (diff < 2) then
  return
 else
  diff = diff / 2
  call mergesort(lst, a, a + diff)
  call mergesort(lst, a + diff, b)
  call _merge(lst, a, a + diff, b)
 endif
end subroutine mergesort
\end{lstlisting}

Mergesort ist nicht in situ. $T(n)=\mathcal{O}(n\log_2 n)$. All Lese- und Schreiboperationen sind streng sequenziell.

Die iterative Variante verläuft ähnlich, verwendet aber 4 Listen $L_1$, $L_2$, $L_3$ und $L_4$.
\begin{enumerate}
	\item Init: Teile $L$ in 2 möglichst gleich große Teillisten $L_1$ und $L_2$
	\item Erzeuge 2 Listen sortierter Paare, $L_3$ und $L_4$, indem positionell sich entsprechende Elemente von $L_1$ und $L_2$ jeweils zu einem sortierten Paar gemacht werden und in die zuletzt nicht benutzte Liste $L_3$ bzw. $L_4$ (immer abwechselnd) geschrieben werden
	\item Erzeuge 2 Listen sortierter Quadrupel in $L_1$ und $L_2$
	\item Erzeuge 2 Listen sortierter Oktupel in $L_3$ und $L_4$
	\item ...
\end{enumerate}

Die Komplexität ist auch hier $T(n)=\mathcal{O}(n\log_2 n)$, tatsächlich ist die Zeit $T(n)=\Theta(n\log_2 n)$ immer die selbe, egal ob best- oder worst case.

\subsection{$k$-Wege-Mergesort}

Hier existiert nur eine rekursive Variante, in der die Liste in $k$ Teillisten aufgeteilt wird. Man kann aber auch mit $k$ Input-Listen und $k$ Output-Listen arbeiten. Noch einige Bemerkungen:
\begin{itemize}
	\item Für den Mergeschritt wird ein $k$-elementiger Vektor von Schlüsselwerten benötigt, um die jeweils aktuellen Kopfelemente der $k$ zu verschmelzenden Listen sortiert zu speichern.
	\item Die Anzahl der Durchläufe reduziert sich gegenüber dem 2-Wege-Mergesort von $\lceil\log_2 n\rceil$ auf $\lceil\log_k n\rceil$, also um den Faktor $\frac{1}{\log_2 k}=\log_k 2$, zum Beispiel bei $k=1024=2^{10}$ Teillisten auf $\frac{1}{10}$.
\end{itemize}

\begin{tabularx}{\textwidth}{X|p{0.13\textwidth}p{0.1\textwidth}p{0.09\textwidth}p{0.1\textwidth}p{0.08\textwidth}|p{0.17\textwidth}}
	\rowcolor{lightgray} \textbf{Markoschritt} & \textbf{produziert} & \textbf{$k$-Tupel sortieren} & \textbf{Anzahl Elemente} & \textbf{Aufwand Insertion-Schritt} & \textbf{Anzahl Tupel} & \textbf{Aufwand} \\
	\hline
	\textbf{1} & $k$-Tupel: $\mathcal{O}($ & $(k\log_2 k+$ & $0\cdot$ & $\mathcal{O}(k))$ & $\cdot\frac{n}{k})$ & $\mathcal{O}(n\log_2 k)$ \\
	\textbf{2} & $k^2$-Tupel: $\mathcal{O}($ & $(k\log_2 k+$ & $k(k-1)\cdot$ & $\mathcal{O}(k))$ & $\cdot\frac{n}{k^2})$ & $\mathcal{O}(nk)$ \\
	\textbf{3} & $k^3$-Tupel: $\mathcal{O}($ & $(k\log_2 k+$ & $(k^3-k)\cdot$ & $\mathcal{O}(k))$ & $\cdot\frac{n}{k^3})$ & $\mathcal{O}(nk)$ \\
	\vdots & \vdots & \vdots & \vdots & \vdots & \vdots & \vdots \\
	\hline
	&&&&&& $\Sigma = \mathcal{O}(kn\log_2 n)$ \\
\end{tabularx}