\section{Queues}

Eine \begriff{Queue} ist eine Warteschlange und sollte mit \texttt{pop} und \texttt{inject} implementiert werden. Es ist dabei in 2 verschiedene Prinzipien zu unterscheiden:
\begin{itemize}
	\item Beim \begriff{FIFO-Prinzip} (first-in-first-out) wird das erste Element, was in die Warteschlange kommt, bearbeitet und verlässt die Warteschlange (so wie bei der Kassenschlange in der Mensa).
	\item Beim \begriff{LIFO-Prinzip} (last-in-first-out) wird das Element, was zuletzt in die Warteschlange kommt, bearbeitet (nach dem Prinzip bearbeite ich Mails: die neuste beantworte ich zuerst).
\end{itemize}

\smiley{} Professor Walter bevorzugt übrigens das LIFO-Prinzip in der Mensa. Er kommt zuletzt, hat aber als Erster sein Essen. \smiley{}

Weiterhin gibt es noch Output-resticted-queues bzw. Input-restricted-queues. Das sind deques mit \texttt{push}, \texttt{pop}, \texttt{inject}, aber ohne \texttt{eject} bzw. eine deque mit \texttt{pop} und \texttt{eject} oder \texttt{push} und \texttt{inject}.

\subsection{Grundoperationen auf einer Queue}

Eine Queue hat 4 wichtige Funktionen:
\begin{itemize}
	\item \texttt{init(Q,n)}
	\item \texttt{empty(Q)} bzw. \texttt{full(Q)}
	\item \texttt{enqueue(Q,neu)}: \texttt{inject} am tail
	\item \texttt{dequeue(Q)}: \texttt{pop} am head
\end{itemize}

Implementiert wird dies mit einem eindimensionalen Feld mit \texttt{maxlengh}, \texttt{index\_head}, \texttt{index\_tail} und \texttt{elems} (Pointer auf Feld \texttt{Q}). Hier sind die notwendigen Funktionen nur angedeutet, Details kann sich jeder selber denken.

\begin{lstlisting}
subroutine init(Q,n)
 type(queue) :: Q
 integer :: n
 
 allocate(Q(0:n-1))
 maxlengh = n
 index_head = 0
 index_tail = n-1
end subroutine init

function empty(Q)
 type(queue) :: Q
 logical :: empty
 
 empty = mod(index_tail+1,n) == index_head

end function empty

function full(Q)
type(queue) :: Q
logical :: full

 full = mod(index_tail+2,n) == index_head
 ! ein Element bleibt ungenutzt

end function full

subroutine enqueue(Q,neu)
 type(queue) :: Q
 type(element) :: neu
 
 ! ...
 index_tail = mod(index_tail+1,n)
end subroutine enqueue

subroutine dequeue(Q)
type(queue) :: Q

! ...
index_head = mod(index_head+1,n)
end subroutine dequeue
\end{lstlisting}