\section{Messbarkeit}\setcounter{equation}{0}
Wir führen zunächst das \lebesque-Maß ein und behandeln dann messbare Mengen und messbare Funktionen.

\subsection{\lebesque-Maß}
\begin{*definition}[Quader, Volumen]
	Wir definieren die Menge 
	\begin{align*}
		\mathcal{Q} &:= \left\{ I_1 \times \dotsc \times I_n \subset\mathbb{R}^n \mid I_j\subset\mathbb{R}\text{ beschränktes Intervall} \right\}
	\end{align*}
	$\emptyset$ ist auch als beschränktes Intervall zugelassen. $Q\in\mathcal{Q}$ heißt Quader.
	
	Sei $\vert I_j\vert :=$ Länge des Intervalls $I_j\subset\mathbb{R}$ (wobei $\vert\emptyset\vert = 0$), dann heißt 
	\begin{align}
		\proplbl{messbarkeit_definition_volumen_eq}
		v(Q) &:= \vert I_1\vert \cdot \dots \cdot \vert I_n\vert \quad \text{für}\; Q = I_1\times \dotsc\times I_n \in\mathbb{Q}\notag
	\end{align}
	Volumen von $Q$
\end{*definition}

\begin{*definition}[\person{Lebesgue}-Maß]
	Dafür betrachte eine (Mengen-) Funktion $\vert .\vert :\mathcal{P}(\mathbb{R}^n)\to [0,\infty]$ mit \begin{align}
		\proplbl{messbarkeit_definition_lebesque_mass}
		\vert \mu \vert &= \inf \left\{ \left. \sum_{j=1}^{\infty} v(Q_j) \;\right|\; M\subset\bigcup\limits_{j=1}^\infty Q_j, \; \text{$Q_j\in\mathcal{Q}$ Quader} \right\}\quad\forall M\subset\mathbb{R}^n,
	\end{align}
	die man \person{Lebegue}-Maß auf $\mathbb{R}^n$ nennt.
\end{*definition}

\begin{proposition}
	Es gilt: 
	\begin{align}
		M_1 \subset M_2 &\Rightarrow \vert M_1 \vert \le \vert M_2\vert\notag
	\end{align}
	und die Abbildung $\mu\mapsto \vert \mu\vert$ ist $\sigma$-subadditiv, d.h. 
	\begin{align}
		\left\vert \bigcup_{j=1}^\infty M_k\right\vert &\le \sum_{k=1}^\infty \vert M_k\vert, \quad\text{für } M_j\subset\mathbb{R}^n, \;j\in\mathbb{N}_{\ge 1}\notag
	\end{align}
\end{proposition}

\begin{proof}
	\begin{itemize}
		\item klar
		\item Finde Quader $Q_{k_j}$ mit $M_k\subset\bigcup Q_{k_j}$, $\sum v(Q_{k_j})\le\vert M_k\vert+\frac{\varepsilon}{2^k}$. Wegen $\bigcup_{k=1}^\infty M_k\subset \bigcup_{j,k=1}^\infty v(Q_{k_j}) \le \sum_{k=1}^\infty \vert M_k\vert + \epsilon$ folgt 
		\begin{align*}
		\left\vert\bigcup_{k=1}^\infty M_k\right\vert \le \sum_{j,k=1}^\infty v(Q_{k_j}) \le \sum_{k=1}^\infty \vert M_k\vert + \epsilon
		\end{align*}
	\end{itemize}
\end{proof}

\begin{*definition}[Nullmenge]
	$N\subset\mathbb{R}^n$ heißt Nullmenge, falls $\vert N \vert = 0$. Offenbar gilt:
\end{*definition}

\begin{conclusion}
	Es ist $v(Q) = \vert Q\vert$ $\forall Q\in\mathcal{Q}$
	
	Damit im folgenden Stets $\vert Q\vert$ statt $v(Q)$
\end{conclusion}
\begin{proof}
	$v(Q) = v(\cl Q)$ und $\vert Q\vert = \vert \cl Q\vert\Rightarrow Q$ abgeschlossen. Finde neue Quader $Q_j$ mit $Q\subset\bigcup Q_j$ und $\sum v(Q_j)\le\vert Q\vert+\varepsilon$. Da $Q$ kompakt $\Rightarrow$ Überdeckung durch endlich viele $Q_j$, geeignete Zerlegung von $Q_j\Rightarrow v(Q)\le\sum v(Q_j)\Rightarrow \vert Q\vert\le v(Q)\le\vert Q\vert+\varepsilon$
\end{proof}

\begin{*definition}
	Eine Eigenschaft gilt f.ü. auf $M\subset\mathbb{R}^n$, falls eine Nullmenge existiert, sodass die Eigenschaft $\forall x\in M\setminus N$ gilt. Man sagt auch, dass die Eigenschaft für fast alle $x\in M$ gilt.
\end{*definition}

\subsection{Messbare Mengen}
\begin{*definition}[messbar]
	Eine Menge $M\subset\mathbb{R}^n$ heißt messbar, falls 
	\begin{align}
		\vert \tilde{M}\vert = \vert \tilde{M}\cap M\vert + \vert \tilde{M}\setminus M\vert \quad\forall \tilde{M}\in\mathbb{R}\notag
	\end{align}
	Beim Nachweis der Messbarkeit muss man nur "`$\ge$"' prüfen.
\end{*definition}

\begin{proposition}
	\begin{enumerate}[label={(\alph*)}]
		\item $\emptyset$, $\mathbb{R}^n$ sind messbar
		\item $M\subset\mathbb{R}^n$ messbar $\Rightarrow$ $M^C = \mathbb{R}^n\setminus M$ messbar 
		\item $M_1, M_2, \dotsc\subset\mathbb{R}^n$ messbar $\Rightarrow$ $\bigcup_{j=1}^\infty M_j$, $\bigcap_{j=1}^\infty M_j$ messbar
	\end{enumerate}
\end{proposition}
\begin{proof}\hspace*{0pt}
	\begin{itemize}[topsep=\dimexpr-\baselineskip / 2\relax]
		\item wegen $\vert\emptyset\vert = 0$ und: $\vert \tilde{M}\vert \le \vert\tilde{M}\setminus\emptyset\vert = \vert\tilde{M}\vert$
		\item wegen $\tilde{M}\cap M = \tilde{M}\setminus M^C$, $\tilde{M}\setminus M = \tilde{M}\cap M^C$ $\Rightarrow$ Behauptung
		\item offenbar $M_1\cap...\cap M_k$ messbar und $M_1\cup...\cup M_k$ messbar, wähle $A=\bigcup M_j\Rightarrow A$ messbar
	\end{itemize}
\end{proof}

\begin{proposition}
	Es gilt: \begin{enumerate}[label={(\alph*)}]
		\item alle Quader sind Messbar ($Q\in\mathcal{Q}$)
		\item Offene und abgeschlossene $M\subset\mathbb{R}^n$ sind messbar
		\item alle Nullmengen sind messbar
		\item Sei $M\subset\mathbb{R}^n$ messbar, $M_0\subset\mathbb{R}^n$, beide Mengen unterscheiden sich voneinander nur um eine Nullmenge, d.h. $\vert (M\setminus M_0)\cup (M_0\setminus M)\vert = 0$ \\
		$\Rightarrow$ $M_0$ messbar.
	\end{enumerate}
\end{proposition}

\subsection{Messbare Funktionen}
\begin{*definition}[messbar]
	Eine Funktion $f:D\subset\mathbb{R}\to\overline{\mathbb{R}}$ heißt messbar, falls $D$ messbar ist und $f^{-1}(U)$ für jede offene Menge $U\subset\overline{\mathbb{R}}$ messbar ist.
\end{*definition}

\begin{*definition}[charakteristische Funktion]
	Für $M\subset\mathbb{R}^n$ heißt $\chi_\mu:\mathbb{R}^n\to\mathbb{R}$ mit \begin{align*}
		\chi_\mu = \begin{cases}
			1, &x\in M\\ 0, &x\in\mathbb{R}^n\setminus M
		\end{cases}
	\end{align*}
	charakteristische Funktion von $M$.
\end{*definition}

\begin{*definition}[Treppenfunktion]
	Eine Funktion $h:\mathbb{R}^n\to\mathbb{R}$ heißt Treppenfunktion, falls es $M_1, \dotsc, M_k\subset\mathbb{R}^n$  und $c_1,\dotsc,c_k\in\mathbb{R}$ gibt mit 
	\begin{align}
		h(x) = \sum_{j=1}^k a_j \chi_{\mu_j}(x)\notag
	\end{align}
\end{*definition}

\begin{*definition}[Nullfortsetzung]
	Für $f:D\subset\mathbb{R}^n\to\overline{\mathbb{R}}$ definieren wir die Nullfortsetzung $\overline{f}:\mathbb{R}^n\to\overline{\mathbb{R}}$ durch \begin{align}
		\overline{f}(x) &:= \begin{cases}
			f(x), &x\in D\\ 0,&x\in\mathbb{R}^n\setminus D
		\end{cases}\notag
	\end{align}
\end{*definition}

\rule{0.4\linewidth}{0.1pt}

\begin{example}
	Folgende Funktionen sind messbar
	\begin{itemize}
		\item Stetige Funktionen auf offenen und abgeschlossenen Mengen, insbesondere konstante Funktionen sind messbar
		\item Funktionen auf offenen und abgeschlossenen Mengen, die f.ü. mit einer stetigen Funktion übereinstimmen
		\item $\tan$, $\cot$ auf $\mathbb{R}$ (setzte z.b. $\tan\left(\frac{\pi}{2}+k\pi\right) = \cot(k\pi) = 0$ $\forall k$)
		\item $x\to \sin\frac{1}{x}$ auf $[-1,1]$ (setzte beliebigen Wert in $x=0$)
		\item $\chi_M:\mathbb{R}\to\mathbb{R}$ ist für $\vert\partial M\vert = 0$ messbar auf $\mathbb{R}$ (dann ist $\chi$ auf $\inn M$, $\ext M$ stetig)
	\end{itemize}
\end{example}
