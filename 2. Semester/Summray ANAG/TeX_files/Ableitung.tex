\section{Ableitung} \setcounter{equation}{0}
\proplbl{section_ableitung}

\begin{*definition}[differenzierbar, Ableitung]
	Sei $f: D\subset \mathbb{R}^n \to K^m$, $D$ offen, heißt \begriff{differenzierbar} in $x\in D$, falls es lineare Abbildung $A\in \Lin(K^n, K^m)$ gibt mit \begin{align}
		\proplbl{definition_ableitung}
		\Aboxed{f(x) &= f(x_0) + A(x-x_0) + o(\vert x-x_0 \vert), x\to x_0}\\
		\text{mit } A(x-x_0) = f'(x_0) \cdot (x - x_0)
	\end{align}
	
	Abbildung $A$ heißt dann \begriff{Ableitung} von $f$ in $x_0$ und wird mit $f'(x_0)$ bzw. $\mathrm{D}f(x_0)$ bezeichnet.
\end{*definition}

\begin{*remark}
	Affin lineare Abbildung $\tilde{A}(x) := f(x_0) + f'(x_0)\cdot(x-x_0)$ approximiert die Funktion $f$ in der Nähe von $x_0$ und heißt \begriff{Linearisierung} von $f$ in $x_0$ (man nennt \propref{definition_ableitung} auch Approximation 1. Ordnung von $f$ in der Nähe von $x_0$).
\end{*remark}
\begin{conclusion}[Wann ist $f$ diffbar?]
	$f: D \subset K^n \to K^m, D$ offen, $x_0 \in D$. Für jedes $A \in \Lin(K^n,K^m)$ sei $D\to K^m$ zugeh. Restfkt. gegeben durch
	\begin{align}
		f(x) = f(x_0) + A(x-x_0) +r_a(x) \quad \forall x \in D
	\end{align}
	Dann:
	\begin{align}
		f \text{ ist diffbar in } x_0 \text{ mit Abl.} A &\Leftrightarrow \exists A \in \Lin(K^n,K^m): r_A(x) = o(\vert x-x_0\vert) x \to x_0 \notag\\
		& \quad \text{d.h. } \lim\limits_{\substack{x\to x_0 \\ x\neq x_0}} \frac{r_A(x)}{\vert x - x_0\vert} = 0 \notag\\
		&\Leftrightarrow \exists A \in \Lin(K^n,K^m): \limits_{\substack{x\to x_0 \\ x\neq x_0}} \frac{r_A(x) -f(x_0) -A(x-x_0)}{(x-x_0)} = 0 \notag
	\end{align}
\end{conclusion}

\begin{definition}[diffbar auf $D$, stetig diffbar]
	\begin{itemize}
		\item falls $f$ difbar in allen $x_0 \in D$, heißt $f$ diffbar auf $D$
		\item $f^{'}: D \to \Lin(K^n, K^m) (\cong K^{m\times n})$ Abl. von $f$ (matrixwertig)
		\item f \emph{stetig diffbar} bzw. $C^1$-Fkt., wenn $f^{'}$ stetig auf $D$
		$C^1(D,K^m) = \{f: D \to K^m \mid f \text{ stetig diffbar auf } D} = C^1(D)$
	\end{itemize}
\end{definition}


\begin{proposition}
	\proplbl{diffbar_impl_stetig}
	Sei $f:D\subset K^n \to K^m$, $D$ offen, differenzierbar in $x_0\in D$. Dann:
	\begin{enumerate}[label={\arabic*)}]
		\item $f$ ist stetig in $x_0$
		\item Die Ableitung $f'(x_0)$ ist eindeutig bestimmt.
	\end{enumerate}
\end{proposition}

\begin{proof}
	\begin{enumerate}
		\item Sei $A,\tilde{A}\in L(K^n,K^m)$ Ableitungen von $f$ in $x_0$, betrachte $x=x_0+ty$, wobei $y\in K^n$ mit $\vert y\vert =1$ fest, $t\in\real_{>0}$ (offenbar $\vert x-x_0\vert=t$) \\
		$\Rightarrow (A-\tilde{A})(ty)=o(\vert ty\vert)\Rightarrow (A-\tilde{A})(y)=\frac{o(t)}{t}\to 0$ \\
		$\Rightarrow (A-\tilde{A})(y)=0\Rightarrow A-\tilde{A}=0\Rightarrow A=\tilde{A}\beha$
		\item $\lim f(x)=1=\lim\big(f(x_0)+f'(x_0)(x-x_0)+o(\vert x-x_0\vert)\big)=f(x_0)\beha$
	\end{enumerate}
\end{proof}

\subsection{Spezialfälle für \texorpdfstring{$K=\mathbb{R}$}{K=R}}
\begin{enumerate}[label={\arabic*)},leftmargin=\widthof{1)\ },topsep=-5pt]
	\item \proplbl{spezialfall_ableitung_m1_item} \uline{$m=1\negthickspace:\, f\negthickspace:\mathbb{R}^n\to \mathbb{R}$}\\[0.6ex]
	$f'(x_0)\in \mathbb{R}^{1\times n}$ Zeilenvektor, $f'(x_0)$ betrachtet als Vektor im $\mathbb{R}^n$ auch \begriff{Gradient} genannt.
	
	Offenbar gilt $f'(x_0)\cdot y = \langle f'(x_0), y\rangle\;\forall y\in\mathbb{R}^n$ (Matrizenmultiplikation = Skalarprodukt) \\
	$\Rightarrow$ \propref{definition_ableitung_zwei} hat die Form \begin{align}
		\proplbl{spezialfall_ableitung_m1}
		f(x) = \underbrace{f(x_0) + \langle f'(x_0), x - x_0\rangle}_{\mathclap{\text{affin lineare Funktion: }\tilde{A}: \mathbb{R}\to \mathbb{R} \,(\text{in }x)}} + o\big( \vert x - x_0\vert \big)
	\end{align}
	Graph von $f$ ist Fläche im $\mathbb{R}^{n\times 1}$, genannt \begriff{Tangentialebene} vom Graphen von $f$ in $\big(x_0, f(x_0)\big)$.
	
	\item \proplbl{spezialfall_ableitung_n1} \uline{$n=1\negthickspace: f\negthickspace: D\subset \mathbb{R}\to \mathbb{R}^n$}\\[0.6ex]
	$f$ (bzw.  Bild $f[D]$) ist Kurve im $\mathbb{R}^n$ ($\cong \mathbb{R}^{m\times 1}$). \propref{definition_ableitung_zwei} kann man schreiben als \begin{align*}
		f(x_0 + t) = \underbrace{f(x_0) + t\cdot f'(x_0)}_{\mathclap{\text{Affin lineare Abb. }\tilde{A}:\mathbb{R}\to \mathbb{R}^m \text{ (in $t$)}}} + o(t), t\to 0, t\in\mathbb{R}
	\end{align*}
	\zeroAmsmathAlignVSpaces
	\begin{alignat}{2}
		\notag &\Leftrightarrow\quad& \underbrace{\frac{f(x_0 + t) - f(x_0)}{t}}_{\mathclap{\text{\begriff{Differenzenquotient} von $f$ in $x_0$}}} &= f'(x_0) + o(1), t\to 0 \\
		\proplbl{differentialquotient} &\Leftrightarrow& \underbrace{\lim\limits_{t\to 0} \frac{f(x_0 + t) - f(x_0)}{t}}_{\mathclap{\text{Differentialquotient}}} &= f(x_0)
	\end{alignat}
	
	\emph{beachte:} \begin{itemize}
		\item $f$  \gls{diffbar} in $x_0$ $\Leftrightarrow$ Differentialquotient existiert in $x_0$
		\item \propref{differentialquotient} nicht erklärt im Fall von $n>1$
	\end{itemize}

	\begin{interpretation}[ für $m > 1$]
		$f'(x_0)$ heißt \begriff{Tangentenvektor} an die Kurve in $f(x_0)$. Falls $f$ nicht  \gls{diffbar} in $x_0$ bzw. $x_0$ Randpunkt in $D$ und ist $f(x_0)$ definiert, so betrachtet man in \propref{differentialquotient} auch einseitige Grenzwerte (vgl. \propref{einseitige_grenzwerte}).
		
		$\lim\limits_{t\downarrow 0} \frac{f(x_0 + t) - f(x_0)}{t} = f_r'(x_0)$ heißt \begriff[Ableitung!]{rechtsseitige} \uline{Ableitung} von $f$ in $x_0$ (falls existent), analog ist $\lim\limits_{t\uparrow 0}$ die \begriff[Ableitung!]{linksseitige} \uline{Ableitung} $f_l'(x_0)$.
	\end{interpretation}

	\item \uline{$n=m=1\negthickspace:\;f\negthickspace: D\subset \mathbb{R}\to \mathbb{R}$} (vgl. Schule)\\[0.6ex]
	$f'(x_0)\in \mathbb{R}$ ist Zahl und \propref{differentialquotient} gilt (da Spezialfall von \propref{spezialfall_ableitung_n1}).
	
	\emph{Beobachtung:} \propref{spezialfall_ableitung_n1} gilt allgemein für $n=1$, nicht für $n>1$!
\end{enumerate}
\vspace*{1.5
	em}

\begin{conclusion}
	Sei $f:D\subset K\to K^n$, $D$ offen. Dann:
	\begin{align}
		\notag& \text{$f$ ist differenzierbar in $x_0\in D$ mit Ableitung $f'(x_0)\in L(K, K^m)$} \\
		\Leftrightarrow\quad
		& \proplbl{differentialquotient_prop} \exists f'(x_0) \in L(K, K^m): \lim\limits_{y\to 0} \frac{f(x_0 + y) - f(x_0)}{y} = f'(x_0) \\
		\notag 
		& \text{alternativ: } \lim\limits_{x\to x_0} \frac{f(x) - f(x_0)}{x - x_0} = f'(x_0)
	\end{align}
\end{conclusion}

\subsection{Einfache Beispiele für Ableitungen}
\begin{example}[affin lineare Funktionen]
	\proplbl{ableitung_linear}
	Sei $f:K^n\to K^m$ affin linear, d.h. \begin{align*}
		f(x) = A\cdot x + a\quad \forall x\in K^n, \text{ mit } A\in L(K^n, K^m), \, a\in K^m \text{ fest}
	\end{align*}
	Dann gilt für beliebiges $x_0\in K^n$:
	\zeroAmsmathAlignVSpaces**
	\begin{align*}
		f(x) &= A\cdot x_0 + a + A(x - x_0) \\
		&=f(x_0) + A(x - x_0)
	\end{align*}
	\zeroAmsmathAlignVSpaces
	\begin{align*}
		\xRightarrow{(\ref{definition_ableitung})}\;\; \text{$f$ ist  \gls{diffbar} in $x_0$ mit } f'(x_0) = A
	\end{align*}
	Insbesondere gilt für konstante Funktionen $f'(x_0) = 0$
\end{example}

\begin{example}[quadratische Funktion]
	\proplbl{ableitung_beispiel_euklidische_norm}
	Sei $f:\real^n\to \real$ mit $f(x)=\vert x\vert^2$ \\
	für beliebiges $x_0$ gilt:
	\begin{align}
		\vert x-x_0\vert^2 &= \langle x-x_0,x-x_0\rangle \notag \\
		&= \vert x\vert^2 - \vert x_0\vert^2 - 2\langle x_0,x-x_0\rangle \notag
	\end{align}
	$\Rightarrow f(x) = f(x_0) + 2\langle \underbrace{2x_0}_{\text{Ableitung}},x-x_0\rangle + \underbrace{\vert x-x_0\vert^2}_{o(\vert x-x_0\vert)}$ \\
	$\Rightarrow f$ ist differenzierbar in $x_0$ mit $f'(x_0)=2x_0$, offenbar ist $f'$ stetig, also $f\in C^1(\real^n)$
\end{example}

\begin{example}[Funktionen mit höherem Exponent]
	Sei $f:K\to K$, $f(x) = x^k$, $k\in\mathbb{N}$.
	\begin{itemize}[leftmargin=\widthof{$\,k=0$:\ }]
		\item[$k=0$:] $f(x) = 1\;\forall x$ $\Rightarrow$ $f'(x_0) = 0\;\forall x_0\in\mathbb{C}$ (vgl. \propref{ableitung_linear})
		\item[$k\ge 1$:] Es gilt \\
		\renewcommand{\arraystretch}{1.2}
		\begin{tabularx}{\linewidth}{r@{\ \ }r@{$\,$}X}
			& $(x_0 + y)^k$ & $\displaystyle = \sum_{j=0}^{k}\binom{k}{j} x_0^{k-j}\cdot y^j = x_0^k + k\cdot x_0^{k-1}\cdot y + o(y),\;y\to 0$ \\
			$\Rightarrow$ & $f(x_0 + y)$ & $= f(x_0) + k\cdot x_0^{k-1}\cdot y + o(y), y\to 0$ \\
			$\xRightarrow{(\ref{definition_ableitung})}$ & $f'(x_0)$ & $= k\cdot x_0^{k-1}$
		\end{tabularx}
	\end{itemize}
	\emph{beachte:} gilt in $\mathbb{C}$ und $\mathbb{R}$.
\end{example}

\begin{example}[Exponentialfunktion]
	$f:K\to K$ mit $f(x)=e^x$ \\
	mit Differentialquotient $\Rightarrow$ $f$ ist differenzierbar mit $f'(x_0)=e^{x_0}\Rightarrow f\in C^1(K)$
\end{example}

\begin{example}[Betragsfunktion]
	\proplbl{ableitung_beispiel_betrag}
	$f:\real^n\to \real$ mit $f(x)=\vert x\vert$ \\
	$f$ ist nicht differenzierbar in $x_0=0$, denn angenommen, $f'(x_0)\in\real^n$ existiert und fixiere $y\in\real^n$, $\vert y\vert=1$ \\
	$\Rightarrow \vert ty\vert=0+\langle f'(0),ty\rangle + o(\vert t\vert)$, $t\to 0$ \\
	$\Rightarrow t\neq 0\Rightarrow \frac{\vert t\vert}{t}=\langle f'(0),y\rangle+\frac{o(t)}{t}\Rightarrow\pm 1 =$ feste Zahl in $\real_+\to 0\Rightarrow\lightning\beha$
	
	\emph{Folglich:} $f$ stetig in $x_0\not\Rightarrow f$ differenzierbar in $x_0$, das heißt Umkehrung von \propref{diffbar_impl_stetig} gilt nicht!
\end{example}

\begin{proposition}[Rechenregeln]
	\proplbl{ableitung_rechenregeln}
	Sei $D\in K^n$ offen, $f,g: D\to K^m$, $\lambda: D\to K$  \gls{diffbar} in $x_0\in D$ \\
	$\Rightarrow$ $(f\pm g): D\to K^m, (\lambda\cdot f):D\to K^m, (f\cdot g):D\to K$ sind  \gls{diffbar} in $x_0\in D$ und $\frac{1}{\lambda}:D\to K$ ist  \gls{diffbar} in $x_0$, falls $\lambda(x_0)\neq 0$
	mit
	\begin{enumerate}[label={\alph*)}]
		\item $(f\pm g)'(x_0) = f'(x_0) \pm g'(x_0)\in K^{m\times 1}$
		\item $(\lambda\cdot f)'(x_0) = \lambda (x_0)\cdot f'(x_0) + f(x_0)\cdot \lambda'(x_0)\in K^{m\times n}$
		\item $(f\cdot g)' (x_0) = \transpose{f(x_0)}\cdot g'(x_0) + \transpose{g(x_0)}\cdot f'(x_0)\in K^{m\times n}$
		\item $\left( \frac{\mu}{\lambda}\right)'(x_0) = \frac{\mu'(x_0)\cdot\lambda(x_0)-\mu(x_0)\cdot \lambda'(x_0)}{(\lambda(x_0))^2}$
	\end{enumerate}
\end{proposition}
\begin{proof}
	\begin{enumerate}
		\item nutze Definition diffbar.
		\item nutze a) für $g=\lambda$
		\item analog zu b)
		\item zeige $\left(\frac{1}{\lambda}\right)'(x_0)=-\frac{\lambda'(x_0)}{\lambda(x_0)^2}$, Rest folgt mit $f=\mu$.
	\end{enumerate}
\end{proof}

\begin{example}
	Sei $f:D\in K^n\to K^m$, $c\in K$, $f$  \gls{diffbar} in $x_0\in D$\\
	$\xRightarrow{\ref{ableitung_rechenregeln}\ b)} (c\cdot f) = c\cdot f'(x_0)$ (da $c$ konst. Funktion $D\to K$)
\end{example}

\begin{example}[Polynom]
	Sei $f:K\to K$, Polynom $f(x) = \sum\limits_{l=0}^{k}a_l x^l$ \\
	$\Rightarrow$ $f$  \gls{diffbar} $\forall x_0\in K$ mit $f'(x_0) = \sum\limits_{l=1}^k l a_l x_0^{l-1}$
\end{example}

\begin{example}
	Sei $f=\frac{f_1}{f_2}$ rationale Funktion auf $\mathbb{R}$ (d.h. $f_1, f_2:K\to K$ Polynom) \\
	$\Rightarrow$ $f$ ist  \gls{diffbar} auf $K\setminus \{ \text{Nullstellen von }f_2 \}$
\end{example}

\begin{example}[Sinus und Cosinus]
	$\sin, \cos: K\to K$ ($\mathbb{R}$ bzw. $\mathbb{C}$) $\forall x_0\in K$.
	
	Denn:{\zeroAmsmathAlignVSpaces
	\begin{align*}
		 \frac{\sin y}{y} = \frac{e^{iy} - e^{-iy}}{2iy} = \frac{1}{2}\cdot \left( \frac{e^{iy} - 1}{iy} + \frac{e^{-iy} - 1}{-iy} \right) \xrightarrow[\text{vgl. \eqref{exp_limit_1}}]{y\to 0} 1,
	\end{align*}}
	nutze $\exp$ Definition für $\sin$, Differentialquotient und Additionstheoreme. Analog für den Kosinus.
\end{example}

\subsection{Rechenregeln}
\begin{*definition}
	Sei $f:D\subset K^n \to K^m$, $D$ offen.
	
	Falls $f$  \gls{diffbar} in allen $x_0\in D$, dann heißt $f$ \begriff{differenzierbar} auf $D$ und Funktion $f':D\to L(K^n, K^m)$ heißt \begriff{Ableitung} von $f$.
	
	Ist zusätzlich Funktion $f': D\to L(K^n, K^m)$ stetig, dann heißt Funktion $f$ \begriff{stetig differenzierbar} (auf $D$) bzw. \mathsymbol{C1}{$C^1$}\emph{-Funktion} (auf $D$).
	
	$C^1(D, K^m):= \left\lbrace f: D\to K^m \mid f \text{ stetig  \gls{diffbar} auf } D \right\rbrace$
\end{*definition}

\begin{example}
	\begin{enumerate}[label={\alph*)}]
		\item $f(x) = x^k\;\forall x\in\mathbb{R},\, k\in\mathbb{N}_{\ge 0}$ \\
		$\Rightarrow$ $f'(x) = k\cdot x^{k-1}\;\forall x\in \mathbb{R}$ \\
		$\Rightarrow$ offenbar stetige Funktion \\ $\Rightarrow$ $f\in C^1(\mathbb{R}, \mathbb{R})$
		
		\item $f(x) = e^x\;\forall x\in\mathbb{C}$ \\
		$\Rightarrow f'(x) = e^x \;\forall x\in\mathbb{C}$ stetig \\
		$\Rightarrow$ $f\in C^1(\mathbb{C},\mathbb{C})$
		
		\item $f(x) = \vert x \vert^2\;\forall x\in\mathbb{R}^n$ \\
		$\Rightarrow$ $f(x) = 2x\;\forall x\in\mathbb{R}^n$, offenbar stetig \\
		$\Rightarrow$ $f\in C^1(\mathbb{R}^n, \mathbb{R})$
	\end{enumerate}
\end{example}

\begin{example}
	\proplbl{ableitung_beipsiel_unstetige_ableitung}
	Sei $f:\mathbb{R}\to \mathbb{R}$ mit $f(0) = 0$, $f(x)=x^2\cdot \sin\left(\frac{1}{x}\right)$ $\forall x\neq 0$.
	
	Wegen \begin{align*}
		\frac{\vert x^2 \cdot \sin \frac{1}{x}\vert}{\vert x \vert} \le \vert x \vert \xrightarrow{x\neq 0} 0
	\end{align*}
	folgt{ \zeroAmsmathAlignVSpaces \begin{align*}
		& f(x) = o(\vert x \vert), x\to 0 \\
		\Rightarrow\;& f(x) = f(0) + 0\cdot (x - 0) + o(\vert x - 0\vert), x\to 0 \\
		\Rightarrow\;& f \text{  \gls{diffbar} in $x=0$ mit $f'(0) = 0$}
	\end{align*}}
	
	Rechenregeln liefern $x\neq 0$: \begin{align*}
		f'(x) = 2x\cdot\sin\frac{1}{x}- \cos\frac{1}{x} \quad\forall x\neq 0
	\end{align*}
	
	Für $x_k := \frac{1}{k\pi}$ gilt: \begin{align*}
		& \lim\limits_{k\to\infty} 2 x_k \cdot \sin \frac{1}{x_k} = 0,\; \lim\limits_{k\to\infty} \cos \frac{1}{x_k} = \pm 1 \\
		\Rightarrow\;& \lim\limits_{x\to 0} f'(x) \text{ existiert nicht} \\
		\Rightarrow\;& f\notin C^1(\mathbb{R}, \mathbb{R}),
	\end{align*}
	d.h. Ableitung einer stetigen Funktion muss \emph{nicht} stetig sein.
\end{example}

\begin{conclusion}
	\proplbl{ableitung_quotientenregel}
	Seien $\lambda$, $\mu:D\to K$  \gls{diffbar} in $x_0$, $D$ offen und $\lambda(x_0)\neq 0$ \\
	$\Rightarrow$ $\left( \frac{\mu}{\lambda} \right): D\to K$  \gls{diffbar} in $x_0$ mit \begin{align*}
		\left( \frac{\mu}{\lambda} \right)' (x_0) = \frac{\lambda(x_0)\cdot \mu'(x_0) - \mu(x_0) \cdot \lambda'(x_0)}{\lambda(x_0)^2}\in K^{1\times n}
	\end{align*}
\end{conclusion}

\begin{proof}[\propref{ableitung_quotientenregel}]
	Setzte in \propref{ableitung_rechenregeln} $f=\mu$ (d.h. $m=1$) und betr. Produkt $\frac{1}{\lambda}\cdot \mu$.
\end{proof}

\begin{proposition}[Kettenregel]
	\proplbl{ableitung_kettenregel}
	Sei $f:D\subset K^n\to K^m$, $g:\tilde{D}\subset K^m\to K^l$, $D$,$\tilde{D}$ offen, $f$  \gls{diffbar} in $x_0\in D$, $g$  \gls{diffbar} in $f(x_0)\in\tilde{D}$ \\
	$\Rightarrow$ $g\circ f: D\to K^l$  \gls{diffbar} in $x_0$ mit $(g\circ f)' = g'(f(x))\cdot f'(x)$ ($\in K^{l\times n}$)
\end{proposition}
\begin{proof}
	\begin{align}
		(g\circ f)(x)  = g(f(x)) &= g(f(x_0)) + g'(f(x_0))(f(x)-f(x_0)) + o(\vert f(x)-f(x_0)\vert) \notag \\
		&= (g\circ f)(x_0) + g'(f(x_0))\cdot f(x_0)(x-x_0) + o(\vert x-x_0\vert)
	\end{align}
	$\Rightarrow$ Behauptung
\end{proof}

\begin{example}[$x$ im Exponenten]
	\proplbl{ableitung_beispiel_exponentialfunktion}
	$f:\mathbb{R}\to \mathbb{R}$, $f(x) = a^x$ ($a\in\mathbb{R}_{\ge 0}$, $a\neq 1$).
	Offenbar $a^x = \left(e^{\ln a}\right)^x = e^{x\cdot \ln a}$\\
	$\Rightarrow$ $f(x) = g(h(x))$ mit $g(y) = e^y$, $h(x) = x\cdot \ln a$
	$\Rightarrow g'(y)=e^y$, $h'(x)=\ln a\Rightarrow f'(x)=e^{x\cdot \ln a}\cdot \ln a=a^x\cdot\ln a$
\end{example}

\begin{example}[Logarithmus]
	\proplbl{ableitung_beispiel_logarithmus}
	$f:\real_{>0}\to \real$ mit $f(x)=\log_a x$, $a\in\real_{>0}$ und $a\neq 1$, $x_0\in \real_{>0}$ \\
	mit $y=\log_a x$, $y_0=\log_a x_0$ ist $x-x_0=a^y-a^{y_0}$ \\
	Differentialquotient $\Rightarrow f'(x)=\frac{1}{x\cdot\ln a}$, also $f\in C^1(\real_{>0})$
	
	Spezialfall: $(\ln(x))' = \frac{1}{x}$ $\forall x>0$
\end{example}

\begin{example}
	Sei $f:\mathbb{R}_{>0}\to \mathbb{R}$, $f(x) = x^r$ ($r\in\mathbb{R}$)
	
	Wegen $x^r = e^{r\cdot \ln x}$ liefert Kettenregeln (analog zu \propref{ableitung_beispiel_exponentialfunktion}) \begin{align*}
	f'(x_0) = \frac{r\cdot e^{r\cdot \ln x_0}}{x_0} = \frac{r\cdot x_0^r}{x_0} = r\cdot x_0^{r - 1} \quad\forall x_0>0
	\end{align*}
	
	Spezialfall: $f(x) = \frac{1}{x^k}$ $\Rightarrow$ $f'(x) = - \frac{k}{x^{k+1}}$
	
	Zu \propref{ableitung_beipsiel_unstetige_ableitung}:\begin{align*}
	f'(x) = 2x\cdot \sin\frac{1}{x} + x^2\cdot \cos\frac{1}{x} \cdot \left( - \frac{1}{x^2}\right) = 2x\cdot \sin\frac{1}{x} - \cos\frac{1}{x}
	\end{align*}
\end{example}

\begin{example}[Tangens und Cotangens]
	\proplbl{ableitung_beispiel_tangens}
	$\tan: K\setminus \{ \frac{\pi}{2} + k\cdot \pi \mid k\in\mathbb{Z} \}\to K$, $\cot:K\setminus \{ k\cdot \pi \mid k\in\mathbb{Z} \} \to K$ \\[\dimexpr - \baselineskip / 2 \relax]
	\zeroAmsmathAlignVSpaces \begin{alignat*}{3}
	\xRightarrow{\text{Quotientenregel}}&\;\;& \tan'(x_0)&= \frac{\sin'(x_0)\cos (x_0) - \cos (x_0) \cdot \sin(x_0)}{\left( \cos(x_0)\right)^2} &&\\
	&& &= \frac{\cos^2(x_0) + \sin^2(x_0)}{\cos^2(x_0)} = \frac{1}{\cos^2(x_0)} && \forall x_0\in \text{ Definitionsbereich} \\
	&& \cot'(x_0) &= - \frac{1}{\sin^2(x_0)}&&\forall x_0\in\text{ Definitionsbereich}
	\end{alignat*}
\end{example}





\begin{proposition}[Reduktion auf skalare Funktionen]
	\proplbl{ableitung_proposition_reduktion}
	Sei $f=(f_1, \dotsc, f_m): D\subset K^n\to K^m$, $D$ offen, $x_0\in D$. Dann gilt:\begin{center}
		$f$  \gls{diffbar} in $x_0$ $\Leftrightarrow$ alle $f_j$  \gls{diffbar} in $x_0$ $\forall j=1,\dotsc,m$
	\end{center}

	Im Fall der Differenzierbarkeit hat man: \begin{align}
		\proplbl{ableitung_jacobimatrix}
		f'(x_0) = \begin{pmatrix}
			f_1'(x_0) \\
			\vdots \\
			f_m'(x_0)
		\end{pmatrix} \in K^{m\times n}
	\end{align}
\end{proposition}

\begin{remark}
	Mit \propref{ableitung_proposition_reduktion} kann man die Berechnungen der Ableitungen stets auf skalare Funktionen $f:D\subset K^n\to K$ zurückführen. Die Matrix in \propref{ableitung_jacobimatrix} besteht aus $m$ Zeilen $f_j'(x_0)\in K^{1\times m}$.
\end{remark}

\begin{example}
	Sei $f:\mathbb{R}\to \mathbb{R}^2$ mit \begin{align*}
		f(t) &= \begin{pmatrix}
			t\cdot \cos( 2\pi t) \\ t\cdot \sin(2\pi t)
		\end{pmatrix}, & f'(t) &= \begin{pmatrix}
			\cos(2\pi t) - t\cdot \sin(2\pi t)\cdot 2\pi \\ \sin(2\pi t)+ t\cdot\cos(2\pi t)\cdot 2\pi
		\end{pmatrix} \in \mathbb{R}^{2\times 1},
	\end{align*}
	und $f'(0) = \binom{1}{0}$, $f'(1) = \binom{1}{2\pi}$.
\end{example}

\begin{lemma}
	\proplbl{ableitung_spezialfall_reduktion_proposition}
	Sei $f=(f_1, f_2):D\subset K^n\to K^k\times K^l$, $D$ offen, $x_0\in D$.
	
	Funktion $f$ ist  \gls{diffbar} in $x_0$ genau dann, wenn $f_1:D\to K^k$ und $f_2 :D\to K^l$  \gls{diffbar} in $x_0$.
	
	Im Falle der Differenzierbarkeit gilt\begin{align}
		\proplbl{ableitung_spezialfall_reduktion}
		f'(x_0) = \begin{pmatrix}
			f_1'(x_0) \\ f_2'(x_0)
		\end{pmatrix} \in K^{(k+l)\times n}
	\end{align}
	
	\begin{hint}
		Da $K^k\times K^l$ mit $K^{k+l}$ identifiziert werden kann, kann man $f$ auch als Abbildung von $D$ nach $K^{k+l}$ ansehen. Dementsprechend kann die Matrix in \propref{ableitung_spezialfall_reduktion} der Form \begin{align*}
			\begin{pmatrix}
				(k\times n) \text{ Matrix} \\
				(l\times n) \text{ Matrix}
			\end{pmatrix}
		\end{align*}
		auch als $((k+l)\times n)$-Matrix aufgefasst werden.
	\end{hint}
\end{lemma}

\begin{proof}\hspace*{0pt}
	\NoEndMark
	\begin{itemize}[topsep=\dimexpr - \baselineskip / 3\relax]
		\item["`$\Rightarrow$"'] Man hat
		\zeroAmsmathAlignVSpaces[3pt][3pt]
		\begin{alignat}{2}
				\proplbl{ableitung_beweis_lemma_spezialfall_reduktion} && f(x) &= f(x_0) + f'(x_0)\cdot(x - x_0) + R(x)\cdot (x - x_0), \, \;R(x) \xrightarrow{x\to x_0}0
			\intertext{da $f'(x_0)$, $R(x)\in L(K^n, K^k\times K^l)$}
				\notag\Rightarrow&\;\;& f'(x_0) &= (A_1, A_2), \; R(x) = \big( R_1(x), R_2(x) \big))
			\intertext{mit $A_1, R_1(x)\in L(K^n, K^k)$, $A_2, R(x)\in L(K^n, K^l)$}
				\proplbl{ableitung_beweis_lemma_spezialfall_reduktion_einzelableitung} \xRightarrow{\eqref{ableitung_beweis_lemma_spezialfall_reduktion}}&& f_j(x)&= f_j(x_0) + A_j \cdot (x - x_0) + R_j(x) (x - x_0),\;R_j(x)\xrightarrow{x\to x_0}0 \\
				\notag\Rightarrow&& f_j & \text{ ist  \gls{diffbar} in $x_0$ mit $f_j'(x_0) = A_j$, $j=1,2$}
		\end{alignat}
		$\Rightarrow$ Behauptung
		\item["`$\Leftarrow$"'] (es gilt auch \eqref{ableitung_beweis_lemma_spezialfall_reduktion_einzelableitung} mit $A_j = f_j'(x_0)$)
		
		Setzte \begin{align*}
		 &A=\begin{pmatrix}
			f_1'(x) \\ f_2'(x)
		\end{pmatrix},\; R(x) = \begin{pmatrix}
			R_1(x) \\ R_2(x)
		\end{pmatrix} \\
		\xRightarrow{\eqref{ableitung_beweis_lemma_spezialfall_reduktion_einzelableitung}}\;& A, R(x)\in L(K^n, K^k\times K^l) \\
		\xRightarrow{\text{mit }A_j=f_j'(x_0)}\; & f(x)= f(x_0) + A(x - x_0) + R(x)(x - x_0), R(x)\xrightarrow{x\to x_0}0
		\end{align*}
		$\Rightarrow$ $f$  \gls{diffbar} in $x_0$ und \eqref{ableitung_spezialfall_reduktion} gilt.\hfill\csname\InTheoType Symbol\endcsname
	\end{itemize}
\end{proof}

\begin{proof}[\propref{ableitung_proposition_reduktion}]
	Mehrfache Anwendung von \propref{ableitung_spezialfall_reduktion_proposition} (z.B. mit $k=1, l = m - j$ für $j=1,\dotsc, m-1$)
\end{proof}