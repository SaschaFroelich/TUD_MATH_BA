\section{Extremwerte} \setcounter{equation}{0}
\subsection{Lokale Extrema ohne Nebenbedingung}

\begin{*definition}[definit, semidefinit, indefinit]
	$f^{(k)}(x)$ für $k\ge $ heißt positiv definit (negativ definit), falls \begin{align}
		f^{(k)}(x) y^k > 0 \;(< 0) \quad\forall y\in\mathbb{R}\setminus \{0\}\notag
	\end{align}
	und positiv (negativ) semidefinit mit $\ge$ ($\le$).
	
	$f^{(k)}$ heißt indefinit, falls \begin{align}
		\exists y_1, y_2\in\mathbb{R}^n\setminus \{0\}: f^{(k)}(x) y_1^k < 0 < f^{(k)} (x) y_2^k\notag
	\end{align}
\end{*definition}

\begin{proposition}[Hinreichende Extremwertbedingung]
	Sei $f:D\subset\mathbb{R}^n\to\mathbb{R}$, $D$ offen, $f\in C^k(D,\mathbb{R})$, $x\in D$, $k\ge 2$ und sei \begin{align}
		f'(x) = \dotsc = f^{(k-1)} = 0\notag
	\end{align}
	Dann: \begin{enumerate}[label={\alph*)}]
		\item $f$ hat strenges lokales Minimum (Maximum), falls $f^{(k)}(x)$ positiv (negativ) definit
		\item \proplbl{extremwerte_hinreichende_bedinung_b}
		$f$ hat weder Minimum noch Maximum, falls $f^{(k)}(x)$ indefinit.
	\end{enumerate}
\end{proposition}

\begin{proof}\hspace*{0pt}
	Taylorscher Satz!
\end{proof}

\begin{boldenvironment}[Test Definitheit in Anwendungen]
	\person{Hesse}-Matrix $A\in\mathbb{R}^{n\times}$ ist \begin{itemize}
		\item positiv (negativ) definit $\Leftrightarrow$ alle Eigenwerte sind positiv (negativ) 
		\item indefinit $\Rightarrow$ $\exists$ positive und negative Eigenwerte
	\end{itemize}
\end{boldenvironment}

\begin{boldenvironment}[\person{Sylvester}'sches Definitheitskriterium]
	Für $n=2$ gilt
	\begin{itemize}
		\item $\det(A)<0\Leftrightarrow$ indefinit
		\item $\det(A)>0, a_1<0\Leftrightarrow$ negativ definit (Maximum)
		\item $\det(-A)>0, a_1>0\Leftrightarrow$ positiv definit (Minimum)
	\end{itemize}
\end{boldenvironment}

\begin{boldenvironment}[Algo Bestimmung Art und Lage Extremwerte]\vspace*{0pt}
	Gegeben sei eine Fkt. $f: \real^n \to \real$
	\begin{enumerate}[label={\alph*)},topsep=\dimexpr-\baselineskip/2\relax]
		\item Bestimme $n$-partielle Ableitung von $f$ und suche Pkte., die $f=0$ erfüllen
		\item Bestimme Hesse-Matrix von $f$
		\item setze gefundene Pkte.  in Hesse Matrix von $f$ ein und berechne die $\det(Hess(f))$, damit bekommt man die Art der Extremwerte.
	\end{enumerate}
\end{boldenvironment}

\subsection{Lokale Extrema mit Gleichungsnebenbedingung}
\begin{proposition}[Lagrange-Multiplikatorregel, notwendige Bedingung]
	Seien $f:D\subset\mathbb{R}^n\to\mathbb{R}$, $g:D\to\mathbb{R}^m$ stetig, diffbar, $D$ offen und sei $x\in D$ lokales Extremum von $f$ bezüglich $G$, d.h. \begin{align*}\exists r > 0: f(x)\; \substack{\le \\ \ge}\; f(y)\quad\forall y\in B_r(x)\end{align*} mit $g(y) = 0$.
	
	Falls $g'(x)$ regulär, d.h. \begin{align}
	\mathrm{rang}\; g'(x) = m\notag \end{align}dann
	\begin{align}
	\exists \lambda\in\mathbb{R}^m: f'(x) + \transpose{\lambda} g'(x) = 0\notag\end{align}
\end{proposition}

\begin{*definition}[Lagrangescher Multiplikator]
	$\lambda$ oben heißt Lagrangescher Multiplikator
\end{*definition}

\begin{boldenvironment}[Finden Extremwerte mit Lagrange-Multiplikatoren]\vspace*{0pt}
	Gegeben seien $f: \real^n \to \real$ und $g: \real^n \to \real^m$
	\begin{enumerate}[label={\alph*)},topsep=\dimexpr-\baselineskip/2\relax]
		\item Berechnene $f'(x) + \transpose{\lambda} g'(x) = 0$, wobei $\lambda \in \real^m$ ($f'$ und $g'$ mit vollst. Reduktion in Kap. 18)
		\item Beachte es müssen gleich viele Gleichung für Unbekannte sein
		\item Gleichungssystem lösen und alle Unbekannten angeben
	\end{enumerate}
\end{boldenvironment}

\subsection{Globale Extrema mit Abstrakter Nebenbedinung}
\begin{boldenvironment}[Frage]
	Bestimme sogenannte globale Extremalstelle $x_{\min}$, $x_{\max}$.
\end{boldenvironment}

\begin{boldenvironment}[Strategie]\vspace*{0pt}
	\begin{enumerate}[label={\alph*)},topsep=\dimexpr-\baselineskip/2\relax]
		\item Bestimmte lokale Extrema in $D$
		\item Bestimme globale Extrema auf $\partial D$
		\item Vergleiche Extrema aus a) und b)
	\end{enumerate}
\end{boldenvironment}