\section{Reihen}
\begin{*definition}[Partialsumme]
	Sei $X$ normierter Raum. $\{x_n\}$ Folge im normierten Raum.\\
	$s_n :=\sum_{k=1}^n x_k = x_0 + \dotsc + x_n$ heißt \begriff{Partialsumme}.
	
	Folge $\{s_n\}$ der Partialsumme heißt \highlight{(unendliche)}\begriff{Reihe} mit Gliedern $x_k$.\\
	Notation: durch Symbol $\sum_{k=0}^\infty x_k = x_0 + \dotsc = \sum_k x_k = \{s_k\}_{k\in\mathbb{N}}$
	
	Existiert der Grenzwert $s = \lim\limits_{n\rightarrow\infty} s_n$, so heißt der \begriff[Reihe!]{Summe} der Reihe.\\
	Notation: $s = \sum_{k=0}^\infty x_n$.
\end{*definition}

\begin{proposition}[\person{Cauchy}-Kriterium]
	\propref{satz_cauchy_kriterium}
	Sei $X$ normierter Raum, $\{x_k\}$ Folge in $X$. Dann
	\begin{enumerate}[label={\arabic*)}]
		\item $\sum_k x_k$ konvergiert $\Rightarrow\;\forall \epsilon > 0\,\exists n_0: \left|\left|\sum_{k=n}^m x_k\right|\right| < \epsilon\,\forall m\ge n\ge n_0$
		\item falls $x$ vollständiger, normierter Raum, gilt auch $\Leftarrow$ oben.
	\end{enumerate}
\end{proposition}
\begin{proof}
	Übungsaufgabe, benutze $\Vert s_m-s_{n-1}\Vert=\Vert\sum_{k=n}^m x_k\Vert$
\end{proof}

\begin{conclusion}
	\proplbl{folgerung_konvergenz_to_zero}
	Sei $X$ normierter Raum, $\{x_n\}$ Folge in $X$. Dann:\\
	$\sum_k x_k$ konvergiert $\Rightarrow$ $x_k\overset{k\rightarrow \infty}{\longrightarrow}0$
\end{conclusion}
\begin{proof}
	\proplbl{satz_cauchy_kriterium} mit $m=n$
\end{proof}

\begin{example}
	\begriff{geometrische Reihe} $X=\mathbb{C}, a_k:= z^k, z\in\mathbb{C}$ fest.
	
	$\sum_{k=0}^\infty z^k = \frac{1}{1-z}\,\forall z\in\mathbb{C}$ mit $|z|<1$
	$\sum_{k=0}^\infty z^k$ divergent, falls $|z|>1$
\end{example}
\begin{example}
	\begriff{harmonische Reihe} $X=\mathbb{R}, x_k := \frac{1}{k}\;(k>1)$. Reihe divergiert.
\end{example}
\begin{example}
	$X=\real$, $x_k=\frac{1}{k(k+1)}$ \\
	$s_n=^\frac{1}{2}+\frac{1}{6}+...=1-\frac{1}{2}+\frac{1}{2}-\frac{1}{3}+...=1-\frac{1}{n+1}\Rightarrow$ konvergiert gegen 1 \\
	Derartige Reihen heißen auch \begriff[Reihe!]{Teleskopreihen}: $\sum_{k=0}^{\infty} (y_k-y_{k+1})$. Diese konvertieren genau dann, wenn $\{y_k\}$ konvergiert.
\end{example}
\begin{example}
	$X=\mathbb{R}$:\[ \sum_{k=1}^\infty \frac{1}{k^s}\;\begin{cases}
	\text{konvergiert},& \text{für }s > 1\\ \text{divergiert},& \text{für }s \le 1
	\end{cases} \]
	Summe heißt \begriff{\person{Riemann}'sche Zetafunktion} \mathsymbol{zeta}{$\zeta(s)$} (für $s > 1$). Diese ist beschränkt und konvergent.
\end{example}

\begin{proposition}
	Sei $X$ normierter Raum, $\{x_n\}, \{y_n\}$ in $X, \lambda,\mu\in K$ ($\mathbb{R}$ oder $\mathbb{C}$). Dann:\\
	$\sum_k x_k, \sum_k y_k$ konvergent $\Rightarrow\;\sum_{k=0}^\infty \lambda x_k + \mu x_k$ konvergent gegen $\lambda\sum_k x_k + \mu \sum_k y_k$.
\end{proposition}
\begin{proof}
	benutze Rechenregeln für Folgen
\end{proof}

\begin{*definition}
	Reihe $\sum_k x_k$ heißt \begriff[Reihe!]{absolut konvergent}, falls $\sum_k \Vert x_k\Vert$ konvergiert.
\end{*definition}

\begin{proposition}
	Sei $X$ vollständiger, normierter Raum. Dann:\\
	$\sum_k x_k$ absolut konvergent $\Rightarrow\;\sum_k x_k$ konvergent
\end{proposition}
\begin{proof}
	Es ist $\Vert\sum_{k=n}^m x_k\Vert\le \sum_{k=n}^m \Vert x_k\Vert$ (1) \\
	$\Rightarrow \tilde{s_m}=\sum_{k=0}^m \Vert x_k\Vert$ ist CF in $\real \overset{(1)}{\Rightarrow} \sum_{k=0}^m x_k$ ist CF in $X\beha$
\end{proof}

\begin{proposition}[Konvergenzkriterien für Reihen]
	Sei $X$ normierter Raum, $\{x_k\}$ in $X, k_0\in\mathbb{N}$
	\begin{enumerate}
		\item Sei $\{x_k\}$ Folge in $\mathbb{R}$ \hfill\begriff{Majorantenkriterium}
		\begin{enumerate}[label={\alph*)}]
			\item $\Vert x_k\Vert \le \alpha_k\,\forall k\ge k_0,\sum_k \alpha_k$ konvergent $\Rightarrow\;\sum_k \Vert x_k\Vert$ konvergent
			\item $0 \le \alpha_k \le \Vert x_k\Vert\,\forall k\ge k_0,\sum_k \alpha_k$ divergent $\Rightarrow\sum_k\Vert x_k\Vert$ divergent.
		\end{enumerate}
		\item Sei $x_k\neq 0\,\forall k\ge k_0$\hfill\begriff{Quotientenkriterium}
		\begin{enumerate}[label={\alph*)}]
			\item $\frac{\Vert x_{k+1}\Vert}{\Vert x_k\Vert} \le q < 1\,\forall k\ge k_0 \;\Rightarrow\;\sum_k \Vert x_k\Vert$ konvergiert
			\item $\frac{\Vert x_{k+1}\Vert}{\Vert x_k\Vert}\,\forall k\ge k_0\;\Rightarrow \sum_k\Vert x_k\Vert$ divergiert.
		\end{enumerate}
		\item \hfill\begriff{Wurzelkriterium}
		\begin{enumerate}[label={\alph*)}]
			\item $\sqrt[k]{\Vert x_k\Vert}\le q < 1\,\forall k\ge k_0\;\Rightarrow\;\sum_k\Vert x_k\Vert$ konvergiert
			\item $\sqrt[k]{\Vert x_k\Vert} \ge 1\,\forall k\ge k_0\;\Rightarrow\;\sum_k \Vert x_k\Vert$ divergent.
		\end{enumerate}
	\end{enumerate}
\end{proposition}
\begin{proof}
	\begin{enumerate}
		\item $s_n=\sum_{k=0}^n \Vert a_k\Vert$ monoton wachsend
		\begin{enumerate}[label={\alph*)}]
			\item $\{s_n\}$ beschränkt $\Rightarrow$ konvergent
			\item $\{s_n\}$ unbeschränkt $\Rightarrow$ divergent
		\end{enumerate}
		\item 
		\begin{enumerate}[label={\alph*)}]
			\item $\Vert x_k\Vert\le q^2\Vert x_{k-2}\Vert\le ...\le q^k\Vert x_1\Vert=:a$, da $\sum_{k=0}^m a_k=\Vert x_k\Vert\sum_{k=0}^{\infty} q^k$ konvergent
			\item ist $\Vert x_k\Vert \not\to 0\overset{\text{\propref{folgerung_konvergenz_to_zero}}}{\Rightarrow}$ Behauptung
		\end{enumerate}
		\item analog zu 2., verwende $\Vert x_k\Vert\le q^k$
	\end{enumerate}
\end{proof}

\begin{example}
	\begriff{Exponentialreihe} $\exp z := \sum_{k=0}^\infty \frac{z^k}{k!}$ absolut konvergent $\forall z\in \mathbb{C}$.
	
	\mathsymbol{e}{$e$}$:=\exp(1)$ \begriff{\person{Euler}'sche Zahl}
\end{example}
\begin{example}
	\begriff{Potenzreihe}: $\sum_{k=0}^\infty a_k(z-z_0)^k$ für $z\in\mathbb{C}, a_k\in\mathbb{C}, z_0\in\mathbb{C}$.
	
	Sei \[L:=\begin{cases} \limsup\limits_{n\rightarrow\infty} \sqrt[k]{|a_k|},&\text{falls existiert}\\ \infty,&\text{sonst}\end{cases}\qquad R:=\frac{1}{L} \;(\text{mit }0 = \frac{1}{\infty}, \frac{1}{0} = \infty)\]
	
	$ |z - z_0| < R$: absolute Konvergenz,\\
	$|z-z_0| > R$: Divergenz,\\
	$|z-z_0| = R$: i.A. keine Aussage möglich.
	
	$B_R(z_0)$ heißt \begriff{Konvergenzkreis}, $R$ \begriff{Konvergenzradius}
\end{example}
\begin{example}
	\begriff{$p$-adische Brüche}. Sei $p\in\mathbb{N}_{\ge 2}$: betrachte $0,x_1x_2x_3\dotsc :=\sum_{k=1}^\infty x_k\cdot p^{-k}$ für $x_k\in\{0,1,\dotsc,p-1\}\,\forall k\in\mathbb{N}$.
\end{example}
\begin{proposition}[\person{Leibnitz}-Kriterium für alternierende Reihen in $\mathbb{R}$]
	Sei $\{x_n\}$ monoton fallende Nullfolge in $\mathbb{R}$. Dann:\\
	alternierende Reihe $\sum_{k=0}^\infty (-1)^k x_k = x_0 - x_1 + x_2 - \dotsc$ ist konvergent.
\end{proposition}

\begin{example}
	Alternierende harmonische Reihe $\sum_{k=1}^{\infty} (-1)^k\cdot\frac{1}{k}=1-\frac{1}{2}+\frac{1}{3} - \frac{1}{4}+...$ ist konvergent \\
	man kann zeigen, dass $\sum_{k=1}^{\infty} (-1)^k\cdot\frac{1}{k}=\ln 2$
\end{example}

\begin{boldenvironment}[Frage]
	Ist die Summationsreihenfolge bei Reihen wichtig?
\end{boldenvironment}
\begin{boldenvironment}[Antwort]
	im Allgemeinen nicht.
\end{boldenvironment}

\begin{*definition}[Umordnung]
	Sei $\beta:\mathbb{N}\rightarrow\mathbb{N}$ bijektive Abbildung: $\sum_{k=0}^\infty x_{\beta(k)}$ heißt \begriff{Umordnung} der Reihe $\sum_k x_k$.
\end{*definition}
\begin{proposition}
	\proplbl{satz umordnung}
	Sei $X$ normierter Raum. Dann:\\
	$\sum_{k=0}^\infty x_k = x$ absolut konvergent $\Rightarrow\;\sum_{k=0}\infty x_{\beta(k)}$ absolut konvergent für jede Umordnung.
\end{proposition}
\begin{proof}
	wegen Konvergenz der Partialsummen: $\forall\varepsilon>0\;\exists n_0:\sum_{k=n_0}^{\infty} \Vert x_k\Vert<\varepsilon$ \\
	da $b:\natur\to\natur$ bijektiv $\exists n_1:\{0,1,...,n_0\}\subset \{b(0),...,b(n_1)\}\Rightarrow \Vert\sum_{k=0}^{\infty} x_k - \sum_{k=0}^m x_{b(k)}\Vert\le \sum_{k=n_0}^{\infty} \Vert x_k\Vert<\varepsilon\Rightarrow\sum_{k=0}^m x_{b(k)}\to \sum_{k=0}^{\infty} x_k=x$ \\
	wegen $\sum_{k=0}^m \Vert x_{b(k)}\Vert\le \sum_{k=0}^{\infty} \Vert x_k\Vert\Rightarrow$ Umordnung ist absolut konvergent
\end{proof}

\begin{underlinedenvironment}[Hinweis]
	Satz \propref{satzumordnung} ist falsch, falls $\sum_{k=0}^{\infty} x_k$ nicht absolut konvergent
\end{underlinedenvironment}

\begin{proposition}
	Sei $\sum_{k=0}^\infty x_k$ konvergierende Reihe in $\mathbb{R}$, die nicht absolut konvergent ist. Dann:\\
	$\forall s\in\mathbb{R}\cup \{\pm\infty\}$ existiert $\beta:\mathbb{N}\rightarrow\mathbb{N}$ bijektiv mit $s=\sum_{k=0}^\infty x_{\beta_k}$
\end{proposition}
\begin{proof}
	für $s\in\real$: Seien $x^+_k$ und $x^-_k$ positive bzw. negative Glieder $\Rightarrow$ Reihe konvergent $\Rightarrow\sum_{k=0}^{\infty} x^{\pm}_k=\pm\infty$ \\
	summiere nun in folgender Reihenfolge: $x^+_1+x^+_2+...+x^+_n$ (Summe erstmals $>s$) $+x^-_{n+1}+x^-_{n+2}...$ (Summe erstmals $<s$) $\Rightarrow$ Partialsummen schwanken um $s\Rightarrow$ wegen $x_k\to 0$ konvergiert umgeordnete Reihe gegen $s$
\end{proof}

\begin{proposition}[\person{Cauchy}-Produkt]
	Sei $X$ normierter Raum über $\mathbb{K}$, $\sum_j x_j$ und $\sum_i \lambda_i$ absolut konvergent in $X$ bzw. $\mathbb{K}$. $\beta:\mathbb{N}\times \mathbb{N}\rightarrow \mathbb{N}$ bijektiv, $Y_{\beta(i,j)} = \lambda_i x_i\,\forall i,j\in\mathbb{N}$
	
	$\Rightarrow \sum_{l=0}^\infty Y_l = \sum_{i=0}^\infty \lambda_i \sum_{j=0}^\infty x_j$, wobei linke Reihe absolut konvergiert in $X$.
	
	\begin{tabular}{ll}
		\highlight{Spezialfall:} & $\beta(i,j) = \frac{(i+j)(i+j+1)}{2} + i$ liefert\\[5pt]
		& $\sum_{k=0}^\infty \sum_{l=0}^k \lambda_k x_{k-l} = \sum_{i=0}^\infty \lambda_i \sum_{j=0}^\infty x_j$
	\end{tabular}
\end{proposition}
\begin{proof}
	Sei $m(k,l)=\max\{k,l\}=m$ und $\tilde{b}(k,l)=m(k,l)^2+m(k,l)+k-l=n$ \\
	$\Rightarrow m(\tilde{b}^{}-1,n)\to\infty$ für $n\to\infty$ \\
	$\Rightarrow\Vert\sum_{l=0}^{n} y_l-\sum_{i=0}^{m(k,l)} \lambda_i\cdot\sum_{j=0}^{m(k,l)} x_j\Vert\le \Vert x_m\Vert\cdot\sum_{i=0}^m \vert\lambda_i\vert+\vert\lambda_m\vert\cdot\sum_{j=0}^m \Vert x_j\Vert\le \tilde{\lambda}\cdot\Vert x_m\Vert+\tilde{j}\cdot\vert \lambda_m\vert\to 0$ (3) \\
	$\sum_{i=0}^m \vert\lambda_i\vert\le \tilde{\lambda}$ \\
	$\sum_{j=0}^m \Vert x_j\Vert\le \tilde{j}$ \\
	da $\sum_{i=0}^m \lambda_i := \lambda, \sum_{j=0}^m x_j=:x$ folgt $\sum_{l=0}^n y_l=\lambda\cdot x$ für $l=\tilde{b}(i,j)$ mit $\Vert y_l\Vert,\vert\lambda_i\vert,\vert x_j\vert$ links in (3) folgt absolute Konvergenz von $\sum_{l=0}^n y_l \Rightarrow$ Behauptung für beliebige $b$ folgt mit \propref{satzumordnung}
\end{proof}

\begin{example}
	$\exp(z_1+z_2)=\exp(z_1)\cdot\exp(z_2)$, denn \\
	$\exp(z_1)\cdot\exp(z_2)=\sum_{k=0}^{\infty} \frac{z_1^k}{k!}\cdot\sum_{l=0}^{\infty} \frac{z_2^l}{l!}=\sum_{n=0}^{\infty}\sum_{m=0}^{n}\frac{z_1^m\cdot z_2^{n-m}}{m!\cdot (n-m!)}\Rightarrow$ Erweiterung mit $\frac{n!}{n!}$ gibt $\binom{n}{m}\Rightarrow\sum_{n=0}^{\infty} \frac{(z_1+z_2)^n}{n!}=\exp(z_1+z_2)$
\end{example}

\begin{proposition}[Doppelreihenproposition]
	Sei $\{x_{k,l}\}_{k,l\in\mathbb{N}}$ Doppelfolge im \person{Banach}-Raum $X$ und mögen $\sum_{l=0}^\infty \Vert x_{k,l}\Vert =:\alpha_k\,\forall k$ und $\sum_{k=0}^\infty x_k =: \alpha$ existieren.
	
	$\Rightarrow \sum_{k=0}^\infty \left(\sum_{l=0}^\infty x_{k,l}\right) = \sum_{l=0}^{\infty}\left( \sum_{k=0}^\infty x_{k,l}\right)$, wobei alle Reihen absolut konvergent sind.
\end{proposition}
\begin{proof}
	\begin{itemize}
		\item als Konvergenz der Reihen: links klar nach Vorraussetzungen \\
		$\Vert x_{kl}\Vert\le a_k\overset{\text{Maj.-Krit.}}{\Rightarrow}\sum_{k=0}^{}\infty x_{kl}:=b_l$ absolut konvergent $\sum_{l=0}^{n}\Vert b_l\Vert=\sum_{l=0}^{n}\Vert\sum_{k=0}^{\infty} x_{kl}\Vert\le \sum_{l=0}^n\sum_{k=0}^{\infty}\Vert x_{kl}\Vert\overset{\text{Add.}}{\le} \sum_{k=0}^{\infty}\sum_{l=0}^n \Vert x_k\Vert\le \sum_{k=0}^{\infty} a_k=a \Rightarrow\sum_{l=0}^{\infty} b_l$ ist absolut konvergent $\Rightarrow$ Reihen rechts sind absolut konvergent
		\item Sei nun $\varepsilon>0\Rightarrow\exists n_0:\sum_{k=n+1}^{\infty} a_k<\frac{\varepsilon}{2}, \sum_{l=n+1}^{\infty} \Vert b_l\Vert <\frac{\varepsilon}{2}\Rightarrow\Vert\sum_{k=0}^{\infty} \left( \sum_{l=0}^{\infty} x_{kl}\right) - \sum_{k=0}^n \sum_{l=0}^{n} x_{kl}\Vert =: s-s_n\le \sum_{k=n+1}^{\infty} a_k+\sum_{l=n+1}^{\infty} \Vert b_l\Vert <\varepsilon$ \\
		$\Rightarrow s_n\to s$, analog $s_n\to\sum_{l=0}^{\infty} \sum_{k=0}^{\infty} x_{kl}=:\tilde{s}\Rightarrow s=\tilde{s}\beha$
	\end{itemize}
\end{proof}