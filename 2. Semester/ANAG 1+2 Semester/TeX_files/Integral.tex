\section{Integral}\setcounter{equation}{0}
\subsection{Integral für Treppenfunktionen}
Sei $h:\mathbb{R}\to \mathbb{R}$ messbare Treppenfunktion mit \begin{align*}
	h &= \sum_{j=1}^{k} c_j \chi_{M_j}, \text{d.h. $c_j\in\mathbb{R}$, $M_j\subset\mathbb{R}$ messbar}
\end{align*}

\begin{*definition}[integrierbar, Integral, Integralabbildung]
	Sei $M\subset\mathbb{R}$ messbar.
	
	$h$ heißt \begriff{integrierbar} auf $M$, falls $\vert M_j\cap M\vert < \infty$ $\forall j: c_j\neq 0$ und \begin{align}
		\proplbl{integral_treppenfunktion_definition}
		\int_M h \D x := \int_M h(x) \D x := \sum_{j=1}^k c_m \vert M_j\cap M\vert
	\end{align}
	heißt (elementares) \begriff{Integral} von $h$ auf $M$.
	
	Menge der auf $M$ integrierbaren Treppenfunktionen ist \mathsymbol{T1}{$T^1(M)$}. $\int_M:T^1(M)\to\mathbb{R}$ mit $h\to \int_M h\D x$ ist die \begriff{Integral-Abbildung}.
\end{*definition}

Man verifiziert leicht
\begin{conclusion}
	\proplbl{integral_treppenfunktion_grundlegende_folgerung}
	Sei $M\subset\mathbb{R}^n$ messbar. Dann gilt:\begin{enumerate}[label={\alph*)}]
		\item (Linearität) Integralabbildung $\int_M:T^1(M)\to\mathbb{R}$ ist linear
		\item (Monotonie) Integral-Abbildung ist monoton auf $T^1(M)$ ,.d.h \begin{align*}
			h_1 \le h_2 \text{ auf $M$} \;\Rightarrow\;\int_M h_1 \D x \le \int_M h_2 \D x
		\end{align*}
		\item \proplbl{integral_treppenfunktion_grundlegende_folgerung_beschraenktheit}
		(Beschränktheit) Es ist $\vert \int_M h\D x \vert \le \int _M \vert h \vert \D x$ $\forall h\in T^{1}(M)$
		\item Für $h\in T^1(M)$ gilt: \\
		\begin{tabularx}{\linewidth}{X@{\ \ }c@{\ \ }X}
			\hfill $\displaystyle \int_M \vert h \vert \D x = 0$ & $\Leftrightarrow$ & $h=0$ \gls{fue} auf $M$
		\end{tabularx}
	\end{enumerate}

	\begin{underlinedenvironment}[Hinweis]
		$\int_M \vert h \vert \D x$ ist Halbnorm auf dem Vektorraum $T^1(M)$.
	\end{underlinedenvironment}
\end{conclusion}

\subsection{Erweiterung auf messbare Funktionen}
sinnvoll:
\begin{itemize}[topsep=-2\baselineskip]
	\item Linearität und Monotonie erhalten
	\item eine gewisse Stetigkeit der Integral-Abbildung
\end{itemize}
\vspace*{1em}
\begin{align}
	\proplbl{integral_messbare_funktion_forderung}
	h_k\to f\text{ in geeigneter Weise }\;\; &\Rightarrow\;\;\int_M h_k \D x \to \int_m f \D x
\end{align}
nach \propref{messbarkeit_funktion_approximation} sollte man in \eqref{integral_messbare_funktion_forderung} eine Folge von Treppenfunktionen $\{ h_k\}$ mit $h_k(x)\to f(x)$ \gls{fue} auf $M$ betrachten, \emph{aber} es gibt zu viele konvergente Folgen für einen konsistenten Integralbegriff.

\begin{example}
	\proplbl{integral_funktion_beispiel_striktere_konvergenz}
	Betrachte $f=0$ auf $\mathbb{R}$, wähle beliebige Folge $\{\alpha_k\}\subset\mathbb{R}$, dazu eine Treppenfunktion \begin{align*}
		h_k(x) = \begin{cases}
			k\cdot \alpha_k&\text{auf }(0,\frac{1}{k}) \\ 0&\text{sonst}
		\end{cases}
	\end{align*}
	Offenbar konvergiert $h_k$ gegen $0$ \gls{fue} auf $\mathbb{R}$ und man hat $h_k\to 0$ \gls{fue} auf $\mathbb{R}$ und $\int_{\mathbb{R}} h_k \D x = \alpha_k$
	\begin{tabularx}{\linewidth}{r@{\ \ }X}
		$\Rightarrow$ & je nach Wahl der Folge $\alpha_n$ liegt ganz unterschiedliches Konvergenzverhalten der Folge $\int_{\mathbb{R}} h_k \D x$ vor \\
		$\Rightarrow$ & kein eindeutiger Grenzwert in \eqref{integral_messbare_funktion_forderung} möglich \\
		$\Rightarrow$ & stärkerer Konvergenzbegriff in \eqref{integral_messbare_funktion_forderung} nötig
	\end{tabularx}
\end{example}

\begin{boldenvironment}[Motivation] \hspace*{0pt}
	\begin{itemize}[topsep=\dimexpr -\baselineskip / 2\relax]
		\item Nur monotone Folgen von Treppenfunktionen, oder
		\item Beschränktheit aus \propref{integral_treppenfunktion_grundlegende_folgerung} erhalten
	\end{itemize}
	$\Rightarrow$ jeweils gleiches Ergebnis, jedoch ist die 1. Variante technisch etwas aufwendiger
	
	Beschränktheit aus \propref{integral_treppenfunktion_grundlegende_folgerung} \ref{integral_treppenfunktion_grundlegende_folgerung_beschraenktheit} bedeutet insbesondere \begin{align*}
		\left\vert \int_M h_k\D x - \int_M f \D x \right\vert = \left\vert \int_M h_k - f \D x \right\vert \le \int_M \vert h_k - f\vert \D x \quad\forall k
	\end{align*}
\end{boldenvironment}
	
\begin{boldenvironment}[man definiert] $h_k\to f$ \gls{gdw} $\int_M \vert h_k - f\vert \D x\to 0$\\
	$\Rightarrow$ Integralabbildung stetig bezüglich dieser Konvergenz.
	
	Wegen $\int_M \vert h_k - h_l\vert \D x \le \int_m \vert h_k - f\vert \D x + \int_M \vert h_l -f \vert \D x$ müsste $\int_M \vert h_k - h_l\vert \D x$ klein sein $\forall h,l$ groß.
\end{boldenvironment}

\subsection{\lebesque-Integral}
\begin{*definition}[$L^1$-\person{Chauchy}-Folge, \person{Lebesgue}-Integral]
	Sei $M\subset\mathbb{R}^n$ messbar, Folge $\{ h_k\}$ in $T^1(M)$ heißt \begriff{$L^1$-\person{Cauchy}-Folge} (kurz $L1$-CF), falls \begin{align*}
		\forall \epsilon > 0 \; \exists k_0\in\mathbb{N}:\;\int_M \vert h_k - h_l\vert \D x < \epsilon \quad\forall h,l > k_0
	\end{align*}
	
	\stepcounter{equation}
	Messbare Funktion $f:D\subset\mathbb{R}^n\to\overline{\mathbb{R}}$ heißt \begriff{integrierbar} auf $M\subset D$, falls Folge von Treppenfunktionen $\{ h_k\}$ in $T^1(M)$ existiert mit $\{ h_k\}$ ist $L1$-CF auf $M$ und $H_k\to f$ \gls{fue} auf $M$.\marginnote{\leqnos\begin{align}\proplbl{integral_funktion_definition}\,\end{align}Formel (3) unbekannt}[-1.5\baselineskip]
	
	Für integrierbare Funktion $f$ heißt eine solche Folge $\{h_k\}$ \begriff{zugehörige $L^1$-CF} auf $M$.
	
	Wegen\begin{align}
		\left\vert\int_M h_k\D x - \int_M h_l\D x\right\vert = \left\vert \int_M (h_k - h_l) \D x\right\vert \overset{\propref{integral_treppenfunktion_grundlegende_folgerung}}{\le} \int_M \vert h_k - h_l\vert \D x
	\end{align}
	ist $\{\int_M h_k\D x\}$ \person{Cauchy}-Folge in $\mathbb{R}$ und somit konvergent.
	
	Der Grenzwert \begin{align}
		\proplbl{integral_lebesque_funktion_definition}
		\int_m f \D x &:= \int_M f(x) \D x := \lim\limits_{k\to\infty} \int_M h_k\D x
	\end{align}
	
	heißt (\lebesque)-\begriff{Integral} von $f$ auf $M$.
\end{*definition}
\begin{underlinedenvironment}[Hinweis]
	Integrale unter dem Grenzwert in \eqref{integral_lebesque_funktion_definition} sind elementare Integrale gemäß \eqref{integral_treppenfunktion_definition}.
\end{underlinedenvironment}
\begin{boldenvironment}[Sprechweise]
	$f$ integrierbar auf $M$ bedeutet stets $f:D\subset\mathbb{R}^n\to\overline{\mathbb{R}}$ messbar und $M\subset D$ messbar
\end{boldenvironment}

\begin{*definition}[Menge der integrierbaren Funktionen]
Menge der auf $M$ integrierbaren Funktionen ist \mathsymbol*{L1}{$L^1$} \begin{align*}
	L^1(M) := \left\{ f:M\subset\mathbb{R}^n\to\overline{\mathbb{R}} \mid f \text{ integierbar auf $M$} \right\}
\end{align*}
\end{*definition}

\begin{remark}\vspace*{0pt}
	\begin{enumerate}[label={\alph*)},topsep=\dimexpr -\baselineskip / 2\relax]
		\item Integral in \eqref{integral_lebesque_funktion_definition} kann als vorzeichenbehaftetes Volumen des Zylinders im $\mathbb{R}^{n+1}$ unter (über) dem Graphen von $f$ interpretiert werden.
		\item Sei $0\le h_1 \le h_2 \le \dotsc$ monotone Folge von integrierbaren Treppenfunktionen mit $h_k\to f$ \gls{fue} auf $M$ und sei Folge $\{ \int_M h_k\D x\}$ in $\mathbb{R}$ beschränkt \\
		$\Rightarrow$ \eqref{integral_lebesque_funktion_definition} gilt und monotone Folge $\{ \int_m h_k \D x \}$ konvergiert in $\mathbb{R}$ (d.h. $\{ h_k \}$ ist $L^1$-CF zu $f$)
		\item $\{ h_k\}$ aus \propref{integral_funktion_beispiel_striktere_konvergenz} ist nur dann $L^1$-CF, falls $\alpha_k\to 0$.
	\end{enumerate}
\end{remark}

\begin{boldenvironment}[Frage]
	Ist die Definition des Integrals in \eqref{integral_lebesque_funktion_definition} unabhängig von der Wahl einer konkreten $L^1$-CF $\{ h_k\}$ zu $f$?
\end{boldenvironment}

\begin{proposition}
	\proplbl{integral_funktion_unabhaengigkeit_leins_folge}
	Definition des Integrals in \eqref{integral_lebesque_funktion_definition} ist unabhängig von der speziellen Wahl einer $L^1$-CF $\{h_k\}$ zu $f$.
\end{proposition}

Vgl. Integral $\int_{M} h \D x$ einer Treppenfunktion gemäß \eqref{integral_treppenfunktion_definition} mit dem in \eqref{integral_lebesque_funktion_definition}:

Offenbar ist konstante Folge $\{ h_k\}$ mit $h_k = h$ $\forall k$ $L^1$-CF zu $h$ \\
$\xRightarrow[\eqref{integral_lebesque_funktion_definition}]{\propref{integral_funktion_unabhaengigkeit_leins_folge}}$ Integral $\int_M h \D x$ in \eqref{integral_lebesque_funktion_definition} stimmt mit elementarem Integral in \eqref{integral_treppenfunktion_definition} überein.

\begin{conclusion}
	Für eine Treppenfunktion stimmt das in \eqref{integral_treppenfunktion_definition} definierte elementare Integral mit dem in \eqref{integral_lebesque_funktion_definition} definierte Integral überein. Insbesondere ist der vor \eqref{integral_treppenfunktion_definition} eingeführte Begriff integrierbar mit dem in \eqref{integral_funktion_definition} identisch\\
	$\Rightarrow$ wichtige Identität \eqref{integral_treppenfunktion_definition} mit Treppenfunktion $\chi_M$ für $\vert M \vert < \infty$: \begin{align*}
		\vert M \vert &= \int_M 1\D x = \int_M \D x\quad\forall M\in\mathbb{R},\text{ $M$ messbar},
	\end{align*}
	d.h. das Integral liefert Maß für messbare Mengen.
\end{conclusion}

\begin{proof}[\propref{integral_funktion_unabhaengigkeit_leins_folge}]
	\NoEndMark
	beachte: alle Integrale im Beweis sind elementare Integrale gemäß \eqref{integral_treppenfunktion_definition}.
	
	\begin{itemize}
	
	\item Sei $f:M\subset\mathbb{R}\to\overline{\mathbb{R}}$ integrierbar und seien $\{ h_k\}$, $\{ \tilde{h}_k \}$ zugehörigen $L^1$-CF in $T^1(M)$.\\
	\begin{tabularx}{\linewidth}{r@{\ \ }X}
		$\Rightarrow$ & $\forall \epsilon > 0$ $\exists k_0$ mit \[ \int_M \vert (h_k + \tilde{h}_k) - (h_l + \tilde{h}_l)\vert \D x \le \int_M \vert h_k - h_l\vert + \vert \tilde{h}_k - \tilde{h}_l \vert \D x < \epsilon \quad\forall k,l\ge k_0  \] \\
		$\Rightarrow$ & $\{ h_k - \tilde{h}_k \}$ ist $L^1$-CF mit $(h_k - \tilde{h}_k)\to 0$ \gls{fue} auf $M$.
	\end{tabularx}

	Da $\{ \int_M h_k \D x\}$, $\{ \int_M \tilde{h}_k \D x \}$ in $\mathbb{R}$ konvergieren, bleibt zu zeigen: $\{ h_k\}$ ist $L^1$-CF 	in $T^1(M)$ mit $h_k\to 0$ \gls{fue} auf $M$
	\begin{flalign}
		\Rightarrow &\int_M h_k \D x \xrightarrow{k\to\infty} 0&
	\end{flalign}
	
	Da Konvergenz von $\{ \int_M h_k \D x \}$ bereits bekannt ist, reicht es, den Grenzwert für eine \gls{tf} zu zeigen.
	
	\item Wähle \gls{tf} derart, dass $\int_M \vert h_k - h_l \vert \D x \le \frac{1}{2^l}$ $\forall k\ge l$
	
	Fixiere $l\in\mathbb{N}$ und definiere $M_l := \{ x\in M\mid h_l(x) \neq 0 \}$, offenbar ist $M$ messbar mit $\vert M_l\vert < \infty$.
	
	Sei nun $\epsilon_l := \frac{1}{2^l \cdot \vert M_l\vert}$ falls $\vert M_l\vert > 0$ und $\epsilon_l = 1$ falls $\vert M_l\vert = 0$.
	
	Weiterhin sei $M_{l,k} := \{ x\in M_l \mid \vert h_k(x)\vert > \epsilon_l \}$, und für $k > l$ folgt
	\begin{align*}
		\left\vert\int_M h_k\D x \right\vert &\le \int_M \vert h_k\vert \D x = \int_{M_l} \vert h_k\vert \D x + \int_{M\setminus M_l} \vert h_k\vert \D x \\
		&\le \int_{ M\setminus M_{l,k}} \vert h_k\vert \D x + \int_{M_{l,k}} \vert h_k\vert \D x + \int_{M\setminus M_l} \vert h_k - h_l\vert \D x + \underbrace{\int_{M\setminus M_l} \vert h_l\vert \D x}_{=0} \\
		&\le \epsilon_l \vert M_l\vert + \int_{M_{l,k}} \vert h_k - h_l\vert \D x + \int_{M_{l,k}} \vert h_l\vert \D x + \frac{1}{2^l} \\
		&\le \frac{1}{2^l} + \frac{1}{2^l} + c_l \cdot \vert M_{l,k}\vert + \frac{1}{2^l}
	\end{align*}
	mit $c_l := \sup\limits_{x\in M} \vert h_l(x)\vert$, $\exists k_l > l$ mit \propref{messbarkeit_funktion_egorov} folgt $\vert \{ x\in M_l\mid \vert h_k(x)\vert > \epsilon_l \} \vert \le \frac{1}{2^l \cdot (c_l + 1)}$ $\forall k > k_l$
	
	\begin{tabularx}{\linewidth}{r@{\ \ }X}
	$\Rightarrow$ & $\displaystyle \left\vert \int_M h_k\D x \right\vert \le \frac{4}{2^l}$ $\forall k>k_l$ \\
	$\xRightarrow[\text{beliebig}]{l\in\mathbb{N}}$ & $\displaystyle \int_M h_k \D x\to 0$
	\end{tabularx}
	\end{itemize}
	\ \hfill\csname\InTheoType Symbol\endcsname
\end{proof}

\begin{proposition}[Rechenregeln]
	\proplbl{integral_funktion_rechenregeln}
	Seien $f$, $g$ integrierbar auf $M\subset\mathbb{R}^n$, $c\in\mathbb{R}$. Dann
	\begin{enumerate}[label={\alph*)}]
		\item \proplbl{integral_funktion_rechenregeln_a}
		(Linearität) $f\pm g$, $cf$ sind integrierbar auf $M$ mit \begin{align*}
			\int_M f \pm g \D x &= \int_M f\D x + \int_M g \D x \\
			\int_M c f \D x &= c \int_M f \D x
		\end{align*}
		\item \proplbl{integral_funktion_rechenregeln_b}
		Sei $\tilde{M}\subset\mathbb{M}$ messbar\\
		\begin{tabularx}{\linewidth}{r@{\ \ }X}
			$\Rightarrow$ & $f \chi_{\tilde{M}}$ ist integrierbar auf $M$ und $f$ ist integrierbar auf $\tilde{M}$ mit \[
				\int_M f\cdot \chi_{\tilde{M}} \D x = \int_{\tilde{M}} f \D x \]
		\end{tabularx}
		\item\proplbl{integral_funktion_rechenregeln_c}
		Sei $M = M_1\cup M_2$ für $M_1$, $M_2$ disjunkt und messbar \\
		\begin{tabularx}{\linewidth}{r@{\ \ }X}
			$\Rightarrow$ & $f$ ist integrierbar auf $M_1$ und $M_2$ mit 
		\end{tabularx}
		\begin{align*}
			\int_M f \D x &= \int_{M_1}  f \D x + \int_{M_2} f \D x
		\end{align*}
		\item\proplbl{integral_funktion_rechenregeln_d}
		Sei $f = \tilde{f}$ \gls{fue} auf $M$ \\\begin{tabularx}{\linewidth}{r@{\ \ }X}
			$\Rightarrow$ & $\tilde{f}$ ist integrierbar auf $M$ mit
		\end{tabularx}
		\begin{align*}
			\int_M f \D x = \int_M \tilde{f} \D x
		\end{align*}
		\item\proplbl{integral_funktion_rechenregeln_e}
		Die Nullfortsetung $\tilde{f}:\mathbb{R}^n\to\overline{\mathbb{R}}$ von $f$ (vgl. \propref{messbarkeit_funktion_nullfortsetzung}) ist auf jeder messbaren Menge $\tilde{M}\subset\mathbb{R}^n$ integrierbar mit \begin{align*}
			\int_{M\cap \tilde{M}} f \D x &= \int_{\tilde{M}} \overline{f}\D x
		\end{align*}
	\end{enumerate}
\end{proposition}

Aussage \ref{integral_funktion_rechenregeln_d} bedeutet, dass eine Änderung der Funktionswerte von $f$ auf einer Nullmenge das Integral nicht verändert.

\begin{proof}
	Seien $\{ h_k\}$ und $\{ \tilde{h}_k \}$ aus $T^1(\mathbb{R})^n$ $L^1$-CF zu $f$ und $g$.
	
	\begin{enumerate}[label={zu \alph*)},leftmargin=\widthof{\texttt{zu a) }},topsep=\dimexpr-\baselineskip/2\relax]
		\item Es ist $h_k + \tilde{h}_k\to f + g$ \gls{fue} auf $M$.
		
		Wegen \begin{align*}
			\int_M \vert (h_k + \tilde{h}_k) - (h_l + \tilde{h}_l)\vert \D x &\le \underbrace{\int_M \vert h_k - h_l\vert \D x}_{=\text{$L^1$-CF, $<\epsilon$}} + \underbrace{\int_M \vert \tilde{h}_k - \tilde{h}_l \vert \D x}_{=\text{$L^1$-CF, $<\epsilon$}}
		\end{align*}
		ist $\{ h_k + \tilde{h}_k\}$ $L^1$-CF zu $f+g$. \\
		$\Rightarrow$ $f+g$ ist integrierbar auf $M$ und Grenzübergang in \begin{align*}
			\int_M h_k + \tilde{h}_k \D x &= \int_M h_k \D x + \int_M \tilde{h}_k \D x
		\end{align*}
		liefert die Behauptung für $f+g$.
		
		Analog zu $cf$. Wegen $f - g$ = $f + (-g)$ folgt die letzte Behauptung.
		
		\item Offenbar ist $\{ \chi_{\tilde{m} h_k} \}$ $L^1$-CF zu $\chi_{\tilde{M}}f$ und $\{ h_k \}$ $L^1$-CF zu $f$ auf $\tilde{M}$.
		
		Mit \begin{align*}
			\int_M h_k \chi_{\tilde{M}} \D x &= \int_{\tilde{M}} h_k \D x\quad\forall k\in\mathbb{N}
		\end{align*}
		folgt die Behauptung durch Grenzübergang.
		\item Nach \ref{integral_funktion_rechenregeln_b} ist $f$ auf $M_1$ und $M_2$ integrierbar. Wegen $f = \chi_{M_1} f + \chi_{M_2} f$ folgt die Behauptung aus \ref{integral_funktion_rechenregeln_a} und \ref{integral_funktion_rechenregeln_b}.
		\item Da $\{ h_k\}$ auch $L^1$-CF zu $\tilde{f}$ ist, folgt die Integrierbarkeit mit dem gleichen Integral.
		\item Es ist $\{ \chi_{M\cap \tilde{M}} h_k \}$ $L^1$-CF zu $f$ auf $M\cap \tilde{M}$ und auch zu $\overline{f}$ auf $\tilde{M}$. Damit folgt die Behauptung.
	\end{enumerate}
\end{proof}

\begin{proposition}[Eigenschaften]
	\proplbl{integral_funktion_eigenschaften}
	Es gilt \begin{enumerate}[label={\alph*)}]
		\item \proplbl{integral_funktion_eigenschaften_integrierbarkeit}
		(Integierbarkeit) Für $f:M\subset\mathbb{R}^n\to\overline{\mathbb{R}}$ messbar gilt:\begin{center}
				$f$ integrierbar auf $M$ \ \ $\Leftrightarrow$ \ \  $\vert f \vert$ integrierbar auf $M$
		\end{center}
		\item\proplbl{integral_funktion_eigenschaften_beschraenktheit}
		(Beschränktheit)
		Sei $f$ integrierbar auf $M$, dann \begin{align*}
			\left\vert \int_M f \D x \right\vert &\le \int_M \vert f \vert \D x
		\end{align*}
		\item\proplbl{integral_funktion_eigenschaften_monotonie}
		(Monotonie)
		Seien $f$, $g$ integrierbar auf $M$. Dann \begin{center}
			$f\le g$ \gls{fue} auf $M$ \ \ $\Rightarrow$ \ \ $\displaystyle\int_M f\D x \le \int_M g \D x$
		\end{center}
		\item\proplbl{integral_funktion_eigenschaften_nullfunktion}
		 Sei $f$ integrierbar auf $M$, dann \begin{center}
				$\displaystyle \int_M \vert f \vert \D x = 0$\ \ $\Leftrightarrow$ \ \ $f = 0$ \gls{fue}
		\end{center}
	\end{enumerate}
	In Analogie zur Treppenfunktion ist $\Vert f\Vert _1 := \int_M \vert f \vert \D x$ auf $L^1(M)$ eine Halbnorm, aber keine Norm ($\Vert f \Vert = 0$ $\cancel{\Leftrightarrow}$ $f = 0$). $\Vert f\Vert_1$ heißt \begriff{$L^1$-Halbnorm} von $f$.
\end{proposition}

\begin{underlinedenvironment}[Hinweis]
	Eine lineare Abbildung $A:X\to Y$ ist beschränkt, wenn $\Vert Ax\Vert_Y \le c\Vert x \Vert _X$ \\
	$\Rightarrow$ Begriff der Beschränktheit in \ref{integral_funktion_eigenschaften_beschraenktheit}.
\end{underlinedenvironment}
\begin{proof}\hspace*{0pt}
	\NoEndMark
	\begin{enumerate}[label={zu \alph*)},topsep=\dimexpr -\baselineskip / 2\relax,leftmargin=\widthof{\texttt{zu b)\ }}]
		\item Sei $f$ integrierbar auf $M$ und sei $\{ h_k \}$ $L^1$-CF zu $f$ \\
		\ $\Rightarrow$ $\vert h_k \vert\to \vert f \vert$ \gls{fue} auf $M$.
		
		Wegen $\int_M \left\vert \vert h_k \vert - \vert h_l \vert\right\vert \D x$\marginnote{$\vert\vert \alpha\vert - \vert\beta\vert\vert \le \vert \alpha - \beta\vert$ $\forall \alpha,\beta\in\mathbb{R}$} $\overset{\cref{integral_treppenfunktion_grundlegende_folgerung}}{\le}$ $\int_M \vert h_k - h_l \vert \D x$ ist $\{ \vert h_k\vert \}$ $L^1$-CF zu $\vert f \vert$ \\
		\ $\Rightarrow$ $\vert f \vert$ ist integrierbar.
		
		\emph{beachte:} andere Richtung später
		
		\item Für eine $L^1$-CF $\{ h_k\}$ zu $f$ gilt nach \propref{integral_treppenfunktion_grundlegende_folgerung} c): \begin{align*}
			\left\vert \int_M h_k \D x \right\vert \le \int_M \vert h_k\vert \D x
		\end{align*}
		Da $\{ \vert h_k \vert \}$ $L^1$-CF zu $\vert f \vert$ ist, folgt die Behauptung durch Grenzübergang.
		
		\item Nach den Rechenregeln ist $g - f$ integrierbar, wegen $\vert g - f\vert = g - f$ \gls{fue} auf $M$ folgt \begin{align*}
			0 \le \left\vert \int_M g - f\D x\right\vert\overset{\ref{integral_funktion_eigenschaften_beschraenktheit}}{\le} \int_M \vert g - f\vert \D x \overset{ \cref{integral_funktion_rechenregeln}\;\ref{integral_funktion_rechenregeln_a}}{=} \int_M g \D x - \int_M f \D x
		\end{align*}
		$\Rightarrow$ Behauptung
		
		\item[zu a)] für "`$\Leftarrow$"' wähle $f^\pm$ ($ f = f^+ - f^-$) jeweils eine monotone Folge von \gls{tf} $\{ h_k^\pm \}$ gemäß \propref{messbarkeit_funktion_existenz_monotone_treppenfunktionen}. Folglich liefert $H_k = h_k^+ - h_k^-$ eine Folge von \gls{tf} mit $h_k\to f$ \gls{fue} auf $M$.
		
		Wegen $\vert h_k \vert \le \vert f \vert$ \gls{fue} auf $M$ ist $\int_M \vert h_k\vert \D x \le \int_M \vert f \vert \D x$.
		
		Folglich ist die monotone Folge $\int_M \vert h_k\vert \D x$ in $\mathbb{R}$ beschränkt \\
		$\Rightarrow$ konvergent.
		
		Da $h_k^\pm$ jeweils das Vorzeichen wie $f^\pm$ haben und die Folge monoton ist, gilt \begin{align*}
			\left\vert \vert h_l\vert - \vert h_k\vert \right\vert &= \vert h_l\vert - \vert h_k\vert = \vert h_l -  h_k \vert \quad\forall l>k
		\end{align*}
		und somit auch \begin{align*}
			\int_M \vert h_l - h_k \vert \D x &= \int_M \vert h_l\vert - \vert h_k\vert \D x = \left \vert \int_M \vert h_l\vert \D x - \int_M \vert h_k\vert \D x\right\vert \quad\forall l>k
		\end{align*}
		Als konvergente Folge ist $\{ \int_M \vert h_k \vert \D x \}$ \person{Cauchy}-Folge in $\mathbb{R}$ und folglich ist $\{ h_k \}$ $L^1$-CF und sogar $L^1$-CF zu $f$ \\
		$\Rightarrow$ $f$ integrierbar
		\item Für $f=0$ \gls{fue} auf $M$ ist offenbar $\int_M \vert f\vert \D x = 0$.
		
		Sei nun $\int_M \vert f \vert \D x = 0$, mit $M_k := \{ x\in M \mid \vert f \vert \ge \frac{1}{k} \}$ $\forall k\in\mathbb{N}$ ist \begin{align*}
			0 = \int_{M\setminus M_k} \vert f \vert \D x + \int_{M_k} \vert f \vert \D x \ge \int_{M\setminus M_k}0 \D x + \int_{M_k}\frac{1}{k}\D x \ge \frac{1}{k}\vert M_k\vert \ge 0
		\end{align*}
		\begin{tabularx}{\linewidth}{r@{\ \ }X}
			$\Rightarrow$ & $\vert M_k\vert = 0$ $\forall k$, wegen $\{ f \neq 0\} = \bigcup_{k\in\mathbb{N}} M_k$ \\
			$\Rightarrow$ & $\displaystyle \vert \{ f\neq 0 \} \vert \le \sum_{k=1}^\infty \vert M_k\vert= 0$ \\
			$\Rightarrow$ & Behauptung\hfill\csname\InTheoType Symbol\endcsname
		\end{tabularx}
	\end{enumerate}
\end{proof}

\begin{conclusion}
	\proplbl{integral_funktion_lemma_weitere_eigenschaften}
	Sei $f$ auf $M$ integrierbar\begin{enumerate}[label={\alph*)}]
		\item Für $\alpha_1$, $\alpha_2\in\mathbb{R}$ gilt:\begin{center}
			$\alpha_1\le f \le \alpha_2$ \gls{fue} auf $M$ \ \ $\Rightarrow$ \ \ $\displaystyle \alpha_1 \vert M \vert \le \int_M f \D x \le \alpha_2 \vert M \vert$
		\end{center}
		\item Es gilt $f\ge 0$ \gls{fue} auf $M$ \ \ $\Rightarrow$ \ \ $\int_M f \D x\ge 0$
		\item Es gilt: $\tilde{M}\subset M$ messbar, $f\ge 0$ \gls{fue} auf $M$ \\
		\ $\Rightarrow$ \ \ $\displaystyle \int_{\tilde{M}} f \D x \le \int_M f \D x$
		
		(linkes Integral nach \propref{integral_funktion_rechenregeln} \ref{integral_funktion_rechenregeln_b})
	\end{enumerate}
\end{conclusion}

\begin{proof}\hspace*{0pt}
	\NoEndMark
	\begin{enumerate}[label={zu \alph*)},topsep=\dimexpr-\baselineskip/2\relax,leftmargin=\widthof{\texttt{zu a)\ }}]
		\item 
		Wegen $\int_M \alpha_j \D x = \alpha_j \vert M \vert $ für $\vert M \vert$ endlich folgt a) direkt aus der Monotonie des Integrals.
		\item folgt mit $\alpha_1=0$ aus a)
		\item folgt, da $\chi_{\tilde{M}}\cdot f \le f$ \gls{fue} auf $M$ und aus der Monotonie\hfill\csname\InTheoType Symbol\endcsname
	\end{enumerate}
\end{proof}

In der Vorüberlegung zum Integral wurde eine gewisse Stetigkeit der Integralabbildung angestrebt. Das Integral ist bezüglich der $L^1$-Halbnorm stetig.
\begin{proposition}
	\proplbl{integral_funktionen_differenz_null_gleichheit}
	Seien $f$, $f_k:D\subset\mathbb{R}^n\to\overline{\mathbb{R}}$ integrierbar auf $M\subset\mathbb{R}^n$ und sei \begin{align*}
		& \lim\limits_{k\to\infty} \int_M \vert f_k - f\vert \D x = 0 \quad(\Vert f_k - f\Vert\to0)\\
		\Rightarrow\;\;&\lim\limits_{k\to\infty} \int_M f_k \D x = \int_M f\D x
	\end{align*}
	Weiterhin gibt es eine Teilfolge $\{ f_{k'}\}$ mit $f_{k'}\to f$ \gls{fue} auf $M$.
\end{proposition}

\begin{proof}
	\NoEndMark
	Aus der Beschränktheit nach \propref{integral_funktion_eigenschaften} folgt \begin{align*}
		\left\vert\int_M f_k\D x - \int_M f \D x \right\vert \le \int_M \vert f_k - f\vert \D x \xrightarrow{k\to 0} 0
	\end{align*}
	\ $\Rightarrow$\ \ 1. Konvergenzaussage
	
	Wähle nun eine \gls{tf} $\{ f_{k_l}\}_l$ mit $\int_M \vert f_{k_l} - f\vert \D x \le \frac{1}{2^{l+1}}$ $\forall l\in\mathbb{N}$.
	
	Für $\epsilon>0$ sei $M_\epsilon := \{ x\in M \mid \limsup\limits_{l\to\infty} \vert f_{k_l} - f\vert > \epsilon \}$ \\
	\begin{tabularx}{\linewidth}{r@{\ \ }X}
	$\Rightarrow$ & $\displaystyle M_\epsilon \subset\bigcup_{l=j}^\infty \{ \vert f_{k_l} - f \vert > \epsilon \}$ $\forall j\in\mathbb{N}$ \\
	$\Rightarrow$ & $\displaystyle M_\epsilon \le \sum_{l=j}^\infty \left\vert \left\{ f_{k_l} - f\vert > \epsilon \right\} \right\vert \le \frac{1}{\epsilon} \sum_{l=j}^\infty \int _M \vert f_{k_l} - f\vert \D x \le \frac{1}{\epsilon} \sum_{l=j}^\infty \frac{1}{2^{l+1}} = \frac{1}{2^j \epsilon}\quad\forall j\in\mathbb{N}$\\
	$\Rightarrow$ & $M_\epsilon = 0$ $\forall\epsilon > 0$ \\
	$\Rightarrow$ & $f_{k_l} \xrightarrow{l\to\infty} f$ \gls{fue} auf $M$ \hfill\csname\InTheoType Symbol\endcsname
	\end{tabularx}
\end{proof}

\begin{proposition}[Majorantenkriterium]
	\proplbl{integral_funktion_majorantenkriterium}
	Seien $f$, $g:D\subset\mathbb{R}^n\to\overline{\mathbb{R}}$ messbar, $M$ messbar, $\vert f \vert \le g$ \gls{fue} auf $M$, $g$ integrierbar auf $M$ \\
	$\;\Rightarrow$ $f$ integrierbar auf $M$
	
	Man nennt $g$ auch \begriff{integrierbare Majorante} von $f$.
\end{proposition}

\begin{lemma}
	\proplbl{integral_funktion_lemma_majorante}
	Sei $f:D\subset\mathbb{R}^n\to\overline{\mathbb{R}}$ messbar auf $M$, sei $f\ge 0$ auf $M$ und sei $\{ h_k\}$ Folge von Treppenfunktionen mit \begin{align}
		\proplbl{integral_funktion_lemma_majorante_eq}
		0 \le h_1 \le h_2 \le \dotsc \le f\quad \text{ und }\quad  \int_M h_k \D x\text{ beschränkt}
	\end{align}
	$\Rightarrow$ $\{ h_k\}$ ist $L^1$-CF zu $f$ und falls $\{ h_k\}\to f$ \gls{fue} auf $M$ ist $f$ integrierbar (vgl \propref{messbarkeit_funktion_existenz_monotone_treppenfunktionen})
\end{lemma}

\begin{proof}
	Offenbar sind alle $h_k$ integrierbar und wegen der Monotonie gilt \begin{align*}
		\left\vert \int_M h_k\D x - \int_M h_l\D x \right\vert &=\int_M \vert h_k - h_l\vert \D x\quad\forall k\ge l
	\end{align*}
	Da $\{ \int_M h_k \D x \}$ konvergent ist in $\mathbb{R}$ als monoton beschränkte Folge ist diese CF in $\mathbb{R}$ \\
	$\Rightarrow$ $\{ h_k\}$ ist $L^1$-CF
	
	Falls noch $h_k\to f$ \gls{fue} $\Rightarrow$ $\{h_k\}$ ist $L^1$-CF zu $f$ $\Rightarrow$ $f$ ist integrierbar
\end{proof}

\begin{proof}[\propref{integral_funktion_majorantenkriterium}]
	\NoEndMark
	(mit $f$ auch $\vert f \vert$ mesbbar nach \propref{messbarkeit_funktion_existenz_monotone_treppenfunktionen})
	
	Es existiert eine Folge $\{ h_k \}$ von Treppenfunktionen mit \begin{align*}
		0 \le h_1 \le h_2 \le \dotsc \le \vert f \vert \le g
	\end{align*}
	auf $M$ und $\{ h_k\}\to\vert f \vert$ \gls{fue} auf $M$.
	
	Da $\{ \int_M h_k \D x \}$ beschränkt ist in $\mathbb{R}$ da $g$ integrierbar ist \\
	{\renewcommand{\arraystretch}{1.3}\begin{tabularx}{\linewidth}{r@{\ \ }X}
		$\xRightarrow{\text{\propref{integral_funktion_lemma_majorante}}}$ & $\{ h_k\}$ ist $L^1$-Cf zu $\vert f \vert$ \\
		$\Rightarrow$ & $\vert f \vert$ integrierbar \\
		$\xRightarrow{\text{\propref{integral_funktion_eigenschaften}}}$ & $f$ integrierbar auf $M$\hfill\csname\InTheoType Symbol\endcsname
	\end{tabularx}}
\end{proof}

\begin{conclusion}
	Seien $f$, $g:M\subset\mathbb{R}^n\to\overline{\mathbb{R}}$ messbar, $\vert M \vert$ endlich. Dann\begin{enumerate}[label={\alph*)}]
		\item Falls $f$ beschränkt ist auf $M$, dann ist $f$ integrierbar auf $M$
		\item Sei $f$ beschränkt und $g$ integrierbar auf $M$\\
		$\Rightarrow$\ \ $f\cdot g$ ist integrierbar auf $M$
	\end{enumerate}
\end{conclusion}
\begin{underlinedenvironment}[Hinweis]
	Folglich sind stetige Funktionen auf kompaktem $M$ integrierbar (vgl. Theorem von Weierstraß)
\end{underlinedenvironment}

\begin{proof}
	Sei $\vert f \vert \le \alpha$ auf $M$ für $\alpha\in\mathbb{Q}$
	\begin{enumerate}[label={zu \alph*)},topsep=\dimexpr-\baselineskip/2\relax,leftmargin=\widthof{\texttt{zu a)\ }}]
	\item 
	$\Rightarrow$ \ \ konstante Funktion $f_1 = \alpha$ ist integrierbare Majorante von $\vert f \vert$
	\item Mit $f_2 = \alpha\cdot \vert g \vert$ ist $f_2$ integrierbare Majorante zu $\vert f\cdot g\vert$ \ \ 
	$\xRightarrow[\text{kriterium}]{\text{Majoranten-}}$ Behauptung
\end{enumerate}
\end{proof}

\subsection{Grenzwertsätze}
$\int_M f_k\D x \xrightarrow{?} \int_M f\D x$ Vertauschbarkeit von Integration und Grenzübergang ist zentrale Frage $\to$ grundlegende Grenzwertsätze $\int_M \vert f_k - f\vert \D x \to 0$
\begin{theorem}[Lemma von Fatou]
	\proplbl{integral_grenzwertsatz_fatou}
	Seien $f_k:D\subset\mathbb{R}^n\to [0,\infty]$ integrierbar auf $M\subset D$ $\forall k\in\mathbb{N}$ \\
	\ $\Rightarrow$ $f(x) := \liminf\limits_{k\to\infty} f_k(x)$ $\forall x\in M$ ist integrierbar auf $M$ und \begin{align*}
		\left( \int_M f\D x =\right) \int_M \liminf_{k\to\infty} f_k \D x &\le \liminf\limits_{k\to\infty} \int_M f_k \D x,
	\end{align*}
	falls der Grenzwert rechts existiert.
\end{theorem}

Keine Gleichheit hat man z.B. für $\{ h_k\}$ aus \propref{integral_funktion_beispiel_striktere_konvergenz} mit $\alpha_k = 1$ $\forall k$ \begin{align*}
	h_k &= \begin{cases}
		h\cdot \alpha_k & x\in \left[0,\frac{1}{k}\right] \\
		0 & \text{sonst}
	\end{cases}
	\intertext{Dann}
	\int_M \liminf\limits_{k\to\infty} h_k \D x &= \int_M 0 \D x = 0 < \liminf\limits_{k\to\infty} \int_{\mathbb{R}} h_k \D x = 1
\end{align*}

\begin{proof}
	Auf $M$ ist $0\le g_k := \inf\limits_{l\ge k} f_l \le f_j$ $\forall j\ge k$, $k\in\mathbb{N}$, $g_1 \le g_2 \le \dotsc$ und $\lim\limits_{k\to\infty} g_k = \liminf\limits_{k\to\infty} f_k = f$
	
	Alle $g_k$ sind messbar nach \propref{messbarkeit_funktionen_komposition}, \propref{integral_funktion_majorantenkriterium}
	
	Für jedes $k\in\mathbb{N}$ wählen wir gemäß \propref{messbarkeit_funktion_existenz_monotone_treppenfunktionen} eine Folge $\{ h_{k_l} \}_l$ von Treppenfunktionen mit $0\le h_{k_1} \le h_{k_2} \le \dotsc \le g_k$, $h_{k_l}\xrightarrow{l\to\infty} g_k$ \gls{fue} auf $M$.
	
	Nach \propref{integral_funktion_lemma_majorante} ist $\{ h_{k_l}\}_l$ $L^1$-CF zu $g_k$.
	
	Anwendung von \propref{messbarkeit_funktion_egorov} auf $g_k - f$ auf $B_k(0)\cap M$ \\
	$\Rightarrow$ $\exists A_k' \subset\mathbb{R}^n$ messbar mit $\vert A_k'\vert \le \frac{1}{2^{k+1}}$ und (ggf. \gls{tf}) $\vert g_k - f\vert < \frac{1}{k}$ auf $(B_k(0) \cap M)\setminus A_K'$.
	
	Analog für Folge $h_{k_l}\xrightarrow{l\to\infty} g_k: \exists A_K''\subset\mathbb{R}^k$ mit $\vert A_k''\vert < \frac{1}{2^{k+1}}$ und (evtl. \gls{tf}) $\vert h_{k_l} - g_k \vert < \frac{1}{k}$ auf $(B_k(0)\cap M)\setminus A_k''$
	
	Setzte $A_k = A_k'\cup A_k''$, offenbar $\vert A_k\vert < \frac{1}{2^k}$, $h_k := h_{k_l}$
	
	Definiere rekursiv $\tilde{h}_1 := h_1$, $\tilde{h}_k := \max( \tilde{h}_{k-1}, h_k)$ \\
	$\Rightarrow$ $h_k \le \tilde{h}_k \le g_k \le f_k$ und $\tilde{h}_{k-1} \le \tilde{h}_k$ $\forall k\in\mathbb{N}$ \\
	$\Rightarrow$ $\vert \tilde{h}_k - f\vert \overset{\triangle-\text{Ungl}}{\le}\vert \tilde{h}_k - g_k\vert + \vert g_k -f \vert \le \vert h_k - g_k\vert + \vert g_k -f \vert \le \frac{2}{k}$ auf $(B_k(0)\cap M)\setminus A_k$.
	
	Mit $\tilde{A}_l := \bigcup_{k=l}^\infty A_k$ folgt $\vert \tilde{A}_l\vert \le \frac{1}{2^{l-1}}$ und $\vert \tilde{h}_k - f \vert \le \frac{2}{k}$ auf $(B_k(0)\cap M)\setminus \tilde{A}_l$ $\forall k>l$.
	
	Folglich $\tilde{h}_l\to f$ \gls{fue} auf $M$ und wegen der Monotonie ist $\{ \tilde{h}_k\}$ $L^1$-CF zu $f$ \\
	$\Rightarrow$ $\int_M f \D x \overset{\text{Def}}{=} \lim\limits_{k\to\infty} \int_M \tilde{h}_k \D x \overset{\text{Monotonie}}{\le}\liminf\limits_{k\to\infty} \int_M f_k \D x$\\
	$\Rightarrow$ Behauptung
\end{proof}

\begin{theorem}[Monotone Konvergenz]
	\proplbl{integral_grenzwertsatz_monotone_konvergenz}
	Seien $f_k:D\subset\mathbb{R}^n\to\overline{\mathbb{R}}$ integrierbar auf $M\subset D$ $\forall k\in\mathbb{N}$ mit $f_1 \le f_2 \le \dotsc $ \gls{fue} auf $M$ \\
	\ $\Rightarrow$ $f$ ist integrierbar auf $M$ und \begin{align*}
		\left( \int_M f \D x = \right) \int_M \lim\limits_{k\to\infty} f_k(x) \D x &= \lim\limits_{k\to\infty} \int_M f_k \D x
	\end{align*}
	falls der rechte Grenzwert existiert.
\end{theorem}

\begin{remark}
	\propref{integral_grenzwertsatz_monotone_konvergenz} bleibt richtig, falls man $f_1 \ge f_2 \ge \dotsc$ \gls{fue} auf $M$ hat.
	
	Ferner ist wegen der Monotonie die Beschränktheit der Folge $\{ \int_M f_k \D x \}$ für die Existenz des Grenzwertes ausreichend.
\end{remark}

\begin{proof}[\propref{integral_grenzwertsatz_monotone_konvergenz}]
	Nach \propref{integral_grenzwertsatz_fatou} ist $f - f_1 = \lim\limits_{k\to\infty} f_k - f_1$ integrierbar auf $M$ und damit auch $f = (f - f_1) + f_1$
	\begin{align*}
	\Rightarrow\;\;\int_M f - f_1 \D x &\le \lim\limits_{k\to\infty} \int_M f_k - f_1 \D x\\
	&= \lim\limits_{k\to\infty} \int_M f_k \D x - \int_M f_1 \D x
	\overset{\text{Monotonie}}{\le} \int_M f \D x - \int_M f_1\D x\\
	&= \int_M f - f_1 \D x
	\end{align*}
\end{proof}

\begin{theorem}[Majorisierte Konvergenz]
	\proplbl{integral_grenzwertsatz_majorisierte_konvergenz}
	Seien $f_k$, $g:D\subset\mathbb{R}^n\to\overline{\mathbb{R}}$ messbar für $k\in\mathbb{N}$ und sei $g$ integrierbar auf $M\subset D$ mit $\vert f_k\vert \le g$ \gls{fue} auf $M$ $\forall k\in\mathbb{N}$ und $f_k\to:f$ \gls{fue} auf $M$
	\begin{align}
		\proplbl{integral_grenzwertsatz_majorisierte_konvergenz_eq}
		\Rightarrow\;\;\lim\limits_{k\to\infty} \int_M \vert f_k - f\vert \D x = 0
	\end{align}
	und \begin{align*}
		\left(\int_M f\D x = \right) \int_M \lim\limits_{k\to\infty} f_k \D x = \lim\limits_{k\to\infty} \int_M f_k \D x,
	\end{align*}
	wobei alle Integrale existieren.
\end{theorem}

\begin{proof}
	Nach dem Majorantenkriterium sind alle $f_k$ \gls{fue} integrierbar auf $M$.
	
	Nach \propref{integral_grenzwertsatz_fatou} gilt:\begin{align*}
		\int_M 2g \D x = \int_M \liminf\limits_{k\to\infty} \vert 2g - \vert f_k - f\vert \vert \D x \le \liminf\limits_{k\to\infty} \int_M 2g - \vert f_k - f \vert \D x
	\end{align*}
	$\Rightarrow$ $0 = \liminf\limits_{k\to\infty} -\int_M \vert f_k - f\vert \D x$ $\Rightarrow$ \eqref{integral_grenzwertsatz_majorisierte_konvergenz_eq} $\xRightarrow{\text{\propref{integral_funktionen_differenz_null_gleichheit}}}$ Behauptung
\end{proof}

\begin{conclusion}
	\proplbl{integral_grenzwertsatz_folgerung_fatou}
	Seien $f_k:D\subset\mathbb{R}^n\to\overline{\mathbb{R}}$ integrierbar auf $M$ $\forall k\in\mathbb{N}$. Sei $\vert M \vert < \infty$ und konvergieren die $f_k \to: f$ gleichmäßig auf $M$ \\
	\ $\Rightarrow$ $f$ ist integrierbar auf $M$ und $\int_M f \D x = \lim\limits_{k\to\infty} \int_M f_k \D x$
\end{conclusion}

\begin{proof}
	$\exists k_0\in \mathbb{N}$ mit $\vert f_k(x) \vert \le \vert f_{k_0}(x) + 1\vert$ $\forall x\in\mathbb{M}$, $k > k_0$.
	
	Da $f_{k_0}+1$ integrierbar auf $M$ folgt die Behauptung aus \propref{integral_grenzwertsatz_majorisierte_konvergenz}.
\end{proof}

\begin{theorem}[Mittelwertsatz der Integralrechnung]
	\proplbl{integral_grenzwertsatz_mittelwertsatz_integralrechnung}
	Sei $M\subset\mathbb{R}^n$ kompaket und zusammenhängend, und sei $f:M\to\mathbb{R}$ stetig
	\begin{align*}
	\Rightarrow\;\;\exists \xi\in M: \int_M f \D x = f(\xi) \cdot \vert M \vert
	\end{align*}
\end{theorem}

\begin{proof}
	Aussage klar für $\vert M \vert = 0$, deshalb wähle $\vert M \vert > 0$.
	
	Da $f$ stetig auf $M$ kompakt 
	
	{\renewcommand{\arraystretch}{1.3}\begin{tabularx}{\linewidth}{r@{\ \ }X}
	$\xRightarrow[\propref{satz_von_weierstrass}]{\text{Weierstrass}}$ & $\exists$ Minimalstelle $x_1\in M$, Maximalstelle $x_2\in M$ und $\displaystyle\gamma := \int_M f \D x$ \\ $\xRightarrow{\text{\cref{integral_funktion_lemma_weitere_eigenschaften}}}$ & $f(x_1) \le \frac{\gamma}{\vert M \vert} \le f(x_2)$ \\
	$\xRightarrow[\propref{zwischenwertsatz}]{\text{Zwischenwertsatz}}$ & $\displaystyle\exists \xi\in M: f(\xi) = \frac{\gamma}{\vert M \vert}$ \\
	$\Rightarrow$ & Behauptung
	\end{tabularx}}
\end{proof}

\subsection{Parameterabhängige Integrale}
Sei $M\subset\mathbb{R}^n$ messbar, $P\subset\mathbb{R}^n$ eine Menge von Parametern und sei $f:M\times P\to\mathbb{R}$.

Betrachte parameterabhängige Funktion \begin{align}
\proplbl{integral_parameterabhaengig_grundgleichung_eq}
	F(p) &:= \int_M f(x,p) \D x
\end{align}

\begin{proposition}[Stetigkeit]
	Seien $M\subset\mathbb{R}^n$ messbar, $P\subset\mathbb{R}^n$ und $f:M\times P\to\mathbb{R}$ eine Funktion mit \begin{itemize}
		\item $f(\,\cdot\,,p)$ messbar $\forall p\in P$
		\item $f(x,\,\cdot\,)$ stetig für \gls{fa} $x\in M$
	\end{itemize}
Weiterhin gebe es integrierbare Funktion $g:M\to\mathbb{R}$ mit \begin{itemize}
		\item $\vert f(x,p)\vert \le g(x)$ für \gls{fa} $x\in M$
	\end{itemize}

$\Rightarrow$ Integrale in \eqref{integral_parameterabhaengig_grundgleichung_eq} existieren $\forall p\in P$ und $F$ ist stetig auf $P$.
\end{proposition}

\begin{proof}
	$f(\,\cdot\, ,p)$ ist integrierbar auf $M$ $\forall p\in P$ nach \cref{integral_funktion_majorantenkriterium}.
	
	Fixiere $p$ und $\{ p_k\}$ in $P$ mit $p_k\to p$.
	
	Setzte $f_k(x) := f(x, p_k)$
	
	Stetigkeit von $f(x,\,\cdot\,)$ liefert $f_k(x) = f(x, p_k)\xrightarrow{x\to\infty} f(x,p)$ für \gls{fa} $x\in M$.
	 \begin{tabularx}{\linewidth}{r@{\ \ }X}
	$\xRightarrow{\text{\cref{integral_grenzwertsatz_majorisierte_konvergenz}}}$ & $F(p_k) = \int_M f_k(x) \D x \to \int_M f(x,p)\D x = F(p)$ \\
	$\xRightarrow[\text{beliebig}]{p\in P}$ & Behauptung
	\end{tabularx}
\end{proof}

\begin{proposition}[Differenzierbarkeit]
	Seien $M\subset\mathbb{R}^n$ messbar, $P\subset\mathbb{R}^m$ offen und $f:M\times P\to\mathbb{R}$ mit $f(\,\cdot\, ,p)$ integrierbar auf $M$ $\forall p\in P$. und \begin{itemize}
		\item $f(x,\,\cdot\,)$ stetig \gls{diffbar} auf $P$ für \gls{fa} $x\in M$
	\end{itemize}
	Weiterhin gebe es eine integrierbare Funktion $g:M\to\mathbb{R}$ mit \begin{itemize}
		\item $\vert f_P(x,p)\vert \le g(x)$ für \gls{fa} $x\in M$ und $\forall p\in P$
	\end{itemize}

	$\Rightarrow$ $F$ aus \eqref{integral_parameterabhaengig_grundgleichung_eq} ist \gls{diffbar} auf $P$ mit \begin{align}
	\proplbl{integral_parameterabhaengig_differenzierbarkeit_eq}
		F'(p) &= \int_M f_p(x,p) \D x
	\end{align}
\end{proposition}

\begin{underlinedenvironment}[Hinweis]
	Das Integral in \eqref{integral_parameterabhaengig_differenzierbarkeit_eq} ist komponentenweise zu verstehen und liefert für jedes $p\in P$ einen Wert im $\mathbb{R}^m$.
	
	Betrachtet man für $p=(p_1, \dotsc, p_m)\in\mathbb{R}^n$ nur $p_j$ als Parameter und fixiert andere $p_i$, dann liefert \eqref{integral_parameterabhaengig_differenzierbarkeit_eq} die partielle ABleitung $F_{p_j} (p) = \int_m f_{p_j}(x,p) \D x$ für $j=1,\dotsc, m$.
\end{underlinedenvironment}

\begin{proof}
	Königsberger: Analysis 2 (Abschnitt 8.4)
\end{proof}

\subsection{\person{Riemann}-Integral}
Der klassische Integralbegriff hat konzeptionelle Bedeutung (Einführung etwas einfacher, keine messbaren Mengen und Funktionen) \\
$\Rightarrow$ weniger Leistungsfähig (Anwendung nur in speziellen Situationen)

\begin{boldenvironment}[ebenfalls]
	Approximation von der zu integrierenden Funktion $f$ durch geeignete Treppenfunktionen
	
	Sei $f:Q\subset\mathbb{R}^n\to\mathbb{R}$ mit $Q\in\mathcal{Q}$ eine beschränkte Funktion. Betrachte die Menge der Treppenfunktionen $T_{\mathcal{Q}}(Q)$, der Form \begin{alignat*}{3}
		h &= \sum_{j=1}^l c_j \chi_{Q_j} & &\quad\text{mit}\quad & \bigcup_{j=1}^l Q_j&= Q,
	\end{alignat*}
	$Q_j\in\mathcal{Q}$ paarweise disjunkt, $c_j\in \mathbb{R}$.
	
	Quader $\{ Q_j\}_{j=1,\dotsc,l}$ werden als Zerlegung zugehörig zu $h$ bezeichnet.
\end{boldenvironment}

\begin{*definition}[Feinheit, \person{Riemann}-Summe, \person{Riemann}-Folge]
	Für Quader $Q' = F_1'\times \dotsc\times F_n'\in\mathcal{Q}$ mit Intervallen $F_j\subset\mathbb{R}$ heißt $\sigma_{Q'} := \max\limits_{j} \vert I_j'\vert$ ($\vert I_j'\vert$ - Intervalllänge) \begriff{Feinheit} von $Q'$ (setzte $\sigma_\emptyset = 0$).
	
	Für $h=\sum_{j=1}^l c_j \chi_{Q_j}$ heißt $\sigma_h := \max \sigma_{Q_j}$ Feinheit zur \begriff{Treppenfunktion} $h$.
	
	Treppenfunktion $h=\sum_{j=1}^l c_j \chi_{Q_j}\in T_{\mathcal{Q}}(Q)$ heißt \begriff{zulässig} (\person{Riemann}-zulässsig) für $f$ falls $\forall j$ $\exists x_j\in Q_j:$ $c_j = f(x_j)$, d.h. auf jedem Quader $Q_j$ stimmt $h$ mit $f$ in (mindestens) einem Punkt $x_j$ überein.
	
	Zu zulässigen $h$ nennen wir $S(h) := \sum_{j=1}^l c_j \vert Q_j\vert = \sum_{j=1}^l f(x_j) \cdot \vert Q_j\vert$ \begriff{\person{Riemann}-Summe} zu $h$.
	
	Folge $\{ h_k\}$ zulässiger Treppenfunktionen zu $f$, deren Feinheit gegen Null geht (d.h. $\sigma_{h_k}\to 0$) heißt \begriff{\person{Riemann}-Folge} zu $f$.
	
	$f$ heißt \person{Riemann}-integrierbar (kurz R-integrierbar) auf $Q$, falls $S\in \mathbb{R}$ existiert mit \begin{align}
	S = \lim\limits_{k\to\infty} S(h_k)\end{align} für \emph{alle} \person{Riemann}-Folgen $\{ h_k \}$ zu $f$.
	
	Grenzwert $\int_Q f(x) \D x := S$ heißt \begriff{\person{Riemann}-Integral} (kurz R-Integral) von $f$ auf $Q$.
\end{*definition}

\begin{proposition}
	\proplbl{integral_riemann_stetig_r_integrierbar}
	Sei $f:Q\subset\mathbb{R}^n\to\mathbb{R}$ stetig und $Q\in\mathcal{Q}$ abgeschlossen \\
	$\Rightarrow$ $f$ ist (\lebesque) integrierbar und \person{Riemann}-Integrierbar auf $Q$ mit R-$\int_Q f \D x = \int_Q f \D x$.
\end{proposition}

\begin{remark}
	Sei $f:Q\subset\mathbb{R}^n\to\mathbb{R}$ beschränkt und es sei $N:=\{ x\in Q \mid f$ nicht stetig in $x \}$.
	
	Dann kann man zeigen: $f$ ist \person{Riemann}-Integrierbar, wenn $n$ Nullmenge ist.
	
	\begin{center}
	\begin{tabular}{r@{\ \ }c@{\ \ }l}
		$f$ ist $R$-integrierbar & $\Leftrightarrow$ $N$ ist Nullmenge.
	\end{tabular}
	\end{center}

	Man sieht leicht: die \person{Dirichlet}-Funktion (\propref{messbarkeit_einfuehrung_dirichlet_funktion}) ist auf $[0,1]$ nicht R-integrierbar, da die Treppenfunktionen $h_0 = 0$ und $h_1 = 1$ auf $[0,1]$ mit belieb feiner Zerlegung $\{Q_j\}$ jeweils stets zulässig sind, sich jedoch in der \person{Riemann}-Summe $0$ bzw. $1$ unterscheiden. (Die \person{Dirichlet}-Funktion ist jedoch L-integrierbar)
\end{remark}

\begin{proof}[\propref{integral_riemann_stetig_r_integrierbar}]
	Als stetige Funktion ist $f$ auf $Q$ messbar und beschränkt und somit L-integrierbar.
	
	Fixiere $\epsilon > 0$ und sei $h=\sum_{j=1}^{l_k} f(x_{k_j}) \chi_{Q_j}$ \person{Riemann}-Folge von Treppenfunktionen zu $f$.
	
	Für $\vert Q \vert = 0$ folgt die Behauptung leicht, da $S(h_k) = 0$ $\forall k\in\mathbb{N}$
	
	Sei nun $\vert Q \vert > 0$. Da $f$ auf kompakter Menge $Q$ gleichmäßig stetig ist, existiert $\delta > 0$ mit $\vert f(x) - f(\tilde{x})\vert < \frac{\epsilon}{\vert Q \vert}$ falls $\vert x - \tilde{x}\vert < \delta$.
	
	Da $\sigma_{h_k}\to 0$ $\exists k_0\in\mathbb{N}:$ $\sigma_{h_k} < \frac{\delta}{\sqrt{n}}$ $\forall k\ge k_0$ \\
	\begin{tabularx}{\linewidth}{r@{\ \ }X}
	$\Rightarrow$ & $\vert x - \tilde{x}\vert < \delta$ $\forall x,\tilde{x}\in Q_{k_j}$ falls $k\ge k_0$ und $\vert f(x) - f(x_{j})\vert < \frac{\epsilon}{\vert Q \vert}$ $\forall x\in Q_{k_j}$ mit $k\ge k_0$\\
	$\Rightarrow$ & $\left\vert \int_Q f\D x - \int_Q h_k \D x \right\vert \le \int_Q \vert f - h_k\vert \D x \le \frac{\epsilon}{\vert Q \vert}\cdot \vert Q \vert = \epsilon$ $\forall k\ge k_0$
	\end{tabularx}
	
	Da $S(h_k) = \int_Q h_k \D x$ und $\epsilon > 0$ beliebig folgt $S(h_k)\to \int_Q f\D x$.
	
	Für jede \person{Riemann}-Folge $\{h_k\}$ zu $f$ ist $f$ R-integrierbar und Behauptung folgt.
\end{proof}