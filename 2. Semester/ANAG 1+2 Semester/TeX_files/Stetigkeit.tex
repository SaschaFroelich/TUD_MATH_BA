\section{Stetigkeit}
\begin{*definition}
	Sei stets $f:D\subset X\rightarrow Y$, $X,Y$ metrischer Raum, $D=\mathcal{D}(f)\neq \emptyset, y_0\in Y$ heißt \begriff{Grenzwert}[!Funktion] der Funktion $f$ im Punkt $x_0\in \overline{D}$, falls gilt:
	\begin{center}
		$\{x_n\}$ Folge in $D$ mit $x_n\to x_0$ $\Rightarrow$ $f(x_n)\to y_0$
	\end{center}
	Notaton: $\lim\limits_{x\rightarrow x_0} = y_0, f(x)\overset{x\to x_0}{\longrightarrow } y_0$
\end{*definition}
\stepcounter{theorem}
\begin{remark}
	Falls $x_0\in D$ \begriff{isolierter Punkt} von $D$, d.h. kein \gls{hp} von $D$, dann ist stets $\lim\limits_{x\rightarrow x_0} f(x) = f(x_0)$.
\end{remark}

\begin{proposition}[$\epsilon\delta$-Kriterium]
	Sei $f:D\subset X\to Y, x_0\in\overline{D}$. Dann
	\begin{center}
	$\lim\limits_{x\rightarrow x_0} f(x) = y_0 \;\Leftrightarrow \; \forall\epsilon > 0\,\exists \delta > 0: f(B_\delta(x_0)\cap D)\subset B_\epsilon(y_0)$
	\end{center}
\end{proposition}

\begin{proposition}[Rechenregeln] \label{proposition:rechenregel_stetigkeit}
	\begin{enumerate}[label={\arabic*)}]
		\item Sei $Y$ normierter Raum über $\mathbb{R}, f,g:D\subset X\to Y,\lambda: D\to K, x_0\in\overline{D}, f(x)\overset{x\to x_0}{\longrightarrow} y, g(x) \overset{x\to x_0}{\longrightarrow} \tilde{y}, \lambda(x)\overset{x\to x_0}{\longrightarrow} \alpha$. Dann:
		\begin{itemize}
			\item $(f+g)(x) \overset{x\to x_0}{\longrightarrow} y+\tilde{y}$
			\item $(\lambda \cdot f)(x) \overset{x\to x_0}{\longrightarrow} \alpha\cdot y$
			\item $\left(\frac{1}{\lambda}\right)(x) \overset{x\to x_0}{\longrightarrow} \frac{1}{\alpha}$ falls $\alpha\neq 0$
		\end{itemize}
		\item Sei $f: D\subset X\to Y, g:\tilde{D}\subset Y\to Z, \Re(f)\subset\tilde{D}, X,Y,Z$ metrische Räume, $x\in\overline{D}, f(x)\overset{x\to x_0}{\longrightarrow}y, g(y)\overset{y\to y_0}{\longrightarrow} z_0$. Dann:\\
		$g(f(x)) \overset{x\to x_0}{\longrightarrow} z_0$
	\end{enumerate}
\end{proposition}

\begin{*definition}
	\proplbl{einseitige_grenzwerte}
	Für $f:D\subset X\to Y$ mit $X=\mathbb{R}$ definieren wir einen \begriff{einseitiger Grenzwert} $y_0\in Y$ heißt \begriff[einseitiger Grenzwert!]{linksseitig} bzw. \begriff[einseitiger Grenzwert!]{rechtsseitig} von $f$ im \gls{hp} $x_0$ von $D\cap(-\infty, x_0)$ bzw. $D\cap(x_0,\infty)$, falls gilt: $x_n\in D\cap(-\infty, x_0)$ bzw. $x_n\in D\cap (x_0,\infty)$ mit $x_n\to x_0\,\Rightarrow \,f(x_n)\to y_0$
	
	$\begin{aligned}
		\text{Notation: } \lim\limits_{x\uparrow x_0} f(x) &= y_0 =: f(x_0^-)& f(x)&\overset{x\uparrow x_0}{\longrightarrow} y_0 \\
		\lim\limits_{x\downarrow x_0}f(x) &= y_0 =:f(x_0^+) & f(x) &\overset{x\downarrow x_0}{\longrightarrow} y_0
	\end{aligned}$
\end{*definition}

\begin{remark}
	Satz \ref{proposition:rechenregel_stetigkeit} gilt sinngemäß auch für einseitige Grenzwerte.
	
	Für $f:D\subset X\to Y$ mit $X=\mathbb{R}$ bzw. $Y=\mathbb{R}$ heißt der Grenzwert \begriff[Grenzwert!]{uneigentlich}\begriff*[Konvergenz!]{uneigentlich}[!Funktion]: \[\lim\limits_{x\to \pm \infty} f(x) = y_0, \lim\limits_{x\rightarrow x_0} f(x) = \pm \infty, \lim\limits_{x\to \pm \infty} f(x) = \pm \infty,\] indem wir einen Grenzwert definiert als $x_0=\pm \infty$ bzw. $y_0=\pm\infty$ wählen und bestimmte divergenzte Folgen $x_n\to \pm \infty$ mit $x_n\in D$) bzw. $f(x_n)\to \pm \infty$ betrachten.
\end{remark}
\stepcounter{theorem}
\subsection*{Landau-Symbole} (Vgl. von "`Konvergenzgeschwindigkeiten"')
\begin{*definition}
	Sei $f:D\subset X\to Y, X$ metrischer Raum, $Y$ normierter Raum, $g:D\subset X\to \mathbb{R}$, $x_0\in \overline{D}$.
	\begin{itemize}
		\item $f(x)$ ist "`\begriff{klein o}"' von $g(x)$ für $x\to x_0$, falls \[ \lim\limits_{\stackrel{x\to x_0}{x\neq x_0}} \frac{\Vert f(x)\Vert}{g(x)} = 0 \]
		Notation: $f(x) = o(g(x))$\mathsymbol*{o}{$o$} (meist $x\neq x_0$ im "`$\lim$"' weggelassen)
		\item $f(x)$ ist "`\begriff{groß O}"' von $g(x)$ für $x\to x_0$, falls \[ \exists \delta > 0, c \ge 0: \frac{\Vert f(x)\Vert}{|g(x)|} \le c \quad \forall x\in (B_\delta(x_0) \setminus \{x_0\})\cap D \]
		Notation: $f(x) = \mathcal{O}(g(x))$\mathsymbol*{O}{$\mathcal{O}$} für $x\to x_0$
	\end{itemize}
\end{*definition}
\stepcounter{theorem}
\stepcounter{theorem}
\subsection*{Relativtopologie}
\begin{*definition}
	Sei $(X,d)$ metrischer Raum, für $D\subset X$ ist $(D,d)$ ein metrischer Raum mit der induzierten Metrik.
	\begin{itemize}
		\item $M\subset D$ heißt \begriff[Relativtopologie!]{offen} bzw. \begriff[Relativtopologie!]{abgeschlossen} \highlight{relativ zu $D$}, falls $M$ offen bzw. abgeschlossen im metrischen Raum $(D,d)$.
		\item $M\subset D$ heißt \begriff[Relativtopologie!]{Umgebung} von $x\in D$ relativ zu $D$, falls $M$ Umgebung von $x$ im metrischen Raum $(D,d)$.
	\end{itemize}
\end{*definition}
\stepcounter{theorem}
\rule{4cm}{0.4pt}
\begin{*definition}
	Sei $f:D\subset X\to Y$ metrischer Raum, $D=\mathcal{D}(f)$, Fkt. $f$ heißt \begriff{folgenstetig} im Punkt $x_0\in D$, falls \[ f(x_n)\to f(x_0) \forall \text{ Folgen $x_n\to x_0$ in $D$} \]
\end{*definition}
\stepcounter{theorem}
\begin{*definition}
	Funktion $f$ heißt \begriff{stetig} im Punkt $x_0\in D$, falls $\forall $ Umgebungen $V$ von $f(x_0)\,\exists $ Umgebung $U$ von $x_0$ in $D:\,f(U)\subset V$.
	
	\begin{tabularx}{\textwidth}{lX}
		\noindent\highlight{Interpretation:} & Input / Output Steuerung besteht Forderung, dass beliebig kleine Output-Toleranzen $\epsilon$ stets durch hinreichend kleine Input-Toleranzen $\delta$ erreicht werden können.
	\end{tabularx}
\end{*definition}

\begin{proposition}
	Sei $f:D\subset X\to Y$, $X,Y$ metrischer Raum, $x_0\in D$. Dann:
	\begin{center}
		$f$ stetig in $x_0$ $\Leftrightarrow$ $f \,\epsilon\delta$-Stetig in $x_0$ $\Leftrightarrow$ $f$ folgenstetig in $x_0$
	\end{center}
\end{proposition}

\begin{*definition}
	Funktion $f$ heißt stetig (folgen- / $\epsilon\delta$-stetig) auf $M\subset D$, falls $f$ stetig (folgen-/$\epsilon\delta$-stetig) in jedem Punkt $x_0\in M$.
\end{*definition}
\stepcounter{theorem}
\begin{proposition}
	Sei $f:D\subset X\to Y, X,Y$ metrische Räume, dann sind folgende Aussagen äquivalent:
	\begin{enumerate}[label={\arabic*)}]
		\item $f$ stetig auf $D$
		\item $f^{-1}(V)$ offen in $D$ $\forall V\subset Y$ offen
		\item $f^{-1}(A)$ abgeschlossen in $D$ $\forall A\subset Y$ abgeschlossen
	\end{enumerate}
\end{proposition}
\begin{proposition}[Rechenregeln]
	\begin{enumerate}[label={\arabic*)}]
		\item Sei $Y$ normierter Raum über $K$, $f,g:D\subset X\to Y, \lambda: D\to U, f,g, ,y $ stetig in $x_0\in D$\\
		$\Rightarrow$ $f+g, \lambda\cdot f$ stetig in $x_0$, $\frac{1}{\lambda}$ stetig in $x_0$ falls $\lambda(x_0) \neq 0$
		\item Sei $f:D\subset X\to Y, y:\tilde{D}\subset Y\to Z, X, Y, Z$ metrischer Raum, $f$ stetig in $x_0$, $g$ stetig in $f(x_0)\in \tilde{D}$\\
		$\Rightarrow \,g\circ f$ stetig in $x_0$ 
	\end{enumerate}
\end{proposition}
\addtocounter{theorem}{3}
\begin{example}[\person{Dirichlet}-Funktion]
	$f:\mathbb{R}\to \mathbb{R}$ mit \[f(x) = \begin{cases}
	 1,&x\in\mathbb{Q}\\ 0,&\text{sonst}
	\end{cases} \] in keinem $x_0\in\mathbb{R}$ stetig.
\end{example}
\begin{proposition}
	\proplbl{chap_14_19}
	Sei $f_n, f:D\subset X\to X, f_n$ stetig in $x_0\in D$, $\forall n\in\mathbb{N}, f_n\to f$ gleichmäßig\\
	$\Rightarrow \, f$ stetig in $x_0$
\end{proposition}

\begin{conclusion}
	Falls alle $f_n$ stetig auf $M\subset D$ und $f_n\to f$ gleichmäßig auf $M$ \\
	$\Rightarrow\, f$ stetig auf $M$.
\end{conclusion}

\begin{proposition}
	Sei $f(z) := \sum_{k=0}^\infty a_k(z-z_0)^k\,\forall z\in B_r(z_0), R\in(0,\infty]$ Konvergrenzkreis, $a_k\in\mathbb{Z}\, \forall k\in \mathbb{N}$\\
	$\Rightarrow\, f:B_r(z_0) \to \mathbb{C}$ stetig auf $B_R(z_0)$
\end{proposition}
\addtocounter{theorem}{2}
\begin{*definition}
	Bijektive Abbildung $f:D\subset X\to R\subset Y, X,Y$ metrische Räume, $D=\mathcal{D}(f), R=\mathcal{R}(f)$ heißt \begriff{Homöomorphismus}, falls $f$ und $f^{-1}$ stetig.
	
	Mengen $D$ und $R$ heißen \begriff[Menge!]{homöomorph} zueinander, falls es einen Homöomorphismus $f:D\to R$ mit $D=\mathcal{D}(f), R=\mathcal{R}(f)$ gibt.
	
	\highlight{beachte:} Homöomorphismus bildet offene (abgeschlossene) Mengen auf offene (abgeschlossene) Mengen ab.
\end{*definition}
\stepcounter{theorem}
\begin{example}
	\begriff{stereographische Projektion}
	
	$X=\mathbb{R}^{n+1}, X_0 := \{(x_0, \dotsc, x_n{n+1}) \in\mathbb{R}^{n+1} \,|\, x_{n+1}=0\}, N = (0,\dotsc, 0,1)$ (Nordpol), $S_n = \{ x\in\mathbb{R}^{n+1} \,|\, |x|=1\}$ $n$-dimensionale Einheitsspäre.
	
	Betrachte $\sigma: \mathbb{R}^{n+1} \setminus\{ N\} \rightarrow \mathbb{R}^{n+1}$ mit $\sigma(x) = N \frac{2}{(x-N)^2}\langle x-N\rangle$ stetig. $\sigma$ ist Homöomorphismus mit $\sigma^{-1}(y) = N - \frac{2}{(y-N)^2}\langle Y-N\rangle$
\end{example}
\rule{4cm}{0.4pt}
\begin{proposition}
	Sei $f:D\subset \mathbb{R}\to \mathbb{R}$ streng monoton und stetig, $D$ Intervall \\
	$\Rightarrow f^{-1}$ existiert und ist stetig auf $\mathcal{R}(f)$.
\end{proposition}
\stepcounter{theorem}
\begin{proposition}
	Sei $f:X\to Y$ linear, $X,Y$ normierte Räume, $X=\mathcal{D}(f)$. Dann sind folgende Aussagen äquivalent:
	\begin{enumerate}[label={\arabic*)}]
		\item $f$ stetig in $x_0$
		\item $f$ ist stetig auf $X$
		\item $f$ ist beschränkt
	\end{enumerate}
\end{proposition}
\rule{4cm}{0.4pt}
\begin{*definition}
	Funktion $f:D\subset X\to Y, X,Y$ metrische Räume, heißt \begriff{gleichmäßig stetig} auf $M\subset D$, falls \[ \forall \epsilon > 0 \,\exists \delta > 0: d(f(x), f(\tilde{x})) < \epsilon\quad \forall x,\tilde{x}\in M \text{ mit $d(x,\tilde{x}) < \delta$}, \]
	d.h. $f$ ist $\epsilon\delta$-stetig in jedem $\tilde{x}\in M$ \highlight{und} $\delta > 0$ kann unabhängig von $x\in M$ gewählt werden.
\end{*definition}

\begin{proposition}
	Sei $f:D\subset X\to Y, X,Y$ metrischer Raum, $f$ stetig auf kompakten $M\subset D$ \\
	$\Rightarrow \,f$ gleichmäßig stetig auf $M$
\end{proposition}

\begin{*definition}
	Funktion $f:D\subset X\to Y, X,Y$ metrischer Raum, heißt \begriff{\person{Lipschitz}-stetig} auf $M\subset D$, falls \begriff{\person{Lipschitz}-Konstante} $L>0$ existiert mit \begin{align}
		\tag{L} d(f(x), f(\tilde{x})) \le Ld(x,\tilde{x})
	\end{align}
	
	\highlight{Spezialfall:} $X,Y$ normierte Räume, dann hat $L$ die Form \begin{align}
		\tag{L'} \Vert f(x) - f(\tilde{x})\Vert \le L\Vert x - \tilde{x}\Vert \quad\forall x,\tilde{x}\in M
	\end{align}
	
	\highlight{Interpretation:} für $X=Y=\mathbb{R}$ fixiere $\tilde{x}$
	\begin{itemize}
		\item Graph von $f$ liegt im schraffierten Kegel
		\item muss $\forall \tilde{x}\in M$ gelten mit gleichem $L$
	\end{itemize}
\end{*definition}

\begin{proposition}
	Sei $f:D\subset X\to Y$ \person{lipschitz}-stetig auf $M,X,Y$ metrische Räume\\
	$\Rightarrow$ $f$ gleichmäßig stetig auf $M$ (und damit auch stetig)
\end{proposition}

\addtocounter{theorem}{2}

\begin{*definition}[Fortsetzung, Einschränkung]
    Funktion $\tilde{f}: D(\tilde{f}) \to Y$ heißt Fortsetzung (bzw. Einschränkung) von $f \mathcal{D}(f) \to Y$ auf $\mathcal{D}(f)$ falls $\mathcal{D} \subset \mathcal{D}(\tilde{f})$ (bzw. $\mathcal{D}(\tilde{f}) \subset \mathcal{D}(f)$) und $\tilde{f}(x) = f(x) \,\forall x \in \mathcal{D}$ (bzw. $\forall x \in \mathcal{D}(\tilde{f}$). Für eine eingeschränkte Funktion $f$ auf $\mathcal{D}(\tilde{f})$, schreibe $\tilde{f} = f_{\vert \mathcal{D}(\tilde{f})}$.
\end{*definition}

\begin{proposition}
    Sei $f: D \subset X \to Y$ gleichmäßig stetig auf $D$, wobei $X,Y$  sind metrische Räume , $Y$ ist vollständig $\Rightarrow$ es existiert eindeutige stetige Fortsetzung $\tilde{f}$ von $f$ auf $\bar{D}$ und $\tilde{f}$ ist auf gleichmäßige stetige auf $\bar{D}$.
\end{proposition}

\begin{*remark}
    Falls $x_0$ kein Häufungspukt von $D$ ist, so kann man stets stetig auf $D\cup \{x_0\}$ fortsetzen (aber nicht eindeutig).
\end{*remark}

\addtocounter{theorem}{6}

\begin{conclusion}
    Sei $f: D \subset X \to Y$ linear, stetig, $Y$ vollständig $\Rightarrow$ es existiert eindeutig stetige Fortsetzung von $f$ auf $\bar{D}$.
\end{conclusion}