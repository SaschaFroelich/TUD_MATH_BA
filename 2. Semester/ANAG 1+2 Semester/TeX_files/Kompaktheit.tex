\section{Kompaktheit}
\begin{*definition}
Sei $(X,d)$ metrischer Raum, Mengensystem $\mathcal{U}\subset \{ U\subset X | U \text{ offen }\}$ heißt \begriff{offene Überdeckung} von $M\subset X$, falls $M\subset \bigcup_{U\in\mathcal{U}} U$.

Überdeckung $\mathcal{U}$ heißt endlich, falls $\mathcal{U}$ endlich (d.h. $\mathcal{U} = \{U_1,\dotsc,U_n\}$).

Menge $M\subset X$ heißt \highlight{(überdeckungs-)}\begriff[Menge!]{kompakt}, falls jede Überdeckung $\mathcal{U}$ eine endliche Überdeckung $\tilde{\mathcal{U}}\subset \mathcal{U}$ endhält (d.h. $\exists U_1,\dotsc, U_n\subset\mathcal{U}$ mit $M\subset\bigcup_{i=1}^n U_n$).

Menge $M\subset X$ heißt \begriff{folgenkompakt}, falls jede Folge $\{x_n\}$ aus $M$ (d.h. $x_n\in M\,\forall M$) eine konvergente Teilfolge $\{x_{n'}\}$ mit Grenzwert in $M$ besitzt (d.h. $\{x_n\}$ hat \gls{hw} in $M$ nach \ref{tfprinzip}).
\end{*definition}

\begin{boldenvironment}[Warnung]
	existiert endliche offene Überdeckung $\tilde{\mathcal{U}}$ von $M\Rightarrow M$ nicht unbedingt kompakt
\end{boldenvironment}

\begin{underlinedenvironment}[Hinweis]
	Eine Abbildung $A:I\to X$ nennt man auch \begriff{Familie} mit Indexmenge $I$ und schreibt $\{A_n\}_{i\in I}$Definition von "'kompakt"' in Literatur mittels Familien ist gleichwertig.
\end{underlinedenvironment}

\begin{theorem}
	\proplbl{theorem_kompakt_folgenkompakt}
	Sei $(X,d)$ metrischer Raum, $M\subset X$. Dann:\[M\text{ kompakt} \;\Leftrightarrow\; M\text{ folgenkompakt}\]
\end{theorem}
\begin{proof}
	\begin{itemize}
		\item $(\Rightarrow)$ Sei $\{x_n\}$ Folge in $M$, angenommen $\{x_n\}$ hat keinen HW in $M$ \\
		$\Rightarrow\exists\varepsilon_x>0:$ nur endlich viele $x_n\in B_{\varepsilon_x}(x)\Rightarrow M$ kompakt $\Rightarrow$ endlich viele $B_{\varepsilon_x}(x)$ überdecken $M\Rightarrow$ nur endlich viele Glieder $x_n$ in $M\Rightarrow$ aber Folge unendlich vieler Glieder $\Rightarrow\lightning\Rightarrow \{x_n\}$ hat HW in $M\beha$\\
		$(\Leftarrow)$ betrachte für $\varepsilon>0$ fest offene Überdeckung $U_{\varepsilon}:=\{B_{\varepsilon}(x)\mid x\in M\}$ von $M$. Angenommen, es gibt keine endliche Überdeckung $U'_{\varepsilon}\subset U_{\varepsilon}$ von $M$ \\
		$\Rightarrow\exists$ Folge $\{x_n\}$ in $M:x_1\in M$ und $x_{k+1}\in M\backslash\bigcup B_{\varepsilon}(x_i)\Rightarrow d(x_k,x_l)>\varepsilon\Rightarrow \{x_k\}$ hat keinen HW $\Rightarrow M$ folgenkompakt $\Rightarrow\lightning$
		\item Sei $U$ beliebige offene Überdeckung von $M$. Angenommen, es gibt keine endliche Überdeckung $U'\subset U$ von $M$ (1) \\
		nach 2.: $\varepsilon_k:=\frac{1}{k}$ gibt es offene Überdeckung $U_k$ von $M$ mit endlich vielen $\varepsilon_k$-Kugeln $\overset{(1)}{\Rightarrow}\forall k\;\exists x_k\in M:B_k:=B_{\varepsilon_k}(x_k)\in U_k$ und es gibt keine endliche Überdeckung $U'\subset U$ von $B_k\cap M$ (2) \\
		$\Rightarrow M$ folgenkompakt $\exists$ TF $x_{k'}\Rightarrow \tilde{x}\in M\Rightarrow\exists\tilde U\in U:\tilde{x}\in\tilde{U}\Rightarrow\tilde{U}$ offen $\Rightarrow\exists\tilde{\varepsilon}>0:B_{\tilde{\varepsilon}}(\tilde{x})\subset\tilde{U}\Rightarrow\exists k_0:d(x_{k_0},\tilde{x})<\frac{\tilde{\varepsilon}}{2}$ und $\frac{1}{k}=\varepsilon_{k_0}<\frac{\tilde{\varepsilon}}{2}\Rightarrow\forall x\in B_{k_0}: d(x,\tilde{x})\le d(x,x_{k_0})+d(x_{k_0},\tilde{x})<\tilde{\varepsilon}\Rightarrow B_{k_0}\subset B_{\tilde{\varepsilon}}(\tilde{x})\subset \tilde{U}\Rightarrow \{\tilde{U}\}\subset U$ ist endliche Überdeckung von $B_{k_0}\overset{(2)}{\Rightarrow}\lightning\Rightarrow 1$ falsch $\beha$
	\end{itemize}
\end{proof}

\begin{proposition}
	\proplbl{satzfolgenkompaktbeschraenktabgeschlossen}
	Sei $(X,d)$ metrischer Raum, $M\subset X$. Dann
	\begin{enumerate}[label={\arabic*)}]
		\item $M$ folgenkompakt $\Rightarrow$ $M$ beschränkt und abgeschlossen
		\item $M$ folgenkompakt, $A\subset M$ abgeschlossen $\Rightarrow$ $A$ folgenkompakt.
	\end{enumerate}
\end{proposition}
\begin{proof}
	\begin{enumerate}
		\item angenommen $M$ unbeschränkt $\Rightarrow\exists$ unbeschränkte Folge $\{x_n\}$ in $M$ ohne HW $\Rightarrow\exists$ keine konvergente TF $\Rightarrow\lightning\Rightarrow M$ beschränkt \\
		Sei $\{x_n\}$ Folge in $M$ mit $x_n\to x\Rightarrow M$ folgenkompakt $\Rightarrow x\in M\Rightarrow M$ abgeschlossen
		\item Sei $\{x_n\}$ Folge in $A\subset X\Rightarrow M$ folgenkompakt $\Rightarrow\exists$ TF $x_{n'}\to x\in M\Rightarrow A$ abgeschlossen $\Rightarrow x\in A\beha$
	\end{enumerate}
\end{proof}

\begin{theorem}[\person{Heine}-\person{Borell} kompakt, \person{Bolzano}-\person{Weierstraß} folgenkompakt]
	\proplbl{H_B_kompakt_B_W_folgenkompakt}
	Sei $X=\mathbb{R}^n$ (bzw. $\mathbb{C}^n$) mit beliebiger Norm, $M\subset X$. Dann \[ M \text{ kompakt} \;\Leftrightarrow\; M \text{ abgeschlossen und beschränkt} \] \\
	
	\begin{boldenvironment}[Warnung]
		Theorem gilt nicht in beliebigen metrischen Räumen! Betrachte $\real$ mit diskreter Metrik: $[0,1]$ nicht folgenkomakt, da $\left\lbrace \frac{1}{n}\right\rbrace$ keine HW hat.
	\end{boldenvironment}
\end{theorem}
\begin{proof}
	$(\Rightarrow)$ Folgt aus \propref{theorem_kompakt_folgenkompakt} und \propref{satzfolgenkompaktbeschraenktabgeschlossen} \\
	$(\Leftarrow)$ für $\real^n$: Norm in $\real^n$ ist äquivalent zu $\vert\cdot\vert_\infty$ \\
	Sei $\{x_k\}$ Folge in $M, x_k=(x^1_k,...,x^n_k)\in \real^n\Rightarrow M$ beschränkt $\Rightarrow \{\vert x_n\vert_\infty\}$ beschränkt in $\real\Rightarrow \{x^j_k\}$ beschränkt in $\real$ für $j=1,...,n\Rightarrow$ \person{Bolzano-Weierstraß} in $\real$ \\
	$\Rightarrow\exists$ TF $\{x_{k'}\}:x^1_{k'}\to x^1$ \\
	$\Rightarrow\exists$ TF $\{x_{k''}\}:x^2_{k''}\to x^2$, offenbar $x^1_{k''}\to x^1$ \\
	$\vdots$ \\
	$\Rightarrow\exists$ TF $\{x_{k*}\}:x^j_{k*}\to x^j\quad\forall j=1,...,n$ \\
	$\Rightarrow x_{k*}\to x=(x^1,...,x^n)$ in $\real^n\Rightarrow M$ abgeschlossen $\Rightarrow x\in M\Rightarrow M$ kompakt
\end{proof}

\begin{conclusion}
	Sei $\{x_n\}$ Folge in $X=\mathbb{R}^n$ (bzw. $\mathbb{C}^n$). Dann \[ \{x_n\}\text{ beschränkt} \;\Rightarrow \; \{x_n\} \text{ hat konvergente \gls{tf}}\]
\end{conclusion}
\begin{proof}
	folgt direkt aus dem Beweis von \propref{H_B_kompakt_B_W_folgenkompakt}
\end{proof}

\begin{proposition}
	\proplbl{aeqv_norm}
	Je 2 Normen aus $\mathbb{R}^n$ bzw. $\mathbb{C}^n$ sind äquivalent.
\end{proposition}
\begin{proof}
	zeige, dass beliebige Norm $\Vert.\Vert$ äquivalent zu $\vert\cdot\vert_\infty$ ist \\
	Sei $\{e_1,...,e_n\}$ Standardbasis, dann für $x=(x_1,...,x_n)\in \real^n, B=\sum_{j=1}^{n} \Vert e_j\Vert>0$ gilt: $\Vert x\Vert=\Vert\sum_{j=1}^n x_j\cdot e_j\Vert\le \sum_{j=1}^n \Vert x_j\Vert\cdot\Vert e_j\Vert\le B\cdot\vert x\vert_\infty$ (3) \\
	Sei $a:=\inf\{\Vert x\Vert\mid x\in S\}$ mit $S:=\{x\in \real^n\mid \vert x\vert_\infty=1\}$, angenommen, $a=0\Rightarrow^\exists \{x_k\}$ in $S:\Vert x_k\Vert\to 0$ \\
	$S$ beschränkt und abgeschlossen $\Rightarrow\exists$ TF $x_{k'}\to\tilde{x}\in S\Rightarrow \Vert\tilde{x}\Vert\le \Vert\tilde{x}-x_k\Vert+\Vert x_{k'}\Vert\le B\vert \tilde{x}-x_{k_0}\vert_\infty + \Vert x_k\Vert\to 0\Rightarrow \tilde{x}=0$, da $\vert 0\vert_\infty=0\Rightarrow\lightning\Rightarrow a>0\Rightarrow a\cdot\vert x\vert_\infty\le \Vert x\Vert\beha$
\end{proof}