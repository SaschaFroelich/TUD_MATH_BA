\section{Anwendung}\proplbl{chap_5}

Sei stets $f: D \subset X \to Y,X,Y$ metrische Räume, $D = \mathcal{D}(f)$.

\begin{proposition}
	\proplbl{satz_15_1}
    Sei $f: D \subset Y \to Y$ stetig, $M \subset D$ kompakt $\Rightarrow f(M)$ ist kompakt.
\end{proposition}

\begin{proposition}
    Sei $f; D \subset X \to Y$ stetig, injektiv, $D$ kompakt $\Rightarrow f^{-1}:f(D) \to D$ ist stetig.
\end{proposition}

\begin{theorem}[\index{Weierstraß}Weierstraß]\label{weierstrass}
	\proplbl{satz_von_weierstrass}
	\proplbl{chap_15_3}
    Sei $f: D \subset X \to Y$ stetig, $X$ metrischer Raum, $M \subset D$ kompakt, $M \neq \emptyset$
    \begin{align}
	    \Rightarrow \; \exists x_{min}, x_{max} : \left\{
	%\begin{cases}
       \begin{alignedat}{3}
	        f(x_{min}) &= \min&\{f(x)\mid x \in M\} &= \min_{x\in M} f(x),\\
		f(x_{max}) &= \max&\{f(x)\mid x \in M\} &= \max_{x\in M} f(x)
        %\end{cases}
	\end{alignedat}\right.
    \end{align}
    %TODO Fix x \in M under \max und \min
\end{theorem}

\begin{remark}
    Theorem \ref{weierstrass} ist wichtiger Satz für Existenz von Optimallösungen (stetige Funktion beseitzt auf kompakter Menge eine Minimum und Maximum). Folglich sind stetige Funktionen auf kompakten Mengen.
\end{remark}

\begin{proposition}
	\proplbl{chap_15_5}
    Sei $f: \mathbb{R}^n \to Y$ linear, $Y$ normierter Raum $\Rightarrow f$ ist stetig auf $\mathbb{R}^n$.\\
    Hinweis: Etwas allgemeiner hat man sogar $f: X \to Y$ linear, $X,Y$ normierte Räume, $\dim X < \infty \Rightarrow f $ ist stetig. (Ist i.a nicht richtig für $\dim X = \infty$.)
\end{proposition}

\begin{*definition}[\index{Kurve}Kurve]
    Eine stetige Abbildung $f: I \subset X \to Y$, wobei $I$ Intervall und $Y$ metrischer Raum ist heißt Kurve in $Y$ (gelegentlich wird auch Mange $f(I)$ als Kurve und $f$ also zugehörige Parametrisierung bezeichnet).
\end{*definition}

\begin{*definition}[bogenzusammenhängende Menge]
    Menge $M \subset X$, wobei $X$ ist metrische Raum, heißt \begriff[Menge!]{bogenzusammenhängend} (bogenweise zusammenhängend) falls $\forall a,b \in M \,\exists$ Kurve $f: [a,b] \to M$ mit $f(\alpha) = a, f(\beta) = b$.\\
    Bemerkung: Eigentlich ist das die Definition für Wegzusammenhängend, leider ist das in der Literatur nicht eindeutig und manchmal wird zwischen Wegzusammenhängend und zusammenhängend noch das "`echt"' bogenzusammenhängend unterschieden. %TODO definition echt bogenzusammenhängend hinzufügen.
\end{*definition}

\begin{*definition}[\index{zusammenhängende Menge}zusammenhängende Menge]
    Menge $M \subset X$ heißt zusammenhängend, falls
    \begin{align}
        A, B \subset M \text{ sind offen in }M\text{, disjunkt, }\emptyset \Rightarrow M \neq A \cup B.
    \end{align}
\end{*definition}

\begin{example}
    \begin{enumerate}[label={\arabic*)}]
    \item $x \in [0,2\pi] \to (x,\sin x) \in \mathbb{R}^2$ ist Kurve in $\mathbb{R}^2$
    \item $x \in [0,1] \to e^{i\pi x} \in \mathbb{C}$ oder $x \in [0,\pi]\to e^{i\pi} \in \mathbb{C}$ sind Kurven in $\mathbb{C}$
    \item Sei $Y$ normierter Raum, $a,b \in Y,f:[0,1] \to Y$ mit $f(t) = (1-t)\cdot a + t\cdot b$ ist Kurve (Strecke von $a$ nach $b$)
    \end{enumerate}
\end{example}

\begin{example}
    Sei $X=\mathbb{R}^2, M = \{(x,\sin x) \mid x \in (0,1]\} \cup \{(0,0)\}$. Dann ist $M$ zusammenhängend aber nicht bogenzusammenhängend.
\end{example}

\addtocounter{theorem}{1}
\begin{proposition}
    Sei $X$ metrischer Raum, $M \subset X$. Dann
    \begin{enumerate}[label={\arabic*)}]
    \item $X = \mathbb{R}: M$ ist zusammenhängend $\Leftrightarrow M$ ist Intervall (offen, abgeschlossen, halboffen, beschränkt, unbeschränkt).
    \item $M$ ist bogenzusammenhängend $\Rightarrow M$ ist zusammenhängend.
    \item Sei $X$ normierter Raum, dann: $M$ ist offen, zusammenhängend $\Rightarrow M$ ist bogenzusammenhängend.
    \end{enumerate}
\end{proposition}

\begin{*definition}[\index{Gebiet}Gebiet]
    Sei $X$ metrischer Raum, $M \subset X$ heißt \begriff{Gebiet} falls $M$ offen und zusammenhängend ist.\\
    Beachte: Gebiet in einem normiertem Raum ist sogar bogenzusammenhängend.\\
    Offenbar: $M \subset X$ ist konvex $\Rightarrow M$ ist bogenzusammenhängend.
\end{*definition}

\begin{proposition}
    Sei $f: D\subset X\to Y$ stetig, wobei $X,Y$ metrische Räume sind, dann gilt: $M \subset D$ ist zusammenhängend $\Rightarrow f(M)$ ist zusammenhängend.
\end{proposition}

\begin{theorem}[\index{Zwischenwertproposition}Zwischenwertproposition]
	\proplbl{zwischenwertsatz} \proplbl{satz_15_8}
    Sei $f: D \subset X \to \mathbb{R}, M \subset D$ zusammenhängend, $a,b \in M \Rightarrow f$ nimmt auf $M$ jeden Wert zwischen $f(a)$ und $f(b)$ an.
\end{theorem}

\addtocounter{theorem}{1}
%TODO add the example here or not?

\begin{example}
    $f:[a,b] \to \mathbb{R}$ sei stetig mit $f([a,b]) \subset [a,b] \Rightarrow$ besitzt \begriff{Fixpunkt}, d.h. $\exists x \in [a,b]\colon f(x)=x$.
\end{example}

\begin{theorem}[\index{Fundamentalproposition der Algebra}Fundamentalproposition der Algebra]\label{Fundam_algebra}
    Sei $f: \mathbb{C} \to \mathbb{C}$ Polynom vom Grad $n\geq 1$ (d.h $f(z) = a_n z^n + \dots + a_1 z + a_0,a_j \in \mathbb{C}, a_n \neq 0, n\geq 1$) $\Rightarrow f$ besitzt (mindestens eine) Nullstelle $z_0 \in \mathbb{C}$ (d.h. $f(z_0) = 0$).
\end{theorem}

\begin{conclusion}
    Jedes Polynom $f: \mathbb{C} \to \mathbb{C}$ von Grad $n, f\neq 0$ besitzt genau $n$ Nullstellen in $\mathbb{C}$ gezählt mit Vielfachen, d.h. $\exists z_1,\dots,z_l \in \mathbb{C}$, paarweise verschieden (=verschieden) $k_1,\dots, k_l \in \mathbb{N}_{\geq 0}$, $a_n \in \mathbb{C}\setminus\{0\}$ mit $k_1 + \dots + k_l = n$ und $f(z) = a_n \cdot (z-z_1)^{k_1}\cdot\dots\cdot(z-z_l)^{l}\,\forall z \in \mathbb{C}$. Hier heißt $k_j$ Vielfachheit der Nullstelle $z_j$.\\
    Hinweis: In dem Satz \ref{Polynomdiv} wurde gezeigt, das $f$ höchstens $n$ Nullstellen besitzt.
\end{conclusion}

\begin{*definition}[\index{analytische Funktion}analytische Funktion]
    Abbildung $f:\mathbb{C} \to \mathbb{C}$ heißt analytisch auf $B_R(z_0)\subset \mathbb{C}$ falls $f$ auf $B_R(z_0)$ durch Potenzreihe in $z_0$ darstellbar ist, d.h.
    \[
    f(z)=\sum_{k=0}^{\infty} a_k(z-z_0)^k \quad \forall z \in B_R(z_0).
    \]
\end{*definition}

\begin{proposition}
	\proplbl{chap_15_20}
    Sei $f:\mathbb{C}\to\mathbb{C}$ analytisch auf $B_R(z_0)$ und sei $B_r(z_1) \subset B_R(z_0)$ für $z_1 \in B_R(z_0),r>0 \Rightarrow f$ ist analytisch auf $B_r(z_1)$.
\end{proposition}

\begin{proposition}[\index{Identitätsproposition}Identitätsproposition]
    Seien $f,g:\mathbb{C} \to \mathbb{C}$ analytisch auf $B_R(z_0)$, sei $z_n \to \tilde{z},z_n\in B_R(z_0)\setminus\{\tilde{z}\}$ und $f(z_n) = g(z_n)\,\forall n \in \mathbb{N} \Rightarrow f(f) = g(z)\,\forall z \in B_R(z_0)$.
\end{proposition}

\begin{remark}
    Analytische Funktionen sind durch Werte auf "`sehr kleinen"' Mengen bereits festgelegt (z.B $\exp$, $\sin$, $\cos$ sind auf $\mathbb{C}$ eindeutig durch Werte auf $\mathbb{R}$ festgelegt).
\end{remark}

\begin{overview}
    Sei $X$ metrischer Raum, $Y$ normierter Raum.
    \begin{itemize}
    \item $B(X,Y):=\{f:X\to Y\mid \Vert f\Vert_{\infty} < \infty\}$ ist normierter Raum der beschränkten Funktionen mit $\Vert f\Vert_{\infty}=\sup\{\Vert f \Vert_{Y} \mid x \in X\}$.
    \item $C_b(X,Y):=\{f:X\to Y\mid \Vert f \Vert_{\infty} < \infty, f \text{ ist stetig}\}$ ist Menge der beschränkten stetigen Funktionen und offenbar eine linearer Unterraum von $B(X,Y)$ und damit auch Kern von $R \text{ mit } \Vert \cdot \Vert_{\infty}$.
    \item $C(X,Y):= \{f: X\to Y\mid f \text{ ist steig}\}$, Menge der stetigen Funktionen ist offenbar ein Vektorraum (enthält unbeschränkte Funktionen, z.B. $f(x)=\frac{1}{x} \text{ mit } x \in X = (0,1)$).
    \end{itemize}
\end{overview}

\begin{remark}
	Falls $X$ kompakt ist, dann kann man den Ausdruck $\Vert f \Vert_{\infty} < \infty$ in der Definition von $C_b(X,Y)$ weglassen (vgl. Theorem \ref{weierstrass}), d.h. $C_b(X,Y) = C(X,Y),f \text{ stetig }\Rightarrow X \to \Vert f(x)\Vert$ ist stetig $\overset{\text{Theorem 15.3}}{\Rightarrow} f$ ist beschränkt auf $X$. In diesem Fall ist auch $C(X,Y)$ mit $\Vert \cdot \Vert_{\infty}$ normierter Raum und $\Vert f\Vert_{\infty} = \max_{x\in M}\Vert f(x)\Vert_{Y}$.
\end{remark}

\begin{proposition}
    Sei $X$ metrischer Raum, $Y$ Banachraum $\Rightarrow B(X,Y)$ und $C_b(X,Y)$ und Banachräume (mit $\Vert \cdot \Vert_{\infty}$).
\end{proposition}

\begin{*definition}[\index{Kontraktion}Kontraktion]
    Funktion $f: D \subset X \to X$, wobei $X$ metrischer Raum ist, heißt \begriff{Kontraktion} (bzw. kontraktiv) auf $M \subset D$ falls
    \[
    \exists L, 0 \leq L < 1\colon d(f(x),f(y)) \leq L\cdot d(x,y) \quad \forall x,y \in M.
    \]
    D.h. $f$ ist Lipschitz-stetig mit Lipschitzkonstante $L < 1$, folglich ist $f$ auch stetig.
\end{*definition}

\begin{theorem}[\begriff*{Banacherscher Fixpunktproposition}Banacherscher Fixpunktproposition]\label{Banach_Fixpunkt}
    Sei $f : D\subset X \to Y$ Kontraktion auf $M \subset D, X$ vollständiger metrischer Raum (z.B. Banachraum), $M$ abgeschlossen und $f(M) \subset M$. Dann
    \begin{enumerate}
	    \item[(1)] $f$ besitzt genau einen Fixpunkt $\tilde{x}$ auf $M$ (d.h. $\exists$ genau ein $\tilde{x} \in M\colon f(\tilde{x}) = \tilde{x}$).
        \item[(2)] Für $\{x_n\}$ in $M$ mit $x_{n+1}=f(x_n),x_0 \in M$ (beliebig) gilt:
            \[
            x_n \to x \text{ und } d(x_n,\tilde{x}) \leq \frac{L^n}{1-L}\cdot d(x_0,x_1) \quad \forall n \in \mathbb{N}.
            \]
        \end{enumerate}
        Hinweis: Theorem \ref{Banach_Fixpunkt} ist eine wichtige Grundlage für Iterationsverfahren in der Numerik.
\end{theorem}

\subsection*{Partialbruchzerlegung}

\begin{*definition}[\index{Pol der Ordnung $k$}Pol der Ordnung $k$]
    Sei $R: \mathbb{C} \to \mathbb{C}$ rationale Funktion, d.h. $R(z) = \frac{f(z)}{g(z)}$ für Polynome $f$, $g$ existieren mit
    \begin{align*}
	    R(z) &= \frac{\tilde{f}(z)}{(z-z_0)^k\cdot \tilde{g}}& &\text{und}& \tilde{f}(z_0) \neq 0,&\;\tilde{g}(z_0) \neq 0.
    \end{align*}
    Motivation: Gelgentlich ist gewisse additive Zerlegung von rationalen Funktionen wichtig (Integration) z.B.
    \[
    \frac{2x}{x^2 - 1} = \frac{2x}{(x-1)(x+1)} = \frac{1}{x+1}+\frac{1}{x-1}.
    \]
\end{*definition}

\begin{lemma}
    Sei $R: \mathbb{C} \to \mathbb{C}$ rationale Funktion, $z_0 \in \mathbb{C}$ Pol der Ordnung $k\geq 1 \Rightarrow \,\exists ! a_1,\dots,a_k \in \mathbb{C},a_k\neq 0$ und $\exists !$ Polynom $\tilde{p}$ mit 
    \begin{align}
	    R(z) = \sum_{i=1}^{k}
	    \frac{a_i}{(z-z_0)^{i}} + \frac{\tilde{p}(z)}{\tilde{g}(z)} = H(z) +\frac{\tilde{p}(z)}{\tilde{g}(z)}
    \end{align}
    $H(z)$ heißt Hauptteil von $R \text{ in } z_0$. Beachte das $\frac{\tilde{p}}{\tilde{g}}$ keine Pole in $z_0$ hat.
\end{lemma}

\begin{proposition}[\index{Partialbruchzerlegung}Partialbruchzerlegung]
    Sei $R: \mathbb{C} \to \mathbb{C}$ rationale Funktion, $R(z)=\frac{f(z)}{g(z)}$ für Polynome $f,g$. Sei $g(z) = \prod_{i=1}^{l}(z-z_i)^{k_i}$ gemäß Fundamentalproposition der Algebra(Theorem \ref{Fundam_algebra}). Seien $z_1,\dots,z_l$ keine Nullstellen von $f$ und seien $H_1,\dots,H_l$ Hauptteile von $R$ in $z_1,\dots,z_l \Rightarrow$
    \[
    \exists \text{ Polynom } p:R(z)=H_1(z)+\dots+H_l(z)+p(z) \quad\forall z \neq z_j \,\forall j = 1,\dots,l
    \]
    wobei $f(z) = p(z)\cdot g(z) + r(z)\,\forall z$ für Polynom $r$. $p=0$ falls $\grad(f) < \grad(g)$ (vgl Satz \ref{Polynomdiv} Polynomdivision)
\end{proposition}