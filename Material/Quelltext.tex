\documentclass{article}
\usepackage{xcolor}
\usepackage{lmodern}
\usepackage{listings}

\definecolor{lightlightgray}{rgb}{0.95,0.95,0.95}
\definecolor{lila}{rgb}{0.8,0,0.8}
\definecolor{mygray}{rgb}{0.5,0.5,0.5}
\definecolor{mygreen}{rgb}{0,0.8,0.26}

\lstset{language=[95]Fortran,
  basicstyle=\ttfamily,
  keywordstyle=\color{lila},
  commentstyle=\color{lightgray},
  morecomment=[l]{!\ },% Comment only with space after !
  stringstyle=\color{mygreen}\ttfamily,
  backgroundcolor=\color{white},
  showstringspaces=false,
  numbers=left,
  numbersep=10pt,
  numberstyle=\color{mygray}\ttfamily,
  identifierstyle=\color{blue}
}
\begin{document}

\begin{lstlisting}
! Der folgende Fortran-Code ist bei Wikipedia geklaut.
SUBROUTINE test( Argument1, Argument2, Argument3 )
   REAL,              INTENT(IN) :: Argument1
   CHARACTER(LEN= *), INTENT(IN) :: Argument2
   INTEGER,           INTENT(IN), OPTIONAL :: Argument3
   ! This makes sense
   write(*,*) "Hallo Welt!"
END SUBROUTINE test
\end{lstlisting}
\end{document}