\section{Norm und Spur}
Sei $L\mid K$ endliche Körpererweiterung und $\alpha \in L$.
\begin{remark}
	$L$ ist ein $K$-Vektorraum $\implies$ $\End_K (L)$ ist ein $K$-Vektorraum und ein (nicht kommutativer) Ring unter Komposition.
\end{remark}
\begin{definition}[Spur, Norm]
	\begin{enumerate} %TODO fix 1) looks ugly
		\item \begin{align*}
		\mu_{\alpha} : \begin{cases}
		L & \to L\\
		x &\mapsto \alpha x
		\end{cases} \in \End_K (L)
		\end{align*}
		\item \begin{enumerate}[label=]
			\item $N_{L \mid K}(\alpha) := \det(\mu_{\alpha}$, die $(L \mid K)$- Norm von $\alpha$
			\item $\Tr_{L \mid K}(\alpha) := \Tr(\mu_{\alpha})$, die $(L\mid K)$-Spur von $\alpha$
		\end{enumerate}
		\item \begin{enumerate}[label=]
			\item $\chi_{\alpha} :=$ charakteristisches Polynom von $\mu_{\alpha}$
			\item $f_{\alpha} :=$ Minimalpolynom von $\mu_{\alpha}$
		\end{enumerate}
	\end{enumerate}
\end{definition}
\begin{lemma}
	\begin{enumerate}
		\item $f_{\alpha} = \MinPol(\alpha \mid K)$
		\item $\chi_{\alpha} = f_^{\alpha}^m$ für $m = [L:K(\alpha)]$
	\end{enumerate}
\end{lemma}
\begin{proof}
	\begin{enumerate}
		\item Sei
		\begin{align*}
			\mu: \begin{cases}
			L & \to \End_K(L)\\
			\beta & \mapsto \mu_{\beta} 
			\end{cases} \tag{$\star$} \label{bew:1_8_3}
		\end{align*}
		ist $K$-linearer Rainghomomorphismus: \checkmark\\
		Sei $g:= \MinPol(\alpha \mid K)$. Dann
		\begin{align*}
			\begin{pmatrix}
			g(\mu_{\alpha}) \overset{\eqref{bew:1_8_3}}{=} \mu_{g(\alpha)} = 0 \in \End_K(L) \implies f_{alpha}\mid g\\
			\mu_{f_{\alpha}(\alpha)} \overset{\eqref{bew:1_8_3}}{=} f_{\alpha}(\mu_{\alpha}) = 0 \in \End_K(L) \xRightarrow{\mu \text{ inj.}} f_{\alpha}(\alpha) = 0 \implies g \mid f_{\alpha}
			\end{pmatrix} \implies f_{\alpha} = g
		\end{align*}
		\item Ccharakteristisches Polynom und Minimalpolynom haben die gleichen irreduziblen Faktoren: $\nearrow$ LAAG VIII.7.6 oder direkt:
		\begin{enumerate}[option]
			\item $V$ $n$-dimensionaler $K$-VR, $\varphi \in \End_K(V)$, $\mathscr{B}$ Basis von $V \rightsquigarrow A = M_{\mathscr{B}}(f)$. $\chi_{\varphi} = \chi_A \in K[X]$ zerfällt in Linearfaktoren in $\bar{K}[X]$\\
			$\implies$ lese $\chi_{\varphi} = \chi_A \und P_{\varphi} = P_A$ aus der Jordan-Normalform von $A$ ab.
		\end{enumerate}
		$f_{\alpha} = \MinPol(\alpha\mid K)$ irreduzibel $\implies \chi_{\alpha} = f_{\alpha}^m$ für ein $m$\\
		\begin{align*}
			\begin{pmatrix}
			\deg(f_{\alpha}) &= \deg(\MinPol(\alpha\mid K)) = \deg(\alpha \mid K) = [K(\alpha : K)]\\
			\deg(\chi_{\alpha}) = [L:K]
			\end{pmatrix}\implies m = \frac{\deg(\chi_{\alpha})}{\deg(f_{\alpha})} = \frac{[L:K]}{[K(\alpha):K]} = [L:K(\alpha)]
		\end{align*}
	\end{enumerate}
\end{proof}
\begin{example}
	Sei $\C = \R + \R\ii, \alpha = x+y\ii \in \C$.
\end{example}
%to add 16 May 2019 %TODO
\begin{conclusion}
	Sei $K \subseteq L \subseteq M$, dann
	\begin{itemize}
		\item $N_{L \mid K}(\alpha) = N_{L\mid K}(N_{L\mid K}(\alpha))$
		\item $\Tr_{L \mid K}(\alpha) = \Tr_{L \mid K}(\Tr_{L \mid K}(\alpha))$
	\end{itemize}
\end{conclusion}
\begin{proof}
	Sei $[L:K] = q_1 \cdot r_1, [M:L] = q_2 \cdot r_2$. $\Hom(L, \bar{K}) = \set{\tau_1, \dots, \tau_{r_1}}, \Hom(M, \bar{L}) = \set{\sigma_1, \dots, \sigma_{r_2}}$. Fixiere Einbettung $L \subseteq \bar{K}$ und setze $\tau_i$ fort zu $\tilde{\tau}_i \in \Aut(\bar{K}, K)$( \propref{1_4_11}) %TODO
	Dann ist
	\[
		\Hom_K(M, \bar{K}) = \set{\tilde{\tau}_i \circ \sigma_j, i =1, \dots, r_1, j = 1, \dots, r_2}
	\]
	denn $\neq \Hom(M, \bar{K}) = [M, K] = r_1 \cdot r_2$ und
	\begin{align*} %TODO check for { } misssing
		\tilde{\tau}_{i'} \circ \sigma_{j'} \implies \sigma_j = (\tilde{\tau}_i^{-1} \circ \tilde{\tau}_{i'})\circ \sigma_{j'}\\
		\implies \tilde{\tau}_i^{-1} \circ \tilde{\tau}_{i\mid L} = \id_L = \tau_i = \tau_{i'} \implies i = i'\\
		\implies \sigma_j = \sigma_{j'} \implies j = j'.\\
		\implies N_{L\mid K}(N_{M\mid K}(\alpha)) = N_{L\mid K}\brackets{\prod_{i=j}^{r_2} \sigma_j (\alpha)}^{q_2}\ \overset{\cref*{1_2_7}}{=}\\
		= \brackets{ \prod_{i=1}^{r_1} \tau_i \brackets{\prod_{j=1}^{r_2} \sigma_j (\alpha) }}{q_1 q_2} = \brackets{\prod_{i,j} (\tilde{\tau}_i \circ \sigma_j (\alpha))}^{q_1 q_2} \overset{\cref{1_8_7}}{=} N_{M \mid K}(\alpha).
	\end{align*}
	Analog für die Spur.
\end{proof}
\begin{theorem}[Unabhängigkeit der Charaktere, \person{Artin}]
	\proplbl{1_8_9}
	Sei $G$ eine Gruppe. Sind $\chi_1, \dots, \chi_n \in \Hom(G, K^{\times})$ paarweise verschieden, so sind sie linear unabhängig im $K$-Vektorraum $\Abb(G,K)$.
\end{theorem}
\begin{proof} %TODO reformat!
	Seien $\chi_1, \dots, \chi_n$ linear abhängig, oE $n \ge 2$ minimal, d.h. $\sum_{i=1}^n a_i \chi_i = 0 \mit a_1, \dots, a_n \in K^{\times}$. Sind $\chi_1 \neq \chi_n \implies \exists g \in G \mit \chi_1(g) \neq \chi_n(g)$. $\sum a_i \chi_i = \implies \forall h \in G$ ist $\sum_{i=1}^n a_i \chi_i (h) = 0$.
	\begin{align*}
		\implies \begin{cases}
			\sum_{i=1}^n a_i \cdot \underbrace{\chi_i (hg)}_{\chi_i(h)\cdot \chi_i(g)} &= 0\\
			\sum_{i=1}^n a_i \cdot \chi_i(h)\cdot \chi_i(g) = 0
		\end{cases}
		\implies 0 &= \sum_{i=1}^n a_i \cdot \chi_i(h)(\chi_i(g) - \chi_n(g))\\
		&= \sum_{i=1}^{n-1} a_i (\chi_i(g) - \chi_n(g))\cdot \chi_i(h)\\
		\implies \sum_{i=1}^{n-1} a_i \cdot (\chi_i(g) - \chi_n(g))\cdot \chi_i = 0\\
		a_n (\chi_1(g) - \chi_n(g)) \neq 0
	\end{align*}
	das ist ein Widerspruch zur Minimalität von $n$.
\end{proof}
\begin{conclusion}
	\proplbl{1_8_10}
	Genau dann ist $\Tr_{L \mid K} \neq 0$, wenn $L \mid K$ separabel.
\end{conclusion}
\begin{proof}\
	\begin{itemize}
		\item ``$\implies$'': \propref{1_8_6}
		\item ``$\Longleftarrow$'': Sei $\Hom_K(L, \bar{K}) = \set{\sigma_1, \dots, \sigma_n}$. $\sigma_{i \mid L^{\times}} \in \Hom_K(L^{\times}, K^{\times})$\\
		$\xRightarrow{\cref{1_8_7}} \sigma_1,\dots, \sigma_n$ sind $\bar{K}$-linear unabhängig. Insbesondere ist $\Tr_{L \mid K} = \sum_{i=1}^n \sigma_i \neq 0$.
	\end{itemize}
\end{proof}