\section{Der algebraische Abschluss}
Sei $L\mid K$ eine Körpererweiterung.
\begin{definition}[algebraisch abgeschlossen]
	$K$ ist algebraisch abgeschlossen $\Longleftrightarrow$ jedes $f \in K[X] \mit \deg(f) > 0$ hat eine Nullstelle in $K$.
\end{definition}
\begin{lemma}
	\proplbl{1_4_2}
	Es ist äquivalent:
	\begin{enumerate}[label=(\alph*)]
		\item $K$ ist algebraisch abgeschlossen. \label{aussage:1_4_2:1}
		\item Jedes $0 \neq f \in K[X]$ zerfällt über $K$ in Linearfaktoren. \label{aussage:1_4_2:2}
		\item $K$ hat keine echte algebraische Erweiterung. \label{aussage:1_4_2:3}
	\end{enumerate}
\end{lemma}
\begin{proof} %TODO ref
	\begin{enumerate}[label=(\alph*)]
		\item \ref{aussage:1_4_2:1} $\Rightarrow$ \ref{aussage:1_4_2:2}: Induktion nach $\deg(f)$ (siehe LAAG)
		\item \ref{aussage:1_4_2:2} $\Rightarrow$ \ref{aussage:1_4_2:3}: Sei $L \mid K$ algebraisch, $\alpha \in L$. Schreibe $f = \MinPol(\alpha \mid K)$. Nach \ref{aussage:1_4_2:2} zerfällt $f$ in Linearfaktoren über $K \Rightarrow \alpha \in K$
		\item \ref{aussage:1_4_2:3} $\Rightarrow$ \ref{aussage:1_4_2:1}: Sei $f \in K[X], \deg(f) > 0$. Nach \propref{1_3_9} existiert ein Zerfällungskörper $L$ von $f$. Da $L\overset{(*)}{=}K$ nach \ref{aussage:1_4_2:3} hat $f$ Nullstellen in $K$. \\
		($(*)$ $L$ ist Erweiterung $\rightarrow$ die nach \ref{aussage:1_4_2:3} trivial ist)
	\end{enumerate}
\end{proof}
\begin{definition}[\begriff{algebraisch Abgeschlossen}]
	$L$ ist algebraischer Abschluss von $K :\Longleftrightarrow L$ ist algebraisch abgeschlossen und $L\mid K$ algebraisch.
\end{definition}
\begin{lemma}
	\proplbl{1_4_4}
	Ist $L$ algebraischer Abschluss, so ist der relative algebraische Abschluss $\tilde{K}$ ein algebraischer Abschluss von $K$.
\end{lemma}
\begin{proof}
	\begin{itemize}
		\item $\tilde{K}$ ist Körper: \propref{1_2_15}
		\item $\tilde{K} \mid K$ ist algebraisch: Definition
		\item $\tilde{K}$ ist algebraisch abgeschlossen: Sei $f \in \tilde{K}[X] \mit \deg(f) > 0$.\\
		$L$ algebraisch abgeschlossen $\Rightarrow$ existiert $\alpha \in L \mit f(\alpha) = 0$ und $f(\alpha) = 0 \Rightarrow \alpha$ algebraisch über $\tilde{K} \xRightarrow{\propref{1_2_15}} \alpha \in \tilde{K}$.
	\end{itemize}
\end{proof}
\begin{example}
	\begin{enumerate}[label=(\alph*)]
		\item $\C$ ist algebraisch abgeschlossen (Fundamentalsatz der Algebra, $\nearrow$ II.) %TODO \nearrow II later
		\item $\C$ ist algebraischer Abschluss von $\R$.
		\item $\tilde{\Q} := \set{\alpha \in \C \mid \alpha \text{ algebraisch über }\Q}$ ist nach \propref{1_4_4} ein algebraischer Abschluss von $\Q$.
	\end{enumerate}
\end{example}
\begin{lemma}
	\proplbl{1_4_6}
	Sei $L\mid K$ algebraisch, $E$ ein algebraisch abgeschlossener Körper und $\varphi \in \Hom(K,E)$. Dann existiert eine Fortsetzung von $\varphi$ auf $L$, d.h. ein $\sigma \in \Hom(L,E) \mit \sigma_{\mid K} = \varphi$.
\end{lemma}
\begin{proof}
	Definiere Halbordnung.
	\begin{align*}
		\Halb &:= \set{(M,\sigma) : K \subseteq M \subseteq L \text{ Zwischenkörper, }\sigma\in \Hom(M,E), \sigma_{\mid K} = \varphi}\\
		&(M,\sigma) \subseteq (M' , \sigma') :\Leftrightarrow m \subset M' \und \sigma'_{\mid M} = \sigma
	\end{align*}
	\begin{itemize}
		\item $\Halb \neq \emptyset$: $(K,\varphi) \in \Halb$
		\item Ist $(M,\sigma)_{i \in I}$ eine Kette in $\Halb$, so definieren wir $M:= \bigcup_{i\in I} M_i$ und $\sigma: M \to E$ durch $\sigma(x) = \sigma_i (x)$ falls $x \in M_i$. Dann ist $(M,\sigma) \in \Halb$ eine obere Schranke der Kette $(M_i , \sigma_i)_{i\in I}$. Nach Lemma von \person{Zorn} existiert $(M, \sigma)$ maximal. 
		Es ist $M = L$: Sei $\alpha \in L, f= \MinPol(\alpha\mid M)$. $f \in E[X]$ hat Nullstelle $\beta \in E$, da $E$ algebraisch abgeschlossen ist.
		$\xRightarrow{\propref{1_3_12}}$ existiert Fortsetzung $\sigma' \in \Hom(M(\alpha), E)$ von $\sigma$\\
		$(M,\sigma) \le (M(\alpha, \sigma')) \in \Halb \xRightarrow{(M(\alpha), \sigma) \text{ max.}} M = M(\alpha), \alpha \in M.$
	\end{itemize}
\end{proof}
\begin{theorem}[\person{Steinitz}, 1910]
	Jeder Körper $K$ besitzt einen bis auf $K$-Isomorphie eindeutig bestimmten algebraischen Abschluss.
\end{theorem} %TODO tikzcd!
\begin{proof}\
	\begin{itemize}
		\item Eindeutigkeit:\\
		Seien $L_1 , L_2$ algebraische Abschlüsse von $K$\\
		$L_1 \mid K$, $L_2$ algebraisch abgeschlossen $\xRightarrow{\propref{1_4_6}}$ existiert $\sigma \in \Hom(L_1 , L_2)$
		\begin{align*} %TODO find a way to have only the right curly bracket?
		\begin{Bmatrix}
		L_1 \text{ algebraisch abgeschlossen }\Rightarrow \sigma(L_1) \cong L_1 \text{ algebraisch abgeschlossen}\\
		L_2 \mid K \text{ algebraisch } \Rightarrow L_2 \mid \sigma(L_1) \text{  algebraisch }
		\end{Bmatrix}\xRightarrow{\propref{1_4_2}} L_2 = \sigma(L_1).
		\end{align*}
		Somit ist $\sigma: L_1 \to L_2$ ein $K$-Isomorphismus.
		\item Existenz: Seien
		\begin{enumerate}[label=(\alph*)]
			\item $\mathscr{F} = \set{f \in K[X] \colon \deg(f) > 0}$
			\item $\Halb = (X_f)_{f \in \mathscr{F}}$ Familie von Variablen
			\item $R := K[\Halb]$ Polynomring in den Variablen $X_f$ ($f \in mathscr{F}$)
			\item $I := (f(X_f) \colon f \in \mathscr{F}) \properideal R$
		\end{enumerate}
			\begin{enumerate}[label=(\alph*)]
				\item Behauptung 1: Es gilt $I \properideal R$.
					\begin{proof}[Behauptung 1]
						Angenommen $I = R$. Dann existieren $f_1, \dots, f_n \in \mathscr{F}$ und $g_1, \dots, g_n \in R$ mit $\sum_{i=1}^n g_i \cdot f_i (X_f{f_i}) = 1$. Sei $L$ ein Zerfällungskörper von $f_1, \dots, f_n$. Dann existieren $\alpha_1, \dots, \alpha_n \in L$ mit $f_i(\alpha_i) = 0$ für alle $i$. Sei $\varphi: R \to L$ der Einsetzungshomomorphismus gegeben durch
						\begin{align*}
							&\varphi_{\mid K} = \id_K \quad \varphi(X_{f_i}) = \alpha_i \quad \varphi(X_f) = 0 \text{ für } f \in \lnkset{\mathscr{F}}{\set{f_1,\dots, f_n}}\\
							&\implies 1 = \varphi(1) = \sum_{i=1}^n \varphi(g_i) \cdot \varphi(f_i (X_f))\\
							&= \sum_{i=1}^n \varphi(g_i) \cdot f_i (\underbrace{\varphi(X_f)}_{= \alpha_i}) = \sum_{i=1}^n \varphi(g_i) \cdot \underbrace{f_i (\alpha_i)}_{=0} = 0
						\end{align*}
					\end{proof}
				\end{enumerate}
		Jedes echte Ideal ist in einem maximalen Ideal von $R$ enthalten (GEO II 2.13) $\implies$ existiert maximales Ideal $m \unlhd R \mit I \subseteq m$. $L_1 := \lnkset{R}{m}$ ist Körpererweiterung von $K$, und jedes $f \in \mathscr{F}$ hat eine Nullstelle in $L_1$, nämlich $f(X_f + m) = f(X_f) + m = 0 + m$. Iteriere dies und erhalte eine Kette von Körpern
		\begin{align*}
			K := L_0 \subseteq L_1 \subseteq L_2 \subseteq \cdots,
		\end{align*}
		wobei jedes $f \in L_i [X], \deg(f) >0$ eine Nullstelle in $L_{i+1}$ hat. Setze nun $L = \bigcup_{i=1}^{\infty}$.
			\begin{enumerate}[label=(\alph*)]
				\item Behauptung 2: $L$ ist algebraisch abgeschlossen.
				\begin{proof}[Behauptung 2]
					Sei $f \in L[X], \deg(f) > 0 \implies f \in L_i [X]$ für ein $i \implies f$ hat eine Nullstelle in $L_{i+1} \subseteq L$
				\end{proof}
			\end{enumerate}
		Nach \propref{1_4_4} ist somit $\tilde{K} = \set{\alpha \in L \colon \alpha \text{ algebraisch über }K}$ ein abgeschlossener Abschluss von $K$.
%		\end{enumerate}
	\end{itemize}
\end{proof}
\begin{definition}[algebraischer Abschluss]
	Mit $\bar{K}$ bezeichnen wir den (bis auf $K$-Isomorphie eindeutig bestimmten) \begriff{algebraischen Abschluss} von $K$.
\end{definition}
\begin{definition}[Automorphismengruppe]
	$\Aut(L\mid K) := \set{ \sigma \in \Hom_K (L,L)\colon \sigma \text{ Isomorphismus} }$, die Automorphismengruppe von $L\mid K$.
\end{definition}
\begin{remark}
	$\Aut(L \mid K)$ ist Gruppe unter $\sigma \cdot \sigma' = \sigma' \circ \sigma$ und wirkt auf $L$ durch $x^{\sigma} := \sigma(x)$.
\end{remark}
\begin{proposition}
	\proplbl{1_4_11}
	Sei $K \subseteq L \subseteq \bar{K}$ ein Zwischenkörper. Jedes $\varphi \in Hom_K (L, \bar{K})$ lässt sich zu einem $\sigma \in \Aut(\bar{K}\mid K)$ fortsetzen.
\end{proposition}
\begin{proof}
	Sei $\bar{K} \mid K$ algebraisch abgeschlossen und $\bar{K}$ algebraisch abgeschlossen\\
	$\xRightarrow{\propref{1_4_6}}$ existiert Fortsetzung $\sigma \in \Hom_K (\bar{K}, \bar{K})$ von $\varpi$\\
	$\bar{K}$ algebraisch abgeschlossen $\implies \sigma(\bar{K})$ algebraisch abgeschlossen\\
	$\bar{K} \mid K$ algebraisch ist $\implies \bar{K} \mid \sigma(\bar{K})$ algebraisch ($\bar{K} = \sigma(\bar{K})$) somsit ist $\sigma \in \Aut(\bar{K}, K)$.
\end{proof}
\begin{definition}[konjugiert]
	$\alpha, \beta \in \bar{K}$ sind $K$-konjugiert $\Longleftrightarrow$ existiert $\sigma \in \Aut(\bar{K}, K)\mit \sigma(\alpha) = \beta$.
\end{definition}
\begin{remark}
	$K$-Konjugiertheit ist eine Äquivalenzrelation auf $\bar{K}$.
\end{remark}
\begin{conclusion}
	$\alpha, \beta \in \bar{K}$ sind $K$-konjugiert $\equi$ $\MinPol(\alpha \mid K) = \MinPol(\beta \mid K)$.
\end{conclusion}
\begin{proof}\
	\begin{itemize}
		\item $\Rightarrow:$ $\sigma(\alpha) = \beta \mit \sigma \in \Aut(\bar{K}\mid K), f \in K[X]$, $f(\alpha) = 0 \implies 0 = \sigma(0) = \sigma(f(\alpha)) = f(\sigma(\alpha)) = f(\beta)$
		\item $\Leftarrow:$ $\MinPol(\alpha \mid K) = \MinPol(\beta \mid K)$\\
		$\xRightarrow{\propref{1_3_5}}$ existiert $K$-Isomorphismus $\varphi: K(\alpha) \to K(\beta) \mit \varphi(\alpha) = \beta$\\
		$\xRightarrow{\propref{1_4_11}}$ existiert Fortsetzung $\sigma \in \Aut(\bar{K}, K)$ von $\varphi$. 
	\end{itemize}
\end{proof}
\begin{example}
	\begin{itemize}
		\item $\ii, -\ii \in\tilde{\Q}$ sind $\Q$-konjugiert: komplex Konjugation (eingeschränkt auf $\tilde{\Q}$)
		\item $\sqrt{2}, -\sqrt{2} \in \tilde{\Q}$ sind $\Q$-konjugiert: $\MinPol(\sqrt{2}\mid \Q) = x^2 -2 = \MinPol(-\sqrt{2}\mid \Q)$
	\end{itemize}
\end{example}