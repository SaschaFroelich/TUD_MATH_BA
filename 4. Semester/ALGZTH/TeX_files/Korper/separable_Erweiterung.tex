\section{Separable Polynome}
Sei $K$ ein Körper, $f \in K[X], n = \deg(f)$.
\begin{definition}[title]
	Sei $a \in K$.
	\begin{enumerate}
		\item $\mu(f,u) := v_{x-a}(f) := \sup \set{k \in \N_0 : (x-a)^k \mid f} \in \N_0 \cup \set{\infty}$ die \begriff{Vielfachheit} der Nullstelle von $f :\iff \mu(f,a) = 1$
		\item 
	\end{enumerate}
\end{definition}
\begin{remark}
	
\end{remark}
\begin{definition}[title]
	Die \begriff{formale Ableitung} von $f = \sum_{i=1}^n i a_i X^{i-1}$ ist $f' := \frac{\d}{\d x} f(x) := \sum_{i=1} i a_i X^{i-1} \in K[X]$
\end{definition}
\begin{lemma}
	Für $f,g \in K[X], a,b \in K$ gelten\begin{enumerate}
		\item $(af + bg)' = a f' + b g'$ (Linearität)
		\item $(fg)' = f'g + fg'$ (Produktregel)
		\item $(f(g(x)))' = f'(g(x))\cdot g'(x)$ (Kettenregel)
	\end{enumerate}
\end{lemma}
\begin{proof}
	Übung.
\end{proof}
\begin{lemma}
	Sei $f \neq 0$. Für $a \in K$ gilt
	\[
	\mu(f' , a) \ge \mu(f,a) = 1
	\]
	mit Gleichheit genau dann, wenn
	\[
	\chara(K) \not\mid \mu(f,a)
	\]
\end{lemma}
\begin{proof}
	Schreibe $f = (X-a)^k \cdot g, k = \mu(f,a)$
	\begin{itemize}
		\item $k=0$: $\mu(f', a) \ge 0 > -1$ und $\chara(K) \mid 0$
		\item $k>0$: $f' = k(X-a)^{k-1}g + (X-a)^k \cdot g'$\\
		$\implies \mu(f',a) \ge k$\\
		\begin{align*}
			\mu(f',a) \ge k &\iff (X-a)^k \mid k(X-a)^{k-1}g\\
			&\iff X-a \mid k
		\end{align*}
	\end{itemize}
\end{proof}
%TODO
\begin{conclusion}
	Sei $f$ irreduzibel
	\begin{enumerate}
		\item Ist $\chara(K) = 0$, so ist $f$ separabel
		\item Ist $\chara(K) = p>0$, so sind äquivalent
		\begin{enumerate}
			\item $f$ ist inseparabel
			\item $f' = 0$
			\item $f(X) = g(X^p)$ für ein $g \in K[X]$
		\end{enumerate}
	\end{enumerate}
\end{conclusion}
\begin{proof}
	$f$ irreduzibel $\implies \underbrace{\ggT(f,f') \sim 1}_{\iff f \text{ sep}} \oder \underbrace{\ggT(f,f') \sim f}_{\iff f \text{ sep.}}$. (Nutze 6.6 für die $\iff$) Da $\deg(f') = \deg(f)$ ist
	\[
		f \mid f' \iff f' = 0 \iff f(x) = g(x^p). \text{ für ein }g
	\]
	Im Fall $\chara(K) = 0$ tritt dieser Fall nicht ein.
\end{proof}
\begin{definition}[vollkommen]
	$K$ ist \begriff{vollkommen} $\iff$ jedes irreduzibel $f \in K[X]$ ist separabel.
\end{definition}
\begin{example}
	\begin{enumerate}
		\item $\chara(K) = 0 \implies K$ ist vollkommen
		\item $K = \bar{K} \implies K$ ist vollkommen
		\item $K = \F_p (t)$ ist nicht vollkommen:
		\begin{align*}
			f = X^p - t \in K[X] \text{ ist irreduzibel}\\
			f' = X^{p-1} = 0 \implies f \text{ nicht seperabel.}
		\end{align*}
		Tatsächlich hat $f$ nur eine Nullstelle in $\bar{K}$:
		\begin{align*}
			f = X^p - t \overset{V1}{=} (X - t^{\frac{1}{t}})^p
		\end{align*}
	\end{enumerate}
\end{example}
\begin{definition}[title]
	Sei $\chara(K) = p > 0$.
	\begin{enumerate}
		\item \begin{align*}
		\varphi_p : \begin{cases}
		K &\to K\\
		x &\mapsto x^p 
		\end{cases}
		\end{align*}
		\person{Frobenius}-Endomorphismus von $K$
		\item $K^p = \Image(\varphi_p) = \set{a^p : a \in K}$
	\end{enumerate}
\end{definition}
\begin{proposition}
	Sei $\chara(K) = p > 0$. Dann ist $\varphi_p \in \End(K): =\Hom(K,K)$
\end{proposition}
\begin{proof}
	Für $a, b \in K$ ist
	\begin{itemize}
		\item $\varphi_p = (ab)^p = a^p \cdot b^p = \varphi_p (a) \cdot \varphi_p(b)$
		\item $\varphi_p(a+b) = (a+b)^p = \sum_{i=0}^p\binom{p}{i} a^i b^{p-i} = b^p + a^p = \varphi_p(a) + \varphi_p(b)$, da $p \mid \binom{p}{i}$ für $i = 1, \dots, p-1$ (V1).
		\item $\varphi_p(1) = 1^p = 1$
	\end{itemize}
\end{proof}
\begin{remark}
	\begin{enumerate}
		\item Da $\varphi_p \in \End(K) $ ist $K^p$ ein Teilkörper von $K$ und $\varphi_p$ ist injektiv.
		\item Insbesondere gibt es zu jedem $a \in K$ ein eindeutig bestimmtes $a^{\frac{1}{p}} \in \bar{K}$ mit
		\[
			\varphi_p(a^{\frac{1}{p}}) = (a^{\frac{1}{p}})^p = a
		\]
		\item Für $a \in \F \cong \F_p$ ist $\varphi_p(a) = a$. (z.B. $\varphi_p(1) = 1$ oder kleiner Satz von \person{Fermat})
	\end{enumerate}
\end{remark}
\begin{lemma}
	Sei $\chara(K) = p > 0, a \in K \setminus K^p$. Dann ist $f = X^p -a$ irreduzibel und inseparabel
\end{lemma}
\begin{proof}
	Sei $\alpha \in \bar{K}$ mit $f(\alpha) = 0, g= \MinPol(\alpha \mid K)$\\
	$\implies g \mid f = X^p -a = (X-\alpha)^p$\\
	$\implies g \equiv (X - \alpha)^k \mit k \le p$. $a \notin K^p \implies \alpha \notin K \implies k >1$\\
	$\implies g$ ist inseperabel\\
	$\xRightarrow{g \text{ irred.}} g(x) = h(x^p)$ für ein $h$\\
	$\implies k = p \implies f = g$ irreduzibel 
\end{proof}