\section{Algebraische Körpererweiterungen}

Sei $L \mid K$ eine Körpererweiterung.

\begin{definition}[algebraisch, transzendent]
	Sei $\alpha \in L$. Gibt es ein $0 \neq f \in K$ mit $f(\alpha) = 0$, so heißt $\alpha$ \begriff{algebraisch} über $K$, andernfalls \begriff{transzendent} über $K$.
\end{definition}

\begin{example}
	\begin{enumerate}[label=(\alph*)]
		\item $\alpha \in K \Rightarrow \alpha$ ist algebraisch über $K$ (denn $f(\alpha) = 0$ für $f = X - \alpha \in K$)
		\item $\sqrt{-1} \in \Q(\sqrt{-1})$ ist algebraisch über $\Q$ (denn $f(\sqrt{-1})=0$ für $f = X^2 + 1 \in \Q$) \\
		$\sqrt{-1} \in \C$ ist algebraisch über $\R$        
	\end{enumerate}
\end{example}

\begin{remark}
	\proplbl{1_2_3}
	Sind $K \subseteq L \subseteq M$ Körper und $\alpha \in M$ algebraisch über $K$, so auch über $L$.
\end{remark}

\begin{lemma} 
	\proplbl{1_2_4}
	Genau dann ist $\alpha \in L$ algebraisch über $K$, wenn $1, \alpha, \alpha^2 , \dots$ $K$-linear abhängig sind.
\end{lemma}

\begin{proof}
	Für $\lambda_0 , \lambda_1 , \dots \in K$, fast alle gleich Null, so ist
	\begin{align}
	\sum_{i=0}^\infty \lambda_i \alpha^i :\Leftrightarrow f(\alpha) = 0 \text{ für } f = \sum_{i=0}^\infty \lambda_i X^i \in K\notag
	\end{align}
\end{proof}

\begin{lemma}
	Betrachte den Epimorphismus
	\begin{align}
	\phi_{\alpha}:\begin{cases}
	K[x] &\to K[\alpha]\\
	f &\mapsto f(\alpha).
	\end{cases}\notag
	\end{align}
	Genau dann ist $\alpha$ algebraisch über $K$, wenn $\Ker(\phi_\alpha) \neq (0)$. In diesem Fall ist $\Ker(\phi_\alpha) = (f_\alpha)$ mit einem eindeutig bestimmten irreduziblen, normierten $f_\alpha \in K$.
\end{lemma}

\begin{proof}
	$K$ Hauptidealring $\Rightarrow \Ker(\phi_\alpha) = (f_\alpha)$, $f_\alpha \in K$, o.E. sei $f_{\alpha}$ normiert. Aus $K[\alpha] \subseteq L$ nullteilerfrei folgt, dass $\Ker(\phi_\alpha)$ prim ist. Somit ist $f_\alpha$ prim und im Hauptidealring also auch irreduzibel.
\end{proof}

\begin{definition}[Monimalpolynom, Grad]
	Sei $\alpha \in L$ algebraisch über $K$, $\Ker(\phi_\alpha) = (f_\alpha)$ mit $f_\alpha \in K$ normiert und irreduzibel.
	\begin{enumerate}
		\item $\MinPol(\alpha\mid K) := f_\alpha$, das \begriff{Minimalpolynom} von $\alpha$ über $K$.
		\item $\deg(\alpha\mid K) :\Leftrightarrow \deg(f_\alpha)$, der \begriff{Grad} von $\alpha$ über $K$.
	\end{enumerate}
\end{definition}

\begin{proposition}
	\proplbl{1_2_7}
	Sei $\alpha \in L$.
	\begin{enumerate}
		\item $\alpha$ transzendent über $K$ \\
		$\Rightarrow K[\alpha] \cong K$, $K(\alpha) \cong_K K(X)$, $[K(\alpha) : K] = \infty$.
		\item $\alpha$ algebraisch über $K$ \\
		$\Rightarrow K[\alpha] = K(\alpha) \cong \lnkset{K}{\MinPol(\alpha\mid K)}$ , $[ K(\alpha) \colon K)]  = \deg(\alpha \mid K) < \infty$ und \\
		$1, \alpha, \dots , \alpha^{\deg(\alpha \mid K) -1}$ ist $K$-Basis von $K(\alpha)$. 
	\end{enumerate}
\end{proposition}

\begin{proof}
	\begin{enumerate}[label=(\alph*)]
		\item $\Ker(\phi_\alpha) = (0) \Rightarrow \phi_\alpha$ ist Isomorphismus (da zusätzlich injektiv) \\
		$\Rightarrow K(\alpha) \cong_K \Quot(K[\alpha]) \cong_K \Quot(K) = K(X)$ \\
		$\Rightarrow [K(\alpha) \colon K] = [K(x) \colon K] = \infty$
		\item Sei $f = f_\alpha = \MinPol(\alpha \mid K)$, $n = \deg(\alpha \mid K) = \deg(f)$.
		\begin{itemize}
			\item $f$ irreduzibel $\Rightarrow (f) \neq (0)$ prim ${\xRightarrow{\text{GEO II.4.7}}} (f)$ ist maximal \\
			$\Rightarrow K[\alpha] \cong \lnkset{K}{(f)}$ ist Körper $\Rightarrow K[\alpha] = K(\alpha)$
			\item $1, \alpha, \dots , \alpha^{n-1}$ sind $K$-linear unabhängig: 
			\begin{align}
			\sum_{i=0}^{n-1} \lambda_i \alpha^i = 0 \Rightarrow \sum_{i=0}^{n-1} \lambda_i X^i \in (f) \quad \overset{\deg f = n}{\Longrightarrow} \quad \lambda_i = 0 \enskip \forall i \notag
			\end{align}
			$1, \alpha, \dots , \alpha^{n-1}$ ist Erzeugendensystem: Für $g \in K$ ist 
			\begin{align}
			g = qf + r \text{ mit } q,r \in K \text{ und } \deg(r) < \deg(f) = n \notag
			\end{align}
			und  
			\begin{align}
			g(\alpha) = q(\alpha) \underbrace{f(\alpha)}_{=0} + r(\alpha) = r(\alpha) \notag
			\end{align}
			somit $K = \Image(\phi_\alpha) = \set{g(\alpha) \colon g \in K} = \set{r(\alpha) \colon r \in K, \deg(r) < n} = \sum_{i=0}^{n-1} K \cdot \alpha^i$
		\end{itemize}
	\end{enumerate}
\end{proof}

\begin{example}
	\begin{enumerate}[label=(\alph*)]
		\item $p \in \Z$ prim $\Rightarrow$ $\sqrt{p} \in \C$ ist algebraisch über $\Q$. \\
		Da $f(X) = X^2 - p$ irreduzibel in $\Q$ ist (GEO II.7.3), ist $\MinPol(\sqrt{p}:\Q) = X^2 - p$, $[\Q(\sqrt{p}) : \Q] = 2$.
		\item Sei $\zeta_p = e^{\frac{2\pi i}{p}} \in \C$ ($p \in \N$ prim). Da $\Phi_p =  \frac{X^p-1}{X-1} = X^{p-1} + X^{p-2} + \cdots + X + 1 \in \Q$ irreduzibel in $\Q$ ist (GEO II.7.9), ist $\MinPol(\zeta_p \mid \Q) = \Phi_p$, $[\Q(\zeta_p) : \Q] = p-1$. Daraus folgt schließlich $[\C : \Q \ge [\Q(\zeta_p) : \Q] = p-1 \enskip \forall p \Rightarrow [\C : \Q] = \infty \Rightarrow [R : \Q] = \infty$.
		\item $e \in \R$ ist transzendent über $\Q$ (\person{Hermite} 1873), 
		$\pi \in \R$ ist transendent über $\Q$ (\person{Lindemann} 1882). \\
		Daraus folgt: $[R : \Q] \ge [\Q(\pi): \Q] = \infty$. Jedoch ist unbekannt, ob z.B. $\pi + e$ transzendent ist.
	\end{enumerate}
\end{example}

\begin{definition}
	$L \mid K$ ist \begriff{algebraisch} $:\Leftrightarrow$ jedes $\alpha \in L$ ist algebraisch über $K$.
\end{definition}

\begin{proposition}
	\proplbl{1_2_10}
	$L \mid K$ endlich $\Rightarrow$ $L \mid K$ algebraisch.
\end{proposition}

\begin{proof}
	$\alpha \in L$, $[L : K] = n$ $\Rightarrow 1, \alpha, \dots , \alpha^n$ $K$-linear abhängig $\xRightarrow{\propref{1_2_4}} \alpha$ algebraisch über $K$.
\end{proof}

\begin{conclusion}
	\proplbl{1_2_11}
	Ist $L = K(\alpha_1, \dots, \alpha_n)$ mit $\alpha_1, \dots, \alpha_n$ algebraisch über $K$, so ist $L \mid K$ endlich, insbesondere algebraisch.
\end{conclusion}

\begin{proof}
	Induktion nach $n$:
	\begin{itemize}
		\item $n=0$: \checkmark
		\item $n > 0$: $K_1 :=  K(\alpha_1, \dots, \alpha_{n-1})$ \\
		$\Rightarrow L=K_1(\alpha_n)$, $\alpha_n$ algebraisch über $K_1$ (\propref{1_2_3}) \\
		$\Rightarrow [L : K] = \underbrace{[K_1(\alpha_n) : K_1]}_{< \infty \text{ nach \propref{1_2_7}}}\cdot \underbrace{[K_1 : K]}_{< \infty \text{ nach IH}}$
	\end{itemize}
\end{proof}

%\begin{proof_induction}[$n$]
%	\ianfang[$n=0$] $\checkmark$
%	\ischritt[$n > 0$] 
%\end{proof_induction} 

\begin{conclusion}
	Es sind äquivalent:
	\begin{enumerate}
		\item $L \mid K$ ist endlich.
		\item $L \mid K$ ist endlich erzeugt und algebraisch.
		\item $L = K(\alpha_1, \dots , \alpha_n)$ mit $\alpha_1, \dots, \alpha_n$ algebraisch über $K$.
	\end{enumerate}
\end{conclusion}

\begin{proof}
	\begin{itemize}
		\item (1) $\Rightarrow$ (2): \propref{1_1_15} und \propref{1_2_10}
		\item (2) $\Rightarrow$ (3): trivial
		\item (3) $\Rightarrow$ (1): \propref{1_2_11}
	\end{itemize}
\end{proof}

\begin{remark}
	Nach \propref{1_2_7} ist
	\begin{align}
	\alpha \text{ algebraisch über } K :\Leftrightarrow K[\alpha] = K(\alpha) \notag
	\end{align}
	Direkter Beweis für $(\Rightarrow)$: \\
	Sei $0 \neq \beta \in K[\alpha]$. Daraus folgt, dass $f(\beta) = 0$ für ein irreduzibles $0 \neq f = \sum_{i=0}^n a_i X^i \in K$. Durch Einsetzen von $\beta$ und Division durch $\beta$ erhält man (auch wegen der aus der Irreduzibilität
	\begin{align}
	\xRightarrow{a_0 \neq 0}\beta^{-1} = -a_0^{-1} ( a_1 + a_2 \beta + \dots + a_n \beta^{n-1}) \in K[\beta] \subseteq K[\alpha] \notag
	\end{align}
\end{remark}
% % % % % % % % % % % % % % % % 3rd lecture % % % % % % % % % % % % % % % % % % %
\begin{proposition}
	\proplbl{1_2_14}
	Seien $K \subseteq L \subseteq M$ Körper. Dann gilt:
	\begin{align*}
		M\mid K \text{ algebraisch } \Leftrightarrow M\mid L \text{ algebraisch und } L \mid K \text{ algebraisch }
	\end{align*}
\end{proposition}

\begin{proof}
	\begin{itemize}
		\item[($\Rightarrow$)] klar, siehe \propref{1_2_3}.
		\item[($\Leftarrow$)] Sei $\alpha \in M$. Schreibe $f=\MinPol(\alpha \mid L) = \sum_{i=0}^{n} a_i x^i, \quad L_0 := K(a_0,\dots,a_n)$\\
		$\Rightarrow f \in L_0[x]$\\
		$\Rightarrow [L_0(\alpha): L_0] \le \deg(f) \le \infty$\\
		$\Rightarrow [K(\alpha: K)] \le [K(a_0,\dots,a_n,\alpha):K] = \underbrace{[L_0(\alpha):L_0]}_{< \infty}\underbrace{[L_0 :K]}_{< \text{ nach } \propref{1_2_7}}$\\
		$\Rightarrow \alpha$ abgebraisch über $K$\\
		$\overset{\alpha \text{ bel.}}{\Rightarrow} M \mid K$ algebraisch.
	\end{itemize}
\end{proof}

\begin{conclusion}
	$\tilde{K} = \set{\alpha \in L\colon \alpha \text{ algebraisch über }K}$ ist ein Körper, und ist $\alpha \in L$ algebraisch über $\tilde{K}$, so ist schon $\alpha \in \tilde{K}$.
\end{conclusion}

\begin{proof} %TODO find a good way to format the RIGHTARROWS?
	\begin{itemize}
		\item $\alpha, \beta \in \tilde{K}:\\
			\Rightarrow K(\alpha, \beta)\mid K$ endlich, insbesondere algebraisch\\
			$\Rightarrow \alpha + \beta, \alpha - \beta, \alpha \cdot \beta, \alpha^{-1} \in K(\alpha,\beta)$ alle algebraisch über $K$, also $K(\alpha, \beta) \subseteq \tilde{K}$.
		\item $\alpha \in L$ algebraisch über $\tilde{K}$:\\
			$\Rightarrow \tilde{K}(\alpha)\mid \tilde{K}$ algebraisch\\
			$\Rightarrow \tilde{K}\mid K$ algebraisch
			$\overset{\propref{1_2_14}}{\Rightarrow} \tilde{K}(\alpha\mid K)$ algebraisch, insbesondere $\alpha \in \tilde{K}$.
	\end{itemize}
\end{proof}

\begin{definition}[relative algebraische Abschluss]
	$\tilde{K} = \set{\alpha \in L\colon \alpha \text{ algebraisch über }K}$ heißt der \begriff{relative algebraische Abschluss} von $K$ in $L$.
\end{definition}

\begin{example}
	$\tilde{\Q} = \set{\alpha \in \C \colon \alpha \text{ algebraisch über }K}$ ist ein Körper, der Körper der algebraischen Zahlen. Es ist $[\tilde{\Q},\Q] = \infty$, z.B. da $[\Q(\zeta_p):\Q] = p-1$ für jedes $p$ prim. (algebraische Erweiterung die nicht endlich ist.)
\end{example}