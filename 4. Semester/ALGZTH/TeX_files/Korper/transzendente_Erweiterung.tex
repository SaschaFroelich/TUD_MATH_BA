\section{Die transzendente Erweiterung}
Sei $L\mid K$ eine Körpererweiterung.
\begin{definition}[algebraisch abhängig]
	\begin{enumerate}
		\item $a_1, \dots, a_n \in L$ \begriff{algebraisch abhängig} über $K$ $: \equi $ existiert \\$0 \neq f \in K(X_1,\dots, X_n) \colon f(a_1, \dots, a_n) = 0$
		\item $(a_i)_{i\in I}$ ist \begriff{algebraisch abhängig} über $K$ $:\equi$ existiert $J \subseteq I$ endlich: $(a_i)_{i\in I}$ algebraisch abhängig über $K$
	\end{enumerate}
\end{definition}
\begin{*example}
	Betrachte die reellen Zahlen $\sqrt{\pi} \und 2\pi +1$, beide sind transzendent über $\Q$. Die Singletons $\set{\sqrt{\pi}}\und \set{2\pi +1}$ sind algebraisch unabhängig über $\Q$. Aber die Vereinigung $\set{\sqrt{\pi}, 2\pi +1}$ ist nicht algebraisch unabhängig in $\Q$, da
	\begin{align*}
		P(x,y) = 2x^2 - y + 1 = 0
	\end{align*}
	ist, wenn $x = \sqrt{\pi} \und y = 2\pi +1$ gesetzt sind.
\end{*example}
\begin{remark}
	\begin{enumerate}
		\item $(a)$ ist algebraisch abhängig über $K \equi a$ ist algebraisch über $K$
		\item $L = K(X_1,\dots, X_n) = \Quot(K([X_1,\dots, X_n])) \implies X_1,\dots, X_n$ sind algebraisch unabhängig über $K$
		\item Sind $\pi, e$ unabhängig über $\Q$?\\
		Falls ``Ja'', wäre z.B. $\pi+e$ transzendent über $\Q$
	\end{enumerate}
\end{remark}
\begin{definition}[rein transzendent]
	$L \mid K$ \begriff{rein transzendent} $:\equi L = K(\Halb) \mit \Halb = (a_i)_{i\in I}$ algebraisch unabhängig über $K$.
\end{definition}
\begin{lemma}
	\proplbl{1_5_4}
	$\Halb = (a_i)_{i \in I}$ algebraisch unabhängig über $K \implies K(\Halb) \cong_K K(X_i \colon i \in I) = \Quot(K[X_i \colon i \in I])$. 
\end{lemma}
\begin{proof}
	Betrachte $K$-Isomorphismus
	\begin{align*}
		\varphi = \begin{cases}
			K[X_i \colon I \in I] &\to K[a_i : i \in I]\\
			f & \mapsto f(\Halb)
		\end{cases} 
	\end{align*}
	($a_i$ für $x_i$ einsetzen.) Da $\Halb$ algebraisch unabhängig über $K$, ist $\Ker(\varphi) = (0)$\\
	$\implies K(\Halb) = \Quot(K[\Halb]) \cong_K \Quot(K[X_i : i \in I])$.
\end{proof}
\begin{proposition}
	$L\mid K$ rein transzendent $\implies \tilde{K} = K$.
\end{proposition}
\begin{proof}
	Nach \propref{1_5_4} o.E. $L = K(X_i : i \in I)$. Weiter o. E. $I = \set{1, \dots,n}$ endlich. Sei $\alpha \in L$ algebraisch über $K$. Definiere $f = \MinPol(\alpha \mid K)$.\\
	$f$ irreduzibel in $K[X] \xRightarrow{\text{Gauß}} f$ irreduzibel in $K[X_1, \dots, X_n][X]$\\
	$\xRightarrow{\text{Gauß}} f$ irreduzibel $K(X_1, \dots, X_n)[X]$\\
	$\implies \deg(f) = 1$\\
	$\implies \alpha \in K$.
\end{proof}