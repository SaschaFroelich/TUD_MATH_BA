\documentclass{standalone}

\usepackage{../../../texmf/tex/latex/mathoperators/mathoperators}
\usepackage{tikz, amsmath, amssymb}
	\usepackage{tikz-qtree}
	\usetikzlibrary{cd}
	\usetikzlibrary{arrows}
	\usetikzlibrary{automata}
	\usetikzlibrary{babel}
	\usetikzlibrary{calc}
	\usetikzlibrary{fit}
	\usetikzlibrary{matrix}
	\usetikzlibrary{positioning}
	\usetikzlibrary{shapes.geometric}
	\usetikzlibrary{arrows.meta,bending}

%\newcommand{\Q}{\mathbb{Q}}

\begin{document}
	\begin{tikzcd}
		& {\Q(\sqrt{2},\sqrt{3})} \arrow[dash, ld] \arrow[dash, d] \arrow[dash, rd] &                         \\
		\Q(\sqrt{2}) \arrow[dash, rd] & \Q(\sqrt{3}) \arrow[dash, d]                                  & \Q(\sqrt{6}) \arrow[dash, ld] \\
		& \Q                                                      &                        
	\end{tikzcd}
	\begin{tikzcd}
		& \set{\id} \arrow[ld] \arrow[d] \arrow[rd]   &                                 \\
		{\set{\id,\sigma_1}} \arrow[rd] & {\set{\id, \sigma_3}} \arrow[d]             & {\set{\sigma_4,\id}} \arrow[ld] \\
		& {\set{\sigma_1,\sigma_2,\sigma_3,\sigma_4}} &                                
	\end{tikzcd}
\end{document}
