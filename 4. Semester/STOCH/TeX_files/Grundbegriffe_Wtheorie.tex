\chapter{Grundbegriffe der Wahrscheinlichkeitstheorie}

\section{Wahrscheinlichkeitsräume}

\subsection*{Ergebnisraum}

Welche der möglichen Ausgänge eines zufälligen Geschehens interessieren uns?\\
Würfeln? Augenzahl, nicht die Lage und die Fallhöhe

\begin{definition}[Ergebnisraum]
	Die Menge der relevanten Ergebnisse eines Zufallsgeschehens nennen wir \begriff{Ergebnisraum} und bezeichnen diesen mit $\Omega$.
\end{definition}

\begin{*example}
	\begin{itemize}
		\item Würfeln: $\Omega = \{1,2, \dots, 6\}$
		\item Wartezeiten: $\Omega = \real_{+} = [0, \infty)$ (überabzählbar!)
	\end{itemize}
\end{*example}

\subsection*{Ereignisse}

Oft interessieren wir uns gar nicht für das konkrete Ergenis des Zufallsexperiments, sondern nur für das Eintreten gewisser Ereignisse.
\begin{*example}
	\begin{itemize}
		\item Würfeln: Zahl ist $\ge 3$
		\item Wartezeit: Wartezeit $\le 5$ Minuten
	\end{itemize}
\end{*example}

$\longrightarrow$ Teilmenge des Ereignisraums, also Element der Potenzmenge $\mathscr{P}(\Omega)$, denen eine Wahrscheinlichkeit zugeordnet werden kann, d.h. welche \begriff{messbar} (mb) sind.

\begin{definition}[Ereignisraum, messbarer Raum]
	Sei $\Omega \neq \emptyset$ ein Ergebnisraum und $\mathscr{F}$ eine $\sigma$-Algebra auf $\Omega$, d.h. eine Familie von Teilmenge von $\Omega$, sodass
	\begin{enumerate}
		\item $\Omega \in \mathscr{F}$
		\item $A \in \mathscr{F} \Rightarrow A^C \in \mathscr{F}$
		\item $A_1, A_2, \dots \in \mathscr{F} \Rightarrow \bigcup_{i \ge 1} \in \mathscr{F}$
	\end{enumerate}
	Dann heißt $(\Omega, \mathscr{F})$ \begriff{Ereignisraum} bzw. \begriff{messbarer Raum}.
\end{definition}

\subsection*{Wahrscheinlichkeiten}

Ordne Ereignissen Wahrscheinlichkeiten zu mittels der Abbildung

\begin{align}
	\mathbb{P}: \mathscr{F} \to [0,1]\notag
\end{align}

sodass

\begin{align}
	\text{Normierung } \mathbb{P}(\Omega) = 1 \tag{N}\label{eq_norm}\\
	\sigma\text{-Additivität für paarweise disjunkte Ereignisse} \tag{A}\label{eq_additive}\\
	A_1, A_2, \dots \in \mathscr{F} \Rightarrow \mathbb{P}(\bigcup_{i \ge 1} A_i) = \sum_{1 \ge 1} \mathbb{P}(A_i)\notag
\end{align}

(\ref{eq_norm}), (\ref{eq_additive}) und die Nichtnegativität von $\mathbb{P}$ werden als \begriff{\person{Kolmogorov}sche Axiome} bezeichnet (nach Kolomogorov: Grundbegriffe der Wahrscheinlichkeitstheorie, 1933)
%TODO find out how to have (A) instead of A in the text with ref!!!

\begin{definition}[Wahrscheinlichkeitsmaß, Wahrscheinlichkeitsverteilung]
	Sei $(\Omega, \mathscr{F})$ ein Ereignisraum und $\mathbb{P}: \mathscr{F} \to [0,1]$ eine Abbildung mit Eigenschaften (\ref{eq_norm}) und (\ref{eq_additive}). Dann heißt $\mathbb{P}$ \begriff{Wahrscheinlichkeitsmaß} oder auch \begriff{Wahrscheinlichkeitsverteilung}.
\end{definition}

Aus der Definition folgen direkt:

\begin{proposition}[Rechenregeln für W-Maße]
	Sei $\mathbb{P}$ ein W-Maß, Ereignisse $(\Omega, \mathscr{F}), A, B, A_1, A_2, \dots \in \mathscr{F}$. Dann gelten:
	\begin{enumerate}
		\item $\mathbb{P}(\emptyset) = 0$
		\item Monotonie: $A \subseteq B \Rightarrow \mathbb{P}(A) \le \mathbb{P}(B)$
		\item endliche $\sigma$-Additivität: $\mathbb{P}(A\cup B) + \mathbb{P}(A\cap B) = \mathbb{P}(A) + \mathbb{P}(B)$ und insbesondere $\mathbb{P}(A) + \mathbb{P}(A^C) = 1$
		\item $\sigma$-Subadditivität:
		\begin{align}
			\mathbb{P}\left(\bigcup_{i \ge 1} A_i\right) \le \sum_{1 \ge 1} \mathbb{P}(A_i)\notag
		\end{align}
		\item $\sigma$-Stetigkeit: Wenn $A_n \uparrow A$ (d.h. $A_1 \subseteq A_2 \subseteq \cdots$ und $A = \bigcup_{i=1}^{\infty} (A_i)$) oder $A_n \downarrow A$, so gilt:
		\begin{align}
			\mathbb{P}(A_n) \longrightarrow \mathbb{P}(A), n \to \infty \notag
		\end{align}
	\end{enumerate}
\end{proposition}

\begin{proof}
	Beweise erst folgende Aussage: $A\cap B = \emptyset \Longrightarrow \probp(A \uplus B) = \probp(A) + \probp(B)$.\\
	Es kann $\sigma$-Additivität verwendet werden, indem ``fehlende'' Mengen durch $\emptyset$ ergänzt werden:
	\begin{align}
		\probp(A \uplus B) = \probp(A \uplus B \uplus \emptyset \uplus \emptyset \dots) = \probp(A) + \probp(B) + \probp(\emptyset) + \dots = \probp(A) + \probp(B),\notag
	\end{align}
	wobei Maßeigenschaften verwendet wurden.
	\begin{enumerate}
		\item Definition des Maßes.
		\item Da $A \subseteq B$ ist auch $B = A \uplus (B \setminus A) = A \uplus (B \setminus (A \cap B))$. Wende wieder Aussage von oben an, damit folgt
		\begin{align}
			\probp(B) = \probp(A \uplus (B \setminus A)) = \probp(A) + \probp(B \setminus A) \ge \probp(A) \label{eq_1_1_4}\tag{*}
		\end{align}
		\item Zerlege $A \cup B$ geschickt, dann sieht man mit oben gezeigter Aussage und (\ref{eq_1_1_4})
		\begin{align}
			\probp(A \cup B) + \probp(A \cap B) &= \probp(A \uplus (B \setminus (A \cap B)) + \probp(A \cap B)\notag \\
			&= \probp(A) + \probp(B \setminus (A \cap B)) + \probp(A \cap B)\notag\\
			&= \probp(A)+\probp(B).\notag	
		\end{align}
		Im letzten Schritt wurde (\ref{eq_1_1_4}) verwendet.
		\item Folgt aus endlicher $\sigma$-Additivität, da $\probp\left(\bigcap_{i\ge 1} A_i \right) \ge 0$.
		\item Definiere $F_1 := A_1, F_2 := A_2 \ A_1, \dots, F_{i+1} := A_{i+1}\ A_n$. Die $F_i$ Mengen sind paarweise disjunkt und damit folgt für $m \to \infty$
		\begin{align}
			A_m = \biguplus_{i=1}^{m} F_i \Rightarrow A = \biguplus_{i=1}^{\infty} F_i = \biguplus_{i=1}^{\infty} A_i\notag
		\end{align}
		und
		\begin{align}
			\probp(A) = \probp\left( \biguplus_{i=1}^{\infty} F_i \right) = \sum_{i=1}^{\infty} \probp(F_i) = \lim\limits_{m \to \infty} \probp\left( \biguplus_{i=1}^{m} F_i \right) = \lim\limits_{m\to \infty} \probp(A_m). \notag
		\end{align}
	\end{enumerate}
	Siehe Schillings Buch. %TODO set literature link to literature??!?!?!
\end{proof}

\begin{example}
	Für ein beliebigen Ereignisraum $(\Omega, \mathscr{F})$ ($\Omega \neq \emptyset$) und eine beliebiges Element $\xi \in \Omega$ definiere
	\begin{align}
		\delta_{\xi}(A := \begin{cases}
		1 & \xi \in A \\
		0 & \text{ sonst}
		\end{cases}\notag
	\end{align}
	eine (degeneriertes) W-Maß auf $(\Omega, \mathscr{F})$, welches wir als \begriff{\person{Dirac}-Maß} oder \begriff{\person{Dirac}-Verteilung} bezeichnen.
\end{example}

\begin{example}
	Würfeln mit fairem, $6$-(gleich)seitigem Würfel mit Ergebnismenge $\Omega=\{1, \dots, 6\}$ und Ereignisraum $\mathscr{F} = \mathscr{P}(\Omega)$ setzen wir als Symmetriegründen
	\begin{align}
		\mathbb{P}(A) = \frac{\# A}{6}.\notag
	\end{align}
	(Wobei $\# A$ oder auch $\vert A \vert$ die Kardinalität von $A$ ist.) Das definiert ein W-Maß.
\end{example}

\begin{example}
	Wartezeit an der Bushaltestelle mit Ergebnisraum $\Omega = \real_{+}$ und Ereignisraum \person{Borel}sche $\sigma$-Algebra $\mathscr{B}(\real_{+}) = \mathscr{F}$. Eine mögliches W-Maß können wir dann durch
	\begin{align}
	\mathbb{P}(A) = \int_{A} \lambda e^{-\lambda x} dx\notag %TODO set a mathoperator for dx!!!
	\end{align}
	für einen Parameter $\lambda > 0$ festlegen. (Offenbar gilt $\mathbb{P}(\Omega) = 1$ und die $\sigma$-Additivität aufgrund der Additivität des Integrals.) Wir bezeichnen diese Maß als \begriff{Exponentialverteilung}. (Warum gerade dieses Maß für Wartezeiten gut geeignet ist $\nearrow$ später) %TODO add later a ref!!!
\end{example} 