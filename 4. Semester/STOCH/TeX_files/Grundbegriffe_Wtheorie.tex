\chapter{Grundbegriffe der Wahrscheinlichkeitstheorie}

\section{Wahrscheinlichkeitsräume}

\subsection*{Ergebnisraum}

Welche der möglichen Ausgänge eines zufälligen Geschehens interessieren uns?\\
Würfeln? Augenzahl, nicht die Lage und die Fallhöhe

\begin{definition}[Ergebnisraum]
	Die Menge der relevanten Ergebnisse eines Zufallsgeschehens nennen wir \begriff{Ergebnisraum} und bezeichnen diesen mit $\Omega$.
\end{definition}

\begin{*example}
	\begin{itemize}
		\item Würfeln: $\Omega = \{1,2, \dots, 6\}$
		\item Wartezeiten: $\Omega = \real_{+} = [0, \infty)$ (überabzählbar!)
	\end{itemize}
\end{*example}

\subsection*{Ereignisse}

Oft interessieren wir uns gar nicht für das konkrete Ergenis des Zufallsexperiments, sondern nur für das Eintreten gewisser Ereignisse.
\begin{*example}
	\begin{itemize}
		\item Würfeln: Zahl ist $\ge 3$
		\item Wartezeit: Wartezeit $\le 5$ Minuten
	\end{itemize}
\end{*example}

$\longrightarrow$ Teilmenge des Ereignisraums, also Element der Potenzmenge $\mathscr{P}(\Omega)$, denen eine Wahrscheinlichkeit zugeordnet werden kann, d.h. welche \begriff{messbar} (mb) sind.

\begin{definition}[Ereignisraum, messbarer Raum]
	Sei $\Omega \neq \emptyset$ ein Ergebnisraum und $\mathscr{F}$ eine $\sigma$-Algebra auf $\Omega$, d.h. eine Familie von Teilmenge von $\Omega$, sodass
	\begin{enumerate}
		\item $\Omega \in \mathscr{F}$
		\item $A \in \mathscr{F} \Rightarrow A^C \in \mathscr{F}$
		\item $A_1, A_2, \dots \in \mathscr{F} \Rightarrow \bigcup_{i \ge 1} \in \mathscr{F}$
	\end{enumerate}
	Dann heißt $(\Omega, \mathscr{F})$ \begriff{Ereignisraum} bzw. \begriff{messbarer Raum}.
\end{definition}

\subsection*{Wahrscheinlichkeiten}

Ordne Ereignissen Wahrscheinlichkeiten zu mittels der Abbildung

\begin{align}
	\mathbb{P}: \mathscr{F} \to [0,1]\notag
\end{align}

sodass

\begin{align}
	\text{Normierung } \mathbb{P}(\Omega) = 1 \tag{N}\label{eq_norm}\\
	\sigma\text{-Additivität für paarweise disjunkte Ereignisse} \tag{A}\label{eq_addivtive}\\
	A_1, A_2, \dots \in \mathscr{F} \Rightarrow \mathbb{P}(\bigcup_{i \ge 1} A_i) = \sum_{1 \ge 1} \mathbb{P}(A_i)\notag
\end{align}

\ref{eq_norm}, \ref{eq_addivtive} und die Nichtnegativität von $\mathbb{P}$ werden als \begriff{\person{Kolmogorov}sche Axiome} bezeichnet (nach Kolomogorov: Grundbegriffe der Wahrscheinlichkeitstheorie, 1933)
%TODO find out how to have (A) instead of A in the text with ref!!!

\begin{definition}[Wahrscheinlichkeitsmaß, Wahrscheinlichkeitsverteilung]
	Sei $(\Omega, \mathscr{F})$ ein Ereignisraum und $\mathbb{P}: \mathscr{F} \to [0,1]$ eine Abbildung mit Eigenschaften \ref{eq_norm} und \ref{eq_addivtive}. Dann heißt $\mathbb{P}$ \begriff{Wahrscheinlichkeitsmaß} oder auch \begriff{Wahrscheinlichkeitsverteilung}.
\end{definition}

Aus der Definition folgen direkt:

\begin{proposition}[Rechenregeln für W-Maße]
	Sei $\mathbb{P}$ ein W-Maß, Ereignisse $(\Omega, \mathscr{F}), A, B, A_1, A_2, \dots \in \mathscr{F}$. Dann gelten:
	\begin{enumerate}
		\item $\mathbb{P}(\emptyset) = 0$
		\item endliche $\sigma$-Additivität: $\mathbb{P}(A\cup B) + \mathbb{P}(A\cap B) = \mathbb{P}(A) + \mathbb{P}(B)$ und insbesondere $\mathbb{P}(A) + \mathbb{P}(A^C) = 1$
		\item Monotonie: $A \subseteq B \Rightarrow \mathbb{P}(A) \le \mathbb{P}(B)$
		\item $\sigma$-Subadditivität:
		\begin{align}
			\mathbb{P}(\bigcup_{i \ge 1} A_i) \le \sum_{1 \ge 1} \mathbb{P}(A_i)\notag
		\end{align}
		\item $\sigma$-Stetigkeit: Wenn $A_n \uparrow A$ (d.h. $A_1 \subseteq A_2 \subseteq \cdots$ und $A = \bigcup_{i=1}^{\infty} (A_i)$) oder $A_n \downarrow A$, so gilt:
		\begin{align}
			\mathbb{P}(A_n) \longrightarrow \mathbb{P}(A), n \to \infty \notag
		\end{align}
	\end{enumerate}
\end{proposition}

\begin{proof}
	MINT bzw Schillings Buch Kapitel 3! %TODO set literature link to literature??!?!?!
\end{proof}

\begin{example}
	Für ein beliebigen Ereignisraum $(\Omega, \mathscr{F})$ ($\Omega \neq \emptyset$) und eine beliebiges Element $\xi \in \Omega$ definiere
	\begin{align}
		\delta_{\xi}(A := \begin{cases}
		1 & \xi \in A \\
		0 & \text{ sonst}
		\end{cases}\notag
	\end{align}
	eine (degeneriertes) W-Maß auf $(\Omega, \mathscr{F})$, welches wir als \begriff{\person{Dirac}-Maß} oder \begriff{\person{Dirac}-Verteilung} bezeichnen.
\end{example}

\begin{example}
	Würfeln mit fairem, $6$-(gleich)seitigem Würfel mit Ergebnismenge $\Omega=\{1, \dots, 6\}$ und Ereignisraum $\mathscr{F} = \mathscr{P}(\Omega)$ setzen wir als Symmetriegründen
	\begin{align}
		\mathbb{P}(A) = \frac{\# A}{6}.\notag
	\end{align}
	(Wobei $\# A$ oder auch $\vert A \vert$ die Kardinalität von $A$ ist.) Das definiert ein W-Maß.
\end{example}

\begin{example}
	Wartezeit an der Bushaltestelle mit Ergebnisraum $\Omega = \real_{+}$ und Ereignisraum \person{Borel}sche $\sigma$-Algebra $\mathscr{B}(\real_{+}) = \mathscr{F}$. Eine mögliches W-Maß können wir dann durch
	\begin{align}
	\mathbb{P}(A) = \int_{A} \lambda e^{-\lambda x} dx\notag %TODO set a mathoperator for dx!!!
	\end{align}
	für einen Parameter $\lambda > 0$ festlegen. (Offenbar gilt $\mathbb{P}(\Omega) = 1$ und die $\sigma$-Additivität aufgrund der Additivität des Integrals.) Wir bezeichnen diese Maß als \begriff{Exponentialverteilung}. (Warum gerade dieses Maß für Wartezeiten gut geeignet ist $\nearrow$ später) %TODO add later a ref!!!
\end{example} 