\begin{enumerate}
	\item \ul{Frage:} Beispiel \propref{4_5} Durchschnittliche Wartezeit? $\to$ Erwartungswert\\
	Wie stark ist die Streuung um den Durchschnitt? $\to$ Varianz
\end{enumerate}
\section{Der Erwartungswert}
\begin{definition}[title]
	Sei $(\O, \F, \P)$ Wahrscheinlichkeitsraum und $X: (\O, \F) \to (\R, \borel(\R))$ Zufallsvariable. Dann ist
	\begin{align*}
		\ew[X] = \int_{\O} X(\omega) \P(\d \omega) = \int_{\R} x \P(X  \in \d x)
	\end{align*}
	der \begriff{Erwartungswert von $X$}.
\end{definition}
\begin{*hint}
	Der Erwartungswert von $X$ existiert, genau dann wenn
	\begin{align*}
		\int_{\O} \abs{X(\omega)}\P(\d \omega) < \infty \bzw \ew[\abs{X}] < \infty
	\end{align*}
	d.h. genau dann wenn $X \in \Ln{1} (\P)$.\\
	Für nichtnegative Zufallsvariablen ist der Erwartungswert immer definiert, wenn wir $+\infty$ als zulässigen Wert annehmen, was wir in der Folge auch tun.
\end{*hint}
\begin{example}
	$(\O, \F, \P)$ Wahrscheinlichkeitsraum, $A \in \F$ und sei $Y: (\O, \F) \to (\R, \borel(\R))$ die Indikatorvariable
	\[
		X(\omega) = \indi_{A} (\omega)
	\]
	Dann gilt: $X \in \Ln{1}(\P)$ und
	\begin{align*}
		\ew[X] = \int_{\O} \indi_{A} (\omega) \P(\d \omega) = \int_{A} \P(\d \omega) = \P(A).
	\end{align*}
\end{example}
\begin{proposition}
	Sei $X: (\O, \F) \to (\Rn, \borel(\Rn)$ Zufallsvariable und
	\begin{align*}
		f: \Rn \to \R \text{ Borel-messbar}.\\
		\intertext{Dann}
		\ew[f(X)] = \int f(X)\d \P = \int_{\O} \ew(X(\omega)) \d \P(\omega) = \int_{\Rn} f(X) \P(X \in \d x). 
	\end{align*}
\end{proposition}
\begin{proof}
	Sei $f(X)$ eine reelle Zufallsvariable. Die Formel folgt direkt auf dem Transformationssatz für Bildmaße ($\nearrow$ Schilling MINT 18.1).
\end{proof}
\begin{proposition}[Erwartungswerte bei Existenz einer (Zähl-)dichte]
	Sei $X: (\O, \F) \to (\Rn, \borel(\Rn))$ Zufallsvariable und
	\begin{align*}
		f: \Rn \to \R \text{ Borel-messbar}.
	\end{align*}
	\begin{enumerate}
		\item \ul{diskrete Fall:} Ist $\P_X$ ein Wahrscheinlichkeitsmaß auf $(\Z, \P(\Z))$ und der Zähldichte $\rho$, so
		\begin{align*}
			\ew[f(X)] = \sum_{x \in \Z} f(x)\rho(x).
		\end{align*}
		\item \ul{stetiger Fall:} Besitzt $\P_X$ eine Dichte $\rho$ (bzgl Lebesguemaß), so
		\begin{align*}
			\ew[f(X)] = \int_{\R} f(x)\rho(x) \d x
		\end{align*}
	\end{enumerate}
\end{proposition}
\begin{proof}
	Klar aus Def. 5.1 und Prop. 5.3. %TODO ref
\end{proof}
\begin{example}
	Sei $X \sim \Bin(n,p)$. dann gilt
	\begin{align*}
		\ew[f(X)] &= \sum_{k=1}^n k(\binom{n}{k})p^k (1-p)^{n-k}\\
		&= \sum_{k=1}^n \frac{n!}{(n-k)!(k-1)!}p^k (1-p)^{n-k}\\
		&= np \sum_{k=1}^n \underbrace{\binom{n-1}{k-1}p^{k-1}(1-p)^{n-1-(k-1)}}_{\text{Zähldichte }\Bin(n-1,p) in k-1}\\
		&= np
	\end{align*}
\end{example}