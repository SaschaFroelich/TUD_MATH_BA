\section{(Un)abhängigkeit} \label{sec_unabhangigkeit}
In vielen Fällen besagt die Intuition über verschiedene Zufallsexperimente / Ereignisse, dass diese sich \emph{nicht} gegenseitig beeinflussen. Für solche $A,B \in \F$ mit $\P(A) > 0, \P(B) > 0$ sollte gelten
\begin{align*}
\P(A\mid B) = \P(A), \quad \P(B\mid A) = \P(B).
\end{align*}

\begin{definition}[(stochastisch) unabhängig]
	\proplbl{3_2_11}
	Sei $(\O, \F, \P)$ Wahrscheinlichkeitsraum. Zwei Ereignisse $A,B \in \F$ heißt \begriff{(stochastisch) unabhängig bezüglich $\P$}, falls
	\begin{align*}
		\P(A\cap B) = \P(A)\P(B)
	\end{align*}
	Wir schreiben auch $A \upmodels B$.
\end{definition}
\begin{example}
	Würfeln mit 2 fairen, sechsseitigen Würfeln:
	\begin{align*}
	\O &= \set{(i,j) \mid i,j \in\set{1,\dots,n}},\quad \F = \pows(\O), \quad \P = \Gleich(\O)
	\intertext{Betrachte}
	A&:= \set{(i,j) \in \O, i \text{ gerade}}\\
	B&:= \set{(i,j) \in \O, j \le 2}
	\end{align*}
	In diesem Fall, erwarten wir intuitiv Unabhängigkeit von $A$ und $B$.\\
	In der Tat ist % start using \P instead of \P!
	\begin{align*}
	\P(A) &= \frac{1}{2}, \quad \P(B) = \frac{1}{3} \und \P(A\cap B) = \frac{1}{6}
	\intertext{was}
	\P(A \cap B) = \P(A) \P(B)
	\intertext{erfüllt}
	\end{align*}
	Betrachte nun
	\begin{align*}
	C&:= \set{(i,j) \in \O \mid i\neq j = 1}\\
	D&:= \set{(i,j) \in \O \mid i = 6}
	\intertext{dann gilt}
	\P(C) = \frac{1}{6}, \quad \P(D) = \frac{1}{6}
	\intertext{und wegen $C \cap D = \set{(6,1)}$ folgt}
	\P(C\cap D) &= \frac{1}{36} = \frac{1}{6} \frac{1}{6} = \P(C \setminus D)
	\end{align*}
	$C$ und $D$ sind also \emph{stochastisch} unabhängig, obwohl eine kausale Abhängigkeit vorliegt!
\end{example}

\begin{definition}[unabhängig bezüglich $\P$]
	Sei $(\O, \F, \P)$ Wahrscheinlichkeitsraum und $I \neq \emptyset$ endliche Indexmenge. Dann heißt die Familie $(A_i)_{i \in I}$ von Ereignissen in $\F$ \begriff{unabhängig bezüglich $\P$}, falls für alle $J \subseteq I, J \neq \emptyset$ gilt:
	\begin{align*}
		\P\brackets{\bigcap_{i\in J}A_i} = \prod_{i\in J} \P(A_i)
	\end{align*}
	Offensichtlich impliziert die Unabhängigkeit einer Familie die paarweise Unabhängigkeit je zweier Familienmitglieder nach \propref{3_2_11}. Umgekehrt gilt dies nicht!
\end{definition}

\begin{example}[Abhängigkeit trotz paarweiser Unabhängigkeit]
	Betrachte zweifaches Bernoulliexperiment mit Erfolgswahrscheinlichkeit $\sfrac{1}{2}$, d.h.
	\begin{align*}
		\O = \set{0,1}^2, \quad \F = \pows(\O), \quad \P = \Gleich(\O)
		\intertext{sowie}
		A &= \set{1}\times \set{0,1} \qquad \text{(Münzwurf: erster Wurf ist Zahl)}\\
		B &= \set{0,1}\times \set{1} \qquad \text{(Münzwurf: zweiter Wurf ist Zahl)}\\
		C &= \set{(0,0), (1,1)} \qquad \text{(beide Würfe haben selbes Ergebnis)}
	\end{align*}
	Dann gelten
	\begin{align*}
		\P(A) = \frac{1}{2} = \P(B) = \P(C)
		\intertext{und}
		\P(A\cap B) = \P(\set{(1,1)}) = \frac{1}{4} = \P(A)\P(B)\\
		\P(A\cap C) = \P(\set{(1,1)}) = \frac{1}{4} = \P(A)\P(C)\\
		\P(B\cap C) = \P(\set{(1,1)}) = \frac{1}{4} = \P(B)\P(C)
	\end{align*}
	also paarweise Unabhängigkeit.\\
	Aber
	\begin{align*}
	\P(A\cap B \cap C) = \P(\set{(1,1)}) = \frac{1}{4} \neq \P(A)\P(B)\P(C)
	\end{align*}
	und $A,B,C$ sind \emph{nicht} stochastisch unabhängig.
\end{example}

\begin{definition}[Unabhängige $\sigma$-Algebren]
	\proplbl{3_2_15}
	% started using \O for \Omega and \E for this special generating set E_i
	Seien $(\O, \F,\P)$ Wahrscheinlichkeitsraum, $I \neq \emptyset$ Indexmenge und $(E_i, \E_i)$ Messräume
	\begin{enumerate}
		\item Die Familie $\F_i \subset \F, i \in I$, heißen \begriff{unabhängig}, wenn für die $J \subseteq I, I \neq \emptyset, \abs{J} < \infty$ gilt
		\begin{align*}
			\P\brackets{\bigcap_{i \in J} A_i} = \prod_{i\in J} \P(A_i) \qquad \text{ für beliebige } A_i \in \F_i, i \in J
		\end{align*}
		\item Die Zufallsvariable $X_i: (\O, \F) \to (E_i, \E_i), i \in I$, heißen \begriff{unabhängig}, wenn die $\sigma$-Algebren
		\begin{align*}
		\sigma(X_i) = X^{-1}(\E_i) = \set{\set{X_i \in F} \colon F \in \E_i}, \quad i \in I
		\end{align*}
		unabhängig sind.
	\end{enumerate}
\end{definition}

\begin{lemma}[Zusammenhang der Definitionen]
	\proplbl{3_2_16}
	Sei $(\O,\F,\P)$ Wahrscheinlichkeitsraum, $I \neq \emptyset, A \in \F, i \in I$. 
	Die folgenden Aussagen sind äquivalent:
	\begin{enumerate}
		\item Die Ereignisse $A_i, i \in I$ sind unabhängig. 
		\item Die $\sigma$-Algebren $\sigma(A_i), i \in I$ sind unabhängig.
		\item Die Zufallsvariablen $\indi_{A_i}, i \in I$ sind unabhängig.
	\end{enumerate}
\end{lemma}
\begin{proof} %TODO add ref?
	Da die Unabhängigkeit über endliche Teilemengen definiert ist, können wir oBdA $I = \set{1, \dots, n}$ annehmen. 
	\begin{itemize}
		\item Da $\sigma(\indi_{A_i}) = \sigma(A_i)$ folgt die Äquivalenz von 2. und 3. direkt aus \propref{3_2_15}.
		\item Zudem ist 2. $\to$ 1. klar!
		\item Für 1 $\to$ 2. genügt es zu zeigen, dass
		\begin{align*}
			A_1, \dots, A_n \text{ unabhängig } &\Rightarrow B_1, \dots, B_n \text{ unabhängig mit } B_i \in \set{\emptyset, A_i, A_i^C, \O}.
			\intertext{Rekursiv folgt dies bereits aus}
			A_1,\dots, A_n \text{ unabhängig } &\Rightarrow B_1, A_2, \dots, A_n \text{ unabhängig mit } B_1 \in \set{\emptyset, A_1, A_1^C, \O}.
		\end{align*}
		Für $B_1 \in \set{\emptyset, A_1, \O}$ ist dies klar.\\
		Sei also $B_1 = A_1^C$ und $J \subseteq I, J \neq \emptyset$. Falls $1 \not \in J$, ist nichts zu zeigen. Sei $1 \in J$, dann gilt mit
		\begin{align*}
			A &= \bigcap_{i\in J, i \neq 1} A_i
			\intertext{sicherlich}
			\P\brackets{A_1^C \cap A} &= \P(A \setminus (A_1 \cap A))\\
			&= \P(A) - \P(A_1 \cap A)\\
			&= \prod_{i\in J\setminus \set{1}} \P(A_i) - \prod_{i\in J}(A_i)\\
			&= (1- \P(A_1))\prod_{i\in J\setminus \set{1}} \P(A_i)\\
			&= \P\brackets{A_1^C})\prod_{i\in J\setminus \set{1}} \P(A_i)
		\end{align*} 
	\end{itemize}
\end{proof}

Insbesondere zeigt \propref{3_2_16}, dass wir in einer Familie unabhängiger Ereignisse beliebig viele Ereignisse durch ihr Komplement, $\emptyset$ oder $\O$ ersetzen können, ohne die Unabhängigkeit zu verlieren.

\begin{proposition}
	\proplbl{3_2_17}
	Sei $(\O, \F, \P)$ Wahrscheinlichkeitsraum und $\F_i \subseteq \F, i \in I$, seien $\cap$-stabile Familien von Ereignissen. Dann gilt
	\begin{align*}
	\F_i, i \in I \text{ unabhängig } \Leftrightarrow \sigma(\F_i), i \in I \text{ unabhängig}. 
	\end{align*}
\end{proposition}

\begin{proof}
	oBdA sei $I = \set{1, \dots, n}$ und $\O \in \F_i, i \in I$.
	\begin{itemize}
		\item $\Leftarrow$: trivial, da $\F_i \subseteq \sigma(\F_i)$ und das Weglassen von Mengen erlaubt ist.
		\item $\Rightarrow$: zeigen wir rekursiv
		\begin{enumerate}
			\item Wähle $F_i \in \F_i, i = 2, \dots,n$ und defniere für $F \in \sigma(\F_i)$ die endlichen Maße
			\begin{align*}
				\mu(F) = \P\brackets{ F \cap F_2 \cap \cdots \cap F_n} \und \nu(F) = \P(F) \, \P(F_2) \, \dots \, \P(F_n)
			\end{align*}
			\item Da die Familien $\F_i$ unabhängig sind, gilt
			$\mu\mid_{\F_1} = \nu\mid_{\F_1}$.
			Nach dem Eindeutigkeitssatz für Maße (\proplbl{1_1_19}) folgt $\mu\mid_{\sigma(\F_1)} = \nu\mid_{\sigma(\F_1)}$ also
			\begin{align*}
				\P\brackets{ F \cap F_2 \cap \cdots \cap F_n} = \P(F) \P(F_2) \dots \P(F_n)
			\end{align*}
			für alle $F \in \sigma(\F_i)$ und $F_i \in \F_i, i = 1, \dots, n$. Da $\O \in \F_i$ für alle $i$ gilt die erhaltene Produktformel auf für alle Teilemengen $J \subseteq I$.\\
			Also sind
			\begin{align*}
			\sigma(\F_1), \F_2, \dots, \F_n \text{unabhängig}
			\end{align*}
			\item Wiederholtes Anwenden von $1$ und $2$ liefert den Satz.
		\end{enumerate}
	\end{itemize}
\end{proof}

Mit \propref{3_2_17} folgen:

\begin{conclusion}
	Sei $(\O,\F,\P)$ Wahrscheinlichkeitsraum und
	\begin{align*}
		\F_{i,j} \subseteq \F, \quad 1 \le i \le n, 1 \le j \le m(i)
	\end{align*}
	unabhängige, $\cap$-stabile Familien.
	Dann sind auch
	\begin{align*}
		\G_i = \sigma(\F_{i,1}, \dots , \F_{i,m(i)}), \quad 1 \le i \le n
	\end{align*}
	unabhängig.
\end{conclusion}

\begin{conclusion}
	Sei $(\O,\F,\P)$ Wahrscheinlichkeitsraum und
	\begin{align*}
		X_{ij}: \O \to E, \quad 1 \le i \le n, 1 \le j \le m(i)
	\end{align*}
	unabhängige Zufallsvariablen. Zudem seinen $f_i: E^{m(i)} \to \R$ messbar. Dann sind auch die Zufallsvariablen
	\begin{align*}
		f_i(X_{i,1}, \dots, X_{i,m(i)}), \quad 1 \le i \le n
	\end{align*}
	unabhängig.
\end{conclusion}

\begin{example}
	$X_1, \dots, X_n$ unabhängige reelle Zufallsvariablen. Dann sind auch
	\begin{align*}
	Y_1 = X_1, Y_2 = X_2 + \cdots + X_n
	\end{align*}
	unabhängig.
\end{example}