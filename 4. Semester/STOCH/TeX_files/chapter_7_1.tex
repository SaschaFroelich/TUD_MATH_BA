\begin{definition}
	Sei $\mu \in \R, \sigma^2 > 0$. Das Wahrscheinlichkeitsmaß $\normal(\mu, \sigma^2)$ auf $(\R, \borel(\R))$ mit Dichtefunktion
	\begin{align*}
		g_{\mu,\sigma^2}(x) := \frac{1}{\sqrt{2\pi\sigma^2}}e^{-\frac{(x-\mu)^2}{2\sigma^2}} \quad x \in \R
	\end{align*}
	heißt \begriff{Normalverteilung mit Parametern $\mu \und \sigma^2$}.
	Im Fall $\mu = 0,\sigma^2 = 1$, so heißt $\normal(0,1)$ \\ \begriff{Standardnormalverteilung}.
	%TODO pic, check eric!
\end{definition}
\begin{*remark}
	\begin{enumerate}
		\item $g_{\mu, \sigma^2}$ ist Wahrscheinlichkeitsdichte, da $\int_{\R} e^{-\frac{x^2}{2}}\d x = \sqrt{2\pi}$.
		\item $\mu \und \sigma^2$ sind gerade \emph{Erwartungswert} und \emph{Varianz} der Normalverteilung $\normal(\mu, \sigma^2)$. Sei $X \sim \normal(\mu, \sigma^2)$
		\begin{align*}
		\E[X] &= \frac{1}{\sqrt{2\pi \sigma^2}} \int_{\R} x e^{-\frac{(x-\mu)^2}{2\sigma^2}} \d x\quad y = \frac{x-\mu}{\sqrt{\sigma^2}}\\
		&= \frac{1}{\sqrt{2\pi}} \int_{\R} (\sqrt{\sigma^2}y + \mu)e^{-\frac{y^2}{2}} \d y\\
		&= \sqrt{\frac{\sigma^2}{2\pi}} \underbrace{\int_{\R} y e^{-\frac{y^2}{2}}\d y}_{=0, \text{ Symmetrie}} + \frac{1}{\sqrt{2\pi}}\mu \underbrace{\int_{\R} e^{-\frac{y^2}{2}}\d y}_{\sqrt{2\pi}} = \mu\\
		\Var(X) 
		&= \frac{1}{\sqrt{2\pi \sigma^2}} \int_{\R} (x-\mu)^2 e^{-\frac{(x-\mu)^2}{2\sigma^2}} \d x \quad y= \frac{x-\mu}{\sqrt{\sigma^2}}\\ 
		&= \frac{\sigma^2}{\sqrt{2\pi}} \int_{\R} y^2 e^{-\frac{y^2}{2}} \d y\\
		&= \frac{\sigma^2}{\sqrt{2\pi}} \int_{\R} (-y)(-y e^{-\frac{y^2}{2}})\d y\\
		\overset{\text{P.I.}}&{=} \frac{\sigma^2}{\sqrt{2\pi}} \brackets{\underbrace{\sqbrackets{-y e^{-\frac{y^2}{2}}}_{-\infty}^{\infty}}_{=0} + 
		\underbrace{\int_{\R}e^{-\frac{y^2}{2}} \d y}_{= \sqrt{2\pi}}}\\
		&= \sigma^2
		\end{align*}
	\end{enumerate}
\end{*remark}
Die Popularität der Normalverteilung erklärt sich insbesondere aus dem zentralen Grenzwertsatz. Hier eine einfache Form davon
\begin{proposition}[de \person{Moivre}-\person{Laplace}, lokaler Grenzwertsatz]
	Seien $p \in (0,1), B_{n,p}(k) = \binom{n}{k}p^k(1-p)^{n-k}$ Zähldichte $\Bin(n,p), g(x)$ Dichte der Standardnormalverteilung. Dann gilt für $c>0$ beliebig mit
	\[
		x_n(k) = \frac{k-np}{\sqrt{np(1-p)}}
	\]
	\begin{align*}
		\lim_{n\to \infty} \max_{k:\abs{x_n(k)} < c} \abs{\frac{\sqrt{np(1-p)}\cdot B_{n,p}(k)}{g(x_n(k))} -1} = 0
	\end{align*}
\end{proposition}
\begin{proof}
	$\nearrow$ pdf im Opal, bzw. später als Korollar des zentralen Grenzwertsatzes.
\end{proof}