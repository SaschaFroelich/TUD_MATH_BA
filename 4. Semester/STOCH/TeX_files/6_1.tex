In \cref{chap3}: $(\O, \F, \P), A,B \in \F$ %TODO add reference!
\begin{align*}
	\P(A \mid B) = \begin{cases}
	\frac{\P(A \cap B)}{\P(B)} &\quad \P(B) > 0\\
	0/ \text{ beliebig} &\quad \P(B) = 0
	\end{cases}
\end{align*}
In Fall $\P(B) > 0$ ist $\P(\ast\mid B)$ ein Wahrscheinlichkeitsmaß und wir können das Integral 
\[
	\E[X \mid B] := \int X(\omega) \P(\d \omega \mid B)
\]
definieren. Wir bezeichnen die als \begriff{bedingten Erwartungswert} von $X$. Für $X= \indi_{A}$ folgt (für $\P(B) > 0$)
\[
	\int X(\omega) \P(\d \omega \mid B) = \P (A \mid B) = \frac{\P(A \cap B)}{\P(B)} = \frac{\E[\indi_{A\cap B}]}{\P(B)} = \frac{\E[X \indi_{B}]}{\P(B)}
\]
und mittels maßtheoretischer Induktion folgt
\begin{align*}
	\E[X \mid B] = \frac{\E[X \indi_B]}{\P(B)}
\end{align*}
allgemein ($X \in \Ln{1}, \P(B) > 0$)
\begin{enumerate}[label=]
	\item \ul{Frage:} (Wie) können wir bedingte Erwartungswerte definieren, wenn $\P(B) = 0$?
\end{enumerate}