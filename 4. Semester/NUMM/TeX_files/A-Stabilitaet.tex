\section{A-Stabilität}

Wir betrachten die Test-AWA
\begin{align}
	\label{3_29}
	y'=\lambda y\quad\mit\quad y(0)=1
\end{align}
wobei $\lambda\in\comp$ ein Parameter ist. Die eindeutige Lösung dieser Aufgabe ist gegeben durch $y(x)=\exp(\lambda x)$ und es gilt insbesondere
\begin{align}
	\Re(\lambda)&< 0 \quad\Rightarrow\quad \abs{y(x)}\to 0 \quad\text{für } x\to\infty \notag \\
	\Re(\lambda)&=0 \quad\Rightarrow\quad\abs{y(x)}=1 \quad\text{für alle } x\in[0,\infty) \notag
\end{align}

\begin{definition}[A-Stabilität]
	Ein Verfahren erzeuge zu einem beliebigen Paar $(h,\lambda)\in(0,\infty)\times\comp$ eine Folge $\{y_k\}$. Dann heißt das Verfahren \begriff{A-stabil}, wenn
	\begin{align}
		\abs{y_{k+1}} \le \abs{y_k}\quad\forall k\in\natur\notag
	\end{align}
	für jedes $(h,\lambda)\in (0,\infty)\times\comp$ mit $\Re(\lambda)\le 0$.
\end{definition}

Bei ESV gilt $y_{k+1}=y_k + h\Phi(x_k,y_k,y_{k+1},h)$. Wir nehmen an, dass für $f(x,y)=\lambda y$ eine Darstellung des ESV in der Form
\begin{align}
	y_{k+1} = g(h\lambda)y_k \notag
\end{align}
mit einer Funktion $g$: $\comp\to\comp$ existiert. Die Funktion $g$ heißt dann auch \begriff{Stabilitätsfunktion}. Falls der \begriff{Stabilitätsbereich} (Bereich der absoluten Stabilität)
\begin{align}
	\mathcal{S} = \{z\in\comp\mid \abs{g(z)}\le 1\}\notag
\end{align}
die Halbebene $\comp_-=\{z\in\comp\mid \Re(z)\le 0\}$ enthält, dann ist das ESV A-stabil (und umgekehrt), denn es gilt $\abs{y_{k+1}} = \abs{g(h\lambda)}\abs{y_k}\le \abs{y_k}$ für $k\in\natur$ und beliebige $(h,\lambda)\in (0,\infty)\times\comp$ mit $h\lambda\in\comp_-$. Für die \begriff{Trapenzregel} (ein implizites ESV)
\begin{align}
	y_{k+1} = y_k + \frac{h}{2}\bigg(f(x_k,y_k) + f(x_{k+1},y_{k+1})\bigg) \notag
\end{align}
erhält aus der Test-AWA \cref{3_29} $y_{k+1} = y_k + \frac{h}{2}(\lambda y_k + \lambda y_{k+1})$ und somit
\begin{align}
	\left(1-\frac{h}{2}\lambda\right)y_{k+1} = y_k\left(1+\frac{h}{2}\lambda\right)\notag
\end{align}
das heißt die Stabilitätsfunktion der Trapezregel ist gegeben durch
\begin{align}
	g(z) = \frac{1+\frac{z}{2}}{1-\frac{z}{2}} = \frac{2+z}{2-z}\notag
\end{align}
Falls $\Re(z)\le 0$, so folgt
\begin{align}
	\abs{g(z)}^2 = \frac{(2+\Re(z))^2 + (\Im(z))^2}{(2-\Re(z))^2 + (\Im(z))^2} \le 1\notag
\end{align}
also die A-Stabilität der Trapezregel.

Für das \begriff[\person{Euler}-Verfahren!]{explizite \person{Euler}-Verfahren} ergibt sich (wegen \cref{3_29})
\begin{align}
	y_{k+1} = y_k + h\lambda y_k = (1+h\lambda)y_k\quad\text{und}\quad g(z) = 1+z\notag
\end{align}
Damit gilt
\begin{align}
	\abs{g(z)}^2 = \abs{1+z}^2 = (1+\Re(z))^2 + (\Im(z))^2 \le 1 \notag
\end{align}
genau dann, wenn $z=h\lambda$ im Einheitskreis um $(-1,0)\in\comp$ liegt. Da der Stabilitätsbereich beim expliziten \person{Euler}-Verfahren nicht alle $z\in\comp$ mit $\Re(z)\le 0$ enthält, ist dieses Verfahren nicht A-stabil. Das explizite \person{Euler}-Verfahren hat Konsistenzordnung 1 (vgl. \propref{3_2_3}) und ist mit dieser Ordnung auch konvergent (vgl. \propref{satz_3_8}). Die fehlende A-Stabilität hat zur Folge, dass zur erfolgreichen numerischen Lösung der Test-AWA \cref{3_29} für $\lambda <0$ zumindest $-2\le h\lambda$ gelten muss. Dies erfordert $h\sim \frac{1}{\abs{\lambda}}$, also gegebenenfalls sehr kleine Schrittweiten. Dies ist neben einem hohen Aufwand auch die Gefahr des Überwiegens von Rundungsfehlern verbunden, vgl. Abschnitt 2.4. Verfahren, die A-stabil sind, bzw. einen hinreichend großen Bereich absoluter Stabilität besitzen, haben außerdem Vorteile bei sogenannten steifen AWA, vgl. Abschnitt 5.

Bei RKV kann man den Stabilitätsbereich untersuchen, indem man sich die Stabilitätsfunktion beschafft. Zum Beispiel betrachten wir das 2-stufige explizite RKV
\begin{align}
	y_{k+1} &= y_k + hc_1k_1 + hc_2k_2 \notag \\
	&= y_k + hc_1f(x_k,y_k) + hc_2f(x_k+\alpha_2h,y_k + h\beta_{21}f(x_k,y_k))\notag
\end{align}
Mit der Test-AWA \cref{3_29} folgt
\begin{align}
	y_{k+1} &= y_k + h\lambda c_1y_k + h\lambda c_2(y_k + h\lambda\beta_{21} y_k) \notag \\
	&= y_k(1+h\lambda c_1 + h\lambda c_2 + (h\lambda)^2 c_2\beta_{21}) \notag
\end{align}
Beim Verfahren von \person{Heun} (vgl. Abschnitt 2.5) mit $c_1=c_2=\sfrac{1}{2}$ und $\beta_{21}=1$ ergibt sich
\begin{align}
	y_{k+1} = y_k\left(1+h\lambda + \frac{1}{2}(h\lambda)^2\right)\quad\text{und also}\quad g(z) = 1+z+\frac{1}{2}z^2 \notag
\end{align}
Man sieht schnell, dass dieses Verfahren nicht A-stabil ist (man wähle $z=(a,0)$ mit $a<-2$).

\begin{remark}
	Es gibt kein explizites lineares MSV und kein explizites RKV, dass A-stabil ist und die A-stabilen impliziten MSV haben höchstens Konsistenzordnung 2 (zweite \person{Dahlquist}-Barriere).
\end{remark}