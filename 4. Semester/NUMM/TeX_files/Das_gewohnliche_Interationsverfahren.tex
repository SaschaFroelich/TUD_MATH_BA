\newpage
\section{Das gewöhnliche Interationsverfahren}
\subsection{Fixpunkte}

Seien ein Vektorraum $V$, eine Menge $U \subseteq V$ und eine Abbildung $\Phi: U \to V$ gegeben.
Dann heißt $x^{*} \in U$ \begriff{Fixpunkt} der Abbildung $\Phi$, falls $\Phi(x^{*}) = x^{*}$ gilt.
Die Aufgabe
\begin{align}
	\Phi(x) = x\notag
\end{align}
eigentlich die Aufgabe, diese Gleichung zu lösen) wird als \begriff{Fixpunktaufgabe} bezeichnet.
Die Abbildung $\Phi$ heißt \begriff{Fixpunktabbildung}. Im Unterschied zur Fixpunktaufgabe heißt
\begin{align}
	F(x) = 0 \notag
\end{align}
\begriff{Nullstellenaufgabe}. 
Zu jeder Nullstellenaufgabe gibt es eine äquivalente Fixpunktaufgabe (z.B. $F(x) = 0 \Leftrightarrow \Phi(x) = x $ mit $\Phi(x) := F(x) + x$) und umgekehrt (z.B.
$\Phi(x) = x \Leftrightarrow F(x) = 0$ mit $F(x) := \Phi(x) -x$).

\subsection{Der Fixpunktsatz von Banach}

Der folgende Satz gibt (unter gewissen Bedingungen) eine konstruktive Möglichkeit an, einen Fixpunkt näherungsweise zu ermitteln.

\begin{proposition}[Banach]
	\proplbl{1_1_1}
	%TODO use \norm here and find out why undefined control sequence for \norm shows up?!?!?
	Seien $(V, \Vert \cdot \Vert)$ ein Banach-Raum, $U \subseteq V$ eine abgeschlossene Menge und $\Phi: U \to V$ eine Abbildung. Die Abbildung $\Phi$ sei \begriff{selbstabbildend}, d.h. es gilt
	\begin{align}
		\Phi(U) \subseteq U.\notag
	\end{align}
	Außerdem sei $\Phi$ \begriff{kontraktiv}, d.h. es gibt $\lambda \in [0,1)$, so dass
	\begin{align}
		\Vert\Phi(x) - \Phi(y)\Vert \le \lambda \Vert x-y\Vert, \text{für alle } x,y \in U.\notag
	\end{align}
	Dann besitzt $\Phi$ genau einen Fixpunkt $x^{*} \in U$. Weiterhin konvergiert die durch
	\begin{align}
		x^{k+1} := \Phi(x^k) \label{eq_1_1_1}
	\end{align}
	erzeugte Folge $\{x^k\}$ für jeden Startwert $x^0 \in U$ gegen $x^{*}$ und es gilt für alle $k \in \N$
	\begin{align}
		\Vert x^{k+1} - x^{*}\Vert \le \frac{\lambda}{1 - \lambda}\Vert x^{k+1} - x^k\Vert \text{ a posteriori Fehlerabschätzung},\\
		\Vert x^{k+1} - x^{*}\Vert \le \frac{\lambda^{k+1}}{1 - \lambda}\Vert x^1 - x^0\Vert \text{ a priori Fehlerabschätzung},\\
		\Vert x^{k+1} - x^{*}\Vert \le \lambda \Vert x^{k} - x^{*}\Vert \text{ Q-lineare Konvergenz mit Ordnung }\lambda.
	\end{align}
\end{proposition}

\begin{proof}
	Vorlesung zur Analysis.
\end{proof}

Die in \propref{1_1_1} vorkommende Zahl $\lambda \in [0,1)$ wird \begriff{Kontraktionskonstante} genannt. 

\subsection{Gewöhnliches Iterationsverfahren}

Durch \ref{eq_1_1_1} erklärte Verfahren heißt \begriff{gewöhnliches Interationsverfahren} oder \begriff{Fixpunktiteration}. Kritisch ist dabei, ob die Vorraussetzungen ($\Phi$ ist selbstabbildend und kontraktiv) erfüllt werden können. Dies wird in diesem Abschnitt im Fall $V = \Rn$ mit einer beliebigen aber festen Vektornorm $\Vert \cdot \Vert$ untersucht. Die zugeordnete Matrixnorm wurde mit $\Vert \cdot \Vert_{\ast}$ bezeichnet.

\begin{lemma}
	\proplbl{1_1_2}
	Sei $S \subseteq \Rn$ offen und konvex und $\Phi: D \to \Rn$ stetig differenzierbar. Falls $L > 0$ existiert mit
	\begin{align}
		\Vert \Phi^{'}(x) \Vert_{\ast} \le L \text{ für alle } x \in D, \label{eq_1_1_5}
	\end{align}
	dann ist $\Phi$ Lipschitz-stetig in $D$ mit der Lipschitz-Konstante $L$, d.h. es gilt
	\begin{align}
		\Vert \Phi(x) - \Phi(y)\Vert \le L \Vert x-y \Vert \text{ für alle } x \in D. \label{eq_1_1_6}
	\end{align}
	Die Umkehrung dieser Aussage ist ebenfalls richtig.
\end{lemma}

\begin{proof}
	\begin{enumerate}
		\item Sei \ref{eq_1_1_5} erfüllt. Mit Satz 5.1 aus der Vorlesung ENM (Taylorformel mit Integralrestglied) folgt % TODO find out which prop is meant!
		\begin{align}
            \Vert \Phi(x) - \Phi(y) \Vert_{\ast} = \left\Vert \int_{0}^{1} \Phi^{'}(y + t(x-y))(x-y) \mathrm{d}t \right\Vert \le \Vert x-y \Vert \sup_{t \in [0,1]} \Vert \Phi^{'}(y+t(x-y))\Vert_{\ast}
		\end{align} % TODO define math operator for "d" in integrals
		für alle $x,y \in D$. Also liefert \ref{eq_1_1_5} unter Beachtung der Konvexität von $D$ die Behauptung.
		\item Sei nun \ref{eq_1_1_6} erfüllt. Angenommen es gibt $\hat{y} \in D$ mit
		\begin{align}
			\Vert \Phi^{'}(\hat{y})\Vert_{\ast} > L. \label{eq_1_1_7}
		\end{align}
		Unter Berücksichtigung der Definition der zugeordneten Matrixnorm $\Vert \cdot \Vert_{\ast}$ folgt, dass $d \in \Rn$ existiert mit $\Vert d \Vert = 1$ und $\Vert \Phi^{'}(\hat{y})d\Vert = \Vert \Phi(\hat{y}) \Vert_{\ast}$. Wendet man nun \ref{eq_1_1_6} und Satz 5.1 der Vorlesung ENM mit $x := \hat{y} + sd$ und $y := \hat{y}$ an, so folgt für alle $s > 0$ hinreichend klein
		\begin{align}
			\Vert \Phi(\hat{y} + sd) - \Phi(\hat{y})\Vert \le L \Vert sd \Vert = sL
		\end{align}
		und 
		\begin{align}
			\Vert \Phi(\hat{y} + sd) - \Phi(\hat{y}) \Vert &= \left\Vert \int_{0}^{1} \Phi^{'}(\hat{y} + tsd)(sd)\mathrm{d}t \right\Vert\notag \\
			&= \left\Vert \int_{0}^{1} \Phi^{'}(\hat{y})(sd)\mathrm{d}t + \int_{0}^{1} (\Phi^{'}(\hat{y}+tsd) - \Phi^{'}(\hat{y}))(sd) \mathrm{d}t \right\Vert\notag \\
			&\ge s \Vert \Phi^{'}(\hat{y})d \Vert - s \Vert d \Vert \sup_{t \in [0,1]}\Vert\Phi^{'}(\hat{y} + tsd) - \Phi^{'}(\hat{y})\Vert_{\ast} \notag \\
			&= s \left( \Vert \Phi^{'}(\hat{y}) \Vert_{\ast} - \sup_{t \in [0,1]}\Vert\Phi^{'}(\hat{y} + tsd) - \Phi^{'}(\hat{y})\Vert_{\ast} \right) \notag\\
			&= sL, \notag
		\end{align}
		wobei sich die letzte Ungleichung wegen \ref{eq_1_1_7} und der Stetigkeit von $\Phi^{'}$ ergibt. Offenbar hat man damit einen Widerspruch, so dass die Annahme falsch ist.
	\end{enumerate}
\end{proof}

\begin{example}
	Die Nullstellenaufgabe $\cos x - 2x = 0$ sei zu lösen. Eine mögliche Formulierung als Fixpunktaufgabe ist
	\begin{align}
		\Phi(x) = x \text{   mit  } \Phi(x) := -x + \cos x \notag
	\end{align}
	Offenbar ist $\Phi: \R \to \R$ selbstabbildend. Weiter ergibt sich
	\begin{align}
		\Phi^{'}(x) = -1 - \sin x \notag
	\end{align}
	Für $x \in D := (0,1)$ gilt daher $\vert \Phi^{'} (x)\vert > 1$. Mit \propref{1_1_2} folgt $\vert \Phi(x) - \Phi(y)\vert \ge \vert x-y\vert$ für mindestens ein Paar $(x,y) \in D \times D$. Somit ist $\Phi$ in $D$ nicht kontrahierend.
	Definiert man $\Phi$ aber durch $\Phi(x) := \sfrac{1}{2}\cos x$, so ist die Fixpunktaufgabe $\sfrac{1}{2}\cos x = x$ wiederum zur Nullstellenaufgabe äquivalent und es folgt
	\begin{align}
		\Phi^{'}(x) = - \frac{1}{2}\sin x. \notag
	\end{align} 
	Damit hat man $\vert \Phi^{'}(x)\vert \le \sfrac{1}{2}$ für alle $x \in \R$. Also ist die zuletzt definierte Abbildung $\Phi$ kontrahierend auf $\R$ (und dort natürlich selbstabbildend), so dass die Vorraussetzungen des Banachschen Fixpunktsatzes erfüllt sind. Die Fixpunktiteration mit $\Phi(x) = \sfrac{1}{2}\cos x$ und $x^0 := 1$ ergibt:
	\begin{align}
        x^1 &= 0.270\dots,  & x^2 &= 0.481\dots,   & x^3 &= 0.433\dots,    & x^4 &= 0.4517\dots, \notag \\
        x^5 &= 0.4498\dots, & x^6 &= 0.45025\dots, & x^7 &= 0.450167\dots, & x^8 &= 0.450187\dots \notag
	\end{align}
\end{example}

Nehmen wir an, die Vorraussetzungen des Banachschen Fixpunktsatzes seien gegeben. Dann hängt die Konvergenzgeschwindigkeit der Fixpunktiteration offenbar von der Kontraktionskonstanten $\lambda \in [0,1)$ ab. Je kleiner $\lambda$ ist, desto schneller ist ist die Konvergenzgeschwindigkeit. Unter Umständen kann die Umformulierung einer Fixpunktaufgabe mit Hilfe einer anderen Fixpunktabbildung helfen, die Konvergenzgeschwindigkeit zu verbessern (ggf. auf Kosten der Größe der Menge $U$, in der die Voraussetzungen des Banachschen Fixpunktsatzes erfüllt sind). Ein Beispiel zur Konstruktion einer Fixpunktabbildung mit lokal beliebig kleiner Kontraktionskonstante gibt Abschnitt 1.4. In Abschnitt 2.1 wird gezeigt, wie Fixpunktabbildungen zur iterativen Lösung von linearen Gleichungssystemen eingesetzt werden können.
Im Weiteren bezeichne $B(x^{\ast}, r) := \{ x \in \Rn \colon \Vert x - x^\ast \Vert \le r \}$ die abgeschlossene Kugel um $x^{\ast}$ mit Radius $r$ (bzgl. einer passenden Norm). 
% TODO decide wether equation is useful - his script just says "B(...) :=" without any mathematical definition

\begin{proposition}[Ostrowski]
	Seien $D \subseteq \Rn$ offen und $\Phi: D \to \Rn$ stetig differenzierbar. Die Abbildung $\Phi$ besitze einen Fixpunkt $x^{\ast} \in D$ mit $\Vert \Phi^{'}(x^{\ast})\Vert_{\ast} < 1$. Dann existiert $r > 0$, so dass das gewöhnliche Iterationsverfahren für jeden Startpunkt $x^0 \in B(x^{\ast}, r)$ gegen $x^{\ast}$ konvergiert.
\end{proposition}

\begin{proof}
	TODO!
\end{proof}