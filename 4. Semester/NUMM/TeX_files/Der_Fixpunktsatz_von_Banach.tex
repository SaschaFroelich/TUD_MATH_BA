\section{Der Fixpunktsatz von Banach}

Der folgende Satz gibt (unter gewissen Bedingungen) eine konstruktive Möglichkeit an, einen Fixpunkt näherungsweise zu ermitteln.

\begin{proposition}[\person{Banach}]
	\proplbl{1_1_1}
	%TODO use \norm here and find out why undefined control sequence for \norm shows up?!?!?
	Seien $(V, \Vert \cdot \Vert)$ ein Banach-Raum, $U \subseteq V$ eine abgeschlossene Menge und $\Phi: U \to V$ eine Abbildung. Die Abbildung $\Phi$ sei \begriff{selbstabbildend}, d.h. es gilt
	\begin{align}
	\Phi(U) \subseteq U.\notag
	\end{align}
	Außerdem sei $\Phi$ \begriff{kontraktiv}, d.h. es gibt $\lambda \in [0,1)$, so dass
	\begin{align}
	\Vert\Phi(x) - \Phi(y)\Vert \le \lambda \Vert x-y\Vert, \text{für alle } x,y \in U.\notag
	\end{align}
	Dann besitzt $\Phi$ genau einen Fixpunkt $x^{*} \in U$. Weiterhin konvergiert die durch
	\begin{align}
	x^{k+1} := \Phi(x^k) \label{eq_1_1_1}
	\end{align}
	erzeugte Folge $\{x^k\}$ für jeden Startwert $x^0 \in U$ gegen $x^{*}$ und es gilt für alle $k \in \natur$
	\begin{align}
	\Vert x^{k+1} - x^{*}\Vert &\le \frac{\lambda}{1 - \lambda}\Vert x^{k+1} - x^k\Vert &\text{ a posteriori Fehlerabschätzung}, \label{eq_1_2_2}\\
	\Vert x^{k+1} - x^{*}\Vert &\le \frac{\lambda^{k+1}}{1 - \lambda}\Vert x^1 - x^0\Vert &\text{ a priori Fehlerabschätzung},\label{eq_1_2_3_}\\
	\Vert x^{k+1} - x^{*}\Vert &\le \frac{\lambda}{1 - \lambda}\Vert x^{k} - x^{*}\Vert &\text{ Q-lineare Konvergenz mit Ordnung }\lambda. \label{eq_1_2_4}
	\end{align}
\end{proposition}

\begin{proof}
	Verlesung zur Analysis.
\end{proof}

Die in \propref{1_1_1} vorkommende Zahl $\lambda \in [0,1)$ wird \begriff{Kontraktionskonstante} genannt. 