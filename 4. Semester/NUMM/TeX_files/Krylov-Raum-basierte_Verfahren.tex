\section{\person{Krylov}-Raum-basierte Verfahren}

\subsection{\person{Krylov}-Räume}

Für $A\in\Rnn$, $r\in\Rn$ und $k\in\natur$ ist der $k$-te \begriff{Krylov-Raum} gegeben durch $\mathcal{K}_0=\{0\}$ und
\begin{align}
	\mathcal{K}_k(r,A) = \Span\{r,Ar,A^2r,...,A^{k-1}r\}\quad\text{ für } k>0\notag
\end{align}
Offenbar ist $\dim(\mathcal{K}_k(r,A))\le\min\{k,n\}$ für alle $k\in\natur$.

\begin{lemma}
	\proplbl{lemma_2_6}
	Es seien $A\in\Rnn$, $r\in\Rn\backslash\{0\}$ und $k\in\natur$ gegeben. Dann sind folgende Aussagen äquivalent:
	\begin{enumerate}[label=(\alph*)]
		\item $\dim(\mathcal{K}_{k+1}(r,A))< k+1$
		\item $\mathcal{K}_k(r,A) = \mathcal{K}_{k+1}(r,A)$
	\end{enumerate}
\end{lemma}
\begin{proof}
	\begin{itemize}
		\item (a) $\Rightarrow$ (b): Nach Voraussetzung gibt es $l\in\{1,...,k\}$ und $\alpha_0,...,\alpha_l\in\R$, so dass
		\begin{align}
			A^lr = \sum_{i=0}^{l-1} \alpha_iA^ir \notag
		\end{align}
		Multiplikation mit $A^{k-l}$ liefert
		\begin{align}
			A^kr = \sum_{i=0}^{l-1} \alpha_iA^{k-l+i}r\in\mathcal{K}_k(r,A)\notag
		\end{align}
		Also folgt $\mathcal{K}_k(r,A) = \mathcal{K}_{k+1}(r,A)$.
		\item (b) $\Rightarrow$ (a): Offensichtlich
	\end{itemize}
\end{proof}

\begin{proposition}
	\proplbl{satz_2_7}
	Es seien $A\in\Rnn$ regulär, $x^0\in\Rn$ mit $r^0=b-Ax^0\neq 0$ gegeben. Dann sind folgende Aussagen für $k\in\natur$ äquivalent:
	\begin{enumerate}[label=(\alph*)]
		\item $\mathcal{K}_k(r^0,A) = \mathcal{K}_{k+1}(r^0,A)$
		\item $x^\ast = A^{-1}b\in x^0 + \mathcal{K}_k(r^0,A)$
	\end{enumerate}
\end{proposition}
\begin{proof}
	\begin{itemize}
		\item (a) $\Rightarrow$ (b): Wegen \propref{lemma_2_6} gibt es $l\in\{0,...,k\}$ und $\mu_l,...,\mu_k\in\R$, so dass $\mu_l\neq 0$ und
		\begin{align}
			0 &= \sum_{i=l}^{k} \mu_iA^ir^0 \notag \\
			&= \mu_lA^lr^0 + \sum_{i=l+1}^{k} \mu_iA^ir^0 \notag
		\end{align}
		wobei der Summationsterm auf der rechten Seite entfällt, wenn $l=k$. Wegen der Regularität von $A$ kann man die Gleichung mit $A^{-(l+1)}$ multiplizieren. Für $l=k$ liefert dies $0=A^{-1}r^0 = A^{-1}b-x^0\in\mathcal{K}_k(r^0,A)$. Für $l<k$ folgt
		\begin{align}
			A^{-1}b - x^0 &= A^{-1}r \notag \\
			&= -\frac{1}{\mu_l} \sum_{i=l+1}^k \mu_iA^{i-l-1}r ^0 \notag \\
			&= -\frac{1}{\mu_l}\sum_{i=0}^{k-l-1} \mu_{i+l+1}A^ir^0\in\mathcal{K}_k(r^0,A) \notag
		\end{align}
		Somit gilt Aussage (b).
		\item (b) $\Rightarrow$ (a): Nach Voraussetzung gilt $x^\ast\in x_0 + \mathcal{K}_k(r^0,A)$. Durch Multiplikation mit $A$ folgt
		\begin{align}
			b&\in Ax^0 + A\mathcal{K}_k(r^0,A)\notag \\
			&= Ax^0 + \Span\{Ar^0,A^2r^0,A^3r^0,...,A^kr^0\}\notag
		\end{align}
		Also ist $r^0=b-Ax^0$ eine Linearkombination der Vektoren $Ar^0,A^2r^0,A^3r^0,...,A^kr^0$ und es gilt
		\begin{align}
			\dim(\mathcal{K}_{k+1}(r^0,A)) < k+1\notag
		\end{align}
		 \propref{lemma_2_6} liefert damit die Gültigkeit von Aussage (a).
	\end{itemize}
\end{proof}

\begin{remark}
	Offenbar gibt es $k^\ast\in\{1,...,n\}$, so dass Aussage (a) von \propref{satz_2_7} für $k^\ast$ zutrifft, aber für kein $k<k^\ast$ erfüllt ist. \propref{satz_2_7} zeigt daher, dass die exakte Lösung $x^\ast$ von $Ax=b$ in $x^0 + \mathcal{K}_{k^\ast}(r^0,A)$ liegt. Man kann also nun versuchen, eine Folge $\{x^k\}$ mit $x^k\in x^0 + \mathcal{K}_k(r^0,A)$ zu bestimmen, so dass $x^k$ das Gleichungssystem $Ax=b$ (geeignet) näherungsweise löst. Dazu werden in nächsten Abschnitt zwei grundlegende Ansätze angegeben (Minimum-Residuum und Galerkin).
\end{remark}

\subsection{Basisalgorithmen zur Lösung von $Ax=b$}

\begin{algorithm}[Minimum-Residuum Basisalgorithmus]
	Input: $x^0\in\R$, $A,B\in\Rnn$ regulär, $b\in\Rn$
\begin{lstlisting}
while %$r^k\neq 0$% do
 k = k + 1;
 compute %$x^k\in\Rn\text{ als Lösung von}$%
 %$\quad f_B(x) = \frac{1}{2}\Vert b-Ax\Vert^2_B\to\min\quad\text{bei } x\in x^0+\mathcal{K}_k(r^0,A)$%
 %$r^k$% = b - A%$x^k$%
enddo
\end{lstlisting}
	Output: $x^\ast=x^k$, $k^\ast=k$
\end{algorithm}

\begin{algorithm}[Galerkin Basisalgorithmus]
	Input: $x^0\in\R$, $A,B\in\Rnn$ regulär, $b\in\Rn$
	\begin{lstlisting}
%$r^0$% = b - A%$x^0$%;
k = 0;
while %$r^k\neq 0$% do
 k = k + 1;
 determine %Untervektorraum $\mathcal{L}_k$ von $\Rn$ mit $\dim(\mathcal{L}_k)=k$%;
 compute %$x^k\in x^0+\mathcal{K}_k(r^0,A)$ mit $b-Ax^k\perp\mathcal{L}_k$%;
 %$r^k$% = b - A%$x^k$%;
enddo
	\end{lstlisting}
	Output: $x^\ast=x^k$, $k^\ast=k$
\end{algorithm}

\subsection{Das CG-Verfahren}

\begin{algorithm}[CG-Verfahren]
	Input: $x^0\in\Rn$, $A\in\Rnn$ symmetrisch und positiv definit, $b\in\Rn$
	\begin{lstlisting}
%$d^0$% = %$r^0$% = b - A%$x^0$%;
k = 0;
while %$r^k\neq 0$% do
 %$t_k$% = %$\frac{\Vert r^k\Vert_2^2}{(d^k)^T Ad^k}$%;
 %$x^{k+1}$% = %$x^k$% + %$t_kd^k$%;
 %$r^{k+1}$% = b - A%$x^{k+1}$% = %$r^k$% - %$t_kAd^k$%;
 %$\beta_k$% = %$\frac{\Vert r^{k+1}\Vert_2^2}{\Vert r^k\Vert_2^2}$%;
 %$d^{k+1}$% = %$r^{k+1}$% - %$\beta_kd^k$%;
 k = k + 1;
enddo
	\end{lstlisting}
	Output: $x^\ast=x^k$
\end{algorithm}

\subsection{Fehlerverhalten des CG-Verfahrens}

\subsection{Vorkonditionierung}

\begin{algorithm}[CG-Verfahren für $\tilde{A}\tilde{x}=\tilde{b}$]
	Input: $\tilde{x}^0\in\Rn$, $\tilde{A}\in\Rnn$ symmetrisch und positiv definit, $\tilde{b}\in\Rn$
	\begin{lstlisting}
	%$\tilde{d}^0$% = %$\tilde{r}^0$% = %$\tilde{b} - \tilde{A}\tilde{x}^0$%;
	k = 0;
	while %$\tilde{r}^k\neq 0$% do
	%$\tilde{t}_k$% = %$\frac{\Vert \tilde{r}^k\Vert_2^2}{(\tilde{d}^k)^T \tilde{A}\tilde{d}^k}$%;
	%$\tilde{x}^{k+1}$% = %$\tilde{x}^k$% + %$\tilde{t}_k\tilde{d}^k$%;
	%$\tilde{r}^{k+1}$% = %$\tilde{b} - \tilde{A}\tilde{x}^{k+1}$% = %$\tilde{r}^k$% - %$\tilde{t}_k\tilde{A}\tilde{d}^k$%;
	%$\tilde{\beta}_k$% = %$\frac{\Vert \tilde{r}^{k+1}\Vert_2^2}{\Vert \tilde{r}^k\Vert_2^2}$%;
	%$\tilde{d}^{k+1}$% = %$\tilde{r}^{k+1}$% - %$\tilde{\beta}_k\tilde{d}^k$%;
	k = k + 1;
	enddo
	\end{lstlisting}
	Output: $x^\ast=x^k$
\end{algorithm}

\begin{algorithm}[Vorkonditioniertes CG-Verfahren]
	Input: $x^0\in\Rn$, $A\in\Rnn$ symmetrisch und positiv definit, $b\in\Rn$
	\begin{lstlisting}
%$r^0$% = b - A%$x^0$%;
compute %$z^0$ als Lösung von $Pz = r^0$% 
%$d^0$% = %$z^0$%
k = 0;
while %$r^k\neq 0$% do
 %$t_k$% = %$\frac{(r^k)^T}{(d^k)^T Ad^k}$%;
 %$x^{k+1}$% = %$x^k$% + %$t_kd^k$%;
 %$r^{k+1}$% = %$r^k$% - %$t_kAd^k$%;
 compute %$z^{k+1}$ als Lösung von $Pz = r^{k+1}$%
 %$\beta_k$% = %$\frac{(r^{k+1})^T z^{k+1}}{(r^k)^T z^k}$%;
 %$d^{k+1}$% = %$z^{k+1}$% + %$\beta_kd^k$%;
 k = k + 1;
enddo
	\end{lstlisting}
	Output: $x^\ast=x^k$
\end{algorithm}

\subsection{Ausblick und Anmerkungen}