\documentclass[ngerman,a4paper,order=firstname]{../../texmf/tex/latex/mathscript/mathscript}
\usepackage{../../texmf/tex/latex/mathoperators/mathoperators}

\title{\textbf{Numerische Mathematik SS 2019}}
\author{Dozent: Prof. Dr. \person{Andreas Fischer}}

\begin{document}
\pagenumbering{roman}
\pagestyle{plain}

\maketitle

\hypertarget{tocpage}{}
\tableofcontents
\bookmark[dest=tocpage,level=1]{Inhaltsverzeichnis}

\pagebreak
\pagenumbering{arabic}
\pagestyle{fancy}

\chapter*{Vorwort}
Schön, dass du unser Skript für die Vorlesung \textit{Lineare Algebra und analytische Geometrie 1} bei Prof. Dr. Arno Fehm im WS2017/18 gefunden hast! \footnote{Obwohl man sagen kann, dass es in dieser Vorlesung nur um Lineare Algebra ging, der Teil mit der analytischen Geometrie wurde vernachlässigt. Liegt wahrscheinlich auch daran, dass es demnächst eine Reform der Studienordnung gibt, in der aus der Vorlesung \textit{Lineare Algebra und analytische Geometrie} die Vorlesung \textit{Einführung in die Lineare Algebra} wird.}

Wir verwalten dieses Skript mittels Github \footnote{Github ist eine Seite, mit der man Quelltext online verwalten kann. Dies ist dahingehend ganz nützlich, dass man die Quelltext-Dateien relativ einfach miteinander synchronisieren kann, wenn man mit mehren Leuten an einem Projekt arbeitet.}, d.h. du findest den gesamten \LaTeX-Quelltext auf \url{https://github.com/henrydatei/TUD_MATH_BA}. Unser Ziel ist, für alle Pflichtveranstaltungen von \textit{Mathematik-Bachelor} ein gut lesbares Skript anzubieten. Für die Programme, die in den Übungen zur Vorlesung \textit{Programmieren für Mathematiker} geschrieben werden sollen, habe ich ein eigenes Repository eingerichtet; es findet sich bei \url{https://github.com/henrydatei/TU_PROG}.

Du kannst dir gerne dort die \LaTeX-Quelldateien herunterladen, die Dateien für exakt dieses Skript sind im Ordner \texttt{1. Semester/LAAG ueberarbeitet}. Es lohnt sich auf jeden Fall während des Studiums die Skriptsprache \LaTeX{} zu lernen, denn Dokumente, die viele mathematische oder physikalische Formeln enthalten, lassen sich sehr gut mittels \LaTeX{} darstellen, in Word oder anderen Office-Programmen sieht so etwas dann eher dürftig aus.

\LaTeX{} zu lernen ist gar nicht so schwierig, ich habe dafür am Anfang des ersten Semesters wenige Wochen benötigt, dann kannte ich die wichtigsten Befehle und konnte den Vorgänger dieses Skriptes schreiben (\texttt{1. Semester/LAAG}, Vorsicht: hässlich, aber der Quelltext ist relativ gut verständlich).

Es sei an dieser Stelle darauf hingewiesen (wie in jedem anderem Skript auch \smiley{}), dass dieses Skript nicht den Besuch der Vorlesungen ersetzen kann. Es könnte sein, dass Prof. Fehm seine Vorlesung immer mal wieder an die Studenten anpasst; wahrscheinlich immer dann, wenn die Prüfungsergebnisse zu schlecht waren. Nichtsdestotrotz veröffentlicht Prof. Fehm sein Skript auf seiner Homepage \url{http://www.math.tu-dresden.de/~afehm/lehre.html}. Allerdings ist dieses Skript recht hässlich, besonders was die Übersichtlichkeit angeht.

Wir möchten deswegen ein Skript bereitstellen, dass zum einen übersichtlich ist, zum anderen \textit{alle} Inhalt aus der Vorlesung enthält, das sind insbesondere Diagramme, die sich nicht im offiziellen Skript befinden, aber das Verständnis des Inhalts deutlich erleichtern. Ich denke, dass uns dies erfolgreich gelungen ist.

Trotz intensivem Korrekturlesen können sich immer noch Fehler in diesem Skript befinden. Es wäre deswegen ganz toll von dir, wenn du auf unserer Github-Seite \url{https://github.com/henrydatei/TUD_MATH_BA} ein neues Issue erstellst und damit auch anderen hilfst, dass dieses Skript immer besser wird.

\begin{itemize}
	\item Bollhöfer/Mehrmann: Numerische Mathematik, Vieweg 2004
	\item Deuflhard/Hohmann: Numerische Mathematik1, de Gruyter 2008
	\item Deuflhard/Bornemann: Numerische Mathematik, de Gruyter 2008
	\item Deuflhard/Weiser: Numerische Mathematik 3, de Gruyter 2011
	\item Freund/Hoppe: Stoer/Bulirsch: Numerische Mathematik 1, Springer 2007
	\item Hämmerlin/Hoffmann: Numerische Mathematik, Springer 2013
	\item Knorrenschild, M: Numerische Mathematik, Fachbuchverlag 2005
	\item Plato, R: Numerische Mathematik kompakt, Vieweg 2009
	\item Preuß/Wenisch: Lehr- und Übungsbuch Numerische Mathematik, Fachbuchverlag 2001
	\item Quarteroni/Sacco/Saleri: Numerische Mathematik 1+2, Springer 2002
	\item Roos/Schwetlick: Numerische Mathematik, Teubner 1999
	\item Schaback/Wendland: Numerische Mathematik, Springer 2004
	\item Stoer/Bulirsch: Numerische Mathematik II, Springer 2005
\end{itemize}
\chapter{Das gewöhnliche Iterationsverfahren}
\section{Fixpunkte}

Seien ein Vektorraum $V$, eine Menge $U \subseteq V$ und eine Abbildung $\Phi: U \to V$ gegeben.
Dann heißt $x^{*} \in U$ \begriff{Fixpunkt} der Abbildung $\Phi$, falls $\Phi(x^{*}) = x^{*}$ gilt.
Die Aufgabe
\begin{align}
\Phi(x) = x\notag
\end{align}
eigentlich die Aufgabe, diese Gleichung zu lösen) wird als \begriff{Fixpunktaufgabe} bezeichnet.
Die Abbildung $\Phi$ heißt \begriff{Fixpunktabbildung}. Im Unterschied zur Fixpunktaufgabe heißt
\begin{align}
F(x) = 0 \notag
\end{align}
\begriff{Nullstellenaufgabe}. 
Zu jeder Nullstellenaufgabe gibt es eine äquivalente Fixpunktaufgabe (z.B. $F(x) = 0 \Leftrightarrow \Phi(x) = x $ mit $\Phi(x) := F(x) + x$) und umgekehrt (z.B.
$\Phi(x) = x \Leftrightarrow F(x) = 0$ mit $F(x) := \Phi(x) -x$).
\section{Der Fixpunktsatz von Banach}

Der folgende Satz gibt (unter gewissen Bedingungen) eine konstruktive Möglichkeit an, einen Fixpunkt näherungsweise zu ermitteln.

\begin{proposition}[Banach]
	\proplbl{1_1_1}
	%TODO use \norm here and find out why undefined control sequence for \norm shows up?!?!?
	Seien $(V, \Vert \cdot \Vert)$ ein Banach-Raum, $U \subseteq V$ eine abgeschlossene Menge und $\Phi: U \to V$ eine Abbildung. Die Abbildung $\Phi$ sei \begriff{selbstabbildend}, d.h. es gilt
	\begin{align}
	\Phi(U) \subseteq U.\notag
	\end{align}
	Außerdem sei $\Phi$ \begriff{kontraktiv}, d.h. es gibt $\lambda \in [0,1)$, so dass
	\begin{align}
	\Vert\Phi(x) - \Phi(y)\Vert \le \lambda \Vert x-y\Vert, \text{für alle } x,y \in U.\notag
	\end{align}
	Dann besitzt $\Phi$ genau einen Fixpunkt $x^{*} \in U$. Weiterhin konvergiert die durch
	\begin{align}
	x^{k+1} := \Phi(x^k) \label{eq_1_1_1}
	\end{align}
	erzeugte Folge $\{x^k\}$ für jeden Startwert $x^0 \in U$ gegen $x^{*}$ und es gilt für alle $k \in \natur$
	\begin{align}
	\Vert x^{k+1} - x^{*}\Vert \le \frac{\lambda}{1 - \lambda}\Vert x^{k+1} - x^k\Vert \text{ a posteriori Fehlerabschätzung}, \label{eq_1_2_2}\\
	\Vert x^{k+1} - x^{*}\Vert \le \frac{\lambda^{k+1}}{1 - \lambda}\Vert x^1 - x^0\Vert \text{ a priori Fehlerabschätzung},\label{eq_1_2_3_}\\
	\Vert x^{k+1} - x^{*}\Vert \le \frac{\lambda}{1 - \lambda}\Vert x^{k} - x^{*}\Vert \text{ Q-lineare Konvergenz mit Ordnung }\lambda. \label{eq_1_2_4}
	\end{align}
\end{proposition}

\begin{proof}
	Verlesung zur Analysis.
\end{proof}

Die in \propref{1_1_1} vorkommende Zahl $\lambda \in [0,1)$ wird \begriff{Kontraktionskonstante} genannt. 
\section{Gewöhnliche Iterationsverfahren}

Durch \cref{eq_1_1_1} erklärte Verfahren heißt \begriff{gewöhnliches Interationsverfahren} oder \begriff{Fixpunktiteration}. Kritisch ist dabei, ob die Voraussetzungen ($\Phi$ ist selbstabbildend und kontraktiv) erfüllt werden können. Dies wird in diesem Abschnitt im Fall $V = \Rn$ mit einer beliebigen aber festen Vektornorm $\Vert \cdot \Vert$ untersucht. Die zugeordnete Matrixnorm wurde mit $\Vert \cdot \Vert_{\ast}$ bezeichnet.

\begin{lemma}
	\proplbl{1_1_2}
	Sei $S \subseteq \Rn$ offen und konvex und $\Phi: D \to \Rn$ stetig differenzierbar. Falls $L > 0$ existiert mit
	\begin{align}
	\Vert \Phi'(x) \Vert_{\ast} \le L \text{ für alle } x \in D, \label{eq_1_1_5}
	\end{align}
	dann ist $\Phi$ Lipschitz-stetig in $D$ mit der Lipschitz-Konstante $L$, d.h. es gilt
	\begin{align}
	\Vert \Phi(x) - \Phi(y)\Vert \le L \Vert x-y \Vert \text{ für alle } x \in D. \label{eq_1_1_6}
	\end{align}
	Die Umkehrung dieser Aussage ist ebenfalls richtig.
\end{lemma}

\begin{proof}
	\begin{enumerate}
		\item Sei \cref{eq_1_1_5} erfüllt. Mit Satz 5.1 aus der Vorlesung ENM folgt %TODO find out which prop is meant!
		\begin{align}
		\norm{\Phi(x) - \Phi(y)}_{\ast} = \norm{\int_{0}^{1} \Phi'(y + t(x-y))(x-y) \diff t} \le  \norm{x-y} \sup_{t \in [0,1]} \norm{\Phi'(y+t(x-y))}_{\ast}
		\end{align}
		für alle $x,y \in D$. Also liefert \cref{eq_1_1_5} unter Beachtung der Konvexität von $D$ die Behauptung.
		\item Sei nun \cref{eq_1_1_6} erfüllt. Angenommen es gibt $\hat{y} \in D$ mit
		\begin{align}
		\norm{\Phi'(\hat{y})}_{\ast} > L. \label{eq_1_1_7}
		\end{align}
		Unter Berücksichtigung der Definition der zugeordneten Matrixnorm $\norm{\cdot}_{\ast}$ folgt, dass $d \in \Rn$ existiert mit $\Vert d \Vert = 1$ und $\norm{\Phi'(\hat{y}d)} = \norm{\Phi(\hat{y})}_{\ast}$. Wendet man nun ENM mit $x := \hat{y} + sd$ und $y := \hat{y}$ an, so folgt für alle $s > 0$ hinreichend klein
		\begin{align}
		\norm{\Phi(\hat{y} + sd) - \Phi(\hat{y})} \le L \norm{sd} = sL
		\end{align}
		und 
		\begin{align}
		\norm{\Phi(\hat{y} + sd) - \Phi(\hat{y})} &= \norm{\int_{0}^{1} \Phi'(\hat{y} + tsd)(sd)\diff t}\notag \\
		&= \norm{\int_{0}^{1} \Phi'(\hat{y} + tsd)(sd)\diff t + \int_{0}^{1} \Phi'(\hat{y})(sd)(sd)\diff t - \int_{0}^{1} \Phi'(\hat{y})(sd)(sd)\diff t}\notag \\
		&\ge s\norm{\Phi'(\hat{y}d)} - s\norm{d} \sup_{t \in [0,1]}\norm{\Phi'(\hat{y} + tsd) - \Phi'(\hat{y})}_{\ast} \notag \\
		&= s (\norm{\Phi'(\hat{y})}_{\ast} - \sup_{t \in [0,1]}\norm{\Phi'(\hat{y} + tsd) - \Phi'(\hat{y})}_{\ast}) \notag\\
		&> sL, \notag
		\end{align}
		wobei sich die letzte Ungleichung wegen \cref{eq_1_1_7} und der Stetigkeit von $\Phi'$ ergibt. Offenbar hat man damit einen Widerspruch, so dass die Annahme falsch ist.
	\end{enumerate}
\end{proof}

\begin{example}
	Die Nullstellenaufgabe $\cos(x) - 2x = 0$ sei zu lösen. Eine mögliche Formulierung als Fixpunktaufgabe ist
	\begin{align}
	\Phi(x) = x \text{   mit  } \Phi(x) := -x + \cos(x) \notag
	\end{align}
	Offenbar ist $\Phi: \R \to \R$ selbstabbildend. Weiter ergibt sich
	\begin{align}
	\Phi'(x) = -1 - \sin(x) \notag
	\end{align}
	Für $x \in D := (0,1)$ gilt daher $\vert \Phi' (x)\vert > 1$. Mit \propref{1_1_2} folgt $\vert \Phi(x) - \Phi(y)\vert \ge \abs{x-y}$ für mindestens ein Paar $(x,y) \in D \times D$. Somit ist $\Phi$ in $D$ nicht kontrahierend.
	Definiert man $\Phi$ aber durch $\Phi(x) := \sfrac{1}{2}\cos(x)$, so ist die Fixpunktaufgabe $\sfrac{1}{2}\cos(x) = x$ wiederum zur Nullstellenaufgabe äquivalent und es folgt
	\begin{align}
	\Phi'(x) = \frac{1}{2}\sin(x). \notag
	\end{align} 
	Damit hat man $\vert \Phi'(x)\vert \le \sfrac{1}{2}$ für alle $x \in \R$. Also ist die zuletzt definierte Abbildung $\Phi$ kontrahierend auf $\R$ (und dort natürlich selbstabbildend), so dass die Voraussetzungen des Banachschen Fixpunktsatzes erfüllt sind. Die Fixpunktiteration mit $\Phi(x) = \sfrac{1}{2}\cos(x)$ und $x^0 := 1$ ergibt:
	\begin{align}
		x^1 &= 0.270\dots \notag \\
		x^2 &= 0.481\dots \notag \\
		x^3 &= 0.433\dots \notag \\
		x^4 &= 0.4517\dots \notag \\
		x^5 &= 0.4498 \dots \notag \\
		x^6 &= 0.45025\dots \notag \\
		x^7 &= 0.450167\dots \notag \\
		x^8 &= 0.450187\dots \notag
	\end{align}
	\begin{center}
		\begin{tikzpicture}
		\begin{axis}[
		xmin=0, xmax=1, xlabel=$x$,
		ymin=0, ymax=1, ylabel=$y$,
		axis x line=middle,
		axis y line=middle,
		samples=400,
		]
		\addplot[mark=none,blue] {x};
		\addplot[mark=none,blue] {0.5*cos(deg(x))};
		\draw[dotted] (axis cs: 0.27,0) -- (axis cs: 0.27,0.8);
		\draw[dotted] (axis cs: 0.481,0) -- (axis cs: 0.481,0.8);
		\draw[dotted] (axis cs: 0.433,0) -- (axis cs: 0.433,0.8);
		\draw[dotted] (axis cs: 0.4517,0) -- (axis cs: 0.4517,0.8);
		\draw[dotted] (axis cs: 0.4498,0) -- (axis cs: 0.4498,0.8);
		\draw[dotted] (axis cs: 0.45025,0) -- (axis cs: 0.45025,0.8);
		\draw[dotted] (axis cs: 0.450167,0) -- (axis cs: 0.450167,0.8);
		\draw[dotted] (axis cs: 0.450187,0) -- (axis cs: 0.450187,0.8);
		\end{axis}
		\end{tikzpicture}
	\end{center}
\end{example}

Nehmen wir an, die Voraussetzungen des Banachschen Fixpunktsatzes seien gegeben. Dann hängt die Konvergenzgeschwindigkeit der Fixpunktiteration offenbar von der Kontraktionskonstanten $\lambda \in [0,1)$ ab. Je kleiner $\lambda$ ist, desto schneller ist ist die Konvergenzgeschwindigkeit. Unter Umständen kann die Umformulierung einer Fixpunktaufgabe mit Hilfe einer anderen Fixpunktabbildung helfen, die Konvergenzgeschwindigkeit zu verbessern (ggf. auf Kosten der Größe der Menge $U$, in der die Voraussetzungen des Banachschen Fixpunktsatzes erfüllt sind.) Ein Beispiel zu Konstruktion einer Fixpunktabbildung mit lokal beliebig kleiner Kontraktionskonstante gibt Abschnitt 1.4. In Abschnitt 2.1 wird gezeigt, wie Fixpunktabbildungen zu iterativen Lösung von linearen Gleichungssystemen eingesetzt werden können.
Im Weiteren bezeichne $B(x^{\ast}, r) :=$ die abgeschlossene Kugel um $x^{\ast}$ mit Radius $r$ (bzgl. einer passenden Norm).

\begin{proposition}[\person{Ostrowski}]
		\proplbl{1_3_4}
	Seien $D \subseteq \Rn$ offen und $\Phi: D \to \Rn$ stetig differenzierbar. Die Abbildung $\Phi$ besitze einen Fixpunkt $x^{\ast} \in D$ mit $\Vert \Phi'(x^{\ast})\Vert_{\ast} < 1$. Dann existiert $r > 0$, so dass das gewöhnliche Iterationsverfahren für jeden Startpunkt $x^0 \in B(x^{\ast}, r)$ gegen $x^{\ast}$ konvergiert.
\end{proposition}

\begin{proof}
	Da $\Phi$ stetig differenzierbar ist und $\norm{\Phi'(x^\ast)}_{\ast} < 1$, gibt es $\lambda \in[0,1]$ und $r > 0$, sodass
	\begin{align}
		\norm{\Phi'(x)}_{\ast} \le \lambda \quad \text{ für alle } x\in B(x^\ast,r).\notag
	\end{align}
	Nach \propref{1_1_2} gilt daher
	\begin{align}
		\norm{\Phi(x) - \Phi(y)}\le \lambda\norm{x-y} \quad\text{ für alle }x,y \in B(x^\ast,r).
	\end{align}
	Insbesondere folgt hieraus
	\begin{align}
		\norm{\Phi(x) - \Phi(x^\ast)} = \norm{\Phi(x) - x^\ast} \le \lambda \norm{x-x^\ast} \quad \text{ für alle } x \in B(x^\ast,r)
	\end{align}
	und damit $\Phi(x) \in B(x^\ast,r)$ für alle $x \in B(x^\ast,r)$. Also ist $\Phi$ bzgl. $B(x^\ast,r)$ selbstabbildend und kontraktiv. Daher liefert \propref{1_1_1} die gewünschte Aussage.
\end{proof}
\section{Das \person{Newton}-Verfahren als Fixpunktiteration}

Sei $D \subseteq \Rn$ offen und $F: D \to \Rn$ stetig differenzierbar. Die Nullstellenaufgabe
\begin{align}
	F(x) = 0\notag
\end{align}
wird nun in eine äquivalente Fixpunktaufgabe überführt. Dazu nehmen wir an, dass $x^{\ast}$ eine reguläre Nullstelle von $F$ ist. Wegen der vorausgesetzten Stetigkeit von $F'$ gibt es $r>0$ hinreichend klein, so dass $F'(x)$ für $x \in B(x^{\ast},r)$ regulär ist. Damit erhält man
\begin{align}
	F(x) = 0 \Leftrightarrow 0 = -F'(x)^{-1}F(x) \Leftrightarrow x = x - F'(x)^{-1}F(x).\notag
\end{align}
für $x \in B(x^{\ast},r)$. Definiert man $\Phi: B(x^{\ast},r) \to \Rn$ durch
\begin{align}
	\Phi(x):= x - F^{\ast}(x)^{-1}F(x). \label{eq_1_3_8}
\end{align}
so kann das Newton-Verfahren als Fixpunktverfahren mit $\Phi$ als Fixpunktabbildung interpretiert werden. Ob $\Phi$ selbstabbildend und kontrahierend ist, müsste noch untersucht werden. Hier soll nur die Kontraktionseigenschaft in $B(x^{\ast},r)$ für $r>1$ hinreichend klein betrachtet werden. Die Eigenschaft der Selbstabbildung ergibt sich dann wie im Beweis zu \propref{1_3_4}.

\begin{lemma}
	Sei $D\subseteq \Rn$ offen und $F: D \to \Rn$ stetig differenzierbar. Weiter sei $x^{\ast}\in D$ eine reguläre Nullstelle von $F$. Dann ist $\Phi$ in $x^{\ast}$ differenzierbar mit $\Phi'(x^{\ast}) = 0$.
\end{lemma}

\begin{proof}
	Wie zuvor gezeigt wurde, ist die durch \cref{eq_1_3_8} definierte Abbildung $\Phi$ in $B(x^{\ast},r)\subset D$ hinreichend kleines $r>0$ wohldefiniert. Falls
	\begin{align}
		\lim_{x\to x^{\ast}} \frac{\norm{\Phi(x) - \Phi(x^{\ast}) - G(x-x^{\ast})}}{\norm{x-x^{\ast}}} \label{eq_1_4_9}
	\end{align}
	mit $G = 0\in \Rnn$ gilt, folgt die Behauptung des Lemmas aus der Definition der Fréchet-Differenzierbarkeit. Unter Beachtung von $\Phi(x^{\ast}) = x^{\ast}$ ergibt sich
	\begin{align}
	\Phi(x) - \Phi(x^{\ast}) = x - F'(x)^{-1}F(x) - x^{\ast} = -F'(x)^{-1}(F'(x))(x^{\ast}-x)+F(x))\notag
	\end{align}
	und mit Satz 5.1 aus der Vorlesung ENM folgt weiter
	\begin{align}
		\Phi(x) - \Phi(x^{\ast}) = F'(x)^{-1}\left( -F(x^{\ast}) + \int_{0}^{1} (F'(x+t(x^{\ast}-x)) - F'(x))(x^{\ast}-x)\diff t\right)\label{eq_1_4_10}
	\end{align}
	für alle $x \inn B(x^{\ast},r)$. Die Stetigkeit von $F'$ auf der kompakten Menge $B(x^{\ast},r)$ impliziert, dass $F'$ dort auch gleichmäßig stetig ist. Also gibt es zu jedem $\epsilon > 0$ ein $\delta(\epsilon) > 0$, so dass auch
	\begin{align}
		\norm{x+t(x^{\ast}-x) - x}\le \delta(\epsilon) \quad \text{ die Beziehung } \norm{F'(x+t(x^{\ast}-x)) -F'(x)}_{\ast} \le \epsilon\notag
	\end{align}
	für beliebige $x \in B(x^{\ast},r)$ und $t \in [0,1]$ folgt. Damit hat man
	\begin{align}
		\lim_{x\to x^{\ast}}\max_{t\in[0,1]} \norm{F'(x+t(x^{\ast}-x)) -F'(x)}_{\ast} = 0 \notag
	\end{align}
	und
	\begin{align}
		\lim_{x\to x^{\ast}} \frac{\norm{\int_{0}^{1} (F'(x+t(x^{\ast} - x)) -F'(x))(x^{\ast}-x)\diff t}_{\ast}}{\norm{x-x^{\ast}}} = 0 \notag
	\end{align}
	Somit erhält man aus \cref{eq_1_4_10} unter Beachtung von $F(x^{\ast}) = 0$ und der Regularität von $F'(x)$
	\begin{align}
		\lim_{x\to x^{\ast}} \frac{\norm{\Phi(x) - \Phi(x^{\ast})}}{\norm{x-x^{\ast}}O(x-x^{\ast})} = 0,\notag
	\end{align} % different to the script!
	d.h. \cref{eq_1_4_9} ist für $G=0$ erfüllt.
\end{proof}

\begin{remark}
	Falls $F$ in einer Umgebung von $x^{\ast}$ sogar zweimal stetig differenzierbar und damit $\Phi$ dort stetig differenzierbar ist, zeigt \propref{1_1_2}, dass $\norm{\Phi'(x)}_{\ast} \le L$ für alle $x \in D \cap B(x^{\ast},r(L))$ gilt. D.h. die Kontraktionskonstante der Fixpunktabbildung $\Phi$ in \cref{eq_1_3_8} in einer Kugel $B(x^{\ast},r)$ konvergiert gegen $0$, wenn man den Radius $r$ gegen $0$ gehen lässt. Ferner gibt es Sätze, bei denen unter geeigneten Voraussetzungen eine bestimmte lokale Konvergenzgeschwindigkeit (Q-Ordnung) gezeigt wird (etwa die Q-Ordnung $2$, wenn insbesondere $\Phi'$ stetig ist und $\Phi'(x^{\ast}) = 0$ gilt).
\end{remark}

\chapter{Iterative Verfahren für lineare Gleichungssysteme}
Seien eine reguläre Matrix $A \in \R^{n \times n}$ und $b \in Rn$ gegeben. In diesem Kapitel werden iterative Verfahren zur Lösung des linearen Gleichungssystems
\begin{align}
Ax = b\label{eq_2_2_1}
\end{align}
betrachtet.
%TODO fix counter, this equation should be the 1 and the one in the following section, 2!
\section{Fixpunktiteration}

Grundidee dieser Verfahren ist die geeignete Umformulierung des System $Ax = b$ als Fixpunktaufgabe und die Anwendung des gewöhnlichen Iterationsverfahrens. Die hier betrachtete (zu \cref{eq_2_2_1} äquivalente) Fixpunktaufgabe lautet
\begin{align}
	x = x - B^{-1}(Ax - b),\notag
\end{align}
wobei $B \in \R^{n \times n}$ eine noch zu wählende reguläre Matrix ist. Bei Wahl eines Startpunktes $x^0 \in \Rn$ ergibt sich das gewöhnliche Iterationsverfahren damit zu
\begin{align}
	x^{k+1} := x^{k} - B^{-1}(A x^k -b) = (I - B^{-1}A)x^{k} + B^{-1}b, \qquad k = 0,1,2,\dots \label{eq_2_2_2} %TODO add underbraces for M = I - B^{-1}A and c = B^{-1}b
\end{align}
Mit den Bezeichnung $M := I - B^{-1}A$ und $c:= B^{-1}b$ untersuchen wir deshalb die Iterationsvorschrift
\begin{align}
	x^{k+1} := Mx^k + c. \label{eq_2_2_3}
\end{align}
Die zugehörige Fixpunktabbildung $\Phi: \Rn \to \Rn$ ist damit offenbar gegeben durch
\begin{align}
	\Phi(x) := Mx + c.\notag
\end{align}
\begin{proposition}
	\proplbl{2_2_1}
	Es sei $B \in \Rnn$ regulär und mit $M:= I - B^{-1}A$ gelte
	\begin{align}
		\lambda := \norm{M}_{\ast} < 1 \label{eq_2_1_4}
	\end{align}
	wobei $\norm{\cdot}_{\ast}$ die einer Vektornorm $\norm{\cdot}$ zugeordnete Matrixnorm bezeichnet. Dann gilt:
	\begin{enumerate} %TODO add alph
		\item Die für eine beliebiges $x^0 \in \Rn$ durch \cref{eq_2_2_3} erzeugte Folge $\set{x^k}$ konvergiert gegen die eindeutige Lösung $x*$ des linearen Gleichungssystems \cref{eq_2_2_1}.
		\item Die Abschätzungen \cref{eq_1_2_2} - \cref{eq_1_2_4} sind für alle $k \in \N$ erfüllt.
	\end{enumerate}
\end{proposition}

\begin{proof}
	Direkte Folgerung aus dem Banachschen Fixpunktsatz (\propref{1_1_1})
\end{proof}

%%%%%%%%%%%%%%%%%%%%%%%%%%%%%%%%%% 3rd lecture %%%%%%%%%%%%%%%%%%%%%%%%%%%%%%%%%%%%%%%%%%

\begin{remark}
	\proplbl{2_2_2}
	In \propref{2_2_1}a) kann die Folgerung \cref{eq_2_1_4} durch die Bedingung
	\begin{align}
	\rho(M) < 1\label{eq_2_1_5}
	\end{align}
	ersetzt werden. Da
	\begin{align}
		\rho(C) \le \norm{C}_{\ast} \quad \text{ für alle } C \in \Rnn \notag
	\end{align}
	für jede beliebige zugeordnete Matrixnorm $\norm{\cdot}_{\ast}$ gilt (vgl. Übungsaufgabe), ist \cref{eq_2_1_5} eine schwächere Forderung als \cref{eq_2_1_4}. Andererseits gibt es zu jedem Paar $(C,\epsilon) \in \Rnn \times (0,\infty)$ eine zugeordnete Matrixnorm $\norm{\cdot}_{(C,\epsilon)}$, so dass
	\begin{align}
		\norm{C}_{(C,\epsilon)} \le \rho(C) + \epsilon. \notag
	\end{align}
	Dabei ist $\rho(C)$ der \begriff{Spektralradius} der Matrix $C \in \Rnn$, d.h.
	\begin{align}
		\rho(C) := \max_{i = 1,\dots,n}\abs{\lambda_i}, \notag
	\end{align}
	wobei $\lambda_1,\dots,\lambda_n \in \C$ die Eigenwerte der Matrix $C \in \Rnn$ bezeichnen. Man kann weiter zeigen, dass \cref{eq_2_1_5} auch notwendig dafür ist, dass die durch \cref{eq_2_2_2} erzeugte Folge $\set{x^k}$ für jedes $x^0$ gegen $x*$ konvergiert.
\end{remark}

Um eine Matrix $B$ zu finden, so dass einerseits der Aufwand pro Iteration \cref{eq_2_2_2} niedrig und andererseits die Bedingung \cref{eq_2_1_4} bzw. \cref{eq_2_1_5} erfüllt ist, betrachten wir die folgende Zerlegung
\begin{align}
	A = L + D + R \notag
\end{align}

der Matrix $A$, wobei $D:= \diag(a_{11}, \dots, a_{nn})$ die aus den Diagonalelementen von $A$ bestehende Diagonalmatrix bezeichnet und $L$ bzw. $R$ eine untere bzw. obere Dreiecksmatrix ist mit

\begin{align}
	L = 
	\begin{pmatrix}
		0      &        &        &           & \\
		a_{21} & 0      &        &           & \\
		a_{31} & a_{32} & 0      &           & \\
		\vdots &        & \ddots & \ddots    & \\
		a_{n1} & \cdots & \cdots & a_{n,n-1} & 0
	\end{pmatrix}
	 \bzw R = 
	 \begin{pmatrix}
	 0      & a_{12} & a_{13} & \dots     & a_{1n}\\
	        & 0      & a_{23} & \dots     & a_{2n}\\
	        &        & \ddots & \ddots    & \vdots\\
	        &        &        & 0         & a_{n-1,n}\\
	        &        &        & a_{n,n-1} & 0
	 \end{pmatrix}.\notag
\end{align}

\subsection{Das \person{Jacobi}-Verfahren}
Wir setzen hier voraus, dass $D$ regulär ist und wählen
\begin{align}
	B:= D \label{eq_1_2_6}
\end{align}
Damit ergibt sich die Iterationsvorschrift
\begin{align}
	x^{k+1} = x^k - D^{-1}(Ax^k - b) = -D^{-1}(L+R)x^k+D^{-1}b. \label{eq_1_2_7}
\end{align}
In \cref{eq_2_2_3} ist entsprechend
\begin{align}
	M:= M_J := -D^{-1}(L+R) \text{ und } c:= c_J := D^{-1}b\notag 
\end{align}
zu wählen. Dieses Verfahren heißt \begriff{Gesamtschrittverfahren} oder \begriff{Jacobi-Verfahren}. Der Aufwand pro Schritt (Berechnung von $x^{k+1}$ aus $x^k$) beträgt $\Landau(n^2)$ bei voll besetzter Matrix $A$ und mindestens $\Landau(n)$, falls $A$ schwach besetzt ist.

\begin{proposition}
	Die Matrix $A$ sei streng diagonaldominant (vgl. Definition 3.1 der Vorlesung ENM). Dann ist die Matrix $B$ aus \cref{eq_1_2_6} regulär und es gilt
	\begin{align}
	\norm{M_J}_{\infty} \le \lambda_{SD} := \max_{i = 1,\dots,n} \frac{1}{\abs{a_{ii}}} \sum_{\substack{j =1 \\ j\neq i}}^n \abs{a_{ij}} < 1. \notag
	\end{align}
\end{proposition}

\begin{proof}
	Die Regularität von $B$ ergibt sich sofort aus der strengen Diagonaldominanz von $A$. Nutzt man die Definition der Zeilensummennorm $\norm{\cdot}_{\infty}$ erhält man sofort
	\begin{align}
		\norm{M_J}_{\infty} = \norm{D^{-1}(L+R)}_{\infty} =\max_{i = 1,\dots,n} \frac{1}{\abs{a_{ii}}} \sum_{\substack{j =1 \\ j\neq i}}^n \abs{a_{ij}}  = \lambda_{SD}.\notag
	\end{align}
	Die vorrausgesetzte strenge Diagonaldominanz von $A$ sichert $\lambda_{SD} < 1$.
\end{proof}

\subsection{Das \person{Gauss-Seidel}-Verfahren}
Wir setzen hier voraus, dass $L + D$ regulär ist und wählen

\begin{align}
	B := L + D \label{1_2_8}
\end{align}

Damit ergibt sich die Iterationsvorschrift

\begin{align}
	x^{k+1} = x^k - (L+D)^{-1}(Ax^k - b) = - (L+D)^{-1}R x^k + (L+D)^{-1}b. \label{1_2_9}.
\end{align}

In \cref{eq_2_2_3} ist entsprechend

\begin{align}
	M:= M{GS} := - (L+D)^{-1}R \text{ und } c:= c_{GS} := (L+D)^{-1}b \notag
\end{align}

zu wählen. Dieses Verfahren heißt \begriff{Einzelschrittverfahren} oder \begriff{Gauß-Seidel-Verfahren}. Der Aufwand pro Schritt beträgt im ungünstigsten Fall $\Landau(n^2)$. Verbesserungen sind möglich, wenn eine Sparse-Struktur in $A$ ausgenutzt werden kann.

\begin{proposition}
	Die Matrix $A$ sei streng diagonaldominant ($\nearrow$ Definition 3.1 der Vorlesung ENM). Dann ist die Matrix $B$ aus \cref{1_2_8} regulär und es gilt
	\begin{align}
		\norm{M_{GS}}_{\infty} \le \lambda_{SD} <1. \notag
	\end{align}
\end{proposition}

\begin{proof}
	Die Regularität von $B$ folgt sofort aus der strengen Diagonaldominanz von $A$. Weiter ergibt sich
	\begin{align}
		\norm{M_{GS}}_{\infty} = \norm{(L+D)^{-1}R}_{\infty} = \sup_{\norm{y}_{\infty}=1} \norm{(L+D)^{-1}Ry}_{\infty}. \notag
	\end{align}
	Um für einen festen Vektor $y$ mit $\norm{y}_{\infty} = 1$ eine Abschätzung für die rechte Seite zu erhalten, setzen wir $z:= (L+D)^{-1}Ry$. Damit gilt
	\begin{align}
		(D+L)z = Ry \label{1_2_10}
	\end{align}
	und
	\begin{align}
	z_1 = \frac{1}{a_{11}} \sum_{j=1}^{n} a_{1j}y_j. \notag
	\end{align}
	Daraus folgt (da $\lambda_{SD} < 1$ wegen der strengen Diagonaldominanz von $A$)
	\begin{align}
		\abs{z_1} 
		\le \frac{1}{\abs{a_{11}}} \sum_{j=2}^{n} \abs{a_{1j}}\abs{y_j} 
		\le \sum_{j=2}^{n} \abs{a_{1j}} \le \lambda_{SD} < 1.\notag
	\end{align}
	Nehmen wir nun an, dass
	\begin{align}
		\abs{z_1} \le \text{ für } i = 1, \dots, k-1, \notag
	\end{align}
	für ein $k \in \set{2,\dots,n}$ gilt. Dann folgt wegen \cref{1_2_10} und $\norm{y}_{\infty} = 1$
	\begin{align}
		\abs{z_k} = \frac{1}{\abs{a_{kk}}} \abs{-\sum_{i=1}^{k-1} a_{ki}z_i + \sum_{i=k+1}^{n} a_{ki}y_i} 
		\le \frac{1}{\abs{a_{kk}}} \brackets{\sum_{i=1}^{k-1} \abs{a_{ki}} + \sum_{i=k+1}^{n} \abs{a_{ki}}} \le \lambda_{SD}. \notag
	\end{align}
	Somit hat man induktiv $\abs{z_k} \le \lambda_{SD}$ für $k = 1, \dots, n$ und damit
	\begin{align}
		\norm{(L+D)^{-1}Ry}_{\infty} = \norm{z}_{\infty} \le \lambda_{SD} \notag
	\end{align}
	für beliebige $y$ mit $\norm{y}_{\infty} = 1$.
\end{proof}

\subsection{SOR-Verfahren}

Um dieses verfahren zu beschreiben, nehmen wir an, dass für ein $\omega \neq 0$ die Matrix

\begin{align}
	B:=L + \frac{1}{\omega}D \label{eq_2_2_11}
\end{align}
regulär ist. Damit ergibt sich die Iterationsvorschrift

\begin{align}
	x^{k+1} := x^k - \brackets{L+\frac{1}{\omega}D}^{-1}(Ax^k - b) = M(\omega)x^k + c(\omega)\notag
\end{align}

\begin{align}
	M(\omega) := I -  \brackets{L + \frac{1}{\omega} D^{-1} A} 
	= \brackets{L + \frac{1}{\omega} D}^{-1}
%	\brackts{\frac{1-\omega}{\omega} D - R} \notag
\end{align}

und

\begin{align}
	c(\omega) :=  \brackets{L+\frac{1}{\omega}D}^{-1}b.
\end{align}

Für $\omega = 1$ erhält man offenbar als Spezialfall das Gauß-Seidel-Verfahren, so dass der folgende Satz auch dafür Anwendung finden kann. Man beachte dazu \propref{2_2_2}.

\begin{proposition}
	Die Matrix $A$ sei symmetrisch und positiv definit. Dann ist die Matrix $B$ aus \cref{eq_2_2_11} regulär (für jedes $\omega \neq 0$). Falls $\omega \in (0,2)$, dann gilt
	\begin{align}
		\rho(M(\omega)) < 1\notag
	\end{align}
	und umgekehrt.
\end{proposition}

\begin{proof}
	Da $A$ positiv definit ist, gilt $e_i^TA e_i = a_{ii} > 0$ für $i = 1, \dots, n$. Also ist $D$ positiv definit und damit $B$ regulär für alle $\omega \neq 0$.\\
	Sei $\lambda \in \C$ ein Eigenwert von $M(\omega)$ und $z \in \Cn$ ein zugehöriger Eigenvektor. Mit
	\begin{align}
		A = A - M(\omega)^T AM(\omega) + M(\omega)^T AM(\omega) \notag
	\end{align}
	sowie (unter Berücksichtigung der Definition von $M$ und von $A = A^T$ und $R = L^T$)
	\begin{align}
	A - M(\omega)^T AM(\omega) 
	&= A - (I - B^{-1}A)^T A(I-B^{-1}A)\notag\\
	&= AB^T A + AB^{-1}A - AB^{T-1}AB^{-1}A\notag\\
	&= (B^{-1}A)^T (B + B^T - A)(B^{-1}A)\notag\\
	&= (B^{-1}A)^T \brackets{L + \frac{1}{\omega}D + L^T + \frac{1}{\omega}D - L - D - L^T})(B^{-1}A)\notag\\
	&= (B^{-1}A)^T\brackets{\frac{2 -\omega}{\omega}D}(B^{-1}A)\notag
	\end{align}
	ergibt sich daher
	\begin{align}
		z^H Az = (AB^{T-1}z)^H\brackets{\frac{2 -\omega}{\omega}D}(B^{-1}Az) + z^H M(\omega)^T AM(\omega)z.\notag
	\end{align}
	Da die Diagonalmatrix $D$ positiv-definit ist, besitzt $\frac{2-\omega}{\omega}D$ dieselbe Eigenschaft für $\omega \in (0,2)$. Es folgt
	\begin{align}
		(AB^{T-1}z)^H \brackets{\frac{2 -\omega}{\omega}D} (B^{-1 Az}) > 0 \notag
	\end{align}
	und damit
	\begin{align}
		\abs{\lambda} < 1.
	\end{align}
	Also gilt $\rho(M(\omega)) = \max_{i = 1,\dots,n}\abs{\lambda_i} < 1$, sofern $\omega \in (0,2)$. Die Umkehrung der Aussage ergibt sich aus dem Satz von \person{Kahan} ($\nearrow$ Übungsaufgabe). %TODO add later.
\end{proof}

Es ist nun naheliegend, dass man $\omega \in(0,2)$ so wählen möchte, dass $\rho(\omega)$ möglichst klein ist. Dies ist in bestimmten Fällen näherungsweise möglich, ansonsten beschränkt man sich auf geeignete Heuristiken zur Wahl von $\omega$. Auf der Fixpunktiteration \cref{eq_2_2_3} beruhende Verfahren werden häufig auch \begriff{Splitting-Methoden} genannt. Es gibt noch weitere solche Verfahren, auf die hier nicht eingegangen wird.
\subsection{Koordinatenweise Darstellung der Splitting Methode}
\begin{align*}
	0 = Ax \cdot b = \begin{pmatrix}
	a_{11}x_1 &+ a_{12}x_2 &+ \cdots &+ a_{1n} x_n &- b_1\\
	\vdots & & \vdots & \vdots\\
	a_{n1}x_1 &+ a_{n2}x_2 &+ \cdots &+ a_{nn}x_n &-b_n
	\end{pmatrix}
\end{align*}
\begin{enumerate}
	\item 
\end{enumerate}
\section{\person{Krylov}-Raum-basierte Verfahren}

\subsection{\person{Krylov}-Räume}

\subsection{Basisalgorithmen zur Lösung von $Ax=b$}

\subsection{Das CG-Verfahren}

\subsection{Fehlerverhalten des CG-Verfahrens}

\subsection{Vorkonditionierung}

\subsection{Ausblick und Anmerkungen}

\chapter{Numerische Behandlung von Anfangswertaufgaben}
\section{Aufgabe und Lösbarkeit}

Es seien $a,b \in \R$ mit $a < b$, eine stetige Funktion $f$: $[a,b] \times \R^m \to \R^m$ und $y^0 \in \R^m$ gegeben. Unter \begriff{Anfangswertaufgabe} (AWA) 1. Ordnung versteht man das Problem, eine stetige Funktion $y$: $[a,b] \to \R^m$ zu ermitteln, so dass $y$ auf $(a,b)$ stetig differenzierbar ist und
\begin{align}
	y'(x) = f(x,y(x)) \quad \mit \quad y(a)=y^0 \notag
\end{align}
für alle $x \in [a,b]$ gilt. Eine solche Funktion wollen wir \begriff[Anfangswertaufgabe!]{Lösung} der AWA nennen. Kürzer schreibt man für die AWA auch
\begin{align}
	\label{3_1_1}
	y'=f(x,y) \quad \mit \quad y(a)=y^0
\end{align}
Die Existenz und Eindeutigkeit einer Lösung einer AWA hängen von den Eingangsinformationen $a,b,f$ und $y^0$ ab. Es gilt folgender Satz zur (globalen) Existenz und Eindeutigkeit einer Lösung auf $[a,b]$:

\begin{proposition}[\person{Picard-Lindelöf}: eine globale Version]
	Es sein $f$: $[a,b] \times \R^m \to \R^m$ stetig und es existiere $L>0$, so dass
	\begin{align}
		\label{3_1_2}
		\norm{f(x,y)-f(x,z)} \le L \norm{y-z} \quad \forall (x,y),(x,z) \in [a,b] \times \R^m
	\end{align}
	Dann besitzt \cref{3_1_1} für jedes $y^0 \in \R^m$ eine eindeutige Lösung.
\end{proposition}

Die Bedingung \cref{3_1_2} ist eine globale Lipschitz-Bedingung an $f$ bezüglich der zweiten Veränderlichen. Es ist leicht, AWA anzugeben, in denen diese Bedingung nicht erfüllt ist und keine Lösung in ganz $[a,b]$ existiert, zum Beispiel
\begin{align}
	y'= y^2 \quad \mit \quad y(0) = 1 \notag
\end{align}
Dafür erhält man für beliebige $x,y,z \in \R$
\begin{align}
	\abs{f(x,y) - f(x,z)} = \abs{y^2 - z^2} = \abs{y+z} \abs{y-z} \notag
\end{align}
das heißt die Bedingung \cref{3_1_2} kann in diesem Beispiel (global) nicht gelten. Die Lösung der AWA lautet $y(x) = \sfrac{-1}{x-1}$ für $x \in [0,1)$. Für Intervalle $[0,b]$ mit $b \ge 1$ existiert keine Lösung. Eine Abschwächung der Lipschitz-Bedingung \cref{3_1_2} gestattet folgender

\begin{proposition}[\person{Picard-Lindelöf}: eine lokale Version]
	Es sei $f$: $[a,b] \times \R^m \to \R^m$ stetig und zu jeder kompakten Menge $\mathcal{Y} \subset \R^m$ existiere $L_Y > 0$, so dass
	\begin{align}
		\norm{f(x,y)-f(x,z)} \le L_y \norm{y-z} \quad \forall (x,y),(x,z) \in [a,b] \times \mathcal{Y} \notag
	\end{align}
	Dann gibt es für jedes $y^0 \in \R^m$ ein Teilintervall $\mathcal{I} \subseteq [a,b]$ mit $a \in \mathcal{I}$, so dass die AWA \cref{3_1_1} auf $\mathcal{I}$ eine eindeutige Lösung besitzt.
\end{proposition}

Seien $g$: $[a,b]\times \R^n \to \R$ stetig und $\eta \in \R^n$. Jede explizite Differentialgleichung $n$-ter Ordnung
\begin{align}
	y^{(n)} = g(x,y,y',y'',...,y^{(n-1)})\notag
\end{align}
mit den Anfangsbedingungen
\begin{align}
	y(a) = \eta_1, \quad y'(a) = \eta_2, \quad y''(a) = \eta_3, \quad \dots \quad y^{(n-1)}(a) = \eta_n\notag
\end{align}
kann mittels Substitution
\begin{align}
	y_1 = y,\quad y_2=y', \quad y_3=y'', \quad \dots \quad y_n = y^{(n-1)}\notag
\end{align}
in eine AWA 1. Ordnung überführt werden:
\begin{align}
	\begin{pmatrix}
		y_1' \\ \vdots \\ y_n'
	\end{pmatrix} = 
	\begin{pmatrix}
		y_2 \\ \vdots \\ y_n \\ g(x,y_1,...,y_n)
	\end{pmatrix} 
	\quad \mit \quad 
	\begin{pmatrix}
	y_1(a) \\ \vdots \\ y_n(a)
	\end{pmatrix} = 
	\begin{pmatrix}
	\eta_1 \\ \vdots \\ \eta_n
	\end{pmatrix} \notag
\end{align}
\section{Einschrittverfahren}
\subsection{Grundlagen}

Anstelle der gesuchten Lösungsfunktion $y$: $[a,b]\to\real^m$ einer AWA ist man an möglichst guten Näherungen $y^k\in\real^m$ ($k=0,1,2,...,N$) für die Funktionswerte $y(x_k)\in\real^m$ der Funktion $y$ an \begriff{Gitterpunkten} $x_k\in [a,b]$ interessiert. Auf Grundlage der Paare $(x_k,y^k)$ ($k=0,1,...,N-1$) ist es auch möglich, eine Näherungsfunktion $y$ zu erzeugen (etwa durch Interpolation).

Einschrittverfahren bilden eine Klasse von Verfahren, die Näherungen $y^k$ zu erzeugen. Das \begriff{Gitter} $\{x_0,...,x_N\}$ is so gewählt, dass
\begin{align}
	x_0=a<x_1<x_2<\dots < x_{N-1} < x_N = b\notag
\end{align}
Außerdem setzen wir
\begin{align}
	h_k = x_{k+1}-x_k\quad\text{für} k=0,...,N-1\notag
\end{align}
und bezeichnen $h_k$ als \begriff{Schrittweite}. Falls $h=h_0=\dots=h_{N-1}$, so heißen die Gitterpunkte bzw. das Gitter \begriff[Gitter!]{gleichabständig} oder \begriff[Gitter!]{äquidistant}.

Ein Verfahren zur Erzeugung einer Folge $y^0,...,y^N$ heißt \begriff{Einschrittverfahren} für das AWA \cref{3_1_1}, wenn
\begin{align}
	\label{3_1_3}
	y^{k+1} = y_k + h_k\Phi(x_k,y_k,y^{k+1},h_k) \quad\text{für} k=0,...,N-1
\end{align}
Dabei bezeichnet $\Phi(x,y,z,h)$ den Funktionswert einer \begriff{Verfahrensfunktion}
\begin{align}
	\Phi: [a,b]\times\real^m\times \real^m\times (0,b-a]\to \real^m\notag
\end{align}
die das jeweilige Einschrittverfahren definiert. Man beachte, dass $y^0$ bereits durch die Anfangsbedingung in \cref{3_1_1} gegeben ist. Ein Einschrittverfahren heißt \begriff[Einschrittverfahren!]{implizit}, falls $\Phi$ tatsächlich von $z$ abhängt. Dann ist zur Bestimmung von $y^{k+1}$ aus \cref{3_1_3} die Lösung eines im Allgemeinen nichtlinearen Gleichungssystems erforderlich. Falls $\Phi$ nicht von $z$ abhängt, heißt das Einschrittverfahren \begriff[Einschrittverfahren!]{explizit}. Das explizite \begriff{\person{Euler}-Verfahren} (auch \begriff{Polygonzugverfahren} genannt) ist gegeben durch
\begin{align}
	\label{3_1_4}
	\Phi(x,y,z,h) = f(x,y)
\end{align}
das heißt
\begin{align}
	y^{k+1} = y^k + h_kf(x_k,y^k) \notag
\end{align}
Für das implizite \person{Euler}-Verfahren gilt die Vorschrift
\begin{align}
	y^{k+1} = y^k + h_kf(x_k + h_k,y^{k+1}) \notag
\end{align}
Um die Güte der Näherungen $y^k$ zu beurteilen, untersuchen wir zunächst den lokalen Diskretisierungsfehler eines Einschrittverfahrens.

\subsection{Lokaler Diskretisierungsfehler und Konsistenz}

\begin{definition}[lokaler Diskretisierungsfehler]
	Seien $y$: $[a,b]\to \real^m$ Lösung des Differentialgleichung $y'=f(x,y)$ und $\Phi$ die Verfahrensfunktion eines Einschrittverfahrens. Für $x\in[a,b)$ und $h>0$ mit $x+h\le b$ heißt
	\begin{align}
		\label{3_1_5}
		\Delta(x,h) = y(x+h) - \bigg( y(x) + h\Phi\big(x,y(x),y(x+h),h\big)\bigg)
	\end{align}
	\begriff{lokaler Diskretisierungsfehler} und 
	\begin{align}
		\frac{\Delta(x,h)}{h} = \frac{y(x+h)-y(x)}{h} - \Phi(x,y(x),y(x+h),h)
	\end{align}
	relativer lokaler Diskretisierungsfehler des Einschrittverfahrens.
\end{definition}

Der lokale Diskretisierungsfehler gibt also die Abweichung zwischen exakter Lösung $y(x+h)$ an der Stelle $x+h$ und der Näherung an dieser Stelle an, wobei angenommen wird, dass die Näherung unter Verwendung der exakten Lösung $y(x)$ (und ggf. $y(x+h)$) berechnet wird. Die Bezeichnung relativer Diskretisierungsfehler ist bezüglich der Schrittweite $h$ zu verstehen.

\begin{definition}[konsistent, Konsistenzordnung]
	Ein Einschrittverfahren heißt \begriff{konsistent} zur Differentialgleichung $y'=f(x,y)$, wenn
	\begin{align}
		\lim\limits_{h\downarrow 0} \left\Vert\frac{\Delta(x,h)}{h}\right\Vert =0\quad\forall x\in [a,b) \notag
	\end{align}
	für jede Lösung $y$: $[a,b]\to\real^m$ der Differentialgleichung gilt. Gibt es außerdem $p\ge 1$, $M>0$, $\tilde{h}>0$, so dass
	\begin{align}
		\left\Vert\frac{\Delta(x,h)}{h}\right\Vert \le Mh^p\quad\forall (x,h)\in [a,b)\times (0,\tilde{h})\text{ mit } x+h\le b \notag
	\end{align}
	für jede Lösung $y$: $[a,b]\to\real^m$ der Differentialgleichung gilt, so hat das Einschrittverfahren (für diese Differentialgleichung) die \begriff{Konsistenzordnung} $p$.
\end{definition}

\begin{proposition}
	Sei $f$: $[a,b]\times \real^m\to\real^m$ stetig differenzierbar. Dann hat das explizite \person{Euler}-Verfahren die Konsistenzordnung 1.
\end{proposition}
\begin{proof}
	Mit \cref{3_1_4} folgt
	\begin{align}
		\Delta(x,h) = y(x+h) - y(x) - hf(x,y(x)) \notag
	\end{align}
	Da $y$ die Differentialgleichung $y'=f(x,y)$ löst und $f$ stetig differenzierbar ist, muss $y$ zweimal stetig differenzierbar sein. Aus der \person{Taylor}-Formel erhält man für $i\in\{1,...,m\}$
	\begin{align}
		\Delta(x,h)_i &= y'_i(x)h + \frac{1}{2}y''_i(\xi_i(x,h))h^2 - hf_i(x,y(x)) \notag \\
		&= \frac{1}{2} y''_i(\xi_i(x,h))h^2 \notag
	\end{align}
	für ein $\xi_i(x,h)\in (x,x+h)$. Die Stetigkeit von $y''$ auf $[a,b]$ und Division durch $h$ liefert die Behauptung mit $M=\frac{1}{2}\max_{1\le i\le m}\max_{\xi\in[a,b]}\Vert y''_i(\xi)\Vert$ und $\tilde{h}=b-a$.
\end{proof}

\subsection{Konvergenz von Einschrittverfahren}

Zum Gitter $G=\{x_0,...,x_N\}\subset [a,b]$ mit $x_0=a$ und $x_N=b$ seien $y^0,...,y^N\in\R^m$ durch ein Einschrittverfahren erzeugt. Weiter bezeichne $y$: $[a,b]\to\R^m$ die eindeutige Lösung der AWA \cref{3_1_1}. Dann seien
\begin{align}
	e(x_k) &= y(x_k)-y^k \notag \\
	e(G) &= \max_{x\in G}\norm{e(x)} \notag
\end{align}
sowie
\begin{align}
	h_{max}(G) = \max_{k=0,...,N-1} h_k\notag
\end{align}
definiert.

\begin{definition}[konvergent]
	Die AWA \cref{3_1_1} besitzt die eindeutige Lösung $y$: $[a,b]\to\R^m$. Ein Einschrittverfahren für diese AWA heißt dann \begriff[Einschrittverfahren!]{konvergent}, falls
	\begin{align}
		\lim_{l\to\infty} e(G_l)=0\notag
	\end{align}
	für alle Gitterfolgen $\{G_t\}$ gilt, für die $\lim_{l\to\infty} h_{max}(G_l)=0$. Gibt es außerdem $p\ge 1$, $C>0$, $\tilde{h}>0$, so dass
	\begin{align}
		e(G) \le C\cdot h_{max}(G)^p\notag
	\end{align}
	für jedes Gitter mit $h_{max}(G)\le\tilde{h}$, so hat das Einschrittverfahren für die gegebene AWA die \begriff[Einschrittverfahren!]{Konvergenzordnung} $p$.
\end{definition}

\begin{lemma}[diskretes \person{Grönwall}sches Lemma]
	\proplbl{3_2_5}
	Falls die Zahlenfolgen $\{\alpha_k\}$, $\{\beta_k\}$, $\{v_k\}\subset [0,\infty)$ den Bedingungen
	\begin{align}
		v_0=0\quad\text{und}\quad v_{k+1}=(1+\alpha_k)v_k+\beta_k\quad\forall k=0,...,N-1\notag
	\end{align}
	genügen, dann folgt
	\begin{align}
		v_{k+1} \le \sum_{i=0}^{k} \beta_i\cdot \exp\left(\sum_{j=i+1}^{k} \alpha_j\right)\quad\text{für } k=0,...,N-1\notag
	\end{align}
	gilt zusätzlich $\alpha_k=\alpha>0$ und $\beta_k=\beta>0$ für jedes $k=0,...,N-1$, dann folgt
	\begin{align}
		v_k \le \frac{\beta}{\alpha}(\exp(k\alpha)-1)\quad\text{für } k=0,...,N-1\notag
	\end{align}
\end{lemma}
\begin{proof}
	Zum Beispiel durch vollständige Induktion (vgl. Übungsaufgabe).
\end{proof}

In der Literatur findet man für vorstehende und ähnliche Aussagen die Bezeichnung \textit{diskretes \person{Grönwall}sches Lemma}.

\begin{proposition}
	Die AWA \cref{3_1_1} besitze die eindeutige Lösung $y$: $[a,b]\to\R^m$. Ein Einschrittverfahren mit der Verfahrensfunktion $\Phi$ habe für die Differentialgleichung $y'=f(x,y)$ die Konsistenzordnung $p$. Es gebe ferner $L_\Phi>0$ und $H>0$, so dass die Lipschitz-Bedingung
	\begin{align}
		\label{3_1_6}
		\norm{\Phi(x,y,z,h) - \Phi(x,\tilde{y},\tilde{z},h)} \le L_\Phi(\norm{y-\tilde{y}} + \norm{z-\tilde{z}})
	\end{align}
	für alle $(x,y,z,h),(x,\tilde{y},\tilde{z},h)\in [a,b]\times \R^m\times \R^m\times (0,H]$ gilt. Dann besitzt das Einschrittverfahren die Konvergenzordnung $p$.
\end{proposition}
\begin{proof}
	Entsprechend \cref{3_1_3} und \cref{3_1_5} gilt
	\begin{align}
		y^{k+1} = y^k + h_k\Phi(x_k,y^k,y^{k+1},h_k) \notag
	\end{align}
	und
	\begin{align}
		y(x_{k+1}) = y(x_k) + h_k\Phi(x_k,y(x_k),y(x_k+h_k),h_k) + \Delta(x_k,h_k) \notag
	\end{align}
	also folgt
	\begin{align}
		e(x_{k+1}) &= y(x_{k+1}) - y^{k+1} \notag \\
		&= y(x_k) - y^k + h_k\big(\Phi((x_k,y(x_k),y(x_k+h_k),h_k) - \Phi(x_k,y^k,y^{k+1},h_k)\big) + \Delta(x_k,h_k) \notag
	\end{align}
	und weiter mit \cref{3_1_6} für $0<h_k\le \tilde{h}=\min\{H,\tilde{h},\frac{1}{2L_\Phi}\}$
	\begin{align}
		\norm{e(x_{k+1})} \le \norm{e(x_k)} + \norm{\Delta(x_k,h_k)} + h_kL_\Phi(\norm{e(x_k)} + \norm{e(x_{k+1})}) \notag
	\end{align}
	Durch Umstellen und Beachtung der Konsistenzordnung ergibt sich
	\begin{align}
		\label{3_1_7}
		\norm{e(x_{k+1})} \le \frac{1+h_kL_\Phi}{1-h_kL_\Phi}\norm{e(x_k)} + \frac{M}{1-h_kL_\Phi}h_k^{p+1}
	\end{align}
	Mit $\alpha_k=4h_kL_\Phi$ hat man (wegen $2h_kL\Phi\le 1$)
	\begin{align}
		\frac{1+h_kL_\Phi}{1-h_kL_\Phi} = 1 + \frac{2h_kL_\Phi}{1-h_kL_\Phi} \le 1+\alpha_k \notag
	\end{align}
	Setzt man weiter $v_k=\norm{e(x_k)}$ und $\beta_k=2Mh_k^{p+1}$, so erhält man aus \cref{3_1_7}
	\begin{align}
		v_{k+1} \le (1+\alpha_k)v_k + \beta_k\quad k=0,...,N-1\notag
	\end{align}
	Nach \propref{3_2_5} folgt daraus (beachte $v_0=\norm{e(x_0)} = \norm{y(x_0) - y^0} = 0$)
	\begin{align}
		v_{k+1} \le \left(\sum_{i=0}^k \beta_i\right)\exp\left(\sum_{i=0}^k \alpha_i\right) \quad\text{für } k=0,...,N-1\notag
	\end{align}
	und damit
	\begin{align}
		\norm{e(x_{k+1})} = v_{k+1} &\le 2M\left(\sum_{i=0}^k h_i^{p+1}\right)\exp\left(4L_\Phi\sum_{i=0}^k h_i\right) \notag \\
		&\le 2Mh_{max}(G)^p(x_{k+1}-x_0)\exp(4L_\Phi(x_{k+1} - x_0)) \notag
	\end{align}
	für $k=0,...,N-1$ sowie
	\begin{align}
		e(G) \le 2M(b-a)\exp(4L_\Phi(b-a))h_{max}(G)^p \notag
	\end{align}
	Also besitzt das Einschrittverfahren die Konvergenzordnung $p$.
\end{proof}

\subsection{Stabilität gegenüber Rundungsfehlern}

Wir betrachten das Einschrittverfahren \cref{3_1_3} für ein gleichabständiges Gitter ($h_k=h$) bei exakter Rechnung, das heißt
\begin{align}
	\label{3_1_8}
	y^{k+1} = y^k + h\Phi(x_k,y^k,y^{k+1},h)\quad\text{für } k=0,...,N-1
\end{align}
Weiter beschreibe
\begin{align}
	\label{3_1_9}
	\tilde{y}^0 = y^0\quad\text{und}\quad \tilde{y}^{k+1}=\tilde{y}^k + h\Phi(x_k,\tilde{y}^k,\tilde{y}^{k+1},h)+\epsilon_k \quad\text{für } k=0,...,N-1
\end{align}
ein gestörtes Verfahren, das heißt $\tilde{y}^1,...,\tilde{y}^N$ sind die tatsächlich im Computer berechneten Größen.

\begin{proposition}
	\proplbl{3_2_7}
	Sei $y^o\in\R^m$ gegeben und $y^1,...,y^N$ sowie $\tilde{y}^1,...,\tilde{y}^N$ entsprechend \cref{3_1_8} und \cref{3_1_9} berechnet, wobei $\norm{\epsilon_k}<\epsilon$ für alle $k=0,...,N-1$ mit einem $\epsilon>0$ gelte. Außerdem sei für gewisse $L_\Phi,H>0$ die Lipschitz-Bedingung \cref{3_1_6} für alle $(x,y,z,h),(x,\tilde{y},\tilde{z},h)\in [a,b]\times \R^m\times \R^m\times (0,H]$ erfüllt. Dann gibt es $\tilde{h}>0$, so dass
	\begin{align}
		\norm{y^k-\tilde{y}^k} \le \frac{\epsilon}{2hL_\Phi}(\exp(4L_\Phi(x_k-a)) - 1)\quad\text{für } k=0,...,N\notag
	\end{align}
	falls $0>h<\tilde{h}$.
\end{proposition}
\begin{proof}
	Für $z^k=y^k-\tilde{y}^k$ folgt
	\begin{align}
		z^{k+1} &= y^{k+1} - \tilde{y}^{k+1} \notag \\
		&= y^k - \tilde{y}^k + h\big(\Phi(x_k,y^k,y^{k+1},h) - \Phi(x_k,\tilde{y}^k,\tilde{y}^{k+1},h)\big) - \epsilon_k \notag
	\end{align}
	und weiter
	\begin{align}
		\norm{z^{k+1}} \le \norm{z^k} + hL_\Phi(\norm{z^k} + \norm{z^{k+1}}) + \epsilon \notag
	\end{align}
	Mit $v_k=\norm{z^k}$, $\alpha=4hL_\Phi$ und $\beta=2\epsilon$ hat man für $0<h\le\tilde{h}=\min{H,\frac{1}{2L_\Phi}}$ die Differenzenungleichung
	\begin{align}
		v_{k+1} \le (1+\alpha)v_k + \beta \notag
	\end{align}
	für $k=0,...,N-1$. \propref{3_2_5} liefert
	\begin{align}
		\norm{y^k-\tilde{y}^k} = \norm{z^k} &= v_k\notag \\
		&\le \frac{\epsilon}{2hL_\Phi}(\exp(k4hL_\Phi) - 1) \notag \\
		&= \frac{\epsilon}{2hL_\Phi}(\exp(4L_\Phi(x_k-a)) - 1) \notag
	\end{align}
\end{proof}

Selbst wenn die Abschätzung in \propref{3_2_7} nicht scharf ist, muss man damit rechnen, dass der Rundungsfehler wie $\sfrac{1}{h}$ wächst. Der Gesamtfehler eines Einschrittverfahrens ans einer Stelle $x_k$ setzt sich aus dem Verfahrensfehler $\norm{y(x_k)-y^k}$ und dem Rundungsfehler $\norm{y^k-\tilde{y}^k}$ zusammen. Für ein Verfahren der Konvergenzordnung $p$ ergibt sich also (bei äquidistantem Gitter) für den Gesamtfehler
\begin{align}
	\norm{y(x_k)-\tilde{y}^k} &\le \norm{y(x_k)-y^k} + \norm{y^k-\tilde{y}^k} \notag \\
	&\le Ch^p + \tilde{C}\frac{\epsilon}{h} \notag
\end{align}
Minimiert man die rechte Seite der Abschätzung in Abhängigkeit von $h$, so folgt, dass man $h$ nicht kleiner als
$\sim \sqrt[p+1]{\epsilon}$ wählen sollte. Setzt man speziell $h=\sqrt[p+1]{\epsilon}$, dann folgt
\begin{align}
	Ch^p + \tilde{C}\frac{\epsilon}{h} = C\exp\left(\frac{p}{p+1}\right) + \tilde{C}\exp\left(\frac{p}{p+1}\right) \notag
\end{align}
Durch Erhöhung der Konvergenzordnung $p$ kann man also versuchen, mit einer größeren Schrittweite einen kleineren Gesamtfehler zu erreichen. Ein weiterer Grund für das Interesse an Verfahren mit höherer Konvergenzordnung liegt in der Möglichkeit, die Gesamtzahl der erforderlichen Funktionswertbestimmungen der Funktion $f$ zu verringern.

\subsection{\person{Runge-Kutta}-Verfahren}
\section{Mehrschrittverfahren}

\subsection{Grundlagen}

Bei Mehrschrittverfahren (MSV) wird eine Näherung $y^{k+l}$ für $y(x_{k+l})$ in bestimmter Weise aus $l$ vorhergehenden Näherungen $y^k,y^{k+1},...,y^{k+l-1}$ bestimmt. Um dies genau zu beschreiben, seien zusätzlich zu $y^0$ (aus AWA) die Startwerte $y^1,...,y^{l-1}\in\R^m$ gegeben. Im Folgenden wollen wir von einem äquidistanten Gitter $G_h=\{x_0,...,x_N\}$ mit Schrittweite $h=\frac{b-a}{N}$ ausgehen. Ein \begriff{lineares Mehrschrittverfahren} mit $l$ Schritten erzeugt dann für $k=0,...,N-l$ die Iterierte $y^{k+l}$ aus $y^k,y^{k+1},...,y^{k+l-1}$ entsprechend
\begin{align}
	\label{3_17}
	\sum_{\nu=0}^{l} \alpha_\nu y^{k+\nu} = h\sum_{\nu=0}^l \beta_\nu f(x_{k+\nu},y^{k+\nu})
\end{align}
wobei $\alpha_\nu$, $\beta_\nu$ ($\nu=0,...,l$) reelle Parameter sind mit $\alpha_l\neq 0$ und $\abs{\alpha_0}+\abs{\beta_0}\neq 0$. Falls $\beta_l=0$, dann spricht man von einem \begriff[lineares Mehrschrittverfahren!]{expliziten} (sonst \begriff[lineares Mehrschrittverfahren!]{impliziten}) linearen MSV. Die MSV \cref{3_17} heißen linear, da die rechte Seite von \cref{3_17} linear von den Funktionswerten $f(x_{k+\nu},y^{k+\nu})$ abhängt. Einem linearen MSV ordnet man sein erstes und zweites \begriff[lineares Mehrschrittverfahren!]{charakteristisches Polynom} $\rho:\comp\to\comp$ und $\sigma:\comp\to\comp$ zu durch
\begin{align}
	\label{3_18}
	\rho(z)=\sum_{\nu=0}^l \alpha_\nu z^\nu\quad\text{und}\quad \sigma(z)=\sum_{\nu=0}^l \beta_\nu z^\nu\quad\forall z\in\comp
\end{align}
Das lineare MSV nach \person{Adams-Bashford} (1883) geht von
\begin{align}
	\label{3_19}
	y(x_{k+l}) - y(x_{k+l-1}) = \int_{x_{k+l-1}}^{x_{k+l}} f(x,y(x))\diff x
\end{align}
aus und approximiert den Integranden $f(x,y(x))$ durch ein Interpolationspolynom, nämlich
\begin{align}
	\label{3_20}
	\sum_{\nu=0}^{l-1} L_\nu(x)f(x_{k+\nu},y(x_{k+\nu}))
\end{align}
Dabei bezeichnen $L_\nu:\R\to\R$ für $\nu=0,...,l-1$ die \person{Langrange}-Polynome mit
\begin{align}
	L\nu(x) = \prod_{\substack{i=k \\ i\neq k+\nu}}^{k+l-1} \frac{x-x_i}{x_{k+\nu} - x_i} \quad\text{für } x\in\R\notag
\end{align}
Definiert man $\beta_\nu$ durch
\begin{align}
	\label{3_21}
	\int_{x_{k+l-1}}^{x_{k+l}} L_\nu(x)\diff x=h\beta_\nu
\end{align}
so liefert die Approximation von \cref{3_19} die Näherungsformel
\begin{align}
	y^{k+l}-y^{k+l-1} &= \sum_{\nu=0}^{l-1} \left(\int_{x_{k+l-1}}^{x_{k+l}} L_\nu(x)\diff x\right)f(x_{k+\nu},y^{k+\nu}) \notag \\
	&= h\sum_{\nu=0}^{l-1} \beta_\nu f(x_{k+\nu},y^{k+\nu}) \notag
\end{align}
also ein explizites $l$-schrittiges lineares MSV mit $\alpha_l=1$, $\alpha_{l-1}=-1$ und den durch \cref{3_21} definierten $\beta_0,...,\beta_{l-1}$ sowie $\beta_l=0$.

Beim linearen MSV nach \person{Adams-Moulton} (1926) wird die Summation in \cref{3_20} von $\nu=0$ bis $\nu=l$ erstreckt und dann analog vorgegangen. Dies ergibt das implizite lineare MSV
\begin{align}
	\label{3_22}
	y^{k+l} - y^{k+l-1} = h\sum_{\nu=0}^l \beta_\nu f(x_{k+\nu},y^{k+\nu})
\end{align}
Es erfolgt die (ggf. näherungsweise) Lösung eines im Allgemeinen nichtlinearen Gleichungssystems für $y^{k+l}$ und kann mit Hilfe des \begriff{Prädiktor-Korrektor-Prinzips} erfolgen. Dabei ermittelt man mit Hilfe eines expliziten linearen MSV (Prädiktor) eine erste Näherung $\zeta^0$ für $y^{k+l}$ und verbessert diese dann mit einem (näherungsweisen) Schritt eines impliziten linearen MSV (Korrektor). Zum Beispiel bestimme man $\zeta^0$ mit \person{Adams-Bashford}, das heißt
\begin{align}
	\zeta^0 = y^{k+l} + h\sum_{\nu=0}^{l-1} \beta_\nu f(x_{k+\nu},y^{k+\nu}) \notag
\end{align}
Danach wird eine Näherungslösung von \cref{3_22} (\person{Adams-Moulton}) etwa mittels Fixpunktiteration ermittelt
\begin{align}
	\zeta^j = y^{k+l-1} + h\beta_l^C f(x_{k+l},\zeta^{j-1}) + h\sum_{\nu=0}^{l-1} \beta_\nu^C f(x_{k+\nu},y^{k+\nu}) \notag
\end{align}
die für ein vorgegebenes $j\ge 1$ abgebrochen wird. Die Bezeichnung $\beta_\nu^C$ dient der Unterscheidung von den im Prädiktor verwendeten Parametern $\beta_\nu$. Für $j=1$ ergibt sich ein nichtlineares MSV (\person{Adams-Bashford-Moulton}-Verfahren). Für $j\to\infty$ kann unter bestimmten Voraussetzungen für hinreichend kleine $h>0$ die Konvergenz der Folge $\{\zeta^j\}$ gegen den eindeutigen Fixpunkt $y^{k+l}$ gezeigt werden.

Eine Klasse von impliziten linearen MSV (sogenannte \begriff{Backward Differentiation Formulas} bzw. \begriff{BDF-Verfahren}) erhält man aus der Idee $y'(x_{k+l}) = f(x_{k+l},y(x_{k+l}))$ durch $\sfrac{1}{h}\sum_{\nu=0}^l \alpha_\nu y(x_{k+\nu})$ (verallgemeinerte Sekantensteigung) zu approximieren. Man hat dann ein lineares MSV der Form
\begin{align}
	\sum_{\nu=0}^l \alpha_\nu y^{k+\nu} = hf(x_{k+l},y^{k+l}) \notag
\end{align}

\subsection{Konsistenz- und Konvergenzordnung für lineare MSV}

Die Konsistenzordnung eines linearen MSV wird in Analogie zur entsprechenden Definition bei den ESV eingeführt. Verallgemeinerungen für beliebige MSV werden hier nicht betrachtet. Der lokale Diskretisierungsfehler eines MSV ergibt sich zu
\begin{align}
	\Delta(x,h) &= y(x+lh) - \frac{1}{\alpha_l}\left(-\sum_{\nu=0}^{l-1} \alpha_\nu y(x+\nu h) + h\sum_{\nu=0}^l \beta_\nu f(x+\nu h,y(x+\nu h))\right) \notag \\
	\label{3_23}
	&= \frac{1}{\alpha_l} \left(\sum_{\nu=0}^l \alpha_\nu y(x+\nu h) - h\sum_{\nu=0}^l \beta_\nu f(x+\nu h,y(x+\nu h))\right)
\end{align}
Wenn es also $p\ge 1$, $M>0$ und $\tilde{h}>0$ gibt, so dass
\begin{align}
	\norm{\frac{\Delta(x,h)}{h}} \le Mh^p\quad\forall (x,h)\in [a,b)\times (0,\tilde{h}]\text{ mit } x+h\le b \notag
\end{align}
für jede Lösung $y$: $[a,b]\to\R^m$ der Differentialgleichung $y'=f(x,y)$ gilt, dann sagt man, dass das MSV die \begriff[Mehrschrittverfahren!]{Konsistenzordnung} $p\ge 1$ besitzt. Unter der Voraussetzung, dass $f$ und damit die Lösungen $y$ der Differentialgleichung hinreichend glatt sind, gelten die Entwicklungen
\begin{align}
	y(x+\nu h) = \sum_{q=0}^{p} \frac{\nu^q}{q!} y^{(q)}(x) h^q + R_p(x,h) \notag
\end{align}
und (mit $\frac{\diff y(x+\nu h)}{\diff h} = y'(x+\nu h)\nu$)
\begin{align}
	f(x+\nu h,y(x+\nu h)) = y'(x+\nu h) = \sum_{q=1}^{p} \frac{\nu^{q-1}}{(q-1)!} y^{(q)}(x)h^{q-1} + r_p(x,h) \notag
\end{align}
wobei für die Restglieder $\norm{R_p(x,h)} \le M_R h^{p+1}$ und $\norm{r_p(x,h)}\le M_r h^p$ bei festem $x$ mit gewissen Konstanten $M_R,M_r>0$ gilt. Aus \cref{3_23} hat man daher
\begin{align}
	\alpha_l\Delta(x,h) = c_0y(x) + \sum_{q=1}^p c_qy(x)^{(q)} h^q + Q(x,h)\quad\mit\quad\norm{Q(x,h)}\le M_Q h^{p+1} \notag
\end{align}
wobei $M_Q>0$ sowie
\begin{align}
	\label{3_24}
	c_0 = \sum_{\nu=0}^l \alpha_\nu\quad\text{und}\quad c_q ? \sum_{\nu=0}^l \left(\frac{\nu^q\alpha_\nu}{q!} - \frac{\nu^{q-1}\beta_\nu}{(q-1)!}\right)\quad\text{für } q=1,...,p
\end{align}
(mit $0^0=1$). Falls $c_0=\dots = c_p=0$, folgt damit
\begin{align}
	\norm{\frac{\Delta(x,h)}{h}}\le \alpha_l M_Q h^p\notag
\end{align}
Also gilt

\begin{proposition}
	\proplbl{satz_3_12}
	Die Funktion $f$: $[a,b]\times\R^m\to\R^m$ sei $p$ mal stetig differenzierbar. Dann hat das MSV \cref{3_17} (mindestens) die Konsistenzordnung $p$, wenn $c_0=\dots=c_p=0$.
\end{proposition}

Unter Beachtung von \cref{3_18} und \cref{3_24} gilt $c_0=c_1=0$ (und damit entsprechend Satz \propref{satz_3_12} Konsistenzordnung $p\ge 1$), genau dann wenn
\begin{align}
	\rho(1)=0\quad\text{und}\quad \rho'(1)-\sigma(1)=0 \notag
\end{align}
Ein $l$-schrittiges explizites lineares MSV \cref{3_17} hat die $2l$ freien Parameter $\alpha_0,...,\alpha_{l-1}$ und $\beta_0,...,\beta_{l-1}$ (o.B.d.A. kann $\alpha_l=1$ gewählt werden, $\beta_l$ kommt nicht vor, da MSV explizit sein sollte). Durch geeignete Wahl dieser Parameter könnte man $c_0=0,...,c_{2l-1}=0$ erreichen und damit die Konsistenzordnung $p=2l-1$. Wie wir aber sehen werden, sind solche Verfahren im Allgemeinen nicht konvergent.

\begin{example}
	Sei $l=2$ und $m=1$. Dann lautet \cref{3_17}
	\begin{align}
		y_{k+2} + \alpha_1y_{k+1} + \alpha_0y_k = h\big(\beta_1 f(x_{k+1},y_{k+1}) + \beta_0 f(x_k,y_k)\big) \notag
	\end{align}
	Um $c_0=c_1=c_2=c_3=0$ und damit Konsistenzordnung 3 zu erreichen, muss man also $\alpha_0,\alpha_1,\beta_0,\beta_1$ so wählen, dass (mit $\alpha_2=1$ und $\beta_2=0$)
	\begin{align}
		\systeme[\alpha_0,\alpha_1,\beta_0,\beta_1]{\alpha_0 + \alpha_1 + 1 = 0@c_0,\alpha_1 - \beta_0 - \beta_1 + 2 = 0@c_1, \frac{1}{2}\alpha_1 - \beta_1 + 2 = 0@c_2, \frac{1}{6}\alpha_1 - \frac{1}{2}\beta_1 + \frac{4}{3} = 0@c_3} \notag
	\end{align}
	gilt. Die Lösung dieses Systems ist gegeben durch $\alpha_0=-5$, $\alpha_1=4$, $\beta_0=2$ und $\beta_1=4$. Also besitzt das MSV
	\begin{align}
		\label{3_25}
		y_{k+2} + 4y_{k+1} - -5y_k = h\big(2f(x_k,y_k) + 4f(x_{k+1},y_{k+1})\big)
	\end{align}
	die Konsistenzordnung 3. Für das Testproblem
	\begin{align}
		y'=-y\quad\mit\quad y(0)=1 \notag
	\end{align}
	lautet die Lösung $y(x)=\exp(-x)$. Das Verfahren \cref{3_25} geht wegen $f(x,y)=-y$ über die homogene lineare Differentialgleichung mit konstanten Koeffizienten
	\begin{align}
		\label{3_26}
		y_{k+2} + (4+4h)y_{k+1} + (-5+2h)y_k=0
	\end{align}
	Mit dem Ansatz $y_k=z^k$ für ein $z\in\R\backslash\{0\}$ erhält man aus \cref{3_26} (nach Division durch $z^k$)
	\begin{align}
		\label{3_27}
		z^2 + (4+4h)z + (-5+2h) = \rho(z) - h\sigma(z) = 0
	\end{align}
	Die Lösungen dieser quadratischen Gleichungen lauten
	\begin{align}
		z_{1/2} = z_{1/2}(h) &= -2(1+h) \pm \sqrt{4(1+h)^2 - 2h+5} \notag \\
		&= -2(1+h) \pm\sqrt{1 + \frac{2h}{3} + \frac{4h^2}{9}} \notag
	\end{align}
	Für $h\to 0$ hat man
	\begin{align}
		z_1(h) = 1-h + \mathcal{O}(h^2)\quad\text{und}\quad z_2(h) = -5 + \mathcal{O}(h) \notag
	\end{align}
	Die allgemeine Lösung der Differentialgleichung \cref{3_26} ist gegeben durch
	\begin{align}
		y_k = c_1z_1 ^k + c_2z_2^k \notag
	\end{align}
	Gibt man sich die Startwerte $y_0=y(0)=1$ und $y_1=y(h)=\exp(-h)$ als exakte Funktionswerte vor, bestimmen sich die Konstanten $c_1=c_1(h)$ und $c_2=c_2(h)$ aus $y_0=c_1+c_2$ und $y_1=c_1z_1 + c_2z_2$. Bei genauerer Betrachtung der Abhängigkeit von $z_1,z_2$ von $h$ ergibt sich dafür $c_1(h)=1+\mathcal{O}(h^2)$ und $c_2(h)=-\sfrac{h^4}{216} + \mathcal{O}(h^5)$. Für festes $x>0$ und $x=kh$ folgt für $k\to\infty$ und $h\to 0$
	\begin{align}
		\abs{c_2(h) z_2(h)^k} = \abs{\mathcal{O}\left(\left(\frac{x}{k}\right)^4\right)}\cdot\abs{-5+\mathcal{O}(h)}^k\to\infty \notag
	\end{align}
	sowie
	\begin{align}
		c_1(h) z_1(h) &= (1+\mathcal{O}(h^2))\cdot (1-h+\mathcal{O}(h^2))^k \notag \\
		&= \left(1-\frac{x}{k}\right)^k + \mathcal{O}(h^2) \notag \\
		&\to\exp(-x)\notag
	\end{align}
	Da die sogenannte parasitäre Lösungskomponente $c_2z_2^k$ die andere sinnvolle Lösungskomponente der Differentialgleichung beliebig übersteigt, kann man für das MSV \cref{3_25} keine Konvergenz erwarten. Um dieses Verhalten bei einem linearen MSV zu verhindern, darf zumindest der Betrag jeder Lösung (Wurzel) der Polynomgleichung $\rho(z)=0$ den Wert 1 nicht übersteigen, vergleiche \cref{3_27} für $h\to 0$.
\end{example}

\begin{definition}[D-stabil, nullstabil]
	Das lineare MSV \cref{3_17} heißt \begriff[Mehrschrittverfahren!]{D-stabil} (oder \begriff[Mehrschrittverfahren!]{nullstabil}), falls es die \begriff{Wurzelbedingung} erfüllt, das heißt wenn der Betrag jeder Nullstelle seines ersten charakteristischen Polynoms $\rho$ durch 1 beschränkt ist und der Betrag jeder mehrfachen Nullstelle von $\rho$ kleiner als 1 ist.
\end{definition}

Die Bezeichnung D-stabil ist zu Ehren von \person{Dahlquist} (1925-2005) für seine Arbeiten zur Stabilität von linearen MSV gewählt worden.

Zur formalen Definition der Konvergenzordnung eines linearen MSV nehmen wir (wie bei ESV) an, dass die AWA \cref{3_1_1} die eindeutige Lösung $y$: $[a,b]\to\R^m$ besitzt. Weiter nehmen wir an, dass zu jeder Schrittweite $h$ Startvektoren $y_0^h,...,y_h^{l-1}$ gegeben sind, aus denen das MSV die Näherungen $y_h^l,...,y_h^N$ erzeugt. Falls für jede Schrittweite $h=\frac{b-a}{N}$ die Startvektoren die Bedingung
\begin{align}
	\label{3_28}
	\norm{y_h^\nu - y(x_o+\nu h)} \le C_1h^p\quad\forall\nu=0,...,l-1
\end{align}
genügen (mit einem von $h$ unabhängigen $C_1>0$), dann heißt ein linearen Mehrschrittverfahren \begriff[Mehrschrittverfahren!]{konvergent mit der Ordnung} $p\ge 1$, wenn es $C_2>0$ und $\tilde{h}>0$ gibt, so dass
\begin{align}
	\norm{y_h^k - y(x_0+kh)} \le C_2h^p \notag
\end{align}
für alle $k=l,...,N$ und alle Schrittweiten $h\in (0,\tilde{h}]$.

Die beiden folgenden Sätze geben wir ohne Beweis an.

\begin{proposition}
	\proplbl{satz_3_15}
	Die AWA \cref{3_1_1} besitze die eindeutige Lösung $y$: $[a,b]\to\R^m$. Das lineare MSV \cref{3_17} sei D-stabil und habe die Konsistenzordnung $p$. Es gebe $L_f>0$, so dass die Lipschitz-Bedingung
	\begin{align}
		\norm{f(x,y) - f(x,\bar{y})} \le L_f\norm{y-\bar{y}}\notag
	\end{align}
	für alle $(x,y),(x,\bar{y})\in [a,b]\times\R^m$ gilt. Weiter gelte die Bedingung \cref{3_28} an die Startvektoren. Dann ist das MSV konvergent mit der Ordnung $p$.
\end{proposition}

\begin{proposition}[Erste \person{Dahlquist}-Barriere]
	\proplbl{satz_3_16}
	Ein $l$-schrittiges lineares MSV \cref{3_17} sei D-stabil. Dann gilt für seine Konsistenzordnung
	\begin{align}
		p \le \begin{cases}
			l+1 & \text{falls } l \text{ ungerade} \\
			l+2 & \text{falls } l \text{ gerade} \\
			l & \text{falls } \sfrac{\beta_l}{\alpha_l} \le 0
		\end{cases} \notag
	\end{align}
\end{proposition}

Für den Einfluss von Rundungsfehlern lassen sich für lineare MSV zu Abschnitt 2.4 vergleichbare Überlegungen durchführen.

Zum Beispiel hat man bei den \person{Adams-Bashford}-Verfahren für die Schrittzahl $l=2$ ein explizites lineares MSV, nämlich 
\begin{align}
	y^{k+2} - y^{k+1} = h(\beta_0 f(x_k,y^k) + \beta_1 f(x_{k+1},y^{k+1})) \notag
\end{align}
Bezogen aus die allgemeine Form \cref{3_17} linearer MSV gilt hier $\alpha_2=1$, $\alpha_1=-1$, $\alpha_0=0$ und $\beta_2=0$. Also ist für dieses spezielle MSV $\sfrac{\beta_2}{\alpha_2}=0$ und nach \propref{satz_3_16} daher bei gewünschter Konvergenz (und damit D-Stabilität) maximal die Konsistenzordnung $p=2$ erreichbar. Um diese zu sichern, müssen $c_0=0$, $c_1=0$ und $c_2=0$ nach \propref{satz_3_12} gelten. Wegen $c_0=\rho(1)=\alpha_0 + \alpha_1 + \alpha_2 = 0-1+1=0$ sind noch $c_1=\rho'(1)-\sigma(1)=2\alpha_2 + \alpha_1 - (\beta_0 + \beta_1 + \beta_2) = 1-\beta_0 - \beta_1 = 0$ und $c_2=\sfrac{\alpha_1}{2} - \beta_1 + \sfrac{4\alpha_2}{2} - 2\beta_2 = -\sfrac{1}{2} - \beta_1 + 2 = 0$ zu erfüllen. Dies liefert $\beta_1 = \sfrac{3}{2}$ und $\beta_0=-\sfrac{1}{2}$. Also hat das Verfahren
\begin{align}
	y^{k+2} - y^{k+1} = \frac{h}{2}\bigg(-f(x_k,y^k) + 3f(x_{k+1},y^{k+1})\bigg)\notag
\end{align}
nach \propref{satz_3_12} die Konvergenzordnung 2. Das charakteristische Polynom $\rho$ des linearen MSV ist offenbar gegeben durch $\rho(z)=z^2-z$. Seine Nullstellen sind $z_1=0$ und $z_2=1$. Folglich ist das Verfahren auch D-stabil und damit nach \propref{satz_3_15} konvergent mit der Ordnung 2.
\section{A-Stabilität}

Wir betrachten die Test-AWA
\begin{align}
	\label{3_29}
	y'=\lambda y\quad\mit\quad y(0)=1
\end{align}
wobei $\lambda\in\comp$ ein Parameter ist. Die eindeutige Lösung dieser Aufgabe ist gegeben durch $y(x)=\exp(\lambda x)$ und es gilt insbesondere
\begin{align}
	\Re(\lambda)&< 0 \quad\Rightarrow\quad \abs{y(x)}\to 0 \quad\text{für } x\to\infty \notag \\
	\Re(\lambda)&=0 \quad\Rightarrow\quad\abs{y(x)}=1 \quad\text{für alle } x\in[0,\infty) \notag
\end{align}

\begin{definition}[A-Stabilität]
	Ein Verfahren erzeuge zu einem beliebigen Paar $(h,\lambda)\in(0,\infty)\times\comp$ eine Folge $\{y_k\}$. Dann heißt das Verfahren \begriff{A-stabil}, wenn
	\begin{align}
		\abs{y_{k+1}} \le \abs{y_k}\quad\forall k\in\natur\notag
	\end{align}
	für jedes $(h,\lambda)\in (0,\infty)\times\comp$ mit $\Re(\lambda)\le 0$.
\end{definition}

Bei ESV gilt $y_{k+1}=y_k + h\Phi(x_k,y_k,y_{k+1},h)$. Wir nehmen an, dass für $f(x,y)=\lambda y$ eine Darstellung des ESV in der Form
\begin{align}
	y_{k+1} = g(h\lambda)y_k \notag
\end{align}
mit einer Funktion $g$: $\comp\to\comp$ existiert. Die Funktion $g$ heißt dann auch \begriff{Stabilitätsfunktion}. Falls der \begriff{Stabilitätsbereich} (Bereich der absoluten Stabilität)
\begin{align}
	\mathcal{S} = \{z\in\comp\mid \abs{g(z)}\le 1\}\notag
\end{align}
die Halbebene $\comp_-=\{z\in\comp\mid \Re(z)\le 0\}$ enthält, dann ist das ESV A-stabil (und umgekehrt), denn es gilt $\abs{y_{k+1}} = \abs{g(h\lambda)}\abs{y_k}\le \abs{y_k}$ für $k\in\natur$ und beliebige $(h,\lambda)\in (0,\infty)\times\comp$ mit $h\lambda\in\comp_-$. Für die \begriff{Trapenzregel} (ein implizites ESV)
\begin{align}
	y_{k+1} = y_k + \frac{h}{2}\bigg(f(x_k,y_k) + f(x_{k+1},y_{k+1})\bigg) \notag
\end{align}
erhält aus der Test-AWA \cref{3_29} $y_{k+1} = y_k + \frac{h}{2}(\lambda y_k + \lambda y_{k+1})$ und somit
\begin{align}
	\left(1-\frac{h}{2}\lambda\right)y_{k+1} = y_k\left(1+\frac{h}{2}\lambda\right)\notag
\end{align}
das heißt die Stabilitätsfunktion der Trapezregel ist gegeben durch
\begin{align}
	g(z) = \frac{1+\frac{z}{2}}{1-\frac{z}{2}} = \frac{2+z}{2-z}\notag
\end{align}
Falls $\Re(z)\le 0$, so folgt
\begin{align}
	\abs{g(z)}^2 = \frac{(2+\Re(z))^2 + (\Im(z))^2}{(2-\Re(z))^2 + (\Im(z))^2} \le 1\notag
\end{align}
also die A-Stabilität der Trapezregel.

Für das \begriff[\person{Euler}-Verfahren!]{explizite \person{Euler}-Verfahren} ergibt sich (wegen \cref{3_29})
\begin{align}
	y_{k+1} = y_k + h\lambda y_k = (1+h\lambda)y_k\quad\text{und}\quad g(z) = 1+z\notag
\end{align}
Damit gilt
\begin{align}
	\abs{g(z)}^2 = \abs{1+z}^2 = (1+\Re(z))^2 + (\Im(z))^2 \le 1 \notag
\end{align}
genau dann, wenn $z=h\lambda$ im Einheitskreis um $(-1,0)\in\comp$ liegt. Da der Stabilitätsbereich beim expliziten \person{Euler}-Verfahren nicht alle $z\in\comp$ mit $\Re(z)\le 0$ enthält, ist dieses Verfahren nicht A-stabil. Das explizite \person{Euler}-Verfahren hat Konsistenzordnung 1 (vgl. \propref{3_2_3}) und ist mit dieser Ordnung auch konvergent (vgl. \propref{satz_3_8}). Die fehlende A-Stabilität hat zur Folge, dass zur erfolgreichen numerischen Lösung der Test-AWA \cref{3_29} für $\lambda <0$ zumindest $-2\le h\lambda$ gelten muss. Dies erfordert $h\sim \frac{1}{\abs{\lambda}}$, also gegebenenfalls sehr kleine Schrittweiten. Dies ist neben einem hohen Aufwand auch die Gefahr des Überwiegens von Rundungsfehlern verbunden, vgl. Abschnitt 2.4. Verfahren, die A-stabil sind, bzw. einen hinreichend großen Bereich absoluter Stabilität besitzen, haben außerdem Vorteile bei sogenannten steifen AWA, vgl. Abschnitt 5.

Bei RKV kann man den Stabilitätsbereich untersuchen, indem man sich die Stabilitätsfunktion beschafft. Zum Beispiel betrachten wir das 2-stufige explizite RKV
\begin{align}
	y_{k+1} &= y_k + hc_1k_1 + hc_2k_2 \notag \\
	&= y_k + hc_1f(x_k,y_k) + hc_2f(x_k+\alpha_2h,y_k + h\beta_{21}f(x_k,y_k))\notag
\end{align}
Mit der Test-AWA \cref{3_29} folgt
\begin{align}
	y_{k+1} &= y_k + h\lambda c_1y_k + h\lambda c_2(y_k + h\lambda\beta_{21} y_k) \notag \\
	&= y_k(1+h\lambda c_1 + h\lambda c_2 + (h\lambda)^2 c_2\beta_{21}) \notag
\end{align}
Beim Verfahren von \person{Heun} (vgl. Abschnitt 2.5) mit $c_1=c_2=\sfrac{1}{2}$ und $\beta_{21}=1$ ergibt sich
\begin{align}
	y_{k+1} = y_k\left(1+h\lambda + \frac{1}{2}(h\lambda)^2\right)\quad\text{und also}\quad g(z) = 1+z+\frac{1}{2}z^2 \notag
\end{align}
Man sieht schnell, dass dieses Verfahren nicht A-stabil ist (man wähle $z=(a,0)$ mit $a<-2$).

\begin{remark}
	Es gibt kein explizites lineares MSV und kein explizites RKV, dass A-stabil ist und die A-stabilen impliziten MSV haben höchstens Konsistenzordnung 2 (zweite \person{Dahlquist}-Barriere).
\end{remark}
\section{Einblick: Steife Probleme}

Für $A\in\R^{m\times m}$ werde die AWA
\begin{align}
	\label{3_30}
	y' = Ay\quad\mit\quad y(a)=y^0
\end{align}
für $x\in [a,b]$ betrachtet. Wir setzen in diesem Abschnitt voraus, dass $A$ eine diagonalisierbare Matrix ist, das heißt es gibt eine reguläre Matrix $S\in\comp^{m\times m}$ und eine Diagonalmatrix $D\in\comp^{m\times m}$ mit $A=SDS^{-1}$. Dann ist die allgemeine Lösung $y$: $[a,b]\to\R^m$ von $y'=Ay$ gegeben durch
\begin{align}
	y(x) = \sum_{i=1}^m c_i\exp(\lambda_i (x-a))v^i \notag
\end{align}
wobei $\lambda_1,...,\lambda_m\in\comp$ die Eigenwerte von $A$ und $v^1,...,v^m\in\comp^m$ ein zugehöriges System linear unabhängiger Eigenvektoren bezeichnet ($A$ diagonalisierbar!). Die Koeffizienten $c_1,...,c_m$ ergeben sich damit eindeutig aus der Anfangsbedingung $y(a)=c_1v^1+\dots+c_mv^m=y^0$.

Falls $\Re(\lambda_i)<0$ für $i=1,...,m$ wird die Zahl
\begin{align}
	\frac{\max_{1\le i\le m}\abs{\Re(\lambda_i)}}{\min_{1\le i\le m}\abs{\Re(\lambda_i)}}\notag
\end{align}
als \begriff{Steifigkeitsquotient} von $A$ bezeichnet. Ist dieser Quotient groß, dann dient dies als Indikator für ein Phänomen, das bei der Anwendung bestimmter numerischer Verfahren aus \cref{3_30} auftreten kann und als \begriff{Steifheit} (stiffness) der AWA \cref{3_30} bezeichnet wird. Ein solches Phänomen wird im folgenden Beispiel beschrieben und führt bei bestimmten Lösungsverfahren (hier explizites \person{Euler}-Verfahren) zum Erfordernis sehr kleiner Schrittweiten.

\begin{example}
	Für $a=0$ und
	\begin{align}
		A = \begin{pmatrix}
			-80.6 & 119.4 \\ 79.6 & -120.4
		\end{pmatrix} \notag
	\end{align}
	ergibt sich als allgemeine Lösung von $y'=Ay$
	\begin{align}
		y(x) = c_1\exp(-x)v^1 + c_2\exp(-200x)v^2 \quad\mit\quad v^1 = \begin{pmatrix}
			3 \\ 2
		\end{pmatrix} \text{ und } v^2 = \begin{pmatrix}
			-1 \\ 1
		\end{pmatrix} \notag
	\end{align}
	Für $y^0=(2,3)^T$ hat man als exakte Lösung von \cref{3_30} $y(x) = c_1\exp(-x)v^1 + c_2\exp(-200x)v^2$. Das explizite \person{Euler}-Verfahren liefert
	\begin{align}
		y^{k+1} = y^k + hAy^k = (\mathbbm{1} + hA)y^k \notag
	\end{align}
	Da $A$ diagonalisierbar ist, gilt $A=SDS^{-1}$ mit $S=(v^1,v^2)$ und $D=\diag(-1,-200)$ und
	\begin{align}
		S^{-1} = \frac{1}{5}\begin{pmatrix}
			1 & 1 \\ -2 & 3
		\end{pmatrix}\notag
	\end{align}
	Damit folgt
	\begin{align}
		S^{-1}y^{k+1} &= S^{-1}y^k + hS^{-1}ASS^{-1}y^k \notag \\
		&= S^{-1}y^k + hDS^{-1}y^k \notag \\
		&= (\mathbbm{1} + hD)S^{-1}y^k \notag
	\end{align}
	Setzt man $z^k = S^{-1}y^k$ ergibt sich weiter
	\begin{align}
		z^{k+1} = (\mathbbm{1} + hD)u^k \notag
	\end{align}
	für $k=0,...$. Wegen $z^0=S^{-1}y^0 = S^{-1}(v^1+v^2) = (1,0)^T + (0,1)^T = (1,1)^T$ erhält man
	\begin{align}
		z^k = (\mathbbm{1} + hD)^k\begin{pmatrix}
			1 \\ 1
		\end{pmatrix} \notag
	\end{align}
	Für $k\to\infty$ folgt $x_k\to\infty$ und $y(x_k)\to 0$. Um die Konvergenz der Folge $\{z^k\}$ und damit der Folge $\{y^k\}$ gegen 0 zu sichern, müssen
	\begin{align}
		\abs{1+\lambda_1 h} = \abs{1-h} < 1\quad\text{und}\quad \abs{1+\lambda_2 h} = \abs{1-200h} < 1 \notag
	\end{align}
	erfüllt sein. Dies impliziert $h < \sfrac{1}{100}$. Der für die exakte Lösung eigentlich unwesentliche (das heißt sehr schnell abklingende) Anteil $\exp(-200x)v^2$ verursacht beim expliziten \person{Euler}-Verfahren sehr kleine Schrittweiten.
\end{example}

Ähnliche Phänomene können bei der Anwendung anderer Verfahren, die nicht A-stabil sind, bzw. deren Bereich der absoluten Stabilität ungeeignet ist, auftreten. Auch bei allgemeineren AWA als \cref{3_30} treten Phänomene der Steifheit auf und erfordern angepasste Verfahren.
\section{Ausblick}

\part*{Anhang}
\addcontentsline{toc}{part}{Anhang}
\appendix

%\printglossary[type=\acronymtype]

\printindex

\end{document}
