\section{Homomorphismen von Ringen}

Seien $R,S$ und $T$ Ringe.

\begin{definition}[Ringhomomorphismus]
	Eine Abbildung $f:R\to S$ ist ein \begriff{Ringhomomorphismus}, wenn für $x,y\in R$ 
	gilt:
	\begin{itemize}
		\item (RH1:) $f(x+y)=f(x)+f(y)$
		\item (RH2:) $f(xy)=f(x)\cdot f(y)$
	\end{itemize}
	Die Menge der Ringhomomorphismen $f:R\to R$ wird mit $\Hom(R,S)$ bezeichnet. Ein Homomorphismus $f:R\to S$ ist ein 
	Mono-, Epi- oder Isomorphismus, wenn $f$ injektiv, surjektiv oder bijektiv ist. Gibt es einen Isomorphismus 
	$f:R\to S$, so nennt man $R$ und $S$ isomorph und schreibt $R\cong S$. Die Elemente von $\End(R):= \Hom(R,R)$ nennt 
	man \begriff{Endomorphismen}. Der Kern eines Ringhomomorphismus $f:R\to S$ ist $\Ker(f):= f^{-1}(\{0\})$.
\end{definition}

\begin{remark}
	Ein Ringhomomorphismus $f:R\to S$ ist ein Gruppenhomomorphismus der abelschen Gruppen $(R,+)$ und 
	$(S,+)$, der mit der Multiplikation verträglich ist, also eine strukturverträgliche Abbildung zwischen Ringen.
\end{remark}

\begin{example}
	\begin{itemize}
		\item $\id_R:R\to R$ ist ein Ringisomorphismus
		\item Ist $R_0\le R$ ein Unterring von $R$, so ist $\iota: R_0 \to R$ ein Ringmonomorphismus
		\item $\mathbb Z \to \mathbb Z\backslash n\mathbb Z$ mit $\overline a\mapsto a+n\mathbb Z$ it ein Ringepimorphismus
		\item Sei $R$ kommutativ mit Einselement. Für $\lambda\in R$ ist die Auswertungsabbildung $R[X]\to R$ mit $f\mapsto 
		f(\lambda)$ ein Ringepimorphismus. Dies ist die Aussage von \propref{1_6_8}
		\item $\mathbb C \to \mathbb C$ mit $z\mapsto \overline z$ ist ein Ringisomorphismus
	\end{itemize}
\end{example}

\begin{proposition}
	Sind $f:R\to S$ und $g:S\to T$ Ringhomomorphismen, so auch $g\circ f:R\to T$.
\end{proposition}
\begin{proof}
	Übung, analog zu \propref{3_2_5}
\end{proof}

\begin{lemma}
	Ist $f:R\to S$ ein Ringisomorphismus, so auch $f^{-1}: S\to R$.
\end{lemma}
\begin{proof}
	Nach \propref{3_2_7} wissen wir: $f^{-1}$ ist ein Isomorphismus der abelschen Gruppen $(S,+)\to (R,+)$. Die Verträglichkeit 
	mit der Multiplikation zeigt man analog.
\end{proof}

\begin{proposition}
	Sei $f\in \Hom(R,S)$. Genau dann ist $f$ ein Ringisomorphismus, wenn es $f'\in \Hom(S,R)$ mit $f'\circ 
	f=\id_R$ und $f\circ f'=\id_S$ gibt.
\end{proposition}
\begin{proof}
	Klar, analog zu \propref{3_2_8}
\end{proof}

\begin{lemma}
	Der Kern $I:=\Ker(f)$ eines Ringhomomorphismus $f:R\to S$ ist eine Untergruppe von $(R,+)$ mit 
	$x\cdot a, a\cdot x \in I$ für alle $a\in I$ und $x\in R$.
\end{lemma}
\begin{proof}
	Nach \propref{3_2_13} ist $I$ eine Untergruppe von $(R,+)$. Für $x\in R$ und $a \in I$ ist $f(xa)=f(x)\cdot 
	f(a)=f(x)\cdot 0=0$. Somit ist $xa\in I$. Analog ist $ax\in I$.
\end{proof}

\begin{proposition}
	Sei $f\in \Hom(R,S)$. Genau dann ist $f$ injektiv, wenn $\Ker(f)=\{0\}$.
\end{proposition}
\begin{proof}
	Klar aus \propref{3_2_14}, da $f:(R,+)\to (S,+)$ ein Gruppenhomomorphismus ist.
\end{proof}

\begin{definition}[Ideal]
	Ist $I$ eine Untergruppe von $(R,+)$ und $xa,ax\in I$ mit $x\in R$ und $a\in I$, so nennt 
	man $I$ ein \begriff{Ideal} von $R$ und schreibt $I\unlhd R$.
\end{definition}

\begin{example}
	Der Kern des Ringhomomorphismus $\mathbb Z\to \mathbb Z\backslash n\mathbb Z$ mit $a\mapsto 
	\overline a$ ist das Ideal $I=n\mathbb Z\unlhd \mathbb Z$.
\end{example}