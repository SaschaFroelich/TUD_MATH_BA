\section{Homomorphismen von Vektorräumen}

Seien $U,V,W$ drei $K$-Vektorraum. \\

\begin{definition}[linear]
	Eine Abbildung $f: V \to W$ heißt $K$-\begriff{linear}er Homomorphismus von $K$-Vektorraum, wenn für 
	alle $x,y\in V$ und $\lambda\in K$ gilt:
	\begin{itemize}
		\item (L1): $f(x+y)=f(x)+f(y)$
		\item (L2): $f(\lambda x)=\lambda \cdot f(x)$
	\end{itemize}
	Die Menge der $K$-linearen Abbildungen $f: V\to W$ wird mit $\Hom_K(V,W)$ bezeichnet. Die Elemente von $\End_K(V)
	:= \Hom_K(V,V)$ nennt man die Endomorphismen von $V$. Ein $f\in \Hom_K(V,W)$ ist ein Mono-, Epi- bzw. Isomorphismus, 
	falls $f$ injektiv, surjektiv bzw. bijektiv ist. Einen Endomorphismus der auch ein Isomorphismus ist, nennt man 
	\begriff{Automorphismus} von $V$ und bezeichnet die Menge der Automorphismen von $V$ mit $\Aut_K(V)$. Der Kern einer linearen 
	Abbildung $f: V\to W$ ist $\Ker(f):= f^{-1}(\{0\})$.
\end{definition}

\begin{remark}
	Eine $K$-lineare Abbildung $f: V\to W$ ist also ein Homomorphismus der abelschen Gruppen $(V,+) 
	\to(W,+)$, der mit der Skalarmultiplikation verträglich ist, d.h. eine strukturverträgliche Abbildung zwischen Vektorräumen.
\end{remark}

\begin{proposition}
	Eine Abbildung $f: V\to W$ ist genau dann $K$-linear, wenn für alle $x,y\in V$ und $\lambda,
	\mu\in K$ gilt: \\
	(L): $f(\lambda x +\mu y)=\lambda f(x) + \mu f(y)$.
\end{proposition}
\begin{proof}
	\begin{itemize}
		\item Hinrichtung: $f(\lambda x +\mu y)=f(\lambda x) + f(\mu y)=\lambda f(x) + \mu f(y)$
		\item Rückrichtung: (L1): $f(x+y)=f(1x+1y)=1f(x)+1f(y)$, (L2): $f(\lambda x)=f(\lambda x+0y)=\lambda f(x)$.
	\end{itemize}
\end{proof}

\begin{example}
	\begin{itemize}
		\item $\id_V: V\to V$ ist ein Automorphismus von $V$
		\item $c_0:V\to W$ mit $x\mapsto 0$ ist $K$-linear
		\item Für einen Untervektorraum $V_0\le V$ ist $\iota: V_0\to V$ ein Monomorphismus
		\item Im $K$-Vektorraum $K[X]$ kann man die (formale) Ableitung definieren: $(\sum_{i=0}^n a_iX^i)' := \sum
		_{i=1}^n ia_iX^{i-1}$. Diese Abbildung $K[X]\to K[X]$ mit $f\mapsto f'$ ist ein $K$-Endomorphismus von $K[X]$.
	\end{itemize}
\end{example}

\begin{example}
	\proplbl{3_4_5}
	Sei $V=K^n$ und $W=K^m$. Wir fassen die Elemente von $V$ und $W$ als Spaltenvektoren auf. Zu einer 
	Matrix $A\in \Mat_{m\times n}(K)$ definieren wir die Abbildung $f_A:V\to W$ mit $x\mapsto Ax$. \\
	Ausgeschrieben: Ist $A=(a_{ij})$ und $x=(x_1,...,x_n)^t$ so ist
	\begin{align}
		f_A(x)=Ax=
		\begin{pmatrix}
		a_{11} & ... & a_{1n}\\
		... &  & ...\\
		a_{m1} & ... & a_{mn}\\
		\end{pmatrix} \cdot \begin{pmatrix} x_1 \\ ... \\ x_n\end{pmatrix} = 
		\begin{pmatrix}
		a_{11}\cdot x_1 + ... + a_{1n}\cdot x_n\\
		...\\
		a_{m1}\cdot x_1 + ... + a_{mn}\cdot x_n\\
		\end{pmatrix}\notag
	\end{align} 
	Nach \propref{3_1_9} und \propref{3_1_7} ist $f_A$ eine lineare Abbildung.
\end{example}

\begin{proposition}
	\proplbl{3_4_6}
	Für ein $f\in \Hom_K(V,W)$. Dann gilt:
	\begin{itemize}
		\item $f(0)=0$
		\item Für $x,y\in V$ ist $f(x-y)=f(x)-f(y)$.
		\item Sind $(x_1)$ aus $V$, $(\lambda_i)$ aus $K$, fast alle gleich 0, so ist $f(\sum_{i\in I} \lambda_i
		\cdot x_i)=\sum_{i\in I} \lambda_i\cdot f(x)$.
		\item Ist $(x_i)$ linear abhängig in $V$, so ist $f(x_i)$ linear abhängig in $W$.
		\item Ist $V_0\le V$ ein Untervektorraum von $V$, so ist $f(V_0)\le W$ ein Untervektorraum.
		\item Ist $W_0\le W$ ein Untervektorraum von $W$, so ist $f^{-1}(W_0)\le V$ ein Untervektorraum.
	\end{itemize}
\end{proposition}
\begin{proof}
	\begin{itemize}
		\item klar
		\item klar
		\item Induktion
		\item $\sum \lambda_i\cdot x_i=0\Rightarrow 0=f(0)=f(\sum \lambda_i\cdot x_i)=\sum \lambda_i\cdot f(x_i)$
		\item $x,y\in V_0\Rightarrow f(x)+f(y)=f(x+y)\in f(V_0)$ \\ 
		$x\in V_0,\lambda\in K\Rightarrow f(x\cdot \lambda= f(\lambda x)\in f(V_0))$
		\item $f(0)=0\in W_0\Rightarrow 0\in f^{-1}(W_0)$, insbesondere ist $f^{-1}(W_0)\neq \emptyset$ \\ 
		$x,y\in f^{-1}(W_0)\Rightarrow f(x+y)=f(x)+f(y)\in W_0$, also $x+y\in f^{-1}(W_0)$ \\
		$x\in f^{-1}(W_0)$ und $\lambda\in K\Rightarrow f(\lambda x)=\lambda f(x)\in W_0$, also $\lambda x\in f^{-1}(W_0)$
	\end{itemize}
\end{proof}

\begin{proposition}
	Sind $f:V\to W$ und $g:W\to U$ $K$-linear, so auch $g\circ f: V\to U$.
\end{proposition}
\begin{proof}
	Für $x,y\in V$ und $\lambda,\mu\in K$ ist $(g\circ f)(\lambda x + \mu y)=g(f(\lambda x + \mu y))=g(\lambda f(x) + 
	\mu f(y))=\lambda (g\circ f)(x) + \mu (g\circ f)(y)$.
\end{proof}

\begin{lemma}
	\proplbl{3_4_8}
	Ist $f:V\to W$ ein Isomorphismus, so auch $f^{-1}:W\to V$.
\end{lemma}
\begin{proof}
	Wir müssen nur zeigen, dass $f^{-1}$ linear ist. Für $x,y\in V$ und $\lambda,\mu\in K$ ist $f(\lambda f^{-1}(x) + 
	\mu f^{-1}(y))=\lambda (f\circ f^{-1})(x) + \mu (f\circ f^{-1})(y)=\lambda x + \mu y$, also $f^{-1}(\lambda x + 
	\mu y)=\lambda f^{-1}(x) + \mu f^{-1}(y)$.
\end{proof}

\begin{proposition}
	\proplbl{3_4_9}
	Sei $f:V\to W$ linear. Genau dann ist $f$ ein Isomorphismus, wenn es eine lineare Abbildung $f':W
	\to V$ gibt mit $(f'\circ f)=\id_V$ und $(f\circ f')=\id_W$.
\end{proposition}
\begin{proof}
	Ist $f$ ein Isomorphismus, so erfüllt $f'=f^{-1}$ nach \propref{3_4_8} die Behauptung. Existiert umgekehrt $f'$ wie angegeben, so muss 
	$f$ bijektiv sein.
\end{proof}

\begin{remark}
	\proplbl{3_4_10}
	Wie auch bei Gruppen sehen wir hier bei Vektorräumen, dass Isomorphismen genau die strukturerhaltenden 
	Abbildungen sind. Wieder können wir uns einen Isomorphismus $f:V\to W$ so vorstellen, dass wir nur die Elemente von 
	$V$ umbenennen. Alle Aussagen, die sich nur aus der Struktur selbst ergeben, bleiben damit wahr, wie z.B. $\dim_K(V)=
	\dim_K(W)\iff V=W$. Insbesondere ist $K^n \cong K^m$ für $n=m$.
\end{remark}

\begin{proposition}
	\proplbl{3_4_11}
	Ist $f:V\to W$ eine lineare Abbildung, so ist $\Ker(f)$ ein Untervektorraum von $V$. Genau dann ist $f$ ein 
	Monomorphismus, wenn $\Ker(f)=\{0\}$.
\end{proposition}
\begin{proof}
	Der erste Teil folgt aus \propref{3_4_6}, der zweite folgt aus \propref{3_2_14}, da $f:(V,+)\to (W,+)$ ein 
	Gruppenhomomorphismus ist.
\end{proof}