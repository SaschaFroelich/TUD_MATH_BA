\section{Das Vorzeichen einer Permutation}

In diesem Kapitel sei $K$ ein Körper und $R$ ein kommutativer Ring mit Einselement.

\begin{remark}
	Wir erinnern uns an die symmetrische Gruppe $S_n$ aus \propref{1_3_7}, die aus den Permutationen der Menge $X=\{1,..,n\}$ (also den 
	bijektiven Abbildungen $X\to X$) mit der Komposition als Verknüpfung. Es ist $\vert S_n\vert =n!$ und $S_2\cong \mathbb Z\backslash 2 \mathbb Z$, 
	doch für $n\ge 3$ ist $S_n$ nicht abelsch. Wir schreiben $\sigma_1\sigma_2$ für $\sigma_1\circ \sigma_2$ und notieren $\sigma\in S_n$ 
	auch als \\
	\begin{align}
		\sigma=\begin{pmatrix}
		1 & 2 & ... & n\\
		\sigma(1) & \sigma(2) & ... & \sigma(n)\\
		\end{pmatrix}\notag
	\end{align}
\end{remark}

\begin{example}
	Für $i,j\in \{1,...,n\}$ mit $i\neq j$ bezeichne $\tau_{ij}\in S_n$ die Transposition 
	\begin{align}
	\tau_{ij}(k)=
	\begin{cases}
	j&\text{falls }$k=i$ \\ i& \text{falls }$k=j$ \\ k& \text{sonst}
	\end{cases}\notag
	\end{align} Offenbar gilt $\tau_{ij}^2=\id$, also $\tau_{ij}^{-1}=\tau_{ij}=\tau_{ji}$.
\end{example}

\begin{proposition}
	Für jedes $\sigma \in S_n$ gibt es ein $r\in \mathbb N_0$ und die Transpositionen $\tau_1,...,\tau_r\in S_n$ mit 
	\begin{align}
		\sigma=\tau_1\circ ... \circ \tau_r\notag
	\end{align}
\end{proposition}
\begin{proof}
	Sei $1\le k \le n$ maximal mit $\sigma(i)=i$ für $i\le k$. Induktion nach $n-k$. \\
	Ist $n-k=0$, so ist $\sigma=\id$ und wir sind fertig. \\
	Andernfalls ist $l=k+1\le n$ und $\sigma(l)>l$. Für $\sigma'=\tau_{l,\sigma(l)}\circ \sigma$ ist $\sigma(l)=l$ und somit $\sigma'(i)=i$ 
	für $1\le i \le k+1$. Nach Induktionshypothese gibt es Transpositionen $\tau_1,...,\tau_r$ mit $\sigma'=\tau_1\circ ...\circ \tau_r$. 
	Es folgt $\sigma=\tau_{l,\sigma(l)}^{-1}\circ \sigma^{-1}=\tau_{l,\sigma(l)}\circ \tau_1\circ ... \circ \tau_r$.
\end{proof}

\begin{definition}[Fehlstand, Vorzeichen]
	Sei $\sigma\in S_n$.
	\begin{itemize}
		\item Ein \begriff{Fehlstand} von $\sigma$ ist ein Paar $(i,j)$ mit $1\le i<j\le n$ und $\sigma(i)>\sigma(j)$.
		\item Das \begriff{Vorzeichen} (oder \begriff{Signum}) von $\sigma$ ist $\sgn(\sigma)=(-1)^{f(\sigma)}\in \{-1,1\}$, wobei $f(\sigma)$ die 
		Anzahl der Fehlstände von $\sigma$ ist.
		\item Man nennt $\sigma$ \begriff[Vorzeichen!]{gerade}, wenn $\sgn(\sigma)=1$, sonst \begriff[Vorzeichen!]{ungerade}.
	\end{itemize}
\end{definition}

\begin{example}
	\proplbl{4_1_5}
	\begin{itemize}
		\item Genau dann hat $\sigma$ keine Fehlstände, wenn $\sigma=\id$. Insbesondere $\sgn(\id)=1$.
		\item Die Permutation 
		\begin{align}
			\sigma=\begin{pmatrix}1 & 2 & 3\\2 & 3 & 1\\\end{pmatrix}\notag
		\end{align} hat die Fehlstände $(1,3)$ und $(2,3)$, somit 
		$\sgn(\sigma)=1$.
		\item Die Transposition 
		\begin{align}
			\tau_{13}=\begin{pmatrix}1 & 2 & 3\\3 & 2 & 1\\\end{pmatrix}\notag
		\end{align} hat die Fehlstände $(1,2)$, $(2,3)$ und 
		$(3,1)$, somit $\sgn(\tau_{13})=-1$.
		\item Eine Transposition $\tau_{ij}\in S_n$ ist ungerade: Ist $i<j$, so sind die Fehlstände $(i,i+1),...,(i,j)$ und $(j+1,j)...
		(j-1,j)$, also $j-(i+1)+1+(j-1)-(i-1)+1=2(j-1)-1$ viele.
	\end{itemize}
\end{example}

\begin{lemma}
	Für $\sigma\in S_n$ ist
	\begin{align}
		\sgn(\sigma)=\prod_{1\le i<j\le n} \frac{\sigma(j)-\sigma(i)}{j-i}\in \mathbb Q\notag
	\end{align}
\end{lemma}
\begin{proof}
	Durchläuft $(i,j)$ alle Paare $1\le i<j\le n$, so durchläuft $\{\sigma(i),\sigma(j)\}$ alle zweielementigen Teilmengen von $\{1,...,
	n\}$. Das Produkt $\prod_{i<j} \sigma(j)-\sigma(i)$ hat also bis auf das Vorzeichen die selben Faktoren wie das Produkt 
	\begin{align}
		\prod_{i<j} j-i=\prod_{i<j} |j-i|\notag
	\end{align}
	 und
	 \begin{align}
	 	\prod_{i<j} \sigma(j)-\sigma(i)&=\prod_{i<j,\sigma(i)<\sigma(j)} 
	 	\sigma(j)-\sigma(i) \cdot \prod_{i<j,\sigma(i)>\sigma(j)} \sigma(j)-\sigma(i) \notag \\
	 	&=(-1)^{f(\sigma)}\cdot \prod_{i<j} |\sigma(j)-\sigma(i)| \notag \\
	 	&=\sgn(\sigma)\cdot \prod_{i<j} j-i \notag
	 \end{align}
\end{proof}

\begin{proposition}
	\proplbl{4_1_7}
	Die Abbildung $\sgn: S_n \to \mathbb Z^{\times}=\mu_2$ ist ein Gruppenhomomorphismus.
\end{proposition}
\begin{proof}
	Seien $\sigma,\tau\in S_n$. Dann ist
	\begin{align}
		\sgn(\sigma\tau)&=\prod_{i<j} \frac{\sigma(\tau(j))-\sigma(\tau(i))}{j-i} \notag \\
		&=\prod_{i<j} \frac{\sigma(\tau(j))-\sigma(\tau(i))}{\tau(j)-\tau(i)}\cdot \prod_{i<j} \frac{\tau(j)-\tau(i)}{j-i} \notag
	\end{align}
	Da mit $\{i,j\}$ auch $\{\tau(i),\tau(j)\}$ alle 
	zweielementigen Teilmengen von $\{1,...,n\}$ und 
	\begin{align}
		\frac{\sigma(\tau(j))-\sigma(\tau(i))}{\tau(j)-\tau(i)}=\frac{\sigma(\tau(i))-
			\sigma(\tau(j))}{\tau(i)-\tau(j)} \notag
	\end{align}
	ist 
	\begin{align}
		\prod_{i<j} \frac{\sigma(\tau(j))-\sigma(\tau(i))}{\tau(j)-\tau(i)}&=\prod_
		{i<j} \frac{\sigma(j)-\sigma(i)}{j-i} \notag \\
		&=\sgn(\sigma) \notag
	\end{align}
	und 
	\begin{align}
		\prod_{i<j} \frac{\tau(j)-\tau(i)}{j-i}=\sgn(\tau) \notag
	\end{align}
	Somit ist $\sgn(\sigma\tau)=\sgn(\sigma)\cdot \sgn(\tau)$.
\end{proof}

\begin{conclusion}
	\proplbl{4_1_8}
	Für $\sigma\in S_n$ ist 
	\begin{align}
		\sgn(\sigma^{-1})=\sgn(\sigma)\notag
	\end{align}
\end{conclusion}
\begin{proof}
	$\sgn(\sigma^{-1})=\sgn(\sigma)^{-1}=\sgn(\sigma)$
\end{proof}

\begin{conclusion}
	Sei $\sigma\in S_n$. Sind $\tau_1,...,\tau_r$ Transpositionen mit $\sigma=\tau_1\circ ... \circ \tau_r$, so ist 
	\begin{align}
		\sgn(\sigma)=(-1)^r\notag
	\end{align}
\end{conclusion}
\begin{proof}
	\propref{4_1_5} und \propref{4_1_7}
\end{proof}

\begin{conclusion}
	\proplbl{4_1_10}
	Die geraden Permutationen $A_n=\{\sigma \in S_n \mid \sgn(\sigma)=1\}$ bilden einen Normalteiler von $S_n$, 
	genannt die \begriff{alternierende Gruppe}. Ist $\tau\in S_n$ mit $\sgn(\tau)=-1$, so gilt für $A_n\tau=\{\sigma\tau \mid \sigma\in A_n\}$: 
	$A_n \cup A_n\tau = S_n$ und $A_n \cap A_n\tau=\emptyset$.
\end{conclusion}
\begin{proof}
	Es ist $A_n=Ker(\sgn)$ und nach \propref{3_2_13} ist dieser auch ein Normalteiler. Ist $\sigma\in S_n\backslash A_n$, so ist 
	\begin{align}
		\sgn(\sigma\tau^{-1})=
		\sgn(\sigma)\cdot \sgn(\tau)^{-1}=(-1)(-1)^{-1}=1\notag
	\end{align}
	also $\sigma=\sigma\tau^{-1}\in A_n\tau$, somit $A_n\cup A_n\tau=S_n$. Ist 
	$\sigma\in A_n$, so ist $\sgn(\sigma\tau)=-1$, also $A_n\cap A_n\tau=\emptyset$.
\end{proof}