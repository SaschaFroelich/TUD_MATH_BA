\section{Homomorphismen von Gruppen}

Seien $G,H$ zwei multiplikativ geschriebene Gruppen.

\begin{definition}[Gruppenhomomorphismus]
	Eine Abbildung $f: G \to H$ ist ein \begriff{Gruppenhomomorphismus}, wenn gilt:
	\begin{itemize}
		\item (GH): $f(xy)=f(x)\cdot f(y)$
	\end{itemize}
	Die Menge der Homomorphismen $f:G\to H$ bezeichnet man mit $\Hom(G,H)$.
\end{definition}

\begin{remark}
	Ein Gruppenhomomorphismus ist also eine Abbildung, welche mit der Verknüpfung, also der Struktur 
	der Gruppe, verträglich ist. Man beachte: für additiv geschrieben Gruppen lautet die Bedingung: $f(x+y)=f(x)+f(y)$.
\end{remark}

\begin{example}
	\begin{itemize}
		\item $\id_G: G \to G$
		\item $c_1:G\to H$ mit $x\mapsto 1_H$
		\item $G_0\le G$ Untergruppe, $\iota:G_0\to G$
		\item $(A,+)$ abelsche Gruppe, $k\in \mathbb Z$, $A\to A$ mit $a\mapsto ka$
		\item $\mathbb Z \to \mathbb Z\backslash n\mathbb Z$ mit $\overline a \mapsto a+n\mathbb Z$
		\item $\mathbb R \to \mathbb R^{\times}$ mit $x\mapsto e^x$
		\item $\Mat_n(K)\to \Mat_n(K)$ mit $A\mapsto A^t$
		\item $\mathbb C\to \mathbb R^{\times}$ mit $z\mapsto |z|$
	\end{itemize}
\end{example}

\begin{proposition}
	\proplbl{3_2_4}
	Sei $f\in \Hom(G,H)$. Dann gilt: 
	\begin{itemize}
		\item $f(1_G)\to 1_H$
		\item Für $x\in G$ ist $f(x^{-1})=(f(x))^{-1}$.
		\item Für $x_1,...,x_n\in G$ ist $f(x_1,...,x_n)=f(x_1)\cdot ... \cdot f(x_n)$.
		\item Ist $G_0\le G$, so ist $f(G_0)\le H$.
		\item Ist $H_0\le H$, so ist $f^{-1}(H_0)\le G$.
	\end{itemize}
\end{proposition}
\begin{proof}
	\begin{itemize}
		\item $f(1)=f(1\cdot 1)=f(1)\cdot f(1) \Rightarrow$ kürzen, weil $H$ Gruppe $\Rightarrow 1=f(1)$
		\item $f(x)\cdot f(x^{-1})=f(x\cdot x^{-1})=f(1)=1$
		\item Induktion nach $n$
		\item $x,y\in G_0\Rightarrow f(x)\cdot f(y)=f(xy)\in f(G_0)$, $f^{-1}(x)=f(x^{-1})\in f(G_0)$
		\item $x,y\in f^{-1}(H_0)\Rightarrow f(x)\cdot f(y)=f(xy)\in H_0\Rightarrow xy\in f^{-1}(H_0)$, $f(x^{-1})=(f(x))
		^{-1}\in H_0\Rightarrow x^{-1}\in f^{-1}(H_0)$, $f(1)=1\in H_0\Rightarrow 1\in f^{-1}(H_0)$, insbesondere 
		$f^{-1}(H_0)\neq \emptyset$
	\end{itemize}
\end{proof}

\begin{proposition}
	\proplbl{3_2_5}
	Seien $G_1,G_2,G_3$ Gruppen. Sind $f_1:G_1\to G_2$, $f_2:G_2\to G_3$ Homomorphismen, so ist auch 
	$f_2\circ f_1:G_1\to G_3$.
\end{proposition}
\begin{proof}
	Für $x,y\in G_1$ ist $(f_2\circ f_1)(xy)=f_2(f_1(xy))=f_2(f_1(x)\cdot f_1(y))=f_2(f_1(x))\cdot f_2(f_1(y))=(f_2
	\circ f_1)(x)\cdot (f_2\circ f_1)(y)$
\end{proof}

\begin{definition}[Arten von Homomorphismen]
	Ein Homomorphismus ist
	\begin{itemize}
		\item ein \begriff{Monomorphismus}, wenn $f$ injektiv ist
		\item ein \begriff{Epimorphismus}, wenn $f$ surjektiv ist
		\item ein \begriff{Isomorphismus}, wenn $f$ bijektiv ist.
	\end{itemize}
	Die Gruppen $G$ und $H$ heißen \begriff{isomorph}, in Zeichen $G\cong H$, wenn 
	es einen Isomorphismus $G\to H$ gibt.
\end{definition}

\begin{lemma}
	\proplbl{3_2_7}
	Ist $f:G\to H$ ein Isomorphismus, so ist auch $f^{-1}:H\to G$ ein Isomorphismus.
\end{lemma}
\begin{proof}
	Da $f^{-1}$ wieder bijektiv ist, müssen wir nur zeigen, dass $f^{-1}$ ein Homomorphismus ist. Seien $x,y\in H$. Dann 
	ist $f(f^{-1}(x)\cdot f^{-1}(y))=f(f^{-1}(x))\cdot f(f^{-1}(y))=xy$, somit $f^{-1}(xy)=f^{-1}(x)\cdot f^{-1}(y)$.
\end{proof}

\begin{proposition}
	\proplbl{3_2_8}
	Sei $f:G\to H$ ein Homomorphismus. Genau dann ist $f$ ein Isomorphismus, wenn es einen Homomorphismus 
	$f':H\to G$ mit $f'\circ f=\id_G$ und $f\circ f'=\id_H$ gibt.
\end{proposition}
\begin{proof}
	Ist $f$ ein Isomorphismus, so erfüllt $f':=f^{-1}$ nach \propref{3_2_7} das Gewünschte. Ist umgekehrt $f'$ wie angegeben, so muss $f$ 
	bijektiv sein:
	\begin{itemize}
		\item $f'\circ f=\id_G$ injektiv $\Rightarrow f$ injektiv
		\item $f\circ f'=\id_H$ surjektiv $\Rightarrow f$ surjektiv
	\end{itemize}
\end{proof}

\begin{conclusion}
	Isomorphie von Gruppen ist eine \begriff{Äquivalenzrelation}: Sind $G,G_1,G_2,G_3$ Gruppen, so gilt:
	\begin{itemize}
		\item $G\cong G$ (Reflexivität)
		\item Ist $G_1\cong G_2$, so ist auch $G_2\cong G_1$ (Symmetrie)
		\item Ist $G_1\cong G_2$ und $G_2\cong G_3$, dann ist auch $G_1\cong G_3$ (Transitivität)
	\end{itemize}
\end{conclusion}
\begin{proof}
	\begin{itemize}
		\item $\id_G$ ist ein Isomorphismus
		\item \propref{3_2_7}
		\item Folgt aus \propref{3_2_5} und der Tatsache, dass die Komposition bijektiver Abbildungen wieder bijektiv ist.
	\end{itemize}
\end{proof}

\begin{remark}
	\propref{3_2_8} erklärt die Bedeutung des Isomorphismus: Eine mit der Struktur verträgliche 
	Abbildung, die eine mit der Struktur verträgliche Umkehrabbildung besitzt, also eine strukturerhaltende Abbildung. 
	Tatsächlich können wir uns einen Isomorphismus $f: G\to H$ so vorstellen, dass wir nur die Elemente von $G$ umbenennen. 
	Alle Aussagen, die sich nur aus der Struktur selbst ergeben, bleiben damit wahr. Zum Beispiel: Ist $G\cong H$ und ist 
	$G$ abelsch, so auch $H$ und umgekehrt.
\end{remark}

\begin{example}
	\begin{itemize}
		\item Es ist $\mathbb Z^{\times} = \mu_2 \cong \mathbb Z\backslash 2\mathbb Z\cong (\mathbb Z\backslash 3\mathbb Z)
		^{\times}\cong S_2$. Je zwei beliebige Gruppen der Ordnung 2 sind zueinander isomorph.
		\item $e: \mathbb R \to \mathbb R_{>0}$, $x\mapsto e^x$ liefert einen Isomorphismus, da $(\mathbb R,+)\to 
		(\mathbb R,\cdot)$.
	\end{itemize}
\end{example}

\begin{definition}[Kern]
	Der \begriff{Kern} eines Gruppenhomomorphismus $f:G\to H$ ist $\Ker(f):= f^{-1}(\{1\})=\{x\in G \mid
	f(x)=1_H\}$.
\end{definition}

\begin{lemma}
	\proplbl{3_2_13}
	Ist $f:G\to H$ ein Homomorphismus, so ist $N:=\Ker(f)$ eine Untergruppe von $G$ mit $x\cdot y\cdot 
	x^{-1}\in N$ für alle $x\in G$ und $y\in N$.
\end{lemma}
\begin{proof}
	Nach \propref{3_2_4} ist $N$ eine Untergruppe. Für $x\in G$ und $y\in N$ ist $f(xyx^{-1})=f(x)\cdot f(y)\cdot f(x^{-1})=f(x)\cdot f(x^{-1}) \cdot 1=
	f(x)\cdot f(x^{-1})=1$, also $xyx^{-1}\in N$.
\end{proof}

\begin{proposition}
	\proplbl{3_2_14}
	Sei $f\in \Hom(G,H)$. Genau dann ist $f$ injektiv, wenn $\Ker(f)=\{1_G\}$.
\end{proposition}
\begin{proof}
	Schreibe $N=\Ker(f)$.
	\begin{itemize}
		\item Hinrichtung: Ist $f$ injektiv, so ist $N\le G$ mit $|N|\le 1$, also $N=\{1_G\}$.
		\item Rückrichtung: Sei $N=\{1_G\}$. Sind $x,y\in G$ mir $f(x)=f(y)$, so ist $1=(f(x))^{-1}\cdot f(y)=f(x^{-1}\cdot y)$, 
		also $x^{-1}\cdot y\in N=\{1\}$ und somit $x=y$. Folglich ist $f$ injektiv.
	\end{itemize}
\end{proof}

\begin{definition}[Normalteiler]
	Ist $N\le G$ mit $x^{-1}y\in N$ für alle $x\in G$ und $y\in N$, so nennt man $N$ 
	einen \begriff{Normalteiler} von $G$ und schreibt $N\vartriangleleft G$.
\end{definition}