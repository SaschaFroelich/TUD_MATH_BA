Schön, dass du unser Skript für die Vorlesung \textit{Programmieren für Mathematiker 1} bei Prof. Dr. Wolfgang Walter im WS2017/18 gefunden hast!

Wir verwalten dieses Skript mittels Github \footnote{Github ist eine Seite, mit der man Quelltext online verwalten kann. Dies ist dahingehend ganz nützlich, dass man die Quelltext-Dateien relativ einfach miteinander synchronisieren kann, wenn man mit mehren Leuten an einem Projekt arbeitet.}, d.h. du findest den gesamten \LaTeX-Quelltext auf \url{https://github.com/henrydatei/TUD_MATH_BA}. Unser Ziel ist, für alle Pflichtveranstaltungen von \textit{Mathematik-Bachelor} ein gut lesbares Skript anzubieten. Für die Programme, die in den Übungen zur Vorlesung \textit{Programmieren für Mathematiker} geschrieben werden sollen, habe ich ein eigenes Repository eingerichtet; es findet sich bei \url{https://github.com/henrydatei/TU_PROG}.

Es lohnt sich auf jeden Fall während des Studiums die Skriptsprache \LaTeX{} zu lernen, denn Dokumente, die viele mathematische oder physikalische Formeln enthalten, lassen sich sehr gut mittels \LaTeX{} darstellen, in Word oder anderen Office-Programmen sieht so etwas dann eher dürftig aus.

\LaTeX{} zu lernen ist gar nicht so schwierig, ich habe dafür am Anfang des ersten Semesters wenige Wochen benötigt, dann kannte ich die wichtigsten Befehle und konnte mein erstes Skript schreiben (\texttt{1. Semester/LAAG}, Vorsicht: hässlich, aber der Quelltext ist relativ gut verständlich). Inzwischen habe ich das Skript überarbeitet, lasse es aber noch für Interessenten online.

Es sei an dieser Stelle darauf hingewiesen (wie in jedem anderem Skript auch \smiley{}), dass dieses Skript nicht den Besuch der Vorlesungen ersetzen kann. Prof. Walter hat nicht wirklich eine Struktur in seiner Vorlesung, ich habe deswegen einiges umstrukturiert und ergänzt, damit es überhaupt lesbar wird. Wenn du Pech hast, ändert Prof. Walter seine Vorlesung grundlegend, aber egal wie: Wenn du noch nicht programmieren kannst, wirst du es durch die Vorlesung auch nicht lernen, sondern nur durch die Übungen; die Vorlesung ist da wenig hilfreich.

Wir möchten deswegen ein Skript bereitstellen, dass zum einen übersichtlich ist, zum anderen \textit{alle} Inhalte aus der Vorlesung enthält, das sind insbesondere Diagramme, die sich nicht im offiziellen Skript befinden, aber das Verständnis des Inhalts deutlich erleichtern. Ich denke, dass uns dies erfolgreich gelungen ist.

Trotz intensivem Korrekturlesen können sich immer noch Fehler in diesem Skript befinden. Es wäre deswegen ganz toll von dir, wenn du auf unserer Github-Seite \url{https://github.com/henrydatei/TUD_MATH_BA} ein neues Issue erstellst und damit auch anderen hilfst, dass dieses Skript immer besser wird.

Und jetzt viel Spaß bei \textit{Programmieren für Mathematiker}!

\begin{flushright}
	Henry, Pascal und Daniel
\end{flushright}