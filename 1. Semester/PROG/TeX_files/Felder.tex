\section{Felder (Arrays)}

Bisher hatten wir nur Skalare als Variablen. Was aber, wenn wir nicht wissen, wie viele Variablen wir brauchen werden. Dann helfen uns Felder. Felder haben eine homogene Datenstruktur, das heißt alle Elemente haben den selben Datentyp, den \begriff{Elementtyp}. Es gibt sowohl ein- als auch mehrdimensionale Felder (die höchste Dimension ist 15), also Vektoren, Matrizen, Tensoren, ... Der lesende als auch schreibende Zugriff erfolgt mittels ganzzahliger Indizees, also \texttt{vector(3)}, \texttt{matrix(2,5)}, \texttt{tensor(3,4,6)}, ... Unzulässige Indexwerte ermöglichen beliebige Speicherzugriffe auch außerhalb des Feldes, wenn sie nicht beim Kompilieren erkannt werden. Es kann also folgender Laufzeitfehler auftreten: \texttt{Index out of bounds}.