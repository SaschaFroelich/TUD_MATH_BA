Eine Programmiersprache ist lexikalisch, syntaktisch und semantisch eindeutig definiert. Eine \begriff{Compiler} übersetzt die \begriff{Programmiersprache} in \begriff{Maschinensprache}. Ein \begriff{Interpreter} arbeitet das Programm dann ab. Ein \begriff{Laufzeitsystem} stellt grundlegende Operationen und Funktionen zur Verfügung.

\section{Bereiche der Informatik}

Die Informatik untergliedert sich in 4 Bereiche:
\begin{itemize}
	\item Technische Informatik
	\item Praktische Informatik
	\item Theoretische Informatik
	\item Angewandte Informatik
\end{itemize}

Die \begriff[Informatik!]{Technische Informatik} beschäftigt sich mit der Konstruktion der Hardware, zum Beispiel der Datenleitungen, um Informationen durch das Internet zu transportieren. Wichtige Firmen sind hier: Intel, Globalfoundries und Infineon.

Die \begriff[Informatik!]{Praktische Informatik} beschäftigt sich mit der Software, also Betriebssystem, Compiler, Interpreter und so weiter. In alltäglicher Software findet sich rund 1 Fehler in 100 Zeilen Quelltext. In wichtiger Software, also Raketen, Betriebssysteme, ...,  ist es nur 1 Fehler pro 10.000 Zeilen Code.

Die \begriff[Informatik!]{Theoretische Informatik} beschäftigt sich mit Logik, formalen Sprachen, der Automatentheorie, Komplexität von Algorithmen, ...

Die \begriff[Informatik!]{Angewandte Informatik} beschäftigt sich mit der Praxis, dem Nutzer, der Interaktion zwischen Mensch und Maschine, ...

\section{Maßeinheiten und Größenordnungen}

Ein \begriff{bit} ist ein Kunstwort aus "'binary"' und "'digit"'. Es kann nur 2 Werte speichern: 0 und 1

Ein \begriff{nibble} ist eine Hexadezimalziffer, bündelt also 4 bits und kann damit 16 Werte annehmen: 0, 1, 2, 3, 4, 5, 6, 7, 8, 9, A, B, C, D, E und F.

Ein \begriff{byte} bündelt 2 nibble, also 8 bit. Er ist die gebräuchlichste, direkt addressierbare, kleinste Speichereinheit. Weitere Speichergrößen sind:
\begin{center}
	\begin{tabular}{c|c||c|c}
		\textbf{Name} & \textbf{Anzahl byte} & \textbf{Name} & \textbf{Anzahl byte} \\
		\hline
		1 KB & $10^3$ & 1 KiB & $2^{10}=1.024$ \\
		1 MB & $10^6$ & 1 MiB & $2^{20}=1.048.576$ \\
		1 GB & $10^9$ & 1 GiB & $2^{30}=1.073.741.824$ \\
		1 TB & $10^{12}$ & 1 TiB & $2^{40}$ \\
		1 PB & $10^{15}$ & 1 PiB & $2^{50}$ \\
		1 EB & $10^{18}$ & 1 EiB & $2^{60}$
	\end{tabular}
\end{center}

Der \begriff{ROM} ("'read-only-memory"') speichert wichtige Informationen auch ohne Strom, wie zum Beispiel die Uhrzeit, Informationen über die Festplatte, ... Er ist nicht mehr änderbar, außer durch Belichtung.

Der \begriff{RAM} ("'random-access-memory"') ermöglicht den Zugriff auf alle Adressen, insbesondere im Hauptspeicher.