\documentclass[ngerman,a4paper]{report}
\usepackage[table]{xcolor}

\usepackage[T1]{fontenc}
\usepackage{amsmath}
\usepackage{amssymb}
\usepackage{amsfonts}
\usepackage{latexsym}
\usepackage{graphicx}
%\usepackage{fontspec}
\usepackage[ngerman]{babel}
\usepackage[utf8]{inputenc}
%\usepackage{url} 
\usepackage[top=1cm,bottom=1.5cm,left=1cm,right=1cm]{geometry}
\usepackage{bbm}
\usepackage[hidelinks]{hyperref}

\usepackage{makeidx}
\makeindex

\usepackage{enumerate}
%\usepackage{enumitem}
\usepackage{tabularx}
\usepackage{booktabs}
\usepackage{cleveref}
\usepackage{xfrac}
\usepackage{parskip}

\usepackage{ulem}

\usepackage{ntheorem}

\usepackage{titlesec}

% diagramms
\usepackage{tikz} %diagrams
\usetikzlibrary{arrows,cd} % for commutative diagrams

\renewcommand{\thesection}{\arabic{section}}
\renewcommand{\thechapter}{\Roman{chapter}}
\titleformat{\chapter}[hang]{\huge\bfseries}{\thechapter}{15pt}{\huge\bfseries}
\titlespacing{\chapter}{0pt}{0pt}{0pt}
\titlespacing{\section}{0pt}{0pt}{0pt}

\theoremstyle{changebreak}
\theorembodyfont{}
\theorempostskip{7pt}
\theorempreskip{3pt}

\newtheorem{theorem}{Theorem}[section]
\newtheorem{example}[theorem]{Beispiel}
\newtheorem{corollar}[theorem]{Korollar}
\newtheorem{lemma}[theorem]{Lemma}
\newtheorem{satz}[theorem]{Satz}
\newtheorem{overview}[theorem]{Überblick}
\newtheorem{definition}[theorem]{Definition}
\newtheorem{remark}[theorem]{Bemerkung}
\newtheorem*{*remark}[theorem]{Bemerkung} %no numbered Remark

\newcommand{\highlight}[1]{\emph{#1}}
\newcommand{\begriff}[2][]{\uline{#2}\index{#1#2}}
\newcommand{\person}[1]{\textsc{#1}}

\DeclareMathOperator{\Abb}{Abb}
\DeclareMathOperator{\id}{id}
\DeclareMathOperator{\Ker}{Ker}
\DeclareMathOperator{\im}{Im}
\DeclareMathOperator{\Sym}{Sym}
\DeclareMathOperator{\spank}{span_K}
\DeclareMathOperator{\Mat}{Mat}
\DeclareMathOperator{\diag}{diag}
\DeclareMathOperator{\End}{End}
\DeclareMathOperator{\aut}{Aut}
\DeclareMathOperator{\Hom}{Hom}
\DeclareMathOperator{\GL}{GL}
\DeclareMathOperator{\rk}{rank}
\DeclareMathOperator{\SR}{SR}
\DeclareMathOperator{\ZR}{ZR}
\DeclareMathOperator{\sgn}{sgn}
\DeclareMathOperator{\SL}{SL}
\DeclareMathOperator{\Tr}{Tr}

\pagestyle{plain}

\begin{document}
\chapter{Grundbegriffe}
\setcounter{section}{1}
\section{Abbildungen}

\begin{overview}{Abbildung}
	Eine \begriff{Abbildung} $f$ von einer Menge $X$ in eine Menge $Y$ ist eine Vorschrift, die jedem $x\in X$ auf eindeutige Weise genau ein $f(x)\in Y$ zuordnet.
	
	Schreibweise: $f: X\rightarrow Y, x\mapsto f(x)$
	Wichtiges:
	\begin{itemize}
		\item $X$ heißt \begriff[Abbildung!]{Definitionsmenge}.
		\item $Y$ heißt \begriff[Abbildung!]{Zielmenge}
		\item Zwei Abbildungen sind gleich, wenn
		\begin{itemize}
			\item Definitionsmenge gleich,
			\item Zielmenge gleich,
			\item jedem $x\in X$ das gleiche Element $y\in Y$ zugeordnet wird.
		\end{itemize}
		\item $\Abb(X,Y)$: Menge der Abbildungen von $X$ nach $Y$.
		\item \begriff[Abbildung!]{Funktionen} Abbildungen mit Zielmenge $\mathbb{R}$.
	\end{itemize}
\end{overview}
\addtocounter{theorem}{1}
\begin{example}
	\begin{enumerate}[(a)]
		\item \begriff[Abbildung!]{Identische Abbildung} für jede Menge $X$ ist $\id_X: X \rightarrow X, x \mapsto x$.
		\item \begriff[Abbildung!]{Inklusionsabbildung} für jede Teilmenge $A \subseteq X$ ist $\iota_A: A \rightarrow X, x \mapsto x$
		\item \begriff[Abbildung!]{Konstante Abbildung} zu je 2 Mengen $X, Y$ und $y_0 \in Y$ ist $c_{y_0}: X \rightarrow Y, x \mapsto y_0$
		\item \begriff[Abbildung!]{Charakteristische Funktion} zu jeder Menge $X$ und Teilmenge $A \subseteq X$ ist \[\chi_A : X \rightarrow \mathbb{R}, x \mapsto \begin{cases}
			1, \text{falls $x\in A$} \\
			0, \text{falls $x\notin A$}
		\end{cases}\]
		\item \begriff[Abbildung!]{Kroneckersymbol} zu jeder Menge $X$ die Abbildung: \[X \times X \rightarrow \mathbb{R}, (x,y) \mapsto \delta_{x,y} := \begin{cases}
			1 \text{ falls $x = y$} \\
			0 \text{ falls $x\neq y$}
		\end{cases}\]
	\end{enumerate}
\end{example}

\begin{definition}[Eigenschaften Abbildung]
	Mit $f: X \rightarrow Y$ Abbildung:
	\begin{enumerate}
		\item $f$ \begriff[Abbildung!]{injektiv}, falls $\forall x, x' \in X: f(x) = f(x') \Rightarrow x = x'$
		\item $f$ \begriff[Abbildung!]{surjektiv}, falls $\forall y\in Y \,\exists x\in X: f(x) = y$
		\item $f$ \begriff[Abbildung!]{bijektiv}, falls $f$ surjektiv und injektiv
	\end{enumerate}
\end{definition}
\addtocounter{theorem}{1}
\begin{definition}[Restriktion, Urbild, Bild]
	Sei $f: X \rightarrow Y$ Abbildung. Dann
	\begin{itemize}
		\item \begriff[Abbildung!]{Restriktion / Einschränkung} Mit $A\subseteq X$ ist $f|_A: A\rightarrow Y, a \mapsto f(a)$
		\item \begriff[Abbildung!]{Bild} von $A\subset X$ unter $f$ ist $f(A) := \{ f(a) | \forall a \in A \}$
		\item \begriff[Abbildung!]{Urbild} von $B \subset Y$ unter $f$ ist $f^{-1}(B) := \{ x\in X | f(x) \in B \}$
		
		\begriff{Bild von f} $\im(f) := f(X)$
	\end{itemize}
\end{definition}
\addtocounter{theorem}{2}
\begin{definition}[Komposition]
	Mit Abbildungen $f: X \rightarrow Y$ und $g: Y \rightarrow Z$ ist \begriff{Komposition} $g\circ f: X \rightarrow Z, x \mapsto g(f(x))$.
	
	Abstrakt: $\circ: \Abb(Y,Z)\times \Abb(X,Y) \rightarrow \Abb(X,Z)$
\end{definition}

\begin{satz}
	Die Komposition von Abbildungen $\circ$ ist \begriff[Komposition!]{assoziativ}. $h\circ(g\circ f) = (f\circ g)\circ f$ mit entsprechend definierten Abbildungen.
\end{satz}

\begin{definition}
	Ist $f:X\rightarrow Y$ bijektive Abbildung, so existiert zu jedem $y\in Y$ ein $x_y \in X$ mit $f(x_y) = y$, folglich $f^{-1}: Y \rightarrow X, y \mapsto x_y$ ist \begriff[Abbildung!]{Umkehrabbildung}
\end{definition}

\begin{satz}
	Ist $f: X \rightarrow Y$ bijektiv, so ist $\id_X = f^{-1}\circ f = f \circ f^{-1}$
\end{satz}
\addtocounter{theorem}{1}
\begin{definition}[Familie]
	Mit $I, X$ Mengen heißt Abbildung $x: I \rightarrow X, i \mapsto x_i$ \begriff{Familie} von Elementen $X$ mit Indexmenge $I$ bzw. $I$-\begriff{Tupel} von Elementen von $X$.
\end{definition}

\begin{example}
	\begriff[Familie!]{Folge} ist Familie $(x_i)_{i\in\mathbb{N}_0}$ mit Indexmenge $\mathbb{N}_0$
\end{example}

\begin{definition}[Graph]
	\begriff[Abbildung!]{Graph} einer Abbildung $f:X \rightarrow Y$ ist $\Gamma_f:=\{ (x,y)\in X\times Y | y = f(x) \}$
\end{definition}

\section{Gruppen}
\begin{definition}
	Sei $G$ Menge. \begriff{Verknüpfung} auf $G$ ist Abbildung $*: G\times G\rightarrow G, (x,y)\mapsto x*y$.
	
	\begin{itemize}
		\item \begriff{Halbgruppe} ist ein Paar $(G,*)$, wenn gilt:
		\item[(G1)] \begriff{Assoziativität} Für $x,y,z\in G: (x*y)*z = x*(y*z)$
		\item \begriff{Monoid} ist Halbgruppe, wenn noch gilt:
		\item[(G2)] Es gibt ein $e\in G$, mit dem für alle $x\in G: x*e = e*x = x$
		\item \begriff{Neutrales Element} der Verknüpfung $*$: ein $e\in G$ wie in (G2) 
	\end{itemize}
\end{definition}
\addtocounter{theorem}{1}
\begin{satz}[Eindeutigkeit des neutralen Elements]
	Ein Monoid $(G,*)$ besitzt genau ein neutrales Element
\end{satz}

\begin{definition}
	\begriff{Gruppe} ist ein Monoid $(G,*)$ mit neutralem Element $e\in G$, für den noch gilt
	\begin{itemize}
		\item[(G3)] Für jedes $x\in G$ existiert ein $x'\in G: x*x' = x'*x = e$
	\end{itemize}

	\begriff[Gruppe!]{Kommutativität} Für alle $x,y\in G: x*y = y*x$. 
	Damit
	\begin{itemize}
		\item \begriff[Gruppe!]{abelsch} Gruppe, welche das Kommutativgesetz einhält
		\item \begriff[Gruppe!]{Inverses Element} heißt ein $x'\in G$ wie in (G3).
	\end{itemize}
\end{definition}
\addtocounter{theorem}{1}
\begin{satz}[Eindeutigkeit des Inversen]
	Ist $(G,*)$ eine Gruppe, so gibt es zu jedem $x\in G$ genau ein inverses Element.
\end{satz}

\begin{example}
	\begin{enumerate}[(a)]
		\item \begriff[Gruppe!]{Triviale Gruppe} besteht nur aus dem neutralen Element: $G:=  \{e\}$
		\item \begriff[Gruppe!]{Permutation} ist Menge $\Sym(X) := \{ f\in\Abb(X,X) \,|\, f \text{ bijektiv} \}$ auf Menge $X$, die mit der Komposition Gruppe bildet: $(\Sym(X),\circ)$, genannt \begriff[Gruppe!]{symmetrische Gruppe} auf X (für $n\in\mathbb{N}$ geschrieben als $S_n := \Sym(\{1,...,n\})$).
	\end{enumerate}
\end{example}
\addtocounter{theorem}{2}
\begin{satz}
	Mit $(G,\cdot\,)$ Gruppe und $x,y\in G$ gilt: $(x^{-1})^{-1} = x, (xy)^{-1} = y^{-1}x^{-1}$.
\end{satz}
\begin{satz}
	Mit $(G,\cdot\,)$ und $a,b\in G$ haben die Gleichungen $a\cdot x = b, y \cdot a = b$ eindeutige Lösungen ($x = a^{-1}\cdot b, y = b\cdot a^{-1}$), damit existieren die Kürzungsregeln.
\end{satz}
\begin{remark}
	\begin{itemize}
		\item \begriff[Gruppe!]{Endlich} Eine Gruppe $(G,\cdot\,)$ ist endlich, falls Menge $G$ endlich
		\item \begriff[Gruppe!]{Ordnung} ist die Mächtigkeit von $G$
		\item Endliche Gruppen können durch \begriff[Gruppe!]{Verknüpungstafeln} beschrieben werden.
	\end{itemize}
\end{remark}

\begin{definition}
	\begriff{Untergruppe} einer Gruppe $(G,\cdot)$ ist \highlight{nichtleere} Teilmenge $H\subseteq G$ mit
	\begin{itemize}
		\item[(UG1)] $x,y\in H: x\cdot y\in H$ (Abgeschlossenheit unter Multiplikation)
		\item[(UG2)] $x\in H: x^{-1}\in X$ (Abgeschlossenheit Inversem)
	\end{itemize}
\end{definition}

\begin{satz}
	Sei $(G,\cdot\,)$ Gruppe und $\emptyset \neq H\subseteq G$. Genau dann ist $H$ Untergruppe von $G$, wenn sich die Verknüpfung $\cdot$ zu einer Abbildung $\cdot_H: H\times H\rightarrow H$ einschränken lässt (d.h. $\cdot|_{H\times H} = \iota_H: H \rightarrow G$ die Inklusionsabbildung ist) und $(H, \cdot_H)$ Gruppe ist.
	
	Notation: $H \le G$
\end{satz}
\addtocounter{theorem}{1}
\begin{example}
	\begin{enumerate}[(a)]
		\item Jede Gruppe enthält \begriff[Untergruppe!]{triviale Untergruppe} $H = G, H = \{e\}$.
	\end{enumerate}
\end{example}

\begin{lemma}
	Ist $G$ eine Gruppe, $(H_i)_{i\in I}$ eine Familie von Untergruppen von $G$, so ist auch $H:= \bigcap_{i\in I} H_i$ Untergruppe von $G$. (Für $I = \emptyset$ setzt man $\bigcap_{i\in I}\in I) H_i = G$).
\end{lemma}

\begin{satz}
	Ist $G$ Gruppe und $X\subseteq G$ Teilmenge, so gibt es eindeutlich bestimmte \highlight{kleinste} Untergruppe $H$ von $G$, die $X$ enthält, d.h. $H$ enthält $X$, und ist $H'$ weitere Untergruppe von $G$, die $X$ enthält, so gilt $H\subseteq H'$.
\end{satz}

\begin{definition}
	Ist $G$ Gruppe und $X\subseteq G$ Teilmenge, so nennt man die kleinste Untergruppe von $G$, die $X$ enthält, die \begriff[Untergruppe!]{von $X$ erzeugte Untergruppe} von $G$.
	Wird $G$ selbst von endlicher Menge erzeugt, so heißt $G$ \begriff[Gruppe!]{endlich erzeugt}.
	
	Notation: $\langle X\rangle$ (falls $X=\{x_1, \dotsc, x_n\}$ endlich auch $\langle x_1, \dotsc, x_n\rangle$).
\end{definition}

\section{Ringe}

\begin{definition}[Ring]
	Ein \begriff{Ring} ist ein Tripel $(R, +, \cdot\,)$ aus Menge $R$ und Verknüpfungen $+:R\times R\rightarrow R$ ("`Addition"') bzw. $\cdot:R\times R \rightarrow R$ ("`Multiplikation"'), das erfüllt:
	\begin{itemize}
		\item[(R1)] $(R,+)$ ist abelsche Gruppe
		\item[(R2)] $(R,\cdot\,)$ ist Halbgruppe
		\item[(R3)] \begriff{Distributivgesetze} gelten für $a,x,y\in R$:
        \[ a\cdot(x+y) = a\cdot x + a\cdot y \qquad \text{und} \qquad (x+y)\cdot a = (x\dot a) + (y\cdot a) \]
	\end{itemize}

	Weiterhin:
	\begin{itemize}
		\item \begriff[Ring!]{kommutativ} ist ein Ring $(R,+,\cdot\,)$, wenn $x\cdot y = y\cdot x \;\forall x,y\in\mathbb{R}$
		\item \begriff{Einselement} ist das neutrale Element der Multiplikation $e\in R: e\cdot x = x \cdot e = x$.
		\item \begriff{Unterring} eines Ringes $(R,+,\cdot\,)$ ist Teilmenge $S\subseteq R$ mit geeigneter Einschränkung von Addition, Multiplikation.\\ 
        Aus Übung 31\\
        Ist $R$ ein Ring und $\emptyset \neq S \subseteqq R$, dann ist $S$ genau dann Unterring von $R$, wenn folgende Bedingungen gelten:
        \begin{itemize}
        \item[(UR1)] $S$ ist abgeschlossen bzgl. Addition
        \item[(UR2)] $S$ ist abgeschlossen bzgl. Bildung additiver Inverser
        \item[(UR3)] $S$ ist abgeschlossen bzgl. Multiplikation
        \end{itemize}
	\end{itemize}
\end{definition}
\addtocounter{theorem}{1}
\begin{example}
	(a) \begriff[Ring!]{Nullring} ist $R=\{0\}$ mit den einzig möglichen Verknüpfungen $+, \cdot$ und ist kommutativ mit $0$ als Einselement.
\end{example}

\begin{remark}
	Ist $R$ ein Ring, so gelten für $x,y\in R$:
	\begin{enumerate}[(a)]
		\item $0\cdot x = x\cdot 0 = 0$
		\item $x\cdot(-y) = (-x)\cdot y = -xy$
		\item $(-x) \cdot (-y) = xy$
	\end{enumerate}
\end{remark}
\addtocounter{theorem}{1}
\begin{theorem}[Division mit Rest in $\boldsymbol{\mathbb{Z}}$]
	Für jedes $a\in\mathbb{Z}$ gibt es eindeutig bestimmte $q,r\in\mathbb{Z}$ mit $a = qb + r$ und $0\le r < |b|$
\end{theorem}
\addtocounter{theorem}{2}
\begin{definition}[Charakteristik]
	Sei $R$ ein Ring mit Einselement $1$. Die \begriff[Ring!]{Charakteristik} von $R$ ist das kleinste $n\in\mathbb{N}$ mit $\underbrace{1 + \dotsc + 1}_{n \text{ viele}} = 0$, falls so ein $n$ existiert -- andernfalls hat $R$ die Charakteristik $0$.
\end{definition}

\begin{definition}[Nullteiler]
	Sei $R$ ein Ring. Ein $0\neq x$ heißt \begriff[Ring!]{Nullteiler} von $R$, wenn es ein $0\neq y\in R$ gibt mit $xy=0$ oder $yx=0$. Ein Ring ohne Nullteiler heißt \begriff[Ring!]{nullteilerfrei}.
\end{definition}

\begin{definition}[Einheit]
	Sei $R$ ein Ring mit Einselement $1$. Ein $x\in R$ heißt \begriff[Ring!]{invertierbar}, oder \begriff[Ring!]{Einheit} von $R$, wenn es $x'\in R$ mit $x x' = x' x = 1$ gibt.
	
	Notation: $R^\times$ ist Menge der invertierbaren Elemente.
\end{definition}
\addtocounter{theorem}{1}
\begin{satz}
	Sei $R$ ein Ring mit Einselement $1$.
	\begin{enumerate}[(a)]
		\item Ist $x\in R$ invertierbar, so ist $x$ kein Nullteiler in $R$.
		\item Die invertierbaren Elemente $R^\times$ von $R$ bilden mit der Multiplikation eine Gruppe.
	\end{enumerate}
\end{satz}

\section{Körper}
\begin{definition}
	Ein \begriff{Körper} ist ein kommutativer Ring $(K,+,\cdot\,)$ mit Einslement $1\neq 0$, indem jedes $0\neq x\in K$ invertierbar ist.
\end{definition}

\begin{remark}
	Ein Körper $K$ ist stets nullteilerfrei, und es gelten
	\begin{itemize}
		\item[(K1)] $(K,+)$ ist abelsche Gruppe mit neutralem Element $0$.
		\item[(K2)] $(K\setminus\{0\},\cdot\,)$ ist abelsche Gruppe mit neutralem Element $1$.
		\item[(K3)] Es gelten die Distributivgesetze (R3).
	\end{itemize}
\end{remark}
\addtocounter{theorem}{1}
\begin{definition}[Teilkörper]
	Ein \begriff{Teilkörper} eines Körpers $(K,+,\cdot\,)$ ist Teilmenge $L\subseteq K$, die mit geeigneter Einschränkung von $+$ und $\cdot$ wieder Körper ist.
\end{definition}
\addtocounter{theorem}{1}
\begin{example}[Komplexe Zahlen]
	\begriff{Komplexe Zahlen} ist Menge $\mathbb{C} := \mathbb{R}\times \mathbb{R}$, mit Addition / Multiplikation definiert als $\big((x_1,y_1), (x_2,y_2)\in\mathbb{C}\big)$ \[
		(x_1, y_1)+(x_2,y_2) := (x_1+x_2,y_1+y_2)\qquad (x_1,y_1)\cdot(x_2,y_2) := (x_1 x_2 - y_1 y_2, x_1 y_2 + x_2 y_1) \] und sind damit Körper. Die \begriff[Komplexe Zahlen!]{imaginäre Einheit} $i := (0,1)$ erfüllt $i^2 = -1$, und jedes Element $z\in\mathbb{C}$ lässt sich als $z = x+iy$ schreiben, $x,y\in\mathbb{R}$.
\end{example}

\begin{lemma}\label{Teiler}
	Sei $a\in\mathbb{Z}$ eine ganze Zahl und $p\in\mathbb{Z}$ eine Primzahl, die $a$ nicht teilt. Dann gibt es $b,k\in\mathbb{Z}$ mit $ab+kp = 1$.
\end{lemma}

\begin{example}[\index{Endlicher Primzahlkörper!Körper}Endlicher Primzahlkörper]
    Für jede Primzahl $p\in\mathbb{Z}$ ist $\mathbb{Z}\slash p\mathbb{Z}$ ein Körper. Ist $\bar{a}\neq \bar{0}$, so gilt $p \nmid a$, und somit gibt es nach \ref{Teiler} $b,k \in \mathbb{Z}$ mit
    \[
    \bar{1}=\overline{ab + kp} = \bar{a} + \bar{b}.
    \]
    Zusammen mit 4.12 und 4.13 erhalt man, dass für ein $n \in \mathbb{N}$ die folgenden Aussagen äquivalent sind:
    \begin{enumerate}[(1)]
    \item Der Ring $\mathbb{Z}\slash n\mathbb{Z}$ ist ein Körper.
    \item Der Ring $\mathbb{Z}\slash p\mathbb{Z}$ ist nullteilerfrei.
    \item $n$ ist Primzahl.
    \end{enumerate}
\end{example}

\section{Polynome}
Hier ist $R$ kommutativer Ring mit Einslement.
\addtocounter{theorem}{1}
\begin{definition}[Polynomring]
	Sei $R[X]$ die Menge der Folgen in $R$, die fast überall $0$ sind, also $R[X] := \{ (a_k)_{k\in\mathbb{N}_0} | \forall k: a_k\in R \text{ und } \exists n_0\,\forall k > n_0: a_k = 0 \}$. Addition und Multiplikation ist gegeben durch \[ (a_k)_{k\in\mathbb{N}_0} + (b_k)_{k\in\mathbb{N}_0} = (a_k + b_k)_{k\in\mathbb{N}_0}  \qquad (a_k)_{k\in\mathbb{N}_0}\cdot(b_k)_{k\in\mathbb{N}_0} = (c_k)_{k\in\mathbb{N}_0}, c_k = \sum_{i+j=k} a_i b_j \] Damit ist $R[X]$ ein kommutativer Ring mit Einselement, der \begriff{Polynomring} (in einer Variablen $X$) über $R$.
	
	Weiterhin
	\begin{itemize}
		\item \begriff{Polynom} ist $(a_k)_{k\in\mathbb{N}_0} \in R[X]$ mit \begriff[Polynom!]{Koeffizienten} $a_0, a_1, \dotsc$.
		\item Mit $x\in R$ und $(x,0,0,\dotsc)$ ist $R$ Unterring von $R[X]$
		\item Mit $X$ als Folge $(0,1,0,\dotsc)$ lässt sich $X^n = (\delta_{k,n})_{k\in\mathbb{N}_0}$ definieren. Damit schreibt sich auch jedes $(a_k)_{k\in\mathbb{N}_0}$ mit $a_k = 0$ für alle $k > n_0$ als
	\end{itemize}
	Notation: \[ f = f(X) = \sum_{k=0}^{n_0} a_k X^k = a_0 + a_1 X + \dotsc \qquad f = \sum_{k \ge 0} a_k X^k = \sum_{k\in\mathbb{N}_0} a_k X^k \]
	
	\begin{itemize}
		\item Der \begriff[Polynom!]{Grad} eines Polynoms $f$ ist $\deg(f) := \max\{ n\in\mathbb{N}_0 | a_n \neq 0 \}$ für $0\neq f(X) = \sum_{n \ge 0} a_k X^k\in R[X]$.
		\begin{itemize}
			\item $\deg(0) = -\infty$ (Grad des Nullpolynoms)
			\item \begriff[Polynom!]{Konstanter Term} ist $a_0$
			\item \begriff[Polynom!]{Leitkoeffizient} ist $a_{\deg(f)}$ von $f$.
			\item Hat $f$ den Grad $0, 1$ oder $2$, so heißt $f$ \begriff[Polynom!]{konstant}, \begriff[Polynom!]{linear} bzw. \begriff[Polynom!]{quadratisch}.
		\end{itemize}
	\end{itemize}
\end{definition}
\addtocounter{theorem}{1}
\begin{satz}
	Seien $f,g\in R[X]$.
	\begin{enumerate}[(a)]
		\item Es ist $\deg(f+g) \le \max\{ \deg(f), \deg(g) \}$
		\item Es ist $\deg(fg) \le \deg(f) + \deg(g)$
		\item Ist $R$ nullteilerfrei, dann ist $\deg(fg) = \deg(f) + \deg(g)$ und $R[X]$ ist nullteilerfrei.
	\end{enumerate}
\end{satz}

\begin{theorem}[Polynomdivision]
	Sei $K$ Körper und sei $0\neq g\in K[X]$. Für jedes $f\in K[X]$ gibt es eindeutig bestimmte $h,r\in K[X]$ mit $f = gh+r$ und $\deg(r) < \deg(g)$.
\end{theorem}
\addtocounter{theorem}{1}
\begin{definition}[Polynomauswertung]
	Sei $f(X) = \sum_{k \ge 0} a_k X^k\in R[X]$. Für $\lambda\in R$ ist die \begriff{Auswertung} von $f$ in $\lambda$ als $f(\lambda) = \sum_{k\ge 0} a_k \lambda^k \in R$.
	
	Das Polynom $f$ definiert so eine Abb. $\tilde{f}: R \rightarrow R, \lambda \mapsto f(\lambda)$.
	Ein $\lambda\in R$ mit $f(\lambda) = 0$ heißt \begriff[Polynom!]{Nullstelle} von $f$.
\end{definition}

\begin{lemma}
	Für $f,g\in R[X]$ und $\lambda\in R$ ist $(f+g)(\lambda) = f(\lambda) + g(\lambda)$ und $(fg)(\lambda) = f(\lambda)g(\lambda)$. 
\end{lemma}

\begin{satz}
	Ist $K$ Körper und $\lambda\in K$ Nullstelle von $f \in K[X]$, so gibt es eindeutiges $h\in K[X]$ mit $f(X) = (X-\lambda)\cdot h(x)$.
\end{satz}

\begin{corollar}
	Sei $K$ ein Körper. Ein Polynom $0 \neq f \in K[X]$ hat höchstens $\deg(f)$ viele Nullstellen in $K$.
\end{corollar}

\begin{corollar}
	Ist $K$ ein unendlicher Körper, so ist die Abbildung $K[X] \rightarrow \Abb(K, K), f\mapsto \tilde{f}$ injektiv.
\end{corollar}
\addtocounter{theorem}{1}
\begin{satz}
	\label{algebraisch_satz}
	Für einen Körper $K$ sind äquivalent:
	\begin{enumerate}[(1)]
		\item Jedes $f\in K[X]$ vom Grad $\deg(f)> 0$ hat eine Nullstelle in $K$.
		\item Jedes $0\neq f\in K[X]$ zerfällt über $K$ in Linearfaktoren, also $f(X) = a\cdot\prod_{i=1}^n (X-\lambda_n)$ mit $n = \deg(f)$, $a\in K$ und $\lambda_1,\dotsc,\lambda_n\in K$.
	\end{enumerate}
\end{satz}

\begin{definition}
	Ein Körper $K$ heißt \begriff{algebraisch abgeschlossen}, wenn er eine Bedingung aus Satz \ref{algebraisch_satz} erfüllt.
\end{definition}

\begin{theorem}[Fundamentalsatz der Algebra, \person{d'Alembert} 1746, \person{Gauss} 1799]
	Der Körper $\mathbb{C}$ der komplexen Zahlen ist algebraisch abgeschlossen.
\end{theorem}


\chapter{Vektorräume}
\section{Definition und Beispiele}
\addtocounter{theorem}{1}
\begin{definition}[Vektorraum]
	Ein $K$-\begriff{Vektorraum} (oder auch Vektorraum über $K$) ist ein Tripel $(V, +, \cdot\,)$ bestehend aus einer Menge $V$, einer Verknüpfung $+: V \times V \rightarrow V$, genannt Addition, und einer Abbildung $\cdot: K \times V \rightarrow V$, genannt \begriff[Vektorraum!]{Skalarmultiplikation}, mit
	\begin{itemize}
		\item[(V1)] $(V,+)$ ist abelsche Gruppe,
		\item[(V2)] Skalarmultiplikation ist verträglich, d.h. für $\lambda,\mu\in K$, $x,y\in V$:
		\begin{enumerate}[(i)]
			\item $\lambda\cdot(x+y) = \lambda x + \lambda y$
			\item $(\lambda + \mu)\cdot x = \lambda \cdot x + \mu \cdot x$
			\item $\lambda\cdot(\mu\cdot x) = (\lambda \cdot \mu) \cdot x$
			\item $1\cdot x = x$
		\end{enumerate}
	\end{itemize}

	Das neutrale Element von $(V,+)$ ist $\boldsymbol{0}$ und heißt \begriff{Nullvektor}.
\end{definition}
\addtocounter{theorem}{1}
\begin{example}[Standardraum]
	\begriff{Standardraum} Für $n\in\mathbb{N}$ ist $V = K^n := \prod_{i=1}^{n} K = \{ (x_1, \dotsc, x_n) | x_i \in K \}$ mit komponentenweiser Addition und komponentenweiser Skalarmultiplikation ein $K$-Vektorraum.
	
	\begriff{Nullraum} ist der Standardraum für $n=0$, d.h. $V = \{0\}$.
\end{example}

\begin{satz}
	Ist $V$ ein $K$-Vektorraum, so gelten für $\lambda\in K$ und $x\in V$:
	\begin{enumerate}[(a)]
		\item $0\cdot x = \boldsymbol{0}$
		\item $\lambda\cdot\boldsymbol{0} = \boldsymbol{0}$
		\item $(-\lambda) \cdot x = \lambda\cdot (-x) = -(\lambda\cdot x)$ (insbes. $-1\cdot x = -x$)
		\item $\lambda\cdot x = \boldsymbol{0} \Rightarrow \lambda = 0 \lor x = \boldsymbol{0}$
	\end{enumerate}
\end{satz}
\addtocounter{theorem}{1}
\begin{definition}[Untervektorraum]
	Sei $V$ ein $K$-Vektorraum. Ein \begriff{Untervektorraum} von $V$ ist \highlight{nichtleere} Teilmenge $W\subseteq V$ mit
	\begin{itemize}
		\item[(UV1)] Für $x,y\in W: x+y\in W$
		\item[(UV2)] Für $x\in W, \lambda\in K: \lambda x\in W$
	\end{itemize}
\end{definition}

\begin{satz}
	Sei $V$ ein $K$-Vektorraum, und $W\subseteq V$ eine Teilmenge. Genau dann ist $W$ ein Untervektorraum von V, wenn $W$ mit geeigneter Einschränkung von Addition und Skalarmultiplikation ein $K$-Vektorraum ist.
\end{satz}

\begin{example}
	\begriff[Untervektorraum!]{Triviale Untervektorräume} hat jeder $K$-Vektorraum $V$, nämlich $W=\{\boldsymbol{0}\}$ und $W = V$.
\end{example}

\begin{lemma}
	Ist $V$ ein $K$-Vektorraum und $(W_i)_{i\in I}$ eine Familie von Untervektorräumen von V, so ist auch $W := \bigcap_{i\in I} W_i$ ein Untervektorraum von $V$.
\end{lemma}

\begin{satz}
	Ist $V$ ein $K$-Vektorraum und $X\subseteq V$ eine Teilmenge, so gibt es einen eindeutig bestimmten kleinsten Untervektorraum $W$ von $V$, der $X$ enthält.
\end{satz}

\begin{definition}
	Ist $V$ ein $K$-Vektorraum und $X\subseteq V$ eine Teilmenge, so nennt man den kleinsten Untervektorraum von $V$, der $X$ enthält, den von $X$ \begriff[Untervektorraum!]{erzeugten} Untervektorraum.
	
	Notation: $\langle X \rangle$
	
	Eine Menge $X\subseteq V$ mit $\langle X \rangle = V$ heißt auch \begriff{Erzeugendensystem} von $V$. Der Vektorraum $V$ heißt \begriff[Vektorraum!]{endlich erzeugt}, wenn er ein endliches Erzeugendensystem $X\subseteq V$ besitzt.
\end{definition}

\section{Linearkombination und lineare Abhängigkeit}
Sei $V$ ein $K$-Vektorraum.

\begin{definition}[Linearkombination]
	\begin{enumerate}
		\item Sei $n\in\mathbb{N}_0$. Ein $x\in V$ ist eine ($K$-)\begriff{Linearkombination} eines $n$-Tupels $(x_1, \dotsc, x_n)$ von Elementen von $V$, wenn es $\lambda_1, \dotsc, \lambda_n\in K$ gibt mit $x = \lambda_1 x_1 + \dotsc + \lambda_n x_n$.
		
		Der Nullvektor ist stets Linearkombination, auch für $n=0$.
		
		\item Ein $x\in V$ ist eine Linearkombination einer Familie $(x_i)_{i\in I}$ von Elementen von $V$, wenn es $n\in\mathbb{N}_0$ und $i_1, \dotsc, i_n\in I$ gibt, für die $x$ eine Linearkombination von $(x_{i_1}, \dotsc, x_{i_n})$ ist.
		
		\item Die Menge $x\in V$, die eine Linearkombination von $\mathcal{F} = (x_i)_{i\in I}$ sind, wird mit $\spank (\mathcal{F})$ bezeichnet.
	\end{enumerate}
\end{definition}
\addtocounter{theorem}{1}
\begin{lemma}
	Für jede Teilmenge $X\subseteq V$ ist $\spank(X)$ ein Untervektorraum von $V$.
\end{lemma}

\begin{satz}
	Für jede Teilmenge $X\subseteq V$ ist $\spank(X) = \langle X \rangle$ der von $X$ erzeugte Untervektorraum von $V$.
\end{satz}

\begin{remark}
	Man nennt $\spank(X)$ auch den von $X$ \begriff[Untervektorraum!]{aufgespannten} Untervektorraum, oder die \begriff{lineare Hülle} von $X$.
\end{remark}

\begin{example}
	Sei $V = K^n$ der Standardraum. Für $i=1,\dotsc,n$ sei $e_i=(\delta_{i,1}, \dotsc, \delta_{i,n})$. Dann ist $\spank (e_1,\dotsc, e_n) = K^n$, und $K^n$ ist endlich erzeugt. Die $e_1,\dotsc,e_n$ heißen \begriff{Standardbasis}.
\end{example}

\begin{definition}[Lineare Abhängigkeit]
	\begin{enumerate}
		\item Sei $n\in \mathbb{N}_0$. Ein $n$-Tupel $(x_1, \dotsc, x_n)$ von Elementen von $V$ sind ($K$-)\begriff{linear abhängig}, wenn es $\lambda_1, \lambda_n\in K$ gibt, die nicht alle gleich Null sind, und $\lambda_1 x_1 + \dotsc + \lambda_n x_n = 0$. Andernfalls heißt $(x_1, \dotsc, x_n)$ \begriff{linear unabhängig}
		
		\item  	Eine Familie $(x_i)_{i\in I}$ von Elementen von $V$ ist linear abhängig, wenn es $n\in \mathbb{N}_0$ und paarweise verschiedene $i_1, \dotsc, i_n\in I$ gibt, für welche das $n$-Tupel $(x_{i_1}, \dotsc, x_{i_n})$ linear abhängig ist. Andernfalls heißt $(x_i)_{i\in I}$ linear unabhängig.
	\end{enumerate}
\end{definition}

\addtocounter{theorem}{1}
\begin{satz}
	Genau dann ist eine Familie $(x_i)_{i\in I}$ linear abhängig, wenn es ein $i_0\in I$ mit $x_{i_0}\in \spank\big( (x_i)_{i\in I\setminus\{i_0\}} \big)$ gibt. In diesem Fall ist $\spank\big( (x_i)_{i\in I} \big) = \spank\big( (x_i)_{i\in I\setminus\{i_0\}} \big)$.
\end{satz}

\begin{satz}
	Genau dann ist eine Familie $(x_i)_{i\in I}$ linear unabhängig, wenn sich jedes $x\in \spank\big( (x_i)_{i\in I} \big)$ in eindeutiger Weise als Linearkombination der $(x_i)_{i\in I}$ schreiben lässt, d.h. ist $x=\sum_{i\in I} \lambda_i x_i = \sum_{i\in I} \lambda'_i x_i$ mit $\lambda_i, \lambda'_i\in K$, fast alle gleich Null, so ist $\lambda_i = \lambda'_i$ für alle $i\in I$.
\end{satz}

\section{Basis und Dimension}
Sei $V$ ein $K$-Vektorraum.

\begin{definition}[Basis]
	Eine Familie $(x_i)_{i\in I}$ von Elementen von $V$ heißt ($K$-)\begriff{Basis} von $V$, wenn gilt:
	\begin{itemize}
		\item[(B1)] Die Familie $(x_i)_{i\in I}$ ist linear unabhängig
		\item[(B2)] Die Familie $(x_i)_{i\in I}$ erzeugt $V$, d.h. $\spank\big( (x_i)_{i\in I} \big) = V$.
	\end{itemize}
\end{definition}
\addtocounter{theorem}{1}
\begin{satz}
	Sei $(x_i)_{i\in I}$ eine Familie von Elementen  von $V$. Genau dann ist $(x_i)_{i\in I}$ eine Basis von $V$, wenn sich jedes $x\in V$ eindeutig als $x=\sum_{i\in I} \lambda_i x_i$ mit $\lambda_i\in K$, fast alle gleich Null, schreiben lässt.
\end{satz}
\addtocounter{theorem}{1}
\begin{satz}
	Für eine Familie $\mathcal{B} = (x_i)_{i\in I}$ von Elementen von $V$ sind folgende Aussagen äquivalent:
	\begin{enumerate}[(1)]
		\item $\mathcal{B}$ ist eine Basis von $V$.
		\item $\mathcal{B}$ ist minimales Erzeugendensystem, d.h. $\mathcal{B}$ ist Erzeugendensystem, aber jede Teilmenge $J\subsetneqq I$ ist $(x_i)_{i\in J}$ kein Erzeugendensystem.
		\item $\mathcal{B}$ ist maximal linear unabhängig, d.h. $\mathcal{B}$ ist linear unabhängig, aber jede Familie $(x_i)_{i\in J}$ mit $J \supsetneqq I$ ist linear abhängig.
	\end{enumerate}
\end{satz}

\begin{theorem}[Basisauswahlsatz]
	Jedes endliche Erzeugendensystem von $V$ besitzt eine Basis von $V$ als Teilfamilie: ist $(x_i)_{i\in I}$ ein endliches Erzeugendensystem, so gibt es eine Teilmenge $J\subseteq I$, für die $(x_i)_{i\in J}$ eine Basis ist.
\end{theorem}

\begin{corollar}
	Jeder endlich erzeugte $K$-Vektorraum besitzt eine endliche Basis.
\end{corollar}
\addtocounter{theorem}{2}
\begin{lemma}[Austauschlemma]
	Sei $\mathcal{B}= (x_1, \dotsc, x_n)$ eine Basis von $V$. Sind $\lambda_1, \dotsc, \lambda_n \in K$ und $y = \sum_{i=1}^{n} \lambda_i x_i$, so ist für jedes $j\in\{ 1, \dotsc, n \}$ mit $\lambda_j\neq 0$ auch $\mathcal{B}' = (x_1, \dotsc, x_{j-1}, y, x_{j+1}, \dotsc, x_n)$ eine Basis von $V$.
\end{lemma}

\begin{theorem}[\person{Steinitz}'scher Austauschsatz]
	Sei $\mathcal{B}=(x_1, \dotsc, x_n)$ eine Basis von $V$ und $\mathcal{F} = (y_1, \dotsc, y_n)$ eine linear unabhängige Familie in $V$. Dann ist $r \le n$ und es gibt $i_1, \dotsc, i_{n-r} \in \{ 1, \dotsc, n \}$, für die $\mathcal{B}' = (y_1, \dotsc, y_r, x_{i_1}, \dotsc, x_{i_{n-r}})$ eine Basis von $V$ ist.
\end{theorem}

\begin{corollar}[Basisergänzungssatz]
	Ist $V$ endlich erzeugt, so lässt sich jede linear unabhängige Familie zu einer Basis ergänzen: ist $(x_1, \dotsc, x_n)$ linear unabhängig, so gibt es $m \ge n$ und $x_{n+1}, \dotsc, x_{m}\in V$ derart, dass $(x_1, \dotsc, x_m)$ eine Basis von $V$ ist.
\end{corollar}

\begin{corollar}
	Sind $(x_i)_{i\in I}$ und $(y_j)_{j\in J}$ Basen von V, und ist $I$ endlich, so ist $|I| = |J|$.
\end{corollar}

\begin{corollar}
	Ist $V$ endlich erzeugt, so haben alle Basen von $V$ dieselbe Mächtigkeit.
\end{corollar}

\begin{definition}[Dimension]
	Ist $V$ endlich erzeugt, so ist die \begriff{Dimension} von $V$ die Mächtigkeit $\dim_K(V)$ eine Basis von $V$. Andernfalls sagt man, dass $V$ unendliche Dimension hat, und schreibt $\dim_K(V) = \infty$.
\end{definition}
\addtocounter{theorem}{2}
\begin{satz}
	Sei $V$ endlich erzeugt, und $W\subseteq V$ ein Untervektorraum.
	\begin{enumerate}[(a)]
		\item Es ist $\dim_K(W) \le \dim_K(V)$. Insbesondere ist $W$ endlich erzeugt.
		\item Ist $\dim_K(W) = \dim_K(V)$, so ist $V = W$.
	\end{enumerate}
\end{satz}

\section{Summen von Vektorräumen}
Sei $V$ ein $K$-Vektorraum und $(W_i)_{i\in I}$ eine Familie von Untervektorräumen von $V$.

\begin{definition}[Summe von Untervektorräumen]
	Die \begriff[Untervektorraum!]{Summe} $(W_i)_{i\in I}$ ist der Untervektorraum $\sum_{i\in I} W_i = \spank \left(\bigcup_{i\in I}W_i\right)$ von $V$. Im Fall $I=\{ 1, \dotsc, n \}$ schreibt man auch $W_1 + \dotsc + W_n$ für $\sum_{i\in I}W_i$.
\end{definition}

\begin{lemma}
	Es ist $\sum_{i\in I} = \left\lbrace \left.\sum_{i\in I} x_i \right| x_i\in W_i, \text{ fast alle gleich Null}\right\rbrace$
\end{lemma}
\addtocounter{theorem}{1}
\begin{satz}
	Es sind äquivalent:
	\begin{enumerate}[(1)]
		\item Jedes $x\in\sum_{i\in I} W_i$ ist eindeutig als $\sum_{i\in I}$ mit $x_i\in W_i$ darstellbar.
		\item Für jedes $i\in I$ ist $W_i\cap \sum_{j\in I\setminus\{i\}} W_j  = \{0\}$
	\end{enumerate}
\end{satz}

\begin{definition}[Direkte Summe von Untervektorräumen]
	Ist jedes $x\in \sum_{i\in I} W_i$ eindeutig als $\sum_{i\in I}x_i$ mit $x_i\in W_i$ darstellbar, so sagt man das $\sum_{i\in I} W_i$ die \begriff[Untervektorraum!]{direkte Summe} der Untervektorräume $(W_i)_{i\in I}$ ist, und schreibt $\bigoplus_{i\in I} W_i$ für $\sum_{i\in I} W_i$. Im Fall $I=\{1,\dotsc,n\}$ schreibt man auch $W_i \oplus \dots \oplus W_n$ für $\bigoplus_{i\in I} W_i$.
\end{definition}
\addtocounter{theorem}{2}
\begin{corollar}
	Seien $W_1, W_2$ Untervektorräume von $V$. Es sind äquivalent:
	\begin{enumerate}[(1)]
		\item $ V = W_1 \oplus W_2$
		\item $ V = W_1 + W_2$ und $W_1\cap W_2 = \{ 0 \}$.
	\end{enumerate}
\end{corollar}

\begin{satz}
	Seien $W_1, W_2$ Untervektorräume von $V$ mit den Basen $(x_i)_{i\in I_1}$ bzw. $(x_i)_{i\in I_2}$, wobei $I_1\cap I_2 = \emptyset$. Es sind äquivalent:
	\begin{enumerate}[(1)]
		\item $V = W_1 \oplus W_2$
		\item $(x_i)_{i\in I_1\cup I_2}$ ist eine Basis von $V$.
	\end{enumerate}
\end{satz}

\begin{corollar}
	Ist $V$ endlichdimensional, so ist jeder Untervektorraum eine direkte Summe, d.h. ist $W$ ein Untervektorraum von $V$, so gibt es (i.A. nicht eindeutig bestimmten) Untervektorraum $W'$ von $V$ (genannt \begriff{lineares Komplement} zu $W$) mit $V = W \oplus W'$. Es ist $\dim_K(W') = \dim_K(V) - \dim_K(W)$.
\end{corollar}
\addtocounter{theorem}{1}
\begin{theorem}[Dimensionsformel]
	Ist $V$ endlichdimensional und sind $W_1, W_2$ Untervektorräume von $V$, so ist $\dim_K(W_1 + W_2) + \dim_K(W_1\cap W_2) = \dim_K(W_1) + \dim_K(W_2)$.
\end{theorem}

\begin{definition}[Externes Produkt von Vektorräumen]
	Das \begriff[Vektorraum!]{(externe) Produkt} einer Familie $(V_i)_{i\in I}$ von $K$-Vektorräumen ist der $K$-Vektorraum $\prod_{i\in I} V_i$ bestehend aus dem kartesischen Produkt der $V_i$ mit komponentenweiser Addition und Skalarmultiplikation.
\end{definition}

\begin{definition}[Externe direkte Summe von Vektorräumen]
	Die \begriff[Vektorraum!]{(externe) direkte Summe} einer Familie $(V_i)_{i\in I}$ von $K$-Vektorräumen ist der Untervektorraum \\ $\bigoplus_{i\in I} V_i := \left\lbrace \left.(x_i)_{i\in I} \in \prod_{i\in I} V_i \right| x_i = 0 \text{ für fast alle $i$}\right\rbrace$ des Produktes $\prod_{i\in I} V_i$.
\end{definition}
\addtocounter{theorem}{1}
\begin{lemma}
	Sei $(V_i)_{i\in I}$ eine Familie von $K$-Vektorräumen und $V:= \bigoplus_{i\in I} V_i$. Für jedes $j\in I$ ist $\tilde{V}_j = V_j\times\prod_{i\in I\setminus\{j\}} \{0\}$ ein Untervektorraum von $V$, und $V = \bigoplus_{i\in I} \tilde{V}_i$.
\end{lemma}

\chapter{Lineare Abbildungen}
In diesem Kapitel sei $K$ ein Körper.

\section{Matrizen}
\begin{definition}[Matrix]
	Seien $m,n\in\mathbb{N}_0$. Eine $m\times n$-\begriff{Matrix} über $K$ ist en rechteckiges Schema \[ A = \begin{pmatrix}
	a_{11} & \dots & a_{1n}  \\ \vdots & & \vdots \\ a_{m1} & \dots & a_{mn}
	\end{pmatrix} \] mit $a_{ij}\in K$ für $i=1,\dotsc, m; j=1,\dotsc,n$. Man schreibt dies auch als \[A = (a_{ij})_{\shortstack{$\scriptstyle i=1,\dotsc,m$\\$\scriptstyle j=1,\dotsc,n$}}\], oder einfach $A = (a_{ij})_{i,j}$, wenn $m$ und $n$ aus dem Kontext hervorgehen.
	\begin{itemize}
		\item Die $a_{i,j}$ heißen \begriff[Matrix!]{Koeffizienten} der Matrix, und wir definieren $(A)_{ij} := a_{ij}$.
		\item Die Menge der $m\times n$-Matrizen wird mit $\Mat_{m\times n}(K)$ oder $K^{m\times n}$ bezeichnet.
		\item Man nennt das Paar $(m,n)$ auch den \begriff[Matrix!]{Typ} (manchmal auch \highlight{Dimension}) der Matrix
		
		Ist $m = n$, so spricht man von \begriff[Matrix!]{quadratisch}en Matrix, und schreibt $\Mat_n(K) := \Mat_{n\times n}(K)$.
		\item Zu einer Matrix $A=(a_{ij})_{i,j}\in \Mat_{m\times n}(K)$ definiert man die \begriff[Matrix!]{transponiert}e Matrix $A^t := (a_{ij})_{j,i}\in\Mat_{n\times m}(K)$.
	\end{itemize}
\end{definition}

\begin{example}
	Seien $m,n\in\mathbb{N}$ fest.
	\begin{enumerate}[(a)]
		\item Die \begriff[Matrix!]{Nullmatrix} ist $0 = (0)_{i,j}\in\Mat_{m\times n}(K)$.
		\item Für $k\in\{1,\dotsc,m\}$ und $l\in\{1,\dotsc,n\}$ ist die $(k,l)-$\begriff[Matrix!]{Basismatrix} gegeben durch $E_{kl} = (\delta_{ik}\delta_{jl})_{i,j}\in\Mat_{m\times n}(K)$.
		\item Die \begriff[Matrix!]{Einheitsmatrix} ist $\mathbbm{1}_n = (\delta_{i,i})_{i,j}\in\Mat_n(K)$. Insbesondere ist $\mathbbm{1}_n = \diag(1,\dotsc,1)$
		\item Für die Permutation $\sigma\in S_n$ definiert man die \begriff[Matrix!]{Permutationsmatrix} $P_\sigma=(\delta_{i,1}\sigma(i),j)_{i,j}\in\Mat_n(K)$
		\item Für $a_1,\dotsc, a_n\in K$ hat man den \begriff{Zeilenvektor} $(a_1,\dotsc, a_n) := (a_1 \; \dotsc \; a_n)\in\Mat_{1\times n}(K)$, sowie \\
		den \begriff{Spaltenvektor} $(a_1,\dotsc,a_n)^t = \begin{pmatrix}
		a_1 \\ \vdots \\ a_n \end{pmatrix} \in \Mat_{n\times 1}(K)$.
	\end{enumerate}
\end{example}

\begin{definition}
	Seien $A=(a_{ij})_{i,j}, B = (b_{ij})_{i,j}\in\Mat_{m\times n}(K)$ und $\lambda\in K$. Man definiert auf $\Mat_{m\times n}(K)$ koeffizientenweise \begriff[Matrix!]{Addition} und \begriff[Matrix!]{Skalarmultiplikation}.
\end{definition}

\begin{satz}
	$(\Mat_{m\times n}(K), +, \cdot\,)$ ist ein $K$-Vektorraum der Dimension $\dim_K (\Mat_{m\times n}(K)) = mn$ mit Basis $\mathcal{B} = (E_{ij})_{(i,j)\in\{1,\dotsc,m\}\times\{1,\dotsc,n\}}$.
\end{satz}

\begin{definition}[Matrizenmultiplikation]
	Seien $m,n,r\in\mathbb{N}_0$. Sind \[ A = (a_{ij})_{\shortstack{$\scriptstyle i=1,\dotsc,m$\\$\scriptstyle j=1,\dotsc,n$}}\in\Mat_{m\times n}(K) \quad\text{und}\quad B = (b_{jk})_{\shortstack{$\scriptstyle j=1,\dotsc,n$\\$\scriptstyle k=1,\dotsc,r$}}\in\Mat_{m\times r}(K) \], so definieren wir $C = A\cdot B$ als die Matrix $C=(c_{ik})_{\shortstack{$\scriptstyle i=1,\dotsc,m$\\$\scriptstyle k=1,\dotsc,r$}}\in\Mat_{m\times r}(K)$ mit $c_{ik} = \sum_{j=1}^{n} a_{ij}b_{jk}$
\end{definition}
\addtocounter{theorem}{1}
\begin{lemma}
	Für $m,n,r\in \mathbb{N}_0, A\in\Mat_{m\times n}(K), \mathcal{B}\in\Mat_{n\times r}(K)$ und $\lambda\in K$ gilt $A(\lambda\cdot B) = (\lambda\cdot A)B= \lambda\cdot AB$.
\end{lemma}

\begin{lemma}
	Matrizenmultiplikation ist assoziativ: für $m,n,r,s\in \mathbb{N}_0, A\in\Mat_{m\times n}(K), B\in\Mat_{n\times r},C\in\Mat_{r\times s}$ ist $A(BC) = (AB)C$.
\end{lemma}

\begin{lemma}
	Für $m,n,r\in\mathbb{N}_0, A, A'\in\Mat_{m\times n}(K)$ und $B,B'\in\Mat_{n\times r}(K)$ ist $(A+A')B = AB + A'B$ und $A(B+B') = AB + AB'$.
\end{lemma}

\begin{satz}
	Mit der Matrizenmultiplikation wird $\Mat_n(K)$ zu einem Ring mit Einselement $\mathbbm{1}_n$.
\end{satz}
\addtocounter{theorem}{1}
\begin{definition}
	Eine Matrix $A\in\Mat_n(K)$ heißt \begriff[Matrix!]{invertierbar} oder \begriff[Matrix!]{regulär}, wenn sie im Ring $\Mat_n(K)$ invertierbar ist, sonst \begriff[Matrix!]{singulär}.
	
	die Gruppe $\text{GL}_n(K) = \Mat_n(K)^\times$ der invertierbaren Matrizen heißt die \begriff[Matrix!]{allgemeine lineare Gruppe}.
\end{definition}
\addtocounter{theorem}{1}
\begin{lemma}
	Für $A, A_1, A_2\in\Mat_{m\times n}(K)$ und $B\in \Mat_{n\times r}(K)$ ist $(A_1 + A_2)^t = A_1^t + A_2^t, (A^t)^t = A$ und $(AB)^t = B^t A^t$.
\end{lemma}

\begin{satz}
	Für $A\in \text{GL}_n(K)$ ist $A^t \in \text{GL}_n(K)$ und $(A^t)^{-1} = (A^{-1})^t$.
\end{satz}

\section{Homomorphismen von Gruppen}
Seien $G$ und $H$ zwei multiplikativ geschriebene Gruppen.

\begin{definition}
	Eine Abbildung $f: G\rightarrow H$ heißt \begriff{Gruppenhomomorphismus} (oder ein \begriff{Homomorphismus} von Gruppen), wenn für alle $x,y\in G$ gilt: \begin{itemize}
		\item[(GH)] $f(xy) = f(x)f(y)$
	\end{itemize}

	Die Menge der Homomorphismen $f:G\rightarrow H$ bezeichnet man mit $\hom(G,H)$.
\end{definition}
\addtocounter{theorem}{2}
\begin{satz}
	Sei $f:G\rightarrow H$ ein Gruppenhomomorphismus. Dann gilt:
	\begin{enumerate}[(a)]
		\item $f(1) = 1$
		\item Für $x\in G$ ist $f(x^{-1}) = f(x)^{-1}$
		\item Für $x_1, \dotsc, x_n \in G$ ist $f(x1 \dots x_n) = f(x_1)\dots f(x_n)$
		\item Ist $G_0\le G$ eine Untergruppe, so ist $f(G_0) \le H$.
		\item Ist $H_0 \le H$ eine Untergruppe, ist ist $f^{-1}(H_0) \le G$.
	\end{enumerate}
\end{satz}

\begin{satz}
	Seien $G_1, G_2$ und $G_3$ Gruppen. Sind $f_1: G_1\rightarrow G_2$ und $f_2: G_2 \rightarrow G_3$ Gruppenhomomorhpismen, so ist auch $f_2\circ f_1: G_1\rightarrow G_3$ ein Gruppenhomomorphimus.
\end{satz}

\begin{definition}
	Ein Homomorphismus $f:G\rightarrow H$ ist ein \begriff{Monomorphismus}, wenn $f$ injektiv ist, ein \begriff{Epimorphismus}, wenn $f$ surjektiv ist, und ein \begriff{Isomorphismus}, wenn $f$ bijektiv ist.
	
	Die Gruppen $G$ und $H$ heißen \begriff{isomorph}, wenn es einen Isomorphismus $f:G\rightarrow H$ gibt.\\
	Notation: $G\cong H$.
\end{definition}

\begin{lemma}
	Ist $f:G\rightarrow H$ ein Isomorphismus, so ist auch $f^{-1}:H \rightarrow G$ ein Isomorphismus.
\end{lemma}

\begin{satz}
	Sei $f:G\rightarrow H$ ein Homomorphismus. Genau dann ist $f$ ein Isomorphismus, wenn es einen Homomorphismus $f':H\rightarrow G$ mit $f'\circ f = \id_G$ und $f\circ f'=\id_H$ gibt.
\end{satz}

\begin{corollar}
	Isomorphie von Gruppen ist eine \highlight{Äquivalenzrelation}: Sind $G, G_1, G_2, G_3$ Gruppen, so gilt:
	\begin{enumerate}[(i)]
		\item $G\cong G$ (Reflexivität)
		\item Ist $G_1\cong G_2$, so auch $G_2\cong G_1$ (Symmetrie)
		\item Ist $G_1\cong G_2$ und $G_2\cong G_3$, so auch $G_1\cong G_3$. (Transitivität)
	\end{enumerate}
\end{corollar}
\addtocounter{theorem}{2}
\begin{definition}
	Der \begriff{Kern} eines Gruppenhomomorphismus $f:G\rightarrow H$ ist $\Ker(f):= f^{-1}(\{1\}) = \{x\in G\,|\,f(x) = 1\}$
\end{definition}

\begin{lemma}
	Ist $f:G\rightarrow H$ ein Homomorphismus, so ist $N:= \Ker(f)$ eine Untergruppe von $G$ mit $x^{-1}yx\in N$ für alle $x\in G, y\in N$.
\end{lemma}

\begin{satz}
	Sei $f:G\rightarrow H$ ein Homomorphismus. Genau dann ist $f$ injektiv, wenn $\Ker(f) = \{1\}$.
\end{satz}

\begin{definition}\label{Normalteiler}
	Ist $N$ eine Untergruppe von $G$ mit $x^{-1}yx\in N$ für alle $x\in G, y\in N$, so nennt man $N$ einen \begriff{Normalteiler} von $G$.\\
	Notation: $N\unlhd G$.
\end{definition}

\section{Homomorphismus von Ringen}
Seien $R, S, T$ Ringe.

\begin{definition}
	Eine Abbildung $f:R\rightarrow S$ heißt \begriff{Ringhomomorphismus} (oder ein \begriff{Homomorphismus von Ringen}), wenn für $x,y\in R$ gilt:
	\begin{itemize}
		\item[(RH1)] $f(x+y) = f(x)+f(y)$
		\item[(RH2)] $f(xy) = f(x) f(y)$
	\end{itemize}

	\begin{itemize}
		\item Die Menge der Homomorphismen $f:R\rightarrow S$ wird mit $\hom(R,S)$ bezeichnet.
		\item Ein Homomorphismus $f:R\rightarrow S$ ist ein \begriff{Monomorphismus}, \begriff{Epimorphismus} oder \begriff{Isomorphismus}, wenn $f$ injektiv, surjektiv oder bijektiv ist.
		\item Gibt es einen Isomorphismus $f:R\rightarrow S$, so nennt man $S$ und $R$ \highlight{isomorph}.\\
		Notation: $S\cong R$
		\item Ein Element aus $\End(R) := \Hom(R,R)$ nennt man \begriff{Endomorphismus} von $R$.
		\item Der \begriff{Kern} eines Ringhomomorphismus $f:R\rightarrow S$ ist $\Ker(f) := f^{-1}(\{0\})$
	\end{itemize}
\end{definition}
\addtocounter{theorem}{2}
\begin{satz}
	Sind $f:R\rightarrow S$ und $g:S\rightarrow T$ Ringisomorphismen, so ist auch $g\circ f:R\rightarrow T$ ein Ringisomorphismus.
\end{satz}

\begin{lemma}
	Ist $f:R\rightarrow S$ ein Ringisomorphismus, so auch $f^{-1}:S\rightarrow S$.
\end{lemma}

\begin{satz}
	Sei $f:R\rightarrow S$ ein Ringhomomorphismus. Genau dann ist $f$ ein Ringisomorphismus, wenn es einen Ringhomomorphismus $f':S\rightarrow R$ mit $f'\circ f = \id_R$ und $f\circ f'=\id_S$ gibt.
\end{satz}

\begin{lemma}
	Der Kern $I:= \Ker(f)$ eines Ringhomomorphismus $f:R\rightarrow S$ ist eine Untergruppe von $(R,+)$ und $xa\in I$ und $ax\in I$ für alle $x\in R, a\in I$.
\end{lemma}

\begin{satz}
	Sei $f:R\rightarrow S$ ein Ringhomomorphismus. Genau dann ist $f$ injektiv, wenn $\Ker(f) = \{0\}$.
\end{satz}

\begin{definition}
	Ist $I$ eine Untergruppe von $(R,+)$ mit $xa\in I$ und $ax\in I$ für alle $x\in R$ und $a\in I$, so nennt man $I$ \begriff{Ideal} von $R$ und schreibt $I\unlhd R$.
\end{definition}

\section{Homomorphismen von Vektorräumen}
Seien $V, W$ und $U$ drei $K$-Vektorräume.

\begin{definition}[Lineare Abbildung]
	Eine Abbildung $f:V\rightarrow W$ heißt ($K$-)\begriff{linear} (oder auch ein \highlight{Homomorphismus von $K$-Vektorräumen}), wenn für alle $x,y\in V$ und $\lambda\in K$ gilt:
	\begin{enumerate}[(L1)]
		\item $f(x+y) = f(x) + f(y)$ (\highlight{Additivität})
		\item $f(\lambda x) = \lambda f(x)$ (\highlight{Homogenität})
	\end{enumerate}
    

	\begin{itemize}
		\item Die Menge der $K$-linearen Abbildungen $f\in\Abb(V,W)$ wird mit $\Hom_K(V,W)$ bezeichnet.
		\item Die Elemente $\End_K(V) := \Hom(V,V)$ nennt man \begriff{Endomorphismus} von $V$.
		\item Eine lineare Abbildung $f: V\rightarrow W$ ist ein \begriff{Monomorphismus}, \begriff{Epimorphismus} bzw. \begriff{Isomorphismus}, falls $f$ injektiv, surjektiv oder bijektiv ist.
		\item Einen Endomorphismus $f:V\rightarrow V$, der auch Isomorphismus ist, nennt man \begriff{Automorphismus} von $V$. \\
		Notation: $\aut_K(V)$ (Menge der Automorphismen)
		\item Der \begriff{Kern} einer linearen Abbildung $f:V\rightarrow W$ ist $\Ker(f) := f^{-1}(\{0\})$
	\end{itemize}
\end{definition}
\addtocounter{theorem}{1}
\begin{satz}
	Eine Abbildung $f:V\rightarrow W$ ist genau dann $K$-linear, wenn für alle $x,y\in V$ und $\lambda,\mu\in K$ gilt:
	\begin{itemize}
		\item[(L)] $f(\lambda x + \mu y) = \lambda f(x) + \mu f(y)$
	\end{itemize}
\end{satz}
\addtocounter{theorem}{1}
\begin{example}
	Sei $V=K^n$ und $W=K^m$. Wir fassen die Elemente von $V$ und $W$ als Spaltenvektoren auf. Zu einer Matrix $A\in\Mat_{m\times n}(K)$ definieren wir eine Abbildung $f_A:V\rightarrow W$ durch $f_a(x) = Ax$.
\end{example}

\begin{satz}
	Sei $f:V\rightarrow W$ eine $K$-lineare Abbildung. Dann gilt:
	\begin{enumerate}[(a)]
		\item $f(0) = 0$
		\item Für $x,y\in V$ ist $f(x-y) = f(x) - f(y)$
		\item Sind $(x_i)_{i\in I}$ aus $V$ und $(\lambda_i)_{i\in I}$ aus $K$, fast alle gleich Null, so ist $f(\sum_{i\in I} \lambda_i x_i) = \sum_{i\in I}\lambda_if(x_i)$.
		\item Ist $(x_i)_{i\in I}$ linear abhängig in $V$, so ist $f\big( (x_i)_{i\in I} \big)$ linear abhängig in $W$.
		\item Ist $V_0\subseteq V$ ein Untervektorraum, so auch $f(V_0)\subseteq W$ von $W$.
		\item Ist $W_0\subseteq W$ ein Untervektorraum, so auch $f^{-1}(W)\subseteq V$ von $V$.
	\end{enumerate}
\end{satz}

\begin{satz}
	Die Komposition $K$-linearer Abbildungen ist wieder $K$-linear: sind $f:V\rightarrow W$ und $g:W\rightarrow U$ zwei $K$-lineare Abbildungen, so auch $g\circ f:V\rightarrow U$.
\end{satz}

\begin{lemma}
	Ist $f:V\rightarrow W$ ein Isomorphismus, so ist auch $f^{-1}:W\rightarrow V$.
\end{lemma}

\begin{satz}
	Sei $f:V\rightarrow W$ linear. Genau dann ist $f$ ein Isomorphismus, wenn eine lineare Abbildung $f':W\rightarrow V$ existiert mit $f'\circ f=\id_V, f\circ f' = \id_W$.
\end{satz}
\addtocounter{theorem}{1}
\begin{satz}
	Ist $f:V\rightarrow W$ eine lineare Abbildung, so ist $\Ker(f)$ ein Untervektorraum von $V$. Genau dann ist $f$ ein Monomorphismus, wenn $\Ker(f) = \{0\}$.
\end{satz}

\section{Der Vektorraum der linearen Abbildungen}
Seien $V, W$ zwei $K$-Vektorräume.

\begin{satz}
	Sei $(x_i)_{i\in I}$ eine Basis von $V$ und $(y_i)_{i\in I}$ eine Familie in $W$. Dann gibt es genau eine lineare Abbildung $f:V\rightarrow W$ mit $f(x_i) = y_i$. für alle $i$. Diese ist durch $f(\sum_{i\in I} \lambda_i x_i) = \sum_{i\in I}\lambda_i f(x_i)$ gegeben und erfüllt
	\begin{enumerate}[(a)]
		\item $\im(f) = \spank\big( (y_i)_{i\in I} \big)$,
		\item genau dann ist $f$ injektiv, wenn $(y_i)_{i\in I}$ linear unabhängig ist.
	\end{enumerate}
\end{satz}

\begin{corollar}
	Ist $V$ endlich dimensional, $(x_1, \dotsc, x_n)$ eine linear unabhängige Familie in $V$ und $(y_1, \dotsc, y_n)$ eine Familie in $W$, so gibt es eine lineare Abbildung $f:V\rightarrow W$ mit $f(x_i) = y_i$ für alle $i$.
\end{corollar}

\begin{corollar}
	Ist $(x_i)_{i\in I}$ eine Basis von $V$ und $(y_i)_{i \in I}$ eine Basis von $W$, so gibt es genau einen Isomorphismus $f:V\rightarrow W$ mit $f(x_i) = y_i$ für alle $i$.
\end{corollar}

\begin{corollar}
	Zwei endlichdimensionale $K$-Vektorräume sind genau dann zueinander isomorph, wenn sie dieselbe Dimension haben.
\end{corollar}

\begin{corollar}
	Ist $\mathcal{B}=(v_1, \dotsc, v_n)$ eine Basis von $V$, so gibt es genau einen Isomorphismus $\Phi_{\mathcal{B}}: K^n\rightarrow V$ mit $\Phi_{\mathcal{B}}(e_i) = v_i$ für $i=1,\dotsc, n$. Insbesondere ist jeder endlich dimensionale $K$-Vektorraum $V$ isomorph zu einem Standardvektorraum $K^n$, nämlich für $n = \dim(V)$.
\end{corollar}

\begin{definition}
	Die Abbildung $\Phi_{\mathcal{B}}$ heißt \begriff{Koordinatensystem} zu $\mathcal{B}$. Für $v\in V$ ist $(x_1, \dotsc, x_n)^t = \Phi_{\mathcal{B}}^{-1}(v)\in K^n$ der \begriff{Koordinatenvektor} zu $v$ bezüglich $\mathcal{B}$, und $x_1, \dotsc, x_n$ sind die \begriff{Koordinaten} von $v$ bezüglich $\mathcal{B}$.
\end{definition}

\begin{satz}
	Die Menge $\Hom_K(V,W)$ ist ein Untervektorraum von $\Abb(V,W)$.
\end{satz}

\begin{lemma}
	Sei $U$ ein weiterer $K$-Vektorraum. Sind $f, f_1, f_2\in \Hom_K(V,W)$ und $g, g_1, g_2\in \Hom_K(U,W)$, so ist $f\circ(g_1+g_2) = f\circ g_1 + f\circ g_2$ und $(f_1 + f_2)\circ g = f_1\circ g + f_2 \circ g$.
\end{lemma}

\begin{corollar}
	Mit der Komposition wird $\End_K(V)$ zu einem Ring mit Einselement $\id_V$, und $\End_K(V)^\times = \aut_K(V)$.
\end{corollar}
\addtocounter{theorem}{1}
\begin{lemma}
	Seien $m,n,r\in \mathbb{N}$ und $A\in\Mat_{m\times n}(K), B\in\Mat_{n\times r}(K)$. Für die linearen Abbildungen $f_A \in\Hom_K(K^n, K^m), f_B\in\Hom_K(K^n, K^r), f_{AB}\in\Mat_K(K^r, K^m)$ gilt dann $f_{AB} = f_A\circ f_B$.
\end{lemma}

\begin{satz}
	Die Abbildung $A\mapsto f_A$ liefert einen Isomorphismus von $K$-Vektorräumen $F_{m\times n}: \Mat_{m\times n}(K)\overset{\cong}{\longrightarrow}\Hom_K(K^n, K^m)$ sowie einen Ringisomorphismus $F_{n\times n}: \Mat_n(K) \overset{\cong}{\longrightarrow}\End_K(K^n)$, der $\text{GL}_n$ auf $\aut_K(K^n)$ abbildet.
\end{satz}

\section{Koordinationdarstellung lineare Abbildungen}

Seinen $V$ und $W$ zwei endlichdimensionale $K$-Vektorräume mit Basen $\mathcal{B} = (x_1,\dots,x_n) \text{ und } \mathcal{C} = (y_1,\dots,y_m)$

\begin{definition}[Darstellende Matrix]
    Sei $f \in \Hom_K(V,W)$. Für $j=1,\dots,n$ schreiben wir
    \[
    f(x_j) = \sum_{i=1}^{m}a_{ij}y_i
    \]
    mit eindeutig bestimmten $a_{ij} \in K$. Die Matrix
    \[
    M_{\mathcal{C}}^{\mathcal{B}}(f) = (a_{ij})_{i,j} \in \Mat_{m\times n}(K)
    \]
    heißt die \begriff{darstellende Matrix} von $f$ bezüglich der Basen $\mathcal{B} \text{ und } \mathcal{C}$.
\end{definition}

\begin{satz}
    Sei $f \in \Hom_K(V,W)$. Die darstellende Matrix $M_{\mathcal{C}}^{\mathcal{B}}(f)$ ist die eindeutige Matrix $A \in \Mat_{m\times n}(K)$, für das Diagramm
\end{satz}

\[
    \begin{tikzcd}
        K^n \arrow{r}{f_A} \arrow[swap]{d}{\Phi_\mathcal{B}} & K^m \arrow{d}{\Phi_\mathcal{C}} \\%
        V \arrow{r}{f}& W
    \end{tikzcd}
\]
kommutiert, d.h. für die $\Phi_\mathcal{C} \circ f_A = f \circ \Phi_\mathcal{B}$ gilt.

\begin{corollar}
	Die Abbildung
    \[
    M_{\mathcal{C}}^{\mathcal{B}}\colon \Hom_K(V,W) \longrightarrow \Mat_{m\times n}(K)
    \]
    ist ein Isomorphismus von $K$-Vektorräumen.
\end{corollar}

\begin{lemma}
    Sei $U$ ein weiterer endlichdimensionaler $K$-Vektorraum mit Basis $\mathcal{A}$. Sind $f \in \Hom_K(V,W) \text{ und } g \in \Hom_K(U,V)$, so ist
    \[
    M_{\mathcal{C}}^{\mathcal{B}}(f)\cdot M_{\mathcal{B}}^{\mathcal{A}}(g)= M_{\mathcal{C}}^{\mathcal{BA}}(f\circ g).
    \]
\end{lemma}

\begin{corollar}
    Sei $f \in \Hom_K(V,W)$. Genau dann ist $f$ ein Isomorphismus, wenn $m=n$ und $M_{\mathcal{C}}^{\mathcal{B}}(f) \in \GL_n(K)$. In diesem Fall ist $M_{\mathcal{C}}^{\mathcal{B}}(f))^{-1} = M_{\mathcal{B}}^{\mathcal{C}}(f^{-1})$.
\end{corollar}

\begin{corollar}
    Die Abbildung
    \[
    M_{\mathcal{B}}\colon = M_{\mathcal{B}}^{\mathcal{B}} \colon \End_K(V) \longrightarrow \Mat_n(K)
    \]
    ist ein Ringhomomorphismus, der $\aut_K(V)$ auf $\GL_n(K)$ abbildet.
\end{corollar}

\begin{definition}[Transformationsmatrix]
    Sind $\mathcal{B} \text{ und } \mathcal{B}^{\prime}$ Basen von $V$, so nennt man
    \[
    T_{\mathcal{B}^{\prime}}^{\mathcal{B}}\colon= M_{\mathcal{B}^{\prime}}^{\mathcal{B}}(\id_V) \in \GL_n(K)
    \]
    die \begriff{Transformationsmatrix} des Basiswechsels von $\mathcal{B} \text{ nach } \mathcal{B}^{\prime}$.
\end{definition}

\addtocounter{theorem}{1}

\begin{satz}[Transformationsformel]
    Seien $\mathcal{B} \text{ und } \mathcal{B}^{\prime}$ Basen von $V$ sowie $\mathcal{C} \text{ und } \mathcal{C}^{\prime}$ Basen von $W$, und sei $f \in \Hom_K(V,W)$. Dann ist
    \[
    M_{\mathcal{C}^{\prime}}^{\mathcal{B}^{\prime}}(f) = T_{\mathcal{C}^{\prime}}^{\mathcal{C}} \cdot M_{\mathcal{C}}^{\mathcal{B}}(f)\cdot\big( T_{\mathcal{B}^{\prime}}^{\mathcal{B}} \big)^{-1}.
    \]
\end{satz}

\begin{corollar}
    Sind $\mathcal{B} \text{ und } \mathcal{B}^{\prime}$ Basen von $V$ und $f \in \End_K(V)$, so gilt
    \[
    M_{\mathcal{B}^{\prime}} = T_{\mathcal{B}^{\prime}}^{\mathcal{B}}\cdot M_{\mathcal{B}}^{\mathcal{B}}(f)\cdot\big( T_{\mathcal{B}^{\prime}}^{\mathcal{B}} \big)^{-1}.
    \] 
\end{corollar}

\section{Quotienträume}

Seien $V \text{ und } W$ zwei $K$-Vektorräume und $U$ ein Untervektorraum von $V$.

\begin{definition}
    Ein \begriff{affiner Unterraum} von $V$ ist eine Teilmenge der Form
    \[
    x + U :=\{ x+u\colon u \in U\} \subseteq V,
    \]
    wobei $U$ ein beliebiger Untervektorraum von $V$ ist und $x \in V$.
\end{definition}

\begin{lemma}
    für $x,x^{\prime} \in V$ sind äquivalent:
    \begin{enumerate}[(1)]
    \item $x+U = x^{\prime} + U$
    \item $x^{\prime} \in x+U$
    \item $x^{\prime} -x \in U$
    \end{enumerate}
\end{lemma}

\begin{lemma}
    Sei $f \in \Hom_K(V,W)$ und $U \Ker(f)$. Für $y \in f(V)$ ist die \begriff{Faser} $f^{-1}(y)\colon=f^{-1}(\{y\}) \text{ von }f$ der affine Unterraum $x+U$ für ein beliebiges $x\inf^{-1}(y)$.
\end{lemma}

\begin{example}
    Sind $V=\mathbb{R}^2 \text{ und } W = \mathbb{R}\text{ und }f(x_1,x_2)=2x_1-x_2$, so sind die Fasern von $f$ genau die Geraden $L\subseteq \mathbb{R}^2$ der Steigung 2.
\end{example}

\begin{lemma}
    Seien $x_1,x_1^{\prime},x_2,x_2^{\prime}\in V \text{ und } \lambda \in K$. Ist $x_1+U=x_1^{\prime}+U$ und $x_2+U=x_2^{\prime}+U$,
    so ist $(x_1+x_2)+U=(x_1^{\prime}+x_2^{\prime})+U \text{ und }\lambda x_1 + U = \lambda x_1^{\prime}+U$.
\end{lemma}

\begin{definition}[Quotientvektorraum]
    Der \begriff{Quotientenvrktorraum} von $V$ modulo $U$ ist Menge der affinen Unterräume:
    \begin{enumerate}[1)]
    \item $V\slash U :=\{x+U\colon x \in V\}$
    \item zusammen mit Addition: $(x_1 +U) + (x_2+U):=(x_1+x_2)+U$
    \item und der Skalarmultiplikation $\lambda\cdot(x+U):=\lambda x + U$
    \end{enumerate}
    Definiere Abbildung $\pi_U: V \to V\slash U$ durch $\pi_U(x) = x + U$.
\end{definition}

\begin{satz}
    Der Quotientenraum $V\slash U$ ist ein $K$-Vektorraum und $\pi_U$ ist ein Epimorphismus mit Kern $U$.
\end{satz}

\begin{theorem}[\index{Homomorphiesatz}Homomorphiesatz]
    Sei $f \in \Hom_K(V,W) \text{ mit }U\subseteq \Ker(f)$. Dann gibt es genau eine lineare Abbildung $\bar{f}:V\slash U \to W \text{ mit } f = \bar{f} \circ \pi_U$. 
    \begin{center}
    	\begin{tikzpicture}
       		\node (V) at (0,0) {$V$};
        	\node (W) at (3,0) {$W$};
        	\node (R) at (1.5,-1.5) {$V\slash U$};
        	\draw[->, above] (V) to node {$f$} (W);
        	\draw[->, below] (V)  to node {$\pi_U$} (R);
        	\draw[->, right, dashed] (W)  to node {$\tilde f$} (R);
    	\end{tikzpicture}
    \end{center}
    Diese erfüllt $\Ker(\bar{f})=\Ker(f)\slash U = \{x+U\colon x\in \Ker(f)\}$.
\end{theorem}

\begin{corollar}
    Für $f \in \Hom_K(V,W)$ ist $\im(f) \cong V\slash \Ker(f)$. Insbesondere gilt: Ist $f$ ein Epimorphismus, so ist $W\cong V\slash \Ker(f)$.
\end{corollar}

\begin{satz}
    Seien $U \text{ und } U^{\prime}$ Unterräume von $V$. Genau dann ist $V = U\oplus U^{\prime}$, wenn $\pi_{U\mid U^{\prime}}:U^{\prime}:\to V\slash U$ ein Isomorphismus ist.
\end{satz}

\begin{corollar}
    Ist $V$ endlichdimensional, so ist $\dim_K(V\slash U) = \dim_K(V) - \dim_K(U)$.
\end{corollar}

\begin{corollar}
    Ist $V$ endlichdimensional und $f \in \Hom_K(V,W)$, so ist $\dim_K(V)=\dim_K(\Ker(f))+\dim_K(\im(f))$.
\end{corollar}

\begin{corollar}
    Ist $V$ endlichdimensional und $f \in \End_K(V)$, so sind äquivalent:
    \begin{enumerate}[(1)]
    \item $f\in \aut_K(V)$
    \item $f$ ist injektiv
    \item $f$ ist surjektiv
    \end{enumerate}
\end{corollar}

\section{Rang}

$V \text{ und }W$ endlichdimensional $K$-Vektorräume, $F\in \Hom_K(V,W)$.

\begin{definition}
    Der \begriff{Rang einer Abbildung} $f$ ist $\rk(f):=\dim_K(\im(f))$.
\end{definition}
\addtocounter{theorem}{1}
\begin{lemma}
% wo kommt f her? f \in \Hom_K(U,V)?
    Sei $U$ ein weiterer $K$-Vektorraum und $g\in\Hom_K(U,V)$.
    \begin{enumerate}[(a)]
    \item Ist $g$ surjektiv, so ist $\rk(f\circ g) = \rk(f)$
    \item Ist $f$ surjektiv, so ist $\rk(f\circ g) = \rk(g)$
    \end{enumerate}
\end{lemma}

\begin{satz}
    Sei $r\in \mathbb{N}_0$. Genau dann ist $\rk(f) = r$, wenn es Basen $\mathcal{B}\text{ von } V \text{ und } \mathcal{C} \text{ von } W$ gibt, für die $M_{\mathcal{B}}^{\mathcal{C}}=E_r:=\sum_{i=1}^{r}E_{ii}$.
\end{satz}

\begin{definition}
    Der \begriff{Rang einer Matrix} $A\in \Mat_{m\times n} \text{ ist } \rk(A):= \rk(f_A)$, wobei $f_A: K^n \to K^m$ die durch $A$ beschriebene lineare Abbildung ist.
\end{definition}

\begin{remark}
    Sei $A=(a_{ij})_{i,j} \in \Mat_{m\times n}(K)$.
    \begin{itemize}
    \item fasst Spalten $a_j = (a_{1j},\dots,a_{mj})^t$ als Elemente des $K^m$ auf und definiert den \begriff{Spaltenraum} $\SR(A) = \spank(a_1,\dots, a_n) \subseteq K^m.$
    \item entsprechend definieren die Zeilen $\tilde{a_i} = (a_{i1},\dots, a_{in})$ und definiert den \begriff{Zeilenraum} $\ZR(A) = \spank(a_1,\dots,a_m^t) \subseteq K^n$
    \end{itemize}
    Dann gelten noch: 
    \begin{itemize}
    \item $\im(f_A) = \SR(A)$ und damit $\rk(A) = \dim_K(\SR(A))$.
    \item $\SR(A^t) = \ZR(A)$, deshalb $\rk(A^t) = \dim_K(\ZR(A))$
    \end{itemize} 
\end{remark}

\begin{lemma}
    Ist $A \in \Mat_{m\times n}(K), S\in \GL_m(K)$ und $T \in \GL_n(K)$ mit $SAT = E_r$, wobei $r = \rk(A)$.
\end{lemma}

\begin{satz}
    Für jedes $A\in\Mat_{m\times n}(K)$ gibt es $S\in \GL_m(K)$ und $T\in \GL_n(K)$ mit $SAT=E_r$, wobei $r=\rk(A)$.
\end{satz}

\begin{corollar}
    Seien $A,B \in \Mat_{m\times n}(K)$. Genau dann gibt es $S\in \GL_m(K)$ und $T\in\GL_n$ mit $B=SAT$, wenn $\rk(A)=\rk(B)$.
\end{corollar}

\begin{satz}
    Für $A \in \Mat_{m\times n}(K)$ ist $\rk(A)=\rk(A^t)$.
\end{satz}

\begin{corollar}
    Für $A\in \Mat_n(K)$ sind äquivalent:
    \begin{enumerate}[(1)]
    \item $A\in\GL_n(K)$, d.h. $A$ sind linear unabhängig
    \item $\rk(A)=n$
    \item Die Zeilen von $A$ sind linear unabhängig.
    \item Die Spalten von $A$ sind linear unabhängig.
    \item Es gibt $S\in\GL_n(K)$ mit $SA=\mathbbm{1}_n$
    \item Es gibt $T\in\GL_n(K)$ mit $AT=\mathbbm{1}_n$
    \end{enumerate}
\end{corollar}

\section{Lineare Gleichungssysteme}
Sei $A \in \Mat_{m\times n}(K)$ und $b\in K^m$.

\begin{definition}
    Unter einem \begriff{lineare Gleichungssystem} verstehen wir eine Gleichung der Form \[Ax = b.\]
    Dieses heißt \begriff{homogen}, wenn $b=0$, sonst \begriff{inhomogen}, und \[L(A,b) = \{ x \in K^n \colon Ax=b \}\] ist sein \begriff{Lösungsraum}.
\end{definition}

\addtocounter{theorem}{1}

\begin{remark}
    \begin{itemize}
    \item homogene System $Ax=0$ hat als Lösungsraum den \emph{Untervektorraum} $L(A,0) = \Ker(f_A)$ der Dimension\\ $\dim_K(L(A,0)) = n - rk(A)$.
    \item inhomogene System $Ax=b$ hat entweder $L(A,b) = \emptyset$, oder \emph{affine Unterraum}\\ $L(A,b) = f_A^{-1}(b) = x_0 + L(A,0)$, $x_0 \in L(A,b)$ bel.
    \item erhält alle Lösungen des inhomogenen Systems, wenn eine Lösung des inhomogenen Systems und alle Lösungen des homogenen Systems
    \item Im Klartext! Wie sieht der Lösungsraum aus?\\
    Die Anzahl der Lösungen lässt sich dann an den $b_i$ ablesen.
        \begin{itemize}
            \item Ist mindestens eines der $b_{k+1},\dotsc ,b_{m}$ ungleich null, so gibt es keine Lösung.
            \item Sind alle $b_{k+1},\dotsc ,b_{m}$ gleich null (oder $k = m$) so gilt:
            \begin{itemize}
                \item Ist $k=n$ , so ist das Gleichungssystem eindeutig lösbar.
                \item Ist $k<n$ , gibt es unendlich viele Lösungen. Der Lösungsraum hat die Dimension $n-k$.
            \end{itemize}
        \end{itemize}
    \end{itemize}
\end{remark}

\begin{definition}\label{Zeilenstufenform}
    Die Matrix $A = (a_{ij})_{i,j}$ hat \begriff{Zeilenstufenform}, wenn es $0\leq r \leq m$ und $1 \leq k_1 < k_2 < \dots < k_r \leq n$ gibt mit:
        \begin{enumerate}[(i)]
        \item für $1 \leq i \leq r$ und $1 \leq j < k_i$ ist $a_{ij} = 0$
        \item für $1 \leq i \leq r$ ist $a_{i k_i} \neq 0$ (sogenanntes \begriff{Pivotelement})
        \item für $1 < i \leq m$ und $1 \leq j < n$ ist $a_{ij} = 0$
        \end{enumerate}
    %TODO mark the last coefficients.
    \[
    \begin{pmatrix}
					0 & ... & 0 & a_{1k_{1}} & * & ... & ... & *\\
					0 & ... & ... & 0 & a_{2k_{2}} & * & ... & *\\
					\vdots & \vdots & \vdots & \vdots & \vdots & \vdots & \vdots & \vdots\\
					0 & ... & ... & ... & ... & ... & ... & a_{rk_{r}}\\
					0 & ... & ... & ... & ... & ... & ... & 0\\
					\vdots & \; & \; & \; & \; & \; & \; & \vdots\\
					0 & ... & ... & ... & ... & ... & ... & 0\\
	\end{pmatrix}
    \]
\end{definition}

\begin{lemma}
    Sei $A$ in Zeilenstufenform wie in \ref{Zeilenstufenform}. Dann ist $f=rk(A)$.
\end{lemma}

\begin{satz}
    Sei $A$ in Zeilenstufenform wie in \ref{Zeilenstufenform}.
    \begin{enumerate}[(a)]
    \item Ist $b_i \neq 0$ für ein $r < i \leq m$, so ist $L(A,b) = \emptyset$.
    \item Ist $b_i = 0$ für alle $r < i \leq m$, so erhält man alle $x\in L(A,b)$, indem erst $x_j \in K$ für $j\in \{1,\dots,n\}\setminus \{k_1,\dots,k_r\}$ beliebig wählt unf für $i= r,\dots,1$ rekursiv
    \[
    x_{k_i}=a_{ik_i}^{-1}\cdot\Bigg(b_i - \sum_{j=k_i+1}^{n}a_{ij}x_j\Bigg)
    \]
    setzt.
    \end{enumerate}
\end{satz}

\begin{definition}[\index{Elementarmatrizen}Elementarmatrizen]
    Für $i,j \in \{1,\dots,m\}$ mit $i\neq j, \lambda\in K^{\times}$ und $\mu \in K$ definiere $m\times m$-Matrizen
    \begin{eqnarray*}
        S_i(\lambda) &:=& \mathbbm{1}_m +(\lambda -1)\cdot E_{ii} = \diag(1,\dots,1,\lambda,1,\dots,1),\\
            Q_{i,j}(\mu) &:=& \mathbbm{1}_m + \mu E_{ii},\\
            P_{i,j} &:=& \mathbbm{1}_m + E_{ij} + E_{ji} - E_{ii} - E_{jj}.
    \end{eqnarray*}
\end{definition}

\begin{lemma}
    Es sind $S_i(\lambda), Q_{ij}(\mu), P_{i,j}\in \GL_m(K)$: Es ist $S_i(\lambda^{-1})=S_{1}(\lambda^{-1}), Q_{i,j}(\mu)^{-1} = Q_{i,j}(-\mu),P_{i,j}^{-1}=p_{i,j}$. Insbesondere gilt: Ist $E$ eine der Elementarmatrizen $S_i(\lambda), Q_{i,j}(\lambda),P_{i,j}$, so ist $\ZR(A)$ und $L(EA,0) = L(A,0)$, insbesondere $\rk(EA)=\rk(A)$.
\end{lemma}

\begin{theorem}[\index{Eliminationsverfahren von Gauß}Eliminationsverfahren von Gauß]
    Zu jeder Matrix $A\in \Mat_{m\times n}(K)$ gibt es $l \in \mathbb{N}_{0}$ und Elementarmatrizen $E_1,\dots,E_l$ vom Typ II, III, für die $E_l\cdots E_1 A$ in Zeilenstufenform.
\end{theorem}

\addtocounter{theorem}{1}

\begin{corollar}
    Zu jeder Matrix $A\in \Mat_{m\times n}(K)$ gibt es eine invertierbare Matrix $S\in \GL_m(K)$, für die $SA$ in Zeilenstufenform ist.
\end{corollar}

\addtocounter{theorem}{1}

\begin{corollar}
    Jedes $A\in \GL_n(K)$ ist ein Produkt von Elementarmatrizen.
\end{corollar}

\chapter{Determinanten}
Sei $n \in \mathbb{N}$.

\section{Das Vorzeichen einer Permutation}

\begin{example}
    Für $i,j \in \{1,\dots,n\}$ mit $i\neq j$ bezeichne $\tau_{ij} \in S_n$ die \begriff{Transposition}
    $ \tau_{ij}(k)=
    \begin{cases}
        j & \text{$k=i$}\\
        i & \text{$k=j$}\\
        k & \text{ sonst}
    \end{cases}$\\
    Offenbar gilt $\tau_{ij}^2 = \tau_{ij}^{-1} = \tau_{ij} = \tau_{ji}$.
\end{example}

\begin{satz}
    Für jedes $\sigma \in S_n$ gibt es $r\in \mathbb{N}_0$ und Transposition $\tau_1,\dots,\tau_r \in S_n$ mit $\sigma = \tau_1\cdots\tau_r$.
\end{satz}

\begin{definition}
    Sei $\sigma \in S_n$, dann
        \begin{enumerate}[(1)]
        \item Ein \begriff{Fehlstand} von $\sigma$ ist ein Paar $(i,j)$ mit $1\leq i \leq j \leq n$ und $\sigma(i) > \sigma(j)$
        \item Das \begriff{Vorzeichen} (oder \begriff{Signum}) von $\sigma$ ist $\sgn(\sigma)=(-1)^{f(\sigma)}\in \{+1,-1\}=\mu_2$, wobei $f(\sigma)$ die Anzahl der Fehlstände von $\sigma$ ist
        \item Man nennt $\sigma$ \emph{gerade}, wenn $\sgn(\sigma)=+1$, sonst \emph{ungerade}
        \end{enumerate}
\end{definition}

\begin{example}
    \begin{enumerate}[(a)]
    \item Genau dann hat $\sigma$ \emph{keine} Fehlstände, wenn $\sigma = \id$ und insbesondere gilt $\sgn(\id) = +1$.
    \item die Permutation $\sigma =
    \begin{pmatrix}
    1 & 2 & 3\\
    2 & 3 & 1
    \end{pmatrix} \in S_n$ hat Fehlstände $(1,3) \text{ und } (2,3)$, sonst $\sgn(\sigma) = (-1)^2 = 1$
    \item Die Transposition $\tau_{1,3}=$
    $\begin{pmatrix}
    1 & 2 & 3\\
    3 & 2 & 1
    \end{pmatrix}$, hat Fehlstände $(1,2), (2,3) \text{ und } (1,3)$, somit $\sgn(\tau_{1,3}) = (-1)^3 = -1$
    \item Eine Transposition $\tau_{ij}\in S_n$ ist \emph{ungerade}. Ist $i<j$, so sind die Fehlstände $(i,i+1),\dots,(i,j)$ und $(i+1,j),\dots,(j-1,j)$ also $j-(i+1)+i+(j-1)+1-(i+1) = 2(j-i)-1$ viele
    \end{enumerate}
\end{example}

\begin{lemma}
    Für $\sigma \in S_n$ ist $\sgn(\sigma)=\prod_{1\leq i < j \leq n} \frac{\sigma(j)-\sigma(i)}{j-i}\in \mathbb{Q}$
\end{lemma}

\begin{satz}
    Die Abbildung $\sgn:S_n \to \mathbb{Z}^{\times} = \mu_2$ ist ein Gruppenhomomosphismus.
\end{satz}

\begin{corollar}
    Für $\sigma \in S_n$ ist $\sgn(\sigma^{-1}) = \sgn(\sigma)$.
\end{corollar}

\begin{corollar}
    Sei $\sigma \in S_n$. Sind $\tau_1,\dots,\tau_r$ Transpositionen mit $\sigma = \tau_1\cdots\tau_r$, so ist $\sgn(\sigma) =(-1)^r$
\end{corollar}

\begin{corollar}
    Die geraden Permutationen
    \[
    A_n:= \{\sigma \in S_n\colon \sgn(\sigma)= +1\}
    \]
    bilden einen Normalteiler der $S_n$, genannt die \begriff{alternierende Gruppe} $A_n$. Ist $\tau \in S_n$ mit $\sgn(\tau) = -1$, so gilt für $A_n\tau := \{\sigma \tau \colon \sigma \in A_n\}$:
    \begin{itemize}
    \item $A_n \cup A_n\tau = S_n$ und
    \item $A_n \cap A_n\tau = \emptyset$
    \end{itemize}
\end{corollar}


\section{Determinanten}

Sei $n \in \mathbb{N}$.

\begin{remark}
    Wir werden nun auch Matrizen mit Koeffzienten im Ring $R$ anstatt $K$ betrachten. Mit der gewohnten Additon und Multiplikation bilden die $n\times n$-Matrizen einen Ring $\Mat_n(R)$ und wir definieren wieder $\GL_n(R) = \Mat_n(R)^{\times}$.
\end{remark}

\begin{remark}
    \begin{itemize}
        \item $(a_1,\dots,a_n)\in R^m$ Spaltenvektoren, so bezeichnen wir mit $A=(a_1,\dots,a_n)\in\Mat_{m\times n}(R)$ die Matrix mit Spalten $(a_1,\dots,a_n)$
        \item $(\tilde{a}_1,\dots,\tilde{a}_m)\in R^n$ Spaltenvektoren, so bezeichnen wir mit $\tilde{A}=(\tilde{a}_1,\dots,\tilde{a}_m)\in\Mat_{m\times n}(R)$ die Matrix mit Zeilen $(\tilde{a}_1,\dots,\tilde{a}_m)$
    \end{itemize}
\end{remark}

\begin{remark}\label{geo_int_det}
    Wir hatten in III.\ref{Normalteiler} definiert:
    \[
    \det A = ad - bc, A = 
    \begin{pmatrix}
    a & b\\
    c & d
    \end{pmatrix} \in \Mat_2(K)
    \] und festgestellt: $\det A\neq 0 \Leftrightarrow A \in \GL_2(K)$.\\
    Interpretation in $\mathbb{R}^2$ (Determinante von $A$, ist die Fläche, welche aufgespannt wird von $x_1 = (a,b) \text{ und } x_2 = (c,d)$, siehe Bild)
\end{remark}

\begin{*remark}
 \begin{enumerate}[(i)]
 \item Für $\lambda \in R$ ist 
 $\det(\lambda x_1, x_2) = \det(x_1,\lambda x_2) = \lambda \det(x_1,x_2)$\\
 und für $x_i = \tilde{x}_1 + \tilde{x}_2$ ist
    \begin{enumerate}[a)]
        \item $\det (x_1,x_2) = \det(\tilde{x}_1,x_2) + \det(\tilde{x}_2,x_2 )$
        \item $\det (x_1,x_2) = \det(x_1,\tilde{x}_1) + \det(\tilde{x}_1, x_2)$.
    \end{enumerate}
 \item Ist $x_1 = x_2$, so ist $\det A = 0$.
 \item $\det \mathbbm{1}_2 = 1$.
 \end{enumerate}
\end{*remark}

Sei $R$ kommutativer Ring mit Einselement, $K$ Körper und $n \in \mathbb{N}$.

\begin{definition}
    Eine Abbildung $\delta: \Mat_n(R) \to R$ heißt \begriff[Determinate]{Determinantenabbildung}, wenn gilt:
        \begin{enumerate}[(D1)]
        \item $\delta$ ist \highlight{linear in jeder Zeile}:\\
        Sind $a_1,\dots,a_n$ die Zeilen von $A\in\Mat_n(R)$ und ist $i\in\{1,\dots,n\}$ und $a_i = \lambda^{'} a^{'} + \lambda^{''}a^{''} \text{ mit } \lambda^{'},\lambda^{''} \in R$ und Zeilenvektoren $a_i^{'},a_i^{''}$, so ist 
        \[
        \delta(A)=\lambda^{'}(a_1^{'},\dots,a_i^{'},\dots,a_n^{'})+\lambda^{''}(a_1^{''},\dots,a_i^{''},\dots,a_n^{''})
        \]
        \item $\delta$ ist \begriff{alternierend}. Sind $a_1,\dots,a_n$ die Zeilen von $A\in\Mat_{R}$ und $i,j\in\{1,\dots,n\},i\neq j$, mit $a_i=a_j$, so ist $\delta(A)=0$
        \item $\delta$ ist \begriff{normiert} $\delta(\mathbbm{1}_n)$
        \end{enumerate}
\end{definition}

\begin{example}
    Sei $\delta: \Mat_n(K) \to K$ eine Determinantenabbildung. Ist $A\in \Mat_n(K)$ \highlight{nicht} invertierbar so ist die Zeile $a_1,\dots,a_n$ von $A$ linear abhängig, es gibt also $i$ mit $a_i=\sum_{j=1}\lambda_j a_j \text{ mit }(\lambda_i \in K)$. Es folgt
    \begin{align*}
        \delta(A) = \delta(a_1,\dots,a_n) &\overset{\text{(D1)}}{=} \sum_{j=1}\lambda_j \delta(a_1,\dots,a_j,\dots,a_n)\\
        &\overset{\text{(D2)}}{=} \sum_{j=1}\lambda_j \cdot 0 = 0
    \end{align*}
\end{example}

\begin{lemma}
    Erfüllt die Abbildung $\delta: \Mat_n(R) \to R$ die Axiome (D1) und somit für jedes $\sigma \in S_n$ und Zeilenvektoren $a_1,\dots,a_n$:
    \[ 
    \delta(a_{\sigma(1)},\dots,a_{\sigma(n)}) = \sgn(\sigma)\cdot\delta(a_1,\dots,a_n).
    \]
\end{lemma}

\begin{lemma}
    Erfüllt die Abbildung $\delta: \Mat_n(R) \to R$ die Axiome (D1) und (D2), so gilt für $A=(a_{{ij}_{i,j}})\in \Mat_n(R)$
    \[
    \delta(A) = \delta(\mathbbm{1}_n)\cdot\sum_{\sigma \in S_n}\sgn(\sigma)\prod_{i=1}^{n}a_{i,\sigma(i)}.
    \]
\end{lemma}

\begin{theorem}
    Es gibt genau \highlight{eine} Determinantenabbildung
    \[
    \det \Mat_n(R) \to R
    \]
    und diese ist gegeben durch die \begriff{\person{Leibniz}-Formel}.
\end{theorem}

\begin{example}
    \begin{enumerate}[(a)]
    \item $n=2$, damit $S_2=\{\id, \tau_{12}\}$
    \[
    \det 
    \begin{pmatrix}
        a_{11} & a_{12} \\
        a_{21} & a_{22}
    \end{pmatrix} = \sum_{\sigma\in S_2}a_{1\sigma(1)} a_{2\sigma(2)} = a_{11} a_{22}-a_{12}a_{21}
    \]
    \item $n=3$, damit $s_3=\{\id, \tau_{12},\tau_{13},\tau_{23}, \sigma_1,\sigma_2\}$ mit 
    $\sigma_1 = 
    \begin{pmatrix}
        1 & 2 & 3\\
        2 & 3 & 1
    \end{pmatrix}$,
    $\sigma_2 = 
    \begin{pmatrix}
        1 & 2 & 3\\
        3 & 2 & 1
    \end{pmatrix}$\\
    $A_3 = \{\id, \sigma_1, \sigma_2\}$ und $S_3\setminus A_3 = \{ \tau_{12},\tau_{13},\tau_{23} \}$
    \begin{align*}
        \det
        \begin{pmatrix}
                a_{11} & a_{12} & a_{13}\\
                a_{21} & a_{22} & a_{23}\\
                a_{31} & a_{32} & a_{33}
        \end{pmatrix}&= \sum_{\sigma\in S_3} a_{1\sigma(1)} a_{2\sigma(2)} a_{2\sigma(3)} a_{2\sigma(2)}\\
        &= a_{11} a_{22}a_{33}+a_{12}a_{23}a_{31}+a_{13}a_{21}a_{32} -a_{31}a_{22}a_{13}-a_{32}a_{23}a_{11}-a_{33}a_{21}a_{12}
    \end{align*}
    die sogenannte \begriff{Regel von Sarrus}.
    \item Ist $A=(a_{ij})_{i,j}$ eine \begriff{obere Dreiecksmatrix} (siehe A108), also $a_{ij} = 0$ für $i=j$, so ist
    \[
    \det A = \det
    \begin{pmatrix}
        a_{ii} & \cdots \\
        0 & \ddots a_{nn}\\
    \end{pmatrix} = \prod_{i=1}^{n}a_{ii}
    \]
    \item Für $i\neq j,\lambda\in K^{\times},\mu \in K$ ist $\det(S_i(\lambda)) = \lambda,\det(Q_{ij}(\mu))
    = 1,\det(P_{ij})=-1$\\
    %(gibt nur eine Permutation $\sigma_{ij} = -1 \text{ und } \sgn(\sigma_{ij}) = -1$) %TODO verify!
    \item Ist $A$ \text{\begriff[!Matrix]{Blockmatrix}} der Gestalt
    \[
    A = \begin{bmatrix}
    A_1 & C \\
    0 & A_2
    \end{bmatrix} \text{ mit } A_1, A_2, C \in \Mat_n(R)
    \]
    So ist $\det(A) = \det(A_1)\cdot\det(A_2) + 0$.
    \end{enumerate}
\end{example}

\begin{corollar}
    Für $A\in \Mat_n(R)$ ist $\det(A) = \det(A^t)$. Insbesondere erfüllt $\det$ die Axiome (D1) und (D2) auch für Spalten statt Zeilen.
\end{corollar}

\begin{theorem}[\index{Determinantenmultiplikationssatz!Determinate}Determinantenmultiplikationssatz]
    Für $A,B\in \Mat_n(R)$ ist $\det(AB) = \det(A)\cdot\det(B)$.
\end{theorem}

\begin{corollar}
    Die Abbildung $\det : \Mat_n(R) \to R$ schränkt sich zu einem Gruppenhomomorphismus $\GL_n(R) \to R^{\times}$. Ist $R = K$ ein Körper, so ist $A\in \Mat_n(K)$ also genau dann invertierbar, wenn $\det(A) \neq 0$, und in diesem Fall ist $\det(A^{^-1}) = (\det(A))^{-1}$.
\end{corollar}

\begin{corollar}
    Die Matrizen von Determinanten $1$ bilden einen Normalteiler $\SL_n(K) = \{ A \in \GL_n(K)\colon \det(A) \}$ der allgemeinen linearen Gruppe, die sog. \begriff[!Matrix]{spezielle lineare Gruppe}.
\end{corollar}

\begin{corollar}
    Elementare Zeilenumformungen von \highlight{Typ II} ändern die Determinante der Matrix $A$ nicht. Elementare Zeilenumformungen von \highlight{Typ III} ändern nur das Vorzeichen.
\end{corollar}

\section{Minoren}
Sei $m,n \in \mathbb{N}$.

\begin{definition}
    Sei $A = (a_{ij})_{i,j} \in \Mat_n(R)$. Für $i,j \in \{ 1,\dots,n \}$ definiere die $n\times n$-Matrix
    \[
    A_{ij} = 
    \begin{pmatrix}
    a_{11} & \dots & a_{1j-1} & 0 & a_{1j+1} & \dots & a_{1n}\\
    \vdots & & & \vdots & & & \vdots\\
    a_{i-11} & \dots & a_{i-1j-1} & 0 & a_{i-1j+1} & \dots & a_{i-1n}\\
    0 & \dots & 0 & 1 & a_{i-1j+1} & \dots & 0\\
    a_{i+11} & \dots & a_{i+1j-1} & 0 & a_{i+1j+1} & \dots & a_{i+1n}\\
    \vdots & & & \vdots & & & \vdots\\
    a_{n1} & \dots & a_{nj-1} & 0 & a_{nj+1} & \dots & a_{mn}
    \end{pmatrix}
    \]
    die durch Ersetzen der $i$-ten Zeile durch $e_j$ und $j$-ten Spalte durch $e_i$ aus $A$ hervorgeht,
    \[
    A_{ij} = 
    \begin{pmatrix}
    a_{11} & \dots & a_{1j-1}  & a_{1j+1} & \dots & a_{1n}\\
    \vdots & &\vdots & \vdots & & \vdots\\
    a_{i-11} & \dots & a_{i-1j-1}  & a_{i-1j+1} & \dots & a_{i-1n}\\
    a_{i+11} & \dots & a_{i+1j-1} & a_{i+1j+1} & \dots & a_{i+1n}\\
    \vdots & &\vdots &\vdots & & \vdots\\
    a_{n1} & \dots & a_{nj-1} & a_{nj+1} & \dots & a_{mn}
    \end{pmatrix}
    \]
    die durch Streichen der $i$-ten Zeile und der $j$-ten Spalte entsteht. Weiter definiere die zu $A$ \begriff{adjungierte Matrix} als
    $A^{\#} = (a_{ij}^{\#})_{i,j} \Mat_n(R)$, wobei $a_{ij}^{\#} = \det(A_{ij})$.
\end{definition}

\begin{lemma}
    Sei $A \in \Mat_n(R)$ mit Spalten $a_1,\dots,a_n$. Für $i,j \in \{1,\dots,n\}$ gilt:
    \begin{enumerate}[(a)]
    \item $\det(A_{ij})=(-1)^{î+j}\det(A_{ij}^{'})$
    \item $\det(A_{ij}) = \det(a_1,\dots, a_{j-1},e_i,a_{j+1},\dots,a_n)$
    \end{enumerate}
\end{lemma}

\begin{satz}
    Für $A\in \Mat_n(R)$ ist $A^{\#}A = A\cdot A^{\#} = \det(A)\mathbbm{1}_n$.
\end{satz}

\begin{corollar}
    Es ist $\GL_n(R) = \{ A \in \Mat_n(R)\colon \det(A) \in R^{\times} \}$ und für $\GL_n(R)$ ist
    \[
    A^{^-1} = \frac{1}{\det(A)}A^{\#}.
    \]
\end{corollar}

\begin{corollar}
    Sei $A=(a_{ij})_{i,j} \in \Mat_n(R)$. Für jedes $i\in\{1,\dots,n\}$ gilt dir Formel für die \highlight{Entwicklung nach der $i$-ten Zeile}
    \[
    \det(A) = \sum_{j=1}^{n}(-1)^{i+j}a_{ij}\det(A_{ij}^{'}),
    \]
    für jedes $j\in\{1,\dots,n\}$ gilt die Formel für die Entwicklung nach der $j$-ten Spalte
    \[
    \det(A) = \sum_{i=1}^{n}(-1)^{i+j}a_{ij}\det(A_{ij}^{'}).
    \]
\end{corollar}

\begin{corollar}[\index{Cramersche Regel}Cramersche Regel]
    Sei $A\in\GL_n(R)$ mit Spalten $a_1,\dots,a_n$ und sei $b \in R^n$. Weiter sei $x=(x_1,\dots,x_n)^t \in R^n$die (eindeutige) Lösung des Linearen Gleichungssystems $Ax=b$.
    Dann ist für $i\in\{1,\dots,n\}$:
    \[
    x_i = \frac{\det(a_1,\dots,a_{i-1},b,a_{i+1},\dots,a_n)}{\det(A)}.
    \]
\end{corollar}

\begin{definition}[\index{Minoren}Minoren]
    Sei $A(a_{ij})_{i,j}\in \Mat_n(R) \text{ und } 1 \leq r \leq m,1 \leq s \leq n$. Eine \highlight{$r \times s$- Teilmatrix} von $A$ ist eine Matrix der Form $(a_{i\mu,j\nu})_{\mu,\nu} \in \Mat_{r\times s}(R)$ mit $1\leq i_{1}, \dots, i_{r} \leq m \text{ und } 1\leq j_{1}, \dots, j_{r} \leq n$. Ist $A^{'}$ eine $r\times s$      -Teilmatrix $A$, so bezeichnet man $\det(A^{'})$ als einen \highlight{$r$-Minor} von $A$.
\end{definition}

\begin{example}
    Ist $A\in \Mat_n(R)$ und $i,j \in\{1,\dots,n\}$, so ist $A_{ij}^{'}$ eine Teilmatrix von $A$ und $\det(A_{i,j}^{'})=(-1)^{i+j}a_{ji}^{\#}$ ein $(n-1)$-Minor von $A$.
\end{example}

\begin{satz}
    Sei $A\in \Mat_{m\times n}(K) \text{ und } r \in \mathbb{N}$. Genau dann is $\rk(A) \geq r$, wenn es eine $r\times r$ Teilmatrix $A^{'}$ von $A$ mit $\det(A^{'}) \neq 0$ gibt. 
\end{satz}

\begin{corollar}
    Sei $A\in \Mat_{m\times n}(K)$. Der Rang von $A$ ist das größte $r \in \mathbb{N}$, für das $A$ eine von Null verschiedenen $r$-Minor hat.
\end{corollar}

\section{Determinanten und Spur von Endomorphismen}
Sei $n \in \mathbb{N}$ und $V$ ein $K$-Vektorraum mit $\dim_K(V) = n$.

\begin{satz}\label{Eindeutigkeit_Transmatrix}
    Sei $f \in \End(K)\text{, } \mathcal{A}$ eine Basis von $V$ und $A \in M_{\mathcal{A}}(f)$. 
    Sei weiter $B \in \Mat_n(K)$. Genau dann gibt es eine Basis $\mathcal{B}$ von $V$ mit $B=M_{\mathcal{B}}(f)$, wenn es $S\in \GL_n(K)$ mit $B=SAS^{-1}$gibt.
\end{satz}

\begin{definition}[\index{Ähnlichkeit!Matrix}Ähnlichkeit von Matrizen]
Zwei Matrizen $A,B \in \Mat_n(R)$ zwei solche Matrizen heißen \highlight{ähnlich} (in Zeichen $A \sim B$), wenn es $S\in \GL_n(R)\text{ mit } B=SAS^{^-1}$ gibt.
\end{definition}

\begin{satz}
    Ähnlichkeit ist eine Äquivalenzrelation auf $\Mat_n(R)$.
\end{satz}

\begin{satz}\label{Ahnlich_det_mat}
    Seien $A,B \in \Mat_{1\times n}(R)$. Ist $A\sim B$, so ist $\det(A) = \det(B)$.
\end{satz}

\begin{definition}
    Die \begriff[!Determinante]{Determinante eines Endomorphismus} $f\in \End(V)$ ist $\det(f):=\det(M_{\mathcal{B}}(f))$, wobei  $\mathcal{B}$ eine Basis von $V$ ist. (nach \ref{Eindeutigkeit_Transmatrix} ist $\det(f)$ wohldefiniert.)
\end{definition}

\begin{satz}
    Für $f,g\in\End(V)$ gilt:
    \begin{enumerate}[(a)]
        \item $\det(\id_V) = 1$
        \item $\det(f\circ g) = \det(f)\cdot \det(g)$
        \item Genau dann ist $\det(f)\neq 0$, wenn $f\in \aut_K(V)$. In diesem Fall ist $\det(f^{-1}) = (\det(f))^{-1}$
    \end{enumerate}
\end{satz}

\begin{definition}[\index{Spur!Matrix}Spur]
    Die Spur einer Matrix $A=(a_{ij})_{i,j} \in \Mat_n(R)$ ist $\Tr(A) = \sum_{i=1}^{n}a_{ii}$.
\end{definition}

\begin{lemma}
    Seien $A,B \in \Mat_n(R)$, dann
    \begin{enumerate}[(a)]
    \item $\Tr: \Mat_n(R) \to R$ ist $R$-linear, d.h. für $A,B \in \Mat_n(R), \lambda, \mu \in R$ is $\Tr(\lambda A + \mu B) = \lambda \Tr(A) + \mu \Tr(B)$
    \item $\Tr(A^t) = \Tr(A)$
    \item $\Tr(AB) = \Tr(BA)$
    \end{enumerate}
\end{lemma}

\begin{satz}\label{ahnlich_spur_mat}
    Sei $A,B \in \Mat_n(R)$. Ist $A\sim B$, so ist $\Tr(A) = \Tr(B)$.
\end{satz}

\begin{definition}
    \highlight{Spur} eines Endomorpshimus $f \in \End_K(V)$ ist $\Tr(f) = \Tr(M_{\mathcal{B}}(f))$, wobei $\mathcal{B}$ eine Basis von $V$ ist. (Nach \ref{Eindeutigkeit_Transmatrix} und \ref{ahnlich_spur_mat} ist $\Tr(f)$ wohldefiniert.)
\end{definition}

\begin{remark}
    Im Fall $K=R$ kan wie in \ref{geo_int_det} den Absolutbetrag der Determinante eines $f\in \End_K(K^n)$ geometrisch intepretieren, nähmlich als das Volumen von $f(\mathcal{Q})$, wobei $\mathcal{Q} = [0,1]^n$ der Einheitsquader ist und somit als \highlight{Volumenänderung} dur $f$. Auch das Vorzeichen von $\det(f)$ hat eine Bedeutung. Es gibt an, ob $f$ \begriff{orientierungserhaltend} ist. Für eine erste Intepretation der Spur siehe A100.
\end{remark}

\printindex

\end{document}