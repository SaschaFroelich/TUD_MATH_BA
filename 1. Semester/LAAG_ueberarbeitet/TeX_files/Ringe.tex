\section{Ringe}

\begin{definition}[Ring]
	Ein \begriff{Ring} ist ein Tripel $(R,+,\cdot)$ bestehend aus einer Menge
	$R$, einer Verknüpfung $+: R \times R \to R$ (Addition) und einer anderen Verknüpfung
	$\cdot: R \times R \to R$ (Multiplikation), sodass diese zusammen die folgenden Axiome 
	erfüllen:
	\begin{itemize}
		\item (R1) $(R,+)$ ist eine abelsche Gruppe.
		\item (R2) $(R,\cdot)$ ist eine Halbgruppe.
		\item (R3) Für $a,x,y \in R$ gelten die Distributivgesetze $a(x+y)=ax+ay$ und $(x+y)a=xa+ya$.
	\end{itemize}
	Ein Ring heißt kommutativ, wenn $xy=yx$ für alle $x,y \in R$.\\
	Ein neutrales Element der Multiplikation heißt Einselement von $R$.\\
	Ein Unterring eines Rings $(R,+,\cdot)$ ist eine Teilmenge, die mit der geeigneten
	Einschränkung von Addition und Multiplikation wieder ein Ring ist.
\end{definition}

\begin{remark}
	Hat ein Ring ein Einselement, so ist dieses eindeutig bestimmt. Notationelle Konfektionen: Das 
	neutrale Element der Addition wird häufig mit 0 bezeichnet; die Multiplikation wird nicht immer
	notiert; Multiplikation bindet stärker als die Addition. \\
	Wenn die Verknüpfungen aus dem Kontext klar sind, schreibt ma $R$ statt $(R,+,\cdot)$.
\end{remark}

\begin{example}
	\begin{itemize}
		\item Der Nullring ist $R=\{0\}$ mit den einzig möglichen Verknüpfungen $+$ und $\cdot$
		auf $R$. Der Nullring ist sogar kommutativ und hat ein Einselement, nämlich die 0.
		\item $(\mathbb{Z},+,\cdot)$ ist ein kommutativer Ring mit Einselement 1, ebenso
		$(\mathbb{Q},+,\cdot)$ und $(\mathbb{R},+,\cdot)$. 
		\item $(2\mathbb{Z},+,\cdot)$ ist ein kommutativer Ring, aber ohne Einselement.
	\end{itemize}
\end{example}

\begin{remark}
	Ist $R$ ein Ring, dann gelten die folgenden Aussagen für $x,y \in R$
	\begin{itemize}
		\item $0 \cdot x=x \cdot 0 = 0$
		\item $x \cdot (-y) = (-x) \cdot y = -xy$
		\item $(-x) \cdot (-y) = xy$
	\end{itemize}
\end{remark}

\begin{remark}
	Wir führen eine wichtige Klasse endlicher Ringe ein. Hierfür erinnern wir uns an eine der Grundlagen
	der Arithmetik in $\mathbb{Z}$.
\end{remark}

\begin{theorem}
	\proplbl{1_4_6}
	Sei $b \neq 0 \in \mathbb{Z}$. Für jedes $a \in \mathbb{Z}$ gibt es 
	eindeutig bestimmte $q,r \in \mathbb{Z}$ ($r$ ist "'Rest"'), mit $a=qb+r$ und $0 \le r < \vert b\vert$.
\end{theorem}
\begin{proof}
	Existenz und Eindeutigkeit
	\begin{itemize}
		\item Existenz: oBdA nehmen wir an, dass $b>0$ (denn ist $a=qb+r$, so ist auch $a=(-q)(-b)+r$). Sei $q \in
		\mathbb{Z}$ die größte Zahl mit $q \le \frac{a}{b}$, und sei $r=a-qb \in \mathbb{Z}$. Dann ist
		$a \le \frac{a}{b}-q < 1$, woraus $0 \le r < b$ folgt.
		\item Eindeutigkeit: Sei $a=qb+r=q'b+r'$ mit $q,q',r,r' \in \mathbb{Z}$ und $0 \le r,r' < |b|$. Dann ist
		$(q-q')b=r-r'$ und $|r-r'|<|b|$. Da $q-q' \in \mathbb{Z}$ ist, folgt $r-r'=0$ und daraus wegen 
		$b \neq 0$, dann $q-q'=0$.
	\end{itemize}
\end{proof}

\begin{example}[Restklassenring]
	Wir fixieren $n \in \mathbb{N}$. Für $a \in \mathbb{Z}$ sei
	$\overline{a} := a+n\mathbb{Z} := \{a+nx \mid x \in \mathbb{Z}\}$ die \begriff{Restklasse} von "$a \bmod n$". 
	Für $a,a' \in \mathbb{Z}$ sind äquivalent:
	\begin{itemize}
		\item $a+n\mathbb{Z}=a'+n\mathbb{Z}$
		\item $a' \in a+n\mathbb{Z}$
		\item $n$ teilt $a'-a$ (in Zeichen $n|a'-a$), d.h. $a'=a+nk$ für $k \in \mathbb{Z}$
	\end{itemize}
\end{example}
\begin{proof}
	\begin{itemize}
		\item $1) \Rightarrow 2)$: klar, denn $0 \in \mathbb{Z}$
		\item $2) \Rightarrow 3)$: $a' \in a+n\mathbb{Z} \Rightarrow a'=a+nk$ mit $k \in \mathbb{Z}$
		\item $3) \Rightarrow 1)$: $a'=a+nk$ mit $k \in \mathbb{Z} \Rightarrow a+n\mathbb{Z}=\{a+nk+nx \mid 
		x \in \mathbb{Z}\}=\{a+n(k+x) \mid x \in \mathbb{Z}\}=a+n\mathbb{Z}$
	\end{itemize}
	Insbesondere besteht $a+n\mathbb{Z}$ nur aus den ganzen Zahlen, die bei der Division durch $n$ den selben Rest lassen wie $a$.
\end{proof}

Aus \propref{1_4_6} folgt weiter, dass $\mathbb{Z}/n\mathbb{Z} := \{\overline{a} \mid a \in \mathbb{Z}\}
= \{\overline{0}, \overline{1},..., \overline{n-1}\}$ eine Menge der Mächtigkeit n ist (sprich: 
"'$\mathbb{Z} \bmod n\mathbb{Z}$"'). \\
$\newline$

Wir definieren Verknüpfungen auf $\mathbb{Z}/n\mathbb{Z}$ durch $\overline{a}+\overline{b} :=
\overline{a+b}$, $\overline{a} \cdot \overline{b} := \overline{ab}$ $a,b \in \mathbb{Z}$. Hierbei
muss man zeigen, dass diese Verknüpfungen wohldefiniert sind, also nicht von den gewählten
Vertretern $a,b$ der Restklassen $\overline{a}$ und $\overline{b}$ abhängen. Ist etwa $\overline{a}
= \overline{a'}$ und $\overline{b}= \overline{b'}$, also $a'=a+nk_1$ und $b'=b+nk_2$ mit $k_1,k_2 \in
\mathbb{Z}$, so ist \\
$a'+b' = a+b+n(k_1+k_2)$, also $\overline{a'+b'} = \overline{a+b}$ \\
$a' \cdot b' = ab+n(bk_1+ak_2+nk_1k_2)$, also $\overline{a'b'} = \overline{ab}$ \\
Man prüft nun leicht nach, dass $\mathbb{Z}/n\mathbb{Z}$ mit diesen Verknüpfungen ein kommutativer
Ring mit Einselement ist, da dies auch für $(\mathbb{Z},+,\cdot)$ gilt. Das neutrale Element der
Addition ist $\overline{0}$, das Einselement ist $\overline{1}$. \\
$\newline$

\begin{example}
	Im Fall $n=2$ ergeben sich die folgenden Verknüpfungstafeln für $\mathbb{Z}
	/2\mathbb{Z} = \{\overline{0}, \overline{1}\}$ \\
	\begin{center}
		\begin{tabular}{|c|c|c|}
			\hline
			$+$ & $\overline{0}$ & $\overline{1}$\\
			\hline
			$\overline{0}$ & $\overline{0}$ & $\overline{1}$\\
			\hline
			$\overline{1}$ & $\overline{1}$ & $\overline{2}=\overline{0}$ \\
			\hline
		\end{tabular}
	\end{center}
	\begin{center}
		\begin{tabular}{|c|c|c|}
			\hline
			$\cdot$ & $\overline{0}$ & $\overline{1}$\\
			\hline
			$\overline{0}$ & $\overline{0}$ & $\overline{0}$\\
			\hline
			$\overline{1}$ & $\overline{0}$ & $\overline{1}$ \\
			\hline
		\end{tabular}
	\end{center}
\end{example}

\begin{definition}[Charakteristik]
	Sei $R$ ein Ring mit Einselement. Man definiert die \begriff{Charakteristik} von
	$R$ als die kleinste natürliche Zahl $n$ mit $1+1+...+1=0$, falls so ein $n$ existiert, andernfalls
	ist die Charakteristik $0$.
\end{definition}

\begin{definition}[Nullteiler]
	Sei $R$ ein Ring mit Einselement. Ein $0 \neq x \in R$ ist ein \begriff{Nullteiler} von 
	$R$, wenn er ein $0 \neq y \in R$ mit $xy=0$ oder $yx=0$ gibt. Ein Ring ohne Nullteiler ist
	nullteilerfrei.
\end{definition}

\begin{definition}[Einheit]
	Sei $R$ ein Ring mit Einselement. Ein $x \in R$ heißt invertierbar (oder
	\begriff{Einheit} von $R$), wenn es ein $x' \in R$ mit $xx'=x'x=1$ gibt. Wir bezeichnen die invertierten
	Elemente von $R$ mit $R^{\times}$.
\end{definition}

\begin{example}
	\proplbl{1_4_12}
	\begin{itemize}
		\item reelle Zahlen sind ein nullteilerfreier Ring der Charakteristik $0$ mit $\mathbb R^{\times}=
		\mathbb R\backslash\{0\}$
		\item $\mathbb Z$ ist ein nullteilerfreier Ring der Charakteristik $0$ mit $\mathbb Z^{\times}=
		\{1,-1\}$
		\item $\mathbb Z/n \mathbb Z$ ist ein Ring der Charakteristik $n$. Ist $n$ keine Primzahl, so
		ist $\mathbb Z$ nicht nullteilerfrei.
	\end{itemize}
\end{example}

\begin{proposition}
	\proplbl{1_4_13}
	Sei $R$ ein Ring mit Einselement. 
	\begin{itemize}
		\item Ist $x \in R$ invertierbar, so ist $x$ kein Nullteiler in $R$.
		\item Die invertierbaren Elemente von $R$ bilden mit der Multiplikation eine Gruppe.
	\end{itemize}
\end{proposition}
\begin{proof}
	\begin{itemize}
		\item Ist $xx'=x'x=1$ und $xy=0$ mit $x',y \in R$, so ist $0=x'\cdot 0=x\cdot xy=1\cdot y=y$, aber
		$y \neq 0$ für Nullteiler
		\item Sind $x,y \in R^{\times}$, also $xx'=x'x=yy'=y'y=1$. Dann ist $(xy)(y'x')=x\cdot 1\cdot x'=1$
		und $(y'x')(xy)=y'\cdot 1\cdot y=1$, somit $R^{\times}$ abgeschlossen unter der Multiplikation. Da 
		$1 \cdot 1=1$ gilt, ist auch $1 \in R^{\times}$. Nach Definition von $R^{\times}$ hat jedes $x \in 
		R^{\times}$ ein Inverses $x' \in R^{\times}$.
	\end{itemize}
\end{proof}