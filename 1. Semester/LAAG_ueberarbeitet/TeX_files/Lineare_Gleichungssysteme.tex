\section{Lineare Gleichungssysteme}

Sei $A\in \Mat_{m\times n}(K)$ und $b\in K^m$.

\begin{definition}[Lineares Gleichungssystem]
	Unter einem \begriff{Linearen Gleichungssystem} verstehen wir eine Gleichung der Form $Ax=b$. 
	Diese heißt \begriff{homogen}, wenn $b=0$, sonst \begriff{inhomogen} und $L(A,b)=\{x\in K^n\mid Ax=b\}$ ist sein \begriff{Lösungsraum}.
\end{definition}

\begin{mathematica}[Lineare Gleichungssysteme]
	Für das Lösen von Linearen Gleichungssystemen gibt es in WolframAlpha bzw. Mathematica verschiedene Verfahren:
	\begin{itemize}
		\item \texttt{Solve[]}:
		\begin{align}
			\texttt{Solve[a == 2 b \&\& b == 5 \&\& c + a == b, \{a, b, c\}]}\notag
		\end{align}
		\item \texttt{LinearSolve[]}: Braucht 2 Argumente: Zum einen die Koeffizientenmatrix $A$ und den Ergebnisvektor $b$. Rückgabe ist dann der Variablenvektor $x$.
		\begin{align}
			\texttt{LinearSolve[\{\{1, 1\}, \{0, 1\}\}, \{6, 10\}]}\notag
		\end{align}
	\end{itemize}
\end{mathematica}

\begin{remark}
	Ist $A=(a_{ij})$, $b=(b_1,...,b_m)^t$, so schreibt man das Lineare Gleichungssystem $Ax=b$ auch
	\begin{align}
	\begin{vmatrix}
		a_{11}x_1 + \dots + a_{1n}x_n & = & b_1\\
		\vdots & \vdots & \vdots\\
		a_{m1}x_1 + \dots + a_{mn}x_n & = & b_m\\
	\end{vmatrix}\notag
	\end{align}
\end{remark}

\begin{remark}
	Das homogene System $Ax=0$ hat als Lösungsraum den Untervektorraum $L(A,0)=\Ker(f_A)$ der Dimension $\dim_K(L(A,0))=n-\rk(A)$. Das 
	inhomogene System hat entweder $L(A,b)=\emptyset$ oder der Lösungsraum ist der affine Unterraum $L(A,b)=f^{-1}(b)=x_0+L(A,0)$, wobei 
	$x_0\in L(A,b)$ beliebig. Man erhält so alle Lösungen des inhomogenen Systems, wenn man eine Lösung und die Lösungen des homogenen 
	Systems kennt.
\end{remark}

\begin{definition}[Zeilenstufenform]
	Die Matrix $A=(a_{ij})$ hat \begriff{Zeilenstufenform}, wenn es ganze Zahlen $0\le r \le m$ und $1\le 
	k_1<...<k_r\le n$ gibt mit:
	\begin{itemize}
		\item für $1\le i \le r$ und $1\le j < k_i$ ist $a_{ij}=0$
		\item für $1\le i \le r$ ist $a_{ik_{i}}\neq 0$ (sogenannte \begriff{Pivotelemente})
		\item für $r<i\le m$ und $1\le j\le n$ ist $a_{ij}=0$
	\end{itemize}
	\begin{align}
		\begin{pmatrix}
		0 & \dots & 0 & a_{1k_{1}} & * & \dots & \dots & *\\
		0 & \dots & \dots & 0 & a_{2k_{2}} & * & \dots & *\\
		\vdots & \vdots & \vdots & \vdots & \vdots & \vdots & \vdots & \vdots\\
		0 & \dots & \dots & \dots & \dots & \dots & \dots & a_{rk_{r}}\\
		0 & \dots & \dots & \dots & \dots & \dots & \dots & 0\\
		\vdots & \; & \; & \; & \; & \; & \; & \vdots\\
		0 & \dots & \dots & \dots & \dots & \dots & \dots & 0\\
		\end{pmatrix}\notag
	\end{align}
\end{definition}

\begin{lemma}
	Sei $A$ in Zeilenstufenform. Dann ist $\rk(A)=r$.
\end{lemma}
\begin{proof}
	Wegen $\rk(A)=\rk(A^t)=\dim_K(\ZR)$ genügt es zu zeigen, dass die ersten $r$ Zeilen $a_1,...,a_r$ linear unabhängig sind. Ist $\sum
	_{i=1}^r \lambda a_i=0$, so ist insbesondere $0=\sum_{i=1}^r \lambda_i a_{ik_{i}}=\lambda_1 a_{1k_{1}}$, also $\lambda_1
	=0$, und dann immer so weiter.
\end{proof}

\begin{proposition}
	\proplbl{3_9_6}
	Sei $A$ in Zeilenstufenform.
	\begin{itemize}
		\item Ist $b_i\neq 0$ für ein $r<i\le m$, so ist $L(A,b)=\emptyset$.
		\item Ist $b_i=0$ für alle $r<i\le m$, so erhält man alle $x\in L(A,b)$, indem man erst $x_j\in K$ für $j\in \{1,..,n\}
		\backslash \{k_1,...,k_r\}$ beliebig wählt und dann für $i=r,r-1,...,1$ rekursiv $x_{k_{i}}=a_{1k_{i}}^{-1}\cdot (b_i-\sum
		_{j=k_i+1}^n a_{ij}\cdot x_j)\quad (*)$ setzt.
	\end{itemize}
\end{proposition}
\begin{proof}
	\begin{itemize}
		\item Klar.
		\item Sicher erhält man auf diese Weise Lösungen $x\in L(A,b)$. Umgekehrt muss jede solche Lösung $(*)$ erfüllen, man erhält auf 
		diese Weise also alle.
	\end{itemize}
\end{proof}

\begin{definition}[Elementarmatrizen]
	Für $i,j\in \{1,...,m\}$, $\lambda \in K^{\times}$ und $\mu\in K$ definieren wir 
	$m\times m$-Matrizen, die sogenannten \begriff{Elementarmatrizen}:
	\begin{itemize}
		\item $S_i(\lambda):=1_m + (\lambda-1)E_{ii}$
		\item $Q_{ij}(\mu):= 1_m + \mu E_{ij}$
		\item $P_{ij}:= 1_m + E_{ij} + E_{ji} - E_{ii} - E_{jj}$
	\end{itemize}
\end{definition}

\begin{remark}
	Multiplikation einer dieser Matrizen von links an die Matrix $A$ hat folgende Wirkung:
	\begin{itemize}
		\item $S_i(\lambda)\cdot A$: Multiplikation der $i$-ten Zeile mit $\lambda$
		\item $Q_{ij}(\mu)\cdot A$: Addition des $\mu$-fachen der $j$-ten Zeile zur $i$-ten Zeile
		\item $P_{ij}$: Vertauschung von $i$-ter und $j$-ter Zeile
	\end{itemize}
	Man spricht dann von sogenannten elementaren Zeilenumformungen der Matrix $A$ von Typ I, II oder III.
\end{remark}

\begin{lemma}
	Es sind $S_i(\lambda),Q_{ij}(\mu),P_{ij}\in \GL_m(K)$. Dann ist $S_i(\lambda)^{-1}=S_i(\lambda^{-1}), Q_{ij}(\mu)
	^{-1}=Q_{ij}(-\mu),P_{ij}^{-1}=P_{ij}$. Insbesondere gilt: Ist $E$ eine der Elementarmatrizen, so ist $\ZR(EA)=\ZR(A)$ und $L(EA,0)=
	L(A,0)$. Weiterhin ist $\rk(EA)=\rk(A)$.
\end{lemma}
\begin{proof}
	Inverse nachprüfen. Da $E\in \GL_m(K)$ sind $f_E,f_{E^t}\in Aut_K(K^m)$, also $\ZR(EA)=\SR((EA)^t)=\Image(f_{A^tE^t})=\Image(f_{A^t}\circ
	f_{E^t})=\Image(f_{A^t})=\ZR(A)$ und $L(EA,0)=\Ker(f_{EA})=\Ker(f_E\circ f_A)=\Ker(f_A)=L(A,0)$.
\end{proof}

\begin{remark}
	Anders gesagt: Elementare Zeilenumformungen verändern den Lösungsraum eines homogenen linearen Gleichungssystems 
	nicht.
\end{remark}

\begin{theorem}[Eliminierungsverfahren nach \person{Gauß}]
	\proplbl{3_9_11}
	Zu jeder Matrix $A\in \Mat_{m\times n}(K)$ gibt es $l\in \mathbb N_0$ und 
	Elementarmatrizen $E_1,...,E_l$ vom Typ II und III für die $E_l\cdot ... \cdot E_1\cdot A$ in Zeilenstufenform ist. 
\end{theorem}
\begin{proof}
	Seien $a_1,...,a_n$ die Spalten von $A$. \\
	Ist $A=0$ so ist nichts zu tun. \\
	Sei nun $A\neq 0$ und sei $k_1$ minimal mit $a_{k_1}\neq 0$. Es gibt also ein $i$ mit $a_{ik_1}\neq 0$. Durch Vertauschen der ersten 
	und der $i$-ten Zeile erreichen wir, dass $a_{1k_1}=0$, d.h. wir multiplizieren $A$ mit $E_1=P_{1i}$. Nun addieren wir für $i=2,..,m$ 
	ein geeignetes Vielfaches der ersten Zeile zur $i$-ten Zeile, um $a_{ik_1}=0$, d.h. wir multiplizieren $A$ mit $E_i=Q_{i1}(\mu_i)$ für 
	$\mu_i=\frac{a_{ik_1}}{a_{1k_1}}$. Nach diesen Umformungen haben wir eine Matrix der Form:
	\begin{align}\begin{pmatrix}
		0 & \dots & 0 & a_{1k_1} & * & \dots & *\\
		0 & \dots & \dots & 0 & \textcolor{red}{*} & \textcolor{red}{\dots} & \textcolor{red}{*}\\
		\vdots & \vdots & \vdots & \vdots & \textcolor{red}{\vdots} & \textcolor{red}{\vdots} & \textcolor{red}{\vdots}\\
		0 & \dots & \dots & 0 & \textcolor{red}{*} & \textcolor{red}{\dots} & \textcolor{red}{*}\\
		\end{pmatrix}\notag\end{align}
	und können nun mit dem \textcolor{red}{Rest der Matrix $A=:A'$} von vorne beginnen. Die nun folgenden Zeilenumformungen werden die 
	erste Zeile und die ersten $k_1$ Spalten nicht mehr ändern, und weil $A'$ weniger Zeilen und Spalten als $A$ hat, bricht das Verfahren 
	nach endlich vielen Schritten ab.
\end{proof}

\begin{mathematica}[\person{Gauss}-Verfahren]
	Auch für das \person{Gauss}-Verfahren hat Mathematica bzw. WolframAlpha eine Funktion. Sie gibt die Matrix nach Ausführung des \person{Gauss}-Algorithmus zurück.
	\begin{align}
		\texttt{RowReduce[\{\{1, 4\}, \{2, 5\}\}]}\notag
	\end{align}
\end{mathematica}

\begin{conclusion}
	Zu jeder Matrix $A$ gibt es eine invertierbare Matrix $S\in \GL_n(K)$ für die $SA$ in Zeilenstufenform ist.
\end{conclusion}
\begin{proof}
	folgt direkt aus \propref{3_9_11} mit $S=E_l\cdot ... \cdot E_1$
\end{proof}

\begin{remark}
	Der Beweis für das Eliminierungsverfahren (\propref{3_9_11}) liefert ein Verfahren, die Elementarmatrizen $E_1,...,E_l$ zu finden. 
	Damit erhält man ein Verfahren ein lineares Gleichungssystem zu lösen. Setzt man $S=E_l\cdot ... \cdot E_1$, $A'=SA$ und $b'=Sb$, so 
	ist $L(A,b)=L(A',b')$: $Ax=b\Rightarrow SAx=Sb$ bzw. $A'x=b' \Rightarrow S^{-1}A'x=S^{-1}b'$. \\
	Das Gleichungssystem kann dann mit \propref{3_9_6} gelöst werden. Praktisch führt man die elementaren Zeilenumformungen an $A$ parallel dazu auch an $b$ 
	durch.
\end{remark}

\begin{remark}
	Es gibt von diesem Verfahren verschiedene Varianten und weitere Anwendungen: So kann man z.B. die Invertierbarkeit 
	einer Matrix $A\in \Mat_n(K)$ prüfen und ggf. das Inverse bestimmen: Ist $E_l\cdot ... \cdot E_1\cdot A$ in Zeilenstufenform, so ist $A$ 
	genau dann invertierbar, wenn alle Zeilen von Null verschieden sind. Ist dies der Fall, so ist $r=n$ und $k_i=i$ für alle $i$, 
	und man findet weitere Elementarmatrizen $E_{l+1},...,E_s$ vom Typ I und II, für die $E_s\cdot ... \cdot E_1\cdot A=1_n$. Dann ist 
	$S'=E_s\cdot ... \cdot E_1\cdot A=A^{-1}$ (vgl. \propref{3_8_11}). Praktisch erhält man $A^{-1}$, indem man die Zeilenumformungen an $A$ parallel dazu 
	auch an $1_n$ ausführt.
\end{remark}

\begin{conclusion}
	Jedes $A\in \GL_m(K)$ ist ein Produkt von Elementarmatrizen.
\end{conclusion}
\begin{proof}
	$A^{-1}=S'=E_s\cdot ... \cdot E_1 \Rightarrow A=(E_s\cdot ... \cdot E_1)^{-1}=E_1^{-1}\cdot ... \cdot E_s^{-1}$
\end{proof}