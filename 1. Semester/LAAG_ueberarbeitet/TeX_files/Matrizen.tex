\section{Matrizen}

Sei $K$ ein Körper.

\begin{definition}[Matrix]
	Seien $m,n \in \mathbb N_0$. Eine $m\times n$-\begriff{Matrix} über $K$ ist ein rechteckiges 
	Schema:
	\begin{align}
		\begin{pmatrix}
		a_{11} & ... & a_{1n}\\
		... &  & ...\\
		a_{m1} & ... & a_{mn}\\
		\end{pmatrix}\notag
	\end{align}
	Man schreibt dies auch als $A=(a_{ij})_{i=1,...,n \; j=1,...,m}$ oder $A=(a_{ij})_{i,j}$, wenn $m$ und $n$ 
	aus dem Kontext hervorgehen. Die $a_{ij}$ heißen die \begriff[Matrix!]{Koeffizienten} der Matrix $A$ und wir definieren $A_{i,j}=
	a_{ij}$. Die Menge der $m\times n$-Matrizen über $K$ wird mit $\Mat_{m\times n}(K)$ oder $K^{m\times n}$ 
	bezeichnet. Man nennt das Paar $(m,n)$ auch den \begriff[Matrix!]{Typ} von $A$. Ist $m=n$, so spricht man von \begriff[Matrix!]{quadratisch}en 
	Matrizen und schreibt $\Mat_n(K)$. Zu einer Matrix $A=(a_{ij}) \in \Mat_{m\times n}(K)$ definiert man die zu $A$ 
	\begriff[Matrix!]{transponierte Matrix} $A^t := (a_{ij})_{j,i} \in \Mat_{n\times m}(K)$.
\end{definition}

\begin{mathematica}[Matrizen]
	Matrizen werden in Mathematica bzw. WolframAlpha folgendermaßen dargestellt:
	\begin{align}
		&\begin{pmatrix}1&2&3\\4&5&6\\7&8&9\end{pmatrix}\notag \\
		\Rightarrow&\Big\lbrace\texttt{\{1,2,3\},\{4,5,6\},\{7,8,9\}}\Big\rbrace\notag
	\end{align}
	Wenn Mathematica als Ergebnis eine Matrix ausgibt, so lässt sich diese als Zeile schlecht lesen. Mit dem Suffix \texttt{//MatrixForm} lässt sich der Output als Matrix formatieren:
	\begin{align}
		&\texttt{\{\{1,2,3\},\{4,5,6\},\{7,8,9\}\}*3 //MatrixForm}\notag \\
		\Rightarrow&\begin{pmatrix}3&6&9\\12&15&18\\21&24&27\end{pmatrix}\notag
	\end{align}
\end{mathematica}

\begin{example}
	\begin{itemize}
		\item Die \begriff{Nullmatrix} ist $0=(0)_{i,j} \in \Mat_{m\times n}(K)$.
		\item Für $k,l \in \{1,...,n\}$ ist die $(k,l)$-\begriff{Basismatrix} gegeben durch $E_{kl}=(\delta_{jk}\delta_{jl})\in 
		\Mat_{m\times n}(K)$.
		\item Die \begriff{Einheitsmatrix} ist $1_n=(\delta_{ii})\in \Mat_n(K)$.
		\item Für $a_i,...,a_n \in K$ definiert man eine \begriff{Diagonalmatrix} $\diag(a_1,...,a_n)=(\delta_{ij}\cdot a_i)$.
		\item Für eine Permutation $\sigma\in S_n$ definiert man die \begriff{Permutationsmatrix} $P_\sigma := (\delta_{\sigma
			(i),j})$.
		\item Für $a_1,...,a_n$ definiert man einen \begriff{Zeilenvektor} $(a_1,...,a_n)\in \Mat_{1\times n}(K)$ bzw. einen 
		\begriff{Spaltenvektor} $(a_1,...,a_n)^t$.
	\end{itemize}
\end{example}

\begin{definition}[Addition und Skalarmultiplikation]
	Seien $A=(a_{ij})$ und $B=(b_{ij})$ desselben Typs und 
	$\lambda \in K$. Man definiert auf $\Mat_{m\times n}(K)$ eine koeffizientenweise \begriff[Matrix!]{Addition} und \begriff[Matrix!]{Skalarmultiplikation}.
\end{definition}

\begin{mathematica}[Matrizenoperationen]
	Die komponentenweise Addition bzw. Skalarmultiplikation von Matrizen $A$ und $B$ lässt sich in Mathematica bzw. WolframAlpha folgendermaßen realisieren:
	\begin{align}
		\texttt{A+B}\notag \\
		\texttt{A*B}\notag
	\end{align}
\end{mathematica}

\begin{proposition}
	\proplbl{3_1_4}
	$(\Mat_{m\times n},+,\cdot)$ ist ein $K$-Vektorraum der Dimension $\dim_K(\Mat_{m\times n})=n\cdot m$ mit 
	Basismatrix als Basis.
\end{proposition}
\begin{proof}
	Dies ist klar, weil wir $\Mat_{m\times n}$ mit dem Standardraum $K^{mn}$ identifizieren können. Wir haben die 
	Elemente nur als $m\times n$-Matrix statt als $mn$-Tupel geschrieben.
\end{proof}

\begin{definition}[Matrizenmultiplikation]
	Seien $m,n,r \in \mathbb N_0$. Sind $A=(a_{ij})\in \Mat_{m\times n}(K)$, 
	$B=(b_{jk})\in \Mat_{n\times r}(K)$ so definieren wir die \begriff[Matrix!]{Matrizenmultiplikation} $C=AB$ als die Matrix $C=(c_{ik})\in \Mat_{m\times r}(K)$ mit 
	$c_{ik}=\sum_{j=1}^n a_{ij}\cdot b_{jk}$. Kurz geschrieben "'Zeile $\cdot$ Spalte"'.
\end{definition}

\begin{mathematica}[Matrizenmultiplikation]
	Die Matrizenmultiplikation in Mathematica und WolframAlpha für Matrizen $A$ und $B$ geht so:
	\begin{align}
		\texttt{A.B}\text{ oder }\texttt{Dot[A,B]}\notag
	\end{align}
\end{mathematica}

\begin{example}
	\proplbl{3_1_6}
	\begin{itemize}
		\item Für $A\in \Mat_n(K)$ ist $0\cdot A=0$ und $1\cdot A=A$.
		\item Für $\sigma \in S_n$ und $A\in \Mat_{n\times r}(K)$ geht $P_{\sigma}\cdot A$ aus $A$ durch Permutation der 
		Zeilen hervor.
	\end{itemize}
\end{example}

\begin{lemma}
	\proplbl{3_1_7}
	Für $m,n,r \in \mathbb N_0$ und $A=(a_{ij})\in \Mat_{m\times n}(K)$, $B=(b_{jk})\in \Mat_
	{n\times r}(K)$ und $\lambda\in K$ gilt:
	\begin{align}
		A(\lambda B)=(\lambda A)B=\lambda(AB)\notag
	\end{align}
\end{lemma}
\begin{proof}
	Schreibe $A=(a_{ij})$, $B=(b_{jk})$. Dann ist $A(\lambda B)=\sum_{j=1}^n a_{ij}\cdot \lambda b_{jk}=\sum
	_{j=1}^n \lambda a_{ij} \cdot b_{jk}=(\lambda A)B=\lambda \cdot \sum_{j=1}^n a_{ij}b_{jk}=\lambda
	(AB)$.
\end{proof}

\begin{lemma}
	\proplbl{3_1_8}
	Matrizenmultiplikation ist assoziativ:
	\begin{align}
		A(BC)=(AB)C\notag
	\end{align}
\end{lemma}
\begin{proof}
	Sei $D=BC\in \Mat_{n\times s}(K)$, $E=AB \in \Mat_{m\times r}(K)$. Schreibe $A=(a_{ij})$ usw. Für $i,l$ ist $(AD)=
	\sum_{j=1}^n a_{ij}d_{jl}=\sum_{j=1}^n a_{ij}\cdot \sum_{k=1}^r b_{jk}c_{kl}=\sum
	_{j=1}^n \sum_{k=1}^n a_{ij}b_{jk}c_{kl}$. \\
	$(EC)=\sum_{k=1}^n e_{ik}c_{kl}=\sum_{k=1}^r \sum_{j=1}^n a_{ij}b_{jk}c_{kl}$. Also ist 
	$AD=EC$.
\end{proof}

\begin{lemma}
	\proplbl{3_1_9}
	Für $m,n,r\in \mathbb N_0$ und $A,A'\in \Mat_{m\times n}(K)$, $B,B'\in \Mat_{n\times r}(K)$ ist
	 \begin{align}
	 	(A+A')B&=AB+A'B\notag \\
	 	A(B'+B)&=AB'+AB\notag
	 \end{align}
\end{lemma}
\begin{proof}
	Schreibe $A=(a_{ij})$ etc. Dann ist $(A+A')B=\sum_{j=1}^n (a_{ij}+a'{ij})b_{jk}=\sum_{j=1}^n 
	a_{ij}+b_{jk} + \sum_{j=1}^n a'_{ij}+b_{jk}=(AB+A'B)$. Rest analog.
\end{proof}

\begin{proposition}
	Mit der Matrizenmultiplikation wird $\Mat_n(K)$ zu einem Ring mit Einselement $1$.
\end{proposition}
\begin{proof}
	Nach \propref{3_1_4}, \propref{3_1_8} und \propref{3_1_9} ist $\Mat_n(K)$ ein Ring und dass $\mathbbm{1}_n$ ein neutrales Element ist, haben wir schon in \propref{3_1_6} gesehen
\end{proof}

\begin{example}
	\begin{itemize}
		\item Für $n=1$ können wir dem Ring $\Mat_n(K)$ mit $K$ identifizieren, der Ring ist also ein Körper, 
		insbesondere ist er kommutativ.
		\item Für $n\ge 2$ ist $\Mat_n(K)$ nicht kommutativ.
	\end{itemize}
\end{example}

\begin{definition}[invertierbar]
	Eine Matrix $A\in \Mat_n(K)$ heißt \begriff[Matrix!]{invertierbar} oder \begriff[Matrix!]{regulär}, wenn sie im Ring 
	$\Mat_n(K)$ invertierbar ist, sonst \begriff[Matrix!]{singulär}. Die Gruppe $\GL_n(K)=\Mat_n(K)^{\times}$ der invertierbaren $n\times n$
	-Matrizen heißt \begriff{allgemeine Gruppe}.
\end{definition}

\begin{mathematica}[Matizen invertieren]
	Das Inverse einer Matrix $A$ in Mathematica bzw. WolframAlpha lässt sich mit der Funktion
	\begin{align}
		\texttt{Inverse[A]}\notag
	\end{align}
	berechnen.
\end{mathematica}

\begin{example}
	\proplbl{3_1_13}
	Sei $n=2$. Zu
	\begin{align}
		A=\begin{pmatrix}a & b\\c & d\\\end{pmatrix} \in \Mat_2(K)\notag
	\end{align} 
	definiert man
	\begin{align}
		\tilde A=
		\begin{pmatrix}d & -b\\-c & a\\\end{pmatrix}\in \Mat_2(K)\notag
	\end{align}
	Man prüft nach, dass $A\cdot \tilde A=\tilde A\cdot A=
	(ad-bc)\cdot 1_2$. Definiert man nun $\det(A)=ad-bc$ so sieht man: Ist $\det(A)\neq 0$, so ist $A$ invertierbar mit 
	$A^{-1}=\det(A)^{-1}\cdot \tilde A$. Ist $\det(A)=0$ so $A$ ist Nullteiler und somit nicht invertierbar (\propref{1_4_13}). Mehr dazu in Kapitel IV.
\end{example}

\begin{lemma}
	\proplbl{3_1_14}
	Für $A,A_1,A_2\in \Mat_{m\times n}(K)$ und $B=\Mat_{n\times r}(K)$ ist 
	\begin{itemize}
		\item $(A^t)^t=A$
		\item $(A_1+A_2)^t=A_1^t + A_2^t$
		\item $(AB)^t=B^tA^t$
	\end{itemize}
\end{lemma}
\begin{proof}
	Übung
\end{proof}

\begin{proposition}
	Für $A\in \GL_n(K)$ ist $A^t\in \GL_n(K)$ und $(A^{-1})^t = (A^t)^{-1}$
\end{proposition}
\begin{proof}
	Aus $AA^{-1}=1$ folgt nach \propref{3_1_14}, dass $(A^{-1})^tA^t=1_n^t=1_n$. Somit ist $(A^{-1})^t$ das Inverse zu $A^t$.
\end{proof}