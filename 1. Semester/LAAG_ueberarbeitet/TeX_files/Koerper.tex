\section{Körper}

\begin{definition}[Körper]
	Ein \begriff{Körper} ist ein kommutativer Ring $(K,+,\cdot)$ mit Einselement 
	$1 \neq 0$, in dem jedes Element $x \neq x \in K$ invertierbar ist.
\end{definition}

\begin{remark}
	Nach \propref{1_4_13} ist ein Körper ist stets nullteilerfrei und $(K\backslash\{0\}, \cdot)$ ist eine abelsche
	Gruppe. Ein Körper ist also ein Tripel $(K,+,\cdot)$ bestehend aus einer Menge $K$ und 2 Verknüpfungen
	$+: K \times K \to K$ und $\cdot: K \times K \to K$, für die gelten: \\
	(K1): $(K,+)$ ist eine abelsche Gruppe \\
	(K2): $(K\backslash\{0\}, \cdot)$ ist eine abelsche Gruppe, deren neutrales Element wir mit 1 bezeichnen \\
	(K3): Es gelten die Distributivgesetze.
\end{remark}

\begin{remark}
	Sei $K$ ein Körper und $a,x,y \in K$. Ist $ax=ay$ und $a \neq 0$, so ist $x=y$.
\end{remark}

\begin{definition}[Teilkörper]
	Ein \begriff{Teilkörper} eines Körpers $(K,+,\cdot)$ ist die Teilemenge $L 
	\subset K$, die mit der geeigneten Einschränkung von Addition und Multiplikation wieder ein
	Körper ist.
\end{definition}

\begin{example}
	\begin{itemize}
		\item Der Nullring ist kein Körper.
		\item Der Körper $\mathbb Q$ der rationalen Zahlen ist ein Teilkörper des Körpers $\mathbb R$ der
		reellen Zahlen.
		\item $(\mathbb Z, + ,\cdot)$ ist kein Körper
	\end{itemize}
\end{example}

\begin{example}[Komplexe Zahlen]
	Wir definieren die Menge $\mathbb C = \mathbb R \times \mathbb R$ und darauf Verknüpfungen wie folgt:
	Für $(x_1,y_1), (x_2,y_2) \in \mathbb C$ ist:
	\begin{itemize}
		\item$(x_1,y_1)+(x_2,y_2) := (x_1+x_2,y_1+y_2)$
		\item$(x_1,y_1)\cdot (x_2,y_2) := (x_1x_2-y_1y_2,x_1y_2+x_2y_1)$
	\end{itemize}
	Wie man nachprüfen kann, ist $(\mathbb C,+,\cdot)$ ein Körper, genannt Körper der komplexen Zahlen.
	Da $(x_1,0)+(x_2,0)=(x_1+x_2,0)$ und $(x_1,0)\cdot (x_2,0)=(x_1x_2,0)$, können wir $\mathbb R$ durch
	"'$x=(x,0)$"' mit dem Teilkörper $\mathbb R \times \{0\}$ von $\mathbb C$ identifizieren. \\
	Die imaginäre Einheit $i=(0,1)$ erfüllt $i^2=-1$ und jedes $z \in \mathbb C$ kann eindeutig geschrieben
	werden als $z=x+iy$ mit $x,y \in \mathbb R$
\end{example}


\begin{lemma}
	\proplbl{1_5_7}
	Sei $a \in \mathbb Z$ und sei $p$ eine Primzahl, die $a$ nicht teilt. Dann gibt es $b,k \in
	\mathbb Z$ mit $ab+kp=1$.
\end{lemma}
\begin{proof}
	Sei $n \in \mathbb N$ die kleinste natürliche Zahl der Form $n=ab+kp$. Angenommen, $n \ge 2$. Schreibe
	$a=qp+r$ mit $q,r \in \mathbb Z$ und $0 \le r < p$ (\propref{1_4_6}). Aus der Nichtteilbarkeit von $a$ folgt $r \neq 0$, also 
	$r \in \mathbb N$. Wegen $r=a\cdot 1-qp$ ist $n\le r$. Da $p$ Primzahl ist und $2\le n\le r < p$, gilt $n$ teilt
	nicht $p$. Schreibe $p=c\cdot n+m$ mit $c,m \in \mathbb Z$ und $0 \le m<n$ (\propref{1_4_6}). Aus $n$ teilt nicht $p$ folgt
	$m \neq 0$, also $m \in \mathbb N$. Da $m=p-cn=-abc+(1-kc)p$, ist $m<n$ ein Widerspruch zur Minimalität
	von $n$. Die Annahme $n \ge 2$ war somit falsch. Es gilt $n=1$.
\end{proof}

\begin{example}[Endliche Primkörper]
	Für jede Primzahl $p$ ist $\mathbb Z /p \mathbb Z$ ein Körper. Ist $\overline{a}\neq \overline{0}$, so gilt 
	$p$ teilt nicht $a$ und somit gibt es nach \propref{1_5_7} $b,k \in \mathbb Z$ mit \\
	\begin{align}
		ab+kp &= 1 \notag \\
		\overline{(ab+kp)} &= \overline{1} = \overline{(ab)} = \overline{a} \cdot \overline{b} \notag
	\end{align}
	und somit ist $\overline{a}$ invertierbar in $\mathbb Z /p \mathbb Z$. Somit sind für $n \in \mathbb N$
	äquivalent:
	\begin{itemize}
		\item $\mathbb Z /n \mathbb Z$ ist ein Körper
		\item $\mathbb Z /n \mathbb Z$ ist nullteilerfrei
		\item $n$ ist Primzahl
	\end{itemize}
\end{example}
\begin{proof}
	\begin{itemize}
		\item 1 $\Rightarrow$ 2: \propref{1_4_13}
		\item 2 $\Rightarrow$ 3: \propref{1_4_12}
		\item 3 $\Rightarrow$ 1: gegeben
	\end{itemize}
	Insbesondere ist $\mathbb Z /p \mathbb Z$ nullteilerfrei, d.h. aus $p\vert ab$ folgt $p\vert a$ oder $p\vert b$.
\end{proof}

\begin{remark}
	Ist $K$ ein Körper und $a,b\in K$, $b\neq 0$, so schreiben wir $\frac{a}{b}$ für $ab^{-1}=b^{-1}a$. Es gelten die bekannten Rechenregeln für Brüche (vgl. \propref{1_3_10}):
	\begin{align}
		\frac{a_1}{b_1}+\frac{a_2}{b_2}&=\frac{a_1b_2+a_2b_1}{b_1b_2}\notag \\
		\frac{a_1}{b_1}\cdot\frac{a_2}{b_2}&=\frac{a_1a_2}{b_1b_2}\notag
	\end{align}
\end{remark}