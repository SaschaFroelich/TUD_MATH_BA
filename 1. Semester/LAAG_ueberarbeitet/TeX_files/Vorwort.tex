Schön, dass du unser Skript für die Vorlesung \textit{Lineare Algebra und analytische Geometrie 1} bei Prof. Dr. Arno Fehm im WS2017/18 gefunden hast! \footnote{Obwohl man sagen kann, dass es in dieser Vorlesung nur um Lineare Algebra ging, der Teil mit der analytischen Geometrie wurde vernachlässigt. Liegt wahrscheinlich auch daran, dass es demnächst eine Reform der Studienordnung gibt, in der aus der Vorlesung \textit{Lineare Algebra und analytische Geometrie} die Vorlesung \textit{Einführung in die Lineare Algebra} wird.}

Wir verwalten dieses Skript mittels Github \footnote{Github ist eine Seite, mit der man Quelltext online verwalten kann. Dies ist dahingehend ganz nützlich, dass man die Quelltext-Dateien relativ einfach miteinander synchronisieren kann, wenn man mit mehren Leuten an einem Projekt arbeitet.}, d.h. du findest den gesamten \LaTeX-Quelltext auf \url{https://github.com/henrydatei/TUD_MATH_BA}. Unser Ziel ist, für alle Pflichtveranstaltungen von \textit{Mathematik-Bachelor} ein gut lesbares Skript anzubieten. Für die Programme, die in den Übungen zur Vorlesung \textit{Programmieren für Mathematiker} geschrieben werden sollen, habe ich ein eigenes Repository eingerichtet; es findet sich bei \url{https://github.com/henrydatei/TU_PROG}.

Du kannst dir gerne dort die \LaTeX-Quelldateien herunterladen, die Dateien für exakt dieses Skript sind im Ordner \texttt{1. Semester/LAAG ueberarbeitet}. Es lohnt sich auf jeden Fall während des Studiums die Skriptsprache \LaTeX{} zu lernen, denn Dokumente, die viele mathematische oder physikalische Formeln enthalten, lassen sich sehr gut mittels \LaTeX{} darstellen, in Word oder anderen Office-Programmen sieht so etwas dann eher dürftig aus.

\LaTeX{} zu lernen ist gar nicht so schwierig, ich habe dafür am Anfang des ersten Semesters wenige Wochen benötigt, dann kannte ich die wichtigsten Befehle und konnte den Vorgänger dieses Skriptes schreiben (\texttt{1. Semester/LAAG}, Vorsicht: hässlich, aber der Quelltext ist relativ gut verständlich).

Es sei an dieser Stelle darauf hingewiesen (wie in jedem anderem Skript auch \smiley{}), dass dieses Skript nicht den Besuch der Vorlesungen ersetzen kann. Es könnte sein, dass Prof. Fehm seine Vorlesung immer mal wieder an die Studenten anpasst; wahrscheinlich immer dann, wenn die Prüfungsergebnisse zu schlecht waren. Nichtsdestotrotz veröffentlicht Prof. Fehm sein Skript auf seiner Homepage \url{http://www.math.tu-dresden.de/~afehm/lehre.html}. Allerdings ist dieses Skript recht hässlich, besonders was die Übersichtlichkeit angeht.

Wir möchten deswegen ein Skript bereitstellen, dass zum einen übersichtlich ist, zum anderen \textit{alle} Inhalt aus der Vorlesung enthält, das sind insbesondere Diagramme, die sich nicht im offiziellen Skript befinden, aber das Verständnis des Inhalts deutlich erleichtern. Ich denke, dass uns dies erfolgreich gelungen ist.

Trotz intensivem Korrekturlesen können sich immer noch Fehler in diesem Skript befinden. Es wäre deswegen ganz toll von dir, wenn du auf unserer Github-Seite \url{https://github.com/henrydatei/TUD_MATH_BA} ein neues Issue erstellst und damit auch anderen hilfst, dass dieses Skript immer besser wird.