\chapter{Konvergenz}
Sei $(X,d)$ metrischer Raum.

\textbf{Ab jetzt alles ohne Bweise, folgen später.}

\begin{mydef}[konvergente Folge, Grenzwert]
    Folge $\{a_n\}_{n\in\natur}$ (d.h. $a_n \in X$) heißt konvergent falls $a\in X$ existiert mit $\forall \epsilon > 0\exists n_0 \in \natur\colon d(a_n,a) <\epsilon \quad \forall n \geq n_0$. Dann heißt $a$ Grenzwert (Limes).\\ Schreibe $a = \lim_{n\to \infty} a_n$ bzw. $a_n \longrightarrow a$ für $n \longrightarrow \infty$ oder $a_n \overset{n \to \infty}{\longrightarrow} a$.
\end{mydef}

Sprich: ``'Für jede Kugel um Grenzwert befinden sich ab einem gewissen Index fasst alle FOlgenglieder innerhalb der Kugel.'' Folge $\{a_n\}$ heißt divergent, falls sie nicht konvergent ist.

\begin{folg}
    Für Folge $\{a_n\}$ gilt: $\forall > 0\quad a = \lim_{n\to \infty} a_n \Leftrightarrow$ jede Kugel $B_{\epsilon}(a)$ enthält fast alle Folgeglieder $a_n$, das heißt alle $a_n$ bis auf endlich viele.
\end{folg}

\begin{exmp}[Konstante Folge]
    Sei $\{a_n\} = \{a\}_{n\in \natur}$ (das heißt $a_n = a \forall n$) $\Rightarrow d(a_n,a) = d(a,a) = 0 < \epsilon \forall \epsilon > 0, n \in \natur \Rightarrow a = \lim_{n\to \infty} a_n$.
\end{exmp}

\begin{exmp}
    
    $\forall \epsilon > 0 \exists n_0 \in \natur\colon \frac{1}{n} = \vert \frac{1}{n} - 0 \vert = d(\frac{1}{n},0)<\epsilon \forall n \geq n_0 \Rightarrow \lim_{n\to \infty} \frac{1}{n} = 0$.
\end{exmp}