\part{Metrische Räume und Konvergenz}
\begin{description}
	\item[Konvergenz:] grundlegender Begriff in Analysis %(benötigt Abstandsbegriff (Metrik))
\end{description}
\chapter{Grundlegen Ungleichungen}

\begin{satz}[Geometrisches und arithmetisches Mittel]\label{satz_7_1_geo_mittel}
	Seien $x_1, \dots, x_n \in \real_{>0}$\\
$\Rightarrow$
	\begin{tabular}{ccc}
		$ \sqrt[n]{x_1, \dots, x_n}$ & $=$ & $\frac{x_1, \dots, x_n}{n}$ \\
		geoemtrisches Mittel &  & arithmetisches Mittel \\
	\end{tabular}\\
Gleichheit gdw $x_1 = \dots = x_n$.
\end{satz}

\begin{proof}
	Zeige zunächst mit vollständiger Induktion\\
	\begin{align} %% add /nonumber to have no numbering
	\prod_{i=1}^{n}x_i= \Rightarrow \sum_{i=1}^{n} x_i \geq n \text{, mit } x_1=\dots=x_n \label{7_1_ind}
	\end{align}
	\begin{itemize}
		\item (IA) $n = 1$ klar
	    \item (IS) (\ref{7_1_ind}) gelte für $n$, zeige (\ref{7_1_ind}) für $n+1$ d.h. $\prod_{i=1}^{n+1} = 1$, falls alle $x_i=1 \beha$ Sonst oBdA $x_n < 1$, $x_{n+1} > 1:$\\ mit $y_n:=x_n x_{n+1}$ gilt $x_1\cdot\dots\cdot x_{n-1}\cdot y_n=1$
	    \begin{align*}
	        \Rightarrow x_1 + \dots + x_{n+1} &= \underbrace{x_1+\dots+x_{n-1}}_{\geq \text{ (IV)}} + y_n - y_n + x_n+x_{n+1}\\ 
            &\geq n + \underbrace{(x_{n+1} -1)}_{>n}\underbrace{(1-x_n)}_{>n}\\ 
            &\Rightarrow (\ref{7_1_ind}) \forall n \in \natur& \text{vollständige Induktion}\\ 
            \shortintertext{allgemein sei nun $g:=\big( \prod_{i=1}^{n} x_i \big)^{\frac{1}{n}} \Rightarrow \prod_{i=1}^{n} \frac{x_i}{g} = 1$}
            &\Rightarrow \sum_{i=1}^{n} \frac{x_i}{g} \geq n \beha& \text{Satz \ref{7_1_ind}}\\ 
            \shortintertext{Aussage über Gleichheit nach nochmaliger Durchsicht.}
	    \end{align*}
	\end{itemize} 
	\QEDA
\end{proof}

\begin{satz}[allg. Bernoulli-Ungleichung]
	Seien $\alpha, x \in \real$. Dann\\
	\begin{align*}
	1)\;(1+x)^{\alpha} &\geq 1 + \alpha x \; \forall x > -1, \alpha > 1\\
	2)\; (1+x)^{\alpha} &\leq 1+\alpha x \; \forall x \geq -1, 0 < \alpha < 1
	\end{align*}
\end{satz}

\begin{proof} % fix alignment
    \begin{enumerate}
    \item[2)] Sei $\alpha =\frac{m}{n} \in \ratio_{<1}\text{, d.h. } m\leq n$
        \begin{align*}
            &\Rightarrow (1+x)^\frac{m}{n} = \sqrt[n]{(1+x)^m\cdot1^{n-m}}& \text{Definition} \\
            &\leq \frac{m(1+x)+(n-m)\cdot1}{n}&\\ 
            &=\frac{n + mx}{n} = 1 + \frac{m}{n}x \text{, für } \alpha \in \ratio \beha&
            \shortintertext{Sei $\alpha \in \real$ angenommen $(1+x)^{\alpha} > 1 + \alpha x$ ($x\neq 0$ sonst klar!)}
            & \Rightarrow \exists \in \ratio_{<1} 
            	\begin{cases*}
            	x > 0&$\alpha<q< \frac{(1+x)^{\alpha}-1}{x}$\\
            	x < 0&$\alpha < q$
            	\end{cases*} &\text{Satz 5.8 } \\
            &\Rightarrow 1+qx < (1+x)^{\alpha} \leq (1+x)^q \Rightarrow \lightning \beha& \text{Satz 5.20}
        \end{align*}
    \item[1)] Sei $1+\alpha x \geq 0$, sonst klar
        \begin{align*}
            &\Rightarrow \alpha x \geq -1 \overset{2)}{\Rightarrow} (1+\alpha x)^{\frac{1}{\alpha}}& \text{mit 2)}\\
            &\geq 1 +\frac{1}{\alpha}\alpha x = 1 +x &\\
            &\Rightarrow \text{ Behauptung und Gleichheit ist Selbststudium.}&
        \end{align*}
    \end{enumerate}
	 
%    
\end{proof}

\begin{satz}[Young'sche Ungleichung]
	Sei $p,q \in \real, p,q>1$ mit $\frac{1}{q} + \frac{1}{q} =1 \Rightarrow ab \leq \frac{a^p}{p}+\frac{b^q}{q}\;\forall a,b \geq 0$ (Gleichheit gdw $a^p = b^q$)\\
	Spezialfall($p=q=2$): $ab \geq \frac{a^2 + b^2}{2}$ gilt $\forall a,b \in \real$ (folgt direkt $0\leq (a-b)^2$)
\end{satz}

\begin{proof} %fix formating
	\begin{align*}
	   \shortintertext{Sei $a,b > 0$ (sonst klar!)}
       &\Rightarrow \big(\frac{b^q}{a^p}\big)^{\frac{p}{q}} = \big(1+\big(\frac{b^q}{a^p} -1\big)\big)^{\frac{p}{q}}&\\ 
       &\leq 1+ \frac{1}{q}\big(\frac{b^q}{a^p} -1\big)& \text{Bernoulli-Ungleichung}\\ 
       &=\frac{1}{p}+\frac{1}{q}+\frac{1}{q}\frac{b^q}{a^p}-\frac{1}{q}\\
       &\Rightarrow a^p\frac{b}a^{\frac{p}{q}} = a^{p(1-\frac{1}{q})}b = ab \leq \frac{a^p}{p} + \frac{b^q}{q}& \cdot a^p 
	\end{align*}\QEDA
\end{proof}

\begin{satz}[Höldersche Ungleichung]
    Sei $p,q \in \real;\;p,q > 0$ mit $\frac{1}{q} + \frac{1}{p} = 1$\\
    $\Rightarrow \sum_{i=1}^{n} \vert x_i y_i\vert \leq \big( \sum_{i=1}^{n} \vert x_i \vert \big)^{\frac{1}{p}} \big( \sum_{i=1}^{n} \vert y_i \vert \big)^{\frac{1}{p}}\;\forall x,y \in \real$
\end{satz}

\begin{remark}
    \begin{enumerate}[label={\arabic*)}]
        \item Ungleichung gilt auch für $x_i,y_i \in \comp$ (nur Beträge gehen ein)
        \item für $p=q=2$ heißt Ungleichung Cauchy-Schwarz-Ungleichung (Gleichheit gdw $\exists x \in \real x_i = \alpha y_i \text{ oder } y_i = \alpha x_i\;\forall i$)
    \end{enumerate}
\end{remark}

\begin{proof}
	Faktoren rechts seien $\mathcal{X} \text{ und } \mathcal{Y}$ d.h.
    \begin{align*}
        \mathcal{X}^p &= \sum_{i=1}^{n} \vert x_i \vert^{\frac{1}{p}}, \mathcal{Y}^p = \sum_{i=1}^{n} \vert y_i \vert^{\frac{1}{q}}\text{, falls } \mathcal{X}=0&\\ &\Rightarrow x_i = 0\;\forall i \beha \text{, analog für } \mathcal{Y} =0&\\
        \shortintertext{Seien $\mathcal{X}, \mathcal{Y} > 0$} 
        &\Rightarrow \frac{\vert x_i y_i \vert}{\mathcal{XY}} \leq \frac{1}{p}\frac{\vert x_i \vert^p}{\mathcal{X}^p}+ \frac{1}{q}\frac{\vert y_i \vert^q}{\mathcal{Y}^p} \forall i& \text{Satz 7.3}\\
        &\Rightarrow \frac{1}{\mathcal{XY}}\sum_{i=1}^{n}\vert x_i y_i \vert \leq \frac{1}{p}\frac{\mathcal{X}^p}{\mathcal{X}^p}+\frac{1}{q}\frac{\mathcal{Y}^p}{\mathcal{Y}^p} = 1 \beha & \cdot \mathcal{XY}
    \end{align*}\QEDA
\end{proof}

\begin{satz}[Minkowski-Ungleichung]
    Sei $p\in \real, p \geq 1 \Rightarrow \big(\sum_{i=1}^{n} \vert x_i + y_i \vert^p \big)^\frac{1}{p} \leq \big(\sum_{i=1}^{n} \vert x_i \vert^p \big)^\frac{1}{p} + \big(\sum_{i=1}^{n} \vert y_i \vert^p \big)^\frac{1}{p}\forall x,y\in \real$
\end{satz}

\begin{remark}
	\begin{enumerate}[label={\arabic*)}]
    \item Ungleichung gilt auch für $x_i, y_i \in \comp$ (vgl. Beweis)
    \item ist $\Delta$-Ungleichung für $p$-Normen (vgl. später)
    \end{enumerate}
\end{remark}

\begin{proof}
	$p=1$ Beh. folgt aus $\Delta$-Ungleichung $\vert x_i + y_i\vert \overset{Satz 5.5}{\leq} \vert x_i \vert + \vert y_i \vert \forall i$\\ $p>1$ sei $\frac{1}{p} + \frac{1}{q} = 1$, $z_i:=\vert x_i + y_i\vert^{p-1}\forall i$
    \begin{align*}
        \mathcal{S}^p &= \sum_{i=1}^{n} \vert z_i \vert^q & q = \frac{p}{p-1}\\
        & = \sum_{i=1}^{n} \vert +x_i+y_i \vert\cdot\vert z_i \vert^q & \\
        & = \sum_{i=1}^{n} \vert x_i + y_i \vert + \sum_{i=1}^{n} \vert z_i \vert & \Delta\text{-Ungleichung}\\
        & \leq \big(\mathcal{X+Y}\big)\big(\sum_{i=1}^{n} \vert z_i\vert^q \big)^\frac{1}{p} & \text{Hölder-Ungleichung}\\
        & = \big(\mathcal{X+Y}\big)\mathcal{S}^\frac{p}{q} & \\
        & \beha & p=\frac{p}{q}+1
    \end{align*}\QEDA
\end{proof}


%continue-+
