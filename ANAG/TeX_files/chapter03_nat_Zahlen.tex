\chapter{Natürliche Zahlen}
$\mathbb N$ sei diejenige Menge, die die \textbf{Peano-Axiome} erfüllt, das heißt
\begin{compactitem}
	\item $\mathbb N$ sei induktiv, d.h. es existiert ein Nullelement und eine injektive Abbildung
	$\mathbb N to \mathbb N$ mit $\nu(n) \neq 0 \quad \forall n$
	\item Falls $N \subset \mathbb N$ induktiv in $\mathbb N$ (0, $\nu(n) \in N$ falls $n \in N
	\Rightarrow N = \mathbb N$
\end{compactitem}
$\to \mathbb N$ ist die kleinste induktive Menge \\
$\newline$

Nach der Mengenlehre ZF (Zermelo-Fraenkel) existiert eine solche Menge $\mathbb N$ der natürlichen
Zahlen. Mit den üblichen Symbolen hat man:
\begin{compactitem}
	\item $0 := \emptyset$
	\item $1 := \nu(0) := \{\emptyset\}$
	\item $2 := \nu(1) := \{\emptyset, \{\emptyset\}\}$
	\item $3 := \nu(2) := \{\emptyset, \{\emptyset, \{\emptyset\}\}\}$
\end{compactitem}
Damit ergibt sich in gewohnter Weise $\mathbb N = \{1; 2; 3; ...\}$ \\
anschauliche Notation $\nu(n) = n+1$ (beachte: noch keine Addition definiert!) \\

\begin{theorem}
	Falls $\mathbb N$ und $\mathbb N'$ die Peano-Axiome erfüllen, sind sie 
	isomorph bez\"uglich Nachfolgerbildung und Nullelement. Das hei{\ss}t alle solche $\mathbb N'$
	sind strukturell gleich und k\"onnen mit obigem $\mathbb N$ identifiziert werden.
\end{theorem}

\begin{satz}[Prinzip der vollständigen Induktion]
	Sei $\{A_n \mid n \in N\}$ eine Menge 
	von Aussagen $A_n$ mit der Eigenschaft:
	\begin{enumerate}[ ]
		\item IA: $A_0$ ist wahr
		\item IS: $\forall n \in \mathbb N$ gilt $A_n \Rightarrow A_{n+1}$
	\end{enumerate}
	$A_n$ ist wahr für alle $n \in \mathbb N$
\end{satz}

\begin{lem} Es gilt:
	\begin{enumerate}
		\item $\nu(n) \cup \{0\} = \mathbb N$
		\item $\nu(n) \neq n \quad \forall n \in \mathbb N$
	\end{enumerate}
\end{lem}

\begin{satz}{(rekursive Definition/Rekursion)} Sei $B$ eine Menge und $b \in B$. Sei $F$ eine 
	Abbildung mit $F: B \times \mathbb N \mapsto B$. Dann liefert nach Vorschrift: $f(0):= b$ und
	$f(n+1) = F(f(n),n) \quad \forall n \in \mathbb N$ genau eine Abbildung $f: \mathbb N \mapsto B$. 
	Das heißt eine solche Abbildung exstiert und ist eindeutig.
\end{satz}
$\newline$

\textbf{Rechenoperationen:}
\begin{compactitem}
	\item Definition Addition '$+$': $\mathbb N \times \mathbb N \mapsto \mathbb N$ auf $\mathbb N$ 
	durch $n+0:=n$, $n+\nu(m):=\nu(n+m) \quad \forall n,m \in \mathbb N$
	\item Definition Multiplikation '$\cdot$': $\mathbb N \times \mathbb N \mapsto \mathbb 
	N$ auf $\mathbb N$ durch $n \cdot 0 := 0$, $n \cdot \nu(m) := n \cdot m + n \quad \forall 
	n,m \in \mathbb N$
\end{compactitem}
Für jedes feste $n \in \mathbb N$ sind beide Definitionen rekursiv und eindeutig definiert. \\
$\forall n \in \mathbb N$ gilt: $n+1=n+\nu(0)=\nu(n+0) = \nu(n)$

\begin{framed}
	\textbf{Satz:} Addition und Multiplikation haben folgende Eigenschaften:
	\begin{compactitem}
		\item es existiert jeweils ein neutrales Element
		\item kommutativ
		\item assoziativ
		\item distributiv
	\end{compactitem}
\end{framed}
$\newline$

Es gilt $\forall k,m,n \in \mathbb N$:
\begin{compactitem}
	\item $m \neq 0 \Rightarrow m+n \neq 0$
	\item $m \cdot n = 0 \Rightarrow n=0$ oder $m=0$
	\item $m+k=n+k \Rightarrow m=n$ (Kürzungsregel der Addition)
	\item $m \cdot k=n \cdot k \Rightarrow m=n$ (Kürzungsregel der Multiplikation)
\end{compactitem}
$\newline$

Ordnung auf $\mathbb N:$ Relation $R := \{(m,n) \in \mathbb N \times \mathbb N \mid m \le n\}$ \\
wobei $m \le n \iff n=m+k$ f\"ur ein $k \in \mathbb N$ \\

\begin{satz}
	Es gilt auf $\mathbb N:$
	\begin{compactitem}
		\item $m \le n \Rightarrow \exists ! k \in \mathbb N: n=m+k$, nenne $n-m:=k$ (Differenz)
		\item Relation $R$ (bzw. $\le$) ist eine Totalordnung auf $\mathbb N$
		\item Ordnung $\le$ ist vertr\"aglich mit der Addition und Multiplikation
	\end{compactitem}
\end{satz}

\begin{proof}
	\item Sei $n=m+k=m+k' \Rightarrow k=k'$
	\item Sei $n=n+0 \Rightarrow n \le n \Rightarrow$ reflexiv \\
	sei $k\le m, m \le n \Rightarrow \exists l,j: m=k+l, n=m+j=(k+l)+j=k+(l+j) \Rightarrow
	k \le n \Rightarrow$ transitiv \\
	sei nun $m \le n und n \le m \Rightarrow n=m+j=n+l+j \Rightarrow 0=l+j \Rightarrow j=0 
	\Rightarrow n=m \Rightarrow$ antisymmetrisch \\
	Totalordnung, d.h. $\forall m,n \in \mathbb N: m\le n$ oder $n\le m$ \\
	IA: $m=0$ wegen $0=n+0$ folgt $0 \le n \forall n$ \\
	IS: gelte $m\le n$ oder $n \le m$ mit festem $m$ und $\forall n \in \mathbb N$, dann \\
	falls $n \le m \Rightarrow n \le m+1$ \\
	falls $m < n \Rightarrow \exists k \in \mathbb N: n=m+(k+1)=(m+)1+k \Rightarrow m+1 \le n$ \\
	$m\le n$ oder $n \le m$ gilt für $m+1$ und $\forall n \in \mathbb N$, also $\forall n,m \in 
	\mathbb N$
	\item sei $m \le n \Rightarrow \exists j: n=m+j \Rightarrow n+k=m+j+k \Rightarrow m+k \le n+k$
	\QEDA
\end{proof}